\section{10}

\subsection{1007}

\subsubsection{概要}

\begin{myeasylist}{itemize}
& 回到主线,集中精力执行Q4开发计划
\end{myeasylist}

\subsubsection{suzaku产品详情}

测试
\begin{myeasylist}{itemize}
& 增强基于gitlab+docker的CI系统
\end{myeasylist}

任务
\begin{myeasylist}{itemize}
& 完成任务
&& 支持python3
& 遗留任务
&& GFM完成最小修改版本,等待测试
&& GFM驱动恢复(计划下周完成)
&& \hl{多点写方案待定,最少要实现基于redirect的功能}(亓武强)
& 新增任务
&& restful API等管理接口(王孝海,采用依赖性最小的python框架)
&& Target管理 (王劲凯)
&& ETCD认证
&& 链式COW snapshot(董冠军, \hl{力争10月份完成第一版})
\end{myeasylist}

计划
\begin{myeasylist}{itemize}
\end{myeasylist}

\subsubsection{长城测试项目详情}

\begin{myeasylist}{itemize}
& 硬件
&& 服务器已发货
&& 等待长城网络设备到位
&& Memblaze P5 hazard数据不一致问题,还没结论(先不用XFS)
& 性能?
&& 性能抖动大,\hl{怀疑与网卡没有配置QoS有关}(Mellanox Connect-3 pro,双口)
&& 紫光云项目用的是mellanox 4代网卡
\end{myeasylist}

\subsubsection{紫光云项目详情}

\begin{myeasylist}{itemize}
& 已发布测试报告
&& 存储网络采用RDMA、前端采用iSCSI
\end{myeasylist}

\subsubsection{研发管理}

\begin{myeasylist}{itemize}
& \hl{团队具有高度使命感、目标感,自我驱动,激励与贡献相一致}
& 取势、明道、优术、利器
% & 集中优势,排除干扰,注重贡献
& PDCA工作理念,明确工作目标,优化工作方法
    && 计划
    && 执行
    && 检查
    && 改进(标准)
& 目标及关键任务,\hl{把握好里程碑节点}
    && 知识管理(文档)
    && 会议
    && Design Review
    && Code Review
    && 难点攻关
    && 测试系统
& 采用小步快走的迭代开发模型,定义工作核心,围绕核心作扩展
& 建立原则和标准
    && 设计原则
    && 编码标准
\end{myeasylist}

\subsubsection{附图}

% \mygraphics{../imgs/weekly/weekly-20191007.png}

\subsection{1014}

\subsubsection{概要}

\begin{myeasylist}{itemize}
& 回到主线,集中精力执行Q4开发计划
& 行百里者半九十,\hl{考虑到测试周期比较长,应在10月份收敛各项任务},后续进入优化环节
& 在保存安全性的基础上,尽量\hl{采用最简设计方案},迭代完成
\end{myeasylist}

\subsubsection{suzaku产品详情}

\begin{myeasylist}{itemize}
& Resource Management
&& 设计原则:管理系统无状态
&& Restful API
&& 性能计数、采用Prometheus?
&& CHECKLIST(+网络QoS)
& GFM完成最小修改版本
&& epoch机制,\hl{hazard测试出现一致性问题(recover或mds切换导致?)}
&& Target管理(redirect、VIP/ucarp)
& JOB Management
&& GFM驱动恢复?
&& + rmvol
&& + QoS
& 多点写
&& 新方案?
& 链式COW snapshot(\hl{力争10月份完成第一版})
&& 初步实现create/delete/revert/list/protect接口和命令行工具
&& 下周实现clone卷操作
&& 操作的事务性
& TODO
&& ETCD认证
&& ETCD v3
\end{myeasylist}

测试
\begin{myeasylist}{itemize}
& 增强基于gitlab+docker的CI系统
& 如何查看文件系统到磁盘LBA的映射
\end{myeasylist}

\subsubsection{长城测试项目详情}

\subsubsection{紫光云项目详情}

% \subsubsection{附图}

% \mygraphics{../imgs/weekly/weekly-20191007.png}

\subsection{1021}

\subsubsection{概要}

\begin{myeasylist}{itemize}
& 回到主线,集中精力执行Q4开发计划
& 行百里者半九十,\hl{考虑到测试周期比较长,应在10月份收敛各项任务},后续进入优化环节
& 在保存安全性的基础上,尽量\hl{采用最简设计方案},迭代完成
\end{myeasylist}

\subsubsection{suzaku产品详情}

\begin{myeasylist}{itemize}
& Resource Management
&& 设计原则:管理系统无状态
&& Restful API,\hl{对接运维平台}
&& 性能计数、采用Prometheus?
&& CHECKLIST(+网络QoS)
& GFM完成最小修改版本
&& epoch机制,\hl{可必现,hazard测试出现一致性问题(recover或mds切换导致?)}
&& 非epoch版本有EIO问题
&& Target管理(redirect已完成、对比VIP/ucarp机制)
& JOB Management
&& GFM驱动恢复,\hl{尽早提交测试}
&& rmvol
&& + GC: 同步和异步
&& + Balance
&& + QoS
& 多点写
&& 新方案?
& 链式COW snapshot(\hl{力争10月份完成第一版})
&& 实现create/delete/revert/list/protect/clone接口和命令行工具
&& + \hl{处理操作的事务性(并发控制和故障恢复)}
&& + 下周收尾,提炼代码,做更全面的测试
& TODO
&& ETCD认证
&& ETCD v3
\end{myeasylist}

测试
\begin{myeasylist}{itemize}
& 增强基于gitlab+docker的CI系统
& 全docker测试环境不稳定?
\end{myeasylist}

设计
\begin{myeasylist}{itemize}
& 可维护性
& 分圈:内核和外部,由内而外,先聚焦到缺之不可的内核特性
\end{myeasylist}

代码
\begin{myeasylist}{itemize}
& \hl{用decorator模式去解决重复代码问题},有RPC、并发控制等
& 异步Job管理要提炼通用模式:作业提交和完成
\end{myeasylist}

\subsubsection{长城测试项目详情}

\subsubsection{紫光云项目详情}

% \subsubsection{附图}

% \mygraphics{../imgs/weekly/weekly-20191007.png}

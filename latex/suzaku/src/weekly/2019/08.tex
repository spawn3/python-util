\chapter{2019}

\section{08}

\newpage

\subsection{0819}

\mygraphics{../imgs/weekly/weekly-20190819.png}

\subsubsection{概要}

\begin{myeasylist}{itemize}
& 冻结长城测试版本rc2,继续进行全面测试(故障+性能)
& 下周转向可靠性机制的开发工作
\end{myeasylist}

\subsubsection{详情}

\begin{myeasylist}{itemize}
& \hl{银河麒麟不支持ib驱动}
& 故障测试增加拔网测试,发现了若干问题,已修正,还在继续验证中
& tgtctl不用task,对性能有一定提升,但测试过程中发现52节点有关bactl负载过高,原因还在分析中
&& \hl{fio client cpu已耗尽,三节点iops 300w左右},中断在cpu上分布情况与fio的压测卷数和num jobs设定有关
&& tgtctl不是瓶颈后,压力传导到bactl?
& \hl{除了性能、可靠性之前,产品化还有大量工作要做},比如:
&& 资源管理(节点、磁盘、pool、volume等的增删改查操作)
&& 支持扩展性(scale out加节点、scale up)
&& 支持工具(分析数据平衡性、各组件的性能等等,以提升诊断问题的效率)
\end{myeasylist}

\subsection{0826}

\subsubsection{团队管理思路}

\begin{myeasylist}{itemize}
& 基本理念:PDCA循环
& 明确目标(短/中/长期)和里程碑节点
& 团队分工协作,先予后取,为美好结果而奋斗
& 细化时间管理粒度,排除干扰,保持专注,提高工作效能
\end{myeasylist}

\subsubsection{概要}

\begin{myeasylist}{itemize}
& 确定长城测试方案
& 切实推进suzaku开发进度,\hl{把握好里程碑节点(Q3的目标)}
\end{myeasylist}

\subsubsection{长城测试项目详情}

\begin{myeasylist}{itemize}
& DM8的数据文件只能用单卷,目前\hl{suzaku通过iSER导出的单卷性能,tpcc测试tpm 26w,期望能达到50w+}。
& 融合部署方案不可行,benchmark工具、DB、存储都很耗CPU。
& 如何提升单卷性能?LVM/multipath/NVMf?
& \hl{测试方法是否能发挥我们系统的优势}?数据量、并发度、整个路径瓶颈何在?
& 验证RoCE
\end{myeasylist}

\subsubsection{suzaku开发进度详情}

\begin{myeasylist}{itemize}
& 支持添加节点操作
& 一致性协议完成设计,开始编码,deadline?
& 统一后台任务处理方式,队列,异步,包括:恢复、平衡、GC等
& Recovery开始编码,deadline?
\end{myeasylist}

\subsubsection{附图}

\mygraphics{../imgs/weekly/weekly-20190826.png}

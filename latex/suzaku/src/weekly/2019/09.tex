\section{09}

\subsection{0902}

\subsubsection{概要}

\begin{myeasylist}{itemize}
& 切实推进suzaku开发进度,\hl{把握好里程碑节点(Q3目标)}
& 发布rc2分支,用于长城测试和紫光云测试
\end{myeasylist}

\subsubsection{suzaku项目详情}

\begin{myeasylist}{itemize}
& \hl{质量和效率是两大主题}
& 支持删除卷操作
& 后台任务调度框架:支持rmvol、start recovery、stop recovery
& 支持节点内磁盘漂移
&& suzaku\_disk dump的\hl{前置条件是nvme盘处在非kernel模式},否则结果未知,产生破坏性后果
& 修正bitmap rebuild过程导致的数据破坏,变量的换算关系:$bitmapindex = offset / pagesize$
& GFM机制的实现涉及面广,验证困难,何时收敛存在不确定性 %(小步迭代,及时验证)
\end{myeasylist}

\subsubsection{长城测试项目详情}

\begin{myeasylist}{itemize}
& 公司内的交换机RoCE支持情况
&& \hl{M 1012通过交换机官方文档中的命令行可以切换,目前只有一台支持}
& 掉盘现象: restart or 掉电
& 飞腾安装Centos 7.5?
& 性能?
&& 对比DM8和oracle
\end{myeasylist}

\subsubsection{附图}

\mygraphics{../imgs/weekly/weekly-20190902.png}

\mygraphics{../imgs/weekly/TODO.png}

\mygraphics{../imgs/weekly/S.jpg}

\mygraphics{../imgs/weekly/PDCA.png}

\subsection{0909}

\subsubsection{概要}

\begin{myeasylist}{itemize}
& \hl{质量和效率是两大主题}
& 明确下一阶段的工作职责,\hl{每人建好自己的根据地,同时加强团队学习},达到分工共创的目的
& 固化优化日常工作流程,减少精力和时间的浪费
& 切实推进suzaku开发进度,\hl{把握好里程碑节点(Q3目标)}
& 紫光云测试进行中
\end{myeasylist}

\subsubsection{suzaku项目详情}

\begin{myeasylist}{itemize}
& 后台任务调度框架:支持start balance、同步GC
& GFM在实现chunk状态机、mds策略,预计下周能收敛
& 多点写方案还需改进
\end{myeasylist}

\subsubsection{长城测试项目详情}

\begin{myeasylist}{itemize}
& 飞腾安装Centos 7.5,验证RoCE v2通过
& 所需硬件(服务器、交换机、网卡)没有到位,磁盘故障还在诊断中
& 性能?
&& 对比DM8和oracle
&& Oracle测试性能是达梦的一半,TPM 35w,压力传导不到存储,why?
\end{myeasylist}

\subsubsection{附图}

\mygraphics{../imgs/weekly/weekly-20190909.png}

\mygraphics{../imgs/weekly/MAP.png}

\subsection{0916}

\subsubsection{概要}

\begin{myeasylist}{itemize}
& \hl{质量和效率是两大主题}
& 明确分工、职责,根据地建设
& 固化优化日常工作流程,改善质量,减少浪费
& 具体任务定义和进度/质量控制, 推进suzaku开发进度,\hl{把握好里程碑节点(Q3目标)}
\end{myeasylist}

\subsubsection{suzaku项目详情}

\begin{myeasylist}{itemize}
& 后台任务调度:recovery/balance/gc框架已有,继续改进
& GFM在实现chunk状态机、mds策略,何时收敛?
& 多点写方案v2已出,本周完成etcd based版,进行方案验证
& \hl{CI系统}已上线运行,持续强化
& 技术白皮书
\end{myeasylist}

\subsubsection{紫光云项目详情}

\begin{myeasylist}{itemize}
& 因为SATA SSA,所以必须用aio driver,用bactl直接提供aio服务,比用外部工作线程性能有数倍提升
\end{myeasylist}

\subsubsection{长城测试项目详情}

\begin{myeasylist}{itemize}
& 暂停oracle、dm8性能对比测试,以后再安排攻关
& 预期服务器周五下午到
& 盘的一致性问题原因待定
&& 盘支持的NVMe协议v1.3,而驱动是v1.2
&& test spdk最新nvme driver
\end{myeasylist}

\subsubsection{附图}

\mygraphics{../imgs/weekly/weekly-20190916.png}

\subsection{0923}

\subsubsection{概要}

\begin{myeasylist}{itemize}
    & Q3总结
    & Q4计划
\end{myeasylist}

\subsubsection{suzaku项目详情}

\begin{myeasylist}{itemize}
& 技术白皮书还显单薄、抽象,须进一步细化
& \hl{CI系统}已上线运行,持续强化,加强测试自动化工作
& GFM有延期,后续应当注意从测试到交付之间\hl{完整生命周期}的管理
    && 可以安排早期测试了
    && 优先副本机制,EC放入下一阶段处理
    && 暂时没实现降级模式
& 多点写有延期,优先紫光云性能方案(iSCSI/SATA SSD)
& 适配python3
& 恢复、平衡加入在各节点分布任务的算法
    && 还没有进度展示
    && 还没有与GFM机制对接
& 工程设计原则
    && 多方案设计:plan A/B,或上中下三策,交叉验证
    && 设计之初,实现之时,充分考虑代码的\hl{可维护性,可测试性}
    && 最小可行性产品(MVP),小步快走,迭代改善
& 新需求
    && 快照先实现COW,支持线性快照,不支持tree型快照树
    && 后端增加性能计数器
    && 考虑增强系统可追踪性和可视化能力,以利于优化性能和诊断问题的效率
\end{myeasylist}

\subsubsection{紫光云项目详情}

\begin{myeasylist}{itemize}
& \hl{性能与客户预期有差距}: 读7w+,现在大概4.5w。须定位瓶颈所在。
& 因为SATA SSA,所以必须用aio driver,用bactl直接提供aio服务,比用外部工作线程性能有数倍提升
& iSCSI/SATA SSD下的性能优化方案
    && multipath不适合客户使用场景
    && QEMU版本低,不支持vhost
    && 剩下的方案就是优化iSCSI本身,包括客户端offload卡,采用dpdk等
    && 对比spdk的iscsi tgt?
\end{myeasylist}

\subsubsection{长城测试项目详情}

\begin{myeasylist}{itemize}
& 硬件
    && 网卡已定,没到位
    && 盘的一致性问题原因待定
        &&& 采用spdk,新盘hazard diskfs测试下,xfs依然有问题
        &&& 新盘在测试过程中,会出现soft lockup kernel message
        &&& 下一步对比ext4和xfs
& 软件
        && lighting有bug,不支持用no大于32的cpu core,已修正
\end{myeasylist}

\subsubsection{管理}

\begin{myeasylist}{itemize}
& \hl{培养团队使命感、目标感,激励与贡献相一致}
& PDCA工作理念,明确工作目标,优化工作方法
    && \hl{质量和效率是两大主题}
    && 明确分工、责任,根据地建设
    && 固化优化日常工作流程,改善质量,减少浪费
    && 及时总结,做好记录工作
& 建立原则和标准
    && 设计原则
    && 编码标准
& 目标及关键任务:具体任务定义和进度/质量控制, 推进suzaku开发进度,\hl{把握好里程碑节点}
    && 会议
    && Design Review
    && Code Review
    && 难点攻关
\end{myeasylist}

\subsubsection{附件}

% \mygraphics{../imgs/weekly/weekly-20190916.png}

\section{09}

\subsection{0902}

\subsubsection{概要}

\begin{myeasylist}{itemize}
& 切实推进suzaku开发进度,\hl{把握好里程碑节点(Q3目标)}
& 发布rc2分支,用于长城测试和紫光云测试
\end{myeasylist}

\subsubsection{suzaku项目详情}

\begin{myeasylist}{itemize}
& \hl{质量和效率是两大主题}
& 支持删除卷操作
& 后台任务调度框架:支持rmvol、start recovery、stop recovery
& 支持节点内磁盘漂移
&& suzaku\_disk dump的\hl{前置条件是nvme盘处在非kernel模式},否则结果未知,产生破坏性后果
& 修正bitmap rebuild过程导致的数据破坏,变量的换算关系:$bitmapindex = offset / pagesize$
& GFM机制的实现涉及面广,验证困难,何时收敛存在不确定性 %(小步迭代,及时验证)
\end{myeasylist}

\subsubsection{长城测试项目详情}

\begin{myeasylist}{itemize}
& 公司内的交换机RoCE支持情况
&& \hl{M 1012通过交换机官方文档中的命令行可以切换,目前只有一台支持}
& 掉盘现象: restart or 掉电
& 飞腾安装Centos 7.5?
& 性能?
&& 对比DM8和oracle
\end{myeasylist}

\subsubsection{附图}

\mygraphics{../imgs/weekly/weekly-20190902.png}

\mygraphics{../imgs/weekly/TODO.png}

\mygraphics{../imgs/weekly/S.jpg}

\mygraphics{../imgs/weekly/PDCA.png}

\subsection{0909}

\subsubsection{概要}

\begin{myeasylist}{itemize}
& \hl{质量和效率是两大主题}
& 明确下一阶段的工作职责,\hl{每人建好自己的根据地,同时加强团队学习},达到分工共创的目的
& 固化优化日常工作流程,减少精力和时间的浪费
& 切实推进suzaku开发进度,\hl{把握好里程碑节点(Q3目标)}
& 紫光云测试进行中
\end{myeasylist}

\subsubsection{suzaku项目详情}

\begin{myeasylist}{itemize}
& 后台任务调度框架:支持start balance、同步GC
& GFM在实现chunk状态机、mds策略,预计下周能收敛
& 多点写方案还需改进
\end{myeasylist}

\subsubsection{长城测试项目详情}

\begin{myeasylist}{itemize}
& 飞腾安装Centos 7.5,验证RoCE v2通过
& 所需硬件(服务器、交换机、网卡)没有到位,磁盘故障还在诊断中
& 性能?
&& 对比DM8和oracle
&& Oracle测试性能是达梦的一半,TPM 35w,压力传导不到存储,why?
\end{myeasylist}

\subsubsection{附图}

\mygraphics{../imgs/weekly/weekly-20190909.png}

\mygraphics{../imgs/weekly/MAP.png}

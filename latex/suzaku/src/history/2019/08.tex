\section{08}

\subsection{01}

dmidecode可以查询服务器型号

\subsection{02}

理解target,各种各样的target。host-target之间的transport和protocol是区分的关键。
\hl{类比TCP协议栈}去理解各种新的网络技术。

tgtctl是target和storage的交接点,体现在文件\hl{nvmf\_suzaku\_io}里。

spdk的NVMf导出bdev。如何对接分布式存储?

把libnvme用git管理起来\todo{git-libnvme}。

尝试用一台vm把suzaku跑起来。看看具体要求和配置是什么?

完善关键流程,补上漏洞。采用\hl{用系统来工作}的理念,完善过程。

test是什么状态?应该怎么做?

hazard相关文档。

排兵布阵,上知天文下知地理。

NVMe中buffer的表示,sge?

\subsection{06}

通过ipmi控制服务器。

一块nvme盘加不上,不知为什么?51,52,53上都是如此。51重新插拔盘解决,52、53拔掉电源,重启解决。

实则性能不如8.1版,为什么?观察到disk延迟高,对disk单独进行性能测试,剔除慢盘。
用4盘测试,性能达到600w+,但latency double了。

测量每块盘的平均队列深度和延时。为什么disk的latency突然变大了呢?
\begin{myeasylist}{itemize}
& 没有读过的盘,非稳态性能?
& bactl有问题?
& remote first后,iops显著下降,latency显著升高,磁盘压力小
& \hl{把单卷大小改为80G之后,性能提升上去了}。
\end{myeasylist}

mds\_rpc\_paget,并发高,导致rangectl的内存耗尽?

加入节点,rehash,等待lease timeout,io会中断。

三个client不要同时启动,而是错开几秒钟。

rdma 在提交和完成之间,可能会占用大量内存,导致内存耗尽。怎么解决?
内存不足时使用后备内存,以处理峰值情况。或者core内存管理动态化。

\hl{拆分为两个库,都需要用静态库},不能用so。

\subsection{07}

运行起来\hl{softroce},更容易构建测试环境。

setenforce 0

\subsection{08}

core配比

超线程?

中断平衡,用万兆以太网卡。

login后,配置设备队列深度和调度算法。

\subsection{09}

\subsection{12}

找到了分析内存泄漏的新方法: trace方法。然后用python脚本分析,即可发现哪里分配了,而没有释放。

用两个版本来解决一致性问题:epoch、clock。epoch用来标记故障,clock用来跟踪对象修订号。
明确此两个版本的维护和规则。

看各个ctl的rpc功能,即可理解各个ctl的职责,包括:frctl, rangectl, mdctl, bactl等。
\hl{tgtctl只是解析前端请求,然后派遣给frctl去执行}。

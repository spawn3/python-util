\section{08}

\subsection{01}

dmidecode可以查询服务器型号

\subsection{02}

理解target,各种各样的target。host-target之间的transport和protocol是区分的关键。
\hl{类比TCP协议栈}去理解各种新的网络技术。

tgtctl是target和storage的交接点,体现在文件\hl{nvmf\_suzaku\_io}里。

spdk的NVMf导出bdev。如何对接分布式存储?

把libnvme用git管理起来\todo{git-libnvme}。

尝试用一台vm把suzaku跑起来。看看具体要求和配置是什么?

完善关键流程,补上漏洞。采用\hl{用系统来工作}的理念,完善过程。

test是什么状态?应该怎么做?

hazard相关文档。

排兵布阵,上知天文下知地理。

NVMe中buffer的表示,sge?

\subsection{06}

通过ipmi控制服务器。

一块nvme盘加不上,不知为什么?51,52,53上都是如此。51重新插拔盘解决,52、53拔掉电源,重启解决。

实则性能不如8.1版,为什么?观察到disk延迟高,对disk单独进行性能测试,剔除慢盘。
用4盘测试,性能达到600w+,但latency double了。

测量每块盘的平均队列深度和延时。为什么disk的latency突然变大了呢?
\begin{myeasylist}{itemize}
& 没有读过的盘,非稳态性能?
& bactl有问题?
& remote first后,iops显著下降,latency显著升高,磁盘压力小
& \hl{把单卷大小改为80G之后,性能提升上去了}。
\end{myeasylist}

mds\_rpc\_paget,并发高,导致rangectl的内存耗尽?

加入节点,rehash,等待lease timeout,io会中断。

三个client不要同时启动,而是错开几秒钟。

rdma 在提交和完成之间,可能会占用大量内存,导致内存耗尽。怎么解决?
内存不足时使用后备内存,以处理峰值情况。或者core内存管理动态化。

\hl{拆分为两个库,都需要用静态库},不能用so。

\subsection{07}

运行起来\hl{softroce},更容易构建测试环境。

setenforce 0

\subsection{08}

core配比

超线程?

中断平衡,用万兆以太网卡。

login后,配置设备队列深度和调度算法。

\subsection{09}

\subsection{12}

找到了分析内存泄漏的新方法: trace方法。然后用python脚本分析,即可发现哪里分配了,而没有释放。

用两个版本来解决一致性问题:epoch、clock。epoch用来标记故障,clock用来跟踪对象修订号。
明确此两个版本的维护和规则。

看各个ctl的rpc功能,即可理解各个ctl的职责,包括:frctl, rangectl, mdctl, bactl等。
\hl{tgtctl只是解析前端请求,然后派遣给frctl去执行}。

\subsection{19}

识别不到nvme,掉电后可以重新识别,是memblaze的bug?

128k以上的io,libnvme写入不成功,内部按128k做了拆分,为什么?

hazard在本地有ib的情况下,报错,与ifconfig有关?ifconfig 2> /dev/null后也不行,为什么?
转而用vm测试。\hl{整个过程需要管理起来}。

tgtctl去task改造后性能有一定提升,如何表达sche\_sleep等错误处理逻辑呢?

dm8如何使用导出的盘?需要启用RDMA,否则性能不够好。
性能优化还能做什么?fio client irq影响大。

取法乎上,高标准是正确选择。

排查一致性问题效率太低。

单节点上一个卷可以导出4条路径,配置成多路径后,性能无法达到聚合性能,大约20w左右。
所以,linux的multipath无法产品化。

目前target以\hl{iSCSI/iSER}为主,NVMf还不够成熟,用户态的NVMf initiator?

\subsection{20}

读华为文档,对标华为,学一家就可。

AIO的盘符会变化,需要记录到磁盘元数据上。

\subsection{21}

银河麒麟不支持ib。所以需要\hl{前端iSCSI、后端RDMA}。

\subsection{22}

client scale out,server scale up。

client的cpu耗尽。中断分布均匀性与num jobs有关。

blktrace研究fio的行为。

iet的默认配置参数,有无影响性能的?如r2t。

卷,从18卷到24卷,每个client压测8个卷,在tgt ring的情况下iops 120w。
两节点时,影响较小。三节点时,其中两节点性能下降一半左右。why?
观察到mem ring scan,则52上bactl压力较大。

再次回到六个卷,52上bactl依然压力大。

\hl{单个bactl线程上的counter特别大},普通比另外两个节点上的大很多。

采用tgt ring版才会出现影响较大的情况,难道是tgtctl不再是瓶颈后,
压力传导到bactl,数据不均衡所致?

200G的卷,切换成80G的卷试试,是否大卷跨range修改引发的问题。

开发判断数据分布平衡性的工具,如不平衡,执行再平衡操作。

\hrulefill

\hl{资源管理,产品化方面有很大工作量}。

CRUD操作

scale out:add node

scale up:add disk

删除disk、node、纳入recovery和balance任务。

pool操作

volume操作
\begin{myeasylist}{itemize}
& rm
\end{myeasylist}

\subsection{23}

bactl通过cds\_rpc提供服务。

diskid是全局资源,注册在etcd上。副本位置即按diskid指定。

0.config的0是disk slot里的索引,由diskid找到该idx,并进而定位到对应的disk。
disk->device是可变的,需要时刻保持正确无误。

diskid可以转化为对应的nid(对应bactl coremask),chkid hash到对应的bactl coreid上。
diskid保存在diskinfo元数据区。bactl再访问对应的disk,由此可见,bactl与disk之间是多对多的关系,
不是让一个bactl负责整块disk,单块disk是并发访问的。有性能问题吗?最佳访问模式是什么?

diskid (global) -> disk slot (local) -> disk,通过mapping实现了disk slot的zero-based。
\hl{这层mapping限制了disk总数和diskid取值范围}。

为什么会观察到某个bactl负载过高的现象呢?

有独立线程周期性scan /opt/suzaku/data/ioctl/config目录下的disk配置文件,可以发现新的盘。
\hl{无diskinfo的新盘如何处理}? scan阶段找不到对应diskinfo信息,按new disk处理?

Redis里存放了什么信息?

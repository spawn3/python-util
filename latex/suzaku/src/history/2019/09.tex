\section{09}

\subsection{02}

开始recovery模块的开发。通过一两天搞定编码。逐步迭代。
scan阶段要做到不重不漏。

如何回收chunk?不重不漏的原则,redo要安全。

先mark,再删除redis,再回收bitmap。需要同步alloc和delete两种操作。
在存在mark的情况下,对应的磁盘空间不能被reuse。

统一后台任务处理框架已基本搭建好。\hl{明天可完成删除卷的工作}。

测试方面:磁盘漫游。nvme需要先切换回kernel态,需要先还给kernel?

用ramdisk测试dm8,性能如何?

采用oracle作横向对比?能得出什么结论?存储-DB-测试工具三个层面,瓶颈何在?

最重要的是什么?

\subsection{03}

\subsection{04}

mellanox的1012支持ib、以太模式切换,6012不支持。现在发现1012也不支持,license问题?

\hl{如何从ib切换到以太},交换机和网卡,配置HOWTO手册。

明日用rc2分支发版,测试手册里要说明已知条件和问题,同时指明如何配置。

\subsection{05}

\subsection{06}

\subsection{07}

新来的盘,导致没通过hazard一致性测试。如何验证盘本身的可靠性?还是driver和disk之间的兼容问题。

\subsection{08}

\subsection{09}

长城测试进展:
\begin{myeasylist}{itemize}
& 硬件:
&& 一个交换机能切换到RoCE
&& 飞腾上安装CentOS 7.5?
& suzaku rc2:周五hazard测试发现不一致现象,经反复定位,换回原来的nvme盘,可以稳定运行
& Oracle性能对比测试?
\end{myeasylist}

估算内存使用量的方法

\hrulefill

平衡:多趟scan,第一趟得到卷的所有chunk的副本在disk上的分布情况。
第二趟依次从最满的disk往最空的disk上迁移副本,直到各disk的used或利用率的差值小于等于预定值。

一个chunk可以移动的条件:
\begin{myeasylist}{itemize}
& 该副本位于大于disk平均利用率的disk上
& 满足故障域规则
\end{myeasylist}

需要考虑range的平衡性吗?交叉选取chunk?

这样,即可以满足卷上数据的平衡性,又可以满足disk利用率的平衡性。

\subsection{11}

平衡和恢复机制
\begin{myeasylist}{itemize}
& job manager
& range\_chunk\_move
& cds\_gc
\end{myeasylist}

gc支持sync和async两种模式。

如何在各节点之间分配任务?
\begin{myeasylist}{itemize}
& 获取pool内nodelist
& 按nid升序排列,依次获取编号0、1等
& 获取pool内所有卷列表
& \hl{每个节点选取属于自己的那部分数据}进行处理
\end{myeasylist}

if 节点列表发生变化

if 磁盘列表发生变化

if pool发生变化

if 卷列表发生变化

少数大卷的情况,如何尽量均衡负载?

平衡和恢复有何不同?

\subsection{12}

分工并明确责任到人,做好\hl{制度设计}。

分工是相对的划分,不是一成不变的,有主次之分。

开展团队学习活动。

好书
\begin{myeasylist}{itemize}
& \hl{计算机体系结构精髓}
& 并行多核体系结构基础
\end{myeasylist}

coremask的HOWTO,哪些配置是不合法的?

两个网卡配了同一ip地址

\subsection{16}

whitepaper和性能之巅的培训ppt。

\subsection{17}

存储一致性,缓存一致性。

在bus体系结构下,目录式缓存一致性协议比广播式更具伸缩性。

\hrulefill

质量

P = f(R),C是物理资源、f是体系结构,\hl{P是输出,代表客户价值}。f是R -> P的映射。
体系结构的重要性在此。算法、架构、设计的实用性和趣味性在此。

\hl{响应式宣言突出了系统的质量属性},这也是架构设计的主要考虑方向。
质量考量渗透到每项功能之中。

CAP定理及其v2,关注的也主要是质量属性。

怎么理解阵列、分布式存储的控制器概念?

控制器/target是导出服务的组件,一个LUN可以通过任一controller导出/挂载。

卷控、子卷控制器是内部实现细节。

CPU指令集架构:存储系统也可抽象成指令集,指令集的设计遵循正交原则。

\hl{精读论文、书籍、产品文档},提炼关键问题,持续思考其解决方案。

场景,考虑系统的使用场景,不同场景的要求和特点是什么?

\hrulefill

紫光云测试:
\begin{myeasylist}{itemize}
& 网卡不能做bridge,极大降低性能。
& 能加入同一disk多次?
& 通过disk、network估算最高性能。先评估network的latency和带宽。
& SATA SSD只能采用aio方式,aio thread的配比怎么优化?
& RoCE需要交换机配置QoS
\end{myeasylist}

\subsection{18}

默认coremask不工作,会产生coredump, why?

如果异常,要有补偿措施,拨乱反正,回归到稳态。观察正态分布、正弦曲线。

开启snapshot?

mdctl提供的是kv接口(raw slice),内部则采用pa接口。key是chkid,v是chkinfo。
volume slice、range slice由mdctl内部管理,上层应用查询和更新的是raw slice的元数据。

memblaze实现的是nvme v1.3,libnvme实现的是v1.2,有没有兼容性问题?
每个关键组件都要进行全面评估。

git: 已push到remote,如何回到到某个commit
\begin{myeasylist}{itemize}
& git reset
& git push --force origin rc2:rc2
& git revert
\end{myeasylist}

target实现为何改的面目全非,而不是用plugin的方式?

当前的几个重要任务:
\begin{myeasylist}{itemize}
& GFM
& Recovery and balance
& Restful API
& snapshot
& others
&& nvme driver (采用最新spdk?)
&& 把spdk/dpdk纳入深度学习计划
\end{myeasylist}

\hrulefill

\hl{spdk本质上是一个单机软件},提供了端到端的组件:target、driver、以及对应linux kernel的块存储层次。

spdk没有副本、分布式等,主要是如何单进程环境的主要任务。

单卷是靠单核提供服务的,每个core是同样的,不像suzaku一样为了完成一个io,可以调动多核资源。

先看其driver。driver是对NVMe协议标准的实现,非常基础。driver之上,bdev、blobfs等,增强易用性。

\hl{提取-保存范式}适用于任何io设备。

看其thread抽象,thread采用polling模式,采用QPair方式。可以lockless。
内存也是分核预留的,binding thread到cpu core,因为局部性,cache的效用显著。

这是event-driven的进一步演化。\hl{epoll wait支持两种模式:硬件中断和PMD},但因为要陷入kernel,不是最好做法。
进一步的做法就是DMA,kernel bypass。不再由kernel维护QPair,而是在用户空间维护。\hl{设备可以读写主机内存缓冲区}。

访问一个nvme设备有两种方式:一、kernel driver;二、spdk driver。spdk driver通过pci号进行访问。
通过专用io指令、或\hl{mmio进行存取}。往一块内存区域写数据,就相当于向设备控制器发送了指令。
指令遵循NVMe规范。注意:这里出现了controller。controller广泛出现在盘阵、OS等技术领域。
设备控制器对内存区域包含的指令进行解码、执行。所以、设备控制器可以看作是另外一个处理器。
\hl{io过程看作主机处理器与设备处理器直接协作的过程}。两者通过bus进行通信。跑满总线带宽是最理想的结果。

访问线程与pci设备的映射关系是多对多。多个线程访问同一NVMe disk,不会降低性能?

\begin{myeasylist}{itemize}
& user space (kernel bypass)
& zero copy
& polling mode
& async
& lockless
& highly parallel
\end{myeasylist}

所谓polling,主要指完成队列。

bus和switch结构的同异。

suzaku block应封装在bdev后面,作为一bdev实现而对接到spdk?

应分层来看suzaku block。target - libsuzaku/suzaku,突出target与suzaku的边界。
\hl{suzaku藏于bdev之后?nvmf等target可拿来直接用?}

\subsection{19}

发挥优势,优势是什么?健康、优势、专注、关系、目标。

\hrulefill

ringlock,主要用途即在于检测ring是否发生了变化。

在partition里维护ring map信息,同步更新。每个rangectl持有的本地信息,
与最新信息的版本不一致时,发生ESTALE。此时尝试刷新本地信息,直至成功。

两个版本比较

\hl{从数据模型开始},怎么分片、怎么复制。怎么表达树状结构?怎么保证复制之间的一致性。

自然地划分range。

卷的元数据,引入中间层,增加扩展性。再切换到flat view,转化为KV DB。

\subsection{20}

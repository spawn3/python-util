\chapter{备忘录}

\hl{此处作为缓冲区,记录每日问题,可以一周一整理}。

\subsection{问题列表}

\mygraphics{../imgs/test/test-qperf.png}

测试环境:
\begin{myeasylist}{itemize}
& hardware
& driver
& library
& config
\end{myeasylist}

checklist
\begin{myeasylist}{itemize}
& hardware
&& check network: ip a
&& check network: \hl{/etc/init.d/opensmd start}
&& lstopo: ib非对称布局
&& check hugepage: cat /proc/meminfo
& balance
&& NUMA: memory, ib, disk
&& data balance
&& load balance
& perf -q <depth> 不能等于1。
& check log: backtrace on/off
& centos 7.2上没有nvme rdma内核模块,7.6才有。
& 官方驱动安装时须指定--with-nvmf选项。
& 网络是否可ping通
\end{myeasylist}

问题列表
\begin{myeasylist}{itemize}
& \hl{RDMA连接timeout} RDMA连接时间长,如果集群节点数多了,会构成问题,采用单个线程处理所有连接请求导致的。
& 大io的nvme落盘
& RDMA本地ipc机制
& all to all
& 影响性能的关键参数
& 参考资料、注意事项
\end{myeasylist}

TODO
\begin{myeasylist}{itemize}
& 提取出target(iSCSI/iSER/NVMf)与fss的接口层。
& 统一硬件、操作系统、驱动
& new cq poll接口
& *** DONE
& etcd短连接
& 优化NUMA资源分布
& 按目标节点聚合提交
& ring方式的跨core通信
\end{myeasylist}

优化项
\begin{myeasylist}{itemize}
& Client
&& systemctl stop irqbalance
&& mlnx\_affinity start
& ===
& cpupower minitor
& ===
& O3
& TCP -> RDMA
& aio -> libnvme
& performance\_analysis
& SUZAKU\_DEBUG
& chunk\_replica\_write
\end{myeasylist}



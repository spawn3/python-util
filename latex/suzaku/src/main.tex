% -*- coding: UTF-8 -*-
% hello.tex

\documentclass[UTF8,oneside]{ctexbook}

% \usepackage{xeCJK}
\usepackage[utf8]{inputenc}

% load paralist before enumitem
\usepackage{paralist}

\usepackage{hyperref}
\hypersetup{pdftex,colorlinks=true,allcolors=blue}
\usepackage{hypcap}

\usepackage{color}
\usepackage[usenames, dvipsnames, svgnames, table]{xcolor}
% \pagecolor{gray}

\usepackage{makeidx}
\makeindex

\usepackage{amsmath}
\usepackage{mathtools}

\usepackage{listings}
\usepackage{multicol}
\usepackage{fancybox}
\usepackage{tcolorbox}
\usepackage{enumitem}

\usepackage{indentfirst}

\newenvironment{enumbox}[0]{
    \begin{tcolorbox}
    \begin{compactenum}
} {
    \end{compactenum}
    \end{tcolorbox}
}

\newenvironment{itembox}[0]{
    \begin{tcolorbox}
    \begin{compactitem}
} {
    \end{compactitem}
    \end{tcolorbox}
}

\usepackage[ampersand]{easylist}

\tcbset{colback=yellow!5!white,colframe=yellow!75!black,boxrule=0.1mm}

\newenvironment{myeasylist}[1]{
    \Activate
    \begin{tcolorbox}
    \begin{easylist}[#1]

} {
    \end{easylist}
    \end{tcolorbox}
    \Deactivate
}

\newcommand{\mygraphics}[1] 
{
    \begin{center}
        \includegraphics[width=10cm]{#1}
    \end{center}
}

\newcommand{\mygraphicsh}[1]
{
    \begin{center}
        \includegraphics[height=10cm]{#1}
    \end{center}
}


% table
\setlength{\arrayrulewidth}{1pt}
\setlength{\tabcolsep}{16pt}
\renewcommand{\arraystretch}{2.5}
\newcolumntype{s}{>{\columncolor[HTML]{AAACED}} p{3cm}}

\arrayrulecolor[HTML]{DB5800}

\usepackage{tikz,mathpazo}
\usetikzlibrary{positioning, fit, matrix, shapes, arrows, chains, trees, arrows.meta}

% \bibliographystyle{plain}
% \bibliography{math}

\tikzset{%
  >={Latex[width=2mm,length=2mm]},
  % Specifications for style of nodes:
            base/.style = {rectangle, rounded corners, draw=black,
                           minimum width=4cm, minimum height=1cm,
                           text centered, font=\sffamily},
  activityStarts/.style = {base, fill=blue!30},
       startstop/.style = {base, fill=red!30},
    activityRuns/.style = {base, fill=green!30},
         process/.style = {base, minimum width=2.5cm, fill=orange!15,
                           font=\ttfamily},
}

% 摘录
\usepackage{verbatim}
\usepackage{libertine}
\usepackage{graphicx}
\usepackage{framed}

\newcommand*\openquote{\makebox(25,-22){\scalebox{5}{``}}}
\newcommand*\closequote{\makebox(25,-22){\scalebox{5}{''}}}
\colorlet{shadecolor}{Azure}

\makeatletter
\newif\if@right
\def\shadequote{\@righttrue\shadequote@i}
\def\shadequote@i{\begin{snugshade}\begin{quote}\openquote}
\def\endshadequote{%
\if@right\hfill\fi\closequote\end{quote}\end{snugshade}}
\@namedef{shadequote*}{\@rightfalse\shadequote@i}
\@namedef{endshadequote*}{\endshadequote}
\makeatother

\usepackage[normalem]{ulem}

\newcommand{\hl}{\bgroup\markoverwith
  {\textcolor{yellow}{\rule[-.5ex]{2pt}{2.5ex}}}\ULon}

%\usepackage{soul}

%\newcommand{\hlc}[2][yellow]{{%
%    \colorlet{foo}{#1}%
%    \sethlcolor{foo}\hl{#2}}%
%}

% todonode
\usepackage{lipsum}                     % Dummytext
\usepackage{xargs}                      % Use more than one optional parameter in a new commands
% 
\usepackage[colorinlistoftodos,prependcaption,textsize=tiny]{todonotes}
\newcommandx{\unsure}[2][1=]{\todo[linecolor=red,backgroundcolor=red!25,bordercolor=red,#1]{#2}}
\newcommandx{\change}[2][1=]{\todo[linecolor=blue,backgroundcolor=blue!25,bordercolor=blue,#1]{#2}}
\newcommandx{\info}[2][1=]{\todo[linecolor=OliveGreen,backgroundcolor=OliveGreen!25,bordercolor=OliveGreen,#1]{#2}}
\newcommandx{\improvement}[2][1=]{\todo[linecolor=Plum,backgroundcolor=Plum!25,bordercolor=Plum,#1]{#2}}
\newcommandx{\thiswillnotshow}[2][1=]{\todo[disable,#1]{#2}}
%

\usepackage[simplified]{pgf-umlcd}

\title{SUZAKU架构文档}
\author{董冠军}
\date{\today}

\begin{document}

\maketitle
\tableofcontents

\listoftodos[Notes]

\part{项目}

\chapter{项目}

PM的质量三角
\includegraphics[width=8cm]{../imgs/quality.jpeg}

\section{范围}

道法自然

奥卡姆剃刀

滚雪球,定义MVP:
\begin{enumbox}
\item \hl{定义存储引擎},用各项特性对设计进行压力测试
\item CRC分析,明确职责,划分模块、定义接口
\item 尽快验证性能、一致性和可靠性
\item TDD 完善自动化测试
\item ***
\item Storage Driver
\item MM
\item ***
\item EC
\item Snapshot
\item Consistency Group
\end{enumbox}

\section{成本}

\section{时间}

行百里者半九十

群起而攻之

\chapter{测试}

\section{指标}

性能指标
\begin{enumbox}
\item 性能
\item 故障下的性能
\item 快照下的性能
\item ***
\item IO中断时长
\item 空间利用率
\end{enumbox}

\section{方法}

跟踪每个版本的性能,每个版本都有记录

先验证TCP,测试iscsi、NVMf,导出卷测试。

\section{工具}

perf方式

驱动

网络

RDMA

\chapter{MISC}

\section{GIT}

\begin{lstlisting}[language=bash,frame=single]
git remote add upstream http://gitlab.taocloud.com/suzaku2019/suzaku.git
git pull upstream master (将suzaku2019的内容更新到我本地)
git add .
git commit -m "desc"
git push origin master
\end{lstlisting}

\section{Hosts}

\chapter{参考产品}

\begin{easylist}[itemize]
& \url{https://www.excelero.com}
\end{easylist}

\section{XSKY}

\mygraphics{../imgs/xsky/xsky-ebs.png}


\part{架构}

\chapter{软件架构}

meta管理采用对称的中心化架构。在节点中选举出admin节点,
管理全局状态和数据分配工作。当admin节点发生故障时,会发生
failover过程,选举出新的admin节点来。

lichd进程内嵌各种server,包括iscsi等。

作为存储系统,主要考虑是元数据组织和IO,恢复等关键过程。
功能之外,可靠性、性能,可扩展性至关重要。

client可以和每个节点进行通信,推荐采用VIP机制,简化连接管理。

\section{架构演化}

\begin{enumbox}
\item{引入etcd}
\item{引入存储池}
\item{引入Bcache}
\item{引入VDO}
\item{引入ROW3}
\end{enumbox}

\section{节点}

两类服务器节点:
\begin{compactenum}
\item admin
\item normal
\end{compactenum}

从meta节点中选举出admin。meta节点是静态指定的吗?meta节点列表构成一个小的集群,类似于ceph的monitor。

normal节点是数据节点,存放元数据和数据。元数据和数据都是按1M的chunk组织。
元数据包括四类chunk: pool,subpool,vol,subvol。每个chunk具有固定数目的槽位bucket,指向管理的下一级节点。
集群内的所有chunk,构成一个单根树。每个pool,每个volume,都是这个大树下的子树。

引导信息bootstrap:rootable。记录了根分区的位置,即所在chunk的位置信息。根分区是一个特殊的pool。

所有的控制器,包括pool和volume控制器,通过lease机制保证集群内的唯一性。
其节点位置,由所在子树的根chunk的主副本确定。

chunk副本的节点分布:基于diskmap的随机分布算法,并记录在元数据里。同时,遵循故障域规则。

chunk副本的磁盘分布:本地数据管理(bitmap+sqlite)

chunk副本之间的一致性:强一致性协议。在每次IO操作前,检查各副本的clock和dirty状态,必要的情况下,进行修复。

为了提升性能,需要充分考虑聚合和并发。聚合优于并发,先考虑聚合/批处理。
聚合,一次提交多个chunk。
并发的粒度,多个chunk,一个chunk的多个副本。同时,锁的粒度要恰如其分。

副本数据的管理,没有采用资源池模式,与chkid紧耦合,非共享,不利于分配和释放。

\subsection{admin节点}

职责:
\begin{compactenum}
\item paxos leader
\item lease server
\item 管理系统引导信息
\item 集群节点列表
\item 维护diskmap
\item VIP(控制器当前位置)
\item 分配卷Id
\item 分配\hl{chunk副本位置}
\end{compactenum}

持久化信息和上报信息,心跳机制

可扩展性

admin上持久化的信息,\hl{是如何避免单点故障的,即如何同步到各个meta节点上的}?
可以rootable为例,参见\verb|mq_master.c|。

引入etcd后,admin处理逻辑会简单些。同时引入了多存储池,寻址过程变得复杂。

\subsection{数据目录}

/opt/fusionstack/data:
\begin{itembox}
\item node/rootable (\hl{global info})
    \begin{compactitem}
    \item sysroot/root
    \item misc/fileid
    \item node/t53
    \end{compactitem}
\item disk (local, disk management)
\item chunk (local, sqlite3)
\item status
\end{itembox}

\section{元数据管理}

元数据包括:
\begin{enumbox}
\item \hl{引导信息,用于加载相应对象}
\item 目录下包含哪些文件?
\item 文件包含哪些chunk?
\item chunk的副本位置(节点和节点上的磁盘)
\item 快照以及快照树
\item xattr
\item 各实体对象描述信息,包括创建时间,id等
\end{enumbox}

关键问题有:
\begin{enumbox}
\item 空间管理,分配和回收
\item 数据分布
\item 复制/EC
\item 数据恢复和平衡
\end{enumbox}

在顶层设计下的case by case。

每一个卷的元数据和数据,共有三层chunk,L1和L2是元数据,L3是数据,构成chunk的单根树。
快照和clone出的卷采用统一的chunk树结构,不过增加了交叉引用关系。

L1原先只有一个chunk,有固定数目的槽位,指向L2的chunk,L2的chunk,有固定条目的槽位,指向L3的数据chunk。

table1包含了指向全部L2 chunk的指针数组(table\_proto),table2包含了指向全部L3 chunk的指针数组(chkinfo+chkstat)
table\_proto内在也包含chkinfo和chkstat。每一个chunk,都需要在其父节点上登记,\hl{卷的第一个chunk}登记在pool里。

parent的界定:\hl{raw和subvol的parent都是volume的chkid,subpool的parent是pool的chkid。
vol的parent是pool的chkid},即parent都是可寻址的实体对象(具有控制器)。\hl{sqlite记录的parent遵循该语义}。

如果\verb|table1->table_count|和\verb|table2->chknum|比较大,会有扩展性和性能方面的问题,同时会消耗比较多的内存。

\section{VIP}

\section{微控制器}

每一个加载的卷,对应一个集群内唯一的卷控制器,负责卷的IO等操作。

\section{数据分布}

数据分布首先要满足规则要求,其次则需做到均衡和局部性。
规则是强制的,极端条件下可能退化。均衡和局部性则影响系统性能。

卷属于存储池,存储池上定义副本数和副本放置规则。
保护域和故障域,可以用存储池来统一。

FusionStor通过元数据来管理,\hl{与CEPH的CRUSH有重大不同}。

数据分布规则有:
\begin{compactitem}
\item 存储池规则
% \item 保护域规则
\item 故障域规则(\ref{rule:faultset})
\end{compactitem}

负载均衡和本地化两方面考虑,平衡包括数据平衡和任务平衡。

chunk在节点上的分布,节点内chunk在磁盘上的分布(包括分层)

controller在节点上的分布,controller在core上的分布

各种任务的分布情况,如数据恢复。

\subsection{负载均衡}

\subsection{本地化}

卷控制器所在节点,具有所有chunk的副本。

当切换控制器的时候,需要控制本地化过程的QoS。

\section{复制一致性}

\begin{tabular}{|s|p{6cm}|}
    %\hline
    %\rowcolor{lightgray} \multicolumn{5}{|c|} {Snapshot} \\
    \hline
    pool & LICH\_REPLICA\_METADATA  \\
    \hline
    subpool & LICH\_REPLICA\_METADATA  \\
    \hline
    vol & fileinfo \\
    \hline
    subvol & fileinfo  \\
    \hline
    raw & fileinfo  \\
    \hline
\end{tabular}

恢复和再平衡过程,是怎么工作的?

控制器协调所有IO操作,通过(owner, magic, clock)维护副本一致性。
owner,magic联合起来,处理controller切换的情况。
clock处理一个chunk内所有变更的顺序。

若干问题:
\begin{enumbox}
\item 因为有owner和magic,是否不需要再维护controller的唯一性?
\item 没有使用日志
\end{enumbox}

\subsection{正常情况}

chunk内的更新按clock顺序提交,这样做影响并发度,进一步影响到性能。
如果不遵循clock序提交,需要有一假设:所有并发任务无重叠,或者是在有重叠的地方按clock串行化。
上层应用不会提交重叠的IO,内部元数据涉及小IO,比如小于512B的IO,多个更新可能指向同一扇区。
如果多个副本不按同样的顺序,就会出现数据一致性问题。

如果收到clock连续的io,且这些io无重叠部分,可以直接提交。
如果io有重叠,按clock序依次提交。采用lattice,可以实施聚合优化。

一种做法是保证并发io无重叠。需要分析每一种io情况,包括应用层io,内部元数据等。
另一种做法是维护一数据结构,能够跟踪io重叠情况。

\subsection{故障情况}

控制器切换

在降级写的时候,没有参与的副本标记持久化状态stale,保证在reload的时候,不会被选为权威副本。

网络分区

节点掉电

集群掉电

disk and raid cache

\section{精简配置}

每个raw chunk有三者状态:ENOENT,alloc,alloc and zero。第一种和第三种等价。

分配一个chunk,开销较大,影响到精简配置和快照的性能。
分配一个chunk,涉及到更新元数据。

快照、克隆卷天然就是精简配置的。

磁盘空间分配器,db记录chkid到磁盘位置的映射(采用rocksdb?)。

\section{分层}

卷的xattr,有目标分层设定:priority。默认是-1,即开启自动分层。先落入tier 0,根据数据热度,
通过异步过程进行数据迁移(纵向的数据流动)。所以有两个异步任务:
\begin{compactenum}
\item 统计访问热度
\item 按策略执行数据迁移(目标分层的偏离,数据热度)
\end{compactenum}

每个chunk有实际分层:tier,不是在副本级。

\section{SSD缓存}

SSD Cache实现了写缓存,没有实现读缓存, 通过内存实现读缓存。

\subsection{配置项及系统行为}

相关控制参数:
\begin{enumbox}
\item lich.conf/disk\_keep: 10G (废弃)
\item lich.conf/disk\_cache: 10G
\item 卷的xattr: writeback
\item 卷的属性:priority (设定卷的目标存储分层)
\end{enumbox}

disk\_cache配置磁盘cache区,对新盘有效,若一个盘已经配置,就固定了,不再改变。
此配置对SSD和HDD都起作用,预留磁盘末尾的空间。

priority是手工分层机制,持久化到卷属性上。

tier是自动分层机制,默认行为是什么? 优先落到SSD上,若SSD满,落到HDD上。

所以,分三种情况:
\begin{compactenum}
\item 自动分层,priority == -1,优先落盘到SSD上,如果SSD满,落盘到HDD上
\item priority == 0, 同自动分层机制
\item priority == 1,落盘到HDD上(此时根据writeback的设置,决定是否走SSD cache)
\end{compactenum}

写IO流程

以下两种情况,会落盘到HDD上:
\begin{compactenum}
\item priority == 1
\item SSD满
\end{compactenum}

落盘到HDD的写IO,\hl{其行为受xattr.writeback影响},分两种情况:
\begin{compactenum}
\item xattr.writeback == 1,写入ssd的cache区+内存后,返回。
\item xattr.writeback == 0, 写入HDD。
\end{compactenum}

落盘到SSD的写IO,不经过Cache, 走原来的IO路径。
\begin{compactenum}
\item 自动分层,会优先落到SSD上
\item priority == 0, 会导致落盘到SSD上
\end{compactenum}

已知问题

\begin{compactenum}
\item 原来的分层机制,默认情况下,会写满SSD,导致后续写HDD(先快后慢)。
\item 异步线程的工作机制 (周期性执行,20min执行一次, 最后才会置换proirity == 0的数据)
\item disk cache区开启后,就不能再改变,空间无法回收,且不能关闭(不可逆过程)
\end{compactenum}

不确定的地方

考虑以下问题,初次分配dispatch\_newdisk,等固定了分层后,后续的变更规则是什么?

\section{分区容忍性}

参考fence.c

\section{关键过程}

\subsection{启动集群}

\subsection{创建存储池}

创建存储池后,会在etcd上记录引导信息。
创建存储池后,必须添加磁盘到该存储池。一块磁盘只能属于一个存储池。

\subsubsection{磁盘管理}

每块磁盘对应一个bitmap,用于该盘的空间管理。

磁盘有分层属性,通常0表示SSD,1表示HDD。有三种分层策略:tier==0,表示写入SSD,tier==1表示写入HDD,
tier==-1表示自动分层,先写入SSD,通过异步后台线程flush不活跃的数据到HDD。

在分配每一个chunk的时候,可以指定tier。没有指定的情况下,默认为卷的priority设定。

chunk\_id到磁盘物理地址的映射,是一随机过程,定位到空闲的bitmap上。

如何确定磁盘的分层?

RAID管理,disk和raid都有cache,需要注意掉电情况下是否丢数据。

\subsection{创建卷}

\subsection{IO过程}

写过程,可能内在地包含了分配chunk的过程,缺页分配。当在末尾写入时,还可能扩展了卷的大小。

大范围内的随机写入,造成很多的缺页分配,分配过程会成为性能瓶颈。

\subsection{分配一个chunk的过程}

分为两阶段:分配空间,chkid和磁盘位置的映射。两阶段按SEDA方式组织,没按pipeline组织。

分配空间线程池:每个磁盘一个线程,多线程共享任务队列。


函数:
\begin{compactitem}
\item \verb|__table2_chunk_create|
\item \verb|replica_srv_create|
\item \verb|disk_create|
\end{compactitem}

与admin交互,返回节点列表,即各副本所在节点。

需要持久化的信息:
\begin{compactitem}
\item disk bitmap,记录磁盘上每个chunk的分配状态
\item sqlite3,记录chkid(副本)到物理地址的映射关系
\item table2 meta,记录chunk info(副本位置)
\item 填充chunk内容为全0?
\end{compactitem}

分配chunk的过程,会影响到若干特性,如精简配置,快照、恢复,再平衡,写入等,都产生新chunk。

优化allocate的性能:
\begin{compactitem}
\item 加大lich.inspect线程数到20
\item table2 lock粒度 \change{dynamical lock table}
\item 异步化sqlite,每个db一个工作线程
\end{compactitem}

\subsection{事务处理}

一个复杂的主题是事务处理,即如何保证在故障条件下,过程执行的ACID属性。
需要做出事务保证的典型过程有:
\begin{compactenum}
\item 创建卷(metadata表出现垃圾记录)
\item 创建快照(snap\_version得不到有效维护)
\item 分配chunk(meta和副本不一致)
\end{compactenum}

以分配chunk为例,基本操作有:
\begin{compactenum}
\item 申请chkid和chkinfo
\item 分配disk bitmap
\item 分配sqlite记录
\item 写入meta
\end{compactenum}

任何两个操作之间发生故障,都会导致问题。

\subsection{异步过程:删除卷}

\subsection{异步过程:删除快照}

\subsection{异步过程:回滚快照}

\subsection{异步过程:flat快照}

\subsection{异步过程:数据恢复}

\section{高性能}

网络层:iSER,NVMf

设备层:libnvme/SPDK

libiscsi

tgt

\chapter{SSAN一致性问题}

\section{原理}

\includegraphics[width=11cm]{../imgs/consistency-splice.png}

从逻辑上讲,一致性是由任一对象的变更历史决定的。任一对象的多个副本/分片,可以看作有限状态机,
须按同一顺序执行变更。变更通常包括写IO和修复IO。

相比于副本机制,EC的各分片具有严格顺序。

从实现机制上来看,副本或EC的一致性,需要从\hl{对象版本、控制器和日志}几个方面来考虑。

恢复过程的关键是\hl{选择到正确的副本/分片}。分为几种情况:
\begin{enumbox}
\item EC的节点故障
\item EC的磁盘故障
\item 副本
\end{enumbox}

\section{EC一致性}

\subsection{对象版本}

从概念上来说,SSAN按epoch组织对象,节点故障时提升epoch,磁盘故障时epoch不变,
通过强制升级epoch来模拟节点/磁盘混合故障。

epoch是集群级别的版本,epoch内节点成员关系不变。在SSAN实现里epoch被用作粗粒度的对象版本。

\subsection{控制器}

IO控制器是gateway,SSAN原始实现无恢复控制器,后针对任一对象引入primary数据分片作为恢复控制器。
这样就形成\hl{IO和恢复的双控架构},为了对象一致性,需要同步IO控制器和恢复控制器。

\subsection{日志}

无日志,难以处理特定情况下的恢复问题。

\subsection{对象组织及其cache}
\label{subsec:object-dir}

因为可能在工作目录创建不同epoch的对象,工作目录下的对象名字也要包括epoch。

进一步可以考虑按epoch组织目录,这样可以简化关键操作,比如消除rename和link操作。
磁盘故障时,因为不升级epoch,所以需要特别处理,\hl{校正对象的磁盘位置,但不需要link了}。
需要保证过程的原子性。

\includegraphics[width=11cm]{../imgs/object-dir.png}

维护磁盘对象结构的内存cache,在其上面提供API,合并stale cache和object list cache。
需要实现的API包括:
\begin{enumbox}
\item get\_obj\_list,获取一节点上所有对象的oid
\item get\_obj\_history,获取一个object的历史版本
\item get\_obj\_history2,获取一个object的历史版本,wd==0
\item stale\_cache\_compact,清除无效磁盘相关的记录
\end{enumbox}

\subsection{恢复实例}

\includegraphics[width=11cm]{../imgs/recovery-fsm.png}

恢复实例可以看作有限状态机。在恢复期间,SSAN进程运行一个恢复实例。
如果有新的故障,则执行上下文切换,切换到下一恢复实例。需要保证切换恢复实例过程的正确性。
任一时刻,最多有一个恢复实例在运行。

恢复状态机的每步转换都要满足safety和liveness条件,特别需要注意的是:
\begin{enumbox}
\item update epoch过程务必成功执行
\item 若一节点收不到recover peer,无法进入NOTIFY STANDBY DONE状态
\end{enumbox}

\section{EC一致性改进之处}

具体见git仓库的提交日志。

\subsection{增强系统可追踪性}

主要通过日志机制来实现。把每个对象、io、恢复实例等等实体看作对象,追踪其生命周期行为,便于分析异常现象。

\subsection{create and write采用sync模式}

出现虽然写成功,后来发现对象内容为全零的情况。

\subsection{优化oidlist的索引}

优先修复vdi object。

采用bitmap检索data object和ledger object。\todo{摘除优先修复对象}\hl{数据量大时,优先修复的oid依然效率低}。

\subsection{改进对象组织方式}

参考小节~\ref{subsec:object-dir}

\subsection{改进stale object cache}

改进stale object cache模块,用于追踪对象在磁盘上的分布,可以理解为磁盘目录结构的cache。
通过支持所需API,来替代原来的object list cache和stale cache。同时也方便stale object的GC过程。

\subsection{恢复状态机引入新状态}

重构恢复实例的状态机。

引入RW\_INIT:为了实现没有进入prepare状态的恢复实例,可以切换到下一实例。一旦进入prepare阶段,则切换过程有所不同。
\hl{通用原则是确保rinfo上下文信息的安全性}。在有引用计数的情况下,不能被free掉。

引入RW\_UPDATE\_EPOCH:因为update epoch执行时间过长,为了不堵塞main线程,须放到工作线程中去做。

引入RW\_NOTIFY\_STANDBY\_DONE:放入同步点,以确保object list cache准备妥当,才能保证后续prepare object list过程无误。

避免prepare object list重复入队,导致修复崩溃

\subsection{磁盘空间不足时的恢复过程}

\todo{磁盘空间不足时的恢复过程}可以在finish object list过程,加入检查逻辑。检查项:
\begin{enumbox}
\item 每个disk的容量是否够用(执行hash运算后分布到该磁盘上的对象)
\item 对象的历史版本可能没及时回收
\item 在恢复过程中会有新的create and write
\end{enumbox}

如果不能通过检查,则标记节点状态为NODE\_NOSPC,影响到的操作:
\begin{enumbox}
\item 读写io
\item 退出恢复过程
\end{enumbox}

在此状态下,运行执行删除卷操作,以回收空间。回收完成后,重新进入修复状态。

\subsection{retry机制}

retry机制的使用需要具体分析,内部过程慎用retry,避免堵塞main线程,使系统失去响应能力。

重试次数和timeout值的选择也影响到故障切换时长和IO中断时长。

\subsection{Too many open files}

文件句柄数量控制,由最大1024改为1048576。直接在SSAN进程内设定。

\subsection{unlikely使用不当}

\includegraphics[width=11cm]{../imgs/unlikely.png}

\section{副本一致性}

现象:观察到恢复完成后,有时vdi对象并不一致。

\todo{副本一致性}目前副本的一致性实现,机制上恐有问题。
恢复的选择步骤,各个副本独立运行选择过程,所依据的并非该对象各个副本的全局信息,而是相当局部的信息。
并不能保证一定选择到正确副本。

需要参考EC一致性的机制,选出primary协调IO和recovery活动。
副本的选择步骤相对简单:可用的最大版本的副本,以之为权威副本,覆盖其余。

\section{小结}

指导原则
\begin{enumbox}
\item 一致性问题要对标相关参考模型
\item 采用流体动力学模型分析性能瓶颈
\item 工欲善其事必先利其器
\end{enumbox}

工具方面
\begin{enumbox}
\item 完整日志追踪系统,细粒度地追踪程序运行时行为
\item 加入PROFILE日志,辅助分析各个过程的性能特征
\item 多用断言,以捕获程序中的不变式,尽早暴露问题
\item 生成COREDUMP
\item 采用valgrind分析内存问题
\item 采用egrep分析日志,保留相关日志的相对顺序
\item ***
\item 尽量保障开发和测试环境
\end{enumbox}

egrep的使用示例:
\begin{lstlisting}[language=bash,frame=single]
egrep 'start_recovery|free_recovery_info' ssan.log
egrep 'start_recovery|iops' ssan.log
\end{lstlisting}

关于日志子系统,需要从内容和形式上进一步规范化。
\begin{enumbox}
\item 可动态调整日志等级
\item 管理对象的生命周期活动
\item 捕获尽可能多的上下文信息
\item 提高日志的信息密度
\item 关键字
\end{enumbox}

性能分析
\begin{enumbox}
\item 流出等于流入
\item 下游处理能力大于流入流量
\item 调度能力大于下游处理能力
\end{enumbox}

重点是提升下游节点的处理能力和中间节点的调度能力。
以修复为例,下游处理能力对应恢复性能,调度能力对应main线程的调度能力。

\chapter{性能}

\section{性能估算}

\begin{enumbox}
\item 单卷性能
\item 网络
\item 磁盘
\item 副本数
\end{enumbox}

\chapter{精简配置}

% \chapter{QOS}

\section{概述}

先需明确问题,是单点控制,还是分布式控制?

学习的方法:
\begin{enumbox}
\item \hl{对标}:行业的标准做法是什么?
\item 如何才能更好地学习?
\item 可以参考的资料有哪些?
\item *
\item 先选出几篇经典论文,顺藤摸瓜,建立相关的知识体系。
\item 与专业人士交流,获取有价值的线索。
\item 还需要主动去悟,提问、消化、守破离,推陈出新
\end{enumbox}

参考网络QoS,存储QoS的核心算法与网络QoS相同。

排队论

态势感知?

在高IOPS的情况,QoS的开销过大,极大地拉低了性能,这是不可接受的。

每次请求都要获取一次时间,是不是必要的?

\subsection{参考}

\begin{enumbox}
\item OS中进程、线程调度算法
\item Disk IO调度算法
\item VM IO调度算法
\item Network QoS and Storage QoS
\item TCP/IP
\item iSCSI
\item SPDK QoS
\item Ceph dmClock
\item SolidFire QoS
\end{enumbox}

\section{算法}

采用了两种曲线

开放控制参数

比较指标:理论和实测值的距离,\hl{也可以考虑夹角的大小}。\change{距离函数}

底层采用token bucket,需要能容忍一定的jitter。

在调度器内加入QoS控制逻辑的设想: 每个core调度器对应一个或若干卷控制器。基于优先级队列,由core线程处理队列(scheduler队列?)。
每个卷控制器在对应的scheduler上注册自己的队列(IO任务、恢复任务)。\hl{core上的每个卷,向scheduler注册自己,从而实现解耦}。
调度器不仅可以处理单个卷的QoS,也可以处理多个卷的QoS。

\hl{队列和线程}往往紧密结合为一体,参见SEDA、actor。

\hl{多mode调度器},根据实际负载条件动态地调整调度器策略。

何时从请求队列移入调度队列是QoS调度器的中心任务。

\section{已知问题}

顺序io,在上层聚合,导致vctl上的qos不准确。

\chapter{快照和克隆}


\part{开发者指南}

\chapter{Getting Started}

\section{配置}

\subsection{solomode}

\chapter{代码}

集中兵力,各个击破

管理和技术,管理,不仅是运营管理,还有技术管理

可重用性

可测试性

\begin{enumbox}
\item 每个组件有rpc,导出接口
\end{enumbox}

结构
\begin{enumbox}
\item 命名规则
\end{enumbox}

函数
\begin{enumbox}
\item 行数
\end{enumbox}

MM
\begin{enumbox}
\item buffer\_t
\item coroutine stack
\item 小对象
\end{enumbox}

\include{devguide/overlayos}
\chapter{实现相关}

\begin{itemize}
    \item polling
    \item coroutine and scheduler
    \item mbuffer
    \item kernel bypass
\end{itemize}

\section{Safe Mode}

卷级进行检查,处在保护模式的卷,不允许iscsi连接,返回错误码。

加载时间较长的模块,用half sync/half async模式来处理。分为两阶段:同步+异步。

一个卷,加载成功,依赖于几个条件:
\begin{itemize}
    \item raw/lsv
    \item module load
\end{itemize}

\section{Coroutine}

使用规则:
\begin{itemize}
    \item \verb|schedule_task_get|和\verb|schedule_yield|要匹配
    \item task有数量上的限制:1024,超出后容易引起死锁
\end{itemize}

\section{Memory Management}

\subsection{ymalloc}

\subsection{mbuffer}

\subsection{huge page}

2M

\section{AIO}

生产者-消费者模型。

\section{sqlite3}

异步sqlite3。

\begin{itemize}
    \item 共10个db,每个一个线程,每个线程管理一个队列
    \item 所有sqlite3操作,泛化为统一的结构,放入线程队列
    \item 消费者线程批量处理队列中的任务
    \item 生产者线程和消费者线程通过sem进行通信
    \item 采用协程机制(yield/resume)同步任务执行顺序
\end{itemize}

消费者线程wait在sem上,生产者线程有消息的时候,调用\verb|sem_post|。

\section{RPC}

\begin{itemize}
    \item minirpc
    \item rpc
    \item corerpc
\end{itemize}

\section{Algorithm}

\begin{itemize}
    \item paxos/raft 选举admin和meta节点
    \item vector clock 副本一致性
    \item lease controller的唯一性
\end{itemize}


\part{用户指南}

\chapter{Configuration}

\section{Hardware}

\subsection{NUMA}

\subsection{Network}

\section{Configuration}

\subsection{core mask}

\chapter{iSCSI}

\section{Getting Started}

iscsiadm

setenforce 0

\section{Concepts}

\mygraphics{../imgs/iscsi/iscsi-conf.png}


Target/LUN: target和lun的discovery机制

Session/Connection

每个suzaku卷对应一个target,每个target只有一个lun?

discovery的target与volume数和tgtctl的core数有关,是两者的乘积。

target的port?多路径?

VAAI

与NVMf规范的相关性

\section{Code}

与suzaku相关的关键文件
\begin{myeasylist}{itemize}
& target.c
& volume.c
& efs\_io.c
\end{myeasylist}

\section{Key Processes}

\subsection{Discovery}

\mygraphics{../imgs/iscsi/iscsi-discovery.png}

\subsection{Connect}

\mygraphics{../imgs/iscsi/iscsi-connect.png}

\section{iSER -- iSCSI over RDMA}

iSER利用了部分iet代码,特别是对接后端存储的部分。

参考\hl{iser\_cmds/iser\_scsi\_cmd\_iosubmit}函数。

\section{Reference}

RFC
\begin{myeasylist}{itemize}
& rfc 3720
& rfc 3721
& rfc 7143
\end{myeasylist}

\chapter{NVMf}

\section{Getting started}

默认NVMf不监听后端网络,即suzaku.conf里配置的网络,所以至少需要一个不同的前端网络。

\subsection{nvme-cli}

\mygraphics{../imgs/nvme-list.png}

\subsection{RDMA}

NVMf卷attr,只能被该协议访问。

no handler found for RDMA transport

\begin{myeasylist}{itemize}
    & modprobe nvme\_rdma
    & modprobe nvme\_fabrics
    & ***
    & ERROR: RDMA listen 0.0.0.0 
    & ERROR: link static libibverbs
    & ***
    & use github nvme-client
    & ERROR: mlnx mln\_compat
\end{myeasylist}

\subsection{NVMf}

NVMf的initiator的安装
\begin{myeasylist}{itemize}
& CentOS 7.6
& IB Driver
& client
&& nvme-cli (nvme\_rdma, nvme\_fabrics)
&& spdk/perf
&& multipath
& info
&& /sys/class/nvme/
\end{myeasylist}

\section{Concepts}

NVMf的RDMA所以一个一个处理,是因为重用req?

如何标识一个卷?在分布式系统中,卷的标识应独立于节点。

subsystem和ns如何映射到分布式环境下?nqn也不因为在节点之间漂移而变化?

subsystem是节点内的概念吗?不是,需要有全局标识。多个host可以通过不同节点连接同一subsystem。

采用\hl{网络协议栈的分层架构模型}去理解NVMf,以及代码阅读的经验谈。

NVMf的RDMA实现性能如何?

nvmf上每个core上启动一个subsystem,每个subsystem包含若干session,session包含connections。
cq是connection级别的。

poll线程不能太多?

nvme-cli为什么能列出PCI NVme和NVMf挂载的设备?这两种设备有着相同特征。

NVMf:从RDMA transfer看起,怎么建立连接,怎么send and poll。 
每个core对应一个subsystem,每个subsystem包含若干session、每个session包含若干连接。连接关联到transport。

在core map里维护卷到core的映射。

discovery机制:

\section{Code Reading}

\begin{myeasylist}{itemize}
& nvmf
& transport (rdma)
& request
& subsystem
& session
& volume
& suzaku\_io
\end{myeasylist}

\subsection{nvmf-session-connect}

\mygraphicsh{../imgs/nvmf/nvmf-session-connect.png}

只有一个tgt的情况,建立两个session,每个session包含1个admin连接和2个io连接。

如果有多个tgt,可以横向扩展。

单卷的性能,既受前端网络的影响(listen了所有的前端网络),又受tgtctl数量的影响。

% \include{ssan/rr}

\part{知识库}

\chapter{learning}

\section{学习方法}

查理芒格的模型:学科的重要模型。

数学概观:现实-模型-理论三元组。模型是对现实的抽象,把逻辑运用到模型推演出理论体系。
通过一问一答解决现实问题。模型的验证,一是事实,而是逻辑。

找到某些基础模型、或如高焕堂老师说的:form。作为构建更复杂系统的基本单元,
有助于达成以简御繁的目的。

化整为零

临摹

师法造化,内得心源。

\subsection{包围式学习}

E=K/I,温故知新,通过包围式学习构建知识网络。这是主动的开疆辟土,步步为营。

学习、学问这些词汇,有很深内涵,回到中庸的论述。

学的是思维方式和方法论,习是刻意练习,体用、知行、道器一体。
器是作品集。

\subsection{戒定慧}

六度架起此岸、彼岸的桥梁。

\begin{enumbox}
\item 11点之前睡觉、六点起床
\item 问题驱动
\item 一心二本
\end{enumbox}

\subsection{守破离}

以算法为中心,贯通多个领域。

% https://en.wikipedia.org/wiki/Algorithm#Informal\_definition

\section{学习计划}

知识体系:编程语言、数据结构和算法、架构和系统。
最重要的系统有:操作系统、编译原理和数据库。

分布式存储系统能把这些知识点贯通起来。

\begin{itemize}
    \item SCSI
    \item NVMe
    \item SPDK
\end{itemize}

\section{学习资料-书籍}

\subsection{老马识途}

斩码三刀:猜测-实证-建构

阅读源码的方法:调试-阅读-调试。以调试为方法,动起来,把握主线。所以要熟练掌握gdb等工具。

逆向工程是匕首,答疑解惑。

主要目的在于培养系统观,不以记住知识为高,而以培养系统观为能,就是学会学习的方法。

\subsection{算法导论}

\subsection{伟大的计算原理}

六类计算原理:计算、存储、通信、协作、评估、设计。

算法、架构、设计三部曲。

架构:高可用、高性能、负载均衡。

设计准则:需求、正确性、容错性、时效性、适用性。
软件系统的设计原理:层级式聚合、层级、封装、虚拟机、对象、C/S。

\subsection{完美软件设计}

设计的几个原则:
\begin{enumbox}
\item 正交
\item 分层
\item 时序下的数据流
\item 封装
\item 名实
\end{enumbox}

\subsection{设计原本}

\subsection{计算机程序的构造与解释}

\section{Paper}

\chapter{Tools}

\section{Deployment}

\subsection{ntpdate}

\mygraphics{../imgs/tool/ntp-date.png}

\subsection{ansible}

\section{Development}

\subsection{cmake}

\mygraphics{../imgs/tool/cmake-link-static.png}

生成静态库
\begin{myeasylist}{itemize}
& SHARED  -> STATIC
& LIBRARY -> ARCHIVE
\end{myeasylist}

\subsection{gdb}

\begin{myeasylist}{itemize}
& ~/.gdbinit
& info registers
& info sharedlibrary
& gdb -p
\end{myeasylist}

gdb -p发现了mbuffer\_writefile进入死循环,原因是count==0。

猜想是重入了一个锁。

\subsection{debug}

trace msgid来跟踪消息流。

\subsection{wireshark}

\subsection{SoftRoce}

spdk/scripts/setup.sh

\section{Test}

\subsection{fio}

\subsection{spdk/perf}

\subsection{hazard}

\mygraphics{../imgs/tool/hazard-1.jpeg}
\mygraphics{../imgs/tool/hazard-2.jpeg}
\mygraphics{../imgs/tool/hazard-3.jpeg}

\chapter{Linux}

\section{ksoftirqd}

\mygraphics{../imgs/linux/ksoftirqd.png}

\chapter{Version}

更新冲突

\section{Logical Clock}

\section{Vector Clock}

\section{Examples}

\subsection{cas}

\subsection{http etag}

\subsection{etcd idx}

\subsection{mysql}

\subsection{session consistency}




\part{备忘录}

% -*- coding: UTF-8 -*-
% hello.tex

\documentclass[UTF8]{book}

\usepackage{xeCJK}


\usepackage{hyperref}
\hypersetup{pdftex,colorlinks=true,allcolors=blue}
\usepackage{hypcap}

\usepackage{color}
\usepackage[usenames, dvipsnames, svgnames, table]{xcolor}
% \pagecolor{gray}

\usepackage{makeidx}
\makeindex

\usepackage{amsmath}
\usepackage{mathtools}

\usepackage{listings}
\usepackage{multicol}
\usepackage{fancybox}
\usepackage{tcolorbox}
\usepackage{enumitem}

\title{LICH架构文档}
\author{董冠军}
\date{\today}

% \bibliographystyle{plain}
% \bibliography{math}

\begin{document}

\maketitle
\tableofcontents

\part{任务清单}

\chapter{任务清单}

%\pagestyle{empty}
\todo[inline]{The original todo note withouth changed colours.\newline Here's another line.}
\lipsum[11]\unsure{Is this correct?}\unsure{I'm unsure about also!}
\lipsum[11]\change{Change this!}
\lipsum[11]\info{This can help me in chapter seven!}
\lipsum[11]\improvement{This really needs to be improved!\newline\newline What was I thinking?!}
\lipsum[11]
\thiswillnotshow{This is hidden since option `disable' is chosen!}
\improvement[inline]{The following section needs to be rewritten!}
\lipsum[11]
%\newpage

\section{原生卷}

\begin{tcolorbox}
\begin{compactenum}
    \item 存储池
    \item Mapping
    \item 数据恢复性能
    \item Redis Cache
    \item 快照树
    \item 精简配置,快照对性能的影响
    \item 故障下的性能及其抖动
    \item \hl{async sqlite} [DONE]
    \item \hl{batch sqlite}
    \item Allocate的性能
    \item 单卷快照的数量
    \item VAAI
    \item SSD Cache
    \item 异步远程复制
    \item FC (+VAAI)
\end{compactenum}
\end{tcolorbox}

\section{关键特性和过程}

\begin{compactenum}
    \item flush, load and recovery
    \item 保护模式 safe mode
    \item 存储分层
    \item 命令行工具,扩展卷和快照相关操作到LSV
\end{compactenum}

\section{兼容性}

\begin{compactenum}
    \item 版本演进
\end{compactenum}

\section{Pool}

\begin{compactenum}
    \item Resource Pool
\end{compactenum}

\section{Volume}

\begin{compactenum}
    \item new format: row2
    \item new format: lsv
    \item vol max size 256T+
    \item vol resize?
    \item all zero's chunk
\end{compactenum}

\section{快照}

\begin{compactenum}
    \item 支持60000+快照
    \item consistency group
    \item 每个snap的大小等信息
    \item snap大小对GC策略的影响
\end{compactenum}

\section{一致性/正确性}

\begin{tcolorbox}
\begin{compactenum}
    \item 底层数据检验工具(chunk0, volume, log/gc, bitmap, wal, rcache)
    \item 内置质量,各模块添加自校验机制,方便诊断数据正确性问题(assert + log + test)
    \item 加强断言:pre和post条件,变量变化规则,不变式,基本假设等
    \item 日志用tag/keywork和timeline,以便于跟踪一个对象的变化历史,用一个或多个维度贯穿起来,用于辅助诊断
    \item 增加CHUNK\_HISTORY,以时间线方式,跟踪记录CHUNK变化的生命周期
\end{compactenum}
\end{tcolorbox}

\section{性能}

性能是负载和资源的函数, $P=F(W, R)$。

\begin{tcolorbox}
\begin{compactenum}
    \item 创建卷时,rcache分配了4096M的SSD cache,可以延迟分配
    \item \textcolor{red}{rcache 顺序IO随机化问题}
    \item wbuf 顺序IO随机化问题
    \item 系统启动时间
    \item 预填充lich chunk
    \item GC策略和算法
    \item 统计基础操作的开销,作为性能分析的基础
\end{compactenum}
\end{tcolorbox}

\section{负载}

\begin{tcolorbox}
\begin{compactenum}
    \item IO队列深度
    \item IO平均大小
    \item IO读写大小
\end{compactenum}
\end{tcolorbox}

\section{资源}

\begin{tcolorbox}
\begin{compactitem}
    \item 内存使用量过大
    \item 内存泄漏
    \item 磁盘利用率不足
    \item 网络带宽:瓶颈或利用率不足
    \item 中断
    \item soft lock up?
\end{compactitem}
\end{tcolorbox}

\section{故障处理}

\begin{compactenum}
    \item \change{故障域,不能中断IO}
    \item 节点间负载均衡(<20\%)
\end{compactenum}

\section{Misc}

\begin{tcolorbox}
\begin{compactitem}
    \item FC
    \item Remote copy
    \item SSD cache
    \item EC
    \item 有效容量的比例
    \item 热插拔
    \item 磁盘漫游
    \item 在线扩容
    \item 滚动升级
\end{compactitem}
\end{tcolorbox}

\section{DONE}

\begin{compactenum}
    \item VAAI [+xcopy]
\end{compactenum}


\part{FusionStorage}

\chapter{模型}

\begin{tikzpicture}[show background grid]
    \begin{class}{Disk}{6, 0}
    \end{class}
    \begin{class}{Storage Pool}{6, 2}
    \end{class}
    \begin{class}{Volume}{6, 4}
    \end{class}
    \begin{class}{Host}{6, 6}
    \end{class}
    \begin{class}{Cluster}{0, 2}
    \end{class}
    \begin{class}{Snapshot}{0, 4}
    \end{class}

    \composition{Cluster}{pools}{1..*}{Storage Pool}
    \composition{Storage Pool}{disks}{1..*}{Disk}
    \composition{Storage Pool}{volumes}{1..*}{Volume}
    \composition{Volume}{mapping}{*..*}{Host}
    \composition{Volume}{snapshots}{1..*}{Snapshot}
\end{tikzpicture}

\section{Cluster}

整体

\section{Protection Domain}

把物理节点划分为不同的保护域,一个卷的所有数据只出现在一个保护域内。卷可以跨保护域进行复制和迁移。

默认一个,包括所有节点。

% 保护域是物理节点的划分,存储池是存储介质的划分。每块盘只能出现在一个存储池里。

\section{Pool}

逻辑容器

\section{Storage Pool}

与存储池有什么同和异?存储池可以看做关联了磁盘的pool,可以看做pool的子类。

属性:
\begin{enumbox}
    \item 磁盘列表
    \item 定义精简池
    \item 存储池上可以指定卷的副本数
    \item \hl{有足够的故障域,且不同故障域配置一致的资源量}
\end{enumbox}

操作:
\begin{enumbox}
    \item 创建
    \item 删除
    \item 扩展(添加磁盘到\hl{已存在的存储池},该映射关系持久化到本地,同步到admin节点)
    \item 缩容(从存储池中移除磁盘,引发数据重建过程)
    \item \hl{自动或手动按磁盘速率进行存储池分级划分}
    \item 不同存储池之间,卷的复制
    \item 不同存储池之间,卷的迁移,可在线或离线
    \item 存储池级别的统计信息
\end{enumbox}

% 存储池是disk的集合,与节点无关。但disk所在的节点构成存储池的节点列表,不同存储池的节点可能覆盖。

存储池下,可以创建volume。没有关联磁盘的存储池,不能创建卷。

\hl{chunkid到磁盘物理位置有两级映射:chunk的副本节点列表,节点内chunkid到物理地址的映射}。

在为卷分配chunk的时候,需要确定各个副本的物理存储位置。当前实现是返回不同副本的节点列表。
如果指定了存储池,就需要在存储池所在的节点范围内进行分配。同时要满足故障域和数据均衡规则。

\begin{tcolorbox}
移动采集中存储池要求,相比于目前的逻辑pool,更多是一种设计上的退步。
存储虚拟化的目标,是物理位置无关。我们可以基于逻辑容器,实现基于策略的管理。
所以,\hl{从实现层面,要保留当前pool的功能,按照系统配置确定pool的类型}。
\end{tcolorbox}

% 存储池内,要满足故障域规则(\ref{rule:faultset})

\section{Fault Set}

故障域有粒度之分,如磁盘,节点,机架,机柜,数据中心。

存储池内,要满足故障域规则:一个chunk的不同副本,分布在不同的故障域内。\label{rule:faultset}

在初次分配,再平衡和恢复等过程中,都需遵循这些规则。

\section{Volume}

属性:

操作:
\begin{compactenum}
    \item rename
    \item resize \info{在线扩容}
    \item mv
    \item copy \change{全量拷贝/增量拷贝} \change{跨存储池拷贝} % change不能出现在box里
\end{compactenum}

\section{Snapshot}

snapshot隶属于卷,无卷则无快照,快照组织成快照树,其中有且只有一个快照是可写快照,即卷的写入点。

\section{Mapping}

数据隔离/ACL,数据保护

卷对主机的可见性。一个卷只有映射给了某主机,才可以被该主机访问。

采用白名单机制,但是,cinder需要无验证地访问一个pool。

在建立mapping时,需要host信息。如果有host列表,则可让管理系统去选择。

iscsi采用chap认证。

每个pool上,需要一个属性,表明是开放的,还是封闭的访问模式。如果是封闭的,检查白名单进行认证。


\section{Consistency Group}

一致性卷组

\begin{shadequote}
Consistency Groups could be useful for Data Protection (snapshots, backups) and
Remote Replication (Mirroring).

The Mirroring support will allow to setup mirroring of multiple volumes in the
same consistency group (i.e. attaching multiple RBD images to the same journal
to ensure consistent replay).

There is already an interest to implement this functionality as a part Mirroring feature:
http://tracker.ceph.com/issues/13295

The snapshot support will allow snapshots of multiple volumes in the same
consistency group to be taken at the same point-in-time to ensure data
consistency.
\end{shadequote}

\chapter{硬件架构}

\section{Disk}

\subsection{Cache}

RAID

关闭磁盘 cache,防止出现数据不一致情况。带电池的情况下,可以打开RAID cache。

\subsection{Tier}

检测磁盘分层,支持两个磁盘分层:0和1,0是SSD,1是HDD。

\subsection{Meta}

磁盘管理元数据目录:/opt/fusionstack/data/disk:
\begin{compactitem}
\item disk
\item \hl{bitmap}
\item info
\item tier
\end{compactitem}

对每块磁盘,开头的1M是引导信息,通过bitmap来进行空间管理。
引导信息包含了所在节点信息,所以只能在节点内进行磁盘漫游。

所在源文件是\emph{diskmd.c}。调用fnotify\_register监控磁盘目录的变化,进而添加或移除相应磁盘。

每个节点最多可以添加256个磁盘。

sqlite3划分为10个db文件,chkid信息hash到相应的db。每个db包含两个table:metadata和raw。

\begin{lstlisting}[frame=single]
CREATE TABLE metadata (key text primary key, 
    disk integer, 
    offset integer, 
    parent text, 
    priority integer, 
    meta_version integer, 
    fingerprint integer, 
    wbdisk integer);

CREATE TABLE raw (key text primary key, 
    disk integer, 
    offset integer, 
    parent text, 
    priority integer, 
    meta_version integer, 
    fingerprint integer, 
    wbdisk integer);
\end{lstlisting}

\section{Advanced}

NVMe

RDMA/DPDK/SPDK

AFA

\chapter{优化项}

\section{时间优化}

\begin{itemize}
    \item localize
    \item auto tier
    \item ssd cache
\end{itemize}

\section{空间优化}

\begin{itemize}
    \item 精简配置 (Thin provisioning)
    \item EC
    \item Dedup
    \item Compress
\end{itemize}



\chapter{企业级特性}

\section{Security}

iscsi CHAP认证

\section{QOS}

token bucket

距离函数

\section{Quota}

\section{Multipath}

\section{DR}

snapshot

io journalling

\section{CDP}

\chapter{LSV}

现有Lich raw卷,存在性能问题,COW快照也不便于扩展。所以实现了Log structured Volume,
转化随机IO为顺序IO,基于其上,实现了ROW快照。

特别要注意的是,实现中应着力避免顺序IO随机化,会引起IO放大,从而极大地降低性能。


\section{Volume}

Volume模块负责空间管理。提供malloc/free接口,也可批量分配和回收。采用bitmap和free list多种管理方式。
freelist充当分配缓冲区的角色,可持久化,也可不持久化。

lsv-lich raw-disk的chunk空间存在两级映射关系,会影响到读写性能。

底层空间宜按固定大小的段来组织。每个段空间管理的开销是固定的。
目前支持两级存储分级:
\begin{tcolorbox}
    \begin{multicols}{2}
        \begin{itemize}
            \item 0:ssd
            \item 1:hdd
        \end{itemize}
    \end{multicols}
\end{tcolorbox}

\section{Bitmap}

Bitmap更合适的叫法是页表,与操作系统里的页表类似,负责虚拟地址到物理地址的映射关系。Bitmap有两层:L1和L2,
按类似页表的方式组织。和Log层数据一起,构成三层。

L1是Bitmap的头部,大小固定,属于卷或快照私有。L2按需分配,在快照之间共享。在Clone的情况下,会涉及跨卷读。

通过Bitmap层,支持快照的全部特性,多个快照构成快照树。快照树分两种方式展示:树状或列表。

\section{Log/Chunk}

底层物理空间,划分为固定大小为1M的数据块,进行统一管理:分配/释放。

在Volume模块之上做了简单封装,表示卷的数据,Bitmap表示卷的元数据。在覆盖更新的情况下,Bitmap指向新的数据页,
导致原来的数据页失效,可以回收。在有快照的情况下,会变得较为复杂。

Log模块无需要持久化的信息。

在Lich卷空间映射到磁盘的时候,目前实现为一个随机过程。\textcolor{red}{磁盘的1M随机和顺序,差别较大}。

\section{WBuf}

Wbuf有两个序列:WAL和Wbuf的提交序。在wbuf中读出的最新数据和提交后通过bitmap+log读取的数据,应该一致。

IO内,LBA不同,无冲突,页序;IO间,LBA可能相同,有冲突,需要串行化。

\section{RCache}

多级缓存机制,需要注意针对多种读场景进行优化,如顺序读。因为经过虚拟页表映射,虚拟地址空间和物理地址空间,顺序可能是交叉的。
应着力避免出现顺序变随机导致读放大的情况。

预读很重要,也比较困难,需要构建学习模型。

\section{GC}

log功能单一化,gc模块独立出来。gc要解决的问题有二:
\begin{enumerate}
    \item 跟踪所有log
    \item 在所有log中,根据一定策略(qos),选择回收价值最大者进行回收
\end{enumerate}

目前的实现,是局域的解,而不是全局最优解,是bottom-up的分代垃圾回收器。可增量并行执行,与前台赋值器需要同步机制。
回收器和赋值器需要读写barrier。

优化GC Check过程:每一页的信息,只会出现在部分的bitmap记录里,与快照树的拓扑结构有关。
在创建快照时,分配snap id。 snap id组织成单调递增的序列。如果中间没有删除或rollback操作,
很容易定位到某页所属的快照点。经过rm或rollback之后,情况有所不同,但依然有迹可循。

\section{Recovery}

正常关机的情况下,各个模块会flush必要的数据,下次启动的时候,load出来即可。

异常关机的情况下,各个模块没有机会flush数据,导致丢失部分内存状态信息。
这样,在下次启动的时候,需要执行恢复过程。

需要flush数据的模块有:
\begin{itemize}
    \item Wbuf
    \item GC
    \item Volume
\end{itemize}

提出几个问题:
\begin{tcolorbox}
\begin{enumerate}
    \item 正常关机时,需要flush什么信息?
    \item 恢复过程,从X恢复出Y,X是什么?Y是什么?(X是日志,Y是最新状态)
    \item 怎么理解提交等基础操作?
    \item 恢复的性能如何?如何通过检查点机制改善恢复性能?
\end{enumerate}
\end{tcolorbox}

针对以上问题,每个模块的恢复机制有所不同,但分析方法具有通用性。

\subsubsection{Volume Recovery}

 $U = (A - B) + C + D$

tail标记了可见空间,可见空间=已分配+可分配(free list)。free list组织成内存和磁盘两部分。flush时,需要持久化freelist的内存部分。

在调用malloc和free接口的时候,会同步更新用于空间管理的bitmap。为1的为已分配,为0的为可分配,这个关系总成立。

为了支持批量malloc和free接口,引入dirty page bitmap,类似于GC中提到的卡表,可以实现\textcolor{red}{多次更新,一次提交}的设计模式。

主要操作:
\begin{tcolorbox}
\begin{itemize}
    \item malloc操作:依次从C,D,U里取可用chunk。
    \item free操作:把释放的chunk放入C,如果C满,则转化为D。
\end{itemize}
\end{tcolorbox}

这里的提交操作可以理解为:C转化为D的过程,并没有记录检查点。
所以恢复操作,要全扫描bitmap,从bitmap重建C和D。

\subsubsection{GC Recovery}

GC recovery过程可以理解为:从gc bitmap重建内存状态。

所有的log,分为两部分:old storage和bitmap。bitmap相当于journalling。进入check queue的logctrl,先登记到bitmap。
在提交时,即从heap移入old storage时,清除/注销相应的bitmap项。

\subsubsection{Wbuf Recovery}

谁充当了日志的角色?在wbuf模块很明确,有专门的WAL。写入阶段登记,commit阶段回收。

\section{LSV测试}

LSV(\textcolor{red}{Log Structured Volume})基于Lich原生卷,实现了日志结构的卷格式,支持快照的各种操作。

相对于Lich原生卷,LSV有几点优势:
\begin{tcolorbox}
    \begin{itemize}
        \item 转化随机IO为顺序IO,混合存储情况下有更高性能
        \item 实现为ROW快照,zero-copy快照,\textcolor{red}{支持快照树}
    \end{itemize}
\end{tcolorbox}

LSV的关键过程:
\begin{tcolorbox}
    \begin{description}[style=nextline]
        \item [写] 写入wal和wbuf后,即可返回。wbuf积聚到1M时,提交log+bitmap后台异步任务。
        \item [读] 从wbuf读取最新数据,如果没有命中,则依次从rcache,bitmap+log读取。
        \item [GC] 垃圾回收,后台异步任务,按一定策略,回收无效页。
        \item [重启] 分两种情况,正常和异常情况。正常情况下会刷新内存状态,重启时直接加载即可;异常情况下,进入recovery过程。
    \end{description}
\end{tcolorbox}


LSV测试,主要分为功能,正确性和性能几个方面,\textcolor{red}{正确性和性能按标准测试用例}执行即可。下面列出一期测试计划。

\subsubsection{GIT分支}

lsv\_pipeline

\subsubsection{特性}

%\begin{tcolorbox}
\begin{lstlisting}[language=bash,frame=single]
# 创建LSV卷:
lichbd vol create p1/v1 --size 100Gi -F lsv -p iscsi

# 快照功能,与原来一样,部分命令示例:
lichbd snap create p1/v1@snap1 -p iscsi
lichbd snap create p1/v1@snap2 -p iscsi
lichbd snap ls p1/v1 -p iscsi

# 暂不支持flat操作

\end{lstlisting}
%\end{tcolorbox}

\subsubsection{性能/正确性测试清单}

\begin{itemize}
    \item 与lich原生卷全面的性能对比
    \item 资源利用率(包括磁盘,内存)
    \item 系统启动时间
    \item 重启系统的恢复过程
    \item 存储分层
    \item 扩展到多卷
\end{itemize}

\subsubsection{注意事项}

\begin{itemize}
    \item 日志满:/opt/fusionstack/log/lich.log (echo 5 > /dev/shm/lich4/msgctl/level)
\end{itemize}


\chapter{实现相关}

\begin{itemize}
    \item polling
    \item coroutine and scheduler
    \item mbuffer
    \item kernel bypass
\end{itemize}

\section{Safe Mode}

卷级进行检查,处在保护模式的卷,不允许iscsi连接,返回错误码。

加载时间较长的模块,用half sync/half async模式来处理。分为两阶段:同步+异步。

一个卷,加载成功,依赖于几个条件:
\begin{itemize}
    \item raw/lsv
    \item module load
\end{itemize}

\section{Coroutine}

使用规则:
\begin{itemize}
    \item \verb|schedule_task_get|和\verb|schedule_yield|要匹配
    \item task有数量上的限制:1024,超出后容易引起死锁
\end{itemize}

\section{Memory Management}

\subsection{ymalloc}

\subsection{mbuffer}

\subsection{huge page}

2M

\section{AIO}

生产者-消费者模型。

\section{sqlite3}

异步sqlite3。

\begin{itemize}
    \item 共10个db,每个一个线程,每个线程管理一个队列
    \item 所有sqlite3操作,泛化为统一的结构,放入线程队列
    \item 消费者线程批量处理队列中的任务
    \item 生产者线程和消费者线程通过sem进行通信
    \item 采用协程机制(yield/resume)同步任务执行顺序
\end{itemize}

消费者线程wait在sem上,生产者线程有消息的时候,调用\verb|sem_post|。

\section{RPC}

\begin{itemize}
    \item minirpc
    \item rpc
    \item corerpc
\end{itemize}

\section{Algorithm}

\begin{itemize}
    \item paxos/raft 选举admin和meta节点
    \item vector clock 副本一致性
    \item lease controller的唯一性
\end{itemize}

\chapter{运维视图}

FAQ

\section{问题集}

问题集
\begin{enumbox}
\item lichd --init做了什么
\item 每个节点的nid保存在哪儿,如何分配的?
\item /opt/fusionstack/data目录布局
\item /dev/shm/lich4目录布局
\item cleanup
\item clock机制
\item hsm
\end{enumbox}

\begin{compactitem}
\item coroutine and scheduler
\item polling \change{polling}
\item kernel bypass
\item mbuffer
\end{compactitem}

问题集:
\begin{enumbox}
\item /的位置信息
\item 当前分配的最大卷ID?
\end{enumbox}

P1: IOMeter测试,256K,Lich顺序和随机IO性能差别大

P2: lsv\_gc\_check断言失败

P3: Error Handling

\section{约束}

\subsection{最大副本数}

6

\subsection{单卷最大快照数}


\section{Performance}

\subsection{估算}

HDD:
\begin{itembox}
\item 顺序1M 200
\item 随机1M 50
\item 顺序4K IOPS
\item 随机4K IOPS
\end{itembox}

集群部署
\begin{enumbox}
\item 网络
\item 磁盘
\item 副本数
\end{enumbox}

\subsection{精简卷顺序1M写入性能低}

subvol写锁,导致串行化

用allocate命令先批量填充,性能有显著提升。


\part{基础理论}

\chapter{基础理论}

Lich涉及许多底层理论和系统,包括并行计算和分布式系统,操作系统,文件系统,数据库系统,网络,还包括相应的底层硬件架构。
需要对数据结构和算法,有良好基础。所以,对基本问题和理论,要有清晰和深入的掌握,才能“运用之妙,存乎一心”。

按道法术器组织结构,写成大文章。



\end{document}

\section{201807}

\subsection{0702}

虚静,摄无量义。

无我曰虚,归根曰静。无我而归诸道,心与道合,是为真人。

淡泊明志,宁静致远。

\subsection{0703}

123哲学是分子结构,再往上就是系统论。一个系统由子系统构成,形成层次结构。
系统具有分形属性,一即一切,一切即一。一花一世界,一叶一菩提。

抽身物外,胜物而不伤,勿死于物下。道提供了与物沟通的另一维度,
道者,万物之奥。道者,物之极。架构师与程序员的不同,主要也是在此。
精于道者兼物物,精于物者以物物,下学而上达。

道物,粗分有两个层次,然上通九天,下贯九野,一层功夫一层理。
合中有分,分中有合。

这一关确实不好过了,走还是留,是个问题。不管怎样,都要做好充分的准备。

管理不上路,财务不合规,关键是能不能虚心听取意见,
从中获得成长,一时的成败不是决定性的。

\subsection{0704}

我注六经,六经注我。我与六经之间是超越线性的关系。为今之计,发明心地,明心见性。

寻章摘句,君子不为。以虚壹而静之心态,拥抱现实及其变化,确立道为最高原则,尊道贵德。

归纳整理出我的原则,至关重要。

系统化的决策流程,决策攸关成败,有底层逻辑,有道有术。

守、破、离对应心物,心道、道物三线,成三角形。

\hl{做决策不是我什么什么还没准备好,要相信自己的基本功与学习能力}。
精于道者兼物物,致力于道,物不会是严重障碍。

顶角即是道,也是机器、系统,看到二中之一,看着物理学之后的形而上的东西。
形而上者谓之道,形而下者谓之器。此一上行下行的路径,揭示了更多可能性。

人生算法有认知闭环:感知-认知-决策-行动,是动词构成的,心道物三者,是名词构成的。
内核与外环,内核是最小化的那个点,外环是动力与使命。认知闭环发生在心物之间,
三角形的每个边都是一个认知闭环,PDCA循环。这些小车轮,架起了友谊的桥梁。

是节点问题,还是边问题?居于中心的是什么?

把道、原则、人生算法、多元思维模型、混沌大学课程这些模型融合起来。
打造自己的模型。

取势、明道、优术,取势在心,明道在道,优术在物。
外环由心发动。

夸克构成质子和中子,1:2的比例关系。

把最近围绕道的认知,应用到工作中,在知行中螺旋上升。
一是道心物三角,二是认知闭环,三是体道方法与心态虚壹而静。

稳住,静下来,搞点大事,五年磨一剑,一战定江山。

原则:心态、机器、系统。分生活、工作、投资等领域内归纳出的一些原则。

算法:认知闭环。

多元思维模型:从硬学科里提炼基础模型,形成体系,运用到各种决策场景。

混沌大学:用第一性原理,跨越第二曲线的不连续区。

道具有最终的统一性,众星拱之。

\hl{把分布式块存储系统列入最小内核},运用即即为广泛,深度也够,待解决问题也很重要。
怎么让它最大化呢?占据铁三角的物之一角。要呵护珍惜!

同时需要从别的领域吸收养分,但这个是核心,如果能立下来一个核心,
来华云不管遇到什么,都值了。

\subsection{0705}

体道者逸而不穷,任数者劳而无功。双线法则

战略,不在战,而在略。亮独观其大略。

用心体会虚壹而静四字。

道不欲杂,道是朴素的,一立而万物生矣。

% 如果钱能到,很好。任何时候,成长都是第一位的,如果因为钱影响了成长,就得不偿失。

成长如何衡量?曰道。道是一种信仰,有道则吉,无道则凶。道之有无取决于目标。

\hl{NLP思维逻辑层次}:精神、身份、信念、能力、行为、环境。

前五个都是我,把握当下。

精神=道,身份=我,信念=原则,自此以往,皆算法。

养神之所归诸道,身份是入口、枢纽、关节点。无我,上通于道,惟道是从。
道居于太极至尊的位置,至尊而不独尊。

内静外敬,性将大定。

\subsection{0706}

正念、良知是体道见性知天命的方法。

虚静,一是尊道;二是正念,如如不动。

\subsection{0709}

惟精惟一

打磨三合架构,整合原则、算法,去分析问题

\begin{enumbox}
\item 过以原则为基础的生活
\item 更高层次思考
\item 做一个超级现实的人
\item 极度的头脑开放
\item 五步流程方法
\item 如何做出好决策?
\end{enumbox}

在心物道三者间,持续转动。用三合结构分析达里奥的原则一书。

道者,物之极。升维思考物的真实价值。回到心,是否足够空灵高效有力量,心智模式。

心,极度真实、极度开放。

原则一书,也是升维降维,上帝视角,引入机器、系统,进行控制。
机器位于物的节点,分解为目标与结果、团队与规则。

五步流程法等同于设定目标+认知闭环(感知、认知、决策、行动)。

怎么做出好决策?

限制一下悟道的时间,不必太多,时时提起。

设定下一阶段的目标,全力以赴

帝者体太一、王者法阴阳、霸者则四时、君者用六律。
太一者,理解为目标、内核,有着更为深刻的内涵。

\subsection{0710}

霍金斯能量等级,让人耳目一新。正负能级的分水岭在勇气,
知耻近乎勇,勇气是自我成长的关键要素。

可以看做情商的元素周期表,每次考察一个元素,
改善之,努力向下一能级跃迁。

舍去这些人与事,内心才能真的平静,聚焦在最重要的事情上。
没有舍怎么得呢?幻想没有什么用,拥抱赤裸裸的真实才有出息。

如果让某些人与事影响内心的平静,真的非常不好,牢牢锁定自己的方向、做自己可控的事情。

胜人者有力,自胜者强。不能自胜,何谈其它?
有些事情,转念就好,顺其自然,岂能妄为?

好好消化原则一书,能极大地促进对道的理解。不要一头扎入细节之中,
做到以道观之,按原则行事。类似的书,还有用系统来工作,管道的故事。

先成长为真正的专家,比盲目地开公司,是更可控更现实的事情。
阿里P9、P10的财务收入已不少。在这个基础上,或投资、或创业,更可期待。
再给自己三年或五年的时间吧,不要急、慢慢来。

最近对道的探索,收获颇丰,感觉离大道更近了点,心态也变得更自主、积极,思路更开阔。
这是非常正确的选择,但不能着急。孔德之容,惟道是从。孔者,从容状。

体道会影响到很多方面,心与物。

用PCDA统筹目标五阶段,1+4。一是设定目标,4是四时,PDCA、元亨利贞、认知闭环、四象限等。

如何做出好决策?这是贯彻始终的一个大问题,渗透到每一个环节。

\subsection{0711}

昨日聚餐谈及PDCA,结合原则的1+4,霸者则四时,四时交替、运转不息,
3/4也是很重要的模式。

金字塔的逻辑结构,与太极生两仪暗合,金字塔原理更着重形式逻辑,
太极两仪偏重辩证逻辑。

机器生目标与结果两极,阳变阴合而生金木水火土,五气顺布,四时行焉。
这是二与五的结合,三与四在其中。无极之真,二五之精,妙合而凝。

有此六个数,足矣。

古之王者,建国君民,教学为先。学术是大本大源,故荀子开篇即是劝学。
博学、审问、慎思、明辨、笃行,环环相扣,一气流行。
气没有固定的形状,表示空。

思维格栅,如何才能形成?道、原则的体系展开。
广泛吸收重要学科的核心概念、理论与工具。

道法术器,一气流行。气韵生动,气表明一股存在的无形力量。
不仅要看到形,更要看到神。

\subsection{0712}

太极图说,黄帝阴符经要内化于心。宇宙在乎手,万化生乎身。宇宙、万化皆一心之所裁,本出于身手,由近及远。
三合结构普遍存在,如三盗既宜、三才既安的天地-人-万物之关系。我与非我,统一于大梵之境。
建立自我、追求无我,如此我无我皆入道矣。
偏于任何一方都非究竟之道。二元对立统一方为中道正见。一而不二,是谓知道。如何统一呢?

求道予人一大格局、大视野、大机趣、大静大动。

一存在于二中为三,一存在于四中为五,大部分情况已够用。
一二四是变化序列,三五隐然其中。运转PDCA而不知一,则怠,有术无道。

变化是维度的增加。

PDCA是个周而复始的循环过程,每一个循环就带来了新的可能性,把系统带入一个新高度。

管子有四时一章,论述详备。

李中莹心智力:这个世界由无数个系统组成,每个系统都用着同一套法则运行,称为系统动力。

\subsection{0713}

\section{08}

\subsection{01}

论自由,不仅是论,更在于得到。自由不仅是认知问题,更是实践问题。恒以一德,此一德就是自由。

庄子的自由离活泼泼的现实生活有点远。行动自由是兵家必争,自由则含义更广。抓手在哪里?

2019是自由元年,经过多年磨砺,心智渐趋成熟,可以更自由地去呼吸、去奋争,去为所欲为。

\hrulefill

计算机体系结构要好好学习,胡伟武的教材为主。在一个更广泛的范围内考虑架构问题。
\begin{myeasylist}{itemize}
& 伟大的计算原理
& 多核应用架构关键技术
& 排队论
& 响应式架构
& 计算机体系结构
& 性能之巅
\end{myeasylist}

\subsection{02}

\hl{hegel的哲学,与毕建勋的三合之道}很匹配,互参互证。精神从逻辑学开始,下降到自然界,再回归到精神自身,
从空泛到充实,充实之美。在征途中,攻城略地,海纳百川的气象,又有着攻城略地的开拓进取。

为什么把逻辑置于山顶?心物二元借着道的中介作用,交融为一。

年薪百万是个坎,并不难逾越。应有计划地突破之。此为目标的量化。

拿出三分精力做管理工作,应有章法,目标导向,严格要求。

\subsection{06}

思维科学
\begin{myeasylist}{itemize}
& 一二三哲学
& 轩辕三书
& 道德经
& 王阳明
& 毛泽东
& 大秦帝国
& 黑格尔
& 波普尔
\end{myeasylist}

绝利一源,用师十倍。

\subsection{07}

\begin{shadequote}
读书就应该读功用最大,价值最高的书。人们所做的一切不都是为了有所功用吗?小功用的书比比皆是,通俗易懂,无须费多少心力即可掌握一技一长。
但这是舍本逐末了,既然要读书就应当选择探究终极之道的书,以道贯通天下万事万物,实现最大的功用。

像这样至高经典的学习,决不能跟学习普通知识一样,看几遍就扔了。若想达到极高的效用,一定要熟读成诵本书,一时理解参透不了的,也不必着急,就像牛反刍一样,经典熟记了,
就会在你的思想中蕴化,为你的思想贯通万事万物打下基础。若要思想升华,只读《阴符经》肯定是不够的,因为道是幽微玄奥的,几部经典互通参照,才能更好的认识道。
在后面的文章中。道易学宫还会继续推出诸如《老子》这样的经典。
\end{shadequote}

\say{阴符经、道德经皆为至高经典。吾道一以贯之,用道贯通天地万物。三合之道中道居于顶点,无复多疑,一定要理解其中深意。
阳明良知说,弘扬人的主体性,但并非可以取代道的绝对性。在hegel哲学里,绝对理念是贯穿始终的东西,一步一步到达真理之境。}

\enquote{原则是道的简化,有限版的道。道是开放体系,hegel哲学并非如马克思所云是封闭的脚手架。是有限与无限的统一体。
知性和理性的区分非常重要。\hl{西方辩证法如何过渡到东方辩证法,情、势、节}?}

\hrulefill

\hl{大处着眼,小处着手}。好好悟这个,大处是思维境界,小处是工作境界,吾道一以贯之。
思维、工作、道法构成三合之道。升维思考,降维贯通。一升一降,生命之圆运动是也。

狐狸和刺猬,

继续\hl{租房}住吧,省事了,主要是省时间,省下来的时间好好用来学习,提高自己,再给一年时间,需要一个质的变化。

\subsection{21}

\hl{轩辕三书},千古之绝唱,无尽之宝藏。

用行动改造理念并形成制度。做事、成事,又不是盲目地做。

\hl{一之解,察于天地;一之理,施于四海}。一即是道,也是事。实事求是,才能一以贯之。
两者是贯通的、不是支离的。道寓于事物之中。在观念上是形而上的。
hegel哲学所揭示的,精神的力量、理念的力量、认识的力量甚至比物质的力量要大。
谋事在人成事也在人。

谋势、积微,缺一不可。

提出\hl{一的哲学},合一的力量。三合之道也不是究竟,上遂而及于一的哲学。

知行合一。

实践论、矛盾论

唯物论、辩证法

\subsection{23}

我的第一本书将叫做第一哲学、一的哲学。

\subsection{26}

一的哲学,简称一学。一学非一休,一文一武,一张一弛。

夫为一而不化,得道之本,得事之要。抱道执度,天下可一也。

化书,御一。

一能贯五,五能综一。

吾道一以贯之。知行合一。

用pcda做好管理工作。

\subsection{27}

一学,配合pdca作为主要的管理工具,pdca的每个字母都是一大课题。转动pdca循环,解决现实问题。
比如,联想方法论的复盘,就是c的深化。

目的是什么?思想方法和工作方法,更重要的是学以致用,用方法指导实践活动。
归纳下来,就是一个知一个行,知行合一。

\hl{道治天下},心道物三合,道具中枢地位。以道通物,以道观天下。
原则、多元思维模型都是法,法自道生。事督乎法,法出乎权,权出乎道。

四度,\hl{春生夏长秋收冬藏},周而复始,与pdca相当。


\end{document}

\chapter{COW}

\mygraphics{../imgs/snapshot/snapshot-cow.png}

以volume为主体,如此看来,cow是写入快照,cow(-1)是读取快照。
关键是要能根据chkid构造快照上对应的chkid。\hl{参考get token或recovery过程}。

先实现线性快照。
\begin{myeasylist}{itemize}
    & 快照与物理卷的关系
    & 如何管理卷的快照结构
    & 对任一snapshot而言,有无映射表、元数据头
\end{myeasylist}

两个节点之间如果没有写入,没有发生COFW过程,则上游快照对应slice为空。

\section{属性}

卷和快照增加以下属性:
\begin{myeasylist}{itemize}
    & snap\_version
    & snap\_parent
    & snap\_rollback
\end{myeasylist}

slice增加以下属性:
\begin{myeasylist}{itemize}
    & snap\_version
\end{myeasylist}

Math Mode: $1+1$

判定条件:
\begin{myeasylist}{itemize}
    & 是否需要COW: op == write && volume.snap\_version != slice.snap\_version
    & 是否处在rollback状态:volume.snap\_version != volume.snap\_rollback
\end{myeasylist}

\section{操作}

\subsection{create}

类似2pc过程,在所有rangectl上加锁,成功后,广播新的快照结构信息。
在此期间,io被堵塞。

每个rangectl上维护一lock table,lock table的key是卷的volid。

frctl open volume时,获取当前snap version。如此,client和rangectl都持有snap version,可以用于版本比较。
如果client的snap version较低,则重新加载。如果rangectl的版本较低,也需要重新加载。

\begin{myeasylist}{itemize}
    & 记录任务标记
    & 创建快照对应的物理卷,并持久化卷的快照信息到etcd上
    & 在所有rangectl上,对该卷加锁
    & 广播snaptree到所有rangectl
    & 标记任务完成
    & unlock
    & 清除任务标记
\end{myeasylist}

io携带有snap epoch,如不匹配,则按规则处理。

特殊情况:
\begin{myeasylist}{itemize}
    & 创建第一个快照,是否生成root snapshot
\end{myeasylist}

异常情况:
\begin{myeasylist}{itemize}
\end{myeasylist}

\subsection{list}

组织成tree的形式。etcd上有一个卷的最新快照信息。

\subsection{read}

读卷与读snapshot不同,读卷不涉及快照,因为卷具有完整数据,故不需回溯。这是COW快照的特点。

读快照则向\hl{下游节点}进行追溯。分为两种情况:一是快照后没有写入;二是快照后有写入。

分两种:读卷和读snapshot。读snapshot可能涉及读多个\hl{snapshot的递归过程}。

已知chkid,即可对slice进行各项操作。

\subsection{revert}

进入revert状态时,\hl{发生COW过程的逆过程}。读目标快照的数据,写入卷。

\hl{revert过程中,io可以继续}。对任一slice,如处在revert状态,则先revert后进行io。
如在revert过程中的slice,需要优先触发COW(-1)。

\subsection{delete}

线性结构下,一快照的私有数据是其下游节点之间有写入的slice。

向上归并:merge到父快照。

特殊情况:
\begin{myeasylist}{itemize}
    & 删除第一个快照,没有父节点。可以采用标记删除。
\end{myeasylist}

\subsection{clone}

clone卷使用方式与普通卷完全一样,支持所有普通卷的操作。

clone卷记录其源快照。

相关操作:
\begin{myeasylist}{itemize}
    & flat操作解除clone卷与其源快照的依赖关系,clone卷变成一个普通卷
    & 如果一个快照存在基于该快照的clone卷,则不能删除该快照
\end{myeasylist}

\subsection{flat}

同revert一样,实现为异步操作,集成到job management模块。

\section{IO}

\subsection{write}

如需COW过程。读取卷上数据,写入其父快照。\hl{卷 -> 快照}

相关操作:
\begin{myeasylist}{itemize}
    & 处在rollback状态时,先rollback后write
\end{myeasylist}

\subsection{read}

先关掉token cache。

相关操作:
\begin{myeasylist}{itemize}
    & 处在rollback状态时,先rollback后read
\end{myeasylist}

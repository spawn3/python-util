\chapter{suzaku}

\section{ID}

\subsection{NID}

参考nodeid.c。

\subsection{CoreID and DiskID}

coreid内置nid,diskid通过d2n\_nid函数映射到nid。都需要两次映射进行定位。

\hl{core和disk都是归属于node的资源},导出进行全局调度,why不用同一种形式?

考虑支持\hl{服务器之间的disk漂移特性}。

\subsection{ChunkID}

\mygraphics{../imgs/arch/volume-meta.png}

卷有两层元数据,由此可以算出卷的最大大小。支持精简配置。

\section{ETCD}

\subsection{etcd idx}

更新etcd的KV是个cas过程,避免并发冲突。

\subsection{Pool}

\mygraphics{../imgs/etcd/etcd-pool.png}

\subsection{Metadata}

\mygraphics{../imgs/etcd/etcd-metadata.png}

\subsection{Coreid and Diskid}

\mygraphics{../imgs/etcd/etcd-instance.png}

两个hash ring:rangectl and mdctl。

\subsection{Network}

\mygraphics{../imgs/etcd/etcd-network.png}

\section{MDS}

\subsection{Leader Election}

\subsection{Cluster Map}

\mygraphics{../imgs/partition/mds-master.png}

\mygraphics{../imgs/partition/partition-update.png}

mds master维护两个hash ring信息,如有变化更新到etcd上,slave定期poll该信息。

\section{IO}

\subsection{Allocate}

diskmap

\subsection{Write}

\subsection{Read}

\section{Recover}

\mygraphics{../imgs/rangectl/chunk-get-token.png}

io内恢复

\mygraphics{../imgs/rangectl/recovery-file.png}

外部线程触发恢复,\hl{io、恢复、平衡都是rangectl协作}进行。

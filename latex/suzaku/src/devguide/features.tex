\chapter{特性系列}

\section{存储池}

可以把各个节点的磁盘划分到不同的存储池里,存储池是对集群的物理划分。

存储池可以方便地进行扩容或缩容,只需要把盘加入存储池即可实现存储池的扩容。
后台平衡过程会采用智能策略进行数据再平衡。

\section{精简配置}

\section{自修复}

在检测到故障时,按存储池进行自修复。修复过程采用并行架构,可以进行快速修复,
也可以通过QoS策略控制恢复占用的带宽。

\section{自平衡}

独立的后台任务调度器按预定策略执行数据再平衡,保证每个卷的数据均匀地分布在所在存储池的所有磁盘上,从而最优化磁盘利用率。

\section{QoS}

有两类QoS策略:
\begin{myeasylist}{itemize}
& 卷的QoS
& 恢复的QoS
\end{myeasylist}

可以通过卷的QoS策略属性限制卷的IOPS、带宽,以避免热点卷占用太多的存储资源,影响到别的业务。

为了最小化故障情况下恢复进程对前端业务的影响,可以通过QoS策略限制恢复的带宽占用。

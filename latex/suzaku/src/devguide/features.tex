\chapter{特性系列}

\section{存储池}

存储池是对集群的物理划分,可以把各节点的磁盘划分到不同的存储池里,

存储池易规划,易扩容,只需要把盘加入存储池即可实现存储池的扩容。
后台平衡过程会采用智能策略进行数据再平衡。整个扩容过程无需业务系统停机,对用户完全透明。

\section{精简配置}

精简配置按需分配存储资源,当所需存储资源不足时,可以及时进行扩容。

卷的数据块记录在元数据里,只需要分配实际使用过的数据块即可,没有访问过的数据块不占用任何资源。

\section{QoS}

系统的用户体验至关重要,各类任务会竞争共享资源,不同卷也有不同的IO访问模式和负载,
必须统筹调度,以保障所有任务能井然有序地执行。

QoS是个动态过程,存在两类QoS策略:
\begin{myeasylist}{itemize}
& 卷的QoS
& 恢复的QoS
\end{myeasylist}

可以通过卷的QoS策略属性限制卷的IOPS、带宽,以避免热点卷占用太多的存储资源,影响到别的业务。

为了最小化故障情况下恢复进程对前端业务的影响,可以通过QoS策略限制恢复的带宽占用。
恢复的QoS策略是按存储池定义的。

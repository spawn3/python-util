\chapter{特性系列}

\section{存储池}

存储池是对集群的物理划分,可以把各个节点的磁盘划分到不同的存储池里,

存储池可以方便地进行扩容或缩容,只需要把盘加入存储池即可实现存储池的扩容。
后台平衡过程会采用智能策略进行数据再平衡。整个扩容过程无需业务系统停机,对用户完全透明。

\section{故障域}

故障域规则指的是:一个数据块的各副本存在不同的故障域里。
通常按节点、机架等集群拓扑结构定义故障域。

系统在任何情况下都不能违反故障域规则。

通过定义故障域,可以降低多副本同时发生故障的概率,有效地提升了系统可靠性。

\section{精简配置}

精简配置按需分配存储资源,当所需存储资源不足时,可以及时进行扩容。

卷的数据块由元数据进行跟踪,只需要分配实际使用过的数据块即可,没有访问过的数据块不占用任何资源。

\section{自修复}

在检测到故障时,系统按存储池自动执行修复任务。

修复过程采用并行架构,通过调动存储池内的相关资源,可以快速修复,
也可以通过QoS策略控制恢复过程占用的带宽。

\section{自平衡}

在扩容或缩容后,数据可能处在不平衡状态,磁盘利用率大小不一,系统需要能够重新回到平衡状态。

独立的后台任务调度器按预定策略执行数据再平衡任务,保证每个卷的数据均匀地分布在所在存储池的所有磁盘上,
这样数据在存储池的各个磁盘上是平衡分布的,从而最优化磁盘利用率,并提高磁盘的平均使用寿命。

\section{QoS}

系统的用户体验至关重要,各类任务会竞争共享资源,不同卷也有不同的IO访问模式和负载,
必须统筹调度,以保障所有任务能井然有序地执行。

QoS是个动态过程,存在两类QoS策略:
\begin{myeasylist}{itemize}
& 卷的QoS
& 恢复的QoS
\end{myeasylist}

可以通过卷的QoS策略属性限制卷的IOPS、带宽,以避免热点卷占用太多的存储资源,影响到别的业务。

为了最小化故障情况下恢复进程对前端业务的影响,可以通过QoS策略限制恢复的带宽占用。
恢复的QoS策略是按存储池定义的。

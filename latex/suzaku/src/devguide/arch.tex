\chapter{系统架构}

\section{硬件架构}

\section{软件架构}

% \mygraphics{../imgs/arch/system-arch.png}

\mygraphics{../imgs/suzaku/suzaku-function.png}

\subsection{块级虚拟化}

\subsection{分布式架构}

分布式架构具有良好的可扩展性,可以方便地进行垂直或水平扩展。

\subsection{全用户态}

采用kernel bypass技术,达到极致性能。

\subsection{元数据服务}

内置元数据服务,每个卷的元数据均匀分布在后端磁盘上,支持的最大单卷容量可达4PB。

在内置元数据的辅助下,可以受控地进行数据恢复、平衡等后台任务,以最小化对前端业务的影响。

\subsection{负载均衡}

首先有数据均衡,然后有负载均衡。

数据均衡有两个过程保证:首次分配和再平衡过程。

卷可以通过任一tgt控制器导出,卷进一步划分为子卷,每个子卷负责管理该卷的一部分数据,
子卷通过hash过程均匀分布在不同存储节点的多个子卷控制器上。
这样,数据和负载都能平衡地分布在存储池内的所有节点和磁盘上。

% \subsection{控制路径和数据路径分离}

% \subsection{丰富的存储特性}

\subsection{面向全闪的系统优化}

SSD盘的优势在于随机IO性能好,时延低,劣势在于擦写次数有限。
系统采取了适当措施,充分利用相关硬件特性,以实现更好的可靠性和性能。

\chapter{Code}

\section{Principles}

\begin{myeasylist}{itemize}
& 可重用性
& 可测试性
& 多重安全机制
\end{myeasylist}

\section{进程结构}

目前每个服务器节点启动\hl{两个进程:ioctl和mdctl}。
ioctl包括tgtctl/frctl/rangectl/bactl等控制器。mdctl包括mdctl控制器。

\subsection{控制器结构}

\mygraphics{../imgs/arch/transport.png}

RDMA从poll开始,深度优先的遍历策略。收到消息后,按消息类型派遣到不同的handler去处理。

tgtctl是target与suzaku的界面,\hl{target逻辑运行在tgtctl上,支持多种target:iSCSI/iSER/NVMf}。

每种controller都监听网络,外部可以与之通信。tgtctl与frctl之间通过ringbuffer进行单向通信。

\mygraphics{../imgs/arch/ctl-dep.png}

按服务去理解,各个服务支持什么rpc
\begin{myeasylist}{itemize}
& tgtctl (\hl{frctl/volume/target.c})
& frctl (frctl/volume/volume.c)
& rangectl (frctl/controller/range.h)
& mdctl (mdctl/mds\_rpc.h)
& bactl (bactl/cds\_rpc.h)
\end{myeasylist}

\section{代码规范}

结构
\begin{enumbox}
\item 命名规则
\end{enumbox}

函数
\begin{enumbox}
\item 行数
\end{enumbox}

MM
\begin{enumbox}
\item buffer\_t
\item coroutine stack
\item 小对象
\end{enumbox}

\section{Debug}

调试代码,要跟踪backtrace,要跟踪消息流向,即消息的生命周期,要比对时间线。
\begin{myeasylist}{itemize}
& module
& assert
& log
&& message flow
&& timeline
&& backtrace
\end{myeasylist}

\hl{按时间线trace消息流向}是强有力的跟踪法。

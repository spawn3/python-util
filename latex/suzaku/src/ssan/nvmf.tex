\chapter{NVMf}

\section{Getting started}

默认NVMf不监听后端网络,即suzaku.conf里配置的网络,所以至少需要一个不同的前端网络。

\subsection{RDMA}

NVMf卷attr,只能被该协议访问。

no handler found for RDMA transport

\begin{myeasylist}{itemize}
    & modprobe nvme\_rdma
    & modprobe nvme\_fabrics
    & ***
    & ERROR: RDMA listen 0.0.0.0 
    & ERROR: link static libibverbs
    & ***
    & use github nvme-client
    & ERROR: mlnx mln\_compat
\end{myeasylist}

\subsection{nvme-cli}

\mygraphics{../imgs/nvme-list.png}

\subsection{NVMf}

NVMf的initiator的安装
\begin{myeasylist}{itemize}
& CentOS 7.6
& IB Driver
& client
&& nvme-cli (nvme\_rdma, nvme\_fabrics)
&& spdk/perf
&& multipath
& info
&& /sys/class/nvme/
\end{myeasylist}

\section{Concepts}

NVMf的RDMA所以一个一个处理,是因为重用req?

如何标识一个卷?在分布式系统中,卷的标识应独立于节点。

subsystem和ns如何映射到分布式环境下?nqn也不因为在节点之间漂移而变化?

subsystem是节点内的概念吗?不是,需要有全局标识。多个host可以通过不同节点连接同一subsystem。

采用\hl{网络协议栈的分层架构模型}去理解NVMf,以及代码阅读的经验谈。

NVMf的RDMA实现性能如何?

nvmf上每个core上启动一个subsystem,每个subsystem包含若干session,session包含connections。
cq是connection级别的。

poll线程不能太多?

nvme-cli为什么能列出PCI NVme和NVMf挂载的设备?这两种设备有着相同特征。

NVMf:从RDMA transfer看起,怎么建立连接,怎么send and poll。 
每个core对应一个subsystem,每个subsystem包含若干session、每个session包含若干连接。连接关联到transport。

在core map里维护卷到core的映射。

discovery机制:

\section{Code Reading}

\begin{myeasylist}{itemize}
& nvmf
& transport (rdma)
& request
& subsystem
& session
& volume
& suzaku\_io
\end{myeasylist}

\subsection{nvmf-session-connect}

\mygraphicsh{../imgs/nvmf/nvmf-session-connect.png}

只有一个tgt的情况,建立两个session,每个session包含1个admin连接和2个io连接。

如果有多个tgt,可以横向扩展。

单卷的性能,既受前端网络的影响(listen了所有的前端网络),又受tgtctl数量的影响。

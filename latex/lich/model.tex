\chapter{领域模型}

定义基本物理概念和逻辑概念。

节点,存储介质

集群,保护域,存储池,故障域

pool, volume, snapshot等基本概念。


\begin{tikzpicture}[show background grid]
    \begin{class}{Disk}{8, 0}
    \end{class}
    \begin{class}{Storage Pool}{8, 2}
    \end{class}
    \begin{class}{Volume}{8, 4}
    \end{class}
    \begin{class}{Host}{8, 6}
    \end{class}
    \begin{class}{Cluster}{0, 2}
    \end{class}
    \begin{class}{Snapshot}{0, 4}
    \end{class}

    \composition{Cluster}{pools}{1..*}{Storage Pool}
    \composition{Storage Pool}{disks}{1..*}{Disk}
    \composition{Storage Pool}{volumes}{1..*}{Volume}
    \composition{Volume}{mapping}{*..*}{Host}
    \composition{Volume}{snapshots}{1..*}{Snapshot}
\end{tikzpicture}

\section{Cluster}

整体

\section{Protection Domain}

把物理节点划分为不同的保护域,一个卷的所有数据只出现在一个保护域内。卷可以跨保护域进行复制和迁移。

\section{Storage Pool}

\begin{tcolorbox}
移动采集中存储池要求,相比于目前的逻辑pool,更多是一种设计上的退步。
存储虚拟化的目标,是物理位置无关。我们可以基于逻辑容器,实现基于策略的限制。
\end{tcolorbox}

支持的操作:
\begin{compactenum}
    \item 创建
    \item 删除
    \item 扩展
    \item 缩容
    \item 自动或手动按磁盘速率进行存储池分级划分
    \item 存储池上可以指定卷的副本数
    \item 不同存储池之间,卷的复制
    \item 不同存储池之间,卷的迁移,可在线或离线
    \item 故障域规则?
\end{compactenum}

不应该出现在IQN里面。

保护域是物理节点的划分,存储池是存储介质的划分。每块盘只能出现在一个存储池里。


卷和存储池的关系,存储池的CRUD。

diskmap

chunkid到磁盘物理位置有两级映射:chunk的副本节点列表,节点内chunkid到物理地址的映射。

在为卷分配chunk的时候,需要确定各个副本的物理存储位置。当前实现是返回不同副本的节点列表。
如果指定了存储池,就需要在存储池所在的节点范围内进行分配。

存储池内,要满足故障域规则(\ref{rule:faultset})
都需遵循这些规则。

CRC协作类:
\begin{compactenum}
    \item node and disk
    \item volume
    \item pool
    \item protection domain
    \item fault set
\end{compactenum}

存储池是disk的集合,与节点无关。但disk所在的节点构成存储池的节点列表,故障域如果定义在节点级。

存储池下,可以创建pool和volume。

存储池和pool是什么关系?一对一,一对多,多对多。

存储池和保护域什么关系?

存储池包含故障域。

\section{Volume}

支持的操作:
\begin{compactenum}
    \item rename
    \item resize \info{在线扩容}
    \item mv
    \item copy \change{全量拷贝/增量拷贝} \change{跨存储池拷贝}
\end{compactenum}

\section{Snapshot}

snapshot隶属于卷,无卷则无快照,一个卷可以有一个快照树,其中有且只有一个快照是可写快照,即卷的写入点。

\section{Fault Set}

故障域有粒度之分,如节点,磁盘,机架,机柜。

存储池内,要满足故障域规则:一个chunk的不同副本,分布在不同的故障域内。\label{rule:faultset}

在初次分配,再平衡和恢复等过程中,都需遵循这些规则。

\section{映射}

数据隔离/ACL,数据保护

卷对主机的可见性。一个卷只有映射给了某主机,才可以被该主机访问。

应遵循最小权限的原则。

\section{Pool}

与pool有什么同和异。

\section{Consistency Group}

一致性卷组

\begin{shadequote}
Consistency Groups could be useful for Data Protection (snapshots, backups) and
Remote Replication (Mirroring).

The Mirroring support will allow to setup mirroring of multiple volumes in the
same consistency group (i.e. attaching multiple RBD images to the same journal
to ensure consistent replay).

There is already an interest to implement this functionality as a part Mirroring feature:
http://tracker.ceph.com/issues/13295

The snapshot support will allow snapshots of multiple volumes in the same
consistency group to be taken at the same point-in-time to ensure data
consistency.
\end{shadequote}

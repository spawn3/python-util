\chapter{对象模型}

pool, volume, snapshot等基本概念。

\section{Pool}
\section{Volume}
\section{Snapshot}
\section{Volume Group}

\chapter{数据管理}

\section{分配和回收}

\subsection{分配一个chunk的过程}



\section{分布}

\section{平衡}

\section{恢复}

\subsection{需求}

恢复性能和Qos机制。

控制系统后台数据恢复进程所占用的带宽和 IO 处理资源,提供多种数据恢复的优先级策略。

至少提供优先前端应用或优先数据修复这两种优先级策略。在优先前端应用的策略中,数据
修复仅在资源较空闲时进行;在优先数据修复策略中,后台恢复以最快速度完成,
但卷仍然在线保持可用,仅性能有所降低。

作为作业管理,展示恢复进度,关键参数和资源消耗情况。

\subsection{设计}

目标,偏离,反馈,调节,达到控制的目的。分析上下限,临界情况,执两用中。

串行恢复一个chunk的性能,并发情况下的加速比(串行是基准)。

在不受限制的情况下,先优化到最大性能,能否达到硬件瓶颈?

需要有约束的机制和策略。控制并发度和等待时间。

在每个节点上运行恢复线程,分两个阶段:
\begin{itemize}
    \item 扫描,确定需要恢复的chunk
    \item 恢复,多线程执行
\end{itemize}

每个节点只处理属于本节点的数据,即主副本位于本地节点。从sqlite扫描chkid,逐个检查其repnum,
是否与fileinfo一致,如果不一致,追加到一临时文件
\begin{itemize}
    \item chkinfo.repnum是实际副本数(需要进一步查看各副本的真实状态)
    \item fileinfo.repnum是目标副本数
\end{itemize}

控制参数:
\begin{itemize}
    \item 线程数
    \item 线程处理完一个chunk后的sleep时间
    \item 数据恢复的最大带宽
\end{itemize}

\subsection{实现}

配置目录:/dev/shm/lich4/nodectl/recovery

\begin{itemize}
    \item immediately       功能开关
    \item interval          恢复任务的时间间隔
    \item thread            工作线程数
    \item suspend
    \item qos\_maxbandwidth 设置最大带宽
    \item qos\_sleep        设置sleep多少豪秒
    \item qos\_cycle        设置计算带宽的时间间隔
    \item info              输出结果
\end{itemize}

\subsection{测试}

\subsection{运行}

\begin{tcolorbox}
    自动触发:Recover 每10分钟执行

    手动触发:echo 1 > /dev/shm/lich4/nodectl/recovery/immediately
\end{tcolorbox}

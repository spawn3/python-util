\chapter{领域模型}

集群

节点,存储介质

资源池,保护域,故障域

pool, volume, snapshot等基本概念。

\section{Resource Pool}

不应该出现在IQN里面。

保护域是物理节点的划分,资源池是存储介质的划分。每块盘只能出现在一个资源池里。

不同资源池之间,卷的copy和migration。

卷和资源池的关系,资源池的CRUD。

diskmap

chunkid到磁盘物理位置有两级映射:chunk的副本节点列表,节点内chunkid到物理地址的映射。

在为卷分配chunk的时候,需要确定各个副本的物理存储位置。当前实现是返回不同副本的节点列表。
如果指定了资源池,就需要在资源池所在的节点范围内进行分配。

资源池内,要满足故障域规则(\ref{rule:faultset})
都需遵循这些规则。

CRC协作类:
\begin{itemize}
    \item node and disk
    \item volume
    \item pool
    \item protection domain
    \item fault set
\end{itemize}

资源池是disk的集合,与节点无关。但disk所在的节点构成资源池的节点列表,故障域如果定义在节点级。

资源池下,可以创建pool和volume。

资源池和pool是什么关系?一对一,一对多,多对多。

资源池和保护域什么关系?

资源池包含故障域。

\section{Fault Set}

资源池内,要满足故障域规则:一个chunk的不同副本,分布在不同的故障域内。\label{rule:faultset}

在初次分配,再平衡和恢复等过程中,都需遵循这些规则。

\section{Pool}

\section{Volume}

\section{Snapshot}

snapshot隶属于卷,无卷则无快照,一个卷可以有一个快照树,其中有且只有一个快照是可写快照,即卷的写入点。

\section{Volume Group}

与pool有什么同和异。

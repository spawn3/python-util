\chapter{Scheduler}

\section{实现}

系统可以指定几个core,专门处理IO。每个core关联一个core thread,运行scheduler代码。

本质上,所谓的scheduler,就运行特定指令序列的操作系统线程,通过操作特定的数据结构和上下文切换,
模拟协程行为。

别的线程要与core thread通信,需要进行同步。core thread内部,不需要同步措施。

调度器和协程代码,交错执行。调度器切换到task,则执行task指令;
task让出控制权,或者返回,则执行调度器指令。

创建task,通过makecontext管理task要执行的过程,通过swapcontext在调度器和task之间切换上下文。

task的生灭过程,一个task的生命周期,状态变迁:FREE, RUNABLE/READY, RUNNING, SUSPEND, BACKTRACE。

BACKTRACE用于打印日志,扫描每一个task,检查task的存活时间。目前,不允许有特长时间的任务存在。
BACKTRACE是由调度器发起的。BACKTRACE不检查sleep状态的task。

每个task,有parent,从而构成task树。

目前维护有三个task队列(runable, reply\_local, reply\_remote),两个新的申请队列(wait\_task, request\_queue)。

task队列占用tasks槽位,总数受限,\hl{目前默认为1024个槽位}。

调度器支持优先级队列:引入不同于runable的队列,多队列之间分配时间片。

\section{使用}

\subsection{接口}

\subsection{使用模式}

\subsection{FAQ}

\begin{compactenum}
\item schedule\_yield1 timeout。检查的是\hl{从yield到再次唤醒的时间,而不是任务的age,一个task可以多次yield/resume}
\item tasks槽位占满后,再加入的请求会放入wait list。task和wait list之间,可能形成相互依赖的deadlock。
\item schedule\_task\_get和schedule\_yield1必须配对使用
\end{compactenum}

\chapter{企业级特性}

\section{Snapshot}

\begin{tcolorbox}
需要支持的快照操作:
    \begin{itemize}
        \item create
        \item rm
        \item list
        \item rollback
        \item clone
        \item cat
        \item flatten
        \item protect/unprotect
    \end{itemize}
\end{tcolorbox}

采用COW,ROW或两者的组合形式,各有优缺点。

COW方式的快照,卷有完整的索引结构。别的快照点,只有增量的索引结构和发生更新的数据。
每个快照,存储的是创建之后,到下一个快照点之间发生更新的所有数据块。所以需要尽量降低发生copy的开销。
适用于频繁且具有局部性的热点负载场景,固定时间段内,每次copy的开销以及\textcolor{red}{复制集的大小}。

ROW方式下,如果meta不发生COW,新的快照点并无完整的索引结构,读过程需要沿着快照链向上回溯。
并且,rm,rollback等操作需要合并快照点。

如果发生了COW,索引项和数据项引用关系不再是1:1,而是多对多,需要专门的GC机制。
但rm,rollback等操作实现起来变得简单。

数据项的粒度,定长或变长,不同的负载,读写性能有不同的影响。
发生COW的粒度,索引结构管理的粒度决定了数据项的粒度。

IO粒度和数据项粒度的关系,如果IO粒度大于数据项粒度

如果IO粒度小于数据项粒度

这条规则是否永远成立:\textcolor{red}{快照树的任一路径,都需要具有完整的索引结构和数据项集,可以有冗余和共享}。

\begin{tcolorbox}

假设有新旧两个快照点S1, S2,每个快照点包括meta和data,提炼出的几个引导性问题:

\begin{enumerate}
    \item 数据块的粒度,page,chunk,extent?
    \item 哪一个是写入点?
    \item 每个快照点是否有完整的索引结构?
    \item meta是否发生COW?
    \item data是否发生COW?
    \item 卷(写入点)是否有完整的索引结构?
    \item 快照操作(create, rm, rollback, clone, flatten, ls, cat, protect/unprotect)的复杂度?
\end{enumerate}

\end{tcolorbox}

\subsection{快照树}

一颗树的节点,可以分为三类:
\begin{itemize}
    \item 根节点
    \item 中间节点(有无分支)
    \item 叶子节点
\end{itemize}

\subsection{创建快照}

原有写入点变成只读,创建新的写入点,并复制L1元数据。

\subsection{删除快照}

快照树上不同的节点,需要不同的删除过程。对于叶子节点,直接删除即可,\textcolor{red}{需要回收数据吗}?

\subsection{列出快照树}

\subsection{写}

写是ROW的重定向过程,可能发生复制第二层元数据的过程。

\subsection{读}

如果不复制元数据,ROW实现的读过程需要回溯快照树,性能不佳。
如果复制元数据,则每一快照点都具有完整的索引结构,可以做到一次即可定位。

复制元数据,快照和数据具有多对多的引用关系,相当于共享数据块。

\subsection{回滚}

回滚并不会重用回滚到的快照点,而是相当于把写入点切换到目标快照。
写入点本身是一个独立的快照结构。

回滚后的写入过程

\subsection{Clone}

clone后,会用到跨卷读快照,类似于ROW过程,所以,创建和clone过程具有相似性。

cat, protect, unprotect, flatten

\section{Security}

iscsi CHAP认证

\section{QOS}

token bucket

距离函数

\section{Quota}

\section{Multipath}

\section{DR}

snapshot

io journalling

\section{CDP}

\chapter{移动集采}

产品评估:功能和质量。质量包括:可靠性,性能,QoS,可扩展性(负载均衡),用户体验(管控,交互,接口)等方面。

性能不是一个值,而是不同场景下的特征曲线。性能是系统配置和负载的函数:$P=F(S, W)$。
精简配置,快照,故障,实现机制等因素都会影响性能及其抖动。

\section{重要问题}

\subsection{存储池}

模型:概念及其关系。

重建状态和速度按存储池进行统计。

\subsection{卷}

\begin{enumbox}
    \item 精简配置
    \item 三副本
    \item 扩容
    \item 卷的QoS:IOPS
    \item 全量拷贝和增量拷贝
\end{enumbox}

三副本 \info{三副本性能}

\subsection{快照}

\begin{enumbox}
    \item 显示卷和快照的创建时间,名称,宿主,\hl{容量,存储池}。
    \item 快照树是必须的
    \item 快照验证:每个快照关联一个数据集。
\end{enumbox}

\subsection{映射}

数据访问和隔离机制,应按\hl{最小权限原则设计}。主机仅能访问映射到该主机的卷

\subsection{故障处理}

\begin{enumbox}
    \item IO抖动
    \item 重建
\end{enumbox}

\subsection{数据恢复}

触发策略,QoS,性能,状态。

\begin{enumbox}
    \item 在线调整策略:应用优先,或恢复优先。
    \item 负载调整到原来的20\%时,数据重建效率。
    \item 显示恢复速度和状态,区分实际重构的,和跳过的
    \item 拔出硬盘的存储池降级(数据冗余度发生下降,目前恢复进程是节点级的,与存储池没有直接关系,并且,
        在节点的维度上,存储池是有覆盖的,overlay network,\hl{按卷进行汇总})
    \item 磁盘漫游,存储系统不发生重建,且数据无异常
    \item 模拟故障:单磁盘,单节点,单机架。要求:无读写中断。
\end{enumbox}

\subsection{负载均衡}

均衡有数据均衡和负载均衡之分。

单卷的负载均衡,因为FusionStor卷控制器绑定到core上,只能利用单核能力。
CPU利用率上有瓶颈。同时,开启polling模式,会独占core cpu。

\subsection{Misc}

VAAI

\section{不需要做的项}

\begin{enumbox}
    \item 一致性组
    \item 同步/异步远程复制
    \item EC
\end{enumbox}

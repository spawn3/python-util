\chapter{缓存}

\section{Redis Cache}

卷控制器所在节点,IO入口处,分页。当切换卷控制器的时候,清空redis cache。

采用redis LRU或FRU算法。

\begin{compactitem}
    \item \verb|volume_proto_write|
    \item \verb|volume_proto_read|
\end{compactitem}

read过程,从io的两段向中间压缩,选取中间最大的加载区。

若发生volume controller切换,源端能否及时感知该事件,会有清理过程吗?怎么使源端的cache数据失效?
在cache里,有两级检查机制:卷和页。有总开关,分级,如果设定卷级状态失效,则全体页也是失效的。

为什么会发生volume controller切换?主动move,节点故障,iscsi session切换等。

切走,失效化,切回。如果在切回之前没有失效化,则有\hl{cache一致性问题}。

如何检测到该事件?

\begin{tcolorbox}
每发生一次切换,\verb|info_version|即+1,在redis key里加入该信息。
利用redis自身的置换机制,则切换不会造成cache一致性问题。
\end{tcolorbox}

\begin{compactenum}
\item config: \hl{功能开关}
\item 初始化
\item \verb|info_version|
\item 并发
\item 内存copy
\item batch update redis (mset and mget)
\item 存在,则不kset?
\item redis连接可以用unix domain,相比tcp方式提速1倍左右
\end{compactenum}

遗留问题:
\begin{enumbox}
\item 顺序读,分页后,性能下降
\item 随机读,破坏局部性
\item 内存cache?
\item 必须考虑基准性能,SSD read和redis read,IOPS哪个多?
\item \hl{快照回滚对cache的影响}
\end{enumbox}

即使是连接本地redis,tcp开销也是很大的,最多3w IOPS。如果缓存命中率低,或者分页引起的tcp开销,反而会导致读写性能下降。

实现cache有一些通用问题和瓶颈,可以提前指出。

\section{08}

\subsection{0813}

制定下一阶段的专业学习计划,为下一步的发展奠定良好基础,争取在不远的将来,取得事业发展的大突破。

念念不忘的10X成长,缺少可以操作的形式,需要进行细化。

一整天纠结在为学为道的关系上,显得有点焦虑,没有归一,各种事务格格不入,应化为一事,以做事来涵养心体。
这样无论做何事,都不会陷入单纯事务的境地,而是牢牢把握住头脑,以心为主,不为物役,物物而不物于物。

道是求之不得、弃之不离。紧巴着心去求道,反而不得安闲。道者,贵宽贵舒,贵周贵密。若无故纠结彷徨,即是非道也。
平常心是道,自然无为,清静以为天下正。这层意思要多体会。

行有不得,反求诸己。及时地觉知不良习惯、偏见、习性,而断然改进之。

\hl{问、学、习三者,把问放到第一位}。学与习皆在解决问题。如何问是一门大学问,也是领导力的首要原则,
学记:善问者如攻坚木。先其易者,次其节目,及其久也,涣然而解。问题能把诸多主题贯通起来。
这符合行动学习的基本原则。

思方三式: what,why,拓展式。

\subsection{0814}

块、文件、对象构成统一存储,目前统一存储平台有ceph等。块是最底层的,文件应用在哪些场景?对象为什么会取代文件?
硬件方面的改进,如何影响到软件架构?什么样的软件架构能适用于未来硬件技术的变革?

存储公司的盈利能力如何? 全闪存的市场格局是什么?

未来要研发的功能:删重、压缩、EC等容量优化技术,多活、备份等数据安全技术。

内核的块处理模块,kernel bypass技术,高性能架构。

常用标准、协议、算法等。

以块为中心,慢慢过渡到统一存储,一个存储架构提供多个接口是否是合理选择?why not?

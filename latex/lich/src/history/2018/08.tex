\section{08}

\subsection{0813}

制定下一阶段的专业学习计划,为下一步的发展奠定良好基础,争取在不远的将来,取得事业发展的大突破。

念念不忘的10X成长,缺少可以操作的形式,需要进行细化。

一整天纠结在为学为道的关系上,显得有点焦虑,没有归一,各种事务格格不入,应化为一事,以做事来涵养心体。
这样无论做何事,都不会陷入单纯事务的境地,而是牢牢把握住头脑,以心为主,不为物役,物物而不物于物。

道是求之不得、弃之不离。紧巴着心去求道,反而不得安闲。道者,贵宽贵舒,贵周贵密。若无故纠结彷徨,即是非道也。
平常心是道,自然无为,清静以为天下正。这层意思要多体会。

行有不得,反求诸己。及时地觉知不良习惯、偏见、习性,而断然改进之。

\hl{问、学、习三者,把问放到第一位}。学与习皆在解决问题。如何问是一门大学问,也是领导力的首要原则,
学记:善问者如攻坚木。先其易者,次其节目,及其久也,涣然而解。问题能把诸多主题贯通起来。
这符合行动学习的基本原则。

思方三式: what,why,拓展式。

\subsection{0814}

块、文件、对象构成统一存储,目前统一存储平台有ceph等。块是最底层的,文件应用在哪些场景?对象为什么会取代文件?
硬件方面的改进,如何影响到软件架构?什么样的软件架构能适用于未来硬件技术的变革?

存储公司的盈利能力如何? 全闪存的市场格局是什么?

未来要研发的功能:删重、压缩、EC等容量优化技术,多活、备份等数据安全技术。

内核的块处理模块,kernel bypass技术,高性能架构。

常用标准、协议、算法等。

以块为中心,慢慢过渡到统一存储,一个存储架构提供多个接口是否是合理选择?why not?

今天要解决什么问题?
\begin{enumbox}
\item 无RAID卡时,调用RAID相关功能出现异常 [y]
\item cron触发lich.node --disk\_load [y]
\item 反思lich.node --start时重建symlink的合理性 (跳过标记为不可用的disk)
\item \hl{如何验证是否影响到NVMe设备}
\item \hl{bcache没有reformat,能否作为新盘加入}?
\item 故障磁盘上线后,终止相应的磁盘恢复过程,是否需要独立的GC?
\end{enumbox}

需要管理几个层次上的事情:物理设备/RAID,bcache,lich。

检测到磁盘故障,设置该盘进入offline状态,同时清除对应的disk文件,关闭文件描述符(否则bcache无法register)。
待恢复完成后,清除其他的磁盘文件(block, info, bitmap等)。

若在恢复的过程中,调用了lich.node --start,则会重建disk文件,再次触发检测过程。
这个过程,可以模拟拔盘插盘后自动上线的场景。

检测到磁盘故障时,做标记,一是是否能做上标记;二是即便能标记成功,何时清除该标记?
在拔盘插盘的情况下,不清除标记,就无法自动加入。

问题在于,write EIO会导致lichd进程退出,如果配置restart=on,会自动重启lichd。
需要重新思考的是,\hl{write EIO为什么会导致lichd重启,如何才能做到不重启}?

如果把lich.node --disk\_load做成周期性任务,则无非通过删除symlink来模拟磁盘故障。
同时会干扰整个逻辑:\hl{处在恢复中的磁盘,会进入检测-删除软链接-软链接被修复的循环}。

\hl{只load前后之差值}?即是通过修复raid与bcache之后,多出来的设备。不改变lich.node --start的语义。

\begin{enumbox}
\item Ceph是怎么处理这个问题的?其它系统呢?
\end{enumbox}

\section{08}

\subsection{0813}

制定下一阶段的专业学习计划,为下一步的发展奠定良好基础,争取在不远的将来,取得事业发展的大突破。

念念不忘的10X成长,缺少可以操作的形式,需要进行细化。

一整天纠结在为学为道的关系上,显得有点焦虑,没有归一,各种事务格格不入,应化为一事,以做事来涵养心体。
这样无论做何事,都不会陷入单纯事务的境地,而是牢牢把握住头脑,以心为主,不为物役,物物而不物于物。

道是求之不得、弃之不离。紧巴着心去求道,反而不得安闲。道者,贵宽贵舒,贵周贵密。若无故纠结彷徨,即是非道也。
平常心是道,自然无为,清静以为天下正。这层意思要多体会。

行有不得,反求诸己。及时地觉知不良习惯、偏见、习性,而断然改进之。

\hl{问、学、习三者,把问放到第一位}。学与习皆在解决问题。如何问是一门大学问,也是领导力的首要原则,
学记:善问者如攻坚木。先其易者,次其节目,及其久也,涣然而解。问题能把诸多主题贯通起来。
这符合行动学习的基本原则。

思方三式: what,why,拓展式。

\subsection{0814}

块、文件、对象构成统一存储,目前统一存储平台有ceph等。块是最底层的,文件应用在哪些场景?对象为什么会取代文件?
硬件方面的改进,如何影响到软件架构?什么样的软件架构能适用于未来硬件技术的变革?

存储公司的盈利能力如何? 全闪存的市场格局是什么?

未来要研发的功能:删重、压缩、EC等容量优化技术,多活、备份等数据安全技术。

内核的块处理模块,kernel bypass技术,高性能架构。

常用标准、协议、算法等。

以块为中心,慢慢过渡到统一存储,一个存储架构提供多个接口是否是合理选择?why not?

今天要解决什么问题?
\begin{enumbox}
\item 无RAID卡时,调用RAID相关功能出现异常 [y]
\item cron触发lich.node --disk\_load [y]
\item 反思lich.node --start时重建symlink的合理性 (跳过标记为不可用的disk)
\item \hl{如何验证是否影响到NVMe设备}
\item \hl{bcache没有reformat,能否作为新盘加入}?
\item 故障磁盘上线后,终止相应的磁盘恢复过程,是否需要独立的GC?
\end{enumbox}

需要管理几个层次上的事情:物理设备/RAID,bcache,lich。

检测到磁盘故障,设置该盘进入offline状态,同时清除对应的disk文件,关闭文件描述符(否则bcache无法register)。
待恢复完成后,清除其他的磁盘文件(block, info, bitmap等)。

若在恢复的过程中,调用了lich.node --start,则会重建disk文件,再次触发检测过程。
这个过程,可以模拟拔盘插盘后自动上线的场景。

检测到磁盘故障时,做标记,一是是否能做上标记;二是即便能标记成功,何时清除该标记?
在拔盘插盘的情况下,不清除标记,就无法自动加入。

问题在于,write EIO会导致lichd进程退出,如果配置restart=on,会自动重启lichd。
需要重新思考的是,\hl{write EIO为什么会导致lichd重启,如何才能做到不重启}?

如果把lich.node --disk\_load做成周期性任务,则无非通过删除symlink来模拟磁盘故障。
同时会干扰整个逻辑:\hl{处在恢复中的磁盘,会进入检测-删除软链接-软链接被修复的循环}。

\hl{只load前后之差值}?即是通过修复raid与bcache之后,多出来的设备。不改变lich.node --start的语义。

\begin{enumbox}
\item Ceph是怎么处理这个问题的?其它系统呢?
\end{enumbox}

\subsection{0815}

负面情绪仍太多,诸如YN这样的,何必放在心上。当下有更重要的事情等待去办理,最最重要的是什么?
成长离不开良好的平台,机会那么多,能不能抓住一两个?老子:居善地,心善渊。

成长从善于提问出发。学思辩行,非问不行,不能持久贯通。问,从道心物三合关系开始。心与物对,道则无对。
什么才是好问题?本质的、具体的,抽象的也有大价值。5W2H+XXX,XXX可以是NLP的理解层次,
也可以是道心物、天地人等思考框架。

四不光对应PDCA、也是2x2的分析框架,如SWOT,平衡积分卡、重要-紧急矩阵等,从正交维度去分析。
要特别重视维度一概念,升维思考、降维贯通。道与别的事务不同,一个大的原因就在于维度。存储的关键维度是质量。

计算、存储、网络都是大课题,存储与数据相关,贯穿计算、网络,是个大产业。放开了格局去看,可做的事情很多。
不必计较一时得失,应该做的是充实自己。

经验可以加以反刍,以提炼最大的价值。未经反思的生活不值得过,反思则能极大丰富生活的意义。

要有强烈的问题意识与紧迫感,
今天要解决什么问题?
\begin{enumbox}
\item 继续完善学习计划
\item 以问答的形式,整理存储的知识体系,致广大而尽精微
\item 添加一工具,集中显示recovery、qos的配置和状态信息
\end{enumbox}

\subsection{0816}

大道即可通过易经而理解,细读系辞传,简练以为揣摩,富有之谓大业,日新之谓盛德,生生之谓易。
盛德大业,内圣外王的梦想,通过使命去完成,小小情绪皆当封存。使命、身份、信念是一端,
能力、行动、环境是另一端。由内而外,分威散势,则遂其志向。

功业可为不可期,侍命以待,可为者在自我磨炼,如精金白玉,如切如磋,如琢如磨,下学而上达。

以此洗心,退藏于密,大易足矣,一以贯之。归宗大易,奠定思想基础。

易与天地准,故能弥论天地之道。一切即易,易即一切。一阴一阳之谓道,此语点出易之形上精义。
立二参一,阴阳三合。

定心在中,万物得度。

U形理论的核心就是沉下来,重为轻根,静为燥君。是以君子终日行,不离其辎重。虽有荣观,燕处超然。

居则观其象而玩其辞,动则观其变而玩其占。以一卦为基本单元,六十四种场景了了分明。
动起来,立足于当前形势,向理想态势进行过渡。或交换上下,错综复杂等各种卦变爻变之手法,为我所驱使。
每种场景下,就有合宜的对策可供选择。此完全操之在我也。

今天要解决什么问题?
\begin{enumbox}
\item 评估bcache dirty data的影响
\item 拔cache盘,disable cache (做RAID1?)
\item \hl{下电节点,性能下降(vfm,clock)}
\item \sout{write EIO}
\item recovery when pool deleted
\item 100G文件MD5sum需要多久
\item 配合测试解决问题
\item 完善诊断工具
\end{enumbox}

vfm标记节点故障,可以简化部分检测工作。
io带内恢复,会极大地降低性能。先标记,延迟恢复。

clock不一致引发的恢复流量,对性能也有一定影响。

延迟重建,故障发生后,按规则执行重建,如果只有一处故障,延迟一定时间进行重建。

诊断工具的前提:
\begin{enumbox}
\item 本地、集群(汇总本地状态)
\item lichd在线,lichd无法启动
\item 分离MVC
\end{enumbox}

\subsection{0817}

\begin{enumbox}
\item bcache cache disk破坏后,重启服务器能否恢复正常?
\item 对cache disk做RAID1,效果如何?
\item bcache的dirty data在各种场景下是否行为一贯正确?
\item 作为新盘加入,需要做什么?
\end{enumbox}

拔数据盘,盘符变化,如sda变成sdai,导致/sys/fs/bcache/<uuid>下的链接失效,如何校正?

重启后,能恢复,其中一次出现readitems core,数据损坏。

新盘sdai

\subsection{0820}

Week Plan:
\begin{enumbox}
\item bcache可靠性
\item 增删盘的流程
\end{enumbox}

有待解决的问题:
\begin{enumbox}
\item \hl{进一步完善诊断工具,对系统运行状态了了分明}。
\item 观察lich health的输出是否正确?
\item 关闭chunk内并发,可能存在io error
\item dirty状态的chunk没有恢复?
\item \hl{故障与恢复之间的延迟}?
\end{enumbox}

今天解决的问题:
\begin{enumbox}
\item 对disk的部分访问没有并发保护,导致close文件描述符时出错,应该还有别的并发问题。
\item write EIO不再coredump,效果有待进一步观察。
\end{enumbox}

升级到kernel-4.18.3,bcache表现符合预期。升级方法:
\begin{itembox}
\item rpm -Uvh http://www.elrepo.org/elrepo-release-7.0-3.el7.elrepo.noarch.rpm
\item yum --enablerepo=elrepo-kernel install -y kernel-ml
\item grub2-set-default 0
\item reboot
\end{itembox}

\subsection{0821}

性能优化
\begin{enumbox}
\item 阿里盘古系统
\item Lustre
\item GlusterFS
\end{enumbox}

存储引擎2.0,其上有块,文件,对象等。ElasticSearch运行在大容量、高性能的块上。

TODOList:
\begin{enumbox}
\item 移动不允许预先采用allocate,须优化精简卷性能
\item 节点下电后的性能
\item 性能抖动
\item etcd规模
\end{enumbox}

\subsection{0822}

TODOList:
\begin{enumbox}
\item 拔bcache数据盘,导致kernel hangup,重启服务器后,出现readitems的magic错误
\item udev机制监听热插拔事件,触发lich.node --disk\_load调用,不用cron机制
\item 比较bcache与无bcache模式,以定位问题
\end{enumbox}

udevadm可以监听设备相关事件。

\begin{enumbox}
\item /dev/bcacheN何时出现?
\item cache盘离线后,data盘是否能独立工作?
\item cache盘离线又上线,能否继续工作
\item data盘离线又上线,能否继续工作?
\item RAID strip大小,影响bcache的data offset设置? 见bcache官方文档
\item 采用WT模式如何?
\item schedule.c的诊断方法
\end{enumbox}

Linux的IO架构:
\begin{enumbox}
\item VFS
\item Page Cache
\item Block Layer
\item SCSI
\item Drivers
\end{enumbox}

\subsection{0823}

DayPlan:
\begin{enumbox}
\item 进一步诊断增删盘引起的vdbench IO中断,细分每一个关键步骤,并profile
\item mount -o sync模式copy文件的性能,即100G需要多长时间
\end{enumbox}

加盘分为几个步骤:
\begin{enumbox}
\item raid miss
\item raid load
\item disk link
\item lichd内部处理
\end{enumbox}

RAID配置改变是否会影响IO中断?可以验证,RAID控制器会引起IO中断。

sync, async,direct IO区别是什么,dd,mount等命令可以指定这些选项。

计算、网络、存储是云计算三大基石,存储最有搞头。存储范围很广,企业存储、统一存储、云存储、大数据等,
有同有异。

计算有操作系统,最好能深入理解kernel的整体架构,网络是一个难点,做到理解基本原理。
存储要围绕关键问题,多问多想多动手。

云计算、大数据、AI也是构成一个序列。云计算解决底层基础架构,其上大数据+AI,双剑合璧。

搜索引擎、推进系统、计算广告都是大数据的应用,都涉及机器学习,
知识体系也要围绕这个ABC三角形进行构建。

\hl{先做整理资料的工作}。围绕存储这个支点,构建全方位的知识体系。

标准、协议、算法

任何文件(包括设备文件)都可以格式化为文件系统,然后mount。

按上下左右模式理解块存储的定位,下是硬件,上是应用场景,左是运维平台,右是云平台。
这种思维方式也可以用来理解kernel的各个模块之间的关系。比如VFS。
在Linux的世界里,一切皆文件。

四面八方,立于中央。

拔出一块数据盘,bcache会如何处理与该盘相关的dirty data呢?
该数据盘不在线,故cache无法flush该盘相关的dirty data。\hl{理想的行为,应是一直保有该dirty data}。
如果丢弃了部分dirty data,会导致数据不一致。cache disk的dirty data加上data disk的数据,才是完整数据。

合并成一个分支,这是需要的,资源分散,重复劳动,效率极为底下。

高性能版本通过配置文件启用。增加的模块主要有:
\begin{enumbox}
\item SPDK/NVMe
\item RDMA
\item \hl{DPDK}?
\item iSCSI/iSER
\item NVMf
\end{enumbox}

从存储+网络+计算三方面去加深理解,四面八方。

介质上增加了NVMe支持,\hl{网络走RDMA,对应的网络协议变更:iSER, NVMf}。
RDMA的支持有两处:一是target的RDMA支持,二是存储网络之间的RDMA通信。

NVMe设备有两种使用方式:一是通过/dev/设备文件,采用aio方式; 二是通过pci号直接访问,与控制器直接通信。
后者性能要好很多。

kernel bypass,用户态代码直接与设备控制器通信,设备控制器具有独立的处理器(offload)。
在上而言是target,导出服务的接口、协议,在下而言是driver,资源管理。

介质、网络、协议的细节如何?

\section{09}

\subsection{0901}

\subsection{0902}

知几其神乎?三合之道的三角对应心道物,相应的修行原则对应诚神几。

几有危微两方面的意思。致诚、用神、制机,从而实现改造现实达成目标的现实意义。

鬼谷有持枢中经残篇,枢纽、枢机、枢要,皆重要之点。主要矛盾,当今主要矛盾依然是财务自由这一主题。
实现财务自由即是最大的胜利,至于精神自由固然重要,却易于自我欺骗,不思进取,而贻误战机。

夫道者,体常而应变,常是原则,随机应变,应变要随机。如带中心点的圆环,后顺得常。

\subsection{0903}

删除pool没有完成,此时重启lichd,会残留pool相关状态,如相关的disk文件,影响后续过程。

\begin{enumbox}
\item 显示disk依然处在使用状态,无法加入新的pool里。
\item pool已经不存在,disk依然能够加载成功?
\item 暂时控制恢复进程的命令行工具
\item \hl{用RAFT实现副本一致性}?
\end{enumbox}

添加etcd task,用于处理pool remove过程中的异常中断。\hl{其它bh bask也可以切换到etcd方式}。
因为pool name可能被reuse,会引起难以预料的问题。

\hl{把数据结构、算法和架构的学习和思考放入内功修炼},持续地不中断地打牢基础。

反思recovery的线程与队列组织,有诸多不便之处,无法方便地\hl{hash不同的数据到不同的线程}。
\hl{线程最好有自己的队列},封装成SEDA模型,按策略分配不同的任务到不同的线程。

在拔盘的时候,EIO要封装在底层(EGAIN),不能返回给前端,目前已能正确处理。

问题集
\begin{enumbox}
\item 把chunk作为object layer,如何?其上构建统一存储。
\item ceph架构: CRUSH pool、pg、object
\item 节点内:RAID、LVM、Bcache、VDO等
\item 分布式:RAFT
\item iSCSI, tgt、spdk项目
\item VFM
\item 多存储网
\item Sqlite/LevelDB/RocksDB
\item 从阵列到分布式架构
\item QoS
\end{enumbox}

多个\hl{索引}如果转化成LevelDB的KV模式?

上下左右看,ABC

往下看,看到操作系统乃至硬件,往上看,看到虚拟机、docker/k8s、\hl{企业存储、云存储、数据库、大数据}、乃至AI。
右手螺旋法则,立体河图

dd、mount、fio、vdbench、hazard

\subsection{0904}

分几个方面建立知识体系:
\begin{enumbox}
\item 项目:实战(立足手头工作)
\item 架构:编程、架构(数据结构、算法、模式)
\item 系统:\hl{操作系统、文件系统、数据库、网络}、分布式系统、缓存、MQ
\item 四面八方延伸至ABC
\item 要包含数学这个异常重要的基础
\end{enumbox}

数据模型即是访问层,如统一存储的对象、块、文件,NoSQL的KV、文档、列存、图、关系等。
对块存储系统(SAN)而言,可分两层去思考,一是对象层,二是块层。

大规模分布式存储系统一书的架构组织不错,从理论到实践、从下到上循序展开。

基础知识分单机存储引擎和分布式架构,质量属性是自始至终都要考虑的要素,包括性能、可靠性、可用性与扩展性等(视角)。
\hl{视图(来自视点,以视点为模板)+ 视角的架构描述方法}值得参考。

\hl{事务}的观点、并发控制与恢复的观点、虚拟化和持久化的观点贯彻始终。

单机存储系统需要考虑的要素有哪些?分布式系统增加了哪些复杂度?

从上至下、从下至上双通路考察本地存储系统。

不要受到当前实现方式的限制,从更基础的层面提出问题,并思索方案的可能性空间。
先发散、后收敛。

存在两类数学,一种是\hl{易经上、或神圣几何里的大衍之数},一种就是数学学科,包括纯粹数学和应用数学。
大衍之数体现了数学的艺术性、哲学性。

QoS是一个大问题
\begin{enumbox}
\item 控制目标:下限、上限、突发
\item 集中式控制、分布式控制
\item 优先级
\item \hl{高速pool的时候},QoS机制本身带来的开销太大,导致IOPS严重地低于设定值。
\item dbg模块会生成/dev/shm/lich4/msgctl目录下的若干文件
\item bcache在docker内部不显示虚拟设备
\end{enumbox}

QoS的参考资料:
\begin{enumbox}
\item Ceph的dmClock算法
\item XSKY的漏桶与令牌桶结合的策略
\item SolidFire
\item NetApp
\item SPDK bdev
\end{enumbox}

\subsection{0905}

pci设备?通过pci设备号操作该设备,不经过kernel。

本地存储引擎,带索引的关系模式怎么转换为KV模式?
磁盘空间管理、chkid到磁盘位置的映射关系。

在allocate或discard时,会建立这些关系。读写时,只是访问这些关系。
allocate过程或独立进行、或在使用的时候按需分配(精简配置)。

chunk的metadata,记录了副本的位置信息。所以\hl{最小的故障域是节点}。

副本或条带一致性是最困难的部分。

数据分布(分配、再平衡、恢复、GC)的一般规则:
\begin{enumbox}
\item 按pool组织
\item pool内满足故障域规则
\item chunk满足副本数要求,且副本之间是一致的
\item 容量和负载在节点之间尽量均衡分布
\item 考虑节点的权重
\end{enumbox}

异常情况的处理,比如故障域小于副本数要求。

pool资源组织成一树状结构,pool-rack-node\hl{-disk}。

负责空间分配的admin在内存中维护着这样的结构,在节点启动之后,
会周期性地向admin上报自己本地的pool拓扑信息。

pool的ns,通过pool controller和volume controller跟踪维护。

给定chkid,如何\hl{定位controller所在位置呢}?
后续所有controller相关操作,都要调度到该controller去处理。

已知问题:
\begin{enumbox}
\item controller在节点以及节点内core上的分布是否均匀?
\item 大卷无法有效地进行负载均衡,瓶颈受限于一个core
\end{enumbox}

数据的复制、迁移。

一个系统可分解为基础结构kernel和高级特性,宏内核架构,尽量组织成模块化。
宏内核方便模块之间的通信,但因隔离性差,也容易引入严重问题,影响到全局。

\subsection{0906}

把disk故障也纳入vfm,即不再使用check方法,而是按subvol进行标记,
被标记的subvol就处在可疑状态,待全部确认后,再清除其上的标记。

check标记的是副本,\hl{subvol标记的是一个subvol范围内的所有chunk在某节点上的所有副本}。

subvol标记怎么做到io和恢复的隔离,从而提高性能的?

\hl{扫描顺序的影响}:如果按顺序进行扫描,待扫描完成后,即可清理其上的标记。
如果是在一个大的范围内随机扫描,则只有全部完成,才能清理每个subvol上的恢复标记。

\subsection{0908}

两种故障要尽量统一处理。

加班
\begin{enumbox}
\item disk recovery,增加控制逻辑,动态调节参数,包括开关、恢复速度(统一为disk\_fill\_rate)
\item QoS,精确控制在高IOPS下难以实现,退而求其次,控制node recovery的fill\_rate
\item 在scheduler层面,iops与recovery采用不同的队列
\item fill rate == 0时,\hl{不退出恢复过程,循环等待},scan过程及其结果要保留,此约定适用于两种故障
\end{enumbox}

恢复控制参数
\begin{enumbox}
\item 控制开关 (静态、动态)
\item 带宽
\item 线程数
\item 每请求提交到控制器的chunk数 (最小最大)
\end{enumbox}

两个pool的隔离性,\hl{拔出一个pool的盘为什么会严重干扰另一个pool的iops}?
\begin{enumbox}
\item RAID (WT)
\item DB
\item schedule
\end{enumbox}

\subsection{0910}

TODO list:
\begin{enumbox}
\item \hl{被deleted的卷/快照导致recovery rescan}
\item recovery 合并重复事件
\item 没有IO的时候,限制recovery不合理(Q4)
\item 删除pool过程如果中断,残留下的垃圾影响到后续过程
\item fnotify
\item qos schedule\_sleep delay是固定值,不合理
\item recovery recover的并行化,多次RPC
\item \hl{VFM机制有放大可疑区间的问题}
\item 与节点启动时间变长有关?etcd lock?
\item 验证VFM是否把写时恢复的流量给成功剥离出来了,启用background\_recovery配置选项
\item VFM:IO时skip vfm存在的副本,解耦IO时恢复,交给外部恢复过程处理
\item \hl{ENABLE\_CHUNK\_DEBUG 跟踪chunk的生命周期活动,在controller上捕获尽可能多的信息、条件、事件}
\item 连接频繁断开,触发recovery过程
\end{enumbox}

\subsection{0911}

在整个recovery过程中,如果有卷/快照被删除,如何处理?
目前的策略是退出recovery过程,然后重新扫描?但如果一直在删除快照,会导致频繁地重新扫描。

李刚彬工作交接:
\begin{enumbox}
\item 数据布局
\item 切换时间控制
\item lookup过程,gc特性关,误删数据
\item chunk check
\item snapshot, 数据view,树型快照
\item snapshot,连续删除、标记删除
\item EC 条带一致性,故障下的恢复,log,cache,或先写入临时区域
\item cache
\item etcd选主依赖于物理时间,容易引发问题
\item admin的切换过程,ctl的切换过程,两者可能有叠加。
\item ***
\item 故障诊断 hazard
\item iscsi session切换,原session的IO还在继续处理
\item clock
\item chunk check
\item chunk tree里parent的chkinfo的rep set变了,沿着chunk tree向上跟踪
\item mbuffer不配对,忘了free,导致内存泄露
\item \hl{诊断方法见: 192.168.1.7 文档}
\item 常见问题:可在git仓库里搜索write failure
\end{enumbox}

\subsection{0912}

理解VFM和clock机制
\begin{enumbox}
\item 解决了什么问题?
\item 需要管理什么信息,粒度,这些信息需要持久化吗?
\item 选择什么样的标记粒度?vol, subvol?
\item CRUD 增删改查的条件
\item 不正确设置VFM会带来什么后果?
\item 更好的方案是什么?
\item 可配置选项有哪些?
\item \hl{恢复chunk集是扫描chunk集的子集}
\item 动态重配置问题
\item 一旦标记subvol,则该subvol的所有chunk皆处在可疑状态,需要确认。
\item \hl{一致性分析:异步恢复违反副本的强一致性原则,会带来不可预期的影响}
\item VFM失效,导致iops低
%\item \hl{chunk check的core\_request影响到跨pool性能}?
\end{enumbox}

\hl{故障下的写时恢复机制}极大地影响了IO性能,解耦IO与恢复两种活动。

在检测到故障时,记录nid到VFM,在恢复完成后,清除相关标记。
在读写的时候,skip VFM列表中的副本。处在vfm列表里的副本,可能处在\hl{stale状态}。
如果按chunk标记该状态信息,因为需要持久化,成本比较高。

\hl{在恢复的时候,如何确定scan和恢复的策略}?

只对raw chunk采用vfm机制来实现异步恢复,meta chunk数量少,且更重要,最好同步恢复。

vfm按subvol进行标记,有放大扫描范围的倾向。

\subsection{0913}

理解QoS
\begin{enumbox}
\item QoS: 分层分类控制。分层:节点、core、controller,分类:IO、Recovery。
\item QoS尽量部署在上游,靠近输入端,什么是end-to-end?
\item 基于类是最佳实践
\item 所有算法都是基于token bucket。token bucket有很多种实现形式
\item 测量:依赖于高精度的定时器,且测量干扰性能(\hl{测不准原理})
\item 调度:network scheduler,基于队列(显式或隐式)
\end{enumbox}

高IOPS就类似于从牛顿力学进入了量子世界。

性能、可用性、QoS三特性一起考虑。故障分两种:节点故障和磁盘故障。

先分析跨pool影响,再分析pool内影响。

从资源使用的角度分析故障的跨pool影响,故障下,恢复流量会用到内存、网络资源,sqlite?
磁盘是隔离的,core也是隔离的。

节点之间的corerpc只有一个连接。两个卷映射到不同的core hash上。

简化测试环境:\hl{两个pool,每个pool一个卷(映射到不同的core hash上),做故障}。
从简单到复杂,先确定简单情况下没有问题,然后再推广到更复杂情况。

\begin{enumbox}
\item background recovery off
\item \hl{确认是否有带内恢复流量},统计chunk\_push被调用的次数,按pool/vol分组
\item clock merge带来的性能抖动
\item chunk bh是干什么的
\end{enumbox}

\subsection{0914}

问题集
\begin{enumbox}
\item getnode返回estale
\item disk故障,没有加入vfm,导致带内恢复
\item recv/send是block操作,堵塞scheduler主循环
\item qos disable overflow
\end{enumbox}

\subsection{0917}

\begin{enumbox}
\item 对标ceph,如何体现竞争力?
\item AFA是大势所趋,如何把握机会窗?
\item 客户在哪里?
\end{enumbox}

为平滑删除资源,采用垃圾桶、skip、维护模式等\hl{设计模式}进行。
本质上是设计资源的状态机,不同的状态允许不同的操作,通过区分不同的状态来达到目的。

问题是如何持久化资源的状态,以满足ACID等基本属性?

\subsection{0919}

\begin{enumbox}
\item volume\_lease use 3s
\item recovery immediately
\item 单磁盘故障,恢复影响io性能,为什么
\item vfm影响io性能
\item 恢复影响io性能
\item etcd版本不一致导致问题
\end{enumbox}

工作职责
\begin{enumbox}
\item 解决产品中遇到的问题
\item 维护存储引擎以及恢复、平衡、QoS、快照等模块
\item 优化现有系统架构,评估并引入关键功能
\end{enumbox}

\subsection{0920}

目前,主要有两类问题:\hl{资源管理和故障处理}。资源管理特别是资源的删除,一是要考虑事务性,二是要考虑性能。
比节点、pool、卷、快照等。

故障处理,包括磁盘和节点故障,事件触发恢复,需要综合考虑可用性、性能和QoS等指标。

以磁盘故障为例。从\hl{事件处理}的角度来看,事件、事件处理、新的事件可能会中断前面的事件处理过程等。
从资源的角度来看,资源建模为状态机,状态机模型可以融合事件处理。\hl{状态-事件构成状态机,或矩阵}。

节点状态分两个维度:out/in,up/down,实际状态是以上维度的组合。把状态按维度 
从执行线程的角度来看

事件:
\begin{enumbox}
\item 磁盘故障,开始恢复
\item 磁盘上线
\end{enumbox}

用CAP理论分析VFM的设计,采用了异步复制,提升了A,而降低了C。
对分布式块存储系统,C是第一位的,任何情况下都不允许丢失数据的后果。

\subsection{0921}

推广事务概念到分布式系统,CAP理论是分布式系统所特有。融合两个理论,中心是C和A。A有二义:原子性、可用性。

\hl{数据模型分布与复制}是根本需求,由此带来了要解决的一系列问题,如一致性、故障处理、负载均衡、性能、容量等。

先看数据模型或范式,如文件系统、数据库、KV、文档、对象、列存、图、搜索引擎等。
NoSQL、NewSQL如雨后春笋,否定之否定螺旋进化。虚拟化、分层堆叠。

围绕着数据来思考,数据结构、模型、关系、存储、提取、分析等等。
数据管理在整个业务开发中最为复杂。

对象是统一的概念模型,在对象层之上构建更复杂的数据模型。文档的V是结构化的、列存是KV之间有一定的模式。

\hl{事件、模式、结构冰山模型},要从结构去识别。数据模型决定数据结构,进一步影响到数据分布。

以分布式块存储为例,volume分成固定大小的chunk,chunk采用了复制或EC机制。
这是最基础的数据模型,pool、dir、snapshot是次一级的数据模型。

pool是对磁盘的一次物理划分,故障域是对pool的又一次物理划分。按chunk组织副本,需要较多的元数据。

Ceph引入了PG概念,pool分PG,PG内的所有对象,具有相同的副本位置。一个对象映射到一个PG,就决定了其副本位置。
这样,所需元数据骤减。PG与OSD具有多对多的映射关系。

使用场景,从用到体。金融机构的OLAP,需要在一定时间内完成。

两个卷、指定范围内的数据合并。

\subsection{0925}

以分布式存储系统为核心,向上下左右、四面八方拓展。

\section{09}

\subsection{0901}

\subsection{0902}

\subsection{0903}

删除pool没有完成,此时重启lichd,会残留pool相关状态,如相关的disk文件,影响后续过程。

\begin{enumbox}
\item 显示disk依然处在使用状态,无法加入新的pool里。
\item pool已经不存在,disk依然能够加载成功?
\item 暂时控制恢复进程的命令行工具
\item \hl{用RAFT实现副本一致性}?
\end{enumbox}

添加etcd task,用于处理pool remove过程中的异常中断。\hl{其它bh bask也可以切换到etcd方式}。
因为pool name可能被reuse,会引起难以预料的问题。

\hl{把数据结构、算法和架构的学习和思考放入内功修炼},持续地不中断地打牢基础。

反思recovery的线程与队列组织,有诸多不便之处,无法方便地\hl{hash不同的数据到不同的线程}。
\hl{线程最好有自己的队列},封装成SEDA模型,按策略分配不同的任务到不同的线程。

在拔盘的时候,EIO要封装在底层(EGAIN),不能返回给前端,目前已能正确处理。

问题集
\begin{enumbox}
\item 把chunk作为object layer,如何?其上构建统一存储。
\item ceph架构: CRUSH pool、pg、object
\item 节点内:RAID、LVM、Bcache、VDO等
\item 分布式:RAFT
\item iSCSI, tgt、spdk项目
\item VFM
\item 多存储网
\item Sqlite/LevelDB/RocksDB
\item 从阵列到分布式架构
\item QoS
\end{enumbox}

多个\hl{索引}如果转化成LevelDB的KV模式?

上下左右看,ABC

往下看,看到操作系统乃至硬件,往上看,看到虚拟机、docker/k8s、\hl{企业存储、云存储、数据库、大数据}、乃至AI。
右手螺旋法则,立体河图

dd、mount、fio、vdbench、hazard

\subsection{0904}

分几个方面建立知识体系:
\begin{enumbox}
\item 项目:实战(立足手头工作)
\item 架构:编程、架构(数据结构、算法、模式)
\item 系统:\hl{操作系统、文件系统、数据库、网络}、分布式系统、缓存、MQ
\item 四面八方延伸至ABC
\item 要包含数学这个异常重要的基础
\end{enumbox}

数据模型即是访问层,如统一存储的对象、块、文件,NoSQL的KV、文档、列存、图、关系等。
对块存储系统(SAN)而言,可分两层去思考,一是对象层,二是块层。

大规模分布式存储系统一书的架构组织不错,从理论到实践、从下到上循序展开。

基础知识分单机存储引擎和分布式架构,质量属性是自始至终都要考虑的要素,包括性能、可靠性、可用性与扩展性等(视角)。
\hl{视图(来自视点,以视点为模板)+ 视角的架构描述方法}值得参考。

\hl{事务}的观点、并发控制与恢复的观点、虚拟化和持久化的观点贯彻始终。

单机存储系统需要考虑的要素有哪些?分布式系统增加了哪些复杂度?

从上至下、从下至上双通路考察本地存储系统。

不要受到当前实现方式的限制,从更基础的层面提出问题,并思索方案的可能性空间。
先发散、后收敛。

存在两类数学,一种是\hl{易经上、或神圣几何里的大衍之数},一种就是数学学科,包括纯粹数学和应用数学。
大衍之数体现了数学的艺术性、哲学性。

QoS是一个大问题
\begin{enumbox}
\item 控制目标:下限、上限、突发
\item 集中式控制、分布式控制
\item 优先级
\item \hl{高速pool的时候},QoS机制本身带来的开销太大,导致IOPS严重地低于设定值。
\item dbg模块会生成/dev/shm/lich4/msgctl目录下的若干文件
\item bcache在docker内部不显示虚拟设备
\end{enumbox}

QoS的参考资料:
\begin{enumbox}
\item Ceph的dmClock算法
\item XSKY的漏桶与令牌桶结合的策略
\item SolidFire
\item NetApp
\item SPDK bdev
\end{enumbox}

\subsection{0905}

pci设备?通过pci设备号操作该设备,不经过kernel。

本地存储引擎,带索引的关系模式怎么转换为KV模式?
磁盘空间管理、chkid到磁盘位置的映射关系。

在allocate或discard时,会建立这些关系。读写时,只是访问这些关系。
allocate过程或独立进行、或在使用的时候按需分配(精简配置)。

chunk的metadata,记录了副本的位置信息。所以\hl{最小的故障域是节点}。

副本或条带一致性是最困难的部分。

数据分布(分配、再平衡、恢复、GC)的一般规则:
\begin{enumbox}
\item 按pool组织
\item pool内满足故障域规则
\item chunk满足副本数要求,且副本之间是一致的
\item 容量和负载在节点之间尽量均衡分布
\item 考虑节点的权重
\end{enumbox}

异常情况的处理,比如故障域小于副本数要求。

pool资源组织成一树状结构,pool-rack-node\hl{-disk}。

负责空间分配的admin在内存中维护着这样的结构,在节点启动之后,
会周期性地向admin上报自己本地的pool拓扑信息。

pool的ns,通过pool controller和volume controller跟踪维护。

给定chkid,如何\hl{定位controller所在位置呢}?
后续所有controller相关操作,都要调度到该controller去处理。

已知问题:
\begin{enumbox}
\item controller在节点以及节点内core上的分布是否均匀?
\item 大卷无法有效地进行负载均衡,瓶颈受限于一个core
\end{enumbox}

数据的复制、迁移。

一个系统可分解为基础结构kernel和高级特性,宏内核架构,尽量组织成模块化。
宏内核方便模块之间的通信,但因隔离性差,也容易引入严重问题,影响到全局。

\subsection{0906}

把disk故障也纳入vfm,即不再使用check方法,而是按subvol进行标记,
被标记的subvol就处在可疑状态,待全部确认后,再清除其上的标记。

check标记的是副本,\hl{subvol标记的是一个subvol范围内的所有chunk在某节点上的所有副本}。

subvol标记怎么做到io和恢复的隔离,从而提高性能的?

\hl{扫描顺序的影响}:如果按顺序进行扫描,待扫描完成后,即可清理其上的标记。
如果是在一个大的范围内随机扫描,则只有全部完成,才能清理每个subvol上的恢复标记。

\subsection{0908}

两种故障要尽量统一处理。

加班
\begin{enumbox}
\item disk recovery,增加控制逻辑,动态调节参数,包括开关、恢复速度(统一为disk\_fill\_rate)
\item QoS,精确控制在高IOPS下难以实现,退而求其次,控制node recovery的fill\_rate
\item 在scheduler层面,iops与recovery采用不同的队列
\item fill rate == 0时,\hl{不退出恢复过程,循环等待},scan过程及其结果要保留,此约定适用于两种故障
\end{enumbox}

恢复控制参数
\begin{enumbox}
\item 控制开关 (静态、动态)
\item 带宽
\item 线程数
\item 每请求提交到控制器的chunk数 (最小最大)
\end{enumbox}

两个pool的隔离性,\hl{拔出一个pool的盘为什么会严重干扰另一个pool的iops}?
\begin{enumbox}
\item RAID (WT)
\item DB
\item schedule
\end{enumbox}

\subsection{0910}

TODO list:
\begin{enumbox}
\item 被deleted的卷/快照导致recovery rescan
\item 删除pool过程如果中断,残留下的垃圾影响到后续过程
\item 没有IO的时候,限制recovery不合理(Q4)
\end{enumbox}

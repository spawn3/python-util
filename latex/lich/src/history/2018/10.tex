\section{10}

\subsection{08}

进入Q4。既要进一步熟悉当前代码,又要提升认知水平,努力把握存储前沿阵地。

一、整理资料:书籍、论文、产品、人才等。市场上已有哪些产品?有哪些在研制中?做全闪,哪些方面是前沿?哪些方面能做出优势来?
标准、协议、算法等。

二、VFM机制违反了强一致性要求,有造成数据丢失的风险。同时,当前实现不够精准。

三、选主、lease、副本一致性、分布式事务管理是分布式系统的重点和难点。

四、卷的QoS已可用,恢复的QoS有难度,因为要协调多种活动,另外,应该在哪个粒度上考虑此事,卷、节点、pool?
设定什么样的策略?

五、与全闪密切相关的特性有哪些?NUMA、内存管理、RDMA、SPDK、NVMf等。

六、快照、删重、压缩、LSV等

七、\hl{用户存储方面的痛点是什么}?

八、引入etcd后,一些事情处理起来就方便多了。但目前\hl{etcd严重依赖于每个节点的物理时间},是我们的用法有问题吗?

\subsection{09}

用菱形结构组织专业知识体系,则为原理-基础,问题-系统,架构是宏观视野,统揽全局。数据结构和算法是基本构造单元。
从问题出发,构建系统,在构建系统的过程中,提出新的问题。

架构,视图+视角。还有更丰富的内容,比如面向对象、函数式、数据流等。侧重于从宏观上把握软件系统的基本属性。

基础包括数据结构、算法、编程语言、工具、协议等等内容。

算法范围很广,包括教科书上的常规算法,也包括新兴的机器学习、AI、并行和分布式算法、事务处理的算法。致广大而尽精微。

从具体产品上来说,横向看,从问题到系统,从方针到程序,需要以天地为中介(宏观/微观,大小,多少,全局/局部),
问题-系统构成横向的一个环,原理、实践构成纵向的一个环。

以上分析是侧重从技术的角度说。

\hl{战略罗盘}也采用了上述的四象分析法。

尽快完成数据平衡的任务:
\begin{enumbox}
\item 按pool进行组织(每个pool包括一个主线程、若干工作线程、每个工作线程建模为actor模型,包含私有任务队列)
\item 可以手工start/stop
\item 可以动态控制带宽
\end{enumbox}

工作机制:
\begin{enumbox}
\item 主线程扫描本pool本地非删除卷,如需要平衡,则启动若干工作线程,然后进入循环,选取平衡任务放入工作线程的任务队列。
\item 工作线程的任务很简单,依次处理本地队列的任务。为了控制带宽,引入token bucket机制。
\end{enumbox}

每个pool维护有node列表(按利用率从小到大排序),卷列表。对任一chunk,先得到各副本对应的节点列表(利用率从大到小排序)。
对高利用率节点上的副本,选取合适的位置。既要达到平衡之目的,也须满足故障域规则的要求。

现有实现中卷结构上记录有可用节点列表,在引入pool的情况下,与pool的节点列表是一致的,所以是不必要的。

pool有两重含义:物理(与节点的关系)和逻辑(与卷的关系)。区分这两种关系至关重要。

\subsection{10}

调试时,注意跟踪指针地址是否正确。现象在此,引起现象的原因可能是一个莫名其妙的地方。

用OO理念进行程序设计。

先骨架,后填空。从顶往下,分而治之,逐步求精。

识别设计模式,重用设计

\subsection{11}

无vfm,无故障,执行lich health scan做什么事情?

跟踪vfm的标记和清除标记过程。为什么vfm没有清理成功?与load有关?

单独提出config,管理/dev/shm/lich4、/opt/fusionstack/data等目录下的文件。
接口比较简单,复杂一点的是fnotify相关功能,支持注册。
\hl{写入配置文件时,采用了rename方式,会触发多个fnotify事件}。

按vol跟踪iscsi,包括target、connection等。\hl{iSCSI和iSER采用了不同的代码,也共用了一部分代码。有些逻辑需要两处都处理}。
对一个vol而言,iSCSI一个session可以有多个connection,否则不能支持multipath?

\hl{协议}:每个iscsid服务可以有多个target,按target名字指定。每个target下包括若干lun。从访问来看,conn/session等概念。
协议交互过程,包括了登录、重定向等过程。

login后,session才正式建立,\hl{存在sess为NULL的conn}。

clock没有写入journal是设计选择。clock丢失是低频事件。

\subsection{15}

TODOList
\begin{enumbox}
\item data balancer
\item vfm cleanup
\end{enumbox}

顺序IO,QoS不起作用,因为\hl{上层进行了IO聚合}?

\subsection{17}

关于数据分布的思考:用一个更高的视野去看恢复、平衡、gc等过程。

引入存储池后,诸过程都按pool进行,在pool内满足故障域、平衡等约束条件。
需要开发相关校验工具,以简化该过程。

控制参数的设定不够灵活,如何更自动、更智能地去满足业务需求?

数据平衡提交测试! 数据恢复及其QoS遗留问题尚多,需要继续!

TODOLIST
\begin{enumbox}
\item QoS控制不够智能,无分级控制
\item 与vfm的配合,无法区分扫描和恢复两个速度
\item 清除vfm标记,似仍有问题?
\item vfm本身引入的问题,如违反强一致性
\item clock丢失导致性能下降(引发恢复)
\item 删除卷导致重新开始
\end{enumbox}

\subsection{19}

\hl{两点取speed法}:保持p1不动,p2代表当前点,随时间而运动。到达时间差时,可以得出速度。
此时,p1跳变,与p2重合。周而复始。

rethink VFM机制本身!

恢复QoS放在客户端还是服务端? 目前的token机制是放在发送端的,控制器端的QoS实现的有问题。

进入\hl{修复模式}后,才执行相关策略。如果只有io,或只有recovery,则不需限制。

应在\hl{控制器端/目标端}控制恢复带宽!

如果策略放在pool上,会横跨多个节点,是一个分布式算法。
如果策略放在pool的各个节点上,是一个局部策略。
如果策略放在单个的vctl上。策略的粒度通常有以下几种:
\begin{enumbox}
\item pool distributed
\item pool node
\item pool node vctl
\end{enumbox}

支持多粒度,如何?不是非此即彼,而是综合治理,分层设定。

场景:一个节点多个磁盘故障,恢复过程中,发生二次故障。

\hl{分离check速度和恢复速度},限制限制的是恢复速度,而不能按check速度进行。

\subsection{21}

如果需要总控,可先写入etcd,每个节点轮询该任务。

Balancer TODOList
\begin{enumbox}
\item 每个pool的状态,在有恢复的情况下
\item 已放入actor队列的任务,如返回ESTALE,ECANCELD等返回值
\item vol\_pop,如返回ESTALE,ECANCELD等返回值
\item \_\_chunk\_read\_getnode返回ESTALE,该分支没有调用check(调用stat时也遇到了该问题)?
\item chunk\_push导致hugepage OOM
\item 策略问题
\item 没有处在允许中的平衡过程,无法发现vctl。
\item \hl{卷控制器分布不均匀}
\end{enumbox}

Recovery TODOList
\begin{enumbox}
\item 区分check、success和fail等计数器
\item 恢复QoS
\item 动态暂停、停止恢复?
\end{enumbox}

资源管理

性能优化
\begin{enumbox}
\item Redis代替Sqlite
\item U-TCP
\item ucache?
\item 内存管理、按需分配
\end{enumbox}

\subsection{24}

一、以中医的世界观为主体,参考武术、书画等艺术形式,理解系统。彭子益的圆运动的古中医学,值得参考。
\hl{中医之根在于河图},此观点值得细细思考、论证。

古之三皇,得道之统,立于中央,神与化游,以抚四方。

恢复、平衡、QoS做一小结。几种活动都与一致性、性能有关,性能抖动。
\hl{整个fusionstor就像一个有机生命体}。

按圆运动的古中医学的结构,中气是存储引擎,包括本地和分布式两部分。
数据分布、副本、EC,精简配置、GC、存储分级、缓存、快照、卷访问控制等功能,内在于存储引擎。

客户端为火,资源管理为水,IO读写、恢复、平衡、卷copy,远程复制等等活动乃经气升降沉浮,体现迁移之美,
QoS体现控制先行,经气流行通不通也。

中央与四方的相互影响、感通。

每个polling core又是一个小的生命体,小的圆运动过程。

二、恢复、平衡都是基于pool进行控制。每个节点上,每个pool有零个或多个卷控制器,恢复、平衡过程处理的就是本节点内pool内的非删除卷/快照。
对每一个卷分布进行处理。恢复、平衡与控制器的交互比较简单。

\hl{线程结构受SEDA架构的启发},分模块主线程,每个pool分主线程、工作线程,工作线程可以进一步细分(如扫描、恢复阶段采用不同的线程池)。
线程之间通过mq进行通信。需要注意的是,要能灵活地进行任务调度,以满足一定的顺序性要求。由此可知,采用工作线程私有队列会更灵活。
若采用一个总的任务队列,则不方便调度。\hl{线程结构决定了数据流的节点和方向}。其中的每个线程可以应用一定的控制策略,比如发送速度、处理速度等。

不同职能与线程之间的对应关系。进度控制/终止条件,更新状态,流量控制,输入输出,处理逻辑、异常处理等。

三、QoS: 明确问题,难度等级越来越高。

卷级的QoS采用令牌桶和漏桶结合的方式,可以控制带宽及其波动。这是本地化的算法。

不同活动之间的QoS,比如IO和恢复,涉及权重、比例、负载等问题。
如按比例,一种活动会拉低另一种活动,负载的饱和点难以确定。
一种思路:若两者之和大于某个阈值,则进行控制。如小于该阈值,则不进行限制。
问题是如何设定阈值?可否通过学习法,自动学习得到?

更高层次的QoS控制策略难以实现,如pool内,需要分布式QoS算法。

四、反思:

架构,以前采用了X型描述法,分为物理资源和虚拟资源。
后来采用视图视点2D描述法,下一步考虑圆运动描述法。

\begin{enumbox}
\item pool
\item dir层次不必要存在,pool下直接放lun。该结构可以在etcd上维护。(映射关系, name到chkinfo的映射)
\item volume (线性父子关系,bitmap, array)
\item snapshot
\item *
\item etcd
\item meta数据管理过于复杂,树状结构。应采用flat结构(不能采用etcd)。
\item 本地存储引擎采用sqlite不当 (bitmap,sqlite相关信息放入redis?)
\item clock也采用redis,如何?
\item *
\item bcache
\item EC、RM等都处于荒废状态
\end{enumbox}

资源管理上最大的变化是引入了pool概念,pool需要与disk建立映射关系。

资源管理,增删改查,删除是比创建更为复杂的过程。资源层次、连接情况会影响删除过程。
以删卷为例,有访问时不能删,有快照时,删除要做更多额外的事情。
用\hl{前置条件、后置条件、不变式}去规范化该问题。

全闪
\begin{enumbox}
\item SPDK
\item RDMA,内存使用上的变化
\item iSCSI/iSER
\end{enumbox}

全面回忆lich的设计与实现

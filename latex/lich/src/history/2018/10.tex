\section{10}

\subsection{08}

进入Q4。既要进一步熟悉当前代码,又要提升认知水平,努力把握存储前沿阵地。

一、整理资料:书籍、论文、产品、人才等。市场上已有哪些产品?有哪些在研制中?做全闪,哪些方面是前沿?哪些方面能做出优势来?
标准、协议、算法等。

二、VFM机制违反了强一致性要求,有造成数据丢失的风险。同时,当前实现不够精准。

三、选主、lease、副本一致性、分布式事务管理是分布式系统的重点和难点。

四、卷的QoS已可用,恢复的QoS有难度,因为要协调多种活动,另外,应该在哪个粒度上考虑此事,卷、节点、pool?
设定什么样的策略?

五、与全闪密切相关的特性有哪些?NUMA、内存管理、RDMA、SPDK、NVMf等。

六、快照、删重、压缩、LSV等

七、\hl{用户存储方面的痛点是什么}?

八、引入etcd后,一些事情处理起来就方便多了。但目前\hl{etcd严重依赖于每个节点的物理时间},是我们的用法有问题吗?

\subsection{09}

用菱形结构组织专业知识体系,则为原理-基础,问题-系统,架构是宏观视野,统揽全局。数据结构和算法是基本构造单元。
从问题出发,构建系统,在构建系统的过程中,提出新的问题。

架构,视图+视角。还有更丰富的内容,比如面向对象、函数式、数据流等。侧重于从宏观上把握软件系统的基本属性。

基础包括数据结构、算法、编程语言、工具、协议等等内容。

算法范围很广,包括教科书上的常规算法,也包括新兴的机器学习、AI、并行和分布式算法、事务处理的算法。致广大而尽精微。

从具体产品上来说,横向看,从问题到系统,从方针到程序,需要以天地为中介(宏观/微观,大小,多少,全局/局部),
问题-系统构成横向的一个环,原理、实践构成纵向的一个环。

以上分析是侧重从技术的角度说。

\hl{战略罗盘}也采用了上述的四象分析法。

尽快完成数据平衡的任务:
\begin{enumbox}
\item 按pool进行组织(每个pool包括一个主线程、若干工作线程、每个工作线程建模为actor模型,包含私有任务队列)
\item 可以手工start/stop
\item 可以动态控制带宽
\end{enumbox}

工作机制:
\begin{enumbox}
\item 主线程扫描本pool本地非删除卷,如需要平衡,则启动若干工作线程,然后进入循环,选取平衡任务放入工作线程的任务队列。
\item 工作线程的任务很简单,依次处理本地队列的任务。为了控制带宽,引入token bucket机制。
\end{enumbox}

每个pool维护有node列表(按利用率从小到大排序),卷列表。对任一chunk,先得到各副本对应的节点列表(利用率从大到小排序)。
对高利用率节点上的副本,选取合适的位置。既要达到平衡之目的,也须满足故障域规则的要求。

现有实现中卷结构上记录有可用节点列表,在引入pool的情况下,与pool的节点列表是一致的,所以是不必要的。

pool有两重含义:物理(与节点的关系)和逻辑(与卷的关系)。区分这两种关系至关重要。

\subsection{10}

调试时,注意跟踪指针地址是否正确。现象在此,引起现象的原因可能是一个莫名其妙的地方。

用OO理念进行程序设计。

先骨架,后填空。从顶往下,分而治之,逐步求精。

识别设计模式,重用设计

\subsection{11}

无vfm,无故障,执行lich health scan做什么事情?

跟踪vfm的标记和清除标记过程。为什么vfm没有清理成功?与load有关?

单独提出config,管理/dev/shm/lich4、/opt/fusionstack/data等目录下的文件。
接口比较简单,复杂一点的是fnotify相关功能,支持注册。
\hl{写入配置文件时,采用了rename方式,会触发多个fnotify事件}。

按vol跟踪iscsi,包括target、connection等。\hl{iSCSI和iSER采用了不同的代码,也共用了一部分代码。有些逻辑需要两处都处理}。
对一个vol而言,iSCSI一个session可以有多个connection,否则不能支持multipath?

\hl{协议}:每个iscsid服务可以有多个target,按target名字指定。每个target下包括若干lun。从访问来看,conn/session等概念。
协议交互过程,包括了登录、重定向等过程。

login后,session才正式建立,\hl{存在sess为NULL的conn}。

clock没有写入journal是设计选择。clock丢失是低频事件。

\subsection{15}

TODOList
\begin{enumbox}
\item data balancer
\item vfm cleanup
\end{enumbox}

顺序IO,QoS不起作用,因为\hl{上层进行了IO聚合}?

\subsection{17}

关于数据分布的思考:用一个更高的视野去看恢复、平衡、gc等过程。

引入存储池后,诸过程都按pool进行,在pool内满足故障域、平衡等约束条件。
需要开发相关校验工具,以简化该过程。

控制参数的设定不够灵活,如何更自动、更智能地去满足业务需求?

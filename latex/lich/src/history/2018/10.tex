\section{10}

\subsection{08}

进入Q4。既要进一步熟悉当前代码,又要提升认知水平,努力把握存储前沿阵地。

一、整理资料:书籍、论文、产品、人才等。市场上已有哪些产品?有哪些在研制中?做全闪,哪些方面是前沿?哪些方面能做出优势来?
标准、协议、算法等。

二、VFM机制违反了强一致性要求,有造成数据丢失的风险。同时,当前实现不够精准。

三、选主、lease、副本一致性、分布式事务管理是分布式系统的重点和难点。

四、卷的QoS已可用,恢复的QoS有难度,因为要协调多种活动,另外,应该在哪个粒度上考虑此事,卷、节点、pool?
设定什么样的策略?

五、与全闪密切相关的特性有哪些?NUMA、内存管理、RDMA、SPDK、NVMf等。

六、快照、删重、压缩、LSV等

七、\hl{用户存储方面的痛点是什么}?

八、引入etcd后,一些事情处理起来就方便多了。但目前\hl{etcd严重依赖于每个节点的物理时间},是我们的用法有问题吗?

\section{12}

\subsection{03}

交换机SM HA不可用,更新OS到相同版本后可用。

熟悉RDMA

两副本,一副本处在chunk状态,且vfm非空,包括另一副本

\subsection{10}

熟悉iscsi/iser架构和代码,进一步需要熟悉spdk、nvmf等,这是块存储的趋势所在。

\begin{enumbox}
\item NVMe设备,通过设备文件和pci直接访问的不同
\item linux kernel的block layer
\item udev
\item mount是指定dirsync选项
\item dd在测试快照时,指定direct io
\item async和non blocking的异同
\item buffer和cache的异同
\item session和cookie
\item 内核提供了什么功能?现在又为什么在网络和存储上提倡kernel bypass。
\item vip and multipath
\item ceph
\item fs and object
\item db
\end{enumbox}

除了协议和生态部分,内核部分的主要模块也要持续思考,包括编程模型、元数据管理、HA等。
要具有提要钩玄的能力,勾勒出主要脉络,分清主次,带动全局。中医对生命科学的思考方式,很有借鉴意义。
人的生命是一个动态发展的整体。如何维持此整体的健康?

\subsection{11}

要深入体会网络协议栈的架构设计实现,充分地体现了分层架构的巨大威力。

要想富,先修路。网络编程的第一步就是连接管理。连接的建立、维护到断开。
一个或多个连接构成session。一个session见得唯一的target。target与lun什么关系?
为什么需要多个连接构成session?

关于学习:重视第一手资料,道听途说不能解决问题,善于提问,不要有畏难情绪。
比如RDMA,从原始文献好好体会,在iSER里进一步体会,温故知新,没有过不去的坎。

再回头看RDMA代码,当明白了一些原则后,就显得一目了然了。
iscsi和iser代码也是如此。为什么呢?需要理解消息交换的基本原则。
在一个循环内,既有post,又有poll。

scst与lio解决的是同一类问题,scst的代码分为三部分:core、target driver and storage driver。
lich也有如此模块。core起到存储虚拟化的作用。

学习一定要破除神秘感,循着正确方法,很容易攻克。即便如AI、数学,也不是不能捡起来。应该捡起来。

\subsection{12}

对个人来说,做什么才能产生复利效应?专业、认知。

一个lich系统,一个生命系统,都是系统。系统论具有普适性。

网络和scheduler的事件循环,也是一个圆运动。

\begin{enumbox}
\item 列出所有controller及其分布
\item 控制器为什么加载不成功?
\end{enumbox}

专注全闪,不做它想。不知自我克制,终将一事无成。

在docker里用文件模拟设备文件,因为文件io接口是一样的。

测试方法和工具:\hl{道法术器}几个层面,工具化、自动化至关重要,
但必须在正确策略的指引之下。

查找控制器为什么用mcast机制,而不是去admin上查(lease机制),
因为客户端有cache,也不会对admin带来额外负担。

vip才会有arp缓存失效的问题,需要主动更新client端arp缓存。

\subsection{13}

linux
\begin{enumbox}
\item 操作系统
\end{enumbox}

ceph
\begin{enumbox}
\item 两篇论文
\item 官方文档
\item 参考书
\item 源码
\end{enumbox}

\subsection{14}

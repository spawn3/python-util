\section{02}

\subsection{02}

\hl{Reactive Programming与CAP关系密切}。追求的value是即时响应、可维护性和可扩展性。在正常情况下是如此,即便发生故障,或负载变化,也是如此。
如何做到?引入了async、non-blocking消息传递,引入了显式的消息队列。通过MQ,如何实现以上诸特性呢?

系统划分成组件树的成绩结构,明确职责、隔离故障,并把故障委托给独立的故障处理单元。

\hrulefill

Lich的故障处理,应该有一套框架去处理所有故障,包括磁盘、节点,甚至把balance也纳入其中。
通过发送消息来触发相应的处理逻辑。

维护显式的消息队列,放在哪儿?client端,卷控端?如放在卷控端,由谁来调度执行?
卷控实则是个非执行体,只是一个内存结构。

可以由卷控所在的core线程去调度、或采用timer的方式。

MQ的task之间,如果由依赖关系,如io读写请求,需要同步机制。

并发提交的每个请求对应一coroutine,在scheduler层面,形成了一MQ。

\hrulefill

Lich的task,与actor有很多不同,如没有mailbox,不能形成监督树。

\subsection{04}

明体达用,理论是体,在各种系统中的实现是用。整理分布式存储系统的方方面面,不能脱离具体的工程实践去抽象地理解理论。
Lich也是诸多系统之一。

按自然的演进过程,从ACID到CAP。单机事务是思考的起点,CAP以及其改进是设计原则。
加入复制后,情况变得更为复杂。

ACID中,I是隔离性,需要CC机制来保障。A/D是原子性和持久性,需要UNDO、REDO日志来保障。C是业务数据完整性约束。
与CAP中的C有所不同。CAP中的C是原子读写,如线性一致性、顺序一致性、因果一致性、会话一致性。

CAP也是数据库的设计的指导原则。\hl{CAP的最新形式是PACELC,与Reactive Programming很类似}。
分布式存储系统是现代大规模高并发网站架构的一个主要部分,按分形学,有很多共性。

线性一致性对读写操作都要执行全序原子多播。顺序一致性需要对写操作执行全序的原子多播。

一致性可以看作物演通论的生存度,一致性递弱需要相应的补偿协议,从强一致性到最终一致性是一个连续的光谱。
一致性需要高于某一阈值,系统才是健康的。一致性降低必有相应的代偿协议。

为什么要降低一致性?或为什么一致性会降低?

\hl{single master与primary copy}不同,前者针对的是真个数据集,后者针对的是某一数据项。
还有一种peer to peer系统,就是一个数据项的更新也会涉及多个primary。

两者之区分实质上也就是\hl{paxos协议中无主和有主的区分}。我们把RAFT看作paxos协议的一种实现。
通过强化约束,来简化实现的复杂度。

\hrulefill

DB引擎,意索引。先考虑CC(隔离性,事务调度器),次考虑提交协议,最后考虑基于日志的故障恢复(A/D)。
扩展到分布式数据库系统,需进一步考虑\hl{CC和提交协议如何去满足ACID属性}。

整理数据库相关知识,包括单机、分布式数据库,从数据模型上,包括SQL/NoSQL/NewSQL。
主要从几个维度进行:
\begin{enumbox}
\item 数据分布
\item 复制
\item index
\item transaction
\item 读/写
\end{enumbox}

\hrulefill

\hl{CAS/ABA,时间戳,向量时钟}。

MVCC里对每行记录增加两个隐藏version。

除了IO之外,还有并发的其他任务,如replica cleanup。

有多种CC方法,主流的是lock,\hl{TO、MVCC都是基于版本号/timestamp}。

\hrulefill

把以上理论进一步落实到Lich和Ceph的设计中。
\begin{enumbox}
\item 更新chunk时递增info version?
\item 在snapshot也实用了version。
\end{enumbox}

\subsection{05}

同心圆模型,最核心的是设计心法,从内往外以此是硬件,软件(OS、FS、DB、PL/Compiler),网络,分布式系统。
设计和算法是两大重点和难点,可放入最内圈。

可靠性、可用性,容错、resilient,可伸缩性、elastic,可扩展性等概念有着细微的差别。

性能\&故障诊断分析方法:
\begin{enumbox}
\item Pipeline
\item Queuing
\item ***
\item PAT
\end{enumbox}

数据存在的几种形态:
\begin{enumbox}
\item Storage
\item Cache
\item MQ (stream)
\end{enumbox}

Storage和Cache具有极大的相似性。

至于async更多是一种架构思想。Reactive Programming强调了MQ的重要性,\hl{message-driven是实现RRE的良方}。

动静分离的静,最适合应用多级cache,御敌于国门之外,最终到达源站。
无状态的业务逻辑层与有状态的数据层,如何进行伸缩是有很大不同。

\subsection{07}

数据层面:复制(EC、dedup、compress)、snapshot、cache等等。

卷的snapshot,cc中快照隔离,日志的检查点技术

\hrulefill

adesk架构演进之路:
\begin{enumbox}
\item 单库无拆分->动静分离->master/slave的replication架构。
\item 加入ElasticSearch,支持全文搜索功能。
\item 采集行为日志,进行数据分析。google analysis,umeng等。
\item 对接广告平台进行变现
\item \hl{上云}:不再自己运维。采用了阿里云和七牛的云平台。
\end{enumbox}

性能:索引,cache(CDN etc)、负载均衡

可伸缩性:动静分离、增加磁盘。

推送

监控:第三方监控、nagios、zabbix

增长黑客

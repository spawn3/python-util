\section{02}

\subsection{02}

\hl{Reactive Programming与CAP关系密切}。追求的value是即时响应、可维护性和可扩展性。在正常情况下是如此,即便发生故障,或负载变化,也是如此。
如何做到?引入了async、non-blocking消息传递,引入了显式的消息队列。通过MQ,如何实现以上诸特性呢?

系统划分成组件树的成绩结构,明确职责、隔离故障,并把故障委托给独立的故障处理单元。

\hrulefill

Lich的故障处理,应该有一套框架去处理所有故障,包括磁盘、节点,甚至把balance也纳入其中。
通过发送消息来触发相应的处理逻辑。

维护显式的消息队列,放在哪儿?client端,卷控端?如放在卷控端,由谁来调度执行?
卷控实则是个非执行体,只是一个内存结构。

可以由卷控所在的core线程去调度、或采用timer的方式。

MQ的task之间,如果由依赖关系,如io读写请求,需要同步机制。

并发提交的每个请求对应一coroutine,在scheduler层面,形成了一MQ。

\hrulefill

Lich的task,与actor有很多不同,如没有mailbox,不能形成监督树。

\section{03}

\hrulefill

范畴化和做类比是认知的核心。RMI原则揭示了这一点。

一致性问题是重中之重。

lich卷提供了类似disk的服务。需要应用层做出补偿。
分别分析单disk、单节点,集群整体掉电的情况。

与raft做类比:leader选举,如何找到leader?日志恢复、正常服务几个阶段。

如何定义提交规则?如何apply到apply?lich保证了提交的持久性,
对uncommitted条目,如何分析其安全性?

nid,magic作为epoch,避免ABA问题。clock/dirty的管理规则

CAS/ABA是个需要重点理解的知识点,lock free ring就是基于类似的原子操作实现的。

多处理器编程的艺术

C++并发编程实战

FS/数据库的一致性

\hrulefill

写和读具有共轭关系。卷的读写就是所谓I/O。

\hl{一个IO可能涉及多个chunk,如何防撕裂?范围锁?}

COW快照的snaptree上,每个阶段记录了该节点到下一节点的变更原像,这是严格按照快照定义得到的。
任何一快照,都保留了该时间点的卷的状态。

考量一个极端情况,一直打快照,而无变更,所有快照的数据均为空。revert和delete都没有太多工作要做。

读卷和读快照有着本质上的不同。读卷不受影响,因为卷上一直有着最新的全部数据。
读快照则不然,需要两步:读该快照及其下游节点,读最低公共祖先到卷所在路径上的节点。

何时需要读快照?revert和clone时。区分读快照与读单个快照。

由此可推到读快照的过程。COW快照能有全量索引吗?有没有更好的做法?

引入auto snap主要是为了节约空间,如果存入snap from,则后续无法区分并清理。

\hrulefill

ROW的快照则保存的是打快照时的卷,相当于双缓冲区,新的写入发生在新的缓冲区。
读卷和读快照都变得复杂,revert和delete操作也需具体分析。

有特殊的情况可以简化删除过程。分解几种情况:如果快照是一个叶子节点,可直接delete,
如果是中间节点或根节点,最简单的方式就是标记删除。

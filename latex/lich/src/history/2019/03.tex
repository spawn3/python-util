\section{03}

\subsection{01}

问到的问题:
\begin{enumbox}
\item open channel SSD
\item NVDIMM/NVRAM
\item DHT的通用问题
\item 分布式cache
\item tgt连接数过多时,不稳定,断开
\item 中存超为的产品
\end{enumbox}

先着力解决sheepdog的痛点问题
\begin{enumbox}
\item 恢复,特别是EC情况
\item 平衡
\item QoS
\end{enumbox}

利用率大于50\%时,会出现空间不足的问题。

\subsection{02}

\subsection{03}

准备入职大道云行。

\subsection{04}

\subsection{05}

掌握sheepdog代码,关键问题:
\begin{enumbox}
\item 如何部署?sheepd与节点和磁盘的关系
\item 存储服务层导出了什么服务?tgt/iSCSI、NBD?
\item 如何确定vnode数量?
\item volume-object-node-disk,几级映射关系如何维护?
\item 如何保证更新和读取的一致性? epoch?
\item 如何recovery?dht不便于scan,从volume or object的角度入手?
\item QoS
\item snapshot and clone?
\end{enumbox}

参考项目
\begin{enumbox}
\item ceph
\item glusterfs
\item sheepdog
\item minio
\item swift
\item scale-io
\item VSAN
\end{enumbox}

\subsection{10}

双线:基础和前沿,缺一不可。要有强烈的问题意识,去解决大问题。

\subsection{11}

来大道云行报道!

主要导出了iscsi接口,tgt和gateway部署在一个节点上,通过127.0.0.1进行通信。有内存争用的现象。

每个vdi只有4T。快照采用cow。在恢复和快照时,性能下降严重。

zk内存消耗大、延迟高。

数据一致性:副本、条带一致性和write your read会话一致性。

\hl{epoch是集群版本号},恢复时用于查找元对象,无对象版本号,无checksum。

\subsection{13}

每个节点包括128个vnode,后续可以在线调整reweight。所有vnode用rbtree来组织。
在查找一个oid对应的node时,还需考虑zone故障域。

when更新epoch?update epoch会引发一系列后果。在update cluster info里也有处理stale object的过程。

recovery object做什么?优先在本地找,如在stale目录找到,还需比对hash才能确定stale object是否可用。

EC在发生故障时,基本上要调整所有分块,开销太大。

分析一个node反复下线又上线的情况。

io和rebuild的相互干扰。会优先rebuild发生io的object,然后wakeup该对象的io请求,继续进行。

一个vdi是不能被多个client端并发访问的。支持主备multipath,pool下的每个node都导出vdi,
只有一条path在工作。object cache无法与multipath协同工作。

自己实现tgt,与gateway在一个进程空间。

EC故障下数据和校验分块需要重组,带来很大的带宽压力,能不能不重组?

QoS的控制点放在哪儿?request or gateway?

没有抽象出RPC接口。

journal特性没开?

mount时,是否影响md5计算结果,因为存在cache的缘故。

写入各个EC object要避免张冠李戴、釜底抽薪。每个EC object的写IO要按相同的顺序进行。

IO和recovery有不同的controller?

每个controller是否看到相同的epoch以及对应的节点view。

在全掉电情况下,如何rollback,以保证原子写?

\subsection{19}

no journaling, nosync and no object version

理解sheepdog里的epoch,可以采用类比的方法
\begin{enumbox}
\item 快照和卷的关系对应stale和wd的关系
\item RAFT里term和index的作用
\end{enumbox}

模拟场景一,一节点下线又上线,又可分为两种情况,没有恢复,恢复未完成,恢复完成

io和recovery中epoch的用法不同。recovery会block io,恢复后wd有最新数据,stale即可回收。读写只读写wd,不读写stale。
恢复应block所有节点上该对象的io。

读写只是读写wd下的内容,修复才会涉及stale。epoch可以识别过期请求。

xfs的journal,就造成\hl{log over log}的多log问题,影响性能。

replication和EC有很大的不同。

部分成功、部分失败

部分写

\subsection{20}

zk感知故障,30s,时间过长。在节点间连接上做hb?hb和io如何组织网络,一个网络还是两个网络?

要区分\hl{临时故障和永久故障}。怎么区分临时故障和永久故障。
永久故障时恢复的工作量是恒定的,临时故障时则不然。尽量减少临时故障时所需恢复的工作量。

节点发生故障到触发恢复,再到开始修复,时间窗口较长。触发恢复之前,发生eio,由client端retry。
开始修复后,发生io的object走绿色通道(direct),优先修复。修复完成后,wakeup该object上的blocking io。

eio会不会导致部分成功、部分失败的场景?特别是EC,如何保证\hl{写入的幂等性?整个过程是可重入的}?

磁盘故障会触发节点级的recovery,不会导致epoch提升。

使用hazard测试ssan的一致性问题。

\chapter{Lich Master}

\section{生态}

全闪架构有两种:阵列、分布式。分布式高一级,更灵活,实现难度也更大。

\begin{enumbox}
\item 有哪些产品,各有什么亮点
\item 硬件的性能数据 (亚毫秒级)
\item 如何评估性能理论上限
\end{enumbox}

latency and iops

万兆网络的延时100微秒,采用用户态协议栈可以达到20微秒左右,IB RDMA大概5-8微秒。

NVMe的延时20微秒

\subsection{华为OceanStor}

华为OceanStor Dorado V3是面向企业关键业务打造的全闪存存储系统。

采用智能芯片、NVMe架构和FlashLink®智能算法,在开启重删、压缩、快照等增值特性后仍能保证0.5ms的稳定时延, 业务性能提升3 倍。

支持平滑扩展到16个控制器,满足未来不可预期的业务增长。一套系统同时支持SAN和NAS,企业级特性齐备,为数据库和文件共享等应用提供更高品质的服务。

支持免网关双活方案,可平滑升级到两地三中心方案和融合数据管理方案,实现方案级99.9999\%可靠性保障。

通过在线重删、在线压缩技术,提供可达5:1的数据缩减率,OPEX节省75\%。

满足数据库、虚拟桌面(VDI)、虚拟服务器(VSI) 和SAP HANA场景所需,助力金融、制造、运营商等行业向闪存时代平滑演进。

\section{取势}

AFA是趋势,SDS向AFA过渡需要在架构上做很多事情,也是ceph下一步演化的重点。

从1+3模型描述硬件资源的使用方式。
\begin{enumbox}
\item NUMA下的多核
\item 内存管理
\item 存储设备
\item 网络通信
\end{enumbox}

网卡、NVMe都是pci设备,是内存设备,即对应一定的内存地址,操作对应内存地址就是与该设备进行通信。
pci设备不同于诸如硬盘等sata设备。\hl{pci设备的精密度远比sata设备为高}。

iscsi和iser的代码是从tgt项目的不同版本取得的,NVMf则用的是spdk。
spdk实现了很多内存管理功能。

什么情况下需要数据对齐、内存对齐?

全用户态的SDS架构,kernel bypass。

\section{RDMA}

latency是最重要的性能指标。

RDMA是一种transfer,采用DPDK USN也是一种选择,比IB慢,但比kernel TCP协议栈要快几倍。

\begin{enumbox}
\item RDMA的编程模型?
\item RDMA的连接管理过程?
\item RDMA的内存使用方式有什么不同?
\item iSCSI/iSER conn是如何关联起来的?
\item 更好的抽象?
\end{enumbox}

RDMA是与TCP并列的一种网络传输方式,需要特定硬件支持,包括RDMA的NIC与交换机。
网络设备不同于存储设备,RDMA可以carry任意网络流量,包括iSCSI/iSER,自定义协议(corenet)等。

ibverbs API屏蔽了链路层的不同,可以用一套API同时支持IB,RoCE、iWARP等。

\subsection{ibverbs}

分为几个层次:node、core/dev、conn、event。

每个设备对应rdma\_info\_t结构,有ibv\_context属性,是唯一的key。
rdma\_cm\_id具有该属性。

\begin{itembox}
\item pd
\item cq (被所有连接共享)
\item mr  (shared)
\end{itembox}

深刻理解一个RDMA连接管理的过程,建立连接的每个阶段需要做哪些工作?

每个RDMA连接,qp关联到cq上。cq\_poll。iov\_mr

qp支持ibv\_post\_recv操作,peer的send操作会消耗这些buffer,
在与远端建立之后,就应注册相关buffer,建立必要的关系,特别是指定wr\_id属性,并post,然后等待接受请求。

在ibv\_cq\_poll之后,得到的ibv\_wc里,包含所需上下午信息(post之前建立的关系,初始化task时)。
可以直接从buffer里recv到的数据。

RDMA是异步通信机制。

\subsection{MM}

RDMA通信包括两类:msg与数据读写,所需内存都需要register。

向RNIC注册内存

\subsection{建立连接}

采用epoll机制

\subsection{通信}

采用ibv\_cq\_poll机制

\subsection{协议-iSER}

iSCSI over RDMA

iSCSI分多阶段,包括Login、Full Feature等。

\section{DPDK}

\section{SPDK}

\subsection{NVMe}

\subsection{协议-NVMf}

NVMe over Fabric

\section{MM}

\subsection{问题}

\begin{enumbox}
\item 简单接口
\item 便于调试
\item 并发性能
\item 内外碎片
\item 动态化
\item RDMA内存
\item replica\_srv\_init不能利用core内存,模块依赖性
\item Hash局部性不好,不利于CPU高速缓存
\end{enumbox}

\subsection{现状}

源文件:
\begin{enumbox}
\item ylib/lib/mem.c
\item ylib/lib/mem\_hugepage.c
\item ylib/lib/mem\_cache.c
\item ylib/lib/mem\_pool.c
\item ylib/lib/buddy.c
\item ylib/lib/buffer.c
\end{enumbox}

两级内存管理

分为core内外两种情况:core使用私有内存。

先生成hugepage,并置0。malloc后得到虚拟地址空间,把hugepage依次mmap到该虚拟地址空间。

hugepage与numa物理内存的关系是怎么样的?什么时候建立起来的?

底层hugepage不一定要用buddy,并且应可动态扩展。
然后是pool层,可动态扩展,用buddy管理每个hugepage。
其上是对象层slot,用于分配应用对象,用大小不等的多个队列管理。

RDMA需要注册内存,目前是把整个core内存一次性注册了。
每个core都调用注册函数?

\section{Performance}

性能优化,需要考虑CPU cache的行为,内存管理的效率等等因素。

\subsection{Hash}

\begin{enumbox}
\item Lock table
\item Replica srv
\end{enumbox}

\subsection{NUMA}

内存、网卡的本地化

多个RDMA、多个NVMe均匀分布之NUMA节点上(服务器通常有两个NUMA节点)

\subsection{Queue Pair}

aio、RAMD、iSER、NVMe等异步通信,大多采用多queue模型,有提交队列和完成队列。维持一定的上下文。
调度器、QoS也基于队列模型。

block和非block的区别,block是事件/中断驱动,非block是polling mode driven(PMD)。

异步意味着非block,把提交和完成解耦,可以并行执行,重要的是完成后可以配对到相应任务。
每一个请求建模为一任务,构成一状态机。如果实现为协程,就是隐式的状态机。

任务组织成一定的数据结构,所谓调度,就是从中选取下一个要运行的task。然后,切换上下文。

网络层、AIO都是如此。sqlite、redis等操作,派生出独立的工作线程,每个线程维护有私有的请求队列。

\subsection{socket}

区分连接socket和监听socket。监听socket构成网络连接的一方,而是起到辅助建立连接的作用。
在epoll等multiplexing机制里,可以同时处理。

aio通过事件抽象后,也纳入事件驱动的方式,其它如定时器等。

\subsection{对齐}

内存缓冲区对齐

offset对齐道sector边界,NVMe对齐到4K,否则性能会非常差

\subsection{知识体系}

ABC,从google三论文开始,hadoop大数据分析平台,spark, openstack+ceph的演进。

分为几类:\hl{协作、存储、计算、缓存}等。存储是最底层,可以把诸多领域贯通起来。

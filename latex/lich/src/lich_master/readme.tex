\chapter{Lich Master}

\section{MM}

\subsection{问题}

\begin{enumbox}
\item 简单接口
\item 便于调试
\item 并发性能
\item 内外碎片
\item 动态化
\item RDMA内存
\item replica\_srv\_init不能利用core内存,模块依赖性
\item Hash局部性不好,不利于CPU高速缓存
\end{enumbox}

\subsection{现状}

源文件:
\begin{enumbox}
\item ylib/lib/mem.c
\item ylib/lib/mem\_hugepage.c
\item ylib/lib/mem\_cache.c
\item ylib/lib/mem\_pool.c
\item ylib/lib/buddy.c
\item ylib/lib/buffer.c
\end{enumbox}

两级内存管理

分为core内外两种情况:core使用私有内存。

先生成hugepage,并置0。malloc后得到虚拟地址空间,把hugepage依次mmap到该虚拟地址空间。

hugepage与numa物理内存的关系是怎么样的?什么时候建立起来的?

底层hugepage不一定要用buddy,并且应可动态扩展。
然后是pool层,可动态扩展,用buddy管理每个hugepage。
其上是对象层slot,用于分配应用对象,用大小不等的多个队列管理。

RDMA需要注册内存,目前是把整个core内存一次性注册了。
每个core都调用注册函数?

\section{RDMA}

\begin{enumbox}
\item RDMA的编程模型?
\item RDMA的连接管理过程?
\item RDMA的内存使用方式有什么不同?
\item iSCSI/iSER conn是如何关联起来的?
\item 更好的抽象?
\end{enumbox}

RDMA是与TCP并列的一种网络传输方式,需要特定硬件支持,包括RDMA的NIC与交换机。
网络设备不同于存储设备,RDMA可以carry任意网络流量,包括iSCSI/iSER,自定义协议(corenet)等。

ibverbs API屏蔽了链路层的不同,可以用一套API同时支持IB,RoCE、iWARP等。

\subsection{ibverbs}

分为几个层次:node、core/dev、conn、event。

每个设备对应rdma\_info\_t结构,有ibv\_context属性,是唯一的key。
rdma\_cm\_id具有该属性。

\begin{itembox}
\item pd
\item cq (被所有连接共享)
\item mr  (shared)
\end{itembox}

深刻理解一个RDMA连接管理的过程,建立连接的每个阶段需要做哪些工作?

每个RDMA连接,qp关联到cq上。cq\_poll。iov\_mr

qp支持ibv\_post\_recv操作,peer的send操作会消耗这些buffer,
在与远端建立之后,就应注册相关buffer,建立必要的关系,特别是指定wr\_id属性,并post,然后等待接受请求。

在ibv\_cq\_poll之后,得到的ibv\_wc里,包含所需上下午信息(post之前建立的关系,初始化task时)。
可以直接从buffer里recv到的数据。

RDMA是异步通信机制。

\subsection{MM}

RDMA通信包括两类:msg与数据读写,所需内存都需要register。

向RNIC注册内存

\subsection{建立连接}

采用epoll机制

\subsection{通信}

采用ibv\_cq\_poll机制

\subsection{协议-iSER}

iSCSI over RDMA

iSCSI分多阶段,包括Login、Full Feature等。

\subsection{协议-NVMf}

NVMe over Fabric

\section{NVMe}

\subsection{SPDK}

\section{Performance}

\subsection{Hash}

\begin{enumbox}
\item Lock table
\item Replica srv
\end{enumbox}

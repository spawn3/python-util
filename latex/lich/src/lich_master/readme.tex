\chapter{Lich Master}

\section{MM}

两级内存管理

\section{RDMA}

RDMA是与TCP并列的一种网络传输方式,需要特定硬件支持,包括RDMA的NIC与交换机。
网络设备不同于存储设备。

ibverbs API屏蔽了链路层的不同,可以用一套API同时支持IB,RoCE、iWARP等。

\subsection{ibverbs}

分为几个层次:node、core/dev、conn、event。

每个设备对应rdma\_info\_t结构,有ibv\_context属性,是唯一的key。rdma\_cm\_id具有该属性。

\begin{itembox}
\item pd
\item cq (被所有连接共享)
\item mr  (shared)
\end{itembox}

每个RDMA连接,qp关联到cq上。cq\_poll。iov\_mr

\subsection{MM}

RDMA通信包括两类:msg与数据读写,所需内存都需要register。

向RNIC注册内存

\subsection{建立连接}

采用epoll机制

\subsection{通信}

采用ibv\_cq\_poll机制

\section{iSCSI}

\subsection{iSER}

iSCSI over RDMA

\section{NVMe}

\subsection{SPDK}

\subsection{NVMf}

NVMe over Fabric

\section{Performance}

\subsection{Hash}

\begin{enumbox}
\item Lock table
\item Replica srv
\end{enumbox}

\chapter{逻辑模型}

\begin{tikzpicture}[show background grid]
    \begin{class}{Disk}{6, 0}
    \end{class}
    \begin{class}{Pool}{6, 2}
    \end{class}
    \begin{class}{Volume}{6, 4}
    \end{class}
    \begin{class}{Host}{6, 6}
    \end{class}
    \begin{class}{Cluster}{0, 2}
    \end{class}
    \begin{class}{Snapshot}{0, 4}
    \end{class}

    \composition{Cluster}{pools}{1..*}{Pool}
    \composition{Pool}{disks}{1..*}{Disk}
    \composition{Pool}{volumes}{1..*}{Volume}
    \composition{Volume}{mapping}{*..*}{Host}
    \composition{Volume}{snapshots}{1..*}{Snapshot}
\end{tikzpicture}

整体,集群按两个维度来理解,一是物理资源,二是逻辑资源。物理资源包括节点以及节点上的磁盘,逻辑资料包括存储池、目录、卷、快照等。

\section{物理资源}

\subsection{添加节点}

\subsection{删除节点}

节点建模为有限状态机。

删除条件:
\begin{enumbox}
\item \hl{在删除该节点后,数据是安全的}
\item 为简化问题,一次只能删除一个节点
\item lichd处在stop状态
\item 节点上有哪些pool
\item 每个pool的磁盘分布满足故障域规则
\item pool下卷和快照的数据完整状态
\end{enumbox}

按pool检查,输出pool的故障域、节点、磁盘拓扑结构(开发工具)

满足删除条件后,确保lichd处在stop状态,同时等待恢复完成后,正式从集群中剔除该节点。

删除过程往往比添加过程要复杂。首先要从数据安全上考虑,任何操作下,都要保证数据是安全的。
删除一个节点,要等全部恢复后,才能继续删除节点。

一个节点进入维护模式意味着什么?

基本流程:
\begin{enumbox}
\item 检测可删除条件,如果满足,注册任务到etcd上
\item 区分force两种情况
\item Force==False的过程
\item Force==TRUE的过程
\item *
\item 节点不在线
\item lichd进程状态
\item pool (rack, node)
\item 如果是pool的最后一个节点,则不允许删除
\item 副本数 (pool的副本数不同于其中卷的副本数)
\end{enumbox}

\section{Pool}

存储池是集群的物理划分。把物理节点划分为不同的保护域,一个卷的所有数据只出现在一个保护域内。卷可以跨保护域进行复制和迁移。

默认一个,包括所有节点(显式创建,不默认)。

\hl{与文件系统的类比}:存储池相当于磁盘分区,构成一个树,mount到根分区,多个pool构成chunk对象的森林。
pool包括目录和卷,相当于文件系统包括目录与文件。卷和文件都是逻辑实体,需要跟踪其包含的数据块的详细信息。

从文件系统出发去理解存储系统是一条可行途径。

文件系统:先有分区或磁盘,格式化为文件系统,然后mount到系统的目录树之下。
所谓格式化成文件系统就是在磁盘上建立文件系统的数据结构,包括superblock、索引结构、日志等。

\hl{嵌套、分形、递归}。

% 保护域是物理节点的划分,存储池是存储介质的划分。每块盘只能出现在一个存储池里。

Pool: 逻辑容器

故障域有粒度之分,如磁盘,节点,机架,机柜,数据中心。

存储池内,要满足故障域规则:一个chunk的不同副本,分布在不同的故障域内。\label{rule:faultset}

在所有的\hl{数据分布(初次分配、再平衡和恢复等)}过程中,都需遵循这些规则。

\begin{tcolorbox}

可以参考ceph的CRUSH实现。bucket和device定义了集群的物理拓扑结构,rule定义了数据存取规则,
pool上关联rule,从而定义了pool中卷数据的放置规则。设备即OSD,对应一个物理磁盘。

***

存储池可以取代保护域,定义所有对象的存放位置是一个节点集。

***

存储池可以用来实现tier cache。重定向IO到cache pool。

***

统一概念:保护域,存储池,pool。Consistency Group不同于pool,与物理存取无关,
而是卷的逻辑集合,卷可以来自不同的pool。

\end{tcolorbox}

与存储池有什么同和异?存储池可以看做关联了磁盘的pool,可以看做pool的子类。

% 存储池是disk的集合,与节点无关。但disk所在的节点构成存储池的节点列表,不同存储池的节点可能覆盖。

存储池下,可以创建volume。没有关联磁盘的存储池,不能创建卷。

\hl{chunkid到磁盘物理位置有两级映射:chunk的副本节点列表,节点内chunkid到物理地址的映射}。

在为卷分配chunk的时候,需要确定各个副本的物理存储位置。当前实现是返回不同副本的节点列表。
如果指定了存储池,就需要在存储池所在的节点范围内进行分配,同时要满足故障域和数据均衡规则。

\begin{tcolorbox}
移动采集中存储池要求,相比于目前的逻辑pool,更多是一种设计上的退步。
存储虚拟化的目标,是物理位置无关。我们可以基于逻辑容器,实现基于策略的管理。
所以,\hl{从实现层面,要保留当前pool的功能,按照系统配置确定pool的类型}。
但是,存储池关联磁盘,可以实现高速池等按物理特征划分的功能。
\end{tcolorbox}

% 存储池内,要满足故障域规则(\ref{rule:faultset})

步骤:
\begin{enumbox}
\item 创建pool,但此时不能创建pool的元数据chunk,因为还没有绑定磁盘(需要一全局的地方,存储pool名字)
\item 添加磁盘到pool,并上报给admin节点
\item 当在pool下面创建卷的时候,前提条件是已准备好磁盘,生成pool元数据chunk(位于同一pool里)。
\end{enumbox}

pool元数据chunk必须位于自己的存储池内,如果分布在不同的存储池,不满足故障域条件。

pool引导信息可以存在rootable里(用etcd取而代之)。

原来的/的元数据chunk放置在哪儿?还是不再需要,每个pool有独立的树,多个pool构成森林。

规则:存储池包括其下的所有卷和快照的元数据和数据,都必须存在该存储池关联的磁盘上。
包括原来/system下的内容。

\subsection{lookup}

\begin{compactitem}
\item 从path到id 拆分path,逐级查找(正向)
\item 从id到path 遍历pool,找到即退出(反向)
\end{compactitem}

\section{Volume}

属性:

操作:
\begin{compactenum}
\item rename
\item resize \info{在线扩容}
\item mv
\item copy \change{全量拷贝/增量拷贝} \change{跨存储池拷贝} % change不能出现在box里
\end{compactenum}

\section{Snapshot}

snapshot隶属于卷,无卷则无快照,快照组织成快照树,其中有且只有一个快照是可写快照,即卷的写入点。

\section{Consistency Group}

一致性卷组 \change{Consistency Group}

\begin{shadequote}
Consistency Groups could be useful for Data Protection (snapshots, backups) and
Remote Replication (Mirroring).

The Mirroring support will allow to setup mirroring of multiple volumes in the
same consistency group (i.e. attaching multiple RBD images to the same journal
to ensure consistent replay).

There is already an interest to implement this functionality as a part Mirroring feature:
http://tracker.ceph.com/issues/13295

The snapshot support will allow snapshots of multiple volumes in the same
consistency group to be taken at the same point-in-time to ensure data
consistency.
\end{shadequote}

\section{Mapping}

数据隔离/ACL,数据保护

卷对主机的可见性。一个卷只有映射给了某主机,才可以被该主机访问。


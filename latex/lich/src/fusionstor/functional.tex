\chapter{功能视图}

\begin{tikzpicture}[show background grid]
    \begin{class}{Disk}{6, 0}
    \end{class}
    \begin{class}{Pool}{6, 2}
    \end{class}
    \begin{class}{Volume}{6, 4}
    \end{class}
    \begin{class}{Host}{6, 6}
    \end{class}
    \begin{class}{Cluster}{0, 2}
    \end{class}
    \begin{class}{Snapshot}{0, 4}
    \end{class}

    \composition{Cluster}{pools}{1..*}{Pool}
    \composition{Pool}{disks}{1..*}{Disk}
    \composition{Pool}{volumes}{1..*}{Volume}
    \composition{Volume}{mapping}{*..*}{Host}
    \composition{Volume}{snapshots}{1..*}{Snapshot}
\end{tikzpicture}

四面八方,对上导出iSCSI等服务,对下整个各种物理资源,左右对接运维平台和云平台。

整体,集群按两个维度来理解,一是物理资源,二是逻辑资源。
物理资源包括节点以及节点上的磁盘,逻辑资料包括存储池、目录、卷、快照等。

\section{Pool}

存储池是集群的物理划分。把物理节点划分为不同的保护域,一个卷的所有数据只出现在一个保护域内。卷可以跨保护域进行复制和迁移。

默认一个,包括所有节点(显式创建,不默认)。

\hl{与文件系统的类比}:存储池相当于磁盘分区,构成一个树,mount到根分区,多个pool构成chunk对象的森林。
pool包括目录和卷,相当于文件系统包括目录与文件。卷和文件都是逻辑实体,需要跟踪其包含的数据块的详细信息。

从文件系统出发去理解存储系统是一条可行途径。

文件系统:先有分区或磁盘,格式化为文件系统,然后mount到系统的目录树之下。
所谓格式化成文件系统就是在磁盘上建立文件系统的数据结构,包括superblock、索引结构、日志等。

\hl{嵌套、分形、递归}。

% 保护域是物理节点的划分,存储池是存储介质的划分。每块盘只能出现在一个存储池里。

Pool: 逻辑容器

故障域有粒度之分,如磁盘,节点,机架,机柜,数据中心。

存储池内,要满足故障域规则:一个chunk的不同副本,分布在不同的故障域内。\label{rule:faultset}

在所有的\hl{数据分布(初次分配、再平衡和恢复等)}过程中,都需遵循这些规则。

\begin{tcolorbox}

可以参考ceph的CRUSH实现。bucket和device定义了集群的物理拓扑结构,rule定义了数据存取规则,
pool上关联rule,从而定义了pool中卷数据的放置规则。设备即OSD,对应一个物理磁盘。

***

存储池可以取代保护域,定义所有对象的存放位置是一个节点集。

***

存储池可以用来实现tier cache。重定向IO到cache pool。

***

统一概念:保护域,存储池,pool。Consistency Group不同于pool,与物理存取无关,
而是卷的逻辑集合,卷可以来自不同的pool。

\end{tcolorbox}

与存储池有什么同和异?存储池可以看做关联了磁盘的pool,可以看做pool的子类。

% 存储池是disk的集合,与节点无关。但disk所在的节点构成存储池的节点列表,不同存储池的节点可能覆盖。

存储池下,可以创建volume。没有关联磁盘的存储池,不能创建卷。

\hl{chunkid到磁盘物理位置有两级映射:chunk的副本节点列表,节点内chunkid到物理地址的映射}。

在为卷分配chunk的时候,需要确定各个副本的物理存储位置。当前实现是返回不同副本的节点列表。
如果指定了存储池,就需要在存储池所在的节点范围内进行分配,同时要满足故障域和数据均衡规则。

\begin{tcolorbox}
移动采集中存储池要求,相比于目前的逻辑pool,更多是一种设计上的退步。
存储虚拟化的目标,是物理位置无关。我们可以基于逻辑容器,实现基于策略的管理。
所以,\hl{从实现层面,要保留当前pool的功能,按照系统配置确定pool的类型}。
但是,存储池关联磁盘,可以实现高速池等按物理特征划分的功能。
\end{tcolorbox}

% 存储池内,要满足故障域规则(\ref{rule:faultset})

步骤:
\begin{enumbox}
\item 创建pool,但此时不能创建pool的元数据chunk,因为还没有绑定磁盘(需要一全局的地方,存储pool名字)
\item 添加磁盘到pool,并上报给admin节点
\item 当在pool下面创建卷的时候,前提条件是已准备好磁盘,生成pool元数据chunk(位于同一pool里)。
\end{enumbox}

pool元数据chunk必须位于自己的存储池内,如果分布在不同的存储池,不满足故障域条件。

pool引导信息可以存在rootable里(用etcd取而代之)。

原来的/的元数据chunk放置在哪儿?还是不再需要,每个pool有独立的树,多个pool构成森林。

规则:存储池包括其下的所有卷和快照的元数据和数据,都必须存在该存储池关联的磁盘上。
包括原来/system下的内容。

\subsection{lookup}

\begin{compactitem}
\item 从path到id 拆分path,逐级查找(正向)
\item 从id到path 遍历pool,找到即退出(反向)
\end{compactitem}

\section{Volume}

属性:

操作:
\begin{compactenum}
\item rename
\item resize \info{在线扩容}
\item mv
\item copy \change{全量拷贝/增量拷贝} \change{跨存储池拷贝} % change不能出现在box里
\end{compactenum}

\section{Snapshot}

snapshot隶属于卷,无卷则无快照,快照组织成快照树,其中有且只有一个快照是可写快照,即卷的写入点。

\section{Consistency Group}

一致性卷组 \change{Consistency Group}

\begin{shadequote}
Consistency Groups could be useful for Data Protection (snapshots, backups) and
Remote Replication (Mirroring).

The Mirroring support will allow to setup mirroring of multiple volumes in the
same consistency group (i.e. attaching multiple RBD images to the same journal
to ensure consistent replay).

There is already an interest to implement this functionality as a part Mirroring feature:
http://tracker.ceph.com/issues/13295

The snapshot support will allow snapshots of multiple volumes in the same
consistency group to be taken at the same point-in-time to ensure data
consistency.
\end{shadequote}

\section{Mapping}

数据隔离/ACL,数据保护

卷对主机的可见性。一个卷只有映射给了某主机,才可以被该主机访问。

\section{属性}

\begin{enumbox}
\item 副本数
\item 故障级
\item 精简池
\item 磁盘列表
\item 复制类型(复制,EC)
\item 配额
\item \hl{有足够的故障域,且不同故障域配置一致的资源量}
\end{enumbox}

相关类:disk、volume

删除操作时,影响到相关类。

\section{操作}

\begin{enumbox}
\item pool create
\item pool rm
\item pool info or stat
\item pool list all
\item pool add disk
\item pool remove disk
\item pool list disk
\item 扩展(添加磁盘到\hl{已存在的存储池},该映射关系持久化到本地,同步到admin节点)
\item 缩容(从存储池中移除磁盘,引发数据重建过程)
\item 不同存储池之间,卷的复制
\item 不同存储池之间,卷的迁移,可在线或离线
\item 存储池级别的统计信息
\item \hl{自动或手动按磁盘速率进行存储池分级划分}
\end{enumbox}

\section{迁移}

离线/在线迁移

不改变卷ID。卷ID和chunkid集群内唯一,迁移过程中保持不变。

怎么判断一个卷是离线还是在线?有无访问者。target上的链接数,
每个卷的volume\_proto都有一个connect\_list。

\begin{lstlisting}[language=bash,frame=single]
# lich.inspect --connection /iscsi/p1/v1
\end{lstlisting}

基于快照实现存储池的迁移和复制。相当于clone了新卷,chunkid皆发生变化。

\section{复制}

卷可以建模为状态机,不同状态允许不同的操作,按条件在不同状态之间切换。

\section{属性}

\begin{lstlisting}[language=c,frame=single]
typedef struct {
        fileid_t id;
        uint16_t magic;
        uint16_t repnum;
        uint64_t snap_rollback;
        uint64_t snap_version;
        uint64_t reference;         //clone reference
        uint32_t attr;
        int32_t  priority;
        uint32_t __pad__[4];

        uint64_t size;
        uint32_t mode;
        uint32_t uid;
        uint32_t gid;
        uint32_t ctime;
        uint32_t mtime;
        uint32_t btime;
        uint32_t atime;
} fileinfo_t;
\end{lstlisting}

卷是ServerSAN核心对象,是pool,snapshot,mapping和cg的中心。

卷的属性记录在L1 chunk的fileinfo段。fileinfo段是vol info区的第一个段。

\subsection{大小}

目前,卷的元数据由两级组成:L1,L2。L1只有一个chunk,每个chunk有8000个槽位。每个槽位指向一个L2 chunk。
L2 chunk有16000个槽位,每个槽位指向一个raw chunk。所以,最大卷大小约为$122TB = 8000 \times 16000 \times 1MB$。

为了支持更大的卷,需要扩展此结构,把L1扩展到多个chunk。改变会影响到:
\begin{compactitem}
\item 加载table1 (加载多个chunk,记录每个chunk的chunkinfo信息)
\item table1的各项操作
\item 遍历卷的chunk
\item 数据恢复过程
\item migrate
\item copy
\item snapshot clone,需要copy L1 chunk
\item ...
\end{compactitem}

L1 chunk:vol.xxx.0, vol.xxx.1, vol.xxx.2。其中,vol.xxx.0的chunkinfo记录在pool。
\hl{vol.xxx.1等chunkinfo记录到vol.xxx.0里info区的第二个块,并需记录其个数}。动态加载。

父:raw和subvol的父chkid为vol.xxx.0,vol.xxx.1的父chkid也为vol.xxx.0。vol.xxx.0的父chkid为所在pool的chkid。

缺页中断:从上到下检查。创建chunk的时候,先检查L1 chunk是否存在,然后检查L2 chunk是否存在。
不存在,则创建。在以下过程会改变卷的数据结构树:
\begin{compactitem}
\item \verb|__pool_proto_mkvol|
\item \verb|volume_proto_load|
\item \verb|volume_proto_chunk_pre_write|
\item \verb|__volume_proto_chunk_allocate|
\end{compactitem}

\begin{lstlisting}[frame=single]
lichbd write /iscsi/p1/v1 hello -o 280375465082880
lichbd cat /iscsi/p1/v1 -o 280375465082880 -l 5
\end{lstlisting}

chunk allocate开销较大,即要申请磁盘空间,也涉及记录元数据到父节点。

遗留问题:
\begin{enumbox}
\item lich.inspect --stat极慢 (统计,load)
\item 分析内存占用量
\item clone
\item recovery
\item balance
\item 回收空间
\end{enumbox}

\subsection{副本数}

\section{操作}

\begin{enumbox}
\item create
\item rm
\item ls
\item info
\item resize
\item rename
\item mv
\item copy (read/write)
\item migrate (move all chunk)
\item duplicate (snapshot-based: clone/flatten)
\item import
\item export
\item mapping
\item \hl{IO}
\end{enumbox}

\subsection{create}

\subsection{rm}

显式回收空间

分为两阶段:标记和回收。先把卷移入/system/unlink,有后台异步线程负责回收。

相关源文件:
\begin{compactitem}
\item pool\_rmvol.c
\item rmvol\_bh.c
\end{compactitem}


\subsection{QoS}

token bucket。IOPS与block size相关,两者乘积等于带宽。
如果各层发生IO聚合,则在流量守恒的情况下,显示IOPS有所不同。

IOPS必须假定一定的block size。

抽象出AbstractVolume,Volume、Snapshot、Clone都是其子类。

\hl{卷或克隆卷扩容后,快照没有相应扩容,会返回ENOENT,需要填充0}。

\section{问题}

\begin{enumbox}
\item session方式,无commit
\item 随机IO,小cache,频繁swap,有问题
\item 如果标记dirty,会出现all dirty的情况
\item 如果有COW,需要pin住源
\end{enumbox}

\section{操作}

需要支持的快照操作:
\begin{enumbox}
\item create
\item rm
\item list
\item rollback
\item clone(跨卷read)
\item read
\item flatten(存储池迁移和复制,需要实现该接口)
\item protect/unprotect
\end{enumbox}

\begin{tabular}{|s|p{0.6cm}|p{0.6cm}|p{0.6cm}|p{0.6cm}| }
    \hline
    \rowcolor{lightgray} \multicolumn{5}{|c|} {Snapshot} \\
    \hline
    Feature & COW & ROW1 & ROW2 & LSV \\
    \hline
    snap tree & N &  & & \\
    \hline
    snap create & N &  & & \\
    \hline
    snap rm -> gc & N &  & & \\
    \hline
    snap rollback & N &  & & \\
    \hline
    snap clone -> read & N & & & \\
    \hline
    write & N &  & & \\
    \hline
    read & & N & & \\
    \hline
    space & N &  & & \\
    \hline
    consistency group & N &  & & \\
    \hline
\end{tabular}

采用COW,ROW或两者的组合形式,各有优缺点。

COW方式的快照,卷有完整的索引结构。别的快照点,只有增量的索引结构和发生更新的数据。
每个快照,存储的是创建之后,到下一个快照点之间发生更新的所有数据块。所以需要尽量降低发生copy的开销。
适用于频繁且具有局部性的热点负载场景,固定时间段内,每次copy的开销以及\textcolor{red}{复制集的大小}。

ROW方式下,如果meta不发生COW,新的快照点并无完整的索引结构,读过程需要沿着快照链向上回溯。
并且,rm,rollback等操作需要合并快照点。

如果发生了COW,索引项和数据项引用关系不再是1:1,而是多对多,需要专门的GC机制。
但rm,rollback等操作实现起来变得简单。

数据项的粒度,定长或变长,不同的负载,读写性能有不同的影响。
发生COW的粒度,索引结构管理的粒度决定了数据项的粒度。

IO粒度和数据项粒度的关系,如果IO粒度大于数据项粒度

如果IO粒度小于数据项粒度

这条规则是否永远成立:\textcolor{red}{快照树的任一路径,都需要具有完整的索引结构和数据项集,可以有冗余和共享}。

\section{COW}

为了支持快照树,需要分析rollback和读快照的过程。

\subsection{Create}

卷,clone卷,快照具有相同的索引结构,不同在于fileinfo的attr。

卷和快照的所有chunk,都是私有的,无法通过索引结构有效地定位chunk的位置。

创建快照是相当简单的过程,创建索引结构,设置正确的fileinfo,同时复制xattr。
卷的snap\_from指向其挂载的快照。

在写入卷的时候,发生COW过程,被修改的chunk先复制到卷的最近一个快照的私有存储空间里,
然后在卷上进行in place写。

\subsection{Rollback}

rollback,卷为源,以目标快照的内容覆盖源卷的内容。同时,为了不丢失信息,需要提前保存一定的数据。
需要保存什么数据?沿着快照树的根,到卷的当前快照(所在分支的最后一个快照),\hl{所有修改过的chunk}。

卷的当前快照:只有在卷的当前快照是叶子节点的时候,才需要保持数据。

读快照的规则:
\begin{compactenum}
\item 自身
\item 下游
\item 回溯至卷(先回溯至公共父节点,然后到卷)
\end{compactenum}

前两步是为了读取修改过的chunk,后一步是为了读取所在分支上从来没有修改过的chunk。
两者合起来,代表快照的数据。

通过fileinfo的snap\_from建立快照树,通过snap\_version为每个快照分配id,通过snap\_rollback指定要回滚的目标快照版本号。

rollback的状态变迁

rollback过程依赖于两个快照路径:
\begin{compactenum}
\item root snap list
\item cross snap list
\end{compactenum}


\section{ROW}

% ROW与LSV的不同,在于同一LBA,只对应一个数据项,而不是对应多个版本的数据项。
% 但这个数据项的引用计数,可以为0,1,或多个。

ROW,快照会产生一个数据项的多个版本。LSV,除了快照之外,一个LBA的覆盖写入也会产生多个版本。

卷和快照,共存于底层lich卷,是否有问题?

元数据的\hl{基本管理单元是page}。按page来组织,每个page unit:chunkid+pageid。
在ROW的过程中,可以只改变一页到新chunk。而bitmap则发生了COW,copy 1M,改变其中一项。
(用chunkid+ page bitmap的方式行不通)。\hl{有一种特殊情况,bitmap管理了连续1M的数据,此时chunkid是重复的}。
如无页级元数据,则需要发生1M的COW。有页级元数据,引入了优化的机会。

bitmap管理1M对应底层1M。管理所需元数据:chunkid+各页的元数据。

snap tree, snap, chunk, page构成四级结构。LSV时,没有chunk那一级(写入位置模式)。

写入模式:顺序写入,需要维护当前写入点。1M模式,需要1M:1M的映射, 而不再需要chunk内页索引。
引用关系是页级的,写入模式是chunk级的。降低了ROW的写入粒度。

vvol:clone时用,跨卷引用。支持多级clone。clone与snapshot基本一致,不同在于跨卷读取。

写入的过程:会检查每个isref,入为1,则表明该chunk归属于本快照,否则,需要创建新的。
对L2 bitmap言,发生1M的COW;对LOG言,申请分配1M,同时写入。

cache一致性问题:

session聚合写入:

CC: 页锁,bitmap单元锁,COW锁。\hl{细粒度的锁,无法保护更大粒度锁上的并发操作},如COW过程,ROW过程,
IO锁最好是区间锁。bitmap单元的提交操作。

页式管理的局限性:连续的有信息冗余。

空间管理:\hl{逻辑空间和物理空间的关系,物理空间的扩展性}。

\subsection{IO过程}

\begin{itembox}
\item CC: 范围锁,层次锁
\item 需要的实际IO次数
\item 页对齐
\item 跨chunk边界
\item 重入分析(数据更新,bitmap更新)
\end{itembox}

两个并发IO,如果落入一个1M,会访问到同一bitmap单元。
应先锁定bitmap单元,再锁定IO范围。遵循树状锁协议。
参考\hl{数据库索引结构上的锁协议}。

\begin{tcolorbox}

假设有新旧两个快照点S1, S2,每个快照点包括meta和data,提炼出的几个引导性问题:

\begin{enumerate}
    \item 数据块的粒度,page,chunk,extent?
    \item 每个快照点是否有完整的索引结构?
    \item 卷(写入点)是否有完整的索引结构?
    \item 哪一个是写入点?
    \item meta是否发生COW?
    \item data是否发生COW?
    \item 快照操作(create, rm, rollback, clone, flatten, ls, cat, protect/unprotect)的复杂度?
\end{enumerate}

\end{tcolorbox}

\section{ROW2}

\subsection{快照树}

快照树的每个节点,都是一个完整的索引结构。其中一个代表着当前写入点,写入点代表着卷。
回滚操作会改变当前写入点。

统一一下,卷和快照具有相同的索引结构。

每个快照结构有唯一的ID,是随着创建快照的过程递增的,卷快照/写入点具有最大的snap id。
LSV log结构记录了每页的snap id。

一颗树的节点,可以分为三类:
\begin{compactenum}
\item 根节点
\item 中间节点(有无分支)
\item 叶子节点
\end{compactenum}

\subsection{创建快照}

原有写入点变成只读,创建新的写入点,并复制L1元数据。

\subsection{删除快照}

快照树上不同的节点,需要不同的删除过程。对于叶子节点,直接删除即可,\textcolor{red}{需要回收数据吗}?

回收快照,涉及meta和data两个部分。

不应改变父快照的内容
保留子节点的共享内容

分两个节点:\hl{标记和回收}。

\subsection{列出快照树}

\subsection{写}

写是ROW的重定向过程,可能发生复制第二层元数据的过程。

如果连续写入,需记录当前的写入点。chunk内数据是逻辑不连续的,或者说,逻辑上连续的在chunk内是随机化了。

如果非连续写入,每页的写入位置计算得到,chunk内数据是逻辑连续的。大范围随机写入的情况下,需要分配很多chunk。
\textcolor{red}{需大力优化chunk的分配过程}。

pagetable是逻辑连续的,按逻辑空间组织。如果已分配位置,则覆盖写入。如果没有分配位置,如何分配?

逻辑地址怎么映射到物理地址?页式,段页式

\hl{写入位置,有三种分配策略:ROW2,ROW3,LSV},都是按页粒度管理物理卷空间。

\subsection{回滚}

回滚并不会重用回滚到的快照点,而是相当于把写入点嫁接到目标快照点。写入点本身是一个独立的快照结构。

除了写入点的所有快照点,都处于只读状态,没有任何操作可以改变其状态。

回滚后的写入过程

\subsection{读}

如果不复制元数据,ROW实现的读过程需要回溯快照树,性能不佳。
如果复制元数据,则每一快照点都具有完整的索引结构,可以做到一次即可定位。

复制元数据,快照和数据具有多对多的引用关系,相当于共享数据块。

读优化: 元数据可以一次定位,但可能碎片化,沿着快照链往上读取。可以通过flatten的过程优化。

\subsection{Clone}

卷和其上的快照位于同一个物理卷内,所以不涉及跨卷读。clone后,会用到跨卷读快照,
类似于ROW过程,所以,创建和clone过程具有相似性。

cat, protect, unprotect, flatten

\subsection{GC}

GC过程和引用计数。每个记录的is\_ref记录是否写入了数据。可以回收写入点,但无法回收一般快照。

LSV: 逐个扫描每个log。

\section{实现}

快照列表信息记录在卷table\_proto的smap和sinfo区,info区的每一项包括name和chkinfo,还包括snap\_attr\_t。
smap用bit标记该位置是否有效。卷上的所有快照,构成快照树。

每个快照有自己的snap\_version和snap\_from,分别表示自己的版本号和父节点的版本号。

新创建的快照,采用卷当时的版本号和父版本号,创建成功后,更新卷版本号和父版本号,rollback版本号与卷版本号一致。

revert的时候,rollback版本号表示目标快照的版本号,如有autosnap生成,采用卷当时的版本号和父版本号。
完成后,更新卷的版本号,父版本号和rollback版本号。

卷的每一chunk具有自身的版本号,revert成功后,所有chunk的版本都等于目标快照版本号。

\subsection{rollback}

主要通过4个变量来控制rollback过程:fileinfo(snap\_version)。

/pool1/system/rollback/vol.32.0

\section{Bitmap Cache}

要解决的关键问题有:
\begin{enumbox}
\item cache一致性
\item 并发与事务
\end{enumbox}

问题导向
\begin{enumbox}
\item 这是最好的设计/实现吗?
\item 需要维护什么序?哪些操作可以并发?
\item chunk的is\_ref与bunit的ref什么关系?
\item 怎么区分snapshot与clone两种情况?
\end{enumbox}

chunk的is\_ref表示引用别的快照的L2 bitmap chunk。
在创建snapshot时,设为1。发生COW后,设为0。

如果chunk的is\_ref == 1,则读出的bunit.ref也等于1。
(发生COW后)
如果chunk的is\_ref == 0,则读出的bunit.ref可能有两种值。

在COW后,chunk的is\_ref等于0,对应的所有bunit都等于1。随着卷上发生写操作,对应的bunit也变为0。
所以两者的含义都是\hl{所指向的对象是否属于卷本身},用于判断COW与数据写入过程是否需要分配数据块的条件,
两者都需要持久化。

每个快照对应一个bitmap,用于映射每个数据页的LBA到底层物理地址。
bitmap分两层:第一层私有,第二层在各个快照之间共享。
在写的时候,触发COW过程,从父快照copy对应的chunk,然后才能写入。

一个卷的多个快照,只有一个快照是可写的,其它快照都处于Read-only状态。
所有写的位置,必须是卷本身的数据空间。所以,对任一数据chunk,都是在可写快照时分配的,创建快照后冻结,产生新的写入点。

为了提高bitmap的存取性能,引入bitmap cache,按页组织(可以类比于VFS的page cache,每个inode对应一个radix tree)。
更新bitmap的粒度是bunit,通常为8B,对应一个底层物理位置,一次io可能对应多个bunit,甚至可能跨越page边界。

\subsection{Read}

\subsection{Write}

IO写的时候,对应的每一个bunit,都存在四种状态:
\begin{enumbox}
\item 本地
\item 没有分配
\item 指向某快照
\item 克隆卷,指向克隆卷的源快照
\end{enumbox}

后三种情况统一处理。

如果是单页更新过程,

如果是多页更新过程,

\section{ConsistencyGroup}

使用场景:如一个分布式数据库,log写入一个卷,数据写入另一个卷,如何打快照才能不违反应用层的一致性?

LVM的快照,涉及底层多个物理卷。clvm

ceph组内各卷,共用一个journal,性能低、架构不灵活

CG是一个独立实体,创建后为空,可以把一些卷加进来,按组进行快照操作。

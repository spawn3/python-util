\chapter{QOS}

\section{概述}

学习的方法:
\begin{enumbox}
\item \hl{对标}:行业的标准做法是什么?
\item 如何才能更好地学习?
\item *
\item 先选出几篇经典论文,顺藤摸瓜,建立相关的知识体系。
\item 与专业人士交流,获取有价值的线索。
\item 还需要主动去悟,提问、消化、守破离,推陈出新
\end{enumbox}

参考网络QoS,存储QoS的核心算法与网络QoS相同。

集中式控制、分布式控制

排队论

态势感知?

在高IOPS的情况,QoS的开销过大,极大地拉低了性能,这是不可接受的。

每次请求都要获取一次时间,是不是必要的?

\subsection{参考}

\begin{enumbox}
\item OS中进程、线程调度算法
\item Disk IO调度算法
\item VM IO调度算法
\item Network QoS and Storage QoS
\item TCP/IP
\item iSCSI
\item SPDK QoS
\item Ceph dmClock
\item SolidFire QoS
\end{enumbox}

\section{算法}

采用了两种曲线

开放控制参数

比较指标:理论和实测值的距离,\hl{也可以考虑夹角的大小}。\change{距离函数}

底层采用token bucket,需要能容忍一定的jitter。

在调度器内加入QoS控制逻辑的设想: 每个core调度器对应一个或若干卷控制器。基于优先级队列,由core线程处理队列(scheduler队列?)。
每个卷控制器在对应的scheduler上注册自己的队列(IO任务、恢复任务)。 \hl{core上的每个卷,向scheduler注册自己,从而实现解耦}。
调度器不仅可以处理单个卷的QoS,也可以处理多个卷的QoS。

\hl{队列和线程}往往紧密结合为一体,参见SEDA、actor。

\hl{多mode调度器},根据实际负载条件动态地调整调度器策略。

何时从请求队列移入调度队列是QoS调度器的中心任务。

% \section{Quota}

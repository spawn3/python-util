\chapter{开发视图}

设计在解决关键问题的同时,要降低实现,测试和维护的复杂度。

加强测试,通过重构降低复杂度。

分解问题,界定边界,降低复杂度

设计的基本原则

分离机制和策略,接口和实现

性能依赖于设计,在一定的设计下,取决于实现。

性能优化手段:并行,聚合,缓存等,根本在于设计,控制复杂度。

识别实体和关系,ERD,DFD等,FSM是机器语言。

核心概念:
\begin{compactenum}
\item 存储池/目录
\item 卷
\item 快照
\item 主机映射
\end{compactenum}

core thread边界,core\_request进入。

分布式副本一致性:clock版本机制,msgqueue离线消息处理。

性能:并发,聚合和cache等

元数据管理,非计算而来

快照树的实现

后台任务统一管理,包括:
\begin{compactenum}
\item recovery
\item balance
\item vol rm
\item snap rm
\item snap rollback
\item snap flat
\end{compactenum}

架构问题:
\begin{compactenum}
\item 元数据管理成本
\item 支持大容量卷
\item 支持ROW快照树
\item 诊断流程和工具
\item 性能profile
\end{compactenum}

\section{故障域}

对任一存储池,设故障域数为M,副本数为N,

当M>=N时,每个故障域内一个副本,随机分布;
当M<N时,
- 策略1,每个故障域内一个副本
- 策略2a,剩余的副本按策略1进行,直到写完所有副本数
- 策略2b,不写剩余副本

按策略2a:

case 1:故障域为2,副本数为3,则副本在故障域的分布为(2,1)或(1,2)
case 2:故障域为1,副本数为3,则副本在故障域的分布为 3

按策略2b:

case 1:故障域为2,副本数为3,则副本在故障域的分布为(1,1)
case 2:故障域为1,副本数为3,则副本在故障域的分布为(1) (副本数不能少于2个,分配失败)

同时,恢复过程须按以上故障域规则进行自动校正!!!

\section{存储池状态}

\begin{compactenum}
\item 不可用
\item 磁盘空间不足/READ ONLY
\item 降级
\item 正常/健康
\end{compactenum}

\section{诊断方法}

对需要改进的流程进行区分,找到最有潜力的改进机会,优先对需要改进的流程实施改进。如果不确定优先次序,企业多方面出手,就可能分散精力,
影响6σ管理的实施效果。业务流程改进遵循五步循环改进法,即DMAIC模式:

\begin{compactenum}
\item 定义[Define]——辨认需改进的产品或过程,确定项目所需的资源。
\item 测量[Measure]——定义缺陷,收集此产品或过程的表现作底线,建立改进目标。
\item 分析[Analyze]——分析在测量阶段所收集的数据,以确定一组按重要程度排列的影响质量的变量。
\item 改进[Improve]——优化解决方案,并确认该方案能够满足或超过项目质量改进目标。
\item 控制[Control]——确保过程改进一旦完成能继续保持下去,而不会返回到先前的状态。
\end{compactenum}

信息有多级:USE。诊断问题依赖于结构化的诊断方法PAT,解决问题也是,构建知识图谱。

欲分析问题,必分析事物发展的完整过程,包括每个参与者的生命周期模型,参与者之间的相互作用。

\section{目录结构}

\begin{enumbox}
\item /opt/fusionstack/data
\item /opt/fusionstack/data/etcd
\item /opt/fusionstack/data/node
\item /opt/fusionstack/data/disk
\item /opt/fusionstack/data/chunk
\item /opt/fusionstack/data/status
\item /opt/fusionstack/data/recovery
\item /opt/fusionstack/data/cleanup
\item 
\item /dev/shm/lich4
\item /dev/shm/lich4/msgctl
\item /dev/shm/lich4/nodectl/*
\item /dev/shm/lich4/maping
\item /dev/shm/lich4/clock
\item /dev/shm/lich4/hb\_timeout
%\item /dev/shm/lich4/recovery
\item /dev/shm/lich4/volume
\end{enumbox}

\section{编程注意事项}

\begin{compactitem}
\item goto之前,先设置ret
\item 大多数情况下,需要检查函数的返回值
\item 函数的可重入性
\end{compactitem}

\section{Protocol}

Protocol与存储卷是正交功能。protocol应该独立于卷进行扩展。
所以,path中包含protocol组件,并非必须,而是实现上的权宜之计。

\section{Algorithm}

\begin{compactitem}
    \item paxos/raft 选举admin和meta节点
    \item lease controller的唯一性
    \item vector clock 副本一致性
\end{compactitem}

\section{Coroutine}

使用规则:
\begin{compactitem}
    \item \verb|schedule_task_get|和\verb|schedule_yield|要匹配
    \item task有数量上的限制:1024,超出后容易引起死锁
    \item schedule\_self 与 schedule\_running不同
\end{compactitem}

首先进行一种区分,core线程内与外;线程内进一步分为下述两张情况。

core调度器本身不同调用blocking操作,同样需要异步化。分两种情况:
一、core调度器本身调用block op;二、core内协程调用block op。
第二种情况可以用schedule\_yield(或core\_request),第一种则不可。如何保序?

\section{plock}

一个task有两种状态:一、持有锁;二、等待。

分为两种task:一、持有锁;二,等待。

读操作与写操作不对称。

序列:
\begin{enumbox}
\item rrrrrrrr
\item wwwwwwww
\item rrwwrrww
\item wwrrwwrr
\end{enumbox}

\section{AIO}

生产者-消费者模型。

\subsection{sqlite3}

异步sqlite3。etcd,RR也会采用该框架。

\begin{compactitem}
    \item 共10个db,每个一个线程,每个线程管理一个队列
    \item 所有sqlite3操作,泛化为统一的结构,放入线程队列
    \item 消费者线程批量处理队列中的任务
    \item 生产者线程和消费者线程通过sem进行通信
    \item 采用协程机制(yield/resume)同步任务执行顺序
\end{compactitem}

消费者线程wait在sem上,生产者线程有消息的时候,调用\verb|sem_post|。

\section{Performance}

4K+1M混合读写,极慢

机械盘关闭localize

IO路径优化
\begin{enumbox}
\item 调用深度
\item DBUG/DINFO等日志
\item 无关功能
\item 代码体积大,导致cpu高速缓存失效: -g
\item lock table用到了hash table
\item hash table用到了ymalloc/yfree机制管理内存
\item table2 subvol wrlock
\item inline
\end{enumbox}

\subsection{Lease}

利用lease机制来保证volume controller的唯一性。

加载卷时,尝试创建lease,成功后才能执行加载过程。

若没有IO,如果发生lease超时,admin会回收发生超时的lease。后续如有IO,需要重新申请。

如同锁一样,lease会发生抢占。lease是带timeout的锁。现在的实现,需要client去频繁检查,而不是通知机制。

(心跳,向量时钟,版本)能改善?

通过VIP机制,访问卷控制器的过程,与不同VIP,有所不同。

EREMCHG错误主要用于控制器发生切换的时候。

\subsection{Hash}

chkid如何选择hash函数?

\subsection{分支预测}

\subsection{Profile}

lich.conf: performance\_analysis: on

kill -USR1 <pid>

tail -f /opt/fusionstack/log/lich.log |grep analysis

\section{故障}

下电,感知有一定延迟,tcp timeout。

重新选举admin

\section{Safe Mode}

卷级进行检查,处在保护模式的卷,不允许iscsi连接,返回错误码。

加载时间较长的模块,用half sync/half async模式来处理。分为两阶段:同步+异步。

一个卷,加载成功,依赖于几个条件:
\begin{compactitem}
    \item raw/lsv
    \item module load
\end{compactitem}

\section{Log}

syslog

日志过滤

统计数据:准确性和实时性
软件工程

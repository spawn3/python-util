\chapter{Pool}

\section{属性}

\begin{enumbox}
\item 副本数
\item 故障级
\item 精简池
\item 磁盘列表
\item 复制类型(复制,EC)
\item 配额
\item \hl{有足够的故障域,且不同故障域配置一致的资源量}
\end{enumbox}

相关类:disk、volume

删除操作时,影响到相关类。

\section{操作}

\begin{enumbox}
\item pool create
\item pool rm
\item pool info or stat
\item pool list all
\item pool add disk
\item pool remove disk
\item pool list disk
\item 扩展(添加磁盘到\hl{已存在的存储池},该映射关系持久化到本地,同步到admin节点)
\item 缩容(从存储池中移除磁盘,引发数据重建过程)
\item 不同存储池之间,卷的复制
\item 不同存储池之间,卷的迁移,可在线或离线
\item 存储池级别的统计信息
\item \hl{自动或手动按磁盘速率进行存储池分级划分}
\end{enumbox}

\section{迁移}

离线/在线迁移

不改变卷ID。卷ID和chunkid集群内唯一,迁移过程中保持不变。

怎么判断一个卷是离线还是在线?有无访问者。target上的链接数,
每个卷的volume\_proto都有一个connect\_list。

\begin{lstlisting}[language=bash,frame=single]
# lich.inspect --connection /iscsi/p1/v1
\end{lstlisting}

基于快照实现存储池的迁移和复制。相当于clone了新卷,chunkid皆发生变化。

\section{复制}

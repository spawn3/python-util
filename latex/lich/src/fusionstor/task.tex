\chapter{任务管理}

后台任务处理,或事件驱动,或定时器驱动。
线程或线程池按一定方式组织,采用actor或csp并发编程模型。

为了在集群级别进行控制,采用主控、或定向策略,提供可控性和性能。
Nutanix的Mapreduce并行执行模型。

\section{平衡}

\subsection{控制器平衡}

每个卷到core的映射是可计算出来的,可按core逐个进行平衡。
先处理core hash等于0的卷,依次类推,同时使迁移量尽量少。

以上算法假定每个卷都是等价的。如果卷的大小和负载不同,目前方案难以做到真正的平衡。

\subsection{数据平衡}

\begin{enumbox}
\item 按pool划分
\item 每个pool,分布在多个节点上,每个节点可得其磁盘利用率
\item 符合move条件的raw chunk,确定其新的副本位置集
\item 迭代执行,重新评估
\end{enumbox}

\section{恢复}
活动的可追踪性,比如scan阶段的时间,是分析系统行为至关重要的指标。
做到重要事件有迹可查,可以审计。

采用SEDA架构组织处理过程,分scan和recovery两阶段,边扫描,边恢复。
pipeline

QoS,参考操作系统的进程调度(MLFQ),网络的流量控制,多级反馈队列等。

采用强化学习的思路动态调整控制参数。

区分节点故障和磁盘故障,磁盘故障时可以利用sqlite信息,直接扫描。

移除磁盘时,调用\verb|md_chunk_set|,设置该磁盘上所有chunk的chkinfo的对应副本状态为\verb|_S_CHECK__|。
恢复过程,根据chkinfo的副本状态,判断是否需要恢复。如果需要恢复,调用\verb|md_chunk_check|。

\verb|md_chunk_set|过程开销较大,故感知磁盘故障的延时较长。\hl{把副本状态持久化到chkinfo里,是否真的必要}?

\hl{节点下线}触发怎么样的过程? \verb|network_connect1|。

故障域规则

单个磁盘

单个节点

单个机柜

重建性能

不中断IO

\subsection{需求}

恢复性能和Qos机制。

控制系统后台数据恢复进程所占用的带宽和 IO 处理资源,提供多种数据恢复的优先级策略。

至少提供优先前端应用或优先数据修复这两种优先级策略。在优先前端应用的策略中,数据
修复仅在资源较空闲时进行;在优先数据修复策略中,后台恢复以最快速度完成,
但卷仍然在线保持可用,仅性能有所降低。

作为作业管理,展示恢复进度,关键参数和资源消耗情况。

当有app io时,恢复性能变慢,原因是任务调度吗?

故障恢复过程中,数据迁移和写入,遵循 tier或 cache策略,优先写SSD,每节点SSD数量有限,
在同时承担业务访问情况下,极易成为瓶径,本次实测恢复数据每节点 10MB/s 以下,
恢复QoS无法体现作用。节点中机械盘不能并发参与恢复。

恢复QoS当前每节点分别命令处理,需要换一种全局可控的方式(例如带宽或网卡百分比,这部分产品先找资料进行参考)

rebalance 是否也是优先写SSD? 应与恢复相同策略,不设置QoS情况下,
应最大限度利用全部机械盘进行并发数据恢复,减少或不占用SSD资源。

\subsection{相关特性}

恢复策略与磁盘热插拔,磁盘漫游相关。

\subsection{设计}

至少可以设定两个等级:恢复优先,或应用优先。至于恢复线程数,等待时间,更多是实现细节,用户很难自己设定。
所以,需要更高级的控制语义。

目标,偏离,反馈,调节,达到控制的目的。分析上下限,临界情况,执两用中。

串行恢复一个chunk的性能,并发情况下的加速比(串行是基准)。

在不受限制的情况下,先优化到最大性能,能否达到硬件瓶颈?

需要有约束的机制和策略。控制并发度和等待时间。

在每个节点上运行恢复线程,分两个阶段:
\begin{itemize}
\item 扫描,确定需要恢复的chunk
\item 恢复,多线程执行
\end{itemize}

每个节点只处理属于本节点的数据,即主副本位于本地节点。从sqlite扫描chkid,逐个检查其repnum,
是否与fileinfo一致,如果不一致,追加到一临时文件
\begin{itemize}
\item chkinfo.repnum是实际副本数(需要进一步查看各副本的真实状态)
\item fileinfo.repnum是目标副本数
\end{itemize}

控制参数:
\begin{itemize}
\item 功能开关
\item 数据恢复的最大带宽(\hl{token bucket 填充速率})
\item 线程数
\item 线程处理完一个chunk后的sleep时间
\end{itemize}

前端优先对应的控制参数是什么?恢复优先对应的控制参数是什么?自定义QoS?

系统级别的配置信息存在哪儿?结构采用KV,有几种方案可供选择:
\begin{itemize}
\item etcd           增加了外部依赖性
\item ZK             增加了外部依赖性
\item /config/       lich运行后,方可访问
\item /dev/shm/lich4 非持久化
\end{itemize}

单次扫描的条目不宜过多。

\subsection{实现}

配置目录:/dev/shm/lich4/nodectl/recovery

\begin{itemize}
\item immediately       功能开关
\item interval          恢复任务的时间间隔
\item thread            工作线程数
\item suspend
\item qos\_maxbandwidth 设置最大带宽
\item qos\_sleep        设置sleep多少豪秒
\item qos\_cycle        设置计算带宽的时间间隔
\item info              输出结果
\end{itemize}

md\_chunk\_getinfo

md\_chunk\_check

\subsection{测评}

从串行到并行

\begin{itemize}
\item 各阶段时间
\item 磁盘利用率
\item 并发度
\end{itemize}

为什么在有前台IO的情况下,性能严重下降?

\subsection{运行}

\begin{tcolorbox}
    自动触发:Recover 每10分钟执行

    手动触发:echo 1 > /dev/shm/lich4/nodectl/recovery/immediately
\end{tcolorbox}

\section{GC}

chunk

\section{Clock Merge}

\section{监视存储池状态}

由admin节点执行

\section{删除卷}

任务规范放入etcd更合适,避免因为ENOSPC,导致删不了。

\section{删除快照}

\section{回滚快照}

\section{flatten}

%\section{HSM}

%分层存储管理
%\begin{itemize}
%\item \verb|__volume_ctl_analysis_init|
%\item \verb|replica_hsm_init|
%\end{itemize}

%\section{SSD Cache}

%\begin{itemize}
%\item \verb|__flush_create|
%\end{itemize}

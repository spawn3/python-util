\chapter{优化项}

\section{时间优化}

快

\begin{enumbox}
\item localize
\item ssd cache
\item auto tier
\end{enumbox}

\section{空间优化}

省

\begin{enumbox}
\item 精简配置 (Thin provisioning)
\item EC
\item Dedup (VDO)
\item Compress
\end{enumbox}

\section{架构优化}

好、多

\begin{enumbox}
\item 副本一致性的等级和证明
\item 控制器采用更小的控制粒度,划分subvolume
\item 控制器与iscsi target的关系
\item ROW3快照
\item 优化chunk allocate效率
\item 恢复:磁盘故障:从哪里读,写到哪里?是否充分并行了?
\item 高性能架构
\item 数据安全(DR)
\item 监控SSD
\end{enumbox}

% \section{Security}
% iscsi CHAP认证

\section{QOS}

token bucket

\change{距离函数}

% \section{Quota}

\section{缓存}

\subsection{Redis Cache}

卷控制器所在节点,IO入口处,分页。当切换卷控制器的时候,清空redis cache。

采用redis LRU或FRU算法。

\begin{compactitem}
    \item \verb|volume_proto_write|
    \item \verb|volume_proto_read|
\end{compactitem}

read过程,从io的两段向中间压缩,选取中间最大的加载区。

若发生volume controller切换,源端能否及时感知该事件,会有清理过程吗?怎么使源端的cache数据失效?
在cache里,有两级检查机制:卷和页。有总开关,分级,如果设定卷级状态失效,则全体页也是失效的。

为什么会发生volume controller切换?主动move,节点故障,iscsi session切换等。

切走,失效化,切回。如果在切回之前没有失效化,则有\hl{cache一致性问题}。

如何检测到该事件?

\begin{tcolorbox}
每发生一次切换,\verb|info_version|即+1,在redis key里加入该信息。
利用redis自身的置换机制,则切换不会造成cache一致性问题。
\end{tcolorbox}

\begin{compactenum}
\item config: \hl{功能开关}
\item 初始化
\item \verb|info_version|
\item 并发
\item 内存copy
\item batch update redis (mset and mget)
\item 存在,则不kset?
\item redis连接可以用unix domain,相比tcp方式提速1倍左右
\end{compactenum}

遗留问题:
\begin{enumbox}
\item 顺序读,分页后,性能下降
\item 随机读,破坏局部性
\item 内存cache?
\item 必须考虑基准性能,SSD read和redis read,IOPS哪个多?
\item \hl{快照回滚对cache的影响}
\end{enumbox}

即使是连接本地redis,tcp开销也是很大的,最多3w IOPS。如果缓存命中率低,或者分页引起的tcp开销,反而会导致读写性能下降。

实现cache有一些通用问题和瓶颈,可以提前指出。

\subsection{SSD Cache}

特性开关:xattr writeback

入口:\verb|writeback_commit|

节点级别,在添加磁盘时指定\hl{--cache}选项,即创建了SSD cache盘。

journalling? 优化随机IO。写入WAL即可返回。

所有非\verb|__FILE_ATTR_DIRECT__| IO,都流经SSD cache,由QoS机制控制流入流出的速率,达到流入流出的平衡。
同时,通过异步线程flush ssd cache的内容到目标位置。

与分层有什么不同:
\begin{compactitem}
\item IO路径(包括入口)
\item 目标位置
\item 容量
\end{compactitem}

tier考虑了热点数据,tier层容量计入总容量,cache层容量不计入总容量。

\subsection{BUG}

BUG: ssd cache假定io 512对齐。如果有非512对齐的IO,会产生core。

\section{DR}

同步RC,异步RC。先实现同步RC,同步RC是异步RC的极限情况。

传输层采用标准协议,如libiscsi。

io journalling

snapshot \& consistency group

% \section{CDP}

\section{Multipath}

\begin{enumbox}
\item 客户端的行为,还是服务端的行为?
\item 什么样的网络拓扑支持什么样的多路径策略?
\item linux支持多路径模块,如何配置?
\end{enumbox}

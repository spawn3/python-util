\chapter{可靠性}

重建延迟时间

如何检测磁盘故障

如何标记一块盘?

受控的磁盘分组

\section{场景}

追踪如下典型场景:
\begin{enumbox}
\item 添加盘
\item 拔盘后插入
\item 删除盘,删除pool,重用该盘
\end{enumbox}

磁盘的状态通过状态机来维护。每种状态下,其关联的资源项的明确规定。

一个磁盘处在out/in状态的标志是什么?

如何快速清除一个pool?

\section{磁盘故障}

拔盘,引起IO错误,或检测线程检测出磁盘不可写状态,此时,把磁盘的内存状态设为offline,并删除对应软链接。
调度进行恢复操作。待恢复完成后,disk unload,才正式从集群中剔除。

\hl{写过程中发生EIO,则会导致lichd进程重启}。在退出之前,调用disklost过程,删除软链接。

如果没有删除完成,重启了lichd,则会修复软链接,加载该盘。

测试中,出现了无软链接,但有别的相关磁盘文件的情况,磁盘加载后进入offline状态,但因为数据库残存有该磁盘的相关记录,
故进入不了完全删除的状态。lich health显示有offline磁盘。

\hl{磁盘故障的处理,同时也作为GC机制}。数据库的记录,不在chunk的当前位置列表中,即可判定为是垃圾数据。

原来掉盘的情况下,没有关闭磁盘对应的文件描述符,导致\hl{bcache的数据盘不能register}。所以,掉盘时需要关闭相关描述符。
再次上线的时候,重新open即可。

\subsection{磁盘元数据}

从状态一致性的角度进行分析。

一个磁盘的状态,依赖于其本身与关联元数据资源。约分为三级:
\begin{enumbox}
\item 物理磁盘+RAID
\item + BCACHE
\item LICH磁盘 (数据文件与内存状态)
\end{enumbox}

其他派生状态

LICH磁盘文件:
\begin{enumbox}
\item /opt/fusionstack/data/disk/disk/\%.disk (soft link)
\item /opt/fusionstack/data/disk/block/\%.block
\item /opt/fusionstack/data/disk/bitmap/\%.bitmap
\item /opt/fusionstack/data/disk/info/\%.info
\item /opt/fusionstack/data/disk/tier/\%.tier
\item /opt/fusionstack/data/disk/speed/\%.speed
\end{enumbox}

LICH内存状态:
\begin{enumbox}
\item offline
\item deleting
\end{enumbox}

\subsection{场景:添加磁盘}

如果bcache关系已存在,仅仅创建disk文件,触发lichd的加盘操作,这样有无问题?毕竟\hl{bcache的数据盘没有被重新格式化}。

\subsection{场景:删除符号链接然后加上}

为什么加上符号链接后,恢复速度反而变慢很多?

一块盘被拔出,经过一段时间后再插入,是否可以停止对应恢复进程?
该盘对应的数据块处在几种状态:最新、过期、或成为垃圾。

仅仅停止恢复进程是不够的,再次调度恢复(包含GC)?

\subsection{场景:拔出cache盘随后插入}

概率性地出现如下问题:cache盘无法上线,data盘无法上线。即register阶段失败。
重启服务器后恢复正常。

data盘无法上线的情况:如果一块盘恢复完成了,可以观察到register成功。说明什么?
lsblk依然能看到虚实数据盘之间的映射关系。

\subsection{场景:删盘}

主动删除盘后,需要恢复。但如果进一步删除了pool,则须退出pool内所有disk的恢复过程。

\subsection{管理线程}

\section{节点故障}

\section{工具}

\subsection{Hazard}

hazard的IO模式:对同一IO extent,先写入,再读取。如果两者不一致,则再次读取。
写和读分两次,每次的大小是随机的。但两次加起来代表一个完整的IO extent。
只有写入与第一次读取不一致时,才有第二次读取进一步验证。故结果分为几种情况:
\begin{enumbox}
\item w != r1, w == r2,为读错误
\item w != r1, r1 == r2,为写错误
\item w != r1, w != r2,为unknown
\end{enumbox}

绘成三角形,如w为顶点,r1、r2为两底点,能更形象地去表示三者的关系。

\hl{定位对应的chunk},区分两种情况:raw和fs。raw的情况容易定位,\hl{(LBA-1) * 512}即可计算出对应的chunk,一个LBA对应512B。
fs需要借助辅助手段来定位。模拟一io,通过底层FusionStor的log来定位。

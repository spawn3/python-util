\chapter{可靠性}

\section{磁盘故障}

拔盘,引起IO错误,或检测线程检测出磁盘不可写状态,此时,把磁盘的内存状态设为offline,并删除对应软链接。
调度进行恢复操作。待恢复完成后,disk unload,才正式从集群中剔除。

写过程中发生EIO,则会导致lichd进程重启。在退出之前,调用disklost过程,删除软链接。

如果没有删除完成,重启了lichd,则会修复软链接,加载该盘。

测试中,出现了无软链接,但有别的相关磁盘文件的情况,磁盘加载后进入offline状态,但因为数据库残存有该磁盘的相关记录,
故进入不了完全删除的状态。lich health显示有offline磁盘。

\subsection{磁盘元数据}

磁盘文件:
\begin{enumbox}
\item /opt/fusionstack/data/disk/disk/\%.disk (soft link)
\item /opt/fusionstack/data/disk/block/\%.block
\item /opt/fusionstack/data/disk/bitmap/\%.bitmap
\item /opt/fusionstack/data/disk/info/\%.info
\item /opt/fusionstack/data/disk/tier/\%.tier
\item /opt/fusionstack/data/disk/speed/\%.speed
\end{enumbox}

内存状态:
\begin{enumbox}
\item offline
\item deleting
\end{enumbox}

\subsection{管理线程}

\section{节点故障}

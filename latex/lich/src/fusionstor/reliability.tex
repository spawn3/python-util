\chapter{可靠性}

重建延迟时间

如何检测磁盘故障

如何标记一块盘?

受控的磁盘分组

\section{大pool的可靠性}

数据模型

C/M/D三角架构的关系

Meta的映射

meta如何存取?如果meta分布到各个节点上,发生故障的概率增加。

对N副本系统,容错等级是N-1。N各副本散布在N个故障域,则发生故障的概率如何计算?
假设故障域发生故障的概率为p,则系统发生故障的概率为p\^N(概率乘法法则)。
\hl{故障域越多,概率越大,故障域内节点越多,概率越大}。

\hl{client能否直接存取osd?}
控制器粒度太大,引入卷控后,卷控相当于底层存储引擎的client,
iscsi必然需要转发,如果iscsi target与卷控在一个节点上,只需转发一次。

只要不share,存取数据项就不需加锁。\hl{是否需要加锁取决于数据是否共享},与控制器粒度无关。

负载均衡和热点:

与此形成对比的是,ceph引入pg,控制器的粒度放在了primary osd上,
ceph中任何一对象的副本大体分布在pg对应的osd上,是受控的。

采用EC机制可以改进可靠性,降低了性能。

\section{场景}

追踪如下典型场景:
\begin{enumbox}
\item 添加盘
\item 拔盘后插入
\item 删除盘,删除pool,重用该盘
\end{enumbox}

磁盘的状态通过状态机来维护。每种状态下,其关联的资源项的明确规定。

一个磁盘处在out/in状态的标志是什么?

如何快速清除一个pool?

\section{磁盘故障}

拔盘,引起IO错误,或检测线程检测出磁盘不可写状态,此时,把磁盘的内存状态设为offline,并删除对应软链接。
调度进行恢复操作。待恢复完成后,disk unload,才正式从集群中剔除。

\hl{写过程中发生EIO,则会导致lichd进程重启}。在退出之前,调用disklost过程,删除软链接。

如果没有删除完成,重启了lichd,则会修复软链接,加载该盘。

测试中,出现了无软链接,但有别的相关磁盘文件的情况,磁盘加载后进入offline状态,但因为数据库残存有该磁盘的相关记录,
故进入不了完全删除的状态。lich health显示有offline磁盘。

\hl{磁盘故障的处理,同时也作为GC机制}。数据库的记录,不在chunk的当前位置列表中,即可判定为是垃圾数据。

原来掉盘的情况下,没有关闭磁盘对应的文件描述符,导致\hl{bcache的数据盘不能register}。所以,掉盘时需要关闭相关描述符。
再次上线的时候,重新open即可。

\subsection{磁盘元数据}

从状态一致性的角度进行分析。

一个磁盘的状态,依赖于其本身与关联元数据资源。约分为三级:
\begin{enumbox}
\item 物理磁盘+RAID
\item + BCACHE
\item LICH磁盘 (数据文件与内存状态)
\end{enumbox}

其他派生状态

LICH磁盘文件:
\begin{enumbox}
\item /opt/fusionstack/data/disk/disk/\%.disk (soft link)
\item /opt/fusionstack/data/disk/block/\%.block
\item /opt/fusionstack/data/disk/bitmap/\%.bitmap
\item /opt/fusionstack/data/disk/info/\%.info
\item /opt/fusionstack/data/disk/tier/\%.tier
\item /opt/fusionstack/data/disk/speed/\%.speed
\end{enumbox}

LICH内存状态:
\begin{enumbox}
\item offline
\item deleting
\end{enumbox}

\subsection{场景:添加磁盘}

如果bcache关系已存在,仅仅创建disk文件,触发lichd的加盘操作,这样有无问题?毕竟\hl{bcache的数据盘没有被重新格式化}。

\subsection{场景:删除符号链接然后加上}

为什么加上符号链接后,恢复速度反而变慢很多?

一块盘被拔出,经过一段时间后再插入,是否可以停止对应恢复进程?
该盘对应的数据块处在几种状态:最新、过期、或成为垃圾。

仅仅停止恢复进程是不够的,再次调度恢复(包含GC)?

\subsection{场景:拔出cache盘随后插入}

概率性地出现如下问题:cache盘无法上线,data盘无法上线。即register阶段失败。
重启服务器后恢复正常。

data盘无法上线的情况:如果一块盘恢复完成了,可以观察到register成功。说明什么?
lsblk依然能看到虚实数据盘之间的映射关系。

\subsection{场景:删盘}

主动删除盘后,需要恢复。但如果进一步删除了pool,则须退出pool内所有disk的恢复过程。

\subsection{管理线程}

\section{节点故障}

每个pool对应一个常驻的恢复主线程。有定向唤醒机制。恢复分三节点:扫描控制器,扫描控制器对应chunk,恢复。
各阶段通过队列进行通信。

标记的是疑似,未必需要恢复,但计入了恢复速度,不妥。

没有按卷组织恢复,难以精确控制恢复的顺序。

异常处理:
\begin{enumbox}
\item 有新ctl加入本节点会如何?
\item 删除pool会如何?
\item 删除vol会如何?
\item 改变控制参数会如何?
\item 恢复过程中,遇到fail的chunk,如何处理?简单重试依然无法成功的情况呢?
\item 怎么才算本轮恢复完成了呢?
\item 什么条件下,标记rescan?
\end{enumbox}

\section{VFM}

在recovery过程中,标记卷的所有subvol,恢复完成后清除标记。

\hl{以意逆志} 在学习时,先站在一个更高的层次上,以道观之,去分析问题。
比如,这个特性要解决什么问题?如何解决的?会不会引入新问题?其它的做法是什么?
不解决这个问题会如何?

没有加入VFM机制,存在什么问题?

全面理解VFM,VFM破坏了强一致性原则,\hl{化同步恢复为异步恢复}。解决\hl{故障下IOPS严重下降}的问题。

统一拓扑结构:单故障域(节点级的故障域)和多故障域

两副本和三副本: 两副本最多标记一个故障域,三副本最多标记两个故障域?\hl{一致性如何}?

推演三节点两副本的情况。

\begin{enumbox}
\item 标记规则:不考虑故障域的情况,3副本只能标记两个节点,2副本最多标记一个节点;要考虑故障域,每个故障域最多只能标记一个节点
\item 何时标记、取消标记?卷控制器自己做标记,外部过程通过vctl利用该信息(感知到问题)
\item 标记什么内容?
\item 标记信息用来做什么?用于读写和恢复过程
\item *
\item 由etcd上节点连接信息判定节点状态(离线或在线)
\item 如某vctl感知到某节点离线,则标记所有subvol
\end{enumbox}

标记后,相关chunk不执行写时恢复逻辑,依赖于外部恢复过程。

如果一个subvol标记了某节点,表明该subvol的所有chunk,如有副本对应该节点,则执行特定的逻辑。
磁盘故障,标记的是节点,范围被放大。\hl{磁盘故障的need recovery是扫描数据库得到的,会放大吗}?

标记的节点,就是代表该副本不可用。在磁盘故障的情况下,相邻chunk不一定处在一个节点上。
比如\hl{同一subvol内两个chunk,可能位于同一节点的不同disk上}。

如何区分真正恢复的和仅仅检查的情况,都计入恢复速度吗?\hl{有两种速度,深度扫描速度和恢复速度}。

三阶段:扫描、深度扫描和恢复。

Tools
\begin{enumbox}
\item lich.inspect --vfmget vol
\end{enumbox}

\section{验证}

Test Case
\begin{enumbox}
\item 恢复编程available后,还需要再次lich health scan,不需第三次(节点的两副本、磁盘故障)
\item 推到一些极限情况,比如两副本,故障两个节点,会如何?
\item 性能影响:默认300的恢复速度情况下,故障对ssd pool的影响大于hdd pool(-10\%)
\end{enumbox}

\section{工具}

\subsection{Hazard}

hazard的IO模式:对同一IO extent,先写入,再读取。如果两者不一致,则再次读取。
写和读分两次,每次的大小是随机的。但两次加起来代表一个完整的IO extent。
只有写入与第一次读取不一致时,才有第二次读取进一步验证。故结果分为几种情况:
\begin{enumbox}
\item w != r1, w == r2,为读错误
\item w != r1, r1 == r2,为写错误
\item w != r1, w != r2,为unknown
\end{enumbox}

绘成三角形,如w为顶点,r1、r2为两底点,能更形象地去表示三者的关系。

\hl{定位对应的chunk},区分两种情况:raw和fs。raw的情况容易定位,\hl{(LBA-1) * 512}即可计算出对应的chunk,一个LBA对应512B。
fs需要借助辅助手段来定位。模拟一io,通过底层FusionStor的log来定位。

\chapter{设计}

设计在解决关键问题的同时,要降低实现,测试和维护的复杂度。

加强测试,通过重构降低复杂度。

分解问题,界定边界,降低复杂度

设计的基本原则

分离机制和策略,接口和实现

性能依赖于设计,在一定的设计下,取决于实现。

性能优化手段:并行,聚合,缓存等,根本在于设计,控制复杂度。

识别实体和关系,ERD,DFD等,FSM是机器语言。

核心概念:
\begin{compactenum}
\item 存储池/目录
\item 卷
\item 快照
\item 主机映射
\end{compactenum}

core thread边界,core\_request进入。

分布式副本一致性:clock版本机制,msgqueue离线消息处理。

性能:并发,聚合和cache等

元数据管理,非计算而来

快照树的实现

后台任务统一管理,包括:
\begin{compactenum}
\item recovery
\item balance
\item vol rm
\item snap rm
\item snap rollback
\item snap flat
\end{compactenum}

架构问题:
\begin{compactenum}
\item 元数据管理成本
\item 支持大容量卷
\item 支持ROW快照树
\item 诊断流程和工具
\item 性能profile
\end{compactenum}

\section{故障域}

对任一存储池,设故障域数为M,副本数为N,

当M>=N时,每个故障域内一个副本,随机分布;
当M<N时,
- 策略1,每个故障域内一个副本
- 策略2a,剩余的副本按策略1进行,直到写完所有副本数
- 策略2b,不写剩余副本

按策略2a:

case 1:故障域为2,副本数为3,则副本在故障域的分布为(2,1)或(1,2)
case 2:故障域为1,副本数为3,则副本在故障域的分布为 3

按策略2b:

case 1:故障域为2,副本数为3,则副本在故障域的分布为(1,1)
case 2:故障域为1,副本数为3,则副本在故障域的分布为(1) (副本数不能少于2个,分配失败)

同时,恢复过程须按以上故障域规则进行自动校正!!!

\section{存储池状态}

\begin{compactenum}
\item 不可用
\item 磁盘空间不足/READ ONLY
\item 降级
\item 正常/健康
\end{compactenum}

\section{诊断方法}

对需要改进的流程进行区分,找到最有潜力的改进机会,优先对需要改进的流程实施改进。如果不确定优先次序,企业多方面出手,就可能分散精力,
影响6σ管理的实施效果。业务流程改进遵循五步循环改进法,即DMAIC模式:

\begin{compactenum}
\item 定义[Define]——辨认需改进的产品或过程,确定项目所需的资源。
\item 测量[Measure]——定义缺陷,收集此产品或过程的表现作底线,建立改进目标。
\item 分析[Analyze]——分析在测量阶段所收集的数据,以确定一组按重要程度排列的影响质量的变量。
\item 改进[Improve]——优化解决方案,并确认该方案能够满足或超过项目质量改进目标。
\item 控制[Control]——确保过程改进一旦完成能继续保持下去,而不会返回到先前的状态。
\end{compactenum}

信息有多级:USE。诊断问题依赖于结构化的诊断方法PAT,解决问题也是,构建知识图谱。

欲分析问题,必分析事物发展的完整过程,包括每个参与者的生命周期模型,参与者之间的相互作用。


\chapter{RDMA}

注意事件循环,消息派遣机制,同步操作。
注意内存使用。

C/S,S端收到RDMA\_READ请求时,push数据到客户端的内存。
S端收到RDMA\_WRITE请求时,从客户端pull数据到本地。

以stor\_rpc\_read和stor\_rpc\_write为例,理解RDMA通信机制。
post、commit和poll分别对应wr队列和wc队列。wr队列又分为发送和接收队列。

rpc构建于RDMA verbs之上,是同步机制。rpc客户端通过post send发送协议参数,
分为读、写两种情况。\hl{RDMA read和write都是由RPC server-side完成的}。

ibv\_post\_recv若干内存区块,以接收ibv\_post\_send消息。处理完一个,再次调用。
就好比有若干空槽,来一个send msg,填充一个,处理完后,reset进入可用状态。
这是不同于tcp send/recv的地方,\hl{send/recv基于字节流,post send/recv基于msg}。
send/recv都不涉及rkey,不需要知道peer的内存地址。

rdma write相当于push本地内存数据到远端应用内存,rdma read相当于从远端应用内存pull数据到本地内存。

通过send/recv交换双方内存信息,或其它方式进行交换。\hl{在接收端,要先准备post recv}。

首先要掌握连接管理,采用c/s架构。

为什么理解epoll机制不够顺利?原因在于这是一个双层结构的东东。理解RDMA面临同样问题。
rdma\_cm\_id约等于一个tcp连接,也需要区分listen socket和connection socket,与客户端共同构成三角形结构。

\chapter{磁盘管理}

作为一个独立的模块,应该导出什么API呢?

disk状态机怎么画?

\section{持久化数据}

磁盘文件
\begin{enumbox}
\item disk
\item bitmap
\item ***
\item block
\item info
\item speed
\item tier
\item rotation
\item ***
\item /dev/shm/lich4/nodectl/diskstat (删盘完成后,才会清除与disk相关的各种文件)
\end{enumbox}

以上文件,以disk文件和bitmap最重要。disk文件指向实际的物理设备,bitmap用于磁盘空间管理。
其它的可以没有,基本信息记录disk头即可,用checksum来检查一致性。

可以把磁盘头备份出来

disk idx记录在disk head元数据区。disk头部包括哪些元数据?

每个节点维护有虚拟磁盘slot,pool通过disk idx引用这些disk slot,形成两级tree结构。

pool的identity是name,disk的identity是idx。\hl{pool有disk\_array成员,与disk slot同构}。即大小相同,位置相同。

\section{Local Storage Engine}

后台存储引擎支持:sqlite3, redis。

\subsection{Sqlite3 Mode}

\subsection{Redis Mode}

redis,采用若干redis-server实例,每个实例管理的数据格式:metadata, raw:vol.10.0, disk:<disk_id>。
disk用于磁盘空间管理,记录已用的chunk块偏移,在启动后构造bitmap,用于空间分配。

\section{Add Disk}

加一块盘,先选出disk idx,通过info、block、bitmap等磁盘文件推导而来。
lichd inotify方式监听到该事件后,处理加盘逻辑。由python脚本和lichd进程共同完成。

\section{Delete Disk}

\section{Recover Disk}

若删除pool,则可终止本过程。

\section{Load Disks}

前置条件检查

多线程执行

\section{malloc}

空间分配:每个pool有一个任务队列,每个disk有一个allocator,这些allocator从任务队列里面取任务。

维护bitmap用于free空间管理。

\section{free}

删除pool、删除vol,回滚垃圾chunk,都会涉及到磁盘空间回收。
且不同任务可能并发执行,多次回收一个disk loc。可重入性,终态一致。

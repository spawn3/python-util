\chapter{情境视图}

\begin{enumbox}
\item 企业存储架构SAN的升级
\item 与云平台对接HCI
\item OLAP
\item OLTP
\item Oracle RAC
\item SAP
\item 统一存储
\item 大数据
\end{enumbox}

往上看,分层架构:提供target(iscsi, NVMf,vhost等)、命令行工具、API等接口。
目前对接的云平台有zstack、openstack(通过cinder驱动)。

目前较稳定的是iSCSI接口,支持VIP、VAAI等。iSCSI target以及所有客户端连接controller层,执行目录、卷、快照相关操作。
对象层没有独立出来(ceph是在对象层之上构建卷,形成\hl{统一存储架构})。

每个控制器管理一个chunk树分支,一个pool的所有数据和元数据构成一颗chunk树,多个pool构成chunk树的森林。
根节点记录在etcd上,中间节点是元数据,叶子节点是卷或快照卷的raw数据。

iSCSI accept客户端链接后,即映射到相应的core的调度器队列。后续该连接上的所有请求响应,都由该调度器进行调度。

多路径?

链路的冗余设计?

检测scheduler是否有block调用,包括syscall等。

\hl{增加RDMA支持}:RDMA链接,不同于TCP链接。RDMA连接既存在于init到target(iSER)之间,也存在于集群的各个节点之间。

为什么要引入RDMA? network层的kernel bypass。iscsi需要强化为iSER,corenet同时支持tcp和rdma。
RDMA在内存使用上也有所不同,都由core scheduler调度。

\hl{增加NVMe支持}:在disk层面(SPDK),在网络层面(NVMf)

不同节点上具有相同hash的core构成corenet,corenet中的节点之间两两建立连接(一条连接)。
如果两个节点彼此并发地建立连接,如何保证有且只有一条连接?

上层采用nid作为节点标示,通过nid查询net table,得到对应的连接,然后执行网络通信。
采用什么数据结构呢?array最为高效。

\hl{TCP之上的multi paxos}:分为两层,通信+协议。

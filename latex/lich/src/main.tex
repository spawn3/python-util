% -*- coding: UTF-8 -*-
% hello.tex

\documentclass[UTF8,oneside]{ctexbook}

% \usepackage{xeCJK}
\usepackage[utf8]{inputenc}

% load paralist before enumitem
\usepackage{paralist}

\usepackage{hyperref}
\hypersetup{pdftex,colorlinks=true,allcolors=blue}
\usepackage{hypcap}

\usepackage{color}
\usepackage[usenames, dvipsnames, svgnames, table]{xcolor}
% \pagecolor{gray}

\usepackage{makeidx}
\makeindex

\usepackage{amsmath}
\usepackage{mathtools}

\usepackage{listings}
\usepackage{multicol}
\usepackage{fancybox}
\usepackage{tcolorbox}
\usepackage{enumitem}

\usepackage{indentfirst}

\newenvironment{enumbox}[0]{
    \begin{tcolorbox}
    \begin{compactenum}
} {
    \end{compactenum}
    \end{tcolorbox}
}

\newenvironment{itembox}[0]{
    \begin{tcolorbox}
    \begin{compactitem}
} {
    \end{compactitem}
    \end{tcolorbox}
}

% table
\setlength{\arrayrulewidth}{1pt}
\setlength{\tabcolsep}{16pt}
\renewcommand{\arraystretch}{2.5}
\newcolumntype{s}{>{\columncolor[HTML]{AAACED}} p{3cm}}

\arrayrulecolor[HTML]{DB5800}

\usepackage{tikz,mathpazo}
\usetikzlibrary{positioning, fit, matrix, shapes, arrows, chains, trees, arrows.meta}

% \bibliographystyle{plain}
% \bibliography{math}

\tikzset{%
  >={Latex[width=2mm,length=2mm]},
  % Specifications for style of nodes:
            base/.style = {rectangle, rounded corners, draw=black,
                           minimum width=4cm, minimum height=1cm,
                           text centered, font=\sffamily},
  activityStarts/.style = {base, fill=blue!30},
       startstop/.style = {base, fill=red!30},
    activityRuns/.style = {base, fill=green!30},
         process/.style = {base, minimum width=2.5cm, fill=orange!15,
                           font=\ttfamily},
}

% 摘录
\usepackage{verbatim}
\usepackage{libertine}
\usepackage{graphicx}
\usepackage{framed}

\newcommand*\openquote{\makebox(25,-22){\scalebox{5}{``}}}
\newcommand*\closequote{\makebox(25,-22){\scalebox{5}{''}}}
\colorlet{shadecolor}{Azure}

\makeatletter
\newif\if@right
\def\shadequote{\@righttrue\shadequote@i}
\def\shadequote@i{\begin{snugshade}\begin{quote}\openquote}
\def\endshadequote{%
\if@right\hfill\fi\closequote\end{quote}\end{snugshade}}
\@namedef{shadequote*}{\@rightfalse\shadequote@i}
\@namedef{endshadequote*}{\endshadequote}
\makeatother

\usepackage[normalem]{ulem}

\newcommand{\hl}{\bgroup\markoverwith
  {\textcolor{yellow}{\rule[-.5ex]{2pt}{2.5ex}}}\ULon}

%\usepackage{soul}

%\newcommand{\hlc}[2][yellow]{{%
%    \colorlet{foo}{#1}%
%    \sethlcolor{foo}\hl{#2}}%
%}

% todonode
\usepackage{lipsum}                     % Dummytext
\usepackage{xargs}                      % Use more than one optional parameter in a new commands
% 
\usepackage[colorinlistoftodos,prependcaption,textsize=tiny]{todonotes}
\newcommandx{\unsure}[2][1=]{\todo[linecolor=red,backgroundcolor=red!25,bordercolor=red,#1]{#2}}
\newcommandx{\change}[2][1=]{\todo[linecolor=blue,backgroundcolor=blue!25,bordercolor=blue,#1]{#2}}
\newcommandx{\info}[2][1=]{\todo[linecolor=OliveGreen,backgroundcolor=OliveGreen!25,bordercolor=OliveGreen,#1]{#2}}
\newcommandx{\improvement}[2][1=]{\todo[linecolor=Plum,backgroundcolor=Plum!25,bordercolor=Plum,#1]{#2}}
\newcommandx{\thiswillnotshow}[2][1=]{\todo[disable,#1]{#2}}
%

\usepackage[simplified]{pgf-umlcd}

\title{LICH架构文档}
\author{董冠军}
\date{\today}

\begin{document}

\maketitle
\tableofcontents

\listoftodos[Notes]

\part{项目管理}

\chapter{移动集采}

产品评估:功能和质量。质量包括:可靠性,性能,QoS,可扩展性(负载均衡),用户体验(管控,交互,接口)等方面。

性能不是一个值,而是不同场景下的特征曲线。性能是系统配置和负载的函数:$P=F(S, W)$。
精简配置,快照,故障,实现机制等因素都会影响性能及其抖动。

\section{重要问题}

\subsection{存储池}

模型:概念及其关系。

重建状态和速度按存储池进行统计。

\subsection{卷}

\begin{enumbox}
    \item 精简配置
    \item 三副本
    \item 扩容
    \item 卷的QoS:IOPS
    \item 全量拷贝和增量拷贝
\end{enumbox}

三副本 \info{三副本性能}

\subsection{快照}

\begin{enumbox}
    \item 显示卷和快照的创建时间,名称,宿主,\hl{容量,存储池}。
    \item 快照树是必须的
    \item 快照验证:每个快照关联一个数据集。
\end{enumbox}

\subsection{映射}

数据访问和隔离机制,应按\hl{最小权限原则设计}。主机仅能访问映射到该主机的卷

\subsection{故障处理}

\begin{enumbox}
    \item IO抖动
    \item 重建
\end{enumbox}

\subsection{数据恢复}

触发策略,QoS,性能,状态。

\begin{enumbox}
    \item 在线调整策略:应用优先,或恢复优先。
    \item 负载调整到原来的20\%时,数据重建效率。
    \item 显示恢复速度和状态,区分实际重构的,和跳过的
    \item 拔出硬盘的存储池降级(数据冗余度发生下降,目前恢复进程是节点级的,与存储池没有直接关系,并且,
        在节点的维度上,存储池是有覆盖的,overlay network,\hl{按卷进行汇总})
    \item 磁盘漫游,存储系统不发生重建,且数据无异常
    \item 模拟故障:单磁盘,单节点,单机架。要求:无读写中断。
\end{enumbox}

\subsection{负载均衡}

均衡有数据均衡和负载均衡之分。

单卷的负载均衡,因为FusionStor卷控制器绑定到core上,只能利用单核能力。
CPU利用率上有瓶颈。同时,开启polling模式,会独占core cpu。

\subsection{Misc}

VAAI

\section{不需要做的项}

\begin{enumbox}
    \item 一致性组
    \item 同步/异步远程复制
    \item EC
\end{enumbox}

\chapter{任务清单}

%\pagestyle{empty}
\todo[inline]{The original todo note withouth changed colours.\newline Here's another line.}
\lipsum[11]\unsure{Is this correct?}\unsure{I'm unsure about also!}
\lipsum[11]\change{Change this!}
\lipsum[11]\info{This can help me in chapter seven!}
\lipsum[11]\improvement{This really needs to be improved!\newline\newline What was I thinking?!}
\lipsum[11]
\thiswillnotshow{This is hidden since option `disable' is chosen!}
\improvement[inline]{The following section needs to be rewritten!}
\lipsum[11]
%\newpage

\section{原生卷}

\begin{tcolorbox}
\begin{compactenum}
    \item 存储池
    \item Mapping
    \item 数据恢复性能
    \item Redis Cache
    \item 快照树
    \item 精简配置,快照对性能的影响
    \item 故障下的性能及其抖动
    \item \hl{async sqlite} [DONE]
    \item \hl{batch sqlite}
    \item Allocate的性能
    \item 单卷快照的数量
    \item VAAI
    \item SSD Cache
    \item 异步远程复制
    \item FC (+VAAI)
\end{compactenum}
\end{tcolorbox}

\section{关键特性和过程}

\begin{compactenum}
    \item flush, load and recovery
    \item 保护模式 safe mode
    \item 存储分层
    \item 命令行工具,扩展卷和快照相关操作到LSV
\end{compactenum}

\section{兼容性}

\begin{compactenum}
    \item 版本演进
\end{compactenum}

\section{Pool}

\begin{compactenum}
    \item Resource Pool
\end{compactenum}

\section{Volume}

\begin{compactenum}
    \item new format: row2
    \item new format: lsv
    \item vol max size 256T+
    \item vol resize?
    \item all zero's chunk
\end{compactenum}

\section{快照}

\begin{compactenum}
    \item 支持60000+快照
    \item consistency group
    \item 每个snap的大小等信息
    \item snap大小对GC策略的影响
\end{compactenum}

\section{一致性/正确性}

\begin{tcolorbox}
\begin{compactenum}
    \item 底层数据检验工具(chunk0, volume, log/gc, bitmap, wal, rcache)
    \item 内置质量,各模块添加自校验机制,方便诊断数据正确性问题(assert + log + test)
    \item 加强断言:pre和post条件,变量变化规则,不变式,基本假设等
    \item 日志用tag/keywork和timeline,以便于跟踪一个对象的变化历史,用一个或多个维度贯穿起来,用于辅助诊断
    \item 增加CHUNK\_HISTORY,以时间线方式,跟踪记录CHUNK变化的生命周期
\end{compactenum}
\end{tcolorbox}

\section{性能}

性能是负载和资源的函数, $P=F(W, R)$。

\begin{tcolorbox}
\begin{compactenum}
    \item 创建卷时,rcache分配了4096M的SSD cache,可以延迟分配
    \item \textcolor{red}{rcache 顺序IO随机化问题}
    \item wbuf 顺序IO随机化问题
    \item 系统启动时间
    \item 预填充lich chunk
    \item GC策略和算法
    \item 统计基础操作的开销,作为性能分析的基础
\end{compactenum}
\end{tcolorbox}

\section{负载}

\begin{tcolorbox}
\begin{compactenum}
    \item IO队列深度
    \item IO平均大小
    \item IO读写大小
\end{compactenum}
\end{tcolorbox}

\section{资源}

\begin{tcolorbox}
\begin{compactitem}
    \item 内存使用量过大
    \item 内存泄漏
    \item 磁盘利用率不足
    \item 网络带宽:瓶颈或利用率不足
    \item 中断
    \item soft lock up?
\end{compactitem}
\end{tcolorbox}

\section{故障处理}

\begin{compactenum}
    \item \change{故障域,不能中断IO}
    \item 节点间负载均衡(<20\%)
\end{compactenum}

\section{Misc}

\begin{tcolorbox}
\begin{compactitem}
    \item FC
    \item Remote copy
    \item SSD cache
    \item EC
    \item 有效容量的比例
    \item 热插拔
    \item 磁盘漫游
    \item 在线扩容
    \item 滚动升级
\end{compactitem}
\end{tcolorbox}

\section{DONE}

\begin{compactenum}
    \item VAAI [+xcopy]
\end{compactenum}

\chapter{问题集}

\section{空间分配和元数据管理}

目前的空间分配和元数据管理,在支持关键特性和过程上,不够高效。

sqlite层采用(chkid, parent, diskid, offset)指定chkid到磁盘空间的映射。
chkid包含了卷ID,可以利用该信息加快卷的回收过程,但不方便reuse该数据空间。

目前的元数据管理分两层:chunk和replica。
分配一个chunk需要多次IO:
\begin{compactenum}
\item disk bitmap
\item sqlite
\item table2/table1 meta
\end{compactenum}

discard操作与allocate操作顺序相反。

关键过程:
\begin{compactenum}
\item chunk allocate
\item chunk discard (+ reuse)
\end{compactenum}

特性:
\begin{compactenum}
\item replication
\item EC
\end{compactenum}

快照实现:
\begin{compactenum}
\item COW
\item ROW2
\item LSV
\end{compactenum}


\part{测试}

\chapter{检查清单}

\section{检查清单checklist}

先宏观,后微观,致广大而尽精微

\begin{compactenum}
\item 集群健康情况
\item 硬件
    \begin{compactenum}
    \item 磁盘
    \item 网络
    \item 内存
    \item CPU
    \item 操作系统
    \end{compactenum}
\item 服务
    \begin{compactenum}
    \item 后台任务,包括恢复,删除,快照后台任务等
    \item 日志
    \item core
    \end{compactenum}
\item 数据一致性检查
\end{compactenum}

\chapter{测试}

\section{指标}

性能指标
\begin{enumbox}
\item 性能
\item 故障下的性能
\item 快照下的性能
\item ***
\item IO中断时长
\item 空间利用率
\end{enumbox}

\section{方法}

跟踪每个版本的性能,每个版本都有记录

先验证TCP,测试iscsi、NVMf,导出卷测试。

\section{工具}

perf方式

驱动

网络

RDMA

\chapter{性能}

\section{性能估算}

\begin{enumbox}
\item 单卷性能
\item 网络
\item 磁盘
\item 副本数
\end{enumbox}


\part{FusionStor}

纵横交织,用视图和视角矩阵重新组织本部分的内容,软件架构、硬件架构、部署都作为特定视图来处理。

每个视图有适用的受众,或利益干系人。视点集是创建视图的结构化模板库,每个视点对应有若干关注点。
视角是看待任意系统的特定维度,主要与质量属性相关,在视图内应用视角来深化质量属性。

根据视点创建视图,根据视角优化视图模型。

\chapter{情境视图}

\begin{enumbox}
\item 企业存储架构SAN的升级
\item 与云平台对接HCI
\item OLAP
\item OLTP
\item Oracle RAC
\item SAP
\item 统一存储
\item 大数据
\end{enumbox}

\chapter{功能视图}

\begin{tikzpicture}[show background grid]
    \begin{class}{Disk}{6, 0}
    \end{class}
    \begin{class}{Pool}{6, 2}
    \end{class}
    \begin{class}{Volume}{6, 4}
    \end{class}
    \begin{class}{Host}{6, 6}
    \end{class}
    \begin{class}{Cluster}{0, 2}
    \end{class}
    \begin{class}{Snapshot}{0, 4}
    \end{class}

    \composition{Cluster}{pools}{1..*}{Pool}
    \composition{Pool}{disks}{1..*}{Disk}
    \composition{Pool}{volumes}{1..*}{Volume}
    \composition{Volume}{mapping}{*..*}{Host}
    \composition{Volume}{snapshots}{1..*}{Snapshot}
\end{tikzpicture}

四面八方,对上导出iSCSI等服务,对下整个各种物理资源,左右对接运维平台和云平台。

整体,集群按两个维度来理解,一是物理资源,二是逻辑资源。
物理资源包括节点以及节点上的磁盘,逻辑资料包括存储池、目录、卷、快照等。

\section{Pool}

存储池是集群的物理划分。把物理节点划分为不同的保护域,一个卷的所有数据只出现在一个保护域内。卷可以跨保护域进行复制和迁移。

默认一个,包括所有节点(显式创建,不默认)。

\hl{与文件系统的类比}:存储池相当于磁盘分区,构成一个树,mount到根分区,多个pool构成chunk对象的森林。
pool包括目录和卷,相当于文件系统包括目录与文件。卷和文件都是逻辑实体,需要跟踪其包含的数据块的详细信息。

从文件系统出发去理解存储系统是一条可行途径。

文件系统:先有分区或磁盘,格式化为文件系统,然后mount到系统的目录树之下。
所谓格式化成文件系统就是在磁盘上建立文件系统的数据结构,包括superblock、索引结构、日志等。

\hl{嵌套、分形、递归}。

% 保护域是物理节点的划分,存储池是存储介质的划分。每块盘只能出现在一个存储池里。

Pool: 逻辑容器

故障域有粒度之分,如磁盘,节点,机架,机柜,数据中心。

存储池内,要满足故障域规则:一个chunk的不同副本,分布在不同的故障域内。\label{rule:faultset}

在所有的\hl{数据分布(初次分配、再平衡和恢复等)}过程中,都需遵循这些规则。

\begin{tcolorbox}

可以参考ceph的CRUSH实现。bucket和device定义了集群的物理拓扑结构,rule定义了数据存取规则,
pool上关联rule,从而定义了pool中卷数据的放置规则。设备即OSD,对应一个物理磁盘。

***

存储池可以取代保护域,定义所有对象的存放位置是一个节点集。

***

存储池可以用来实现tier cache。重定向IO到cache pool。

***

统一概念:保护域,存储池,pool。Consistency Group不同于pool,与物理存取无关,
而是卷的逻辑集合,卷可以来自不同的pool。

\end{tcolorbox}

与存储池有什么同和异?存储池可以看做关联了磁盘的pool,可以看做pool的子类。

% 存储池是disk的集合,与节点无关。但disk所在的节点构成存储池的节点列表,不同存储池的节点可能覆盖。

存储池下,可以创建volume。没有关联磁盘的存储池,不能创建卷。

\hl{chunkid到磁盘物理位置有两级映射:chunk的副本节点列表,节点内chunkid到物理地址的映射}。

在为卷分配chunk的时候,需要确定各个副本的物理存储位置。当前实现是返回不同副本的节点列表。
如果指定了存储池,就需要在存储池所在的节点范围内进行分配,同时要满足故障域和数据均衡规则。

\begin{tcolorbox}
移动采集中存储池要求,相比于目前的逻辑pool,更多是一种设计上的退步。
存储虚拟化的目标,是物理位置无关。我们可以基于逻辑容器,实现基于策略的管理。
所以,\hl{从实现层面,要保留当前pool的功能,按照系统配置确定pool的类型}。
但是,存储池关联磁盘,可以实现高速池等按物理特征划分的功能。
\end{tcolorbox}

% 存储池内,要满足故障域规则(\ref{rule:faultset})

步骤:
\begin{enumbox}
\item 创建pool,但此时不能创建pool的元数据chunk,因为还没有绑定磁盘(需要一全局的地方,存储pool名字)
\item 添加磁盘到pool,并上报给admin节点
\item 当在pool下面创建卷的时候,前提条件是已准备好磁盘,生成pool元数据chunk(位于同一pool里)。
\end{enumbox}

pool元数据chunk必须位于自己的存储池内,如果分布在不同的存储池,不满足故障域条件。

pool引导信息可以存在rootable里(用etcd取而代之)。

原来的/的元数据chunk放置在哪儿?还是不再需要,每个pool有独立的树,多个pool构成森林。

规则:存储池包括其下的所有卷和快照的元数据和数据,都必须存在该存储池关联的磁盘上。
包括原来/system下的内容。

\subsection{lookup}

\begin{compactitem}
\item 从path到id 拆分path,逐级查找(正向)
\item 从id到path 遍历pool,找到即退出(反向)
\end{compactitem}

\section{Volume}

属性:

操作:
\begin{compactenum}
\item rename
\item resize \info{在线扩容}
\item mv
\item copy \change{全量拷贝/增量拷贝} \change{跨存储池拷贝} % change不能出现在box里
\end{compactenum}

\section{Snapshot}

snapshot隶属于卷,无卷则无快照,快照组织成快照树,其中有且只有一个快照是可写快照,即卷的写入点。

\section{Consistency Group}

一致性卷组 \change{Consistency Group}

\begin{shadequote}
Consistency Groups could be useful for Data Protection (snapshots, backups) and
Remote Replication (Mirroring).

The Mirroring support will allow to setup mirroring of multiple volumes in the
same consistency group (i.e. attaching multiple RBD images to the same journal
to ensure consistent replay).

There is already an interest to implement this functionality as a part Mirroring feature:
http://tracker.ceph.com/issues/13295

The snapshot support will allow snapshots of multiple volumes in the same
consistency group to be taken at the same point-in-time to ensure data
consistency.
\end{shadequote}

\section{Mapping}

数据隔离/ACL,数据保护

卷对主机的可见性。一个卷只有映射给了某主机,才可以被该主机访问。

\section{属性}

\begin{enumbox}
\item 副本数
\item 故障级
\item 精简池
\item 磁盘列表
\item 复制类型(复制,EC)
\item 配额
\item \hl{有足够的故障域,且不同故障域配置一致的资源量}
\end{enumbox}

相关类:disk、volume

删除操作时,影响到相关类。

\section{操作}

\begin{enumbox}
\item pool create
\item pool rm
\item pool info or stat
\item pool list all
\item pool add disk
\item pool remove disk
\item pool list disk
\item 扩展(添加磁盘到\hl{已存在的存储池},该映射关系持久化到本地,同步到admin节点)
\item 缩容(从存储池中移除磁盘,引发数据重建过程)
\item 不同存储池之间,卷的复制
\item 不同存储池之间,卷的迁移,可在线或离线
\item 存储池级别的统计信息
\item \hl{自动或手动按磁盘速率进行存储池分级划分}
\end{enumbox}

\section{迁移}

离线/在线迁移

不改变卷ID。卷ID和chunkid集群内唯一,迁移过程中保持不变。

怎么判断一个卷是离线还是在线?有无访问者。target上的链接数,
每个卷的volume\_proto都有一个connect\_list。

\begin{lstlisting}[language=bash,frame=single]
# lich.inspect --connection /iscsi/p1/v1
\end{lstlisting}

基于快照实现存储池的迁移和复制。相当于clone了新卷,chunkid皆发生变化。

\section{复制}

卷可以建模为状态机,不同状态允许不同的操作,按条件在不同状态之间切换。

\section{属性}

\begin{lstlisting}[language=c,frame=single]
typedef struct {
        fileid_t id;
        uint16_t magic;
        uint16_t repnum;
        uint64_t snap_rollback;
        uint64_t snap_version;
        uint64_t reference;         //clone reference
        uint32_t attr;
        int32_t  priority;
        uint32_t __pad__[4];

        uint64_t size;
        uint32_t mode;
        uint32_t uid;
        uint32_t gid;
        uint32_t ctime;
        uint32_t mtime;
        uint32_t btime;
        uint32_t atime;
} fileinfo_t;
\end{lstlisting}

卷是ServerSAN核心对象,是pool,snapshot,mapping和cg的中心。

卷的属性记录在L1 chunk的fileinfo段。fileinfo段是vol info区的第一个段。

\subsection{大小}

目前,卷的元数据由两级组成:L1,L2。L1只有一个chunk,每个chunk有8000个槽位。每个槽位指向一个L2 chunk。
L2 chunk有16000个槽位,每个槽位指向一个raw chunk。所以,最大卷大小约为$122TB = 8000 \times 16000 \times 1MB$。

为了支持更大的卷,需要扩展此结构,把L1扩展到多个chunk。改变会影响到:
\begin{compactitem}
\item 加载table1 (加载多个chunk,记录每个chunk的chunkinfo信息)
\item table1的各项操作
\item 遍历卷的chunk
\item 数据恢复过程
\item migrate
\item copy
\item snapshot clone,需要copy L1 chunk
\item ...
\end{compactitem}

L1 chunk:vol.xxx.0, vol.xxx.1, vol.xxx.2。其中,vol.xxx.0的chunkinfo记录在pool。
\hl{vol.xxx.1等chunkinfo记录到vol.xxx.0里info区的第二个块,并需记录其个数}。动态加载。

父:raw和subvol的父chkid为vol.xxx.0,vol.xxx.1的父chkid也为vol.xxx.0。vol.xxx.0的父chkid为所在pool的chkid。

缺页中断:从上到下检查。创建chunk的时候,先检查L1 chunk是否存在,然后检查L2 chunk是否存在。
不存在,则创建。在以下过程会改变卷的数据结构树:
\begin{compactitem}
\item \verb|__pool_proto_mkvol|
\item \verb|volume_proto_load|
\item \verb|volume_proto_chunk_pre_write|
\item \verb|__volume_proto_chunk_allocate|
\end{compactitem}

\begin{lstlisting}[frame=single]
lichbd write /iscsi/p1/v1 hello -o 280375465082880
lichbd cat /iscsi/p1/v1 -o 280375465082880 -l 5
\end{lstlisting}

chunk allocate开销较大,即要申请磁盘空间,也涉及记录元数据到父节点。

遗留问题:
\begin{enumbox}
\item lich.inspect --stat极慢 (统计,load)
\item 分析内存占用量
\item clone
\item recovery
\item balance
\item 回收空间
\end{enumbox}

\subsection{副本数}

\section{操作}

\begin{enumbox}
\item create
\item rm
\item ls
\item info
\item resize
\item rename
\item mv
\item copy (read/write)
\item migrate (move all chunk)
\item duplicate (snapshot-based: clone/flatten)
\item import
\item export
\item mapping
\item \hl{IO}
\end{enumbox}

\subsection{create}

\subsection{rm}

显式回收空间

分为两阶段:标记和回收。先把卷移入/system/unlink,有后台异步线程负责回收。

相关源文件:
\begin{compactitem}
\item pool\_rmvol.c
\item rmvol\_bh.c
\end{compactitem}


\subsection{QoS}

token bucket。IOPS与block size相关,两者乘积等于带宽。
如果各层发生IO聚合,则在流量守恒的情况下,显示IOPS有所不同。

IOPS必须假定一定的block size。

抽象出AbstractVolume,Volume、Snapshot、Clone都是其子类。

\hl{卷或克隆卷扩容后,快照没有相应扩容,会返回ENOENT,需要填充0}。

\section{问题}

\begin{enumbox}
\item session方式,无commit
\item 随机IO,小cache,频繁swap,有问题
\item 如果标记dirty,会出现all dirty的情况
\item 如果有COW,需要pin住源
\end{enumbox}

\section{操作}

需要支持的快照操作:
\begin{enumbox}
\item create
\item rm
\item list
\item rollback
\item clone(跨卷read)
\item read
\item flatten(存储池迁移和复制,需要实现该接口)
\item protect/unprotect
\end{enumbox}

\begin{tabular}{|s|p{0.6cm}|p{0.6cm}|p{0.6cm}|p{0.6cm}| }
    \hline
    \rowcolor{lightgray} \multicolumn{5}{|c|} {Snapshot} \\
    \hline
    Feature & COW & ROW1 & ROW2 & LSV \\
    \hline
    snap tree & N &  & & \\
    \hline
    snap create & N &  & & \\
    \hline
    snap rm -> gc & N &  & & \\
    \hline
    snap rollback & N &  & & \\
    \hline
    snap clone -> read & N & & & \\
    \hline
    write & N &  & & \\
    \hline
    read & & N & & \\
    \hline
    space & N &  & & \\
    \hline
    consistency group & N &  & & \\
    \hline
\end{tabular}

采用COW,ROW或两者的组合形式,各有优缺点。

COW方式的快照,卷有完整的索引结构。别的快照点,只有增量的索引结构和发生更新的数据。
每个快照,存储的是创建之后,到下一个快照点之间发生更新的所有数据块。所以需要尽量降低发生copy的开销。
适用于频繁且具有局部性的热点负载场景,固定时间段内,每次copy的开销以及\textcolor{red}{复制集的大小}。

ROW方式下,如果meta不发生COW,新的快照点并无完整的索引结构,读过程需要沿着快照链向上回溯。
并且,rm,rollback等操作需要合并快照点。

如果发生了COW,索引项和数据项引用关系不再是1:1,而是多对多,需要专门的GC机制。
但rm,rollback等操作实现起来变得简单。

数据项的粒度,定长或变长,不同的负载,读写性能有不同的影响。
发生COW的粒度,索引结构管理的粒度决定了数据项的粒度。

IO粒度和数据项粒度的关系,如果IO粒度大于数据项粒度

如果IO粒度小于数据项粒度

这条规则是否永远成立:\textcolor{red}{快照树的任一路径,都需要具有完整的索引结构和数据项集,可以有冗余和共享}。

\section{COW}

为了支持快照树,需要分析rollback和读快照的过程。

\subsection{Create}

卷,clone卷,快照具有相同的索引结构,不同在于fileinfo的attr。

卷和快照的所有chunk,都是私有的,无法通过索引结构有效地定位chunk的位置。

创建快照是相当简单的过程,创建索引结构,设置正确的fileinfo,同时复制xattr。
卷的snap\_from指向其挂载的快照。

在写入卷的时候,发生COW过程,被修改的chunk先复制到卷的最近一个快照的私有存储空间里,
然后在卷上进行in place写。

\subsection{Rollback}

rollback,卷为源,以目标快照的内容覆盖源卷的内容。同时,为了不丢失信息,需要提前保存一定的数据。
需要保存什么数据?沿着快照树的根,到卷的当前快照(所在分支的最后一个快照),\hl{所有修改过的chunk}。

卷的当前快照:只有在卷的当前快照是叶子节点的时候,才需要保持数据。

读快照的规则:
\begin{compactenum}
\item 自身
\item 下游
\item 回溯至卷(先回溯至公共父节点,然后到卷)
\end{compactenum}

前两步是为了读取修改过的chunk,后一步是为了读取所在分支上从来没有修改过的chunk。
两者合起来,代表快照的数据。

通过fileinfo的snap\_from建立快照树,通过snap\_version为每个快照分配id,通过snap\_rollback指定要回滚的目标快照版本号。

rollback的状态变迁

rollback过程依赖于两个快照路径:
\begin{compactenum}
\item root snap list
\item cross snap list
\end{compactenum}


\section{ROW}

% ROW与LSV的不同,在于同一LBA,只对应一个数据项,而不是对应多个版本的数据项。
% 但这个数据项的引用计数,可以为0,1,或多个。

ROW,快照会产生一个数据项的多个版本。LSV,除了快照之外,一个LBA的覆盖写入也会产生多个版本。

卷和快照,共存于底层lich卷,是否有问题?

元数据的\hl{基本管理单元是page}。按page来组织,每个page unit:chunkid+pageid。
在ROW的过程中,可以只改变一页到新chunk。而bitmap则发生了COW,copy 1M,改变其中一项。
(用chunkid+ page bitmap的方式行不通)。\hl{有一种特殊情况,bitmap管理了连续1M的数据,此时chunkid是重复的}。
如无页级元数据,则需要发生1M的COW。有页级元数据,引入了优化的机会。

bitmap管理1M对应底层1M。管理所需元数据:chunkid+各页的元数据。

snap tree, snap, chunk, page构成四级结构。LSV时,没有chunk那一级(写入位置模式)。

写入模式:顺序写入,需要维护当前写入点。1M模式,需要1M:1M的映射, 而不再需要chunk内页索引。
引用关系是页级的,写入模式是chunk级的。降低了ROW的写入粒度。

vvol:clone时用,跨卷引用。支持多级clone。clone与snapshot基本一致,不同在于跨卷读取。

写入的过程:会检查每个isref,入为1,则表明该chunk归属于本快照,否则,需要创建新的。
对L2 bitmap言,发生1M的COW;对LOG言,申请分配1M,同时写入。

cache一致性问题:

session聚合写入:

CC: 页锁,bitmap单元锁,COW锁。\hl{细粒度的锁,无法保护更大粒度锁上的并发操作},如COW过程,ROW过程,
IO锁最好是区间锁。bitmap单元的提交操作。

页式管理的局限性:连续的有信息冗余。

空间管理:\hl{逻辑空间和物理空间的关系,物理空间的扩展性}。

\subsection{IO过程}

\begin{itembox}
\item CC: 范围锁,层次锁
\item 需要的实际IO次数
\item 页对齐
\item 跨chunk边界
\item 重入分析(数据更新,bitmap更新)
\end{itembox}

两个并发IO,如果落入一个1M,会访问到同一bitmap单元。
应先锁定bitmap单元,再锁定IO范围。遵循树状锁协议。
参考\hl{数据库索引结构上的锁协议}。

\begin{tcolorbox}

假设有新旧两个快照点S1, S2,每个快照点包括meta和data,提炼出的几个引导性问题:

\begin{enumerate}
    \item 数据块的粒度,page,chunk,extent?
    \item 每个快照点是否有完整的索引结构?
    \item 卷(写入点)是否有完整的索引结构?
    \item 哪一个是写入点?
    \item meta是否发生COW?
    \item data是否发生COW?
    \item 快照操作(create, rm, rollback, clone, flatten, ls, cat, protect/unprotect)的复杂度?
\end{enumerate}

\end{tcolorbox}

\section{ROW2}

\subsection{快照树}

快照树的每个节点,都是一个完整的索引结构。其中一个代表着当前写入点,写入点代表着卷。
回滚操作会改变当前写入点。

统一一下,卷和快照具有相同的索引结构。

每个快照结构有唯一的ID,是随着创建快照的过程递增的,卷快照/写入点具有最大的snap id。
LSV log结构记录了每页的snap id。

一颗树的节点,可以分为三类:
\begin{compactenum}
\item 根节点
\item 中间节点(有无分支)
\item 叶子节点
\end{compactenum}

\subsection{创建快照}

原有写入点变成只读,创建新的写入点,并复制L1元数据。

\subsection{删除快照}

快照树上不同的节点,需要不同的删除过程。对于叶子节点,直接删除即可,\textcolor{red}{需要回收数据吗}?

回收快照,涉及meta和data两个部分。

不应改变父快照的内容
保留子节点的共享内容

分两个节点:\hl{标记和回收}。

\subsection{列出快照树}

\subsection{写}

写是ROW的重定向过程,可能发生复制第二层元数据的过程。

如果连续写入,需记录当前的写入点。chunk内数据是逻辑不连续的,或者说,逻辑上连续的在chunk内是随机化了。

如果非连续写入,每页的写入位置计算得到,chunk内数据是逻辑连续的。大范围随机写入的情况下,需要分配很多chunk。
\textcolor{red}{需大力优化chunk的分配过程}。

pagetable是逻辑连续的,按逻辑空间组织。如果已分配位置,则覆盖写入。如果没有分配位置,如何分配?

逻辑地址怎么映射到物理地址?页式,段页式

\hl{写入位置,有三种分配策略:ROW2,ROW3,LSV},都是按页粒度管理物理卷空间。

\subsection{回滚}

回滚并不会重用回滚到的快照点,而是相当于把写入点嫁接到目标快照点。写入点本身是一个独立的快照结构。

除了写入点的所有快照点,都处于只读状态,没有任何操作可以改变其状态。

回滚后的写入过程

\subsection{读}

如果不复制元数据,ROW实现的读过程需要回溯快照树,性能不佳。
如果复制元数据,则每一快照点都具有完整的索引结构,可以做到一次即可定位。

复制元数据,快照和数据具有多对多的引用关系,相当于共享数据块。

读优化: 元数据可以一次定位,但可能碎片化,沿着快照链往上读取。可以通过flatten的过程优化。

\subsection{Clone}

卷和其上的快照位于同一个物理卷内,所以不涉及跨卷读。clone后,会用到跨卷读快照,
类似于ROW过程,所以,创建和clone过程具有相似性。

cat, protect, unprotect, flatten

\subsection{GC}

GC过程和引用计数。每个记录的is\_ref记录是否写入了数据。可以回收写入点,但无法回收一般快照。

LSV: 逐个扫描每个log。

\section{实现}

快照列表信息记录在卷table\_proto的smap和sinfo区,info区的每一项包括name和chkinfo,还包括snap\_attr\_t。
smap用bit标记该位置是否有效。卷上的所有快照,构成快照树。

每个快照有自己的snap\_version和snap\_from,分别表示自己的版本号和父节点的版本号。

新创建的快照,采用卷当时的版本号和父版本号,创建成功后,更新卷版本号和父版本号,rollback版本号与卷版本号一致。

revert的时候,rollback版本号表示目标快照的版本号,如有autosnap生成,采用卷当时的版本号和父版本号。
完成后,更新卷的版本号,父版本号和rollback版本号。

卷的每一chunk具有自身的版本号,revert成功后,所有chunk的版本都等于目标快照版本号。

\subsection{rollback}

主要通过4个变量来控制rollback过程:fileinfo(snap\_version)。

/pool1/system/rollback/vol.32.0

\section{Bitmap Cache}

要解决的关键问题有:
\begin{enumbox}
\item cache一致性
\item 并发与事务
\end{enumbox}

问题导向
\begin{enumbox}
\item 这是最好的设计/实现吗?
\item 需要维护什么序?哪些操作可以并发?
\item chunk的is\_ref与bunit的ref什么关系?
\item 怎么区分snapshot与clone两种情况?
\end{enumbox}

chunk的is\_ref表示引用别的快照的L2 bitmap chunk。
在创建snapshot时,设为1。发生COW后,设为0。

如果chunk的is\_ref == 1,则读出的bunit.ref也等于1。
(发生COW后)
如果chunk的is\_ref == 0,则读出的bunit.ref可能有两种值。

在COW后,chunk的is\_ref等于0,对应的所有bunit都等于1。随着卷上发生写操作,对应的bunit也变为0。
所以两者的含义都是\hl{所指向的对象是否属于卷本身},用于判断COW与数据写入过程是否需要分配数据块的条件,
两者都需要持久化。

每个快照对应一个bitmap,用于映射每个数据页的LBA到底层物理地址。
bitmap分两层:第一层私有,第二层在各个快照之间共享。
在写的时候,触发COW过程,从父快照copy对应的chunk,然后才能写入。

一个卷的多个快照,只有一个快照是可写的,其它快照都处于Read-only状态。
所有写的位置,必须是卷本身的数据空间。所以,对任一数据chunk,都是在可写快照时分配的,创建快照后冻结,产生新的写入点。

为了提高bitmap的存取性能,引入bitmap cache,按页组织(可以类比于VFS的page cache,每个inode对应一个radix tree)。
更新bitmap的粒度是bunit,通常为8B,对应一个底层物理位置,一次io可能对应多个bunit,甚至可能跨越page边界。

\subsection{Read}

\subsection{Write}

IO写的时候,对应的每一个bunit,都存在四种状态:
\begin{enumbox}
\item 本地
\item 没有分配
\item 指向某快照
\item 克隆卷,指向克隆卷的源快照
\end{enumbox}

后三种情况统一处理。

如果是单页更新过程,

如果是多页更新过程,

\section{ConsistencyGroup}

使用场景:如一个分布式数据库,log写入一个卷,数据写入另一个卷,如何打快照才能不违反应用层的一致性?

LVM的快照,涉及底层多个物理卷。clvm

ceph组内各卷,共用一个journal,性能低、架构不灵活

CG是一个独立实体,创建后为空,可以把一些卷加进来,按组进行快照操作。

  %\chapter{Volume}

\section{属性}

\begin{lstlisting}[language=c,frame=single]
typedef struct {
        fileid_t id;
        uint16_t magic;
        uint16_t repnum;
        uint64_t snap_rollback;
        uint64_t snap_version;
        uint64_t reference;         //clone reference
        uint32_t attr;
        int32_t  priority;
        uint32_t __pad__[4];

        uint64_t size;
        uint32_t mode;
        uint32_t uid;
        uint32_t gid;
        uint32_t ctime;
        uint32_t mtime;
        uint32_t btime;
        uint32_t atime;
} fileinfo_t;
\end{lstlisting}

卷是ServerSAN核心对象,是pool,snapshot,mapping和cg的中心。

卷的属性记录在L1 chunk的fileinfo段。fileinfo段是vol info区的第一个段。

\subsection{大小}

目前,卷的元数据由两级组成:L1,L2。L1只有一个chunk,每个chunk有8000个槽位。每个槽位指向一个L2 chunk。
L2 chunk有16000个槽位,每个槽位指向一个raw chunk。所以,最大卷大小约为$122TB = 8000 \times 16000 \times 1MB$。

为了支持更大的卷,需要扩展此结构,把L1扩展到多个chunk。改变会影响到:
\begin{compactitem}
\item 加载table1 (加载多个chunk,记录每个chunk的chunkinfo信息)
\item table1的各项操作
\item 遍历卷的chunk
\item 数据恢复过程
\item migrate
\item copy
\item snapshot clone,需要copy L1 chunk
\item ...
\end{compactitem}

L1 chunk:vol.xxx.0, vol.xxx.1, vol.xxx.2。其中,vol.xxx.0的chunkinfo记录在pool。
\hl{vol.xxx.1等chunkinfo记录到vol.xxx.0里info区的第二个块,并需记录其个数}。动态加载。

父:raw和subvol的父chkid为vol.xxx.0,vol.xxx.1的父chkid也为vol.xxx.0。vol.xxx.0的父chkid为所在pool的chkid。

缺页中断:从上到下检查。创建chunk的时候,先检查L1 chunk是否存在,然后检查L2 chunk是否存在。
不存在,则创建。在以下过程会改变卷的数据结构树:
\begin{compactitem}
\item \verb|__pool_proto_mkvol|
\item \verb|volume_proto_load|
\item \verb|volume_proto_chunk_pre_write|
\item \verb|__volume_proto_chunk_allocate|
\end{compactitem}

\begin{lstlisting}[frame=single]
lichbd write /iscsi/p1/v1 hello -o 280375465082880
lichbd cat /iscsi/p1/v1 -o 280375465082880 -l 5
\end{lstlisting}

chunk allocate开销较大,即要申请磁盘空间,也涉及记录元数据到父节点。

遗留问题:
\begin{enumbox}
\item lich.inspect --stat极慢 (统计,load)
\item 分析内存占用量
\item clone
\item recovery
\item balance
\item 回收空间
\end{enumbox}

\subsection{副本数}

\section{操作}

\begin{enumbox}
\item create
\item rm
\item ls
\item info
\item resize
\item rename
\item mv
\item copy (read/write)
\item migrate (move all chunk)
\item duplicate (snapshot-based: clone/flatten)
\item import
\item export
\item mapping
\item \hl{IO}
\end{enumbox}

\subsection{create}

\subsection{QoS}

token bucket。IOPS与block size相关,两者乘积等于带宽。
如果各层发生IO聚合,则在流量守恒的情况下,显示IOPS有所不同。

IOPS必须假定一定的block size。

  \chapter{快照和克隆}

\include{fusionstor/info}
\chapter{并发控制}

编程模型:多线程和事件驱动。多线程访问共享内存,需要同步机制。锁会形成死锁。
事件驱动有主循环,不能调用block操作,另外涉及如何动态扩展到多核多CPU的环境。

actor编程模型,SEDA是一个特殊的actor架构。

进程及其通信是基础模型。执行的代换模型与环境模型,事务的页模型与对象模型,
由简单模型推导关键结论。

并行有两种:竞争并行与协作并行。

\section{PCAM设计方法学}

\begin{enumbox}
\item 划分 数据划分、任务划分
\item 通信 任务之间的数据交换
\item 聚合 合并任务,以提升性能
\item 映射 任务调度,并符合负载均衡
\end{enumbox}

\subsection{Example:volume}

volume是一个自然划分,把volume映射到一个core thread上。
问题是这样做的话,粒度过大,无法有效进行负载均衡,受限于单core最大处理能力。

如果一个subvol映射到一个core thread,粒度合适。但需要引入复杂的控制机制。
最好是映射到一个节点的多个core上。

以上分析过程用到了完整的PCAM方法。

\subsection{Example:recovery}

\subsection{Example:rmvol}

无需通过controller去做这些事情,可以直接利用db信息。

\subsection{Example:rollback}

\subsection{Example:flatten}

\section{同步原语}

同步原语组成层次结构。并发原语,层层还原,层层构建

从OS到DB,对并发的讨论在深化,事务要求ACID,且访问特定的索引结构。
另外还有持久化的要求,原子写变得更困难,引入日志。

本质上是序的问题,锁内在地实现了一种序。
逻辑时钟、paxos、raft中,序起到了至关重要的作用。

\chapter{开发视图}

设计在解决关键问题的同时,要降低实现,测试和维护的复杂度。

加强测试,通过重构降低复杂度。

分解问题,界定边界,降低复杂度

设计的基本原则

分离机制和策略,接口和实现

性能依赖于设计,在一定的设计下,取决于实现。

性能优化手段:并行,聚合,缓存等,根本在于设计,控制复杂度。

识别实体和关系,ERD,DFD等,FSM是机器语言。

核心概念:
\begin{compactenum}
\item 存储池/目录
\item 卷
\item 快照
\item 主机映射
\end{compactenum}

core thread边界,core\_request进入。

分布式副本一致性:clock版本机制,msgqueue离线消息处理。

性能:并发,聚合和cache等

元数据管理,非计算而来

快照树的实现

后台任务统一管理,包括:
\begin{compactenum}
\item recovery
\item balance
\item vol rm
\item snap rm
\item snap rollback
\item snap flat
\end{compactenum}

架构问题:
\begin{compactenum}
\item 元数据管理成本
\item 支持大容量卷
\item 支持ROW快照树
\item 诊断流程和工具
\item 性能profile
\end{compactenum}

\section{故障域}

对任一存储池,设故障域数为M,副本数为N,

当M>=N时,每个故障域内一个副本,随机分布;
当M<N时,
- 策略1,每个故障域内一个副本
- 策略2a,剩余的副本按策略1进行,直到写完所有副本数
- 策略2b,不写剩余副本

按策略2a:

case 1:故障域为2,副本数为3,则副本在故障域的分布为(2,1)或(1,2)
case 2:故障域为1,副本数为3,则副本在故障域的分布为 3

按策略2b:

case 1:故障域为2,副本数为3,则副本在故障域的分布为(1,1)
case 2:故障域为1,副本数为3,则副本在故障域的分布为(1) (副本数不能少于2个,分配失败)

同时,恢复过程须按以上故障域规则进行自动校正!!!

\section{存储池状态}

\begin{compactenum}
\item 不可用
\item 磁盘空间不足/READ ONLY
\item 降级
\item 正常/健康
\end{compactenum}

\section{诊断方法}

对需要改进的流程进行区分,找到最有潜力的改进机会,优先对需要改进的流程实施改进。如果不确定优先次序,企业多方面出手,就可能分散精力,
影响6σ管理的实施效果。业务流程改进遵循五步循环改进法,即DMAIC模式:

\begin{compactenum}
\item 定义[Define]——辨认需改进的产品或过程,确定项目所需的资源。
\item 测量[Measure]——定义缺陷,收集此产品或过程的表现作底线,建立改进目标。
\item 分析[Analyze]——分析在测量阶段所收集的数据,以确定一组按重要程度排列的影响质量的变量。
\item 改进[Improve]——优化解决方案,并确认该方案能够满足或超过项目质量改进目标。
\item 控制[Control]——确保过程改进一旦完成能继续保持下去,而不会返回到先前的状态。
\end{compactenum}

信息有多级:USE。诊断问题依赖于结构化的诊断方法PAT,解决问题也是,构建知识图谱。

欲分析问题,必分析事物发展的完整过程,包括每个参与者的生命周期模型,参与者之间的相互作用。

\section{目录结构}

\begin{enumbox}
\item /opt/fusionstack/data
\item /opt/fusionstack/data/etcd
\item /opt/fusionstack/data/node
\item /opt/fusionstack/data/disk
\item /opt/fusionstack/data/chunk
\item /opt/fusionstack/data/status
\item /opt/fusionstack/data/recovery
\item /opt/fusionstack/data/cleanup
\item 
\item /dev/shm/lich4
\item /dev/shm/lich4/msgctl
\item /dev/shm/lich4/nodectl/*
\item /dev/shm/lich4/maping
\item /dev/shm/lich4/clock
\item /dev/shm/lich4/hb\_timeout
%\item /dev/shm/lich4/recovery
\item /dev/shm/lich4/volume
\end{enumbox}

\section{编程注意事项}

\begin{compactitem}
\item goto之前,先设置ret
\item 大多数情况下,需要检查函数的返回值
\item 函数的可重入性
\end{compactitem}

\section{Protocol}

Protocol与存储卷是正交功能。protocol应该独立于卷进行扩展。
所以,path中包含protocol组件,并非必须,而是实现上的权宜之计。

\section{Algorithm}

\begin{compactitem}
    \item paxos/raft 选举admin和meta节点
    \item lease controller的唯一性
    \item vector clock 副本一致性
\end{compactitem}

\section{Coroutine}

使用规则:
\begin{compactitem}
    \item \verb|schedule_task_get|和\verb|schedule_yield|要匹配
    \item task有数量上的限制:1024,超出后容易引起死锁
    \item schedule\_self 与 schedule\_running不同
\end{compactitem}

首先进行一种区分,core线程内与外;线程内进一步分为下述两张情况。

core调度器本身不同调用blocking操作,同样需要异步化。分两种情况:
一、core调度器本身调用block op;二、core内协程调用block op。
第二种情况可以用schedule\_yield(或core\_request),第一种则不可。如何保序?

\section{plock}

一个task有两种状态:一、持有锁;二、等待。

分为两种task:一、持有锁;二,等待。

读操作与写操作不对称。

序列:
\begin{enumbox}
\item rrrrrrrr
\item wwwwwwww
\item rrwwrrww
\item wwrrwwrr
\end{enumbox}

\section{AIO}

生产者-消费者模型。

\subsection{sqlite3}

异步sqlite3。etcd,RR也会采用该框架。

\begin{compactitem}
    \item 共10个db,每个一个线程,每个线程管理一个队列
    \item 所有sqlite3操作,泛化为统一的结构,放入线程队列
    \item 消费者线程批量处理队列中的任务
    \item 生产者线程和消费者线程通过sem进行通信
    \item 采用协程机制(yield/resume)同步任务执行顺序
\end{compactitem}

消费者线程wait在sem上,生产者线程有消息的时候,调用\verb|sem_post|。

\section{Performance}

4K+1M混合读写,极慢

机械盘关闭localize

IO路径优化
\begin{enumbox}
\item 调用深度
\item DBUG/DINFO等日志
\item 无关功能
\item 代码体积大,导致cpu高速缓存失效: -g
\item lock table用到了hash table
\item hash table用到了ymalloc/yfree机制管理内存
\item table2 subvol wrlock
\item inline
\end{enumbox}

\subsection{Lease}

利用lease机制来保证volume controller的唯一性。

加载卷时,尝试创建lease,成功后才能执行加载过程。

若没有IO,如果发生lease超时,admin会回收发生超时的lease。后续如有IO,需要重新申请。

如同锁一样,lease会发生抢占。lease是带timeout的锁。现在的实现,需要client去频繁检查,而不是通知机制。

(心跳,向量时钟,版本)能改善?

通过VIP机制,访问卷控制器的过程,与不同VIP,有所不同。

EREMCHG错误主要用于控制器发生切换的时候。

\subsection{Hash}

chkid如何选择hash函数?

\subsection{分支预测}

\subsection{Profile}

lich.conf: performance\_analysis: on

kill -USR1 <pid>

tail -f /opt/fusionstack/log/lich.log |grep analysis

\section{故障}

下电,感知有一定延迟,tcp timeout。

重新选举admin

\section{Safe Mode}

卷级进行检查,处在保护模式的卷,不允许iscsi连接,返回错误码。

加载时间较长的模块,用half sync/half async模式来处理。分为两阶段:同步+异步。

一个卷,加载成功,依赖于几个条件:
\begin{compactitem}
    \item raw/lsv
    \item module load
\end{compactitem}

\section{Log}

syslog

日志过滤

统计数据:准确性和实时性
软件工程

\chapter{Deploy}

\section{nohosts}

\begin{lstlisting}[frame=single]
[root@node76 ~]# cat /etc/hosts
192.168.1.76 node76
192.168.1.77 node77
192.168.1.78 node78
192.168.1.83 node83

[root@node76 ~]# cat /opt/fusionstack/etc/hosts.conf 
#hosts list for nohost mode
192.168.110.76 node76
192.168.110.77 node77
192.168.110.78 node78
192.168.110.83 node83
\end{lstlisting}

\section{Network}

网段和netmask一致,vip一致

ip addr检查相同IP

arping检查IP冲突 

\section{Disk}

Disk Cache和raid cache下,磁盘读写的行为分析,cache的行为具有通用性。

如果raid cache打开,writeback写,全随机,则r\_await远远大于w\_await。
同时,写入的情况下,iostat util较低。\hl{写采用writeback模式,故性能高;
在全随机的情况下,读不具有局部性,缓存命中率低,故性能低}。

\begin{tcolorbox}
存储池内每一个节点上,都需有SSD,用于tier和cache。
如果一个节点属于多个存储池,也需满足以上条件。
\end{tcolorbox}

\section{节点}

\subsection{添加节点}

负载均衡,需要重新平衡数据

\subsection{删除节点}

恢复


\section{Pool}

\section{Disk}

画出Disk的状态机

\subsection{Tool}

操作磁盘的工具
\begin{enumbox}
\item hdparm set/get hard disk parameters
\item lsscsi
\item udevadm
\item sg\_inq
\item disk2lid (自己实现的)
\item iostat
\end{enumbox}

\subsection{RAID}

Ctrl+R进入bios的RAID控制界面。通过bios进行的管理操作,进入系统后用MegaCli等命令行工具也能完成。
且更为方便。

每个控制器管理一个或多个enclosure,每个enclosure有固定数量的slot。每个slot对应一块物理设备。
每个物理设备处在不同的firm 状态,这些状态可以相互转换。在物理设备之上,构建虚拟设备。

带电池BBU的情况下,可以打开RAID cache。

NVMe在不在RAID里,系统盘呢?看不到系统盘,是因为系统盘不在RAID控制器管理的slot内?

缓存盘做出JBOD,数据盘RAID0

全部是JBOD,性能影响较大。

\subsection{Cache}

Disk Cache

关闭磁盘cache,防止出现数据不一致情况。怎么关闭呢?相关管理工具是什么?

磁盘的设备驱动,linux kernel的块设备IO架构

NVMe具有什么特征?

\subsection{Meta}

/proc/partitions包含所有分区,过滤掉特定的分区,就是lich可用的设备。
注册到lich的设备定义有自身的元数据:\hl{数据盘包括磁盘头的元数据以及元数据文件}。

数据盘对应的cache设备记录在bcache本身的元数据里,通过sys fs,以及用户态工具管理。

diskid的唯一性在\hl{所有场景}中能否得到保证?

磁盘管理元数据目录:/opt/fusionstack/data/disk:
\begin{compactitem}
\item disk (磁盘在线离线的开关)
\item block (super block)
\item info (diskinfo)
\item \hl{bitmap} (不可丢)
\item tier
\item speed
\end{compactitem}

对每块磁盘,开头的1M是引导信息,通过bitmap来进行空间管理。
引导信息包含了所在节点信息,所以只能在节点内进行磁盘漫游。

\begin{figure}[h]
    \centering
    \includegraphics{../images/disk_layout.png}
    \caption{磁盘数据布局}
\end{figure}

磁盘头的元数据布局:MBR(0)+SB(1024)+DISKINFO(4096)。
其中SB包含定义的扩展属性:(cluster, node, type, disk, pool, cache, cached, cset)

每块盘的disk id属性,可用来重建disk/disk目录下的软链接。
block与info文件都可以通过读取磁盘头重建出来。bitmap文件则不可丢。

所在源文件是\emph{diskmd.c}。调用fnotify\_register监控磁盘目录的变化,进而添加或移除相应磁盘。

sqlite3划分为10个db文件,chkid信息hash到相应的db。每个db包含两个table:metadata和raw。

每个节点最多可以添加256个磁盘。

通过lich.node添加磁盘后,创建软链接,触发lichd的相关处理过程。

\begin{lstlisting}[frame=single]
CREATE TABLE metadata (key text primary key,
    disk integer,
    offset integer,
    parent text,
    priority integer,
    meta_version integer,
    fingerprint integer,
    wbdisk integer);

CREATE TABLE raw (key text primary key,
    disk integer,
    offset integer,
    parent text,
    priority integer,
    meta_version integer,
    fingerprint integer,
    wbdisk integer);
\end{lstlisting}

\subsection{状态机}

拔盘-恢复未完成,如何加入?(符号链接丢失,别的文件处在可用状态,重启lichd,修复符号链接)

拔盘-恢复完成,如何加入?

\subsection{Tier}

检测磁盘分层,支持两个磁盘分层:0和1, 0是SSD,1是HDD。

\section{Advanced}

NVMe

RDMA/DPDK/SPDK

AFA

\chapter{运维视图}

FAQ

\section{问题集}

问题集
\begin{enumbox}
\item lichd --init做了什么
\item 每个节点的nid保存在哪儿,如何分配的?
\item /opt/fusionstack/data目录布局
\item /dev/shm/lich4目录布局
\item cleanup
\item clock机制
\item hsm
\end{enumbox}

\begin{compactitem}
\item coroutine and scheduler
\item polling \change{polling}
\item kernel bypass
\item mbuffer
\end{compactitem}

问题集:
\begin{enumbox}
\item /的位置信息
\item 当前分配的最大卷ID?
\end{enumbox}

P1: IOMeter测试,256K,Lich顺序和随机IO性能差别大

P2: lsv\_gc\_check断言失败

P3: Error Handling

\section{约束}

\subsection{最大副本数}

6

\subsection{单卷最大快照数}


\section{Performance}

\subsection{估算}

HDD:
\begin{itembox}
\item 顺序1M 200
\item 随机1M 50
\item 顺序4K IOPS
\item 随机4K IOPS
\end{itembox}

集群部署
\begin{enumbox}
\item 网络
\item 磁盘
\item 副本数
\end{enumbox}

\subsection{精简卷顺序1M写入性能低}

subvol写锁,导致串行化

用allocate命令先批量填充,性能有显著提升。


\chapter{iSCSI}

\section{Getting Started}

iscsiadm

setenforce 0

\section{Concepts}

\mygraphics{../imgs/iscsi/iscsi-conf.png}


Target/LUN: target和lun的discovery机制

Session/Connection

每个suzaku卷对应一个target,每个target只有一个lun?

discovery的target与volume数和tgtctl的core数有关,是两者的乘积。

target的port?多路径?

VAAI

与NVMf规范的相关性

\section{Code}

与suzaku相关的关键文件
\begin{myeasylist}{itemize}
& target.c
& volume.c
& efs\_io.c
\end{myeasylist}

\section{Key Processes}

\subsection{Discovery}

\mygraphics{../imgs/iscsi/iscsi-discovery.png}

\subsection{Connect}

\mygraphics{../imgs/iscsi/iscsi-connect.png}

\section{iSER -- iSCSI over RDMA}

iSER利用了部分iet代码,特别是对接后端存储的部分。

参考\hl{iser\_cmds/iser\_scsi\_cmd\_iosubmit}函数。

\section{Reference}

RFC
\begin{myeasylist}{itemize}
& rfc 3720
& rfc 3721
& rfc 7143
\end{myeasylist}

\chapter{NVMe}

\section{NVMe设备}

\begin{lstlisting}
# 管理NVMe设备:
lspci | grep Non
lsblk

echo '0000:04:00.0' > /sys/bus/pci/drivers/nvme/bind
echo '0000:04:00.0' > /sys/bus/pci/drivers/nvme/unbind

# 支持两种模式: spdk on | off
# SPDK模式,会卸载NVMe驱动,以PCI号的方式使用
# 可以kernel bypass,提升性能
lich.node --start --ttyonly
\end{lstlisting}

uio\_pci\_generic内核驱动

drivers和devices存在多对多的关系。

\subsection{依赖项}

libnvme,还没有采用spdk,NVMf target采用SPDK。

\chapter{SSD}

硬件
\begin{enumbox}
\item *
\item QLC
\item PCM
\item Intel optane
\item PCIe Bus
\end{enumbox}

跟踪产品和项目
\begin{enumbox}
\item \href{https://e8storage.com/}{E8 storage}
\item \href{https://www.excelero.com/}{Excelero}
\item tgt
\item spdk
\item dpdk
\end{enumbox}

全闪做什么?
\begin{enumbox}
\item 全局QoS
\item 全局磨损
\item 加速硬件(FPGA)
\item Oracle ASM
\item 全闪需不需要cache?
\end{enumbox}


% perspective
\chapter{性能}

\section{性能估算}

\begin{enumbox}
\item 单卷性能
\item 网络
\item 磁盘
\item 副本数
\end{enumbox}

\chapter{可靠性}

重建延迟时间

如何检测磁盘故障

如何标记一块盘?

受控的磁盘分组

\section{场景}

追踪如下典型场景:
\begin{enumbox}
\item 添加盘
\item 拔盘后插入
\item 删除盘,删除pool,重用该盘
\end{enumbox}

磁盘的状态通过状态机来维护。每种状态下,其关联的资源项的明确规定。

一个磁盘处在out/in状态的标志是什么?

如何快速清除一个pool?

\section{磁盘故障}

拔盘,引起IO错误,或检测线程检测出磁盘不可写状态,此时,把磁盘的内存状态设为offline,并删除对应软链接。
调度进行恢复操作。待恢复完成后,disk unload,才正式从集群中剔除。

\hl{写过程中发生EIO,则会导致lichd进程重启}。在退出之前,调用disklost过程,删除软链接。

如果没有删除完成,重启了lichd,则会修复软链接,加载该盘。

测试中,出现了无软链接,但有别的相关磁盘文件的情况,磁盘加载后进入offline状态,但因为数据库残存有该磁盘的相关记录,
故进入不了完全删除的状态。lich health显示有offline磁盘。

\hl{磁盘故障的处理,同时也作为GC机制}。数据库的记录,不在chunk的当前位置列表中,即可判定为是垃圾数据。

原来掉盘的情况下,没有关闭磁盘对应的文件描述符,导致\hl{bcache的数据盘不能register}。所以,掉盘时需要关闭相关描述符。
再次上线的时候,重新open即可。

\subsection{磁盘元数据}

从状态一致性的角度进行分析。

一个磁盘的状态,依赖于其本身与关联元数据资源。约分为三级:
\begin{enumbox}
\item 物理磁盘+RAID
\item + BCACHE
\item LICH磁盘 (数据文件与内存状态)
\end{enumbox}

其他派生状态

LICH磁盘文件:
\begin{enumbox}
\item /opt/fusionstack/data/disk/disk/\%.disk (soft link)
\item /opt/fusionstack/data/disk/block/\%.block
\item /opt/fusionstack/data/disk/bitmap/\%.bitmap
\item /opt/fusionstack/data/disk/info/\%.info
\item /opt/fusionstack/data/disk/tier/\%.tier
\item /opt/fusionstack/data/disk/speed/\%.speed
\end{enumbox}

LICH内存状态:
\begin{enumbox}
\item offline
\item deleting
\end{enumbox}

\subsection{场景:添加磁盘}

如果bcache关系已存在,仅仅创建disk文件,触发lichd的加盘操作,这样有无问题?毕竟\hl{bcache的数据盘没有被重新格式化}。

\subsection{场景:删除符号链接然后加上}

为什么加上符号链接后,恢复速度反而变慢很多?

一块盘被拔出,经过一段时间后再插入,是否可以停止对应恢复进程?
该盘对应的数据块处在几种状态:最新、过期、或成为垃圾。

仅仅停止恢复进程是不够的,再次调度恢复(包含GC)?

\subsection{场景:拔出cache盘随后插入}

概率性地出现如下问题:cache盘无法上线,data盘无法上线。即register阶段失败。
重启服务器后恢复正常。

data盘无法上线的情况:如果一块盘恢复完成了,可以观察到register成功。说明什么?
lsblk依然能看到虚实数据盘之间的映射关系。

\subsection{场景:删盘}

主动删除盘后,需要恢复。但如果进一步删除了pool,则须退出pool内所有disk的恢复过程。

\subsection{管理线程}

\section{节点故障}

\section{工具}

\subsection{Hazard}

hazard的IO模式:对同一IO extent,先写入,再读取。如果两者不一致,则再次读取。
写和读分两次,每次的大小是随机的。但两次加起来代表一个完整的IO extent。
只有写入与第一次读取不一致时,才有第二次读取进一步验证。故结果分为几种情况:
\begin{enumbox}
\item w != r1, w == r2,为读错误
\item w != r1, r1 == r2,为写错误
\item w != r1, w != r2,为unknown
\end{enumbox}

绘成三角形,如w为顶点,r1、r2为两底点,能更形象地去表示三者的关系。

\hl{定位对应的chunk},区分两种情况:raw和fs。raw的情况容易定位,\hl{(LBA-1) * 512}即可计算出对应的chunk,一个LBA对应512B。
fs需要借助辅助手段来定位。模拟一io,通过底层FusionStor的log来定位。

\chapter{QOS}

\section{概述}

先需明确问题,是单点控制,还是分布式控制?

学习的方法:
\begin{enumbox}
\item \hl{对标}:行业的标准做法是什么?
\item 如何才能更好地学习?
\item 可以参考的资料有哪些?
\item *
\item 先选出几篇经典论文,顺藤摸瓜,建立相关的知识体系。
\item 与专业人士交流,获取有价值的线索。
\item 还需要主动去悟,提问、消化、守破离,推陈出新
\end{enumbox}

参考网络QoS,存储QoS的核心算法与网络QoS相同。

排队论

态势感知?

在高IOPS的情况,QoS的开销过大,极大地拉低了性能,这是不可接受的。

每次请求都要获取一次时间,是不是必要的?

\subsection{参考}

\begin{enumbox}
\item OS中进程、线程调度算法
\item Disk IO调度算法
\item VM IO调度算法
\item Network QoS and Storage QoS
\item TCP/IP
\item iSCSI
\item SPDK QoS
\item Ceph dmClock
\item SolidFire QoS
\end{enumbox}

\section{算法}

采用了两种曲线

开放控制参数

比较指标:理论和实测值的距离,\hl{也可以考虑夹角的大小}。\change{距离函数}

底层采用token bucket,需要能容忍一定的jitter。

在调度器内加入QoS控制逻辑的设想: 每个core调度器对应一个或若干卷控制器。基于优先级队列,由core线程处理队列(scheduler队列?)。
每个卷控制器在对应的scheduler上注册自己的队列(IO任务、恢复任务)。\hl{core上的每个卷,向scheduler注册自己,从而实现解耦}。
调度器不仅可以处理单个卷的QoS,也可以处理多个卷的QoS。

\hl{队列和线程}往往紧密结合为一体,参见SEDA、actor。

\hl{多mode调度器},根据实际负载条件动态地调整调度器策略。

何时从请求队列移入调度队列是QoS调度器的中心任务。

\section{已知问题}

顺序io,在上层聚合,导致vctl上的qos不准确。

\chapter{Bcache}

\section{管理}

与web管理系统配合,通过web能完成大部分操作

disk list,返回json,不能有别的输出,比如调试信息等。

\section{RAID}

有坏盘,导致RAID进入保护模式,能列出物理设备(Foreign状态),不能列出逻辑设备。

RAID变成writethrough模式?\hl{writeback模式的RAID有丢失数据的风险}。

拔的盘,重新加入raid阵列:
\begin{enumbox}
\item raid miss
\item raid load?Foreign状态,Import 导致服务器hang住
\item raid add?Foreign状态,Clear 导致服务器hang住
\end{enumbox}

\section{原理}

二实一虚,2+1=3,两类物理设备,一类虚拟设备,虚拟设备是入口

三种cache mode

writeback的复杂性

基本用法
\begin{enumbox}
\item 分别格式化cache盘与数据盘
\item 数据盘绑定到cache盘
\item 向kernel注册,每个数据盘出现一个/dev/bcacheXX虚拟设备
\end{enumbox}

\subsection{工具}

\begin{enumbox}
\item make-bcache
\item wipefs
\item lsblk
\item bcache-super-show
\item dmesg
\item lsmod
\item /sys/fs/bcache
\item /sys/block
\item dd if=/dev/zero of=/dev/sdb count=1 bs=1024 seek=4
\end{enumbox}

\section{可靠性}

手动测试,无fusionstor的情况
\begin{enumbox}
\item 拔出数据盘
\item 拔出cache盘
\end{enumbox}

一定需要重启服务器吗?

单节点故障与单磁盘故障有很大不同,对应的恢复过程和性能也有很大不同。

卷控制器rebalance后,不自动触发恢复过程

RECOVERY\_RMQ\_MAX\_RECORD

\subsection{RAID的影响}

% 拔数据的状态机,恢复完成后会踢出该盘

加入raid后,bcache导致kernel hangup。

有什么办法在加入raid前,先禁用bcache。待raid ready后,再启用bcache?
操作范围:cache set。

重启节点,后导致clock丢失,如果一定会发生hangup,最好先stop lich,起到保存clock的目的。

盘符变化后,需要重启lich,重建相关链接

副本多的情况

\section{性能}

测试缓存命中率

单节点故障下的性能
\begin{enumbox}
cache disk的性能
bcache的各项配置 (cache mode, writeback percent, cutoff)
\end{enumbox}

\section{FAQ}

readitems crc error

btree head error

/dev/sd*多出的设备是因为IPMI。

\chapter{优化项}

\section{时间优化}

\begin{itemize}
    \item localize
    \item auto tier
    \item ssd cache
\end{itemize}

\section{空间优化}

\begin{itemize}
    \item 精简配置 (Thin provisioning)
    \item EC
    \item Dedup
    \item Compress
\end{itemize}




% development view
\chapter{作为一个操作系统}

\section{Scheduler}

从操作系统的角度去理解协程。协程也意味着执行的不连续性,与上下午切换。
协程有独立的调度器,负责从就绪队列里选出下一个要执行的任务。

\subsection{实现}

系统可以指定几个core,专门处理IO。每个core关联一个core thread,运行scheduler代码。

本质上,所谓的scheduler,就运行特定指令序列的操作系统线程,通过操作特定的数据结构和上下文切换,
模拟协程行为。

别的线程要与core thread通信,需要进行同步。core thread内部,不需要同步措施。

调度器和协程代码,交错执行。调度器切换到task,则执行task指令;
task让出控制权,或者返回,则执行调度器指令。

创建task,通过makecontext管理task要执行的过程,通过swapcontext在调度器和task之间切换上下文。

task的生灭过程,一个task的生命周期,状态变迁:FREE, RUNABLE/READY, RUNNING, SUSPEND, BACKTRACE。

BACKTRACE用于打印日志,扫描每一个task,检查task的存活时间。目前,不允许有特长时间的任务存在。
BACKTRACE是由调度器发起的。BACKTRACE不检查sleep状态的task。

每个task,有parent,从而构成task树。

目前维护有三个task队列(runable, reply\_local, reply\_remote),两个新的申请队列(wait\_task, request\_queue)。

task队列占用tasks槽位,总数受限,\hl{目前默认为1024个槽位}。

调度器支持优先级队列:引入不同于runable的队列,多队列之间分配时间片。

Actor编程模型,SEDA高并发架构。

\subsection{使用}

\subsubsection{接口}

\subsubsection{使用模式}

\subsubsection{FAQ}

\begin{compactenum}
\item schedule\_yield1 timeout。检查的是\hl{从yield到再次唤醒的时间,而不是任务的age,一个task可以多次yield/resume}
\item tasks槽位占满后,再加入的请求会放入wait list。task和wait list之间,可能形成相互依赖的deadlock。
\item schedule\_task\_get和schedule\_yield1必须配对使用
\end{compactenum}

\section{控制器}

目录和卷,都通过controller进行管理。目录controller的概念有待进一步完善。

每个控制器都可以看着一chunk树,其根节点对应chunk的副本位置列表,决定了控制器的位置:列表中第一副本所在节点。
在迁移控制器时,同样需要遵循该规则,所以需要先调整根chunk的chkinfo。

所有卷的操作都需要通过controller进行。客户端在访问一个卷时,第一步要找到该卷控制器的所在节点nid,
然后把nid作为参数传入后续调用中,如nid是客户端,则进程内;否则,发起rpc调用。
本地访问也可走rpc,可以利用rpc timeout等特性。

md\_map是控制器位置缓存,如可以在cache里找到,直接返回。如缓存不命中,则发起UDP广播。
每个lichd进程有独立线程监听端口:20915。检查cluster uuid,magic,crc等匹配后,尝试加载控制器,然后做出回应。
发起UDP广播的客户端收集各lichd进程的响应,如找到匹配的nid,可以直接退出该过程。

控制器加载过程:第一副本非本节点,返回EREMCHG。加载成功后,用vctl缓存管理起来,缓存项带引用计数和删除标志。
目前,采用lease机制保证vctl的唯一性。其必要性可进一步推演。
加载过程需要保证并发下的唯一性,如果有多个task发起加载过程,只有一个实际执行,别的进入等待队列。加载完成后,唤醒等待队列里的任务。

\section{内存}

每个core线程拥有私有内存,但不应静态分配一个固定值,而是动态按需分配。
因为每个core线程所需内存量可能不均衡,出现总量够,而单个core内存不足的情况。

预留、总量

资源池模式

\section{调度}

\section{I/O}

\subsection{Disk Allocator}

\subsection{Sqlite}

\subsection{FS}

\subsection{SPDK}

\subsection{RPC}


% \chapter{硬件架构}

\section{Disk}

\subsection{Cache}

RAID

关闭磁盘 cache,防止出现数据不一致情况。带电池的情况下,可以打开RAID cache。

\subsection{Tier}

检测磁盘分层,支持两个磁盘分层:0和1,0是SSD,1是HDD。

\subsection{Meta}

磁盘管理元数据目录:/opt/fusionstack/data/disk:
\begin{compactitem}
\item disk
\item \hl{bitmap}
\item info
\item tier
\end{compactitem}

对每块磁盘,开头的1M是引导信息,通过bitmap来进行空间管理。
引导信息包含了所在节点信息,所以只能在节点内进行磁盘漫游。

所在源文件是\emph{diskmd.c}。调用fnotify\_register监控磁盘目录的变化,进而添加或移除相应磁盘。

每个节点最多可以添加256个磁盘。

sqlite3划分为10个db文件,chkid信息hash到相应的db。每个db包含两个table:metadata和raw。

\begin{lstlisting}[frame=single]
CREATE TABLE metadata (key text primary key, 
    disk integer, 
    offset integer, 
    parent text, 
    priority integer, 
    meta_version integer, 
    fingerprint integer, 
    wbdisk integer);

CREATE TABLE raw (key text primary key, 
    disk integer, 
    offset integer, 
    parent text, 
    priority integer, 
    meta_version integer, 
    fingerprint integer, 
    wbdisk integer);
\end{lstlisting}

\section{Advanced}

NVMe

RDMA/DPDK/SPDK

AFA

% \chapter{ETCD}

/opt/fusionstack/etcd


etcdctl ls -r



% operational view
\chapter{BUG}

\section{writeback}

\verb|__writeback_flush__|


% \section{控制器}

目录和卷,都通过controller进行管理。目录controller的概念有待进一步完善。

每个控制器都可以看着一chunk树,其根节点对应chunk的副本位置列表,决定了控制器的位置:列表中第一副本所在节点。
在迁移控制器时,同样需要遵循该规则,所以需要先调整根chunk的chkinfo。

所有卷的操作都需要通过controller进行。客户端在访问一个卷时,第一步要找到该卷控制器的所在节点nid,
然后把nid作为参数传入后续调用中,如nid是客户端,则进程内;否则,发起rpc调用。
本地访问也可走rpc,可以利用rpc timeout等特性。

md\_map是控制器位置缓存,如可以在cache里找到,直接返回。如缓存不命中,则发起UDP广播。
每个lichd进程有独立线程监听端口:20915。检查cluster uuid,magic,crc等匹配后,尝试加载控制器,然后做出回应。
发起UDP广播的客户端收集各lichd进程的响应,如找到匹配的nid,可以直接退出该过程。

控制器加载过程:第一副本非本节点,返回EREMCHG。加载成功后,用vctl缓存管理起来,缓存项带引用计数和删除标志。
目前,采用lease机制保证vctl的唯一性。其必要性可进一步推演。
加载过程需要保证并发下的唯一性,如果有多个task发起加载过程,只有一个实际执行,别的进入等待队列。加载完成后,唤醒等待队列里的任务。

% \chapter{Scheduler}

\section{实现}

系统可以指定几个core,专门处理IO。每个core关联一个core thread,运行scheduler代码。

本质上,所谓的scheduler,就运行特定指令序列的操作系统线程,通过操作特定的数据结构和上下文切换,
模拟协程行为。

别的线程要与core thread通信,需要进行同步。core thread内部,不需要同步措施。

调度器和协程代码,交错执行。调度器切换到task,则执行task指令;
task让出控制权,或者返回,则执行调度器指令。

创建task,通过makecontext管理task要执行的过程,通过swapcontext在调度器和task之间切换上下文。

task的生灭过程,一个task的生命周期,状态变迁:FREE, RUNABLE/READY, RUNNING, SUSPEND, BACKTRACE。

BACKTRACE用于打印日志,扫描每一个task,检查task的存活时间。目前,不允许有特长时间的任务存在。
BACKTRACE是由调度器发起的。BACKTRACE不检查sleep状态的task。

每个task,有parent,从而构成task树。

目前维护有三个task队列(runable, reply\_local, reply\_remote),两个新的申请队列(wait\_task, request\_queue)。

task队列占用tasks槽位,总数受限,\hl{目前默认为1024个槽位}。

调度器支持优先级队列:引入不同于runable的队列,多队列之间分配时间片。

\section{使用}

\subsection{接口}

\subsection{使用模式}

\subsection{FAQ}

\begin{compactenum}
\item schedule\_yield1 timeout。检查的是\hl{从yield到再次唤醒的时间,而不是任务的age,一个task可以多次yield/resume}
\item tasks槽位占满后,再加入的请求会放入wait list。task和wait list之间,可能形成相互依赖的deadlock。
\item schedule\_task\_get和schedule\_yield1必须配对使用
\end{compactenum}


\part{LichMaster}

\chapter{Lich Master}

\section{生态}

全闪架构有两种:阵列、分布式。分布式高一级,更灵活,实现难度也更大。

\begin{enumbox}
\item 有哪些产品,各有什么亮点
\item 硬件的性能数据 (亚毫秒级)
\item 如何评估性能理论上限
\end{enumbox}

latency and iops

万兆网络的延时100微秒,采用用户态协议栈可以达到20微秒左右,IB RDMA大概5-8微秒。

NVMe的延时20微秒

\subsection{华为OceanStor}

华为OceanStor Dorado V3是面向企业关键业务打造的全闪存存储系统。

采用智能芯片、NVMe架构和FlashLink®智能算法,在开启重删、压缩、快照等增值特性后仍能保证0.5ms的稳定时延, 业务性能提升3 倍。

支持平滑扩展到16个控制器,满足未来不可预期的业务增长。一套系统同时支持SAN和NAS,企业级特性齐备,为数据库和文件共享等应用提供更高品质的服务。

支持免网关双活方案,可平滑升级到两地三中心方案和融合数据管理方案,实现方案级99.9999\%可靠性保障。

通过在线重删、在线压缩技术,提供可达5:1的数据缩减率,OPEX节省75\%。

满足数据库、虚拟桌面(VDI)、虚拟服务器(VSI) 和SAP HANA场景所需,助力金融、制造、运营商等行业向闪存时代平滑演进。

\section{取势}

AFA是趋势,SDS向AFA过渡需要在架构上做很多事情,也是ceph下一步演化的重点。

从1+3模型描述硬件资源的使用方式。
\begin{enumbox}
\item NUMA下的多核
\item 内存管理
\item 存储设备
\item 网络通信
\end{enumbox}

网卡、NVMe都是pci设备,是内存设备,即对应一定的内存地址,操作对应内存地址就是与该设备进行通信。
pci设备不同于诸如硬盘等sata设备。\hl{pci设备的精密度远比sata设备为高}。

iscsi和iser的代码是从tgt项目的不同版本取得的,NVMf则用的是spdk。
spdk实现了很多内存管理功能。

什么情况下需要数据对齐、内存对齐?

全用户态的SDS架构,kernel bypass。

\section{RDMA}

latency是最重要的性能指标。

RDMA是一种transfer,采用DPDK USN也是一种选择,比IB慢,但比kernel TCP协议栈要快几倍。

\begin{enumbox}
\item RDMA的编程模型?
\item RDMA的连接管理过程?
\item RDMA的内存使用方式有什么不同?
\item iSCSI/iSER conn是如何关联起来的?
\item 更好的抽象?
\end{enumbox}

RDMA是与TCP并列的一种网络传输方式,需要特定硬件支持,包括RDMA的NIC与交换机。
网络设备不同于存储设备,RDMA可以carry任意网络流量,包括iSCSI/iSER,自定义协议(corenet)等。

ibverbs API屏蔽了链路层的不同,可以用一套API同时支持IB,RoCE、iWARP等。

\subsection{ibverbs}

分为几个层次:node、core/dev、conn、event。

每个设备对应rdma\_info\_t结构,有ibv\_context属性,是唯一的key。
rdma\_cm\_id具有该属性。

\begin{itembox}
\item pd
\item cq (被所有连接共享)
\item mr  (shared)
\end{itembox}

深刻理解一个RDMA连接管理的过程,建立连接的每个阶段需要做哪些工作?

每个RDMA连接,qp关联到cq上。cq\_poll。iov\_mr

qp支持ibv\_post\_recv操作,peer的send操作会消耗这些buffer,
在与远端建立之后,就应注册相关buffer,建立必要的关系,特别是指定wr\_id属性,并post,然后等待接受请求。

在ibv\_cq\_poll之后,得到的ibv\_wc里,包含所需上下午信息(post之前建立的关系,初始化task时)。
可以直接从buffer里recv到的数据。

RDMA是异步通信机制。

\subsection{MM}

RDMA通信包括两类:msg与数据读写,所需内存都需要register。

向RNIC注册内存

\subsection{建立连接}

采用epoll机制

\subsection{通信}

采用ibv\_cq\_poll机制

\subsection{协议-iSER}

iSCSI over RDMA

iSCSI分多阶段,包括Login、Full Feature等。

\section{DPDK}

\section{SPDK}

\subsection{NVMe}

\subsection{协议-NVMf}

NVMe over Fabric

\section{MM}

\subsection{问题}

\begin{enumbox}
\item 简单接口
\item 便于调试
\item 并发性能
\item 内外碎片
\item 动态化
\item RDMA内存
\item replica\_srv\_init不能利用core内存,模块依赖性
\item Hash局部性不好,不利于CPU高速缓存
\end{enumbox}

\subsection{现状}

源文件:
\begin{enumbox}
\item ylib/lib/mem.c
\item ylib/lib/mem\_hugepage.c
\item ylib/lib/mem\_cache.c
\item ylib/lib/mem\_pool.c
\item ylib/lib/buddy.c
\item ylib/lib/buffer.c
\end{enumbox}

两级内存管理

分为core内外两种情况:core使用私有内存。

先生成hugepage,并置0。malloc后得到虚拟地址空间,把hugepage依次mmap到该虚拟地址空间。

hugepage与numa物理内存的关系是怎么样的?什么时候建立起来的?

底层hugepage不一定要用buddy,并且应可动态扩展。
然后是pool层,可动态扩展,用buddy管理每个hugepage。
其上是对象层slot,用于分配应用对象,用大小不等的多个队列管理。

RDMA需要注册内存,目前是把整个core内存一次性注册了。
每个core都调用注册函数?

\section{Performance}

\subsection{Hash}

\begin{enumbox}
\item Lock table
\item Replica srv
\end{enumbox}

\subsection{Queue Pair}

aio、RAMD、iSER、NVMe等异步通信,大多采用多queue模型,有提交队列和完成队列。维持一定的上下文。
调度器、QoS也基于队列模型。

block和非block的区别,block是事件/中断驱动,非block是polling mode driven(PMD)。

异步意味着非block,把提交和完成解耦,可以并行执行,重要的是完成后可以配对到相应任务。
每一个请求建模为一任务,构成一状态机。如果实现为协程,就是隐式的状态机。

任务组织成一定的数据结构,所谓调度,就是从中选取下一个要运行的task。然后,切换上下文。

网络层、AIO都是如此。sqlite、redis等操作,派生出独立的工作线程,每个线程维护有私有的请求队列。

\subsection{socket}

区分连接socket和监听socket。监听socket构成网络连接的一方,而是起到辅助建立连接的作用。
在epoll等multiplexing机制里,可以同时处理。

aio通过事件抽象后,也纳入事件驱动的方式,其它如定时器等。

\chapter{快照和克隆}

\chapter{LSV}

现有Lich raw卷,存在性能问题,COW快照也不便于扩展。所以实现了Log structured Volume,
转化随机IO为顺序IO,基于其上,实现了ROW快照。

特别要注意的是,实现中应着力避免顺序IO随机化,会引起IO放大,从而极大地降低性能。


\section{Volume}

Volume模块负责空间管理。提供malloc/free接口,也可批量分配和回收。采用bitmap和free list多种管理方式。
freelist充当分配缓冲区的角色,可持久化,也可不持久化。

lsv-lich raw-disk的chunk空间存在两级映射关系,会影响到读写性能。

底层空间宜按固定大小的段来组织。每个段空间管理的开销是固定的。
目前支持两级存储分级:
\begin{tcolorbox}
    \begin{multicols}{2}
        \begin{itemize}
            \item 0:ssd
            \item 1:hdd
        \end{itemize}
    \end{multicols}
\end{tcolorbox}

\section{Bitmap}

Bitmap更合适的叫法是页表,与操作系统里的页表类似,负责虚拟地址到物理地址的映射关系。Bitmap有两层:L1和L2,
按类似页表的方式组织。和Log层数据一起,构成三层。

L1是Bitmap的头部,大小固定,属于卷或快照私有。L2按需分配,在快照之间共享。在Clone的情况下,会涉及跨卷读。

通过Bitmap层,支持快照的全部特性,多个快照构成快照树。快照树分两种方式展示:树状或列表。

\section{Log/Chunk}

底层物理空间,划分为固定大小为1M的数据块,进行统一管理:分配/释放。

在Volume模块之上做了简单封装,表示卷的数据,Bitmap表示卷的元数据。在覆盖更新的情况下,Bitmap指向新的数据页,
导致原来的数据页失效,可以回收。在有快照的情况下,会变得较为复杂。

Log模块无需要持久化的信息。

在Lich卷空间映射到磁盘的时候,目前实现为一个随机过程。\textcolor{red}{磁盘的1M随机和顺序,差别较大}。

\section{WBuf}

Wbuf有两个序列:WAL和Wbuf的提交序。在wbuf中读出的最新数据和提交后通过bitmap+log读取的数据,应该一致。

IO内,LBA不同,无冲突,页序;IO间,LBA可能相同,有冲突,需要串行化。

\section{RCache}

多级缓存机制,需要注意针对多种读场景进行优化,如顺序读。因为经过虚拟页表映射,虚拟地址空间和物理地址空间,顺序可能是交叉的。
应着力避免出现顺序变随机导致读放大的情况。

预读很重要,也比较困难,需要构建学习模型。

\section{GC}

log功能单一化,gc模块独立出来。gc要解决的问题有二:
\begin{enumerate}
    \item 跟踪所有log
    \item 在所有log中,根据一定策略(qos),选择回收价值最大者进行回收
\end{enumerate}

目前的实现,是局域的解,而不是全局最优解,是bottom-up的分代垃圾回收器。可增量并行执行,与前台赋值器需要同步机制。
回收器和赋值器需要读写barrier。

优化GC Check过程:每一页的信息,只会出现在部分的bitmap记录里,与快照树的拓扑结构有关。
在创建快照时,分配snap id。 snap id组织成单调递增的序列。如果中间没有删除或rollback操作,
很容易定位到某页所属的快照点。经过rm或rollback之后,情况有所不同,但依然有迹可循。

\section{Recovery}

正常关机的情况下,各个模块会flush必要的数据,下次启动的时候,load出来即可。

异常关机的情况下,各个模块没有机会flush数据,导致丢失部分内存状态信息。
这样,在下次启动的时候,需要执行恢复过程。

需要flush数据的模块有:
\begin{itemize}
    \item Wbuf
    \item GC
    \item Volume
\end{itemize}

提出几个问题:
\begin{tcolorbox}
\begin{enumerate}
    \item 正常关机时,需要flush什么信息?
    \item 恢复过程,从X恢复出Y,X是什么?Y是什么?(X是日志,Y是最新状态)
    \item 怎么理解提交等基础操作?
    \item 恢复的性能如何?如何通过检查点机制改善恢复性能?
\end{enumerate}
\end{tcolorbox}

针对以上问题,每个模块的恢复机制有所不同,但分析方法具有通用性。

\subsubsection{Volume Recovery}

 $U = (A - B) + C + D$

tail标记了可见空间,可见空间=已分配+可分配(free list)。free list组织成内存和磁盘两部分。flush时,需要持久化freelist的内存部分。

在调用malloc和free接口的时候,会同步更新用于空间管理的bitmap。为1的为已分配,为0的为可分配,这个关系总成立。

为了支持批量malloc和free接口,引入dirty page bitmap,类似于GC中提到的卡表,可以实现\textcolor{red}{多次更新,一次提交}的设计模式。

主要操作:
\begin{tcolorbox}
\begin{itemize}
    \item malloc操作:依次从C,D,U里取可用chunk。
    \item free操作:把释放的chunk放入C,如果C满,则转化为D。
\end{itemize}
\end{tcolorbox}

这里的提交操作可以理解为:C转化为D的过程,并没有记录检查点。
所以恢复操作,要全扫描bitmap,从bitmap重建C和D。

\subsubsection{GC Recovery}

GC recovery过程可以理解为:从gc bitmap重建内存状态。

所有的log,分为两部分:old storage和bitmap。bitmap相当于journalling。进入check queue的logctrl,先登记到bitmap。
在提交时,即从heap移入old storage时,清除/注销相应的bitmap项。

\subsubsection{Wbuf Recovery}

谁充当了日志的角色?在wbuf模块很明确,有专门的WAL。写入阶段登记,commit阶段回收。

\section{LSV测试}

LSV(\textcolor{red}{Log Structured Volume})基于Lich原生卷,实现了日志结构的卷格式,支持快照的各种操作。

相对于Lich原生卷,LSV有几点优势:
\begin{tcolorbox}
    \begin{itemize}
        \item 转化随机IO为顺序IO,混合存储情况下有更高性能
        \item 实现为ROW快照,zero-copy快照,\textcolor{red}{支持快照树}
    \end{itemize}
\end{tcolorbox}

LSV的关键过程:
\begin{tcolorbox}
    \begin{description}[style=nextline]
        \item [写] 写入wal和wbuf后,即可返回。wbuf积聚到1M时,提交log+bitmap后台异步任务。
        \item [读] 从wbuf读取最新数据,如果没有命中,则依次从rcache,bitmap+log读取。
        \item [GC] 垃圾回收,后台异步任务,按一定策略,回收无效页。
        \item [重启] 分两种情况,正常和异常情况。正常情况下会刷新内存状态,重启时直接加载即可;异常情况下,进入recovery过程。
    \end{description}
\end{tcolorbox}


LSV测试,主要分为功能,正确性和性能几个方面,\textcolor{red}{正确性和性能按标准测试用例}执行即可。下面列出一期测试计划。

\subsubsection{GIT分支}

lsv\_pipeline

\subsubsection{特性}

%\begin{tcolorbox}
\begin{lstlisting}[language=bash,frame=single]
# 创建LSV卷:
lichbd vol create p1/v1 --size 100Gi -F lsv -p iscsi

# 快照功能,与原来一样,部分命令示例:
lichbd snap create p1/v1@snap1 -p iscsi
lichbd snap create p1/v1@snap2 -p iscsi
lichbd snap ls p1/v1 -p iscsi

# 暂不支持flat操作

\end{lstlisting}
%\end{tcolorbox}

\subsubsection{性能/正确性测试清单}

\begin{itemize}
    \item 与lich原生卷全面的性能对比
    \item 资源利用率(包括磁盘,内存)
    \item 系统启动时间
    \item 重启系统的恢复过程
    \item 存储分层
    \item 扩展到多卷
\end{itemize}

\subsubsection{注意事项}

\begin{itemize}
    \item 日志满:/opt/fusionstack/log/lich.log (echo 5 > /dev/shm/lich4/msgctl/level)
\end{itemize}



\part{生态系统}

\chapter{SNMP}

各种MIB,\hl{有rfc定义的},有厂商私有的

trap

怎么知道一个交换机支持什么OID?用walk遍历所有。

工具
\begin{enumbox}
\item iReasoning MIB Browser
\item pysnmp
\end{enumbox}

MIB文件的位置:
\begin{enumbox}
\item /usr/share/mibs/ietf -> /var/lib/mibs/ietf
\item /usr/share/mibs/iana -> /var/lib/mibs/iana
\item SNMPv2-MIB
\item HOST-RESOURCES-MIB
\item IF-MIB
\item ENTITY-MIB
\end{enumbox}

\include{other/lvm}
\include{other/tgt}
\chapter{CEPH}

从硬件,存储引擎,存储服务,接口,管理等层次分析一款存储产品。评价指标:多快好省。
多是空间,是扩展性,快是性能,是时间。好是品质,可靠性,高可用等;省,在标准下的优化措施。
两个维度,构成纵横交错的矩阵结构。任何一项功能,都服务于一个或多个指标,
都需从多快好省几个维度去分析(雷达图),只是有所偏重。

硬件层:

引擎层:分布,复制/EC,恢复,平衡

模型层:block/file/object pool,卷,快照,一致性组(服务层是模型中实体的属性和操作)

服务层:瘦配置,快照,克隆,cache,分级,QoS,备份,灾备,dedup,压缩,加密(缩减,保护,隔离,安全)

接口层:iscsi

存储池,统一了保护域,故障域,pool, 缓存,EC等,都是通过灵活的存储池配置实现的。

存储池关联到ruleset,ruleset指定了bucket和设备的存取规则。

osd对应到磁盘设备,bucket是容器,构成分层的物理拓扑结构。

\chapter{Nutanix}

zookeeper

cassandra

数据和I/O本地化

ROW快照和克隆

DR

curator的map-reduce框架,用于处理磁盘平衡、恢复、离线EC与离线消重等分布式任务。

区分IO,大的顺序的与小的随机的,使用不同的策略,包括IO路径、压缩策略等。

冷热数据

区分,一阴一阳

%\include{sheepdog}
%\include{openvstorage}
%\include{VSAN}

\part{PBL}

\chapter{操作系统}

资源管理,架起硬件和上层应用之间的桥梁。资源的3+1模型:cpu、memory、disk and network。

\section{CPU}

从单核cpu说起,多核cpu引入了哪些关键问题?

进程、线程、协程调度

通信和同步

经典问题:
\begin{compactenum}
\item 生产者-消费者问题
\item 读者-写者问题
\item 哲学家就餐问题
\end{compactenum}

\section{Memory}

\section{IO和文件系统}

\section{网络}

\section{分布式系统}

\chapter{FusionStor}

数据块及其副本位置是最重要的问题,分配、回收、恢复、平衡、GC都针对数据分布而言。
维护副本数据一致性,EC的情况会更复杂些,需要维护条带数据一致性。

IO路径是主线。

物理资源层次:集群、节点、磁盘。逻辑资源层次:集群、pool、目录、卷+快照。
其中,存储池即是物理资源的划分,又是逻辑卷资源的容器。
这是一个形如X的复用-分用的组织模式。

\section{High Performance}

应从硬件与软件两个层面去理解。

高性能版本意味着什么?首先是硬件的升级,包括网络、存储介质、协议。

对软件架构而言,意味着什么?主要涉及操作系统如何管理资源,访问效率如何。

代码优化

社区在做什么?大家在做什么?趋势是什么?

京东云也做RDMA,六人三个月。

存储有着广泛的标准组织,有什么重要的项目?

关注上下、左右各方面的情况。

\section{移动集采}

\hl{从移动集采项目,看分布式块存储的知识体系}。

分布式块系统面向四个方面:底层资源、主机、运维平台和云平台。
面向主机提供了iSCSI协议,外加vaai,
面向运维平台提供了SNMP trap,REST,CLI等接口。
面向云平台提供了cinder驱动。

面向主机是本质的方面,数据平面;其它面向是控制平面的事情。上下左右,四面八方。

\section{统一存储}

华为出版的\hl{数据存储技术与实践}一书,分为四个板块:企业存储、云存储、数据库、大数据。

统一存储相对于传统阵列来说的,统一了SAN与NAS,另加上对象存储。并不包括数据库、大数据等存储形式。
\hl{NAS网关}是在块的前面加上一层NAS接口组合而成。

分布式架构是相对于传统阵列而言,软件定义存储,智能存储。

云存储可以从openstack的cinder驱动开始理解,cinder提供了标准接口,封装了下层不同的存储产品。
另外是VMWare的VVol,反而不常用。云存储的关注点在云平台下的存储需求,如京东云硬盘组等等。

云平台与块存储的对接,构成了HCI架构。qumu/kvm如何访问块存储?通过rbd协议,或iSCSI等标准协议。
qemu与kvm是如何整合的?docker等容器技术与vm的优缺点,从而规定了不同的适用场景。


\part{理论基础}

\chapter{learning}

\section{学习方法}

查理芒格的模型:学科的重要模型。

数学概观:现实-模型-理论三元组。模型是对现实的抽象,把逻辑运用到模型推演出理论体系。
通过一问一答解决现实问题。模型的验证,一是事实,而是逻辑。

找到某些基础模型、或如高焕堂老师说的:form。作为构建更复杂系统的基本单元,
有助于达成以简御繁的目的。

化整为零

临摹

师法造化,内得心源。

\subsection{包围式学习}

E=K/I,温故知新,通过包围式学习构建知识网络。这是主动的开疆辟土,步步为营。

学习、学问这些词汇,有很深内涵,回到中庸的论述。

学的是思维方式和方法论,习是刻意练习,体用、知行、道器一体。
器是作品集。

\subsection{戒定慧}

六度架起此岸、彼岸的桥梁。

\begin{enumbox}
\item 11点之前睡觉、六点起床
\item 问题驱动
\item 一心二本
\end{enumbox}

\subsection{守破离}

以算法为中心,贯通多个领域。

% https://en.wikipedia.org/wiki/Algorithm#Informal\_definition

\section{学习计划}

知识体系:编程语言、数据结构和算法、架构和系统。
最重要的系统有:操作系统、编译原理和数据库。

分布式存储系统能把这些知识点贯通起来。

\begin{itemize}
    \item SCSI
    \item NVMe
    \item SPDK
\end{itemize}

\section{学习资料-书籍}

\subsection{老马识途}

斩码三刀:猜测-实证-建构

阅读源码的方法:调试-阅读-调试。以调试为方法,动起来,把握主线。所以要熟练掌握gdb等工具。

逆向工程是匕首,答疑解惑。

主要目的在于培养系统观,不以记住知识为高,而以培养系统观为能,就是学会学习的方法。

\subsection{算法导论}

\subsection{伟大的计算原理}

六类计算原理:计算、存储、通信、协作、评估、设计。

算法、架构、设计三部曲。

架构:高可用、高性能、负载均衡。

设计准则:需求、正确性、容错性、时效性、适用性。
软件系统的设计原理:层级式聚合、层级、封装、虚拟机、对象、C/S。

\subsection{完美软件设计}

设计的几个原则:
\begin{enumbox}
\item 正交
\item 分层
\item 时序下的数据流
\item 封装
\item 名实
\end{enumbox}

\subsection{设计原本}

\subsection{计算机程序的构造与解释}

\section{Paper}


% basic
\chapter{Tools}

\section{Deployment}

\subsection{ntpdate}

\mygraphics{../imgs/tool/ntp-date.png}

\subsection{ansible}

\section{Development}

\subsection{cmake}

\mygraphics{../imgs/tool/cmake-link-static.png}

生成静态库
\begin{myeasylist}{itemize}
& SHARED  -> STATIC
& LIBRARY -> ARCHIVE
\end{myeasylist}

\subsection{gdb}

\begin{myeasylist}{itemize}
& ~/.gdbinit
& info registers
& info sharedlibrary
& gdb -p
\end{myeasylist}

gdb -p发现了mbuffer\_writefile进入死循环,原因是count==0。

猜想是重入了一个锁。

\subsection{debug}

trace msgid来跟踪消息流。

\subsection{wireshark}

\subsection{SoftRoce}

spdk/scripts/setup.sh

\section{Test}

\subsection{fio}

\subsection{spdk/perf}

\subsection{hazard}

\mygraphics{../imgs/tool/hazard-1.jpeg}
\mygraphics{../imgs/tool/hazard-2.jpeg}
\mygraphics{../imgs/tool/hazard-3.jpeg}

\chapter{设计}

\section{原则}

策略和机制分离

交叉验证

错误处理

\section{输出}

%\chapter{软件架构}

meta管理采用对称的中心化架构。在节点中选举出admin节点,
管理全局状态和数据分配工作。当admin节点发生故障时,会发生
failover过程,选举出新的admin节点来。

lichd进程内嵌各种server,包括iscsi等。

作为存储系统,主要考虑是元数据组织和IO,恢复等关键过程。
功能之外,可靠性、性能,可扩展性至关重要。

client可以和每个节点进行通信,推荐采用VIP机制,简化连接管理。

\section{架构演化}

\begin{enumbox}
\item{引入etcd}
\item{引入存储池}
\item{引入Bcache}
\item{引入VDO}
\item{引入ROW3}
\end{enumbox}

\section{节点}

两类服务器节点:
\begin{compactenum}
\item admin
\item normal
\end{compactenum}

从meta节点中选举出admin。meta节点是静态指定的吗?meta节点列表构成一个小的集群,类似于ceph的monitor。

normal节点是数据节点,存放元数据和数据。元数据和数据都是按1M的chunk组织。
元数据包括四类chunk: pool,subpool,vol,subvol。每个chunk具有固定数目的槽位bucket,指向管理的下一级节点。
集群内的所有chunk,构成一个单根树。每个pool,每个volume,都是这个大树下的子树。

引导信息bootstrap:rootable。记录了根分区的位置,即所在chunk的位置信息。根分区是一个特殊的pool。

所有的控制器,包括pool和volume控制器,通过lease机制保证集群内的唯一性。
其节点位置,由所在子树的根chunk的主副本确定。

chunk副本的节点分布:基于diskmap的随机分布算法,并记录在元数据里。同时,遵循故障域规则。

chunk副本的磁盘分布:本地数据管理(bitmap+sqlite)

chunk副本之间的一致性:强一致性协议。在每次IO操作前,检查各副本的clock和dirty状态,必要的情况下,进行修复。

为了提升性能,需要充分考虑聚合和并发。聚合优于并发,先考虑聚合/批处理。
聚合,一次提交多个chunk。
并发的粒度,多个chunk,一个chunk的多个副本。同时,锁的粒度要恰如其分。

副本数据的管理,没有采用资源池模式,与chkid紧耦合,非共享,不利于分配和释放。

\subsection{admin节点}

职责:
\begin{compactenum}
\item paxos leader
\item lease server
\item 管理系统引导信息
\item 集群节点列表
\item 维护diskmap
\item VIP(控制器当前位置)
\item 分配卷Id
\item 分配\hl{chunk副本位置}
\end{compactenum}

持久化信息和上报信息,心跳机制

可扩展性

admin上持久化的信息,\hl{是如何避免单点故障的,即如何同步到各个meta节点上的}?
可以rootable为例,参见\verb|mq_master.c|。

引入etcd后,admin处理逻辑会简单些。同时引入了多存储池,寻址过程变得复杂。

\subsection{数据目录}

/opt/fusionstack/data:
\begin{itembox}
\item node/rootable (\hl{global info})
    \begin{compactitem}
    \item sysroot/root
    \item misc/fileid
    \item node/t53
    \end{compactitem}
\item disk (local, disk management)
\item chunk (local, sqlite3)
\item status
\end{itembox}

\section{元数据管理}

元数据包括:
\begin{enumbox}
\item \hl{引导信息,用于加载相应对象}
\item 目录下包含哪些文件?
\item 文件包含哪些chunk?
\item chunk的副本位置(节点和节点上的磁盘)
\item 快照以及快照树
\item xattr
\item 各实体对象描述信息,包括创建时间,id等
\end{enumbox}

关键问题有:
\begin{enumbox}
\item 空间管理,分配和回收
\item 数据分布
\item 复制/EC
\item 数据恢复和平衡
\end{enumbox}

在顶层设计下的case by case。

每一个卷的元数据和数据,共有三层chunk,L1和L2是元数据,L3是数据,构成chunk的单根树。
快照和clone出的卷采用统一的chunk树结构,不过增加了交叉引用关系。

L1原先只有一个chunk,有固定数目的槽位,指向L2的chunk,L2的chunk,有固定条目的槽位,指向L3的数据chunk。

table1包含了指向全部L2 chunk的指针数组(table\_proto),table2包含了指向全部L3 chunk的指针数组(chkinfo+chkstat)
table\_proto内在也包含chkinfo和chkstat。每一个chunk,都需要在其父节点上登记,\hl{卷的第一个chunk}登记在pool里。

parent的界定:\hl{raw和subvol的parent都是volume的chkid,subpool的parent是pool的chkid。
vol的parent是pool的chkid},即parent都是可寻址的实体对象(具有控制器)。\hl{sqlite记录的parent遵循该语义}。

如果\verb|table1->table_count|和\verb|table2->chknum|比较大,会有扩展性和性能方面的问题,同时会消耗比较多的内存。

\section{VIP}

\section{微控制器}

每一个加载的卷,对应一个集群内唯一的卷控制器,负责卷的IO等操作。

\section{数据分布}

数据分布首先要满足规则要求,其次则需做到均衡和局部性。
规则是强制的,极端条件下可能退化。均衡和局部性则影响系统性能。

卷属于存储池,存储池上定义副本数和副本放置规则。
保护域和故障域,可以用存储池来统一。

FusionStor通过元数据来管理,\hl{与CEPH的CRUSH有重大不同}。

数据分布规则有:
\begin{compactitem}
\item 存储池规则
% \item 保护域规则
\item 故障域规则(\ref{rule:faultset})
\end{compactitem}

负载均衡和本地化两方面考虑,平衡包括数据平衡和任务平衡。

chunk在节点上的分布,节点内chunk在磁盘上的分布(包括分层)

controller在节点上的分布,controller在core上的分布

各种任务的分布情况,如数据恢复。

\subsection{负载均衡}

\subsection{本地化}

卷控制器所在节点,具有所有chunk的副本。

当切换控制器的时候,需要控制本地化过程的QoS。

\section{复制一致性}

\begin{tabular}{|s|p{6cm}|}
    %\hline
    %\rowcolor{lightgray} \multicolumn{5}{|c|} {Snapshot} \\
    \hline
    pool & LICH\_REPLICA\_METADATA  \\
    \hline
    subpool & LICH\_REPLICA\_METADATA  \\
    \hline
    vol & fileinfo \\
    \hline
    subvol & fileinfo  \\
    \hline
    raw & fileinfo  \\
    \hline
\end{tabular}

恢复和再平衡过程,是怎么工作的?

控制器协调所有IO操作,通过(owner, magic, clock)维护副本一致性。
owner,magic联合起来,处理controller切换的情况。
clock处理一个chunk内所有变更的顺序。

若干问题:
\begin{enumbox}
\item 因为有owner和magic,是否不需要再维护controller的唯一性?
\item 没有使用日志
\end{enumbox}

\subsection{正常情况}

chunk内的更新按clock顺序提交,这样做影响并发度,进一步影响到性能。
如果不遵循clock序提交,需要有一假设:所有并发任务无重叠,或者是在有重叠的地方按clock串行化。
上层应用不会提交重叠的IO,内部元数据涉及小IO,比如小于512B的IO,多个更新可能指向同一扇区。
如果多个副本不按同样的顺序,就会出现数据一致性问题。

如果收到clock连续的io,且这些io无重叠部分,可以直接提交。
如果io有重叠,按clock序依次提交。采用lattice,可以实施聚合优化。

一种做法是保证并发io无重叠。需要分析每一种io情况,包括应用层io,内部元数据等。
另一种做法是维护一数据结构,能够跟踪io重叠情况。

\subsection{故障情况}

控制器切换

在降级写的时候,没有参与的副本标记持久化状态stale,保证在reload的时候,不会被选为权威副本。

网络分区

节点掉电

集群掉电

disk and raid cache

\section{精简配置}

每个raw chunk有三者状态:ENOENT,alloc,alloc and zero。第一种和第三种等价。

分配一个chunk,开销较大,影响到精简配置和快照的性能。
分配一个chunk,涉及到更新元数据。

快照、克隆卷天然就是精简配置的。

磁盘空间分配器,db记录chkid到磁盘位置的映射(采用rocksdb?)。

\section{分层}

卷的xattr,有目标分层设定:priority。默认是-1,即开启自动分层。先落入tier 0,根据数据热度,
通过异步过程进行数据迁移(纵向的数据流动)。所以有两个异步任务:
\begin{compactenum}
\item 统计访问热度
\item 按策略执行数据迁移(目标分层的偏离,数据热度)
\end{compactenum}

每个chunk有实际分层:tier,不是在副本级。

\section{SSD缓存}

SSD Cache实现了写缓存,没有实现读缓存, 通过内存实现读缓存。

\subsection{配置项及系统行为}

相关控制参数:
\begin{enumbox}
\item lich.conf/disk\_keep: 10G (废弃)
\item lich.conf/disk\_cache: 10G
\item 卷的xattr: writeback
\item 卷的属性:priority (设定卷的目标存储分层)
\end{enumbox}

disk\_cache配置磁盘cache区,对新盘有效,若一个盘已经配置,就固定了,不再改变。
此配置对SSD和HDD都起作用,预留磁盘末尾的空间。

priority是手工分层机制,持久化到卷属性上。

tier是自动分层机制,默认行为是什么? 优先落到SSD上,若SSD满,落到HDD上。

所以,分三种情况:
\begin{compactenum}
\item 自动分层,priority == -1,优先落盘到SSD上,如果SSD满,落盘到HDD上
\item priority == 0, 同自动分层机制
\item priority == 1,落盘到HDD上(此时根据writeback的设置,决定是否走SSD cache)
\end{compactenum}

写IO流程

以下两种情况,会落盘到HDD上:
\begin{compactenum}
\item priority == 1
\item SSD满
\end{compactenum}

落盘到HDD的写IO,\hl{其行为受xattr.writeback影响},分两种情况:
\begin{compactenum}
\item xattr.writeback == 1,写入ssd的cache区+内存后,返回。
\item xattr.writeback == 0, 写入HDD。
\end{compactenum}

落盘到SSD的写IO,不经过Cache, 走原来的IO路径。
\begin{compactenum}
\item 自动分层,会优先落到SSD上
\item priority == 0, 会导致落盘到SSD上
\end{compactenum}

已知问题

\begin{compactenum}
\item 原来的分层机制,默认情况下,会写满SSD,导致后续写HDD(先快后慢)。
\item 异步线程的工作机制 (周期性执行,20min执行一次, 最后才会置换proirity == 0的数据)
\item disk cache区开启后,就不能再改变,空间无法回收,且不能关闭(不可逆过程)
\end{compactenum}

不确定的地方

考虑以下问题,初次分配dispatch\_newdisk,等固定了分层后,后续的变更规则是什么?

\section{分区容忍性}

参考fence.c

\section{关键过程}

\subsection{启动集群}

\subsection{创建存储池}

创建存储池后,会在etcd上记录引导信息。
创建存储池后,必须添加磁盘到该存储池。一块磁盘只能属于一个存储池。

\subsubsection{磁盘管理}

每块磁盘对应一个bitmap,用于该盘的空间管理。

磁盘有分层属性,通常0表示SSD,1表示HDD。有三种分层策略:tier==0,表示写入SSD,tier==1表示写入HDD,
tier==-1表示自动分层,先写入SSD,通过异步后台线程flush不活跃的数据到HDD。

在分配每一个chunk的时候,可以指定tier。没有指定的情况下,默认为卷的priority设定。

chunk\_id到磁盘物理地址的映射,是一随机过程,定位到空闲的bitmap上。

如何确定磁盘的分层?

RAID管理,disk和raid都有cache,需要注意掉电情况下是否丢数据。

\subsection{创建卷}

\subsection{IO过程}

写过程,可能内在地包含了分配chunk的过程,缺页分配。当在末尾写入时,还可能扩展了卷的大小。

大范围内的随机写入,造成很多的缺页分配,分配过程会成为性能瓶颈。

\subsection{分配一个chunk的过程}

分为两阶段:分配空间,chkid和磁盘位置的映射。两阶段按SEDA方式组织,没按pipeline组织。

分配空间线程池:每个磁盘一个线程,多线程共享任务队列。


函数:
\begin{compactitem}
\item \verb|__table2_chunk_create|
\item \verb|replica_srv_create|
\item \verb|disk_create|
\end{compactitem}

与admin交互,返回节点列表,即各副本所在节点。

需要持久化的信息:
\begin{compactitem}
\item disk bitmap,记录磁盘上每个chunk的分配状态
\item sqlite3,记录chkid(副本)到物理地址的映射关系
\item table2 meta,记录chunk info(副本位置)
\item 填充chunk内容为全0?
\end{compactitem}

分配chunk的过程,会影响到若干特性,如精简配置,快照、恢复,再平衡,写入等,都产生新chunk。

优化allocate的性能:
\begin{compactitem}
\item 加大lich.inspect线程数到20
\item table2 lock粒度 \change{dynamical lock table}
\item 异步化sqlite,每个db一个工作线程
\end{compactitem}

\subsection{事务处理}

一个复杂的主题是事务处理,即如何保证在故障条件下,过程执行的ACID属性。
需要做出事务保证的典型过程有:
\begin{compactenum}
\item 创建卷(metadata表出现垃圾记录)
\item 创建快照(snap\_version得不到有效维护)
\item 分配chunk(meta和副本不一致)
\end{compactenum}

以分配chunk为例,基本操作有:
\begin{compactenum}
\item 申请chkid和chkinfo
\item 分配disk bitmap
\item 分配sqlite记录
\item 写入meta
\end{compactenum}

任何两个操作之间发生故障,都会导致问题。

\subsection{异步过程:删除卷}

\subsection{异步过程:删除快照}

\subsection{异步过程:回滚快照}

\subsection{异步过程:flat快照}

\subsection{异步过程:数据恢复}

\section{高性能}

网络层:iSER,NVMf

设备层:libnvme/SPDK

libiscsi

tgt

\chapter{数据结构}

\section{Array}

\section{List}

\section{Stack}

\section{Queue}

scheduler

\section{Hash}

\section{Heap}

\section{Tree}

\subsection{Radix Tree}

\section{Graph}

\section{Bit}

\chapter{算法}

\section{概述}

人生需要核心算法。核心算法解决人生会遇到的大问题,最重要的一个问题就是如何更好的成长。

\subsection{模型}

顶级思维模型。

处处可见圆点哲学的影子。矛盾分析法,双线法则,两眼论,
一阴一阳之谓道,万物负阴抱阳冲气以为和。

回到核心,从核心出发。找到核心的过程,一靠直觉,二靠试错,低成本地试错。

老子第十六章,义理丰富。

以正治国,以奇用兵,以无事取天下。此意渊深,可为座右铭。

孙子兵法提供了一套方案,孙正义有自己的归纳总结。我欲清溪寻鬼谷,不论礼乐但论兵。
兵者是死生存亡的大事,社会暗流涌动,在温泉面纱下,竞争不可谓不激烈。
想要活出自己的人生,不能不考虑更重要的维度。

孙陶然五行管理兵法。

达里奥原则,培根新工具,笛卡尔方法谈,斯宾诺莎伦理学都旨在解决人生算法问题。

开发人生算法,喻颖正做出了表率。按守破离的节奏,先对标,再突破。最小核心最大化,
先要感知自己的核心,通过反复练习,使之价值最大化。

高筑墙,广积粮,缓称王。

\section{战略要素}

\subsection{目标}

第一,列出最重要的五个目标。双列表,10/10/10原则。

事业有成是因,财务自由是果。找到自己的核心算法,其它一切则水到渠成。
反复打磨核心,可以用爱因斯坦质能方程来描述:E=mc\^2。m是核心,c是大量重复练习,E是果。

\subsection{资源}

整合资源:客户,钱,人脉,客户。

\section{将略}

慎言!养成深沉厚重之心态。


\chapter{编程语言}

体系结构、操作系统和汇编语言是底层逻辑。

编程语言通常提供了比较完备的数据结构和算法库,可以结合起来进行学习。

\section{C++}

代码组织方式:file, namespace, exception

\hl{function, class, template}是三个核心概念。class是data type的扩展,template作用于function和class之上。

value、pointer、reference。reference又分为左值和右值。右值引用用来实现move语义,避免不必要的数据copy。
这与mbuffer等zero-copy技术类似。

什么是左值,什么是右值呢?可以取地址的是左值,常数、函数返回的临时变量是右值。


% system
\chapter{计算机体系结构}

组成原理与体系结构,研究内容相当。

\section{结构}

三部分:CPU,存储器,I/O设备。PCIe和NUMA都采用分布式架构,PCIe是串行点对点,
NUMA分远和近,访问远程CPU慢。可以称为分布式共享内存。

二进制

进制之间的转换
\begin{enumbox}
\item 整数(除2取余)
\item 负数(取反加1)
\item 小数(乘2取整)
\item 浮点数采用科学计数法,分单精度与双精度
\item 高位是符号位
\item 表示范围不变,坐标平移
\end{enumbox}

字节顺序:32bit机器上字长是4B,64bit机器上字长是8B。
一个字的字节采用由左到右编码,或从右到左编码。
big-endian表示尾端在高位,little-endian表示尾端在低位。

\section{执行}

一颗CPU的执行可以看着一维的指令流,由顺序和跳转指令,由调度程序模拟多任务的假象。
并发执行需要进行同步,以避免引起的问题,保证数据的一致性和安全性、执行结果的准确性。

基于lock的同步,可能会引起死锁或饥饿现象,须加以避免、检测。

进程的五状态模型,进程调度算法

流水线并行示意图

协同过程,又称协程,控制权转移,可以不从函数开始处执行。

指令执行,需要在寄存器与主存间同步数据。
主存的值,须加载到寄存器。\hl{CPU指令能直接访问主存内容吗?}

物理内存大小固定,需要虚拟化,供多个进程使用。进程的地址空间彼此隔离。
由页表和TLB执行逻辑地址到物理地址的映射。
\hl{物理地址如何管理?分配和回收是否需要free list或bitmap一样的索引结构?}

Hugepage可以减小页表的大小,加快检索性能。

进程空间分用户空间和kernel空间。用户进程空间分段管理:text,data,bss,heap,stack。

进程-进程空间-内存-I/O是核心概念,是理解的主要节点。物理内存不足时,可以借用部分磁盘swap空间,
但这种情况会引起频繁的swap in/out,造成性能抖动。

程序的执行过程:汇编、链接、加载。加载后生成一进程,参与调度。

\chapter{操作系统}

资源管理,架起硬件和上层应用之间的桥梁。资源的3+1模型:cpu、memory、disk and network。

\section{CPU}

从单核cpu说起,多核cpu引入了哪些关键问题?

进程、线程、协程调度

通信和同步

经典问题:
\begin{compactenum}
\item 生产者-消费者问题
\item 读者-写者问题
\item 哲学家就餐问题
\end{compactenum}

\section{Memory}

\section{IO和文件系统}

\section{网络}

\section{分布式系统}

\chapter{Linux Kernel}

操千曲而后晓声,观千剑而后知器。


\chapter{文件系统}

\section{Fuse}

在mount的时候,向kernel注册handler。

\section{GlusterFS}

\chapter{数据库系统}

% \include{basic/compiler}

% design
\include{basic/distributed}
\chapter{并发控制}

编程模型:多线程和事件驱动。多线程访问共享内存,需要同步机制。锁会形成死锁。
事件驱动有主循环,不能调用block操作,另外涉及如何动态扩展到多核多CPU的环境。

actor编程模型,SEDA是一个特殊的actor架构。

进程及其通信是基础模型。执行的代换模型与环境模型,事务的页模型与对象模型,
由简单模型推导关键结论。

并行有两种:竞争并行与协作并行。

\section{PCAM设计方法学}

\begin{enumbox}
\item 划分 数据划分、任务划分
\item 通信 任务之间的数据交换
\item 聚合 合并任务,以提升性能
\item 映射 任务调度,并符合负载均衡
\end{enumbox}

\subsection{Example:volume}

volume是一个自然划分,把volume映射到一个core thread上。
问题是这样做的话,粒度过大,无法有效进行负载均衡,受限于单core最大处理能力。

如果一个subvol映射到一个core thread,粒度合适。但需要引入复杂的控制机制。
最好是映射到一个节点的多个core上。

以上分析过程用到了完整的PCAM方法。

\subsection{Example:recovery}

\subsection{Example:rmvol}

无需通过controller去做这些事情,可以直接利用db信息。

\subsection{Example:rollback}

\subsection{Example:flatten}

\section{同步原语}

同步原语组成层次结构。并发原语,层层还原,层层构建

从OS到DB,对并发的讨论在深化,事务要求ACID,且访问特定的索引结构。
另外还有持久化的要求,原子写变得更困难,引入日志。

本质上是序的问题,锁内在地实现了一种序。
逻辑时钟、paxos、raft中,序起到了至关重要的作用。

\chapter{Performance Tuning}

\section{高性能架构}

从软件与硬件多个层面去梳理整个问题。

首先看资源、架构、算法,以及评估标准,或指标,接着进行设计方法、模式的研究。

\section{学习资料}

阅读yufeng,何登成blog。

\begin{enumbox}
\item context switch 
\item Lock 
\item memory malloc
\item data copy
\item System call 
\end{enumbox}

理论
\begin{enumbox}
\item USE方法
\item 算法分析
\item 排队论
\end{enumbox}

Tool
\begin{enumbox}
\item top
\item iotop
\item slabtop
\item free
\item vmstat
\item iostat
\item dstat
\item pidstat
\item *
\item tcpdump
\item netstat
\item mpstat
\item tcprstat
\item nicstat
\item *
\item strace
\item blktrace
\item stap
\item perf
\item oprofile
\item stap
\item lmbench
\end{enumbox}


% CBA
\chapter{大数据}


\chapter{人工智能}


\chapter{FlowChart}

Some of the greatest \emph{discoveries} in science werer made by accident.

\textit{Some of the greatest \emph{discoveries} in science were made by accident.}

\textbf{Some of the greatest \emph{discoveries} in science were made by accident.}

In physics, the mass-energy equivalence is stated by the equation $E=mc^2$, discovered
in 1905 by Albert Einstein.

The mass-energy equivalence is described by the famous equation 

$$E=mc^2$$

discovered in 1905 by Albert Einstein.
In natural units($c = 1$), the formula expresses the identity

\begin{equation}
E=m
\end{equation}

Subscripts in math mode are written as $a_b$ and superscripts are written as $a^b$. 
These can be combined an nested to write expressions such as

$$T^{i_1 i_2 \dots i_p}_{j_1 j_2 \dots j_q} = T(x^{i_1},\dots,x^{i_p},e_{j_1},\dots,e_{j_q})$$

We write integrals using $\int$ and fractions using $\frac{a}{b}$. 
Limits are placed on integrals using superscripts and subscripts:

$$\int_0^1 \frac{1}{e^x} =  \frac{e-1}{e}$$

Lower case Greek letters are written as $\omega$ $\delta$ etc. 
while upper case Greek letters are written as $\Omega$ $\Delta$.

Mathematical operators are prefixed with a backslash as $\sin(\beta)$, $\cos(\alpha)$, $\log(x)$ etc.

% 流程图定义基本形状
\tikzstyle{startstop} = [rectangle, rounded corners, minimum width=3cm, minimum height=1cm,text centered, draw=black, fill=red!30]
\tikzstyle{io} = [trapezium, trapezium left angle=70, trapezium right angle=110, minimum width=3cm, minimum height=1cm, text centered, draw=black, fill=blue!30]
\tikzstyle{process} = [rectangle, minimum width=3cm, minimum height=1cm, text centered, draw=black, fill=orange!30]
\tikzstyle{decision} = [diamond, minimum width=3cm, minimum height=1cm, text centered, draw=black, fill=green!30]
\tikzstyle{arrow} = [thick,->,>=stealth]

\begin{tikzpicture}[node distance=2cm]
%定义流程图具体形状
\node (start) [startstop] {Start};
\node (in1) [io, below of=start] {Input};
\node (pro1) [process, below of=in1] {Process 1};
\node (dec1) [decision, below of=pro1, yshift=-0.5cm] {Decision 1};
\node (pro2a) [process, below of=dec1, yshift=-0.5cm] {Process 2a};
\node (pro2b) [process, right of=dec1, xshift=2cm] {Process 2b};
\node (out1) [io, below of=pro2a] {Output};
\node (stop) [startstop, below of=out1] {Stop};

%连接具体形状
\draw [arrow](start) -- (in1);
\draw [arrow](in1) -- (pro1);
\draw [arrow](pro1) -- (dec1);
\draw [arrow](dec1) -- (pro2a);
\draw [arrow](dec1) -- (pro2b);
\draw [arrow](dec1) -- node[anchor=east] {yes} (pro2a);
\draw [arrow](dec1) -- node[anchor=south] {no} (pro2b);
\draw [arrow](pro2b) |- (pro1);
\draw [arrow](pro2a) -- (out1);
\draw [arrow](out1) -- (stop);
\end{tikzpicture}

%作者:红伞菌
%链接:https://www.zhihu.com/question/20854046/answer/16400909
%来源:知乎
%著作权归作者所有。商业转载请联系作者获得授权,非商业转载请注明出处。
%作者:红伞菌
%链接:https://www.zhihu.com/question/20854046/answer/16400909
%来源:知乎
%著作权归作者所有。商业转载请联系作者获得授权,非商业转载请注明出处。

% Defines a `datastore' shape for use in DFDs.  This inherits from a
% rectangle and only draws two horizontal lines.
\makeatletter
\pgfdeclareshape{datastore}{
  \inheritsavedanchors[from=rectangle]
  \inheritanchorborder[from=rectangle]
  \inheritanchor[from=rectangle]{center}
  \inheritanchor[from=rectangle]{base}
  \inheritanchor[from=rectangle]{north}
  \inheritanchor[from=rectangle]{north east}
  \inheritanchor[from=rectangle]{east}
  \inheritanchor[from=rectangle]{south east}
  \inheritanchor[from=rectangle]{south}
  \inheritanchor[from=rectangle]{south west}
  \inheritanchor[from=rectangle]{west}
  \inheritanchor[from=rectangle]{north west}
  \backgroundpath{
    %  store lower right in xa/ya and upper right in xb/yb
    \southwest \pgf@xa=\pgf@x \pgf@ya=\pgf@y
    \northeast \pgf@xb=\pgf@x \pgf@yb=\pgf@y
    \pgfpathmoveto{\pgfpoint{\pgf@xa}{\pgf@ya}}
    \pgfpathlineto{\pgfpoint{\pgf@xb}{\pgf@ya}}
    \pgfpathmoveto{\pgfpoint{\pgf@xa}{\pgf@yb}}
    \pgfpathlineto{\pgfpoint{\pgf@xb}{\pgf@yb}}
 }
}
\makeatother
\begin{center}
\begin{tikzpicture}[
  font=\sffamily,
  every matrix/.style={ampersand replacement=\&,column sep=2cm,row sep=2cm},
  source/.style={draw,thick,rounded corners,fill=yellow!20,inner sep=.3cm},
  process/.style={draw,thick,circle,fill=blue!20},
  sink/.style={source,fill=green!20},
  datastore/.style={draw,very thick,shape=datastore,inner sep=.3cm},
  dots/.style={gray,scale=2},
  to/.style={->,>=stealth',shorten >=1pt,semithick,font=\sffamily\footnotesize},
  every node/.style={align=center}]

  % Position the nodes using a matrix layout
  \matrix{
    \node[source] (hisparcbox) {electronics};
      \& \node[process] (daq) {DAQ}; \& \\

    \& \node[datastore] (buffer) {buffer}; \& \\

    \node[datastore] (storage) {storage};
      \& \node[process] (monitor) {monitor};
      \& \node[sink] (datastore) {datastore}; \\
  };

  % Draw the arrows between the nodes and label them.
  \draw[to] (hisparcbox) -- node[midway,above] {raw events}
      node[midway,below] {level 0} (daq);
  \draw[to] (daq) -- node[midway,right] {raw event data\\level 1} (buffer);
  \draw[to] (buffer) --
      node[midway,right] {raw event data\\level 1} (monitor);
  \draw[to] (monitor) to[bend right=50] node[midway,above] {events}
      node[midway,below] {level 1} (storage);
  \draw[to] (storage) to[bend right=50] node[midway,above] {events}
      node[midway,below] {level 1} (monitor);
  \draw[to] (monitor) -- node[midway,above] {events}
      node[midway,below] {level 1} (datastore);
\end{tikzpicture}
\end{center}

\begin{figure}[htb]
\centering
%定义形状样式
\tikzstyle{startstop} = [rectangle, rounded corners, minimum width = 3cm, minimum height = 0.7cm, text centered, draw = black]
\tikzstyle{startstop2} = [rectangle, rounded corners, minimum width = 13cm, minimum height = 0.7cm, text centered, draw = black]
\tikzstyle{io} = [trapezium, trapezium left angle = 30, trapezium right angle = 150, minimum width = 3cm, text centered, draw = black, fill = white]
\tikzstyle{io2} = [trapezium, trapezium left angle = 30, trapezium right angle = 150, minimum width = 2.5cm, draw = black, fill = white]
\tikzstyle{io3} = [trapezium, trapezium left angle = 30, trapezium right angle = 150, minimum width = 2cm, draw = black, fill = white]
\tikzstyle{process} = [rectangle, minimum width = 3cm, minimum height = 1cm, text centered, draw = black]
\tikzstyle{decision} = [diamond, minimum width = 3cm, minimum height = 1cm, text centered, draw = black]
\tikzstyle{arrow} = [thick, -, >= stealth]
\tikzstyle{arrow2} = [thick, ->, >= stealth]

\begin{tikzpicture}[node distance = 1.5cm]
% 定义流程图具体形状
\coordinate[label = left:{\small 输入图像}](A) at(-1.5, 0);
\node(in1) [io] {};
\node(pro1) [startstop, below of = in1] {\small 线性滤波};

\node(in2 - 2)[io3, below of = pro1, yshift = -0.6cm]{};
\node(in3 - 2)[io3, left of = in2 - 2, xshift = -2.5cm]{};
\node(in4 - 2)[io3, right of = in2 - 2, xshift = 2.5cm]{};

\node(in2 - 1)[io2, below of = pro1, yshift = -0.3cm]{};
\node(in3 - 1)[io2, left of = in2 - 1, xshift = -2.5cm]{};
\node(in4 - 1)[io2, right of = in2 - 1, xshift = 2.5cm]{};

\node(in2) [io, below of = pro1] {\small 颜色};
\node(in3)[io, left of = in2, xshift = -2.5cm]{\small 亮度};
\node(in4)[io, right of = in2, xshift = 2.5cm]{\small 方向};

\node(in5)[startstop2, below of = in2 - 2]{\small Center - Surround差异计算及归一化};

\node(in6 - 2)[io3, below of = in5, yshift = -0.6cm]{};
\node(in7 - 2)[io3, left of = in6 - 2, xshift = -2.5cm]{};
\node(in8 - 2)[io3, right of = in6 - 2, xshift = 2.5cm]{};

\node(in6 - 1)[io2, below of = in5, yshift = -0.3cm]{};
\node(in7 - 1)[io2, left of = in6 - 1, xshift = -2.5cm]{};
\node(in8 - 1)[io2, right of = in6 - 1, xshift = 2.5cm]{};

\node(in6) [io, below of = in5] {};
\node(in7)[io, left of = in6, xshift = -2.5cm]{};
\node(in8)[io, right of = in6, xshift = 2.5cm]{};

\coordinate[label = left:{\small 特征图}](B) at(-1, -6.2);
\coordinate[label = left:{\small (12张)}](C) at(-1.5, -7.5);
\coordinate[label = left:{\small (6张)}](D) at(2.7, -7.5);
\coordinate[label = left:{\small (24张)}](E) at(6.7, -7.5);

\node(in9)[startstop2, below of = in6 - 2]{ \small 跨尺度合并及归一化 };

\node(in10) [io, below of = in9] {};
\node(in11)[io, left of = in10, xshift = -2.5cm]{};
\node(in12)[io, right of = in10, xshift = 2.5cm]{};

\coordinate[label = left:{\small 醒目图}](F) at(-1, -9.5);
\node(in13) [startstop, below of = in10] {\small 线性组合};
\node(in14) [io, below of = in13] {};
\coordinate[label = left:{\small 显著图}](G) at(-1, -13);

\node(in15) [startstop, below of = in14] {\small 赢者取全};
\coordinate[label = left:{\small 显著位置}]() at(1, -16.1);
\coordinate[label = left:{\small 反馈抑制}]() at(4.5, -14.7);

%连线
\draw[arrow](pro1) -- (in1);
\draw[arrow](pro1) -- (in2);
\draw[arrow](pro1) -- (in3);
\draw[arrow](pro1) -- (in4);
\draw[arrow](0, -4.75) -- (in2 - 2);
\draw[arrow](-4, -4.75) -- (in3 - 2);
\draw[arrow](4, -4.75) -- (in4 - 2);
\draw[arrow](0, -5.45) -- (in6);
\draw[arrow](-4, -5.45) -- (in7);
\draw[arrow](4, -5.45) -- (in8);
\draw[arrow](0, -8.35) -- (in6 - 2);
\draw[arrow](-4, -8.35) -- (in7 - 2);
\draw[arrow](4, -8.35) -- (in8 - 2);
\draw[arrow](0, -9.05) -- (in10);
\draw[arrow](-4, -9.05) -- (in11);
\draw[arrow](4, -9.05) -- (in12);
\draw[arrow](in13) -- (in10);
\draw[arrow](in13) -- (in11);
\draw[arrow](in13) -- (in12);
\draw[arrow](in13) -- (in14);
\draw[arrow](in14) -- (in15);
\draw[arrow](in15) -- (0, -15.8);
\draw[arrow](0, -15.4) -- (2.5, -15.4);
\draw[arrow](2.5, -14) -- (2.5, -15.4);
\draw[arrow2](2.5, -14) -- (0, -14);
\end{tikzpicture}
\caption{IT算法流程\cite{Itti}}
\end{figure}

% 设置颜色代号
\colorlet{lcfree}{green}
\colorlet{lcnorm}{blue}
\colorlet{lccong}{red}
% -------------------------------------------------
% 设置调试标志层
\pgfdeclarelayer{marx}
\pgfsetlayers{main,marx}
% 标记坐标点的宏定义。交换下面两个定义关闭。
\providecommand{\cmark}[2][]{%
  \begin{pgfonlayer}{marx}
    \node [nmark] at (c#2#1) {#2};
  \end{pgfonlayer}{marx}
  } 
\providecommand{\cmark}[2][]{\relax} 
% -------------------------------------------------
% 开始绘图
\begin{figure}[h]
\centering
\scalebox{.8}{                  %设置缩放	
\begin{tikzpicture}[
    >=triangle 60,              % 箭头的形状
    start chain=going below,    % 从上往下的流程
    node distance=6mm and 60mm, % 全局间距设置
    every join/.style={norm},   % 连接线的默认设置
    ]
% ------------------------------------------------- 
% 节点的样式定义 
% <on chain> 和 <on grid> 可以减少手动调整节点位置的麻烦
\tikzset{
  base/.style={draw, on chain, on grid, align=center, minimum height=4ex},
  proc/.style={base, rectangle, text width=8em},
  test/.style={base, diamond, aspect=2, text width=5em},
  term/.style={proc, rounded corners},
  % coord 用来表示连接线的转折点
  coord/.style={coordinate, on chain, on grid, node distance=6mm and 25mm},
  % nmark 用来表示调试标志
  nmark/.style={draw, cyan, circle, font={\sffamily\bfseries}},
  % -------------------------------------------------
  % 不同的连接线样式
  norm/.style={->, draw, lcnorm},
  free/.style={->, draw, lcfree},
  cong/.style={->, draw, lccong},
  it/.style={font={\small\itshape}}
}
% -------------------------------------------------
% 先放节点
\node [term, densely dotted,fill=lccong!25, it] (p0) {输入};
% 用 join 表示和上一个节点相连 
\node [proc, join]	{使用非线性最小二乘法得到 $X_0$};
\node [proc, join]	{记录 $X=X_0, f=f(X_0)$};
\node [test, join] (t1)	{$T>T_E$?};
\node [proc] (p1)		{$step=0$};
\node [test, join] (t2)	{$step<count$?};
\node [proc] (p2)		{得到新状态$P_N=P+scale\times rand$,计算目标函数差$\Delta f$};
\node [test, join] (t3)	{$F_{Accept}<rand$?};
\node [proc] (p3)		{记录新状态 $X=X_N,f=f(X_N)$};

\node [proc, left=of t1] (p4)	{$T=T\times a,scale=scale\times b$};
\node [term, densely dotted, right=of t1,fill=lcfree!25](p5)	{输出};
\node [proc, right=of t3](p6)	{$step++$};

\node [coord, left=of t2] (c1)  {}; 
\node [coord, right=of t2] (c2)  {}; 
\node [coord, right=of p3] (c3)  {}; 
%先画南北方向的连接线,先画线再画两端的标志和箭头
\path (t1.south) to node [near start, xshift=1em] {$y$} (p1);
  \draw [*->,lcnorm] (t1.south) -- (p1);
\path (t2.south) to node [near start, xshift=1em] {$y$} (p2);
  \draw [*->,lcnorm] (t2.south) -- (p2);
\path (t3.south) to node [near start, xshift=1em] {$y$} (p3);
  \draw [*->,lcnorm] (t3.south) -- (p3);
%接着画东西方向的连接线,方法同上
\path (t1.east) to node [near start, yshift=1em]  {$n$}(p5);
  \draw [o->,lcnorm] (t1.east) -- (p5);
  \draw [->,lcnorm] (p4.east) -- (t1);
\path (t3.east) to node [near start, yshift=1em]  {$n$}(p6);
  \draw [o->,lcnorm] (t3.east) -- (p6);
\path (t2.west) to node [near start, yshift=1em]  {$n$}(c1);
  \draw [o->,lcnorm] (t2.west) -- (c1) -| (p4);
  \draw [->,lcnorm] (p3.east) -- (c3) -| (p6.south);
  \draw [<-,lcnorm] (t2.east) -- (c2) -| (p6.north);
\end{tikzpicture}
}
\label{fig:algorithm}
\end{figure}

\tikzstyle{every node}=[draw=black,thick,anchor=west]
\tikzstyle{selected}=[draw=red,fill=red!30]
\tikzstyle{optional}=[dashed,fill=gray!50]
\begin{tikzpicture}[%
  grow via three points={one child at (0.5,-0.7) and
  two children at (0.5,-0.7) and (0.5,-1.4)},
  edge from parent path={(\tikzparentnode.south) |- (\tikzchildnode.west)}]
  \node {texmf}
    child { node {doc}}
    child { node {fonts}}
    child { node {source}}
    child { node [selected] {tex}
      child { node {generic}}
      child { node [optional] {latex}}
      child { node {plain}}
    }
    child [missing] {}
    child [missing] {}
    child [missing] {}
    child {node {texdoc}};
\end{tikzpicture}

\begin{tikzpicture}[sibling distance=10em,
    every node/.style = {shape=rectangle, rounded corners,
        draw, align=center,
        top color=white, bottom color=blue!20}]]
        \node {root}
        child { node {snap1} }
        child { node {snap2}
            child { node {snap3}
                child { node {snap4} }
                child { node {snap5} }
                child { node {snap6} } }
            child { node {snap7} } };
\end{tikzpicture}

\begin{tikzpicture}
  \begin{scope}[blend group = soft light]
    \fill[red!30!white]   ( 90:1.2) circle (2);
    \fill[green!30!white] (210:1.2) circle (2);
    \fill[blue!30!white]  (330:1.2) circle (2);
  \end{scope}
  \node at ( 90:2)    {Typography};
  \node at ( 210:2)   {Design};
  \node at ( 330:2)   {Coding};
  \node [font=\Large] {\LaTeX};
\end{tikzpicture}

% Drawing part, node distance is 1.5 cm and every node
% is prefilled with white background
\begin{tikzpicture}[node distance=1.5cm,
    every node/.style={fill=white, font=\sffamily}, align=center]
  % Specification of nodes (position, etc.)
  \node (start)             [activityStarts]              {Activity starts};
  \node (onCreateBlock)     [process, below of=start]          {onCreate()};
  \node (onStartBlock)      [process, below of=onCreateBlock]   {onStart()};
  \node (onResumeBlock)     [process, below of=onStartBlock]   {onResume()};
  \node (activityRuns)      [activityRuns, below of=onResumeBlock]
                                                      {Activity is running};
  \node (onPauseBlock)      [process, below of=activityRuns, yshift=-1cm]
                                                                {onPause()};
  \node (onStopBlock)       [process, below of=onPauseBlock, yshift=-1cm]
                                                                 {onStop()};
  \node (onDestroyBlock)    [process, below of=onStopBlock, yshift=-1cm] 
                                                              {onDestroy()};
  \node (onRestartBlock)    [process, right of=onStartBlock, xshift=4cm]
                                                              {onRestart()};
  \node (ActivityEnds)      [startstop, left of=activityRuns, xshift=-4cm]
                                                        {Process is killed};
  \node (ActivityDestroyed) [startstop, below of=onDestroyBlock]
                                                    {Activity is shut down};     
  % Specification of lines between nodes specified above
  % with aditional nodes for description 
  \draw[->]             (start) -- (onCreateBlock);
  \draw[->]     (onCreateBlock) -- (onStartBlock);
  \draw[->]      (onStartBlock) -- (onResumeBlock);
  \draw[->]     (onResumeBlock) -- (activityRuns);
  \draw[->]      (activityRuns) -- node[text width=4cm]
                                   {Another activity comes in
                                    front of the activity} (onPauseBlock);
  \draw[->]      (onPauseBlock) -- node {The activity is no longer visible}
                                   (onStopBlock);
  \draw[->]       (onStopBlock) -- node {The activity is shut down by
                                   user or system} (onDestroyBlock);
  \draw[->]    (onRestartBlock) -- (onStartBlock);
  \draw[->]       (onStopBlock) -| node[yshift=1.25cm, text width=3cm]
                                   {The activity comes to the foreground}
                                   (onRestartBlock);
  \draw[->]    (onDestroyBlock) -- (ActivityDestroyed);
  \draw[->]      (onPauseBlock) -| node(priorityXMemory)
                                   {higher priority $\rightarrow$ more memory}
                                   (ActivityEnds);
  \draw           (onStopBlock) -| (priorityXMemory);
  \draw[->]     (ActivityEnds)  |- node [yshift=-2cm, text width=3.1cm]
                                    {User navigates back to the activity}
                                    (onCreateBlock);
  \draw[->] (onPauseBlock.east) -- ++(2.6,0) -- ++(0,2) -- ++(0,2) --                
     node[xshift=1.2cm,yshift=-1.5cm, text width=2.5cm]
     {The activity comes to the foreground}(onResumeBlock.east);
\end{tikzpicture}


\part{航海日志}

\chapter{2012}

10月,从美地森离职去光点图论,接触移动互联网技术:

\section{经验教训}

\begin{enumbox}
\item 没有把握住重点
\item 没有培养出人才
\item 陷入技术和运维的泥潭
\item 技术上没有深入
\item 没有完全投入
\end{enumbox}

\section{技能矩阵}

\begin{enumbox}
\item Nginx
\item Redis
\item MongoDB
\item ElasticSearch
\item Hadoop/HBase
\end{enumbox}

性能优化:
\begin{enumbox}
\item 增加多级缓存,包括Nginx/App/Redis等
\item 动静分离,独立出imagefs (保存了原图与各分辨率的图)
\end{enumbox}

Mongodb:
\begin{enumbox}
\item 增加MongoDB index
\item 架构:replication
\item 图片渲染服务器
\end{enumbox}

关键字搜索
\begin{enumbox}
\item ElasticSearch
\end{enumbox}

数据分析
\begin{enumbox}
\item Hadoop/HBase
\item MapReduce
\item Clojure
\end{enumbox}

第三方服务
\begin{enumbox}
\item DNSPod
\item 网宿
\item 蓝汛
\item 友盟
\item 广告平台
\item 阿里云
\end{enumbox}

DNS:
\begin{enumbox}
\item DNS负载均衡
\item 解析不到
\end{enumbox}

CDN:
\begin{enumbox}
\item 设置回源地址
\end{enumbox}


\chapter{2016}

FusionStor

\begin{enumbox}
\item 远程复制
\item 一致性卷组
\item EC
\item COW snaptree
\item LSV (2017)
\end{enumbox}


\chapter{2017}

深入FusionStor实现



\chapter{2018}

华云网际,分布式块存储系统开发。

\section{08}

\subsection{01}

论自由,不仅是论,更在于得到。自由不仅是认知问题,更是实践问题。恒以一德,此一德就是自由。

庄子的自由离活泼泼的现实生活有点远。行动自由是兵家必争,自由则含义更广。抓手在哪里?

2019是自由元年,经过多年磨砺,心智渐趋成熟,可以更自由地去呼吸、去奋争,去为所欲为。

\hrulefill

计算机体系结构要好好学习,胡伟武的教材为主。在一个更广泛的范围内考虑架构问题。
\begin{myeasylist}{itemize}
& 伟大的计算原理
& 多核应用架构关键技术
& 排队论
& 响应式架构
& 计算机体系结构
& 性能之巅
\end{myeasylist}

\subsection{02}

\hl{hegel的哲学,与毕建勋的三合之道}很匹配,互参互证。精神从逻辑学开始,下降到自然界,再回归到精神自身,
从空泛到充实,充实之美。在征途中,攻城略地,海纳百川的气象,又有着攻城略地的开拓进取。

为什么把逻辑置于山顶?心物二元借着道的中介作用,交融为一。

年薪百万是个坎,并不难逾越。应有计划地突破之。此为目标的量化。

拿出三分精力做管理工作,应有章法,目标导向,严格要求。

\subsection{06}

思维科学
\begin{myeasylist}{itemize}
& 一二三哲学
& 轩辕三书
& 道德经
& 王阳明
& 毛泽东
& 大秦帝国
& 黑格尔
& 波普尔
\end{myeasylist}

绝利一源,用师十倍。

\subsection{07}

\begin{shadequote}
读书就应该读功用最大,价值最高的书。人们所做的一切不都是为了有所功用吗?小功用的书比比皆是,通俗易懂,无须费多少心力即可掌握一技一长。
但这是舍本逐末了,既然要读书就应当选择探究终极之道的书,以道贯通天下万事万物,实现最大的功用。

像这样至高经典的学习,决不能跟学习普通知识一样,看几遍就扔了。若想达到极高的效用,一定要熟读成诵本书,一时理解参透不了的,也不必着急,就像牛反刍一样,经典熟记了,
就会在你的思想中蕴化,为你的思想贯通万事万物打下基础。若要思想升华,只读《阴符经》肯定是不够的,因为道是幽微玄奥的,几部经典互通参照,才能更好的认识道。
在后面的文章中。道易学宫还会继续推出诸如《老子》这样的经典。
\end{shadequote}

\say{阴符经、道德经皆为至高经典。吾道一以贯之,用道贯通天地万物。三合之道中道居于顶点,无复多疑,一定要理解其中深意。
阳明良知说,弘扬人的主体性,但并非可以取代道的绝对性。在hegel哲学里,绝对理念是贯穿始终的东西,一步一步到达真理之境。}

\enquote{原则是道的简化,有限版的道。道是开放体系,hegel哲学并非如马克思所云是封闭的脚手架。是有限与无限的统一体。
知性和理性的区分非常重要。\hl{西方辩证法如何过渡到东方辩证法,情、势、节}?}

\hrulefill

\hl{大处着眼,小处着手}。好好悟这个,大处是思维境界,小处是工作境界,吾道一以贯之。
思维、工作、道法构成三合之道。升维思考,降维贯通。一升一降,生命之圆运动是也。

狐狸和刺猬,

继续\hl{租房}住吧,省事了,主要是省时间,省下来的时间好好用来学习,提高自己,再给一年时间,需要一个质的变化。

\subsection{21}

\hl{轩辕三书},千古之绝唱,无尽之宝藏。

用行动改造理念并形成制度。做事、成事,又不是盲目地做。

\hl{一之解,察于天地;一之理,施于四海}。一即是道,也是事。实事求是,才能一以贯之。
两者是贯通的、不是支离的。道寓于事物之中。在观念上是形而上的。
hegel哲学所揭示的,精神的力量、理念的力量、认识的力量甚至比物质的力量要大。
谋事在人成事也在人。

谋势、积微,缺一不可。

提出\hl{一的哲学},合一的力量。三合之道也不是究竟,上遂而及于一的哲学。

知行合一。

实践论、矛盾论

唯物论、辩证法

\subsection{23}

我的第一本书将叫做第一哲学、一的哲学。

\subsection{26}

一的哲学,简称一学。一学非一休,一文一武,一张一弛。

夫为一而不化,得道之本,得事之要。抱道执度,天下可一也。

化书,御一。

一能贯五,五能综一。

吾道一以贯之。知行合一。

用pcda做好管理工作。

\subsection{27}

一学,配合pdca作为主要的管理工具,pdca的每个字母都是一大课题。转动pdca循环,解决现实问题。
比如,联想方法论的复盘,就是c的深化。

目的是什么?思想方法和工作方法,更重要的是学以致用,用方法指导实践活动。
归纳下来,就是一个知一个行,知行合一。

\hl{道治天下},心道物三合,道具中枢地位。以道通物,以道观天下。
原则、多元思维模型都是法,法自道生。事督乎法,法出乎权,权出乎道。

四度,\hl{春生夏长秋收冬藏},周而复始,与pdca相当。

\section{201809}

\subsection{0901}

\subsection{0902}

\subsection{0903}

战略几何学、神圣几何:圆是时间,四方形、十字架是空间,三角形是存在,构成时-空-存在的结构。

双环系统可以解释一切,双环相交处是太极图。右手螺旋法则。

周末读书,关注到几个概念,心神、机发,心神论是黄帝内经的精髓,机发论是易道主义的理论枢纽。

机是什么?随机而动,机是变动不居的存在,但可以通过思维与实践的方式去认识和把握。
阳明心学的精髓:此心不动,随机而动,就是圆点结构。

一心一意到专业学习上,有道有术两个层次多个层次。所有的事情,都是培养心体。
要留出足够的时间去反思,并记录下反思的过程与结论。

这么多年,很遗憾的一点就是不能一心一意,也就是不诚,身在曹营心在汉,不能全心全力地投入到手头要紧的事情上,
老是觉得另有更大价值的事情,反而导致手头的机会也白白溜走。

今后当从容规划(转动PDCA循环),稳扎稳打,一步一个脚印,去实现目标。

几有多义,主要是微和危。事物的萌芽状态,看不透、想不明白,\hl{惟精是惟一的功夫},博文是约礼的功夫。这是阳明一贯的主张。
守住底线、抓住关键是方法,围绕一转动PDCA循环。

\hl{如何尽快实现财务自由}?四象限,打工、个体户、创业、投资。贯穿其中的是\hl{专业、工匠精神}。
只是有工匠精神依然不够,要有道。立足于当下,什么才是最重要的事情呢?

\hl{乾之九三给出了答案}。乾坤是易的门户,黄帝垂衣裳而治天下,盖取诸乾坤。

\subsection{0904}

乾坤是易的门户,易是通向现实世界的门户。这是非常重要的论断,因为一是学易之法,二是用易之法,学以致用,解决现实问题。
读书不在乎多,学宗大易,一部易经观天下。透过一部易经,而通达于现实世界,得偿所愿,心想事成。
通过易,撑起开物成务、进德修业的英雄梦想之旅。

六爻之动,三极之道。分而论之,初二为地道,三四为人道,五上为天道,匹配几、诚、神。

用\hl{三级火箭模型}分析创业公司的发展轨迹和着力点,什么是发动机?如何一环套一环?
产品和客户是任何公司的两极。设计理念与客户反馈要综合为用。

易之三义,变易、不易、易简。

\subsection{0905}

努力经营事业,开始物色各类人选,看看水浒传、三国演义,任何事业都不会想当然地一蹴而就,而是长期经营的结果。
事业是男人的第一支柱。

易经在这方面有着深沉的诉求,圣人以神道设教,抛开迷信的成分,易经是第一励志书,也是第一帝王书。
学习易经,方术方面了解即可,不作为主要方面,重要的是开物成务、进德修业方面的启发和辅助。

至九四,始入于上层,开启了自己的平台和事业。上下分际处是着力点。
或跃在渊,此一跃是多方面因素叠加的结果,主要还在于自己的野心、理念、认知。
此一跃,不回头。

一是因果律,二是神圣意志之发扬。乾卦就是这样的精神力量,乘云气而御飞龙,高扬进取意识。

更加open地去思考关键问题,包括行业、事业等等。思考、交流都是需要的。
进一步去了解别的产品,主要是把握趋势。

双环系统可以解释一切,双环相交处是太极图。

怎么通向现实世界呢?

\subsection{0906}

不能控制自己的情绪,太幼稚,这种东西纯粹影响发挥。当前第一要务是什么?事业,不容置疑。迄今没有起色。

第一个是专业环,这是安身立命之本。经过多年的摸索,是整理收割的时候了。

第二个是易经,全面拥抱易经,以之作为进德修业的重要基础。以此洗心,退藏于密,洗心,就是淬炼心智模式。
易经在思维方面,有着深度与广度。进入眼界的思维模型,都挂入易经这个思维格栅中去,易经就是太上老君的八卦炉,
淬炼出了孙悟空的火眼金睛。

另外,黄帝内经所蕴含的神本论以及机发论思想,在易经中也有深入的体现。洁净精微,易之教也。

环环相扣,专业与易经之环,碰撞出火花。工作与生活都需要大设计。

不要急、慢慢来。易经为起点,一部易经观天下,指导生活与工作之设计。专业是工作的一部分。
生活是进德,工作是修业,内外兼备,合内外,一物我。

一切的学习都不仅仅为了学习而学习,为了单纯的知识而学习,而是为了解决问题。

关键思想:
\begin{enumbox}
\item 确定易经作为最高指导思想,第三空间或虚或实,主要指的是这个,过有原则的生活,富有之谓大业、日新之谓盛德
\item PDCA作为执行方法
\item 双环系统分别对应生活和工作
\item 把\hl{视点/视角方法}作为架构描述语言
\item B:确定把分布式存储作为主要的技术领域
\item B:确定把QoS作为主要的研究方向之一
\end{enumbox}

\subsection{0911}

全力以赴到专业方向上,去解决关键问题,太极云尔,是反思框架。

心、道、物的三合之道,适合于下一阶段的学习过程。心就是阳明所谓良知,为学头脑所在,多问多思。道,原则,方法论,架构。
物是要研究的系统,要解决的问题。以道观之,以架构之眼看系统,当如庖丁解牛。

双环,一者三合之道,二者PDCA。双环正交?

对解决问题有腻烦心理,问题是前进的动力,当善待之,乐于去搞定她。

\subsection{0912}

心神主宰,以道观之,落实到物,以道的光华普照世界。寂然不动、感而遂通天下之故,这是二重性。

第一个小目标,100w,1000w,以此类推。明年大概就有100w了,坚定地走下去,不急不躁。重为轻根,静为躁君。

架构驱动的软件开发过程。

坚持用SWOT分析,是战略分析的起步。

\hl{本周末给出一个更明确的路线图}。第一,强化架构思维能力,视图视角是标准做法,IEEE STD 1471-2000。
视图可以视点集为模板,也可以单独定义。运用视角到视图之中,形成纵横交错的架构描述。

\subsection{0913}

\begin{shadequote}

能把诚神几统一起来的为圣人。北宋周敦颐在《通书》中提出的命题。“寂然不动者,诚也。感而遂通者,神也。
动而未形,有无之间者,几也。……诚神儿曰圣人”(《通书·圣》)。
诚是静无的,即“诚无为”(《通书·诚几德》)。神“感而遂通”,是诚的直接表现。几处于静无动有之间,是动之始。
诚是纯粹至善的,是一切道德行为的源泉。
神是诚的直接表现,故亦善。只有几“动于彼”,感外物而动,故兼有善恶。
《宋元学案·濂溪学案上》云:“常人之心,首病不诚。不诚故不儿而著。不几故不神。物焉而已。”
常人不能以诚贯几,受物之累而为恶。只有圣人才能以诚贯几,去几中之恶,把诚神几统一起来,故诚神几曰圣人。
\end{shadequote}

心道物,诚神几,有对应关系。把心置于三角形顶点处,似更体贴。

养心莫善乎诚,致诚则无它事。至诚之道,可以前知。惟天下至诚,为能经纶天下之大经,立天下之大本,成天地之化育。

圣人以神道设教,道则通神,一阴一阳之谓道,阴阳不测之谓神。何为神?妙万物而为言者。

几,人心惟危,道心惟微,几则合多义而言。机发论更提出制机的说法,乃易道主义的理论枢纽。
从机发论的角度理解,\hl{黄帝内经}灵枢,\hl{鬼谷子、阴符经}亦然。

\hl{此三角形居于左侧(符合右手螺旋法则),圆形+十字架构成的几何形状居于右侧(SWOT, PDCA, 2x2矩阵及其延伸,符合左手螺旋法则),
左右交错,形成太极之两仪}。大拇指都指向自己,反求诸己,建立自我,贵我通今,时变是守。
此参伍以变,错综其数的义理架构,实有进一步发挥的余地。

左为知、右为行,以此类推,大商之道的道术、变常、方圆、生死、利害、取予之对立统一,也是如此。

孙正义的25字诀,与\hl{周易、兵经百字、东方战略学},都是以字通道的卓越理念。

\subsection{0914}

观象玩辞,以字通道。建勋画论的三合之道,启人深思。道具太极位,则有商讨的余地。邵雍曰:心者太极也。华严经云:心如工画师,能画诸世间。
阳明心学也是如此。心是能动的一面,也是目的性的一面,使心居于太极位,乃应有之义。心秉道通物,心格物穷理,天性,人也,人心,机也。
立天之道,以定人也。此说并不否定或拉低道的价值,而是在建立自我的阶段,高扬心性,确立为学的头脑。道依然是那个道,
致吾心之良知于事事物物,则事事物物皆得其理。即满足了目的性要求,又满足了道的约束性原则性。

欲正其心者,先诚其意。在明道、格物的过程中,诚其意。事上磨练,皆在涵养此心之体。由物及心,完成此逆时针的环转关系。此右手螺旋法则。

如忽略道的环节,而直奔物的主题,则易于陷入事务主义的泥淖之中,事半功倍,乃至无功而返。
如过于强调理论,也有教条主义的倾向。

神者生之本。

\subsection{0918}

系统思考。

职场与理想的距离,靠三度修炼去完成。三度:态度、气度、厚度。读一艮卦,胜读一部华严。
中秋看王明夫主编的三度修炼,好好想一想下一步的规划。

\subsection{0920}

离开HY的可能较大,离不离开,都要以成长为主要标准。时间并不充裕,接下来到年底的一段时间,好好锤炼专业技能。

\hl{优先考虑开启自己的事业},专业技能的学习、知识体系的构建,不能脱离这个目的,才称得上学以致用。

\hl{全闪时代来临,离自己最近,怎能再次错过}?应采用包围式学习,地毯式学习,既要明确关键,又要面面俱到,点线面体,全面展开。

在多副本复制的场景下,由一控制器负责,如果控制器发生切换,则开启新纪元。在某一控制器的生存期内,
每次提交采用单调递增的版本号,所以二元序号的构成:(epoch, version/clock)。
卷控制器可管理很多chunk及其副本一致性,控制器位置与副本位置不具有对应关系。\hl{卷控制器可迁移}。

关于控制器的若干关键问题:
\begin{enumbox}
\item 如何选取控制器
\item 客户端如何定位控制器
\item 控制器发生切换又如何
\end{enumbox}

paxos的精髓是温故知新,一个实例产生一个值。如何标记序号?序号可以是二元结构,方便处理。

multi paxos与RAFT的异同?每一个控制器的生命周期包括三阶段:\hl{选主、恢复、正常操作}。

进一步,传统的2PC、3PC算法的不足和使用场景。这类算法是分布式系统的精髓,务必加以消化理解。

\hl{算法是程序员的金线},理应是下一阶段的重点。比如,通过token bucket或leaky bucket解决qos问题,实现很简单,设计很精妙。

马云定随舍三部曲,第一曲是定字诀。艮,止也。知止而后有定,定而后能静。

\hl{起居有常、饮食有节},乃养生之道,不仅此也,常与节有深意存焉。
财自道生、利缘义取,是大商之道。菩萨畏因,凡夫畏果。

\hl{多听多问}是领导之道,陈述句不如疑问句。

易经的卦图是直线,加上圆点哲学,三角形集两者之大成,融合双线法则、圆点哲学、三点论、一分为三诸论而为一,
算是多年思考、探索的一点结论。三生万物,由此而展开其广泛的运用过程,进入明体达用的第二阶段。
用太极两仪模式解读三角形各点之间的关系。

道是吾观物的门户、工具,不能僭越心的第一性,道物、道器、体用,分阴分阳,叠用柔刚。
\hl{吾有方圆之形}。五代表圆点哲学,PDCA等。以五为食,哈?口为口、为目,以五观之、观天下。

两个三角形,下一个代表物理资源,上一个代表虚拟资源,中间的交点是集群,物理资源总而为一,进一步生化出虚拟资源。

心道物三角,自身也有两种旋转方向,左手螺旋右手螺旋,标准图以心为顶点。

\subsection{0920}

战略一二三,美团十字架,参伍点成圆,乱环诀中诀。

智仁勇三达德,好学近乎智,力行近乎仁,知耻近乎勇。

在乱环之中,存在第一义,找到她!

架构、算法是内功心法,练拳不练功,到老一场空。

功业之心热灼,怎么开始?如何播种下第一粒种子?离什么最近? 立于中央,由近及远。

无所待,此时就是开始!此时此地,从心开始。

开始不难,终局判断如何?商业计划书?开始吧。

人钱事,搭班子、定战略、带队伍。做什么?怎么做?如何解决启动资金的问题?

如何整合资源?一二三级火箭分别是什么?

\section{201810}

\subsection{1003}

为学头脑处,此阳明念念不忘者。格物穷理,未免支离。头脑处在明明德,在心。龙场一悟,由外而转至于内。
精神之体相用,一而三、三而一,全体大用得以实现。

如何在心体上用功?在念头功夫。慎独之说,净念相继、都摄六根。正念是功夫、良知是本体。

先守住一部大学,体用兼备,兼采朱子阳明意,阳明为主,朱子为辅,尊德性而道问学。

\subsection{1004}

理解阳明心学之真实义,及其演进脉络,首要在于切己体察,作为成长之一助力。以德性融摄知识,在诚意中格物。

熟读大学,以定其规模。大学格物致知,兼采朱子阳明,以阳明为主。三合之道,圆伊三点。

张学智在阐释阳明心学时,采用道德理性与知识理性一主一从、相辅相成的观点,深有启发。

为学日益,为道日损。道统摄学,达以简驭繁之效。

三五以变,为学处事的纲领。三摄太极两仪,五有空间时间。数年方法论探索的一综合结论。混沌大学的第一性原理,第二曲线,不连续性
也纳入这一体系内。三生万物。

\begin{enumbox}
\item 易经
\item 道德经
\item 大学
\item 中庸
\item 孙子兵法
\item 传习录
\item 画道精义
\item 一二三哲学
\item 原则
\item 用系统来工作
\item PDCA
\item 稻盛和夫
\end{enumbox}

读书诸原则:
\begin{enumbox}
\item 有的放矢,精读泛读相结合
\item 读原文、悟原理、知行合一
\end{enumbox}

\subsection{1004}

从诚意去理解大学中庸,修身为本,则有下手处。喜怒哀乐之未发,谓之中;发而皆中节,谓之和。
中也者,天下之大本;和也者,天下之达道。致中和,天地位焉,万物育焉。

不反身,看不出一身毛病。

儒学,心学也。止于一为正,中和一是圣学根本。从内讲,至诚无息,纯亦不已。从外讲,尽性知天。
唯天下至诚,为能尽其性;能尽其性,则能尽人之性;能尽人之性,则能尽物之性。根源在诚意。

诚意与觉知、正念、冥想等略同,为明明德、致良知的功夫所在。

三五的中和一怎么理解?

\subsection{1006}

悟后大有功夫在,专且有恒,不可泛滥无归。大学中庸,入道之书,当熟玩之,以奠定根本。

看三国电视剧,关羽、周瑜、杨修等人,皆以傲字取败。阳明国藩诲人,以傲字为第一凶德,可不警惕乎?
力去此病,劳谦、君子有终,吉。玩易既久,而不得真实受用,则与不读等,枉费精神而已,当思痛改之。

反身而诚,乐莫大焉。反之,反求诸己,不怨天不尤人,实为修身之首务。确立我是一切成败的根源,从而自强不息。

萧天石极言精功、内功之有益,宜重视。圣人定之以中正仁义而主静,立人极也。自然之道静,故天地万物生。
静而能生,宜深思。动有何敝?

专攻读一经,易经是也。易之妙,终身读之不能穷尽。大学中庸道德经等,皆易经之辅翼。太极两仪,此大学三纲领之义理结构,
从内修的角度去理解,修身为本,修身实为进德修业的根本。阳明龙场之悟,点出了一个重要的道路,突出了炼心,合心物内外,而成一元论。
由外而转向内,明明德、致良知皆是心性功夫,心性不废知识,相得益彰,一君二民,逡巡并进。

\subsection{1007}

心道物三者循环往复,心的代表是稻盛和夫,道的代表是范蠡。道商,以道经商,是切合时代精神的选择。工匠精神、企业家精神有共通处,至诚之心,感天动地。
诚意是功夫头脑所在,论语与算盘,经济不仅是个人重要的一面,也是社会最重要的一面。致良知四合院是如何贯通两个领域的?企业家们学习致良知的价值有几何?

产业资本、金融资本之间的矛盾,金融资本得全球化之利,产业资本渐有全球化之害。

心性修养是一切的根源,内圣开出新外王,新时代,教育、科技、企业、资本是重点。

认知极重要,观三国演义、国共之争能强烈地感觉到,正确的见识有多么重要!寥寥数语,化腐朽为神奇。

国庆小结:

一、看电视剧三国演义、毛泽东,深知认知之重要,认知差极难弥补,超认知在事物的发展演进过程中,发挥着极大的设计塑造作用。
性格决定命运,人的事业发展的高度,视其性格即可见八九。由此可知,有无相生,无形的心性修养在事业中占据着极为重要的作用,
不可等闲视之。阳明心学融心学、知识修养与一炉,并非虚言。伟大人物如何看待一个事件,体现了其见识、洞察力。志、量、识不可或缺。

二、国庆之始,意在研读阳明学,中途多有变化,如看了萧天石、道商系列,沿着心道物的认知框架,逐步拓展。反求诸己是国庆最大的收获。
回归到正途,怨天尤人皆无用,风物长宜放眼量。不反身,看不出一身毛病;不反身,无以开发全部的潜能;不反身,就奠定不了以后发展的扎实基础。
行有不得,反求诸己,此君子之行也。不惟如此,一个团队、一个组织,也离不开沉下心来,如切如磋,十年一剑,磨炼内功。重视大学中庸,
如果仅仅在文字上打滚,也不会有大的收获。知行合一,方收大利。

三,既然立定了修身为本的纲领,诚意实为修身之要,慎独、主敬、求仁、习劳,是曾文正的心得。
不诚意,则旧习难改,因循度日,有恶不能改,有善不能迁,所失大矣。诚意是格物的主意,也是为学的头脑。按诸中庸之说,诚意之用,实为首要之责。
在心体上用功夫,则无支离无统之病。

四,重阳立教十五论有论学书一节,很得心学读书法之精要。心学的理念运用到读书、工作、生活中,当大有裨益。

五,道商学提出了六图思维模型。有极、无极、有无相生(太极、中极、真一),乃至大成。分四层。道常无为而无不为。有无之合,有中一至善之境。
真一图有近于圆点哲学。其中对三、五的解读,多有值得留意之处。四正的提法尤为精彩:智慧、生命、事业、兵法,合内外之道,由此可建立完善的知识体系。

六、关注到了八段锦,相比太极拳更为简易,工作闲暇之余,可玩味之,性命双修,养生之术,也当予以留意。神者生之本,另外一说也同样正确,好的身体
真的太重要。生命在于运动,生命也在于不运动(静)。静功性命双修,此南怀瑾、萧天石等前辈所明言。


\section{11}

\subsection{01}

编程注意事项
\begin{enumbox}
\item long time操作之前与之后,需要renew lease,防止lease失效,导致vctl切换
\item 读写、vfm\_get、vfm\_stat、stat等操作之前,必须chunk check。\hl{如果不检查,可能会返回ESTALE, see \_\_chunk\_read\_getnode}
\item chunk组织成tree,就意味着依赖性,需要从上到下依次保护,check
\item chunk上有lock保护
\item alloc/discard与io分离,故io加rdlock
\item \hl{困难之处在并发,tree结构上的并发,并且涉及到持久化、lease}
\end{enumbox}

诊断工具
\begin{enumbox}
\item 怎么快速得到各个controller的状态,包括pool、volume、snapshot等?
\item 怎么检查底层数据一致性?
\item 打印集群chunk tree,以及每个chunk的详细信息
\end{enumbox}

写到LUN结尾处,会自动扩容。ROW3这里有问题?

垃圾数据,干扰恢复,gc replica目前不能打开?

\subsection{02}

尽可能通过\hl{DBUG、GOTO}保留可跟踪线索,以方便线上调试。

mellanox交换机编程:SDK?


\section{12}

\subsection{03}

交换机SM HA不可用,更新OS到相同版本后可用。

熟悉RDMA

两副本,一副本处在chunk状态,且vfm非空,包括另一副本

\subsection{10}

熟悉iscsi/iser架构和代码,进一步需要熟悉spdk、nvmf等,这是块存储的趋势所在。

\begin{enumbox}
\item NVMe设备,通过设备文件和pci直接访问的不同
\item linux kernel的block layer
\item udev
\item mount是指定dirsync选项
\item dd在测试快照时,指定direct io
\item async和non blocking的异同
\item buffer和cache的异同
\item session和cookie
\item 内核提供了什么功能?现在又为什么在网络和存储上提倡kernel bypass。
\item vip and multipath
\item ceph
\item fs and object
\item db
\end{enumbox}

除了协议和生态部分,内核部分的主要模块也要持续思考,包括编程模型、元数据管理、HA等。
要具有提要钩玄的能力,勾勒出主要脉络,分清主次,带动全局。中医对生命科学的思考方式,很有借鉴意义。
人的生命是一个动态发展的整体。如何维持此整体的健康?

\subsection{11}

要深入体会网络协议栈的架构设计实现,充分地体现了分层架构的巨大威力。

要想富,先修路。网络编程的第一步就是连接管理。连接的建立、维护到断开。
一个或多个连接构成session。一个session见得唯一的target。target与lun什么关系?
为什么需要多个连接构成session?

关于学习:重视第一手资料,道听途说不能解决问题,善于提问,不要有畏难情绪。
比如RDMA,从原始文献好好体会,在iSER里进一步体会,温故知新,没有过不去的坎。

再回头看RDMA代码,当明白了一些原则后,就显得一目了然了。
iscsi和iser代码也是如此。为什么呢?需要理解消息交换的基本原则。
在一个循环内,既有post,又有poll。

scst与lio解决的是同一类问题,scst的代码分为三部分:core、target driver and storage driver。
lich也有如此模块。core起到存储虚拟化的作用。

学习一定要破除神秘感,循着正确方法,很容易攻克。即便如AI、数学,也不是不能捡起来。应该捡起来。

\subsection{12}

对个人来说,做什么才能产生复利效应?专业、认知。

一个lich系统,一个生命系统,都是系统。系统论具有普适性。

网络和scheduler的事件循环,也是一个圆运动。

\begin{enumbox}
\item 列出所有controller及其分布
\item 控制器为什么加载不成功?
\end{enumbox}

专注全闪,不做它想。不知自我克制,终将一事无成。

在docker里用文件模拟设备文件,因为文件io接口是一样的。

测试方法和工具:\hl{道法术器}几个层面,工具化、自动化至关重要,
但必须在正确策略的指引之下。

查找控制器为什么用mcast机制,而不是去admin上查(lease机制),
因为客户端有cache,也不会对admin带来额外负担。

vip才会有arp缓存失效的问题,需要主动更新client端arp缓存。

\subsection{13}

linux
\begin{enumbox}
\item 操作系统
\end{enumbox}

ceph
\begin{enumbox}
\item 两篇论文
\item 官方文档
\item 参考书
\item 源码
\end{enumbox}

\subsection{14}


\end{document}

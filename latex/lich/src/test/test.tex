\chapter{测试}

\lstset{numbers=left,
    frame=shadowbox,
    numberstyle= \tiny,
    keywordstyle= \color{ blue!70},commentstyle=\color{red!50!green!50!blue!50}, 
    rulesepcolor= \color{ red!20!green!20!blue!20} 
}

\section{已知问题}

\begin{enumbox}
\item vol resize会产生死锁
\item vol copy的提示
\item flat后保护快照
\end{enumbox}

\section{部署}

基本步骤:
\begin{enumbox}
\item 创建集群
\item 创建存储池
\item 向存储池添加磁盘(Tier, SSD Cache)
\item 创建卷
\item 创建快照
\end{enumbox}

\subsection{创建集群}

\begin{lstlisting}[language=bash]
lich prep t151 t152 t153
lich create t151 t152 t153
\end{lstlisting}

\hl{注意事项}:
\begin{compactenum}
\item 检查IP是否重复
\item 检查子网mask是否匹配
\item ...
\end{compactenum}

\subsection{创建存储池}

\begin{lstlisting}[language=bash]
lichbd pool create p1
\end{lstlisting}

\subsection{向存储池添加磁盘}

\begin{lstlisting}[language=bash]
lich.node --disk_add all --force --pool p1
\end{lstlisting}

\hl{注意事项}:
\begin{compactenum}
\item 存储池内每个节点上需要有SSD,支持tier功能
\item 存储池内每个节点上需要有SSD,支持SSD cache功能
\end{compactenum}

\subsection{创建卷}

\begin{lstlisting}[language=bash]
# 卷路径规范:<pool>/<protocol>/<volume>
# 三副本
# row2格式
lichbd vol create p1/iscsi/v1 --size 4096Gi --repnum 3 -F row2
lich.inspect --localize /iscsi/v1 0 --pool p1
\end{lstlisting}

\hl{注意事项}:
\begin{compactenum}
\item 卷格式:row2 or raw (default)
\item 三副本 (default: 2)
\item 关闭localize
\end{compactenum}

\subsection{创建快照}

\begin{lstlisting}[language=bash]
# 快照路径规范:<pool>/<protocol>/<volume>@<snap>
lichbd snap create p1/iscsi/v1@snap1
\end{lstlisting}

\section{工具}

省略...

\subsection{iSCSI}

\begin{lstlisting}[language=bash]
iscsiadm -m discovery -t st -p 192.168.251.202
\end{lstlisting}

\subsection{fio}

单机测试

\subsection{vdbench}

支持分布式测试。

在master上配置和查看日志,master配置为可以免密登录slave节点。

slave上需要提前把卷挂载好。

\section{故障测试}

每类节点故障行为不同。除选举过程外,还有vip,iscsi连接,controller的切换,lease,io,恢复过程等。
评价可靠性的指标,主要是vdbench测试中,各种故障条件下io无中断。

另外,故障点还会破坏事务执行的原子性,如allocte过程,创建snapshot过程,
导致严重后果,如造成垃圾,数据状态不一致。如何通过可重入性,或事务解决此类问题?

快照的rollback,delete,flat都设计为可重入过程。如果任务执行失败,可以重新调度。
各种持久化状态之间,保持一致性。

\subsection{单磁盘故障}

磁盘有两种角色:数据盘和cache盘。拔cache盘等同于节点故障?

\subsection{节点故障}

节点有多种角色:
\begin{compactenum}
\item etcd master
\item lich admin
\item lich normal
\end{compactenum}

受VIP机制影响,arp协议会影响客户端到iscsi target的网络连接。
需要注意的是,大部分网络会禁用arp广播,单播则可以。

控制器的加载,lease获取等需要一定时间。

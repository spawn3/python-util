\chapter{OS}

\section{存储系统}

\subsection{应用层}

往上,看应用层的使用场景。磁盘空间管理:分配、释放。这是节点内状态。

访问设备文件时,打开了多个文件描述符,访问策略有所不同。

\subsection{Linux IO架构}

同时,要监听磁盘热插拔事件。监控文件系统的文件变更(fnotify)。
开启启动的过程

定位到启动盘的MBR,上面是boot loader程序,由boot loader程序进一步加载kernel镜像文件。

系统盘的结构:多个分区。一个磁盘的分区结构是怎么维护的?每个分区对应一个设备文件,可以格式化为文件系统。
一个磁盘也可以直接作为设备文件使用。

磁盘-分区-格式化为文件系统-mount到文件系统的某一节点上。

设备文件可以直接使用,也可以分区格式化为文件系统使用。直接访问设备文件提供给上层应用更大的灵活性。
类似于普通文件,设备文件也是通过VFS接口进行管理。

\hl{磁盘设备驱动程序}检测可用设备,并挂载到/dev目录下。udev机制用于用户态设备管理,提供了命令行工具udevadm,
可以检测到设备上线、离线事件。/dev不需要持久化,而是动态注册注销的。

访问设备文件有不同的IO策略,比如writethrough、writeback等,内核维护有设备IO缓冲区。
设备有设备\hl{控制器},也可以引入RAID、LVM机制。\hl{cache是一个重要的主题、关注点}。
数据最终会落到存储介质上。

如何设定策略?hdparm工具。

IO体系结构包括以下主题: 页缓存、块缓存,数据同步,页面回收与交换。
缓存是如何组织的呢?bcache与此有什么异同?

\subsection{device mapper}

cgroup等组件(见wiki百科)。io在不同的设备之间变换,\hl{VDO、bcache}也是这个模式,引入了透明的cache层。
由此可见,该机制至关重要。

\chapter{FusionStor}

数据块及其副本位置是最重要的问题,分配、回收、恢复、平衡、GC都针对数据分布而言。
维护副本数据一致性,EC的情况会更复杂些,需要维护条带数据一致性。

IO路径是主线。

物理资源层次:集群、节点、磁盘。逻辑资源层次:集群、pool、目录、卷+快照。
其中,存储池即是物理资源的划分,又是逻辑卷资源的容器。
这是一个形如X的复用-分用的组织模式。

存储服务分基本功能与高级功能。基本功能包括存储池管理,卷管理与访问控制。

\section{High Performance}

应从硬件与软件两个层面去理解。

高性能版本意味着什么?首先是硬件的升级,包括网络、存储介质、协议。

对软件架构而言,意味着什么?主要涉及操作系统如何管理资源,访问效率如何。

代码优化

社区在做什么?大家在做什么?趋势是什么?

京东云也做RDMA,六人三个月。

存储有着广泛的标准组织,有什么重要的项目?

关注上下、左右各方面的情况。

\section{故障处理}

不同的故障会触发不同的事件,对应不同的处理过程。在故障情况下,即便不考虑恢复流量,业务性能也会有所下降。
必须合理地限制恢复流量,使得业务性能的下降变得可控,这个就是恢复过程的QoS策略。

快速恢复也是一项重要的性能指标。影响恢复性能的因素有哪些?如何尽可能地提高恢复的并发度?
对一个数据块的恢复,需要先读后写,即从源端读取数据,写入到目标端。
所以,源端与目标端的并发度是影响性能的两个重要因素。

目前,一个数据块的副本位置是用节点nid表示的,如果数据块对应的节点nid集合没有变化,就不需要更新相关元数据。
所以,在磁盘故障的情况下,采用了优先选择故障盘所在节点作为目标端的策略,然而,该策略限制了目标端的并发度。

\hl{故障情况下,性能下降很大的原因是什么}?为什么vfm机制可以缓和该问题?

\section{移动集采}

\hl{从移动集采项目,看分布式块存储的知识体系}。

分布式块系统面向四个方面:底层资源、主机、运维平台和云平台。
面向主机提供了iSCSI协议,外加vaai,
面向运维平台提供了SNMP trap,REST,CLI等接口。
面向云平台提供了cinder驱动。

面向主机是本质的方面,数据平面;其它面向是控制平面的事情。上下左右,四面八方。

\section{统一存储}

华为出版的\hl{数据存储技术与实践}一书,分为四个板块:企业存储、云存储、数据库、大数据。

统一存储相对于传统阵列来说的,统一了SAN与NAS,另加上对象存储。并不包括数据库、大数据等存储形式。
\hl{NAS网关}是在块的前面加上一层NAS接口组合而成。

分布式架构是相对于传统阵列而言,软件定义存储,智能存储。

云存储可以从openstack的cinder驱动开始理解,cinder提供了标准接口,封装了下层不同的存储产品。
另外是VMWare的VVol,反而不常用。云存储的关注点在云平台下的存储需求,如京东云硬盘组等等。

云平台与块存储的对接,构成了HCI架构。qumu/kvm如何访问块存储?通过rbd协议,或iSCSI等标准协议。
qemu与kvm是如何整合的?docker等容器技术与vm的优缺点,从而规定了不同的适用场景。

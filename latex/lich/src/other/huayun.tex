\chapter{HUAYUN}

倒计时!

需要整理的资料
\begin{enumbox}
\item 代码
\item 文档
\item book
\end{enumbox}

项目,包括后台和管理系统
\begin{enumbox}
\item SAN
\item NAS
\end{enumbox}

从光点到华云的收获,在光点一直浮在使用系统的层次上,在华云能深入底层基础,更全面地理解一个系统。

编程语言
\begin{enumbox}
\item C/C++
\item Go
\item Python
\item Java
\item Erlang
\item Clojure
\item Shell
\end{enumbox}

接触的系统有:
\begin{enumbox}
\item Nginx
\item MongoDB
\item Redis
\item ElasticSearch
\item Hadoop
\item Sqlite
\item Ceph是下阶段重点
\item 阿里盘古
\item spdk、dpdk
\end{enumbox}

从混合存储到全闪,架构上不同的地方在哪里?高速SSD、高速网络、新的target协议等几个层面。

从功能、质量(性能、可靠性、可扩展性、QoS)几个维度展开。难度在于交叉地带。

难点在于如何降低故障下的IO中断latency?

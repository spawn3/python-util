\chapter{CEPH}

从硬件,存储引擎,存储服务,接口,管理等层次分析一款存储产品。评价指标:多快好省。
多是空间,是扩展性,快是性能,是时间。好是品质,可靠性,高可用等;省,在标准下的优化措施。
两个维度,构成纵横交错的矩阵结构。任何一项功能,都服务于一个或多个指标,
都需从多快好省几个维度去分析(雷达图),只是有所偏重。

\begin{enumbox}
\item 接口层:iscsi
\item 服务层:瘦配置,快照,克隆,cache,分级,QoS,备份,灾备,dedup,压缩,加密(缩减,保护,隔离,安全)
\item 模型层:block/file/object pool,卷,快照,一致性组(服务层是模型中实体的属性和操作)
\item 引擎层:分布,复制/EC,恢复,平衡
\item 硬件层:
\end{enumbox}

存储池,统一了保护域,故障域,pool, 缓存,EC等,都是通过灵活的存储池配置实现的。

存储池关联到ruleset,ruleset指定了bucket和设备的存取规则。

osd对应到磁盘设备,bucket是容器,构成分层的物理拓扑结构。

kernel space和user space之划分

ceph的数据位置计算而来,如何保证平衡是一个重点。

按x型结构理解ceph,ceph突出了object层,在object层之上构建块、文件、对象的统一存储系统。

集群由node组成,node上运行monitor和osd服务,一个节点一般有多个osd进程。
monitor维护集群全局信息,如cluster map。

逻辑资源由pool、pg、对象构成,在pg和osd之间是多对多的映射,所以对象的存储位置由pg决定。
位置计算成本低,但如何保证数据平衡,是一个需要深入论证的课题。
如何管理再平衡过程?

三层架构:target driver、core、storage driver。每层的性能和扩展性如何?

从pg角度来看,一个pg对应有若干osd,primary osd负责它以其为主的各有pg的写入请求。
与协程方式处理的优缺点?

理解对象及其关系后,就要重点分析关键过程,如写、读、恢复、平衡等。
副本和EC处理上有何异同?

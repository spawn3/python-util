\chapter{CEPH}

从硬件,存储引擎,存储服务,接口,管理等层次分析一款存储产品。评价指标:多快好省。
多是空间,是扩展性,快是性能,是时间。好是品质,可靠性,高可用等;省,在标准下的优化措施。
两个维度,构成纵横交错的矩阵结构。任何一项功能,都服务于一个或多个指标,
都需从多快好省几个维度去分析(雷达图),只是有所偏重。

硬件层:

引擎层:分布,复制/EC,恢复,平衡

模型层:block/file/object pool,卷,快照,一致性组(服务层是模型中实体的属性和操作)

服务层:瘦配置,快照,克隆,cache,分级,QoS,备份,灾备,dedup,压缩,加密(缩减,保护,隔离,安全)

接口层:iscsi

存储池,统一了保护域,故障域,pool, 缓存,EC等,都是通过灵活的存储池配置实现的。

存储池关联到ruleset,ruleset指定了bucket和设备的存取规则。

osd对应到磁盘设备,bucket是容器,构成分层的物理拓扑结构。

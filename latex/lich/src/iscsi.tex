\chapter{iSCSI}

\section{IQN}

关于iqn的不变性,iqn是卷的公开标示,供上层应用引用该卷。改变iqn,需要通知依赖于iqn的应用,做出相应的改变。

回到lich的情况,iqn包含了路径部分:<pool\_name>.<image\_name>,跨存储池迁移,rename等操作会改变路径部分。

问题: 可否用卷的volid作为iqn的一部分,替代path,同时保证volid在各种操作下具有不变性?

ceph的做法:
\begin{compactenum}
\item rbd访问方式,用的是路径。
\item 通过tgt提供iscsi服务时,通过tgt配置项建立iqn到path的映射
\end{compactenum}

\begin{lstlisting}[frame=single]
<target iqn.2014-04.rbdstore.example.com:iscsi>
    driver iscsi
    bs-type rbd
    # Format is <iscsi-pool>/<iscsi-rbd-image>
    backing-store iscsi/iscsi-rbd  
    initiator-address <clients address allowed>
</target>
\end{lstlisting}

\section{CHAP}

In function \verb|ns_build_auth_chap|
\begin{compactitem}
\item \verb|lich_system_username|
\item \verb|lich_system_password|
\end{compactitem}

\section{白名单}

\begin{compactitem}
\item \verb|is_connect_allowed|
\end{compactitem}

没有设置ip或initiator,默认拥有全部权限,不符合白名单语义,最小权限原则。
可以切换到黑名单机制。

xattr用于保持ip或initiator白名单,如果很长,则溢出。
需要找到更合适的存储方式。

可以采用编/解码方式存入,通过多个xattr进行扩展。

\section{Initiator}

\begin{lstlisting}[language=bash,frame=single]
echo 2 > /sys/block/sdd/device/queue_depth
cat /etc/iscsi/initiatorname.iscsi

iscsiadm -m node -u
iscsiadm -m discovery -t st -p 192.168.110.219 -l
\end{lstlisting}

看日志

VIP与普通方式有别

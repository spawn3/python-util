\chapter{learning}

\section{学习方法}

知识体系:编程语言、数据结构和算法、架构和系统。
最重要的系统有:操作系统、编译原理和数据库。

分布式存储系统能把这些知识点贯通起来。

找到某些基础模型、或如高焕堂老师说的:form。作为构建更复杂系统的基本单元,
有助于达成以简御繁的目的。

化整为零

临摹

\section{学习计划}

\begin{itemize}
    \item SCSI
    \item NVMe
    \item SPDK
\end{itemize}

\section{学习资料}

\subsection{书籍}

\subsubsection{老马识途}

斩码三刀:猜测-实证-建构

阅读源码的方法:调试-阅读-调试。以调试为方法,动起来,把握主线。所以要熟练掌握gdb等工具。

逆向工程是匕首,答疑解惑。

主要目的在于培养系统观,不以记住知识为高,而以培养系统观为能,就是学会学习的方法。

\subsubsection{完美软件设计}

设计的几个原则:
\begin{enumbox}
\item 正交
\item 分层
\item 时序下的数据流
\item 封装
\item 名实
\end{enumbox}

\subsection{Paper}

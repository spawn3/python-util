\chapter{软件架构}

从节点到集群,从本地到全局

平衡利用各种资源。

\section{引言:C1000K问题}

\section{调度}

\subsection{Actor}

controller类似actor概念。

non-blocking IO。

带队列的线程是一个强大的结构,即有执行单元,又有调度单元。一些复杂的任务,比如恢复过程中的并行执行,就最好组织成带队列的线程池。
至于队列是线程私有,还是公共队列,是一个需要进一步探讨的问题。

再比如,协程调度本质上也是一个带队列的线程,队列不止一个,可以组合成相对强大的调度模型。

\subsection{CSP}

显式channel

\subsection{SEDA}

actor + pipeline,多个actor构成SEDA架构。

\section{Memory}

内存分配也是syscall,经过glib包装后,依然不能满足高性能要求。

\begin{enumbox}
\item hugepage
\item 两层allocator,解决内外碎片问题;
\item buffer\_t解决zero-copy。
\item core thread拥有私有内存,解决共享/竞争问题。
\end{enumbox}

\section{EIT}

\begin{itemize}
    \item journalling
    \item update many, commit one
    \item double check
    \item double/multi buffer
    \item half sync, half async
\end{itemize}

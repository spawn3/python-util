\chapter{数据结构}

array和list是存储结构,大部分ADT都依赖于此。

每个数据结构都要考虑创建和销毁,以及元素的CRUD。

可以有多种实现方式。

\section{认识递归}

array/string/list都可以看做[H|T],对支持双向运动的array/string/double linked list来说,
也可以从首尾两端向中间靠拢。

\hl{子结构,子树},由此可见array/string,list、tree都是递归定义的数据结构。

要深刻认识数据结构的递归定义,递归是分治法的基石。
进一步,引入模式匹配和累加器accumulator,可以\hl{改造递归为尾递归},方便实施尾递归优化。

理解递归,可以形象化为\hl{递归调用树}。一个问题如果缩减为一个小的问题,就是stack。
如果分解为两个或多个子问题,就是递归树。以fibonacci序列为例,如果不进行优化,存在大量的重复计算,则复杂度是指数级的。
可以引入cache机制,把以及计算的结果缓存起来,需要的时候直接查表即可。

substructure的表示法,以array为例,用[offset, len]、or [left, right]。
如果仅仅指定len,则默认offset=0,此种表示法适合从后向前遍历。
注意\hl{差一错误},区间的尾端是开还是闭。

\hl{递归=迭代+Stack}。递归和迭代之间的互换,有时容易,有时则不是那么明显。

\section{Bit}

计算机的工作原理

\subsection{Bmap}

\subsection{BloomFilter}

是一个查找结构。

\subsection{Bmap Index}

\section{Array}

\section{String}

String是一特殊数组,支持随机存取,各种方法需要首先检测输入参数。

翻转、旋转,可以通过翻转实现旋转功能。
如s1=xy,s2=yx,s1s1=xyxy,s2是s1s1的substring。

回文、镜像

\hrulefill

\begin{enumbox}
\item c: char array
\item c++: string class
\item Java: StringBuilder
\item re
\end{enumbox}

\section{List}

\subsection{Single list}

是否引入哨兵?在slist头部增减节点会引起头指针的改变。
引入哨兵,即可屏蔽这种不规则性。

二叉树一旦建立后root很少改变,故不需要引入哨兵。为了统一起见,则slist也不引入哨兵。

问题集
\begin{enumbox}
\item kth,顺序或倒序
\item 求中间节点
\item 逆转、反转
\item 判定是否\hl{回文结构}
\item 倒序输出
\item 判定是否有环,以及环的入口节点
\item 两个slist的第一个重合节点
\end{enumbox}

\subsection{Double list}

\subsection{SkipList}

多通路,大步长

\section{Stack}

Stack用数组即可轻松实现,用slist则只在头部push and pop。

正反序,反序的要考虑用stack和post order遍历

问题集
\begin{enumbox}
\item 两个stack实现queue
\end{enumbox}

\section{Queue}

Queue实现方式很多,如ring数组/动态数组、带tail的slist、两个stack。

问题集
\begin{enumbox}
\item BFS
\end{enumbox}


\section{Hash}

hash函数,hash局部性不佳(影响CPU高速缓存性能),大的hash表最好分组,极限情况是radix tree。
Linux Page Cache采用了radix tree,而不是hash。

\section{Heap and Priority Queue}

TopN,多路归并,从TopN中再选出TopN

\section{Tree}

\subsection{Binary Search Tree}

\subsection{Red Black Tree}

\subsection{Trie}

\subsection{Radix Tree}

\subsection{B and B+ Tree}

\subsection{Huffman Code}

\section{Graph}

\section{Disjoint Set}

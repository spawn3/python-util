\chapter{数据库系统}

从操作系统开始到数据库管理系统,对并发现象的讨论,上升到新的高度。
硬件提供了基础原子指令,利用这些原子指令封装成并发原语,
进一步去实现事物的ACID要求。

计算模型:事务的页模型和对象模型。对象模型是页模型的一个推广,对象模型的叶子节点是页,构成一棵树。

没有什么比好的理论更有用。透过页模型可以推演出很多重要结论。
并发控制算法、事务恢复、分布式提交是由简单到复杂的推演路径。
前两个可以在单节点环境下讨论,然后提升到分布式环境下,引出分布式提交的概念。

答案并非唯一,多种答案之间权衡利弊得失。

四个基础问题:分布及复制数据的透明管理、通过分布式事务的可靠的数据存取、
改进的性能以及更为容易的系统扩展。

事务的支持需要实现分布式并发控制和可靠性协议,尤其是2PC和分布式恢复协议,
这些协议比在集中式数据库里要复杂得多。

\section{并发问题}

并发访问共享对象,竞态条件

\begin{enumbox}
\item 过程是否发生了yield,不连续执行?
\item 临界区有哪些共享对象?
\item 冲突操作
\item 如果有多个并发任务重入该过程,会造成什么问题?
\item 加锁的粒度如何,太大,或太小?
\item 乱序?
\item 有无lock free,wait free方案?
\end{enumbox}

\section{事务管理}

事务模型、详解ACID。

\subsection{隔离等级}

\section{可串行化理论}

单站点的可串行化,单副本可串行化(RSM)

冲突操作:优先图的节点是事务,节点之间的边是冲突操作对。无环时是可串行调度。

\section{2PL}

2PL可以保障事务串行化。注意加锁的顺序和解锁的顺序。

索引结构上的加锁规则

lock-free queue:并非无锁,而是小锁。

\section{并发设计模式}

意向锁?

范围锁?

\begin{compactitem}
    \item double check
    \item fork and join
    \item pipeline
    \item half sync, half async
\end{compactitem}

\section{故障恢复}

基于日志的恢复

REDO

UNDO

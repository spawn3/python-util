\chapter{并发控制}

编程模型:多线程和事件驱动。多线程访问共享内存,需要同步机制。锁会形成死锁。
事件驱动有主循环,不能调用block操作,另外涉及如何动态扩展到多核多CPU的环境。

actor编程模型,SEDA是一个特殊的actor架构。

进程及其通信是基础模型。执行的代换模型与环境模型,事务的页模型与对象模型,
由简单模型推导关键结论。

并行有两种:竞争并行与协作并行。

\section{PCAM设计方法学}

\begin{enumbox}
\item 划分 数据划分、任务划分
\item 通信 任务之间的数据交换
\item 聚合 合并任务,以提升性能
\item 映射 任务调度,并符合负载均衡
\end{enumbox}

\subsection{Example:volume}

volume是一个自然划分,把volume映射到一个core thread上。
问题是这样做的话,粒度过大,无法有效进行负载均衡,受限于单core最大处理能力。

如果一个subvol映射到一个core thread,粒度合适。但需要引入复杂的控制机制。
最好是映射到一个节点的多个core上。

以上分析过程用到了完整的PCAM方法。

\subsection{Example:recovery}

\subsection{Example:rmvol}

无需通过controller去做这些事情,可以直接利用db信息。

\subsection{Example:rollback}

\subsection{Example:flatten}

\section{同步原语}

同步原语组成层次结构。并发原语,层层还原,层层构建

从OS到DB,对并发的讨论在深化,事务要求ACID,且访问特定的索引结构。
另外还有持久化的要求,原子写变得更困难,引入日志。

本质上是序的问题,锁内在地实现了一种序。
逻辑时钟、paxos、raft中,序起到了至关重要的作用。

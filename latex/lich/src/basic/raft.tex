\chapter{Consensus}

理解共识算法并没那么困难,一点点推展开来。化整为零,要善于识别要解决的问题和通用的设计模式。

\hl{2PC and 3PC}解决提交原子性的问题,Paxos解决了xx?

lease

\hl{问题:RSM}。所有日志项构成一个total order的序列。采用WAL+Data。
每个server上的状态机:先写入log,再执行状态机。
\hl{apply状态机必须是连续的},提交日志则可以不连续。

问题集
\begin{enumbox}
\item 如何选主?client如何发现leader?
\item 日志提交方式
\item 如何论证\hl{safety and liveness}
\end{enumbox}

异常
\begin{enumbox}
\item 节点故障
\item 成员变化
\end{enumbox}

\section{Paxos}

微信paxosstore

paxos一个实例确定一个值。多个实例确定多个值。

multi paxos,通过paxos protocol进行选主?

\section{RAFT}

已提交的判定条件特别重要。不能仅仅靠统计同一<term, index>的repnum来决定。

本地Log包括两部分:\hl{提交的和没提交的}。如何判断?RAFT论文似乎采用了最大提交的原则?会不会出现幽灵复现现象?

\subsection{幽灵复现问题}

ABA时,查询不到的记录,再次切回A时,能查询到了。

本质上是上层应用逻辑问题,对没有提交成功的记录如何处理?
没有提交成功,为什么允许读呢? \hl{client retry,要做到幂等性}。

每个leader上任的开始,就在日志里添加一条内容为当前term的记录,没有提交的日志采取discard策略。

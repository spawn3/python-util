\chapter{learning}

\section{学习计划}

知识体系:编程语言、数据结构和算法、架构和系统。
最重要的系统有:操作系统、编译原理和数据库。

分布式存储系统能把这些知识点贯通起来。

\subsection{理论知识}

怎么学习理论知识:书籍,论文,网站,与人讨论等。道心物三者融合无间,方好。

常常琢磨一些重要问题,心态开放。由手头工作延伸开来

及时把问题、思考的进展、与人讨论的结果等一切新知记录下来,留出一定的时间进行反思,整理出知识体系来。


工作相关:
\begin{enumbox}
\item LICH架构,以及关键问题
\item NVMe
\end{enumbox}

重要专业知识
\begin{enumbox}
\item 存储基础知识
\item RAID
\item LVM
\item udev
\item iSCSI
\end{enumbox}

基础知识
\begin{enumbox}
\item 编程语言
\item 数据结构与算法
\item 操作系统
\item 数据库
\item 网络
\item 分布式算法
\end{enumbox}

\subsection{常用工具}

\begin{enumbox}
\item Clion/VIM
\item GIT
\item CMake
\item gdb
\item perf
\item valgrind
\item stap
\end{enumbox}

\subsection{流行技术}

存储系统
\begin{enumbox}
\item SheepDog
\item Ceph
\item GlusterFS
\item ZFS
\item SPDK
\item tgt
\end{enumbox}

计算与虚拟化
\begin{enumbox}
\item Linux
\item KVM
\item VMWare
\item Docker
\end{enumbox}

网站架构
\begin{enumbox}
\item Nginx/Envoy
\item MySQL
\item Redis
\item MongoDB
\item MQ
\item Oracle RAC
\end{enumbox}

大数据计算平台
\begin{enumbox}
\item Kafka
\item Hadoop
\item Spark
\item ElasticSearch
\end{enumbox}

\section{学习方法}

查理芒格的模型:学科的重要模型。

数学概观:现实-模型-理论三元组。模型是对现实的抽象,把逻辑运用到模型推演出理论体系。
通过一问一答解决现实问题。模型的验证,一是事实,而是逻辑。

找到某些基础模型、或如高焕堂老师说的:form。作为构建更复杂系统的基本单元,
有助于达成以简御繁的目的。

化整为零

临摹

师法造化,内得心源。

\subsection{戒定慧}

六度架起此岸、彼岸的桥梁。

\begin{enumbox}
\item 11点之前睡觉、六点起床
\item 问题驱动
\item 一心二本
\end{enumbox}

\subsection{守破离}

以算法为中心,贯通多个领域。

\subsection{包围式学习}

包围式学习合乎画小圈、组块化、刻意练习的理念,达到以我为主,为我所用的效果。

学习的本质是联系,内化这种联系,触类旁通,更有利于应用时的回忆和提取。

E=K/I,温故知新,通过包围式学习构建知识网络。这是主动的开疆辟土,步步为营。

学习、学问这些词汇,有很深内涵,回到中庸的论述。学的是思维方式和方法论,习是刻意练习,
体用、知行、道器一体。器是作品集。

\hl{图论提供了一个很好的模型},神经网络也好,网络流也好,都是图论的变体。
知识点相当于图论上的节点,通过已知节点去探索新的节点。
图的遍历,有深度优先和广度优先等次序。

可以在更普适的意义上去理解数据结构,玩味其内涵。

\subsection{Example:进程、线程、协程}

进程、线程、协程都是一种软件抽象和虚拟化技术。

进程是资源管理的基本单元,具有独立的逻辑地址空间。

线程是调度的基本单元,同一进程的多个线程,共享进程资源,线程可以拥有TLS。
线程是抢占式调度,需要时序方面的考量。

协程呢?上下文切换成本低,用户态的非抢占式调度。

单核,多核,多处理器,逐步引入更复杂的架构,以及需要面对的问题。

% https://en.wikipedia.org/wiki/Algorithm#Informal\_definition

\section{Site}

\begin{enumbox}
\item Wiki
\item 知乎
\end{enumbox}

\section{Paper}

\section{Book}

\subsection{老马识途}

斩码三刀:猜测-实证-建构

阅读源码的方法:调试-阅读-调试。以调试为方法,动起来,把握主线。所以要熟练掌握gdb等工具。

逆向工程是匕首,答疑解惑。

主要目的在于培养系统观,不以记住知识为高,而以培养系统观为能,就是学会学习的方法。

\subsection{算法导论}

\subsection{伟大的计算原理}

六类计算原理:计算、存储、通信、协作、评估、设计。

算法、架构、设计三部曲。

架构:高可用、高性能、负载均衡。

设计准则:需求、正确性、容错性、时效性、适用性。
软件系统的设计原理:层级式聚合、层级、封装、虚拟机、对象、C/S。

\subsection{完美软件设计}

设计的几个原则:
\begin{enumbox}
\item 正交
\item 分层
\item 时序下的数据流
\item 封装
\item 名实
\end{enumbox}

\subsection{设计原本}

\subsection{计算机程序的构造与解释}

\section{技术面试}

算法
\begin{enumbox}
\item quick sort
\item B+ tree、Bcache
\item Radix tree
\end{enumbox}

分布式系统
\begin{enumbox}
\item 多副本一致性
\item Paxos与RAFT的区别
\item CAP理论
\end{enumbox}

NFS
\begin{enumbox}
\item 如何实现NFS cache一致性?
\item mount流程
\item IO流程
\end{enumbox}

系统
\begin{enumbox}
\item lustre 论文
\item OneFS
\end{enumbox}

公司
\begin{enumbox}
\item 阿里
\item 华为
\item 京东
\item 小米
\item XSKY
\item 同友
\item 衫岩
\item 浪潮
\item 曙光
\item 联想
\item 大道云行
\item 首都在线
\item 谷数科技
\item 迅雷
\end{enumbox}

HCI
\begin{enumbox}
\item SmartX
\item ZStack
\item 深信服
\item UCloud
\item EasyStack
\item Zetta云
\end{enumbox}

外国公司
\begin{enumbox}
\item EMC
\item HDS
\item NetApp
\item Nutanix
\end{enumbox}

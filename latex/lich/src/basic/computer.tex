\chapter{体系结构}

组成原理与体系结构,研究内容相当。

\section{结构}

三部分:CPU,存储器,I/O设备。PCIe和NUMA都采用分布式架构,PCIe是串行点对点,
NUMA分远和近,访问远程CPU慢。可以称为分布式共享内存。

二进制

进制之间的转换
\begin{enumbox}
\item 整数(除2取余)
\item 负数(取反加1)
\item 小数(乘2取整)
\item 浮点数采用科学计数法,分单精度与双精度
\item 高位是符号位
\item 表示范围不变,坐标平移
\end{enumbox}

字节顺序:32bit机器上字长是4B,64bit机器上字长是8B。
一个字的字节采用由左到右编码,或从右到左编码。
big-endian表示尾端在高位,little-endian表示尾端在低位。

\section{执行}

\subsection{CPU}

一颗CPU的执行可以看着一维的指令流,由顺序和跳转指令,由调度程序模拟多任务的假象。
并发执行需要进行同步,以避免引起的问题,保证数据的一致性和安全性、执行结果的准确性。

基于lock的同步,可能会引起死锁或饥饿现象,须加以避免、检测。

进程的五状态模型,进程调度算法

流水线并行示意图, SIMD是一个指令处理多个字,多发射或超标量是指一个流水线阶段处理多个指令。
进一步提高了执行的并行度。

协同过程,又称协程,控制权转移,可以不从函数开始处执行。

指令执行,需要在寄存器与主存间同步数据。
主存的值,须加载到寄存器。\hl{CPU指令能直接访问主存内容吗?}

cpu指令,就是在寄存器的协助下,操作内存的内容,驱动设备的输入和输出。
可以通过\hl{port或内存映射}控制外设。

\hl{一定要掌握汇编语言},才能洞幽烛微。

每个core都有私有或共享的高速缓存,多个core之间如何维护缓存需要缓存一致性协议,例如MESI。
每个core有独立的调度器和工作队列。进程的cpu亲和性就是进程可以bind到相应的cpu工作队列,进行调度。

false share

NUMA架构

\subsection{Memory}

进程-进程空间-内存-I/O是核心概念,是理解的主要节点。物理内存不足时,可以借用部分磁盘swap空间,
但这种情况会引起频繁的swap in/out,造成性能抖动。

程序的执行过程:汇编、链接、加载。加载后生成一进程,参与调度。

物理内存大小固定,需要虚拟化,供多个进程使用。进程的地址空间彼此隔离。
由页表和TLB执行逻辑地址到物理地址的映射。\hl{页表是进程级的,多进程之间可以共享物理frame}。

\hl{物理地址如何管理?分配和回收是否需要free list或bitmap一样的索引结构?}

Hugepage可以减小页表的大小,加快检索性能。

进程空间分用户空间和kernel空间。用户进程空间分段管理:text,data,bss,heap,stack。
需要考虑mmap,mmap可以分配hugepage。所谓采用hugepage,就是通过某种机制,获取到hugepage,然后在其上,定制allocator。

jemalloc, tcmalloc都是第三方的内存分配库,可以取代标准库的malloc/free函数,以提升性能。

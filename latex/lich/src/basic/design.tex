\chapter{软件架构}

冰山模型,事件-模式-结构,结构最稳定。

从节点到集群,从本地到全局

平衡利用各种资源。

\section{引言:C1000K问题}

\section{调度}

\subsection{Actor}

带队列的线程是一个强大的结构,即有执行单元,又有调度单元。一些复杂的任务,比如恢复过程中的并行执行,就最好组织成带队列的线程池。
至于队列是线程私有,还是公共队列,是一个需要进一步探讨的问题。

比如在扫描卷的过程中,扫描结果是放入公共队列,还是放入线程私有队列?放入公共队列难于控制一个卷上扫描结果的处理顺序,
而放入线程私有队列,既可以在多线程之间进行\hl{负载均衡},又容易控制处理顺序。

由此可以推论,\hl{私有优先于公有}。在core绑定的设计思路中,也体现了这种理念。

再比如,协程调度本质上也是一个带队列的线程,队列不止一个,可以组合成相对强大的调度模型。

生产者、消费者模型,值得进一步研究。排队论是理论武器。

队列的结构,不仅与线程结构有关,也与应用逻辑有关。

controller类似actor概念。

non-blocking IO。

\subsection{SEDA}

actor + pipeline,多个actor构成SEDA架构。

\subsection{CSP}

显式channel

\subsection{Map Redule}

\subsection{Lambda架构}

\section{EIT}

\section{Memory}

内存分配也是syscall,经过glib包装后,依然不能满足高性能要求。

\begin{enumbox}
\item hugepage
\item 两层allocator,解决内外碎片问题;
\item buffer\_t解决zero-copy。
\item core thread拥有私有内存,解决共享/竞争问题。
\end{enumbox}

\section{场景模式}

\begin{itemize}
\item journalling
\item update many, commit one
\item double check
\item double/multi buffer
\item half sync, half async
\end{itemize}

\section{QoS}

卷QoS实现了局部目标,全局QoS控制更难。系统存在多种活动,活动有优先级控制,比如业务IO和恢复流量之间如何平衡?
QoS也与系统拓扑结构有关,即控制粒度,分层。

\chapter{FusionStor}

架构演进之路:v2,v3,v4。TODO:v5

v2: mds ha静态配置

v3:引入paxos,解决mds ha问题。

v4:引入controller,scheduler,sqlite解决性能问题,引入etcd解决ha问题。

v5:
\begin{compactenum}
\item 树形结构纳入etcd
\item vc放入etcd,包括volinfo, xattr, snap
\item subvol封装,头部记录在etcd
\item chunk映射到磁盘id,uint16_t最多65536块盘
\item 取消sqlite,chkid到磁盘映射,bitmap封装到disk
\item disk封装成对象,支持lfs,raw,bluestore,spdk,libnvme,cmb(control memory buffer)
\end{compactenum}

稳定可靠高性能是下一步演进的指导方针,对标solidfire。

分布式系统的编程0-4级金字塔模型,运行在tcp上的multi paxos。

整个性能优化的主要思路,一是减少cs开销,二是减少内存copy。

cpu的执行模型,上下文切换影响性能,破坏局部性。coroutine的上下文切换也有较大开销,但是比线程要小得多。
减少调用嵌套深度和局部变量的使用,有助于提高cpu cache命中率。cache color。

采用pipeline机制,可以降低上下文切换成本。参考tgt。paper:stage。

性能预测:cpu按时间片执行指令序列。用lmbench观察,比如cs过多,且每次cs耗时过久,会验证影响性能。
cpu的执行,可以看着一个时间轴,或cs,或block,或执行有效指令。各种lock机制会涉及多core同步。

ceph采用了很多的线程,所以单核iops只有1w左右。更多并发并不一定有更大的iops。

spinlock和rwlock的实现方式有所不同,需要深入理解不同之处。
在分析高性能网络服务器时,没有可以忽略的小事。

TLB: hugepage, page fault, malloc不一定分配物理内存,只有在后期使用的时候,才会分配,有可能导致缺页中断。
calloc会置零分配的内存区域。采用hugepage内存池,有助于解决这些问题。

内存对齐:分配时对齐,或submit时对齐,哪种效率高?

MESI cache一致性协议。

重要的项目:tgt,kernel,nginx多看。在big picture下的前提下多看。
分析方法和工具:iostat, vmstat, top, perf,lmbench等。

与阿里对比:阿里做的较深,多核调度,qos,io稳定性。(这是一个很好的议题)

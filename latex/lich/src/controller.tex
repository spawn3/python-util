\section{控制器}

目录和卷,都通过controller进行管理。目录controller的概念有待进一步完善。

每个控制器都可以看着一chunk树,其根节点对应chunk的副本位置列表,决定了控制器的位置:列表中第一副本所在节点。
在迁移控制器时,同样需要遵循该规则,所以需要先调整根chunk的chkinfo。

所有卷的操作都需要通过controller进行。客户端在访问一个卷时,第一步要找到该卷控制器的所在节点nid,
然后把nid作为参数传入后续调用中,如nid是客户端,则进程内;否则,发起rpc调用。
本地访问也可走rpc,可以利用rpc timeout等特性。

md\_map是控制器位置缓存,如可以在cache里找到,直接返回。如缓存不命中,则发起UDP广播。
每个lichd进程有独立线程监听端口:20915。检查cluster uuid,magic,crc等匹配后,尝试加载控制器,然后做出回应。
发起UDP广播的客户端收集各lichd进程的响应,如找到匹配的nid,可以直接退出该过程。

控制器加载过程:第一副本非本节点,返回EREMCHG。加载成功后,用vctl缓存管理起来,缓存项带引用计数和删除标志。
目前,采用lease机制保证vctl的唯一性。其必要性可进一步推演。
加载过程需要保证并发下的唯一性,如果有多个task发起加载过程,只有一个实际执行,别的进入等待队列。加载完成后,唤醒等待队列里的任务。

\chapter{Snapshot}

\begin{tcolorbox}
需要支持的快照操作:
    \begin{compactenum}
        \item create
        \item rm
        \item list
        \item rollback
        \item clone
        \item cat
        \item flatten
        \item protect/unprotect
    \end{compactenum}
\end{tcolorbox}

\begin{tabular}{|s|p{0.6cm}|p{0.6cm}|p{0.6cm}|p{0.6cm}| }
    \hline
    \rowcolor{lightgray} \multicolumn{5}{|c|} {Snapshot} \\
    \hline
    Feature & COW & ROW1 & ROW2 & LSV \\
    \hline
    snap tree & N &  & & \\
    \hline
    snap create & N &  & & \\
    \hline
    snap rm -> gc & N &  & & \\
    \hline
    snap rollback & N &  & & \\
    \hline
    snap clone -> read & N & & & \\
    \hline
    write & N &  & & \\
    \hline
    read & & N & & \\
    \hline
    space & N &  & & \\
    \hline
    consistency group & N &  & & \\
    \hline
\end{tabular}

采用COW,ROW或两者的组合形式,各有优缺点。

COW方式的快照,卷有完整的索引结构。别的快照点,只有增量的索引结构和发生更新的数据。
每个快照,存储的是创建之后,到下一个快照点之间发生更新的所有数据块。所以需要尽量降低发生copy的开销。
适用于频繁且具有局部性的热点负载场景,固定时间段内,每次copy的开销以及\textcolor{red}{复制集的大小}。

ROW方式下,如果meta不发生COW,新的快照点并无完整的索引结构,读过程需要沿着快照链向上回溯。
并且,rm,rollback等操作需要合并快照点。

如果发生了COW,索引项和数据项引用关系不再是1:1,而是多对多,需要专门的GC机制。
但rm,rollback等操作实现起来变得简单。

数据项的粒度,定长或变长,不同的负载,读写性能有不同的影响。
发生COW的粒度,索引结构管理的粒度决定了数据项的粒度。

IO粒度和数据项粒度的关系,如果IO粒度大于数据项粒度

如果IO粒度小于数据项粒度

这条规则是否永远成立:\textcolor{red}{快照树的任一路径,都需要具有完整的索引结构和数据项集,可以有冗余和共享}。

% ROW与LSV的不同,在于同一LBA,只对应一个数据项,而不是对应多个版本的数据项。
% 但这个数据项的引用计数,可以为0,1,或多个。

ROW,快照会产生一个数据项的多个版本。LSV,除了快照之外,一个LBA的覆盖写入也会产生多个版本。

卷和快照,共存于底层lich卷,是否有问题?

元数据的\hl{基本管理单元是page}。按page来组织,每个page unit:chunkid+pageid。
在ROW的过程中,可以只改变一页到新chunk。而bitmap则发生了COW,copy 1M,改变其中一项。
(用chunkid+ page bitmap的方式行不通)。\hl{有一种特殊情况,bitmap管理了连续1M的数据,此时chunkid是重复的}。

\begin{tcolorbox}

假设有新旧两个快照点S1, S2,每个快照点包括meta和data,提炼出的几个引导性问题:

\begin{enumerate}
    \item 数据块的粒度,page,chunk,extent?
    \item 每个快照点是否有完整的索引结构?
    \item 卷(写入点)是否有完整的索引结构?
    \item 哪一个是写入点?
    \item meta是否发生COW?
    \item data是否发生COW?
    \item 快照操作(create, rm, rollback, clone, flatten, ls, cat, protect/unprotect)的复杂度?
\end{enumerate}

\end{tcolorbox}

\section{COW}

\section{ROW2}

\subsection{快照树}

快照树的每个节点,都是一个完整的索引结构。其中一个代表着当前写入点,写入点代表着卷。
回滚操作会改变当前写入点。

统一一下,卷和快照具有相同的索引结构。

每个快照结构有唯一的ID,是随着创建快照的过程递增的,卷快照/写入点具有最大的snap id。
LSV log结构记录了每页的snap id。

一颗树的节点,可以分为三类:
\begin{compactenum}
    \item 根节点
    \item 中间节点(有无分支)
    \item 叶子节点
\end{compactenum}

\subsection{创建快照}

原有写入点变成只读,创建新的写入点,并复制L1元数据。

\subsection{删除快照}

快照树上不同的节点,需要不同的删除过程。对于叶子节点,直接删除即可,\textcolor{red}{需要回收数据吗}?

回收快照,涉及meta和data两个部分。

不应改变父快照的内容
保留子节点的共享内容

\subsection{列出快照树}

\subsection{写}

写是ROW的重定向过程,可能发生复制第二层元数据的过程。

如果连续写入,需记录当前的写入点。chunk内数据是逻辑不连续的,或者说,逻辑上连续的在chunk内是随机化了。

如果非连续写入,每页的写入位置计算得到,chunk内数据是逻辑连续的。大范围随机写入的情况下,需要分配很多chunk。
\textcolor{red}{需大力优化chunk的分配过程}。

pagetable是逻辑连续的,按逻辑空间组织。如果已分配位置,则覆盖写入。如果没有分配位置,如何分配?

逻辑地址怎么映射到物理地址?页式,段页式

\subsection{回滚}

回滚并不会重用回滚到的快照点,而是相当于把写入点嫁接到目标快照点。写入点本身是一个独立的快照结构。

除了写入点的所有快照点,都处于只读状态,没有任何操作可以改变其状态。

回滚后的写入过程

\subsection{读}

如果不复制元数据,ROW实现的读过程需要回溯快照树,性能不佳。
如果复制元数据,则每一快照点都具有完整的索引结构,可以做到一次即可定位。

复制元数据,快照和数据具有多对多的引用关系,相当于共享数据块。

读优化: 元数据可以一次定位,但可能碎片化,沿着快照链往上读取。可以通过flatten的过程优化。

\subsection{Clone}

clone后,会用到跨卷读快照,类似于ROW过程,所以,创建和clone过程具有相似性。

cat, protect, unprotect, flatten

\subsection{GC}

GC过程和引用计数。每个记录的is\_ref记录是否写入了数据。可以回收写入点,但无法回收一般快照。

LSV: 逐个扫描每个log。

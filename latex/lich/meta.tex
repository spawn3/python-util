\chapter{软件架构}

meta管理采用对称的中心化架构。在节点中选举出admin节点,
管理全局状态和数据分配工作。当admin节点发生故障时,会发生
failover过程,选举出新的admin节点来。

lichd进程内嵌各种server,包括iscsi等。

作为存储系统,主要考虑是元数据组织和IO,恢复等关键过程。
功能之外,性能,可靠性,可扩展性至关重要。

client可以和每个节点进行通信,推荐采用VIP机制,简化连接管理。

\section{元数据管理}

元数据包括:
\begin{enumbox}
\item \hl{引导信息,用于加载相应对象}
\item 目录下包含哪些文件?
\item 文件包含哪些chunk?
\item chunk的副本位置(节点和节点上的磁盘)
\item 快照以及快照树
\item 各实体对象描述信息,包括创建时间,id等
\item xattr
\end{enumbox}

关键问题有:
\begin{enumbox}
\item 空间管理,分配和回收
\item 数据分布
\item 复制/EC
\item 数据恢复和平衡
\end{enumbox}

在顶层设计下的case by case。

每一个卷的元数据和数据,共有三层chunk,L1和L2是元数据,L3是数据,构成chunk的单根树。
快照和clone出的卷采用统一的chunk树结构,不过增加了交叉引用关系。

L1原先只有一个chunk,有固定数目的槽位,指向L2的chunk,L2的chunk,有固定条目的槽位,指向L3的数据chunk。

table1包含了指向全部L2 chunk的指针数组(table\_proto),table2包含了指向全部L3 chunk的指针数组(chkinfo+chkstat)
table\_proto内在也包含chkinfo和chkstat。每一个chunk,都需要在其父节点上登记,\hl{卷的第一个chunk}登记在pool里。

如果\verb|table1->table_count|和\verb|table2->chknum|比较大,会有扩展性和性能方面的问题。


\section{节点}

三类服务器节点:
\begin{compactenum}
\item admin
\item meta
\item normal
\end{compactenum}

从meta节点中选举出admin。meta节点是静态指定的吗?meta节点列表构成一个小的集群,类似于ceph的monitor。

normal节点是数据节点,存放元数据和数据。元数据和数据都是按1M的chunk组织。
元数据包括四类chunk: pool,subpool,vol,subvol。每个chunk具有固定数目的槽位bucket,指向管理的下一级节点。
集群内的所有chunk,构成一个单根树。每个pool,每个volume,都是这个大树下的子树。

引导信息bootstrap:rootable。记录了根分区的位置,即所在chunk的位置信息。根分区是一个特殊的pool。

所有的控制器,包括pool和volume控制器,通过lease机制保证集群内的唯一性。
其节点位置,由所在子树的根chunk的主副本确定。

chunk副本的节点分布:基于diskmap的随机分布算法,并记录在元数据里。

chunk副本的磁盘分布:本地数据管理(bitmap+sqlite)

chunk副本之间的一致性:强一致性协议。

\subsection{admin节点}

职责:
\begin{compactenum}
\item paxos leader
\item lease server
\item 管理系统引导信息
\item 集群节点列表
\item 维护diskmap
\item VIP(控制器当前位置)
\item 分配卷Id
\item 分配\hl{chunk副本位置}
\end{compactenum}

持久化信息和上报信息,心跳机制

可扩展性

admin上持久化的信息,\hl{是如何避免单点故障的,即如何同步到各个meta节点上的}?可以rootable为例,参见\verb|mq_master.c|。

\subsection{数据目录}

/opt/fusionstack/data:
\begin{itembox}
\item node/rootable (\hl{global info})
    \begin{compactitem}
    \item sysroot/root
    \item misc/fileid
    \item node/t53
    \end{compactitem}
\item disk (local, disk management)
\item chunk (local, sqlite3)
\item status
\end{itembox}

\subsection{引入etcd}

\section{VIP}

\section{微控制器}

每一个加载的卷,对应一个集群内唯一的卷控制器,负责卷的IO等操作。

\section{数据分布}

数据分布首先要满足规则要求,其次则需做到均衡和局部性。
规则是强制的,极端条件下可能退化。均衡和局部性则影响系统性能。

卷属于存储池,存储池上定义副本数和副本放置规则。
保护域和故障域,可以用存储池来统一。

FusionStor通过元数据来管理,\hl{与CEPH的CRUSH有重大不同}。

数据分布规则有:
\begin{compactitem}
\item 存储池规则
\item 保护域规则
\item 故障域规则(\ref{rule:faultset})
\end{compactitem}

负载均衡和本地化两方面考虑,平衡包括数据平衡和任务平衡。

chunk在节点上的分布,节点内chunk在磁盘上的分布(包括分层)

controller在节点上的分布,controller在core上的分布

各种任务的分布情况,如数据恢复。

\subsection{负载均衡}

\subsection{本地化}

卷控制器所在节点,具有所有chunk的副本。

当切换控制器的时候,需要控制本地化过程的QoS。

\section{复制}

\begin{tabular}{|s|p{6cm}|}
    %\hline
    %\rowcolor{lightgray} \multicolumn{5}{|c|} {Snapshot} \\
    \hline
    pool & LICH\_REPLICA\_METADATA  \\
    \hline
    subpool & LICH\_REPLICA\_METADATA  \\
    \hline
    vol & fileinfo \\
    \hline
    subvol & fileinfo  \\
    \hline
    raw & fileinfo  \\
    \hline
\end{tabular}

恢复和再平衡过程,是怎么工作的?

\section{分层}

卷的xattr,有目标分层设定:priority。默认是-1,即开启自动分层。先落入tier 0,根据数据热度,
通过异步过程进行数据迁移(纵向的数据流动)。所以有两个异步任务:
\begin{compactenum}
\item 统计访问热度
\item 按策略执行数据迁移(目标分层的偏离,数据热度)
\end{compactenum}

每个chunk有实际分层:tier,不是在副本级。

\section{SSD缓存}

SSD Cache实现了写缓存,没有实现读缓存, 通过内存实现读缓存。

\subsection{配置项及系统行为}

相关控制参数:
\begin{enumbox}
\item lich.conf/disk\_keep: 10G (废弃)
\item lich.conf/disk\_cache: 10G
\item 卷的xattr: writeback
\item 卷的属性:priority (设定卷的目标存储分层)
\end{enumbox}

disk\_cache配置磁盘cache区,对新盘有效,若一个盘已经配置,就固定了,不再改变。
此配置对SSD和HDD都起作用,预留磁盘末尾的空间。

priority是手工分层机制,持久化到卷属性上。

tier是自动分层机制,默认行为是什么? 优先落到SSD上,若SSD满,落到HDD上。

所以,分三种情况:
\begin{compactenum}
\item 自动分层,priority == -1,优先落盘到SSD上,如果SSD满,落盘到HDD上
\item priority == 0, 同自动分层机制
\item priority == 1,落盘到HDD上(此时根据writeback的设置,决定是否走SSD cache)
\end{compactenum}

写IO流程

以下两种情况,会落盘到HDD上:
\begin{compactenum}
\item priority == 1
\item SSD满
\end{compactenum}

落盘到HDD的写IO,\hl{其行为受xattr.writeback影响},分两种情况:
\begin{compactenum}
\item xattr.writeback == 1,写入ssd的cache区+内存后,返回。
\item xattr.writeback == 0, 写入HDD。
\end{compactenum}

落盘到SSD的写IO,不经过Cache, 走原来的IO路径。
\begin{compactenum}
\item 自动分层,会优先落到SSD上
\item priority == 0, 会导致落盘到SSD上
\end{compactenum}

已知问题

\begin{compactenum}
\item 原来的分层机制,默认情况下,会写满SSD,导致后续写HDD(先快后慢)。
\item 异步线程的工作机制 (周期性执行,20min执行一次, 最后才会置换proirity == 0的数据)
\item disk cache区开启后,就不能再改变,空间无法回收,且不能关闭(不可逆过程)
\end{compactenum}

不确定的地方

考虑以下问题,初次分配dispatch\_newdisk,等固定了分层后,后续的变更规则是什么?

\section{关键过程}

\subsection{磁盘管理}

每块磁盘对应一个bitmap,用于该盘的空间管理。

磁盘有分层属性,通常0表示SSD,1表示HDD。有三种分层策略:tier==0,表示写入SSD,tier==1表示写入HDD,
tier==-1表示自动分层,先写入SSD,通过异步后台线程flush不活跃的数据到HDD。

在分配每一个chunk的时候,可以指定tier。没有指定的情况下,默认为卷的priority设定。

chunk\_id到磁盘物理地址的映射,是一随机过程,定位到空闲的bitmap上。

如何确定磁盘的分层?

RAID管理,disk和raid都有cache,需要注意掉电情况下是否丢数据。

\subsection{分配一个chunk的过程}

函数:
\begin{compactitem}
\item \verb|__table2_chunk_create|
\item \verb|replica_srv_create|
\item \verb|disk_create|
\end{compactitem}

与admin交互,返回节点列表,即各副本所在节点。

需要持久化的信息:
\begin{compactitem}
\item table2 meta,记录chunk info(副本位置)
\item sqlite3,记录chkid(副本)到物理地址的映射关系
\item disk bitmap,记录磁盘上每个chunk的分配状态
\item 填充chunk内容为全0?
\end{compactitem}

分配chunk的过程,会影响到若干特性,如精简配置,恢复,再平衡,写入等,都产生新chunk。

优化allocate的性能:
\begin{compactitem}
\item 异步化sqlite,每个db一个工作线程
\item 加大lich.inspect线程数到20
\item table2 lock粒度 \change{dynamical lock table}
\end{compactitem}

\subsection{IO过程}

写过程,可能内在地包含了分配chunk的过程,缺页分配。当在末尾写入时,还可能扩展了卷的大小。

大范围内的随机写入,造成很多的缺页分配,分配过程会成为性能瓶颈。


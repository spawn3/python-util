\section{05}

\subsection{05}

primary在节点间分布不均,导致各节点恢复进度差别较大。

每个节点上恢复速度受什么制约?

恢复性能分析:用流体动力学模型。分调度器和下游处理节点。调度能力应大于下游处理能力。
怎么评估main thread的调度能力。

性能分析工具
\begin{enumbox}
\item perf
\end{enumbox}

网络层RPC
\begin{enumbox}
\item 并行
\item libev
\end{enumbox}

磁盘修复不回溯stale数据,为什么?

\subsection{05}

测试发现,从某个epoch起,\hl{scli node recovery总是返回nan},即便停止故障模拟,recovery无法进行。

看standby相关日志,可以很容易发现问题所在,后续收到的只有193的消息,没有收到过192和192的消息,为什么?
进而仔细分析\hl{start recovery和run next rw}两个过程。

run next rw里的cur在某个时间点之后再没有变更过,说明了什么问题?其状态为prepare list,无法推进了。
进一步的分析表明,\hl{该rinfo没有进入recovery的任务队列}。

多次recovery被压缩为两次事件,需要小心管理\hl{recovery的上下文切换}过程。
worker线程用到的rinfo,不允许被main thread free掉。上下文切换的过程,目前的实现也涉及与zk的交互。

不变式:任一rinfo,都需经过free过程。满足malloc/free的对称性。可以用上下文切换的语义建模,辅助理解。

每次recovery周期,update epoch可能被调用一次或两次,视start recovery里新创建的rinfo是否放入next rinfo而定。

\hrulefill

可以用egrep的OR方法,保留多个相关日志的相对顺序,便于分析。如\hl{start\_recovery|free\_recovery\_info|standby|prepare}模式。
欲设计出诊断问题的模式,需要对系统的运行机制有全面而清晰的理解。

发现的问题:
\begin{enumbox}
\item retry逻辑,导致当前恢复周期很长时间才退出,
\item 192异常退出后,191一直恢复失败。
\item 数据跑到stale下面,恢复没有完成,vdi list堵塞。
\item 大量数据被放置在stale下,导致磁盘空间满
\end{enumbox}

移动对象容易带来问题,\hl{需要设计不移动对象的数据管理方案}。
另外,如果采用raw disk方式管理磁盘,如果轻量地移动对象呢?raw disk可是没有rename特性的。

完成队列采用polling机制,不一定要main线程去做?

内部retry逻辑影响到run next rw,进而影响到整个系统的响应能力。
block式网络io也有很大问题。

\hrulefill

\hl{worker thread request done}包括多个完成队列,如rx main、tx main、recovery object main等。
rx main和tx main需要进一步测量其性能,不能是慢操作。

或者是有\hl{多个main线程},去承担节点内所有负载?
切分任务到不同的main线程,任务进行局部化管理。参考lich的core线程设计。

选择什么标准划分任务?任务类型包括:IO、recovery等。

rpc和corerpc的区分:corerpc处理与对象IO,其它请求通过rpc来做?
对象IO可以按照对象hash到不同的main thread上。部分请求是不涉及对象的。

\hrulefill

notify standby发起消息,当master收集到足够的消息后,触发recover peer过程。

\hrulefill

QoS,不限制需要优先恢复的对象。

埋点在工作线程上,故需要线程安全。

\subsection{07}

准备切入全闪方面的工作,ssan的工作大概到五月底。

分布式块存储系统是能力核心,扩大了说就是分布式系统,体系结构方面。
AI是下几年的学习重点。\hl{向Jeff Dean学习},能横跨体系结构和AI多个领域。
合而言之,就是ABC的细分领域。

全面认识ssan的价值,部分c代码和设计是可以重用的。

wd和stale可以理解为COW过程。update epoch是创建对象snapshot的过程。

\subsection{08}

192上的ssan无响应,用\hl{strace/gdb跟踪}了一下,发现陷入sd read lock的循环内。
进一步推测在finish object list里访问rinfo是无效的,被free了。

进一步检查日志发现,rw被生成了两次,一次是init状态,一次是prepare状态。
在执行第二个rw时,关联的rinfo已被freed。

sd read lock的实现,类似于retry逻辑,在某些情况下,会严重影响系统的响应能力。
main thread无限循环,系统永远不再工作。

在update epoch之后,如果无法完成修复,会有大麻烦。

\hrulefill

可追踪性是一项重要的能力,举凡log、stap、gdb、perf,都意味着trace。

egrep分析日志的模式:
\begin{enumbox}
\item start recovery | free recovery info
\item default update epoch
\item ***
\item standby | recover peer
\item queue recovery work
\item prepare | finish object list
\item ***
\end{enumbox}

\hrulefill

改进ssan
\begin{enumbox}
\item 大量LOG
\item 引入GOTO、assert
\item 产生coredump
\item check and set rlimit
\item 双进程结构
\item 可扩展的main线程
\item 调整epoch目录结构
\item 并发网络请求
\item rinfo状态机,并发下的引用计数
\end{enumbox}

\hrulefill

战略要义在于一。\hl{舍九取一,取一是主要方面}。有所为有所不为,有所为是主要方面。

没有记录的公司是没有希望的,同样,没有记录的个人成长也值得怀疑。
一旦确立长期观点,就可以深谋远虑,变得从容主动。

一是综合判断,聚焦,不排斥多元的变化,蕴涵无尽可能。

道生一,无极而太极。一即是战略,又是战术。后面的数字偏于分析。

吾道一以贯之。天下之道,可一言而尽也:其为物不二,则其生物不测。

少年得志可喜,大器晚成可贵。有目标,沉住气,踏实干。

天下之动,贞夫一者也。找到人生的这个一,即是使命。

老子传奇里,某位真人对老子的教导,就强调了这个一字。阳明归纳而又归纳,得出致良知一法门。
孔子曰:吾道一以贯之。商君一言,利出一孔、利出一孔,处处显出一的可贵。有个这个一,则万象生。
爰有奇器,是生万象。

\hrulefill

update epoch是非常危险的操作,且耗时,且由main thread来完成,故必须优化。

\subsection{09}

ssan测试出来的问题
\begin{enumbox}
\item update epoch耗时极多,堵塞main thread
\item 加入节点时,do get vdis发生panic,原因是其它节点在进行耗时的update epoch
\item stale object GC
\item disk full,恢复无法继续,会丢数据
\end{enumbox}

\hl{重新format集群时,需先行卸载各个卷}。否则,貌似有问题。

rinfo状态机
\begin{enumbox}
\item 何时放入工作队列
\item oid in recovery
\item 能否取消,开始下一周期
\end{enumbox}

\hrulefill

性能之巅一书,结构很是合理。\hl{文件系统和数据库系统},复杂度很高。

ssan的精髓是什么?

\section{08}

\subsection{01}

论自由,不仅是论,更在于得到。自由不仅是认知问题,更是实践问题。恒以一德,此一德就是自由。

庄子的自由离活泼泼的现实生活有点远。行动自由是兵家必争,自由则含义更广。抓手在哪里?

2019是自由元年,经过多年磨砺,心智渐趋成熟,可以更自由地去呼吸、去奋争,去为所欲为。

\hrulefill

计算机体系结构要好好学习,胡伟武的教材为主。在一个更广泛的范围内考虑架构问题。
\begin{myeasylist}{itemize}
& 伟大的计算原理
& 多核应用架构关键技术
& 排队论
& 响应式架构
& 计算机体系结构
& 性能之巅
\end{myeasylist}

\subsection{02}

\hl{hegel的哲学,与毕建勋的三合之道}很匹配,互参互证。精神从逻辑学开始,下降到自然界,再回归到精神自身,
从空泛到充实,充实之美。在征途中,攻城略地,海纳百川的气象,又有着攻城略地的开拓进取。

为什么把逻辑置于山顶?心物二元借着道的中介作用,交融为一。

年薪百万是个坎,并不难逾越。应有计划地突破之。此为目标的量化。

拿出三分精力做管理工作,应有章法,目标导向,严格要求。

\subsection{06}

思维科学
\begin{myeasylist}{itemize}
& 一二三哲学
& 轩辕三书
& 道德经
& 王阳明
& 毛泽东
& 大秦帝国
& 黑格尔
& 波普尔
\end{myeasylist}

绝利一源,用师十倍。

\subsection{07}

\begin{shadequote}
读书就应该读功用最大,价值最高的书。人们所做的一切不都是为了有所功用吗?小功用的书比比皆是,通俗易懂,无须费多少心力即可掌握一技一长。
但这是舍本逐末了,既然要读书就应当选择探究终极之道的书,以道贯通天下万事万物,实现最大的功用。

像这样至高经典的学习,决不能跟学习普通知识一样,看几遍就扔了。若想达到极高的效用,一定要熟读成诵本书,一时理解参透不了的,也不必着急,就像牛反刍一样,经典熟记了,
就会在你的思想中蕴化,为你的思想贯通万事万物打下基础。若要思想升华,只读《阴符经》肯定是不够的,因为道是幽微玄奥的,几部经典互通参照,才能更好的认识道。
在后面的文章中。道易学宫还会继续推出诸如《老子》这样的经典。
\end{shadequote}

\say{阴符经、道德经皆为至高经典。吾道一以贯之,用道贯通天地万物。三合之道中道居于顶点,无复多疑,一定要理解其中深意。
阳明良知说,弘扬人的主体性,但并非可以取代道的绝对性。在hegel哲学里,绝对理念是贯穿始终的东西,一步一步到达真理之境。}

\enquote{原则是道的简化,有限版的道。道是开放体系,hegel哲学并非如马克思所云是封闭的脚手架。是有限与无限的统一体。
知性和理性的区分非常重要。\hl{西方辩证法如何过渡到东方辩证法,情、势、节}?}

\hrulefill

\hl{大处着眼,小处着手}。好好悟这个,大处是思维境界,小处是工作境界,吾道一以贯之。
思维、工作、道法构成三合之道。升维思考,降维贯通。一升一降,生命之圆运动是也。

狐狸和刺猬,

继续\hl{租房}住吧,省事了,主要是省时间,省下来的时间好好用来学习,提高自己,再给一年时间,需要一个质的变化。

\subsection{21}

\hl{轩辕三书},千古之绝唱,无尽之宝藏。

用行动改造理念并形成制度。做事、成事,又不是盲目地做。

\hl{一之解,察于天地;一之理,施于四海}。一即是道,也是事。实事求是,才能一以贯之。
两者是贯通的、不是支离的。道寓于事物之中。在观念上是形而上的。
hegel哲学所揭示的,精神的力量、理念的力量、认识的力量甚至比物质的力量要大。
谋事在人成事也在人。

谋势、积微,缺一不可。

提出\hl{一的哲学},合一的力量。三合之道也不是究竟,上遂而及于一的哲学。

知行合一。

实践论、矛盾论

唯物论、辩证法

\subsection{23}

我的第一本书将叫做第一哲学、一的哲学。

\subsection{26}

一的哲学,简称一学。一学非一休,一文一武,一张一弛。

夫为一而不化,得道之本,得事之要。抱道执度,天下可一也。

化书,御一。

一能贯五,五能综一。

吾道一以贯之。知行合一。

用pcda做好管理工作。

\subsection{27}

一学,配合pdca作为主要的管理工具,pdca的每个字母都是一大课题。转动pdca循环,解决现实问题。
比如,联想方法论的复盘,就是c的深化。

目的是什么?思想方法和工作方法,更重要的是学以致用,用方法指导实践活动。
归纳下来,就是一个知一个行,知行合一。

\hl{道治天下},心道物三合,道具中枢地位。以道通物,以道观天下。
原则、多元思维模型都是法,法自道生。事督乎法,法出乎权,权出乎道。

四度,\hl{春生夏长秋收冬藏},周而复始,与pdca相当。

\subsection{29}

\hl{代码才最重要,要坚持写代码},在清晰思路下写代码。
所以至少需要在两个层面上思考,\hl{一是产品化,二是代码实现}。

管理丧失了焦点和主线。deadline

场景思维,用户思维,\hl{用起来怎么样}?不能陷入技术思维。
这是amazon的那个不变的一。用户是产品的使用者,客户是埋单的人。

第一性原理演绎出整个系统,逻辑奇点是第一性原理的基石假设。
奇点下移,边界外延的方式,实现成长和进化。

逻辑奇点+公理化方法=第一性原理?还是\hl{第一性原理+公理化方法=系统}。

逻辑奇点是第一性原理的能量之源,第一性原理是太阳,系统就是太阳系,逻辑奇点是太阳的能量来源。

组合创新、错位竞争、价值网络,基本要素组合成生态位。

没必要太强调颠覆式,华为的成功是致广大而尽精微的结果,围绕着主航道持续投入,这是正规军的打法。打的是阵地战。
运动战和游击战穿插其中。

S曲线是事物的成长曲线,分析全闪产品的演进。还没有过破局点,潜龙勿用。
技术上的单一要素,单一要素最大化,击穿破局点后,用战略杠杆模型乘胜出击,建立系统、生态、平台。
单一要素最大化,就是\hl{绝利一源用师十倍}的战略战术。

回到孙正义的双乘兵法。一流攻守群。

道者,一其号也,虚其居也,无为其素也,和其用也。此语极精妙。

\subsection{30}

全面使用pdca循环,做好个人成长和日常工作的管理工作。

pdca之外,还有逻辑奇点的问题,不能忘。

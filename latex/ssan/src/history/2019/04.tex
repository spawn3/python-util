\section{04}

\subsection{01}

区分四种消息类型,特别是local和peer,区别很晦涩。

recovery有main thread get的用法,只能在主线程里使用。什么时候用到呢?

主线程做了太多的事情,包括所有完成队列的处理,会成为性能瓶颈点。

理解ssan线程模型。在main函数里启动了一个event loop,构成主线程。
另外有若干\hl{消息队列及其线程池}。

特别要注意是一个函数是在\hl{主线程或是工作线程}里被调用的。
终于明白为什么要用work和main来命名函数了。

\subsection{02}

\subsection{13}

分析理念:生命周期分析法,每个对象及其上面的操作,构成对象的生命周期。
对象包括:数据对象、一次IO过程、recovery过程等。

对一致性分析而言,每个逻辑对象对应的物理对象,要满足一直的变化规则(状态机模型)。

日志要满足可追踪性。

日志分析很重要的一点,就是对齐时间线,梳理事件之间的因果关系,找到发生问题的最终原因。

IO场景
\begin{enumbox}
\item 无IO
\item 有IO
\end{enumbox}

故障场景
\begin{enumbox}
\item 恢复完成后再次故障
\item 恢复完成前再次故障
\end{enumbox}

新旧恢复过程过渡期间易出问题。

\begin{enumbox}
\item transmit block main thread,移交到work线程。
\item 日志缓冲区过小,会丢日志
\item 节点在prepare list之前,即收到transmit请求,从oid list取出,后又有同一oid被加入oid list,导致一个oid被修复多次
\item 修复要满足exactly once条件,唯一性
\item if-what 如果多个IO落到同一object,会如何?
\item rename与io交叉执行,完善epoch机制
\item fio 报O\_DIRECT错误,/dev/disk/by-path看不到该设备
\end{enumbox}

trace点
\begin{enumbox}
\item start\_recovery
\item finish\_recovery
\item run\_next\_rw
\item finish\_object\_list
\item get\_.*robj
\item ***
\item gateway\_write\_obj
\end{enumbox}

要理解的内容:
\begin{enumbox}
\item tgt
\item zookeeper
\end{enumbox}

分析方法
\begin{enumbox}
\item 完善LOG
\item GOTO
\item ASSERT and coredump
\item PROFILE
\item 对齐各个节点的时间
\item 识别更多不变式,加入断言
\item 返回值和可重入性
\item RAMDISK,可以用fio的verify机制
\end{enumbox}

工具:
\begin{enumbox}
\item fio
\item pdsh
\item timedatectl
\end{enumbox}

工欲善其事必先利其器。下一步重点考虑方法和工具,提升效率。

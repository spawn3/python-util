\section{04}

\subsection{01}

区分四种消息类型,特别是local和peer,区别很晦涩。

recovery有main thread get的用法,只能在主线程里使用。什么时候用到呢?

主线程做了太多的事情,包括所有完成队列的处理,会成为性能瓶颈点。

理解ssan线程模型。在main函数里启动了一个event loop,构成主线程。
另外有若干\hl{消息队列及其线程池}。

特别要注意是一个函数是在\hl{主线程或是工作线程}里被调用的。
终于明白为什么要用work和main来命名函数了。

\subsection{02}

\subsection{13}

分析理念:生命周期分析法,每个对象及其上面的操作,构成对象的生命周期。
对象包括:数据对象、一次IO过程、recovery过程等。

对一致性分析而言,每个逻辑对象对应的物理对象,要满足一直的变化规则(状态机模型)。

日志要满足可追踪性。

日志分析很重要的一点,就是对齐时间线,梳理事件之间的因果关系,找到发生问题的最终原因。

IO场景
\begin{enumbox}
\item 无IO
\item 有IO
\end{enumbox}

故障场景
\begin{enumbox}
\item 恢复完成后再次故障
\item 恢复完成前再次故障
\end{enumbox}

新旧恢复过程过渡期间易出问题。

\begin{enumbox}
\item transmit block main thread,移交到work线程。
\item 日志缓冲区过小,会丢日志
\item 节点在prepare list之前,即收到transmit请求,从oid list取出,后又有同一oid被加入oid list,导致一个oid被修复多次
\item 修复要满足exactly once条件,唯一性
\item if-what 如果多个IO落到同一object,会如何?
\item rename与io交叉执行,完善epoch机制
\item fio 报O\_DIRECT错误,/dev/disk/by-path看不到该设备
\end{enumbox}

trace点
\begin{enumbox}
\item start\_recovery
\item finish\_recovery
\item run\_next\_rw
\item finish\_object\_list
\item get\_.*robj
\item ***
\item gateway\_write\_obj
\end{enumbox}

要理解的内容:
\begin{enumbox}
\item tgt
\item zookeeper
\end{enumbox}

分析方法
\begin{enumbox}
\item 完善LOG
\item GOTO
\item ASSERT and coredump
\item PROFILE
\item 对齐各个节点的时间
\item 识别更多不变式,加入断言
\item 返回值和可重入性
\item RAMDISK,可以用fio的verify机制
\end{enumbox}

工具:
\begin{enumbox}
\item fio
\item pdsh
\item timedatectl
\end{enumbox}

工欲善其事必先利其器。下一步重点考虑方法和工具,提升效率。

\subsection{15}

以后不管每天多忙,都要坚持记日记。清晰思考、清晰表达。
牢记自己的目标,少做无用功。

当下头脑还是挺混乱,没有聚焦到最终的目标上,白白浪费时间和精力。

\subsection{16}

理解ssan的线程和队列模型。每一个放入queue的对象,都内嵌work结构体,包括request、recovery\_work等。
根据work指针可以反向检索到任务所需上下文信息。每个work队列对应一个动态伸缩的线程池。

目前定义了四类request:cluster、gateway、peer和local。如何区分peer和local request?peer是gw发送过来的消息。
local是本地节点执行的消息,\hl{为什么不直接执行},而采用了相同的队列/线程池的执行模型?

\subsection{17}

增加object version,version的变化规则:
\begin{enumbox}
\item create and write 
\item write
\item read
\item recovery (调用create and write,与分配用的区分开)
\end{enumbox}

recovery,version不应该变化,用recovery所选择的version作为权威version,同步gw上version。
gw上所有堵塞的io,都需要用更新后的version值继续。这是双控之间的同步过程。

如果recovery时不更新version,如果区分recovery fail的情况?真的需要区分吗?

\hrulefill

object version用来做什么?
\begin{enumbox}
\item 记录在peer端的日志里,跟踪对象的修订历史
\item 触发动态恢复,动态恢复与优先修复满足相同的条件
\item 取代get hash方法,鉴别对象的新旧
\end{enumbox}

\hrulefill

增加并行版的ssan\_send\_req,优化多节点通信的性能。封装map、或pmap功能。

version是否需要持久化?write op和持久化version的操作,是否需要保证原子性?如何做到?(diry/clock)

gw和magic是否需要传到peer端?

复制方式下的副本和EC下的分片都可由(oid、idx、epoch)三元组唯一确定。
副本和分片的不同在于idx,副本的idx都为0,分片的idx为一个有序序列。

\hrulefill

TODO List
\begin{enumbox}
\item 一致性: node and disk
\item 性能
\end{enumbox}

动态修复与当前的节点修复和disk修复之间如何同步? 一个对象只能被修复一次,如果触发了多次修复,后续修复须具有幂等性。

每次epoch变更,须要做到:
\begin{enumbox}
\item 从所在epoch可以检索到所有有效对象
\item 然后接收新的请求
\end{enumbox}

参考RAFT协议,可以认为epoch更新经历多个阶段:选主、修复、正常操作。
所有节点应看到相同的视图变更历史。

如何处理集群重启这个特殊场景?系统启动后,需要做哪些工作?

\hrulefill

如果gw故障,client配置了multipath。由multipath软件确定何时切换路径。

每个lun在所有gw上都export了,但只有一个处在active状态。

\hrulefill

听老王讲课蹦出来的新概念
\begin{enumbox}
\item 520格式
\item overlay OS
\item token
\end{enumbox}

\subsection{18}

性能优化,\hl{无测量,不优化},说明了定位问题的重要性。
\begin{enumbox}
\item 去掉多余的日志,或控制日志等级
\item 主线程的slow操作
\item 充分并行
\item io 同步 or async
\item 聚合
\item *** 修复相关
\item 去掉notify standby逻辑
\item 引入对象version后,rethink修复过程
\item 修复EC分片时的读改写,若无变化则不需写,若无必要,甚至不需读
\item prepare list概率地出现无效oid,可以先过滤掉,至于根本原因,待查
\item 事件循环里,事件name为何会是乱码
\end{enumbox}

直接回退到旧版本的分片,也可能存在一致性问题。因为分片与io边界不一定一致,
基于分片的回退机制有可能导致已提交数据丢失。

\hrulefill

zookeeper是如何工作的?
\begin{enumbox}
\item 如何选出master
\item watcher机制
\item queue机制,订阅-发布模式
\end{enumbox}

ssan架构方面的问题
\begin{enumbox}
\item 线程和消息队列
\item Memory
\item RPC
\item Disk IO
\item 无supervisor进程
\item journal
\item object cache
\item QoS
\item tgt与ssan进程的IPC通信
\item all user space storage stack
\item *** 工具
\item log
\item GOTO
\item trace
\item coredump
\end{enumbox}

DHT有平衡即恢复的特点。

MULTIPATH和VIP

快照是个重要并难的主题。

引入vdi控制器,至少在概念上要有这个东西。用rb tree组织多个vdi控制器。vdi控制器包括若干属性,
\begin{enumbox}
\item 所有对象以及对应的版本
\item QoS policy
\end{enumbox}

如果由gw控制恢复过程,按vdi进行组织,这样可以进一步完善恢复策略。

最近强化写作,按主题编制文档,最终整合为一,形成知识体系。

凡有所思,皆行诸文字和图表,固化下来,方便优化。溯源传统道兵诸学,奇正之术。
写作的物质载体,首推latex。

\subsection{19}

考驾照列入日程,能在京郊有套别墅当然更好。小目标如何做到?庖丁解牛的寓言里面有答案。
\begin{enumbox}
\item 驾照
\item 车牌
\end{enumbox}

从黑洞模型谈起。护城河,守正出奇。

\subsection{22}

disk故障恢复与节点故障有所不同,没有改变vnode分布,不用提升epoch,只需要由磁盘所在节点处理即可。

disk故障时,disk所在节点接收到start\_recovery。

应尽量优化修复性能,优化掉多余操作。

一致性的修复过程
\begin{enumbox}
\item peer sync
\item move过程有io
\item 两次recovery交替之时,应用新的epoch做是否已修复的条件
\item 对象名加epoch,避免move
\item 按epoch分目录存放object
\item unlikely用法错误,把任意参数转化为bool值
\item 回收:下游节点存在,可回收上游节点
\item purge dir可以采用move并异步purge的策略
\end{enumbox}

制定问题排查清单
\begin{enumbox}
\item 排查依赖性,如\hl{zk、agent、tgt}等
\item 在代码中捕获不变式,用断言处理,fast fail
\end{enumbox}

理解exec\_local\_req,深度学习ssan,力求全面掌握,在这个过程中,整理学习笔记。

理解zk的工作原理,以notify standby为例。每个节点都发出该消息,加入zk消息队列,会广播给所有节点。所有节点都能接收到该消息。
节点注册有事件处理程序。notify standby消息,只在master上处理,别的节点收到后直接返回。这与收不到消息是不同的。

\subsection{23}

zk的工作机制:提供了若干api,模拟消息队列。通过watcher回调函数监控这些节点变化。最后交给epoll事件处理程序。
seq node按序增加,用来模拟消息队列。记录队列头部位置queue pos。

clis/agentd配套使用。

ssan二进制里加入git的branch和commit。

epoch是否可抽象为一对象,多个epoch构成epoch的变更历史。可以方便地查询某一epoch下是否有某节点。

\hrulefill

磁盘故障没有提升epoch,如disk a故障,object存在disk b上,并有更新。disk a再次上线,会link到stale版本上。
可以命名为ABA问题,因为没有版本可以进行区别。

理解ssan的目录结构,多磁盘也用DHT来管理,up/down会导致大量对象的位置变化。

新plugin的盘,可能包含着stale对象。最简单粗暴的处理方式是清空其上的所有数据。
分几种场景来考虑。
\begin{enumbox}
\item 刚下线的disk
\item 刚上线的disk (\hl{上面可能有stale对象,不能作为有效对象使用,如何鉴别?})
\item 其它正常工作的disk (上面的对象也可能需要变动位置)
\end{enumbox}

磁盘故障需要扫描本地所有对象,并重新分布。因为link不支持跨文件系统,所以需要read出并写入。

磁盘故障支持IO对象优先修复。

从update epoch看起,因为磁盘修复不改变epoch,所有update epoch过程有所不同。
update epoch是把各个disk上wd里的对象移入stale目录。
如采用epoch组织目录结构,因为源目录和目标目录相同,如何处理?

修复过程\hl{优先从本地搜索合适的对象},并判断是否可用。如果可用,则link之。
判断是否可用是必须的,否则会导致对象一致性问题。

没有object version,且epoch不变,无法有效地比较对象的新旧。

修复诸阶段
\begin{enumbox}
\item update epoch
\item prepare list
\item recover objs
\item 每个object修复完成后的cleanup操作
\item 整个修复完成后的cleanup操作
\end{enumbox}

\hrulefill

写作更重要。有世界阅读日,更应该有世界写作日。用写作的方式理解ssan,
不仅仅coding,也包括文档,对主题的理解以及引申。
能用文字刻画出系统运行的关键过程,是一种非常好的锻炼。

认知天性是刻意练习的升级版?

ssan的master节点承担了什么特别的职能?zk的所有事件都在main thread里处理?
可以解耦出来,用独立线程处理zk事件吗?如此可以提升main thread的响应能力。

main thread是非常稀缺资源,也是下一步性能优化的一个重点部分。
没有采用supervisor-worker双进程架构。

\subsection{24}

purge directory逻辑导致重启进程后所有数据都被清空。
插入一块新盘,与重启系统应该不同。

如在get tgt epoch的时候。在read请求时,如传入buf小于response的data length,会如何?

epoch从1开始递增,0是无效值。

default init做了什么?

object list cache 和stale list cache,主要用于恢复过程。

\hrulefill

一个vdi可以拥有一个vdi object(支持最大4T的卷),若干data object,若干ledger object。
ledger object与data object采用同一结构组织。一个data oject对应零个或一个ledger对象。

\hl{snapshot vdi与volume vdi的vid不同},采用同一的数据结构。
一个volume vdi可以对应多个snapshot vdi,构成snap tree。

DHT修复的过程,可以看着rehash的过程,从历史epoch里选取对象,分布到新位置(\hl{选择-修复-应用})。
节点和磁盘修复都可以这么看。磁盘故障的特别之处是没有提升epoch,可能会发生ABA问题。缺少版本机制,
所以一定要作为新盘加入,purge掉其上的数据。

oid、vnode、node三者之间的关系。oid hash到vnode上,沿着vnode ring旋转,直到找到不同zone的vnode,
vnode有指向node的指针。

没有另外的平衡过程。

选择是关键操作,选择不好,会导致一致性问题。
\hl{选择阶段为什么可以往前回溯多个epoch}?

ssan有多种场景,用到了不同的选择策略:
\begin{enumbox}
\item replication
\item ec
\item disk (无epoch变化,难以区分)
\end{enumbox}

一旦高epoch修复完毕,则所有低epoch的版本即可回收。

ssan里复制卷各个节点独立进行recovery是否可行?如与IO交错,有无一致性风险?

以\hl{A/B/C/D四节点}的集群为例,创建2+1的EC卷。

\hrulefill

两级元数据架构:object到分片,分片到磁盘。这里object是逻辑对象,包括多个副本或分片。
统一用术语\hl{分片}指代复制的副本或EC的数据分片和校验分片。

EC比副本更复杂,各个分片严格有序。

\hrulefill

一致性割:分析对象的变更历史(\hl{IO粒度}),以及选择的判定标准。
一致性割可由object version来刻画。

leader和journaling是达成一致性的手段。\hl{单控、双控、多控}都可以,只要能保证最终的更新序。
单控最容易实现。journaling为了保证IO的原子写。在处理EC的时候,需要某种UNDO机制,来保证liveness。

从\hl{safety和liveness}两种属性看恢复过程。

需要什么信息才能得到某一对象在某一epoch的分布情况?这个过程必须是唯一确定的过程。
可以认为就是ceph的cluster map?

用raft membership change的算法解决static hash问题?

\subsection{25}

各个关键操作的事务性? 如md\_move\_object。

从对象的角度考虑,一致性意味着数据对象在正确的位置上,且不降级(No Error, No Missing)。

和君四大板块:国势、产业、行业、金融。

模拟一场景:193修复中,192异常下线,导致部分数据修复失败。如果192再次上线,修复能否继续?
还是说永远无法从这种状态进行下去了?

养兵千日用兵一时,养用结合。

\hl{couchbase和redis采用hash slot设计},不同于一般的consistency hash。hash slot是一种static hash机制。

ABA问题,没有版本可以用来鉴别新旧,作为新盘加入。

优先选本地,select and link。其它盘可能有对应数据吗?应该保证不会有,有了也会GC掉,不影响正确性。

只有故障盘对应slot的对象受到影响。需要独立的平衡过程。

\subsection{26}

对hash slot运用多副本,就演化为ceph的pg概念了,一个slot对应多个osd。
slot就不再是通用概念,而是与复制或EC的规则有关联了。

ssan的大环和小环有着重大区别,大环需要支持\hl{副本或EC机制},小环不需要。

向Jeff Dean学习,\hl{语言-系统-AI}跨领域通行无碍。

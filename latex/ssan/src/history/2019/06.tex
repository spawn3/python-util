\section{06}

\subsection{03}

九败一胜,\hl{人生只求一胜}。胜一次就足矣。以空间换时间,积小胜为大胜。

舍九取一,九败一胜,重点在于一。

\hl{一心二门},道心物三合之道,道置于无上地位。一心二门,以心的主体性为至高地位。
统率道和物,真如生灭二门,对唤醒自由无限心,有着积极意义。

没有休息好,一下午头昏脑涨。保证适度睡眠,极为重要。十点睡觉?

\hl{怎么一胜}?渴望胜利太久,又执行不坚决,左右摇摆,缺乏力度,难以达成目的。

\hrulefill

rethink快照实现,重新设计。

\subsection{04}

\hl{一胜九败,能一则胜},单点突破,切勿追求高大上。

败是常态,从失败中学习。胜是罕见之事,值得全情投入。

\subsection{05}

2019好好读孙子兵法,\hl{兵贵胜、不贵久}。一场大胜很重要。孙正义的二乘兵法,即是基于孙子。

钮先钟的孙子三论作为基础。

道生法这一命题仔细打磨。修道而保法,故能为胜败之政。欲求制胜之道,可从其中开始,为第一义。

道生法包括道生一在内,道生一是道生法的一个命题。由道生法展开论述,作为逻辑起点。

为何如此重要呢?法指的是一切法,心生则种种法生。

道生法置于一心二门的统领之下。二门,一道二法。
故道天地将法,将居于中央之地,顶天立地,左道右法。

一切问题都是人的问题,知兵之将,生民之司命,国家安危之主也。
将道将德,这是古典兵论的核心议题,也是现代\hl{领导力}的关键诉求。

将之要在于一心,把握使命,率众乐行。

空谈误国,实干兴邦。既要空谈,又要实干。坐而论道,起而行之。

不止于法,而是顺延下去,道法术器等等。

深陷其中,寻求突围。超越其中,出乎其中,超乎其上。
升维思考,才能降维贯通。

战略之要义在于求得行动自由。

\hrulefill

产品是战略的物质载体。战略是刀尖上的哲学。一切融入产品。哲学和战略是通过产品来落地的。
任正非那么牛,还是通过云管端的产品才得以体现。

\subsection{06}

能量都耗散了,说了多少次专注,\hl{用志不分,乃凝于神}。道理不可谓不懂,为什么不能集中心力呢?
还是像蝴蝶一样飘来飘去。

战略定力那么容易吗?

说YF把一把好牌打烂了,自己何尝不是?一定要认识到深层次的东西。
全情投入、境行果。

\hrulefill

为什么心中充满了loser的情绪,感觉事事不如意?
口号和观念都不能改变,唯有行动可以。\hl{我行故我在}。

盲目的低效的行动又不足取。如何正确地行动?

勤敬简诚能,势/道/梦想。

\subsection{10}

孙氏双乘兵法值得细细品味。不仅是孙正义要教授给软银下一代接班人的核心理论,也可用于分析amazon的大战略。
与道生法的核心命题契合无间。引申一下,可用之分析一下华为的大战略。

道天地将法是正,守正出奇。

制定长期作战计划,三十年、五十年规划。ABC已经无可置疑地成为科技发展的主流。
体系结构+AI两条腿走路是稳妥的,需要慢慢地跟上来,不掉队。

技术是一方面,因为经历的迷失的二十年,现在有点拖后腿了。管理一途,也不能自我设限。
乃至于创业是必然的选择。为之做好充分准备。要时刻问自己一个问题:假如我是一号位,会怎么做?

我思故我在,我行故我在,我胜故我在,军人天生为战胜。过多渲染悲情没多少意义。
除了胜利,一无所求;为了胜利,亦无所惜。

\hrulefill

大学毕业就陷入迷失的二十年,急急慌慌,东奔西走,却做不到择善固执的自信从容。

没有标准、没有规划,没有严格执行。兵法何严苛。对现实的残酷一面认知不足。

如何破局?孙氏双乘兵法提供了良好起点。一切为时不晚,\hl{枕戈待旦,全力以赴}。

寓言、故事、诗歌提供了丰富的战略思想的资源。

由此迷失的二十年,到求胜的意志,进而明白孙氏双乘兵法的重大价值。
务必念兹在兹,内化、优化,为我所用。

路线图目前可以不清晰,但随着更多信息的获取,会变得越来越清晰。
这就是提到的\hl{代数替换法}思维模型。

一定要跳出当前的局限,\hl{高高山顶立,深深海底行}。这种思想是顶情略七斗的一种诠释。
内在于二乘兵法的微言大义之中。

没有行动,什么都不会发生,什么都无法改变。

一即是一个状态量,也是一个过程量。说它是状态量。第一,以及追求第一的雄心壮志。
一当然有更丰富的内涵,这是道家哲学之精义。

攻守兼备极重要。进攻是最好的防守。华云的失败,在于没有防守的意识,进攻也有点盲目。
诚可谓\hl{盲人骑瞎马,夜半临深池}。

\subsection{11}

内化双乘兵法,加入自己的体会。作为一个思维框架,用来分析典型现象。

孙子四求:\hl{求知,求先,求全,求善}。归结到求胜,胜有大小。

华为四句教:以客户为中心,以奋斗者为本,长期艰苦奋斗,坚持自我批判。建立自我,追求无我。
后三句都是讲自我领导。\hl{奋斗、长期、还要在否定之否定}中快速成长。这是hegel辩证法之精义。

雷军的七字诀

\hrulefill

长期主义和专业主义是需要思考的两个维度。年愈不惑,而走向知天命之年。
男人四十一朵花,最是黄金年华,须加倍珍惜。好好规划,做成一件惊天动地的大事来。

守:\hl{保险}算是守一面的具体措施。\hl{居住证/工作居住证、北京户口}、五险一金(配合商业保险)、个税。
\hl{衣食住行,重点在住和行}。住算是初步解决了,距离完美解决距离尚远。
行没有解决,重要性不是那么高,列入计划即可。这是\hl{生活和工作方面的几个保障措施}。

工作居住证目前算是解决了,接下来重点研究一下如何解决\hl{北京户口、房子和车子}的问题。
虽然有了什么了不起,但没有却有着严重不良影响。

更重要的是在职业和事业上的突围破局。破局点在哪儿?如何破局?

书该清理就清理吧,断舍离。守破离。永远不会看到了,却占地方,影响心情。

\hrulefill

从演进的角度来展开讨论。

\subsection{12}

\subsection{13}

一心二门、老子28章,孙子第一章,结合起来好好领悟。一心二门是非常重要的思维模型,悟心、悟道、悟物。
心是居于顶点位置的,体现了心的绝对性。道虽然重要,不可违反,认知和把握的主体依然是人。
同时,离开道和物的心是空心。三者是三位一体的关系。心秉道御物。

一心二门就是画道三合之道,不同的是把心置于顶点了,与把道置于顶点的三合之道有所不同。

品思行,品即是心性,思是道,行是物。品思行三维领导力自成一体。

大乘起信论的架构非常先进,一心二门三大四信五行六度。五行就是六度的一个版本。
层层递进,直至落实,使命必达,成大智慧到彼岸。

渡河之喻,用几何学去解释,此岸彼岸形成双线,一下线一上线,一现实一理想。
如何跨越?从必然王国到自由王国,由此及彼。

孙正义的二乘兵法,可以从这个角度去理解,由了义不了义两层。

以上奠定了思维方式的基础,善学善悟,必将大成。

\hrulefill

勿多言!

\subsection{14}

强烈的问题意思,问题驱动,分析解决大问题。陈述性的描述不如问题式的描述有力。

\section{06}

\subsection{03}

九败一胜,\hl{人生只求一胜}。胜一次就足矣。以空间换时间,积小胜为大胜。

舍九取一,九败一胜,重点在于一。

\hl{一心二门},道心物三合之道,道置于无上地位。一心二门,以心的主体性为至高地位。
统率道和物,真如生灭二门,对唤醒自由无限心,有着积极意义。

没有休息好,一下午头昏脑涨。保证适度睡眠,极为重要。十点睡觉?

\hl{怎么一胜}?渴望胜利太久,又执行不坚决,左右摇摆,缺乏力度,难以达成目的。

\hrulefill

rethink快照实现,重新设计。

\subsection{04}

\hl{一胜九败,能一则胜},单点突破,切勿追求高大上。

败是常态,从失败中学习。胜是罕见之事,值得全情投入。

\subsection{05}

2019好好读孙子兵法,\hl{兵贵胜、不贵久}。一场大胜很重要。孙正义的二乘兵法,即是基于孙子。

钮先钟的孙子三论作为基础。

道生法这一命题仔细打磨。修道而保法,故能为胜败之政。欲求制胜之道,可从其中开始,为第一义。

道生法包括道生一在内,道生一是道生法的一个命题。由道生法展开论述,作为逻辑起点。

为何如此重要呢?法指的是一切法,心生则种种法生。

道生法置于一心二门的统领之下。二门,一道二法。
故道天地将法,将居于中央之地,顶天立地,左道右法。

一切问题都是人的问题,知兵之将,生民之司命,国家安危之主也。
将道将德,这是古典兵论的核心议题,也是现代\hl{领导力}的关键诉求。

将之要在于一心,把握使命,率众乐行。

空谈误国,实干兴邦。既要空谈,又要实干。坐而论道,起而行之。

不止于法,而是顺延下去,道法术器等等。

深陷其中,寻求突围。超越其中,出乎其中,超乎其上。
升维思考,才能降维贯通。

战略之要义在于求得行动自由。

\hrulefill

产品是战略的物质载体。战略是刀尖上的哲学。一切融入产品。哲学和战略是通过产品来落地的。
任正非那么牛,还是通过云管端的产品才得以体现。

\subsection{06}

能量都耗散了,说了多少次专注,\hl{用志不分,乃凝于神}。道理不可谓不懂,为什么不能集中心力呢?
还是像蝴蝶一样飘来飘去。

战略定力那么容易吗?

说YF把一把好牌打烂了,自己何尝不是?一定要认识到深层次的东西。
全情投入、境行果。

\hrulefill

为什么心中充满了loser的情绪,感觉事事不如意?
口号和观念都不能改变,唯有行动可以。\hl{我行故我在}。

盲目的低效的行动又不足取。如何正确地行动?

勤敬简诚能,势/道/梦想。

\subsection{10}

孙氏双乘兵法值得细细品味。不仅是孙正义要教授给软银下一代接班人的核心理论,也可用于分析amazon的大战略。
与道生法的核心命题契合无间。引申一下,可用之分析一下华为的大战略。

道天地将法是正,守正出奇。

制定长期作战计划,三十年、五十年规划。ABC已经无可置疑地成为科技发展的主流。
体系结构+AI两条腿走路是稳妥的,需要慢慢地跟上来,不掉队。

技术是一方面,因为经历的迷失的二十年,现在有点拖后腿了。管理一途,也不能自我设限。
乃至于创业是必然的选择。为之做好充分准备。要时刻问自己一个问题:假如我是一号位,会怎么做?

我思故我在,我行故我在,我胜故我在,军人天生为战胜。过多渲染悲情没多少意义。
除了胜利,一无所求;为了胜利,亦无所惜。

\hrulefill

大学毕业就陷入迷失的二十年,急急慌慌,东奔西走,却做不到择善固执的自信从容。

没有标准、没有规划,没有严格执行。兵法何严苛。对现实的残酷一面认知不足。

如何破局?孙氏双乘兵法提供了良好起点。一切为时不晚,\hl{枕戈待旦,全力以赴}。

寓言、故事、诗歌提供了丰富的战略思想的资源。

由此迷失的二十年,到求胜的意志,进而明白孙氏双乘兵法的重大价值。
务必念兹在兹,内化、优化,为我所用。

路线图目前可以不清晰,但随着更多信息的获取,会变得越来越清晰。
这就是提到的\hl{代数替换法}思维模型。

一定要跳出当前的局限,\hl{高高山顶立,深深海底行}。这种思想是顶情略七斗的一种诠释。
内在于二乘兵法的微言大义之中。

没有行动,什么都不会发生,什么都无法改变。

一即是一个状态量,也是一个过程量。说它是状态量。第一,以及追求第一的雄心壮志。
一当然有更丰富的内涵,这是道家哲学之精义。

攻守兼备极重要。进攻是最好的防守。华云的失败,在于没有防守的意识,进攻也有点盲目。
诚可谓\hl{盲人骑瞎马,夜半临深池}。

\subsection{11}

内化双乘兵法,加入自己的体会。作为一个思维框架,用来分析典型现象。

孙子四求:\hl{求知,求先,求全,求善}。归结到求胜,胜有大小。

华为四句教:以客户为中心,以奋斗者为本,长期艰苦奋斗,坚持自我批判。建立自我,追求无我。
后三句都是讲自我领导。\hl{奋斗、长期、还要在否定之否定}中快速成长。这是hegel辩证法之精义。

雷军的七字诀

\hrulefill

长期主义和专业主义是需要思考的两个维度。年愈不惑,而走向知天命之年。
男人四十一朵花,最是黄金年华,须加倍珍惜。好好规划,做成一件惊天动地的大事来。

守:\hl{保险}算是守一面的具体措施。\hl{居住证/工作居住证、北京户口}、五险一金(配合商业保险)、个税。
\hl{衣食住行,重点在住和行}。住算是初步解决了,距离完美解决距离尚远。
行没有解决,重要性不是那么高,列入计划即可。这是\hl{生活和工作方面的几个保障措施}。

工作居住证目前算是解决了,接下来重点研究一下如何解决\hl{北京户口、房子和车子}的问题。
虽然有了什么了不起,但没有却有着严重不良影响。

更重要的是在职业和事业上的突围破局。破局点在哪儿?如何破局?

书该清理就清理吧,断舍离。守破离。永远不会看到了,却占地方,影响心情。

\hrulefill

从演进的角度来展开讨论。

\subsection{12}

\subsection{13}

一心二门、老子28章,孙子第一章,结合起来好好领悟。一心二门是非常重要的思维模型,悟心、悟道、悟物。
心是居于顶点位置的,体现了心的绝对性。道虽然重要,不可违反,认知和把握的主体依然是人。
同时,离开道和物的心是空心。三者是三位一体的关系。心秉道御物。

一心二门就是画道三合之道,不同的是把心置于顶点了,与把道置于顶点的三合之道有所不同。

品思行,品即是心性,思是道,行是物。品思行三维领导力自成一体。

大乘起信论的架构非常先进,一心二门三大四信五行六度。五行就是六度的一个版本。
层层递进,直至落实,使命必达,成大智慧到彼岸。

渡河之喻,用几何学去解释,此岸彼岸形成双线,一下线一上线,一现实一理想。
如何跨越?从必然王国到自由王国,由此及彼。

孙正义的二乘兵法,可以从这个角度去理解,由了义不了义两层。

以上奠定了思维方式的基础,善学善悟,必将大成。

\hrulefill

勿多言!

\subsection{14}

强烈的问题意思,问题驱动,分析解决大问题。陈述性的描述不如问题式的描述有力。

\subsection{15}

先follow,后成为leader。认真、彻底地学习华为的管理和研发逻辑、工具体系,
争取能为我所用,为下一步发展打下坚实基础。总是要走到前台,去博取更多自由、去赢得未来。

大思想要有大作品相伴,更彰显其价值。

首先要有选择自己战场的智慧,四面出击,得不偿失。
不如一门深入,\hl{制心一处,无事不办}。

卧薪尝胆。

通过解决问题得到成长和发展。解决大问题、重要问题。

\hl{通过追问去学习是最好的学习方法},即深且广。
追问的过程,也是深入参与的过程,不受已有知识体系的束缚。

行动学习的两要素:\hl{L = Q + P}。就像冬瓜哥在大话计算机后记里所说的,原生态的追问是第一动力。
随着问题的展开,逐步形成结构化的知识体系和操作指南。

今后学习任何知识,都要遵照该方法。

\hrulefill

\hl{将领导力发展作为关键战略问题}。通过提问去领导、管理。

重视管理,技术也需要管理。

利用全闪产品小组这次机会,既要完成技术方面的目标,也要达成领导力方面的目标。
\hl{两项任务:技术 and 管理}。管理也是华为成功的要素之一。

设计几个问题:
\begin{enumbox}
\item 按目前的进度,能否达成既定目标?
\item 有没有更好的做法?
\end{enumbox}

\subsection{17}

取高远之势,才能突破当下之困局。不要被大小得失所困扰。

眼下最大之困局,一是中和教育;二是职业发展。后者是抓手,第一则是规则。
如纠结于规则,就会受围困之苦。当取高远之势,徐徐图之。

高远之势为何,又当如何徐徐图之,这是眼下之大问题。

\hl{势、道、梦想}。梦想即是初心、菩提心。

\hrulefill

\hl{取势、明道、优术},势高则围广。

何必舍近而求远?把手头的事情做到极致就是取势了。怎么做到极致?明道优术。

学会双线法则,至少在两个层次去思考问题。比如战略、战术,上限、下限等。

限定能力圈,不是自我设限,而是精益求精。

不需要更多的说明了,\hl{一门深入,长时薰修}。不开始永远都不能成功。

按721法则,2和1也很重要。

向富兰克林学习,制定日课若干条:
\begin{easylist}
& 从容
& 早起
& 聚焦
\end{easylist}

\hl{管理第一,技术第二}。没有好的管理,技术难以变现。
两条腿走路,目前强在技术,弱在管理。技近乎道,\hl{技术和管理统一于道}。

使命是什么?从影响圈、能力圈来看。\hl{全闪产品小组的使命是什么?}

\subsection{18}

勤、敬、诚、简、能,修身五字诀。勤以治惰,敬以治傲。\hl{乾以易知,坤以简能}。

势、道、梦想,着重点在事功,建功立业。

个人势能,如学历、专利、大公司背景等等。这方面考虑太少,打了一手烂牌。
如何反败为胜?

定位清晰,当择善固执,持之以恒。

\hrulefill

向\hl{华为/任正非}学管理,中西合璧,是当前集大成者。

大秦帝国,大争之世,多事之秋,凡有血气,必有争心。
我欲清溪寻鬼谷,不问礼乐但论兵。\hl{兵家、法家、墨家}皆极务实。
德治、法治成双线格局。法治是底线,德治是理想。

多读多悟任正非讲话,把握第一手资料。二手三手资料作为参考。
任的思想博大精深,须整合很多知识才能真的理解。

\hrulefill

带着问题去学习,活学活用。读书贵有疑情。疑情可以多种,以能解决现实问题为归宿。

宋代学者朱熹说:“读书无疑者须教有疑,有疑者却要无疑,到这里方是长进。”
英国哲学家培根也说过类似的话:“如果一个人从肯定开始,必将以疑问告终;如果他准备从疑问着手,则会以肯定结束。”
同学们应该牢记:读书贵有疑。

如冯唐的成事一书,就是有感于如何成事此一疑情。禅宗的参话头、公案都是始于疑情,
阳明龙场悟道,也是始于疑情:圣人处此,更有何道?

理论是灰色的,生命之树长青。不是说理论不重要,而是说理论是灰度的,守正出奇,方可制胜。知白守黑,黄金分割。

\hrulefill

立足最坏,力争最好,充分理解和运用双线法则。
任正非:活着是最低战略,也是最高战略。这是悲观主义吗?非也!
为什么优秀架构师都是悲观主义者,不能叫做悲观主义,而是理性乐观派,革命浪漫主义。拥抱不确定性的勇气。
明知山有虎偏向虎山行。明知山上有大老虎,还什么都不准备,岂不是儿戏人生?

分布式系统的safety和liveness属性。

冒险进取,警备非常。

\hl{就这么决定了,认真学习任正非}。

\hrulefill

提炼做事情的万能方法。

\subsection{19}

抓手,针尖战略,一个针尖捅破天。

\hl{吸纳了科学的玄学将大放异彩}。玄,黑色,吸收能力极强。

双线法则、圆点哲学、三合之道为主要的思维模型。圆点哲学的圆点模型,上文所提到的抓手就是圆点的点,立足于点,运心于圆。
定位后,施以饱和攻击。集中优势兵力,各个歼灭敌人。

守正出奇,持经达变,随缘不变不变随缘,都是圆点的具体化。一切思想都可以用数学来表达。
一门深入,长时薰修。制心一处,无事不变。绝利一源,用师十倍。都揭示了一与多的辩证统一关系。

一花一世界,此花就是一直在寻找的。

先定位一,好好涵养栽培,使之茁壮成长。

\hrulefill

太极哲学,知白守黑。\hl{说一千道一万,归结于太极哲学}。由此而武术、中医,内核在太极。
且看太极图说。从太极哲学里,读出双线、圆点、三合等理念。

矛盾论适用于大争之世。和谐社会依然需要大争,长期艰苦奋斗。

无极而太极,无极类似于灰度。

执一明三定二。以往没能从太极里读出\hl{零、一、二、三}。

总结一下:\hl{太极哲学、双乘兵法、任正非的华为管理和技术}。
聚焦到如何成事,成大事?

顶情略七斗,略,三略六韬。七或者六,黄金分割。斗,大争之世,多事之秋,凡有血气,必有争心。

一流攻守群,群,一体之多元。攻守一体,流不仅是取势,也是流动性。
可以从战术的角度理解此五字。一,集中,流,出奇制胜,灵活性,流动性。攻守兼备,九天九地。
群,一体之多元化,备份,横向扩展,集群。

海有点类似灰度的开放包容精神,格局大。

\hrulefill

二乘法则统一了黄华的事功和修养,内圣外王。\hl{修养要对准事功的需要}。通过修养重塑世界。

黄华的事功和修养论,深化了二乘兵法。如修养论,补充了勤敬诚简能五字诀。事功论,补充了势道梦想。
合内外之道,故时措之宜也。

梦想是牵引力,提供了修养和事功的动力源。一心二门,就是此初心、良知。由此开辟了新天地。

神秀的\hl{身如菩提树,心如明镜台,时时勤拂拭,勿使惹尘埃}。境界也极为高明。
没有这一层功夫,如何能进入更上一层空性境界。

如学习富兰克林的修身十三条,不如以二乘兵法为基础,斟酌损益,时时勤拂拭,勿使惹尘埃。
曾国藩的日课,都是时时勤拂拭的功夫论。

诠释华为四句教:以客户为中心,以奋斗者为本,长期艰苦奋斗,坚持自我批判。
\hl{坚持自我批判}这个很少提及。

\hrulefill

修养论和事功论是一体两翼,一心二门的架构。

\subsection{20}

昨天突然想到\hl{读在职硕士的事情},一举多得。必须为下一步做好计划了。
北京户口是一部分原因,还有长期发展的考虑。

步步踏空,现在想来整个成长过程都缺失规划和指引。教育没把握好,工作也没把握好,
甚至专业技能方面的学习也走了很多弯路,没有达到预定目标。
一直也重视学习方法,但主要的是没有沉下心来,好好切磋琢磨。

值此内外交困之际,最好的投资是投资自己。理念非常好。核心问题是:下一步怎么走?

学好哲学和数学,这项投资有着长期价值。另外就是英语了。
拿出徐特立的精神和方法来,学好\hl{哲学、数学和英文}。

哲学,尤其要中西结合,中学为体,西学为用。核心即太极哲学,千古大道沉沦,只缘误解太极。
充分理解和运用太极哲学。\hl{圆点、双线、三合}都是太极哲学的变体,为太极哲学注入了新鲜血液。

毕业近二十年,看了许多书,不能白看。要从中复盘,提要钩玄,为我所用。
要看到更多积极因素。转危为安,转败为胜。

\hl{周末好好规划下,找人讨论一下}。

\hrulefill

太极哲学不是万灵丹,而是一套概念和思维模型,可以用来分析各种问题。

看看毛主席的两论,实践论和矛盾论,进而是一般的唯物辩证法。太极哲学有着更丰满的内容。

从\hl{长期主义和专业主义}的观点看太极哲学。年纪也可以成为优势,没有了年轻人容易犯的好高骛远,
而是干好手头上的事情,结果纷至沓来。长期主义和专业主义加起来,就是\hl{恒以一德}。

一德在前,加以恒的功夫,所以一德有两种存在形态:观念中的和现实中的。恒是沟通两者的桥梁。

\hrulefill

由二乘兵法,扩展到对孙子兵法的理解。

大秦帝国中\hl{商鞅的势论}发人深省,进而可见其整体治国方略,即道法部分。

大商之道的十二字诀。道法,变常,取予,利害,生死,方圆。
生死、利害是价值判断,变常是任势而为

法术势,法家三派,\hl{法是确定性的力量},术和势是不确定性的力量。

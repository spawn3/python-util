\section{06}

\subsection{03}

九败一胜,\hl{人生只求一胜}。胜一次就足矣。以空间换时间,积小胜为大胜。

舍九取一,九败一胜,重点在于一。

\hl{一心二门},道心物三合之道,道置于无上地位。一心二门,以心的主体性为至高地位。
统率道和物,真如生灭二门,对唤醒自由无限心,有着积极意义。

没有休息好,一下午头昏脑涨。保证适度睡眠,极为重要。十点睡觉?

\hl{怎么一胜}?渴望胜利太久,又执行不坚决,左右摇摆,缺乏力度,难以达成目的。

\hrulefill

rethink快照实现,重新设计。

\subsection{04}

\hl{一胜九败,能一则胜},单点突破,切勿追求高大上。

败是常态,从失败中学习。胜是罕见之事,值得全情投入。

\subsection{05}

2019好好读孙子兵法,\hl{兵贵胜、不贵久}。一场大胜很重要。孙正义的二乘兵法,即是基于孙子。

钮先钟的孙子三论作为基础。

道生法这一命题仔细打磨。修道而保法,故能为胜败之政。欲求制胜之道,可从其中开始,为第一义。

道生法包括道生一在内,道生一是道生法的一个命题。由道生法展开论述,作为逻辑起点。

为何如此重要呢?法指的是一切法,心生则种种法生。

道生法置于一心二门的统领之下。二门,一道二法。
故道天地将法,将居于中央之地,顶天立地,左道右法。

一切问题都是人的问题,知兵之将,生民之司命,国家安危之主也。
将道将德,这是古典兵论的核心议题,也是现代\hl{领导力}的关键诉求。

将之要在于一心,把握使命,率众乐行。

空谈误国,实干兴邦。既要空谈,又要实干。坐而论道,起而行之。

不止于法,而是顺延下去,道法术器等等。

深陷其中,寻求突围。超越其中,出乎其中,超乎其上。
升维思考,才能降维贯通。

战略之要义在于求得行动自由。

\hrulefill

产品是战略的物质载体。战略是刀尖上的哲学。一切融入产品。哲学和战略是通过产品来落地的。
任正非那么牛,还是通过云管端的产品才得以体现。

\section{11}

\subsection{01}

编程注意事项
\begin{enumbox}
\item long time操作之前与之后,需要renew lease,防止lease失效,导致vctl切换
\item 读写、vfm\_get、vfm\_stat、stat等操作之前,必须chunk check。\hl{如果不检查,可能会返回ESTALE, see \_\_chunk\_read\_getnode}
\item chunk组织成tree,就意味着依赖性,需要从上到下依次保护,check
\item chunk上有lock保护
\item alloc/discard与io分离,故io加rdlock
\item \hl{困难之处在并发,tree结构上的并发,并且涉及到持久化、lease}
\end{enumbox}

诊断工具
\begin{enumbox}
\item 怎么快速得到各个controller的状态,包括pool、volume、snapshot等?
\item 怎么检查底层数据一致性?
\item 打印集群chunk tree,以及每个chunk的详细信息
\end{enumbox}

写到LUN结尾处,会自动扩容。ROW3这里有问题?

垃圾数据,干扰恢复,gc replica目前不能打开?

\subsection{02}

尽可能通过\hl{DBUG、GOTO}保留可跟踪线索,以方便线上调试。

mellanox交换机编程:SDK?

\subsection{08}

交换机的任务告一段落,遗留的部分问题:授权,不知如何获取部分信息。

\hl{交换机类似于一个host},如果配置ip地址后,就可以通过SNMP等用户接口进行访问。
不同的是它具有的交换、端口管理能力。

\subsection{09}

session,网站的session,在服务器端维护,记录某用户的相关信息,比如login状态。
客户端浏览器的cookie里如果记录了session id,在server端session没有失效的情况下,不需要再次登录。

iscsi的session什么约定?一个session包含一个或多个连接。

target有多义:一指target server,二指init连接到的某个具体的LUN(portal)

多网段,vip,multipath。

支持\hl{存储多网段},就是有多个存储网段,通过bond机制管理的,只能看做是一个网段。

\subsection{14}

平衡是个关键指标,平衡了,才能充分利用集群的规模效应。平衡分数据、负载的平衡,这是多个层次的事情。

再次理解COW快照,不再做太多改进,进入维护模式。重点是ROW3,但\hl{ROW3依赖于大卷,同时需要空间回收}。
稳定下来,尚需时日。

精简配置、cache、RC、\hl{EC、压缩、重删}等功能也需要先行设计。目前看来,cache、RC优先级要高些。

QoS还需要更精细,但非当前急务,当前急务是一个稳定可靠高性能的版本。覆盖最小功能集。

高性能相关特性,特别是全闪相关的特性,如RDMA、SPDK等也需要尽快熟悉。RDMA不同于TCP。

target方面,iser、NVMe也有理解之必要。如何结合多路径,就会复杂不少。整体上依然是外围的工作。
\hl{存储引擎处理controller以及以下层的事务},包括底层资源管理,可靠性、平衡、性能、QoS特性等等。

bcache是在物理磁盘的层面上处理io的。按\hl{二实一虚的架构},导出虚拟设备供上层应用使用。
进而理解\hl{linux的block io架构}。

SPDK不经过kernel的block层,而是直接用pci号与NVMe控制器直接通信。一个NVMe设备,就是一个独立的pc。
不同的是,没有独立的电源系统。一经插入,就可以直接通信,所以能有极高的iops,极低的latency。

RDMA是transfer,NVMf是target,NVMe是driver,从存储和网络\hl{硬件和协议}的角度,全面提升存储性能。
RDMA是tcp的替代品,而不改变rpc API。

如存储多网段、多路径是从架构的角度提高存储系统的可靠性和性能。

针对每一个特性,就不能泛泛地理解了,需要反复深入。

\subsection{19}

启动时,disk load与etcd的顺序?通过etcd获取pool list,用于判定pool的有效性。
在lich内部,加盘的时候,也需要检查pool是否有效。

如果删除disk symlink时,没有同时删除关联的pci文件,会如何?

删除pool需要满足的规则:
\begin{enumbox}
\item disk何时可重用?
\item 在删除的过程中,不能创建同名pool。
\end{enumbox}

磁盘管理需要rethink。

\subsection{21}

删除pool,提炼模式:由集群内节点并发处理某任务,当所有节点都完成时,任务完成。
同步、异步:任务异步执行。

完成一个操作,要进行模式提炼,方便reuse到其它地方。

etcd里与pool有关的状态:
\begin{enumbox}
\item /lich4/storage/
\item /lich4/poolstatus
\item /lich4/task/poolremove/
\end{enumbox}

删除pool也影响到数据恢复、平衡任务。

lich网络配置,检查netmask,\hl{一个网段不能出现多个网络接口}?

两个存储交换机,每个节点双链路做bond,如果自己实现支持多存储网段,改动会比较大。

\subsection{26}

vfm cleanup必须在成功恢复后才能执行。check vfm
如果scan阶段没有标上vfm,也是有问题的。

\subsection{30}

mellanox SM HA行为不正常,SW1012经过一次failover过程后,sm功能不可用。

下周学习RDMA,解决快照相关问题。

\chapter{思想方法}

毛主席、陈云等前辈都极为重视思想方法和工作方法,方法论是比具体知识更高一个维度的知识,务必引起充分的重视。

\section{思想方法}

战略是一种思想方法,科学是一种思想程序。

两种力量:思想和利剑,利剑不敌思想。

\hl{知其白、守其黑,为天下式,为天下式,恒德不忒,复归于无极}。
从此角度理解任正非,一个命题:战略是一种思想方法。两条路:一哲学,二数学。

生活和工作需要战略思维,战略思维是可以学而知之、学而时习之的。

\hl{张文木、金一南的著作}把思想拉到现实领域。
战略是一种特殊的思想方法,战略是刀尖上的哲学,因为事关生死存亡盛衰成败。
战略是唯物论之上的辩证法艺术,辩证法只有建立在唯物论的基础之上,
才不是玄学,扎根于现实的土壤里,才能发挥现实力量。
所以毛主席的思想和著作,要从两论学起:实践论和矛盾论。

陈云学到家了,提出了十五字决:不唯上、不唯书、只唯实,交换、比较、反复。
前者是唯物论,后者是辩证法。

\hl{老子和孙子相得益彰},孙老二子一动一静相得益彰。老子守静、孙子致动。
兵道二字切题。兵法不如兵道二字。或取二者之和:兵道、兵法。
修道而保法,故能为胜败之政。

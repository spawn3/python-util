\chapter{思想方法}

毛主席、陈云等前辈都极为重视思想方法和工作方法,方法论是比具体知识更高一个维度的知识,务必引起充分的重视。

学习重要,学会学习更重要。授人以鱼不如授人以渔,\hl{中和的教育,着重点在此}。

\section{思想方法}

战略是一种思想方法,科学是一种思想程序。

两种力量:思想和利剑,利剑不敌思想。

\hl{知其白、守其黑,为天下式,为天下式,恒德不忒,复归于无极}。
从此角度理解任正非,一个命题:战略是一种思想方法。两条路:一哲学,二数学。

生活和工作需要战略思维,战略思维是可以学而知之、学而时习之的。

\hl{张文木、金一南的著作}把思想拉到现实领域。
战略是一种特殊的思想方法,战略是刀尖上的哲学,因为事关生死存亡盛衰成败。
战略是唯物论之上的辩证法艺术,辩证法只有建立在唯物论的基础之上,
才不是玄学,扎根于现实的土壤里,才能发挥现实力量。
所以毛主席的思想和著作,要从两论学起:实践论和矛盾论。

陈云学到家了,提出了十五字决:不唯上、不唯书、只唯实,交换、比较、反复。
前者是唯物论,后者是辩证法。

\hl{老子和孙子相得益彰},孙老二子一动一静相得益彰。老子守静、孙子致动。
兵道二字切题。兵法不如兵道二字。或取二者之和:兵道、兵法。
修道而保法,故能为胜败之政。

\section{大乘起信论}

\section{孙正义的二乘兵法}

运用二乘兵法的一个根本前提是有一个长期作战计划。年逾不惑,紧迫感更强烈。
以十年为跨度,制定下一个十年发展纲要,是必要的。

既往不咎,纵情向前。\hl{2016-2020年},作为个人的第一个五年规划的试运行期。
接下来就是二五规划,三五规划等等。

\hl{通过解决重大问题进入顶级专家之列。}

五年规划
\begin{enumbox}
\item 2016 - 2020
\item 2021 - 2025
\item 2026 - 2030
\item 2031 - 2035
\item 2036 - 2040
\end{enumbox}

\hrulefill

回首过去的五年,从2016年开始,就陷入了一系列焦虑的事件之中,所幸没有造成大的麻烦。

至2020年年底,还有一年半的时间,可以做很多事情。

北京户口影响最大,转眼中和即将进入中学,\hl{需要详细了解一下天津方面的相关政策},确定哪一年转过去是合适的。

个人职业方面,一直没有大突破,实则是缺失战略定力,不能择善固执,而是左右摇摆,没有重点所致。
用贝佐斯的说法,没有坚守在不变的事情上。以不变应万变的道理,何尝不明白?
只是很难把理念运用到实际生活中来,没有发挥道理应有的力量。

历练驱动的领导力开发一书,有一些重要的论点,主要是把领导力发展作为关键战略问题来看待。
同时提出了\hl{发展领导力的721法则},即历练、人际关系和培训所起作用的比例关系。
历练所起作用最大。这有点类似联想方法论中的复盘。

从自己、从他人的成功或失败的历练中学习,活学活用,内化为自己的素质和能力。

就像富兰克林的修身十三条,或曾国藩的日课一样,每次的关注点应当是一个,而不是泛泛而论。
每次一个关注点,\hl{集中兵力,各个击破},周而复始,螺旋上升。

集中兵力的原则,不仅是军事上的第一原理,放在工作和生活中的方方面面,都有着积极而重要的价值。

需要注意的是,集中兵力解决了关键问题,善后工作同样重要,孙正义用一个海字来指代。
巩固并扩大战果,压榨历练到极致。海不仅波澜壮阔,也内蕴伟大的原力。
\hl{记录、反思、觉悟},如此反复。高速运转PDCA循环,C包含了检查和记录等职能。

去解决大问题,只要把手头工作做到极致,好的期望中的结果自然而然就来了。
思维力可以纵横分合,工作却需要聚焦。所以思维方法和工作方法乃相辅相成。

远在天边近在眼前。

活着是最低纲领,也是最高纲领。

\hrulefill

磨好豆腐

一定不能被牵着鼻子走,这样很难突破,如何出奇制胜?守正出奇。

一旦有大的突破,现在所看到的困难都不在是困难,就跃迁到了更高的能量级。

突破的抓手就隐藏在一这个字里面了。一流攻守群。

全力以赴。攻呀!最好的防守是进攻。攻!攻!攻!
一切障碍都没有了,患得患失、瞻前顾后真是害人。

林彪的六大战法,\hl{一点两面,攻!攻!攻!}

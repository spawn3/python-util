\chapter{提问}

疑问词:5W1H,why,why not?

\section{网络编程}

Connection,Poller,EventLoop是核心概念。一个Connection最多加入一个EventLoop,EventLoop对应一个Poller,
分离时间到不同的Connection,由该Connection的事件处理程序处理。一个Connection应该具有哪些属性?
连接的建立,断开,错误当如何处理?

两个节点之间的RPC需要建立连接,有且只有一条连接。
\begin{compactenum}
\item 若两节点同时建立连接,每个节点上的netable会增加几项?
\item nid小的发起连接?
\end{compactenum}

corerpc时,情况较为复杂。每个节点上有N个core参与,相同编号的core构成corenet。
一个corenet类似于普通rpc连接管理机制。

\begin{compactenum}
\item 什么时候用普通rpc,什么时候用corerpc?
\item 选中几个core参与corerpc,为什么?
\item 每个节点的polling core数量必须相等吗?
\item polling core的数量可以动态调整吗?
\end{compactenum}

故障,TP,snapshot的性能损耗\%分别是多少?

如etcd与存储网一样,有较多的timeout,etcd应该分网而治吗?(管理网,存储网)

降级写+整体掉电会影响数据的顺序一致性,客户端可见IO Error?

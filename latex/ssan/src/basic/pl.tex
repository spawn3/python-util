\chapter{编程语言}

不应该放在太琐碎的知识点上,要放在\hl{核心概念的融会贯通}上。

多提问,\hl{是什么?为什么?别人是如何做的?},思方学。
视野要开阔,避免坐井观天、夜郎自大。有生态思维。

体系结构、操作系统和汇编语言是底层逻辑。

编程语言通常提供了比较完备的数据结构和算法库,可以结合起来进行学习。

\section{C}

编译和链接

\section{C++}

代码组织方式:file, namespace, exception

\hl{function, class, template}是三个核心概念。class是data type的扩展,template作用于function和class之上。

value、pointer、reference。reference又分为左值和右值。右值引用用来实现move语义,避免不必要的数据copy。
这与mbuffer等zero-copy技术类似。

一个大原则:pointer和reference必须指向有效数据,\hl{局部声明的临时对象,被销毁的对象}都不能再使用指向这些对象的指针或引用。
结合内存模型是很容易理解的。

什么是左值,什么是右值呢?可以取地址的是左值,常数、函数返回的临时变量是右值。

\section{Erlang}

\section{Go}

Go的调度机制

\section{Python}

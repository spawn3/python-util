\chapter{算法分析}

算法分析与排队论要解决的问题不同,算法分析解决单个CPU的性能分析模型,排队论则是排队网络的性能分析,
包括吞吐量、响应时间、并发度等。

排队论的核心是平衡方程及其均值解法,建立在一些简单的假设之上。

\section{算法设计思想}

程序=数据结构+算法,以数据结构和算法统领各领域,比如操作系统、数据库等。

架构是更高的维度。在纷繁复杂的现象背后,寻找其基础元素、结构和灵魂。

首先,定义问题。比如操作系统的进程调度、数据库并发控制的可串行化理论,
都是典型的算法问题。

现实-模型-理论,模型需要实验和理论验证和证明。

列出各领域的关键问题,用算法思维去解决。

暴力、穷举解法,是思考的原点。在此基础上,引入优化策略,提升算法效率。

贪心、分治、动态规划、回溯、分支限界、线性规划等。主要是分合为变。
解空间和解向量,约束条件(隐显)。

分治与动态规划的不同在于子问题是否重叠。分治的子问题相互独立,top down,分而治之。
动态规划bottom up,从子问题逐步合成最终解。

回溯采用DFS,分支限界采用BFS的队列机制,当然队列可以是FIFO,也可以是优先级队列。

\section{排序算法}

\subsection{Insertion Sort}

\subsection{Shell Sort}

\subsection{Select Sort}

\subsection{Heap Sort}

\subsection{Bubble Sort}

\subsection{Quick Sort}

\subsection{Merge Sort}

归并,多路归并,外排序

\subsection{Counting Sort}

\subsection{Bucket Sort}

\subsection{Radix Sort}

\section{查找算法}

\subsection{二分查找}

\subsection{TOP N}

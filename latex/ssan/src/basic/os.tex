\chapter{操作系统}

资源管理,架起硬件和上层应用之间的桥梁。资源的3+1模型:cpu、memory、disk and network。

\section{设计原则}

\begin{enumbox}
\item 分离机制与策略
\item 主奴机制
\item 模块化
\end{enumbox}

\section{CPU}

从单核cpu说起,多核cpu引入了哪些关键问题?

进程调度器本身是什么?

内核可以调度用户态代码吗?怎么做?
fuse由内核调用用户态代码。

nfs不同于posix语义,nfs client是内核态。
曙光文件系统client也是内核态。

进程、线程、协程调度

通信和同步

经典问题:
\begin{enumbox}
\item 生产者-消费者问题
\item 读者-写者问题
\item 哲学家就餐问题
\end{enumbox}

\section{Memory}

多层分配器,解决内外碎片现象。

\section{I/O Device}

\begin{enumbox}
\item /dev (udevd)
\item /dev/mapper
\item /sys (ioctl)
\end{enumbox}

VFS统一管理设备文件,ioctl和sysfs用于特殊的属性控制。

设备导出为/dev下的设备文件,通过设备文件调用device driver。
比如块设备,可以看到分区。lsblk

udev可用于管理磁盘事件。

device mapper是磁盘映射,基于dm实现soft RAID,LVM,VDO等特性。
\hl{RAID导出的是盘,LVM导出的是逻辑卷},LVM over RAID,即LVM可以管理RAID导出的盘。

通常分为内核部分与用户空间部分,实现机制与策略的分离。

\subsection{VFS}

VFS Over Block Driver,提供了page cache。

不同的文件系统实现,共用块设备驱动程序。

direct IO跳过page cache层。sync IO不跳过page cache,但强制flush。

dirty data是指buffer中修改的数据,还没有持久化,有独立的异步flush线程。

\subsection{块设备}

buffer cache,speedup,alignment

\subsection{ioctl}

netlink什么用?怎么用?

\subsection{网络设备}

不在VFS之下

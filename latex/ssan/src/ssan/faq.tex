\chapter{运维视图}

\section{问题集}

三理:生理、病理、治理。治理建立在生理和病理的基础之上,治理要追根究底,三因分析。

通用问题集
\begin{enumbox}
\item 允许的故障节点数?
\end{enumbox}

lich特定问题
\begin{enumbox}
\item 理解table2->chunk\_check做了什么?
\item 拔盘后的行为?
\item lichd --init做了什么
\item 每个节点的nid保存在哪儿,如何分配的?
\item /opt/fusionstack/data目录布局
\item /dev/shm/lich4目录布局
\item cleanup
\item clock机制
\item hsm
\end{enumbox}

\begin{compactitem}
\item coroutine and scheduler
\item polling \change{polling}
\item kernel bypass
\item mbuffer
\end{compactitem}

问题集:
\begin{enumbox}
\item /的位置信息
\item 当前分配的最大卷ID?
\end{enumbox}

P1: IOMeter测试,256K,Lich顺序和随机IO性能差别大

P2: lsv\_gc\_check断言失败

P3: Error Handling

\section{约束}

\subsection{最大副本数}

6

\subsection{单卷最大快照数}


\section{Performance}

\subsection{估算}

HDD:
\begin{itembox}
\item 顺序1M 200
\item 随机1M 50
\item 顺序4K IOPS
\item 随机4K IOPS
\end{itembox}

集群部署
\begin{enumbox}
\item 网络
\item 磁盘
\item 副本数
\end{enumbox}

\subsection{精简卷顺序1M写入性能低}

subvol写锁,导致串行化

用allocate命令先批量填充,性能有显著提升。

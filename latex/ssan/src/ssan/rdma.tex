\chapter{RDMA}

\section{编程模型}

理解异步机制。分提交队列和完成队列。多个cm\_id可以共用一个完成队列。一个dev用一个完成队列。

\subsection{建立连接}

\hl{三层:多核并行-调度-事件}

注意事件循环,消息派遣机制,同步操作。
注意内存使用。

C/S,S端收到RDMA\_READ请求时,push数据到客户端的内存。
S端收到RDMA\_WRITE请求时,从客户端pull数据到本地。

rpc构建于RDMA/IB verbs之上,是同步机制。rpc客户端通过post send发送协议参数,
分为读、写两种情况。\hl{RDMA read和write都是由RPC server-side完成的}。

ibv\_post\_recv若干内存区块,以接收ibv\_post\_send消息。处理完一个,再次调用。
就好比有若干空槽,来一个send msg,填充一个,处理完后,reset进入可用状态。
这是不同于tcp send/recv的地方,\hl{send/recv基于字节流,post send/recv基于msg}。
send/recv都不涉及rkey,不需要知道peer的内存地址。

rdma write相当于push本地内存数据到远端应用内存,
rdma read相当于从远端应用内存pull数据到本地内存。

通过send/recv交换双方内存信息,或其它方式进行交换。\hl{在接收端,要先准备post recv}。

首先要掌握连接管理,采用c/s架构。

为什么理解epoll机制不够顺利?原因在于这是一个双层结构的东东。理解RDMA面临同样问题。
rdma\_cm\_id约等于一个tcp连接,也需要区分listen socket和connection socket,与客户端共同构成三角形结构。

第一层是block point,监听多个fd上的事件。

\subsection{通信}

建立连接后,连接两端处在peer角色。且可以在corenet的router表里检索到。

\section{corerpc}

source files
\begin{enumbox}
\item corenet\_connect\_rdma.c
\item rdma\_event.c
\item corenet\_mapping.c 记录nid到sockid的映射
\item corenet\_rdma.c 消息传递API
\end{enumbox}

以stor\_rpc\_read和stor\_rpc\_write为例,理解RDMA通信机制。
post、commit和poll分别对应wr队列和wc队列。wr队列又分为发送和接收队列。

\section{iSER}

use tgt

iscsi initiator和target之间的RDMA,与lich实现类似,由三个transfer接口组成:send、read、write。
target端负责RDMA read/write。\hl{iSER是iscsi over RDMA},payload协议是scsi。

回到网络的基本模型去理解,上层的协议和接口通常不变,新的硬件和transfer,通常引入了新的协议层去适配。
由此可见,整个\hl{网络协议栈}的设计非常强大,具有很好的扩展性和适应性。

交换机并不是什么神秘的东西,可以理解为一台计算机,导出了网络服务,主要功能是port的交换和路由。

\section{NVMf}

use spdk NVMf target

\chapter{Bcache}

\section{已知问题}

\begin{enumbox}
\item disk管理的改动是否影响NVMe,需要测试
\item 加新盘,如果bcache没有被重新格式化
\item 如果触发disk load过程
\item 停止恢复后,gc怎么做?
\end{enumbox}

\section{管理}

bcache是一内置模块,用lsmod看不到。内核4.17版本的bcache符合预期。

与web管理系统配合,通过web能完成大部分操作

disk list,返回json,不能有别的输出,比如调试信息等。

\section{RAID}

有坏盘,导致RAID进入保护模式,能列出物理设备(Foreign状态),不能列出逻辑设备。

RAID变成writethrough模式?\hl{writeback模式的RAID有丢失数据的风险}。

拔的盘,重新加入raid阵列:
\begin{enumbox}
\item raid miss
\item raid load?Foreign状态,Import 导致服务器hang住
\item raid add?Foreign状态,Clear 导致服务器hang住
\end{enumbox}

\section{原理}

二实一虚,2+1=3,两类物理设备,一类虚拟设备,虚拟设备是入口

三种cache mode

writeback的复杂性

基本用法
\begin{enumbox}
\item 分别格式化cache盘与数据盘
\item 数据盘绑定到cache盘
\item 向kernel注册,每个数据盘出现一个/dev/bcacheXX虚拟设备
\end{enumbox}

\subsection{工具}

\begin{enumbox}
\item make-bcache
\item wipefs
\item lsblk
\item bcache-super-show
\item dmesg
\item lsmod
\item /sys/fs/bcache
\item /sys/block
\item dd if=/dev/zero of=/dev/sdb count=1 bs=1024 seek=4
\end{enumbox}

\section{可靠性}

手动测试,无fusionstor的情况
\begin{enumbox}
\item 拔出数据盘
\item 拔出cache盘
\end{enumbox}

拔出cache disk,执行如下步骤进行修复:
\begin{enumbox}
\item 修复RAID
\item 修复bcache的attach关系
\item 修复lich 磁盘软链接(如果没有恢复完成,此时磁盘没有踢出集群)
\item 如果恢复完成,则作为新盘加入
\end{enumbox}

一定需要重启服务器吗?

单节点故障与单磁盘故障有很大不同,对应的恢复过程和性能也有很大不同。

卷控制器rebalance后,不自动触发恢复过程

RECOVERY\_RMQ\_MAX\_RECORD

\subsection{RAID的影响}

% 拔数据的状态机,恢复完成后会踢出该盘

加入raid后,bcache导致kernel hangup。

有什么办法在加入raid前,先禁用bcache。待raid ready后,再启用bcache?
操作范围:cache set。

cache disk不做成RAID0,做出JBOD模式如何?

重启节点,后导致clock丢失,如果一定会发生hangup,最好先stop lich,起到保存clock的目的。

盘符变化后,需要重启lich,重建相关链接

副本多的情况

\section{性能}

测试缓存命中率

单节点故障下的性能
\begin{enumbox}
\item cache disk的性能
\item bcache的各项配置 (cache mode, writeback percent, cutoff)
\end{enumbox}

\section{FAQ}

readitems crc error

btree head error

/dev/sd*多出的设备是因为IPMI。

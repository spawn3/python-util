\documentclass[12pt]{scrreprt}

\usepackage{datetime}

\title{A Simple Document}
\author{Me}

\begin{document}

\maketitle

\tableofcontents

\begin{abstract}
A brief document to
illustrate how to use \LaTeX.
\end{abstract}

\chapter{Introduction}
\label{ch:intro}

\section{The First Section}

This is a simple \LaTeX\ document.
Here is the first paragraph.
The next chapter is Chapter~\ref{ch:another}
and is on page~\pageref{ch:another}.
The next section is Section~\ref{sec:next}.

\section{The Next Section}
\label{sec:next}

Here is the second paragraph\footnote{with a footnote}. 
As you can see it's a rather short paragraph, but not 
as short as the previous one. This document was 
created on: \today\ at \currenttime.

\chapter{Another Chapter}
\label{ch:another}

Here's another very interesting chapter.
We're going to put a picture here later.
See Chapter~\ref{ch:intro} for an 
introduction.

\chapter{Recommended Reading}

For a basic introduction to \LaTeX\ see Lamport~\cite{lamport94}.
For more detailed information about \LaTeX\ and
associated applications, consult Kopka and Daly~\cite{kopka95}
or Goossens \emph{et al}~\cite{goossens94}.

\chapter*{Acknowledgements}

I would like to acknowledge all those
very helpful people who have assisted
me in my work.

\appendix
\chapter{Tables}

We will turn this tabular environment into a table later.

\begin{tabular}{lrr}
 & \multicolumn{2}{c}{\bfseries Expenditure}\\
 & \multicolumn{1}{c}{Year1} & \multicolumn{1}{c}{Year2}\\
\bfseries Travel & 100,000 & 110,000\\
\bfseries Equipment & 50,000 & 60,000
\end{tabular}

\begin{thebibliography}{1}
\bibitem{goossens94} ``The \LaTeX\ Companion'', Michel Goossens, Frank Mittelbach and
Alexander Samarin, Addison-Wesley, (1994).

\bibitem{lamport94} ``\LaTeX\ : a document preparation system'', Leslie Lamport,
2nd edition (updated for \LaTeX2e), Addison-Wesley (1994).

\bibitem{kopka95} ``A Guide to \LaTeX2e: document preparation for beginners
and advanced users'', Helmut Kopka and Patrick W. Daly, Addison-Wesley (1995).

\end{thebibliography}
\end{document}

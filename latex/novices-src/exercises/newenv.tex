\documentclass[12pt,captions=tableabove]{scrbook}

\usepackage{graphicx}
\usepackage{datetime}
\usepackage{caption,subcaption}

\pagestyle{headings}

\newenvironment{exercise}[1]% environment name
{% begin code
  \par\vspace{\baselineskip}\noindent
  \textbf{Exercise (#1)}\begin{itshape}%
  \par\vspace{\baselineskip}\noindent\ignorespaces
}%
{% end code
  \end{itshape}\ignorespacesafterend
}

\title{A Simple Document}
\author{Me}

\begin{document}

\maketitle

\frontmatter
\tableofcontents
\listoffigures

\chapter{Summary}
A brief document to
illustrate how to use \LaTeX.

\mainmatter
\chapter{Introduction}
\label{ch:intro}

\section{The First Section}

This is a simple \LaTeX\ document.
Here is the first paragraph.
The next chapter is Chapter~\ref{ch:another}
and is on page~\pageref{ch:another}.
The next section is Section~\ref{sec:next}.

\begin{exercise}{An Example}
Here's an example exercise.
\end{exercise}

\section{The Next Section}
\label{sec:next}

Here is the second paragraph\footnote{with a footnote}. 
As you can see it's a rather short paragraph, but not 
as short as the previous one. This document was 
created on: \today\ at \currenttime.

\chapter{Another Chapter}
\label{ch:another}

Here's another very interesting chapter.
See Chapter~\ref{ch:intro} for an 
introduction. Our picture has turned
into Figure~\ref{fig:shapes}.

\begin{figure}[htbp]
\centering
 \includegraphics{shapes}
\caption{Some shapes}
\label{fig:shapes}
\end{figure}

\begin{figure}[hbtp]
 \begin{subfigure}[b]{0.5\linewidth}
   \centering
   \includegraphics{rectangle}
   \caption{Rectangle}\label{fig:rectangle}
 \end{subfigure}%
 \begin{subfigure}[b]{0.5\linewidth}
   \centering
   \includegraphics{circle}
   \caption{Circle}\label{fig:circle}
 \end{subfigure}%
\caption{Two Shapes: (\emph{a}) A Rectangle and
(\emph{b}) A Circle}
\label{fig:shapes2}
\end{figure}

Figure~\ref{fig:shapes2} shows some more shapes.
Figure~\ref{fig:rectangle} is a rectangle and
Figure~\ref{fig:circle} shows a circle.

\begin{exercise}{Another Example}
Here's another example exercise.
\end{exercise}

\chapter{Recommended Reading}

For a basic introduction to \LaTeX\ see Lamport~\cite{lamport94}.
For more detailed information about \LaTeX\ and
associated applications, consult Kopka and Daly~\cite{kopka95}
or Goossens \emph{et al}~\cite{goossens94}.

\chapter*{Acknowledgements}

I would like to acknowledge all those
very helpful people who have assisted
me in my work.

\appendix
\chapter{Tables}

Our tabular environment has now turned into Table~\ref{tab:sample}.

\begin{table}[htbp]
\caption{A Sample Table}
\label{tab:sample}
\centering
\begin{tabular}{lrr}
 & \multicolumn{2}{c}{\bfseries Expenditure}\\
 & \multicolumn{1}{c}{Year1} & \multicolumn{1}{c}{Year2}\\
\bfseries Travel & 100,000 & 110,000\\
\bfseries Equipment & 50,000 & 60,000
\end{tabular}
\end{table}

\begin{thebibliography}{1}
\bibitem{goossens94} ``The \LaTeX\ Companion'', Michel Goossens, Frank Mittelbach and
Alexander Samarin, Addison-Wesley, (1994).

\bibitem{lamport94} ``\LaTeX\ : a document preparation system'', Leslie Lamport,
2nd edition (updated for \LaTeX2e), Addison-Wesley (1994).

\bibitem{kopka95} ``A Guide to \LaTeX2e: document preparation for beginners
and advanced users'', Helmut Kopka and Patrick W. Daly, Addison-Wesley (1995).

\end{thebibliography}
\end{document}


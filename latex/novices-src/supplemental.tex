\setnode{supplemental}
\chapter{Supplementary Material}
\label{ch:supp}

This appendix chapter contains three sections from earlier versions of this
book. The information here is a little dated, but is provided in
case it's still of any use.

%%%%%%%%%%%%%%%%%%% Terminal + Text Editor %%%%%%%%%%%%%%%%%%%

\setnode{editorandterminal}
\section{Text Editor and Terminal Approach}
\label{sec:editorandterminal}

Creating a \LaTeX\ document using a text editor and a terminal
is an approach often favoured by UNIX-type users. If you have never
used a terminal (i.e.\ you have only ever used point-and-click menu
driven applications) then you will be better off using a
front-end, in which case I suggest you
\begin{latexonly}%
turn to Sections~\ref{sec:TeXnicCenter} and~\ref{sec:winedt} which 
describe \appname{TeXnicCenter} and \appname{WinEdt}, respectively.
\end{latexonly}
\begin{htmlonly}
look at the \htmlref{\appname{TeXnicCenter}}{sec:TeXnicCenter} and 
\htmlref{\appname{WinEdt}}{sec:winedt} sections instead.
\end{htmlonly}
\par
To begin, you will first need a text editor\faq{TeX-friendly editors
and shells}{editors}. There are a number available that are suited 
to using with \LaTeX, some people advocate
\iappnamelink{Emacs}{http://www.gnu.org/software/emacs/emacs.html},
others advocate \iappnamelink{vim}{http://vim.sourceforge.net/}, and
there are various others such as
\iappnamelink{NEdit}{http://www.nedit.org/}.  I prefer to use
\iappname{vim}---I'm not overly keen on using the mouse, and I prefer
being able to issue all commands via the keyboard (although 
there is a GUI version of \iappname{vim}). As with some other
editors, it comes with syntax highlighting, regular expression search
and replace, auto-insertion, and a brace matching mechanism which I
find useful. If you are using version~7 of \iappname{vim}, there is an
integrated spell checker, otherwise there is a spell checker plug-in
called
\htmladdnormallink{vimspell}{http://www.vim.org/scripts/script.php?script_id=465},
so you can check your spelling as you type.  If there is already a
text editor that you are comfortable with, then stick with that,
otherwise try out available editors, and decide which one you prefer.

When using the terminal and text editor approach, most people usually
have at least two terminals open: one to run the editor, the other to
run \LaTeX. This means that you don't have to keep quitting the
editor every time you want to \htmlref{\LaTeX\ your
document}{itm:step2}. Some editors allow you to run commands, but
personally I don't like to use this approach. If your editor has a
GUI interface, then you'll probably only need one terminal open.

Let's get started: start up your text editor. This is usually done by
entering the name of the editor at the command prompt in your
terminal. With most editors you can also specify the filename as
well. If the file doesn't exist, a new one will be created when you
save your document. \figureref{fig:startvim} shows my terminal.  The
command prompt looks like \verb|[nlct@nlctltpc examples]$|.  It will
be different for your system. I have typed \texttt{vim sample1.tex}
at the command prompt. This will start \iappname{vim} with a new file
called \texttt{sample1.tex}.  (In this section, I will be using
\iappname{vim} as the text editor, if you are not using
\iappname{vim}, then substitute the editor of your choice.)

\begin{figure}[htbp]
\figconts[alt=Image of a terminal]
 {pictures/terminal1}
 {\caption{Starting vim from a terminal}}
 {fig:startvim}
\end{figure}

Once I have pressed the return key, my terminal looks like
\figureref{fig:terminal2}. Normally \iappname{vim} starts in visual
command mode, which means that when you start typing text, it will be
interpreted as part of a command. In order to type text into your
file, you will need to enter input mode. There are a number of ways
of doing this, but pressing \texttt{i} will do for now\footnote{For a
complete set of available commands, see the \appname{vim} manual at
\url{http://vimdoc.sourceforge.net/}}. \figureref{fig:terminal3}
shows how my terminal looks when I am in input mode. I can now go
ahead and type in my text (\figureref{fig:terminal4}). To go back to
the visual command mode, press the escape key (Esc). Now that you are
back in the visual command mode, you can save your document, either
using the command \texttt{:w} if you have already given your file a
name, or \texttt{:w }\meta{filename} (e.g.\ \texttt{:w sample1.tex})
if you started \iappname{vim} without specifying a file, see
\figureref{fig:terminal5}. When you want to quit \iappname{vim} you
can do \texttt{:wq} to save and quit or \texttt{:q!} to quit without
saving, but I suggest you don't do this just yet if you have another
terminal available.

\begin{figure}[htbp]
 \figconts[alt=Image of a terminal with vim running]
 {pictures/terminal2}
 {\caption{Starting a new file in vim}}
 {fig:terminal2}
\end{figure}

\begin{figure}[htbp]
  \figconts[alt=Image of a terminal with vim running in input mode]
  {pictures/terminal3}
  {\caption{Input mode in vim}}
  {fig:terminal3}
\end{figure}

\begin{figure}[htbp]
  \figconts[alt=Image of a terminal with LaTeX code typed in vim]
  {pictures/terminal4}
  {\caption{Creating a sample document in vim}}
  {fig:terminal4}
\end{figure}

\begin{figure}[htbp]
  \figconts[alt=Image of new file being saved in vim]
  {pictures/terminal5}
  {\caption{Saving your document in vim (the file name should be 
    omitted if the file already has a name)}}
  {fig:terminal5}
\end{figure}

\objectref{Step}{itm:step1} is now complete, and you are now ready to
move on to \objectref{Step}{itm:step2}: using \LaTeX. Go to your
other terminal (or quit your editor if you only have access to one
terminal) and make sure that you are in the same directory as the
file you just created.  Typing \texttt{ls} at the command prompt will
list the contents of your current directory. If you do this, you
should see the file that you have just created (see
\figureref{fig:terminal6}).  At the command prompt type\faq{Makefiles
for LaTeX documents}{make}:
\refstepcounter{object}\label{obj:termlatex}
\begin{verbatim}
latex sample1.tex
\end{verbatim}
as shown in \figureref{fig:terminal7}. You can omit the \texttt{.tex}
extension if you like, \LaTeX\ will automatically add this if
it has been omitted. If you prefer to use \iPDFLaTeX, type
\begin{verbatim}
pdflatex sample1.tex
\end{verbatim}
instead (again the \texttt{.tex} extension may be omitted.)
You should now see something like \figureref{fig:terminal8}.

\begin{figure}[htbp]
  \figconts[alt=Image of directory listing in a terminal]
  {pictures/terminal6}
  {\caption{Listing the contents of the current directory}}
  {fig:terminal6}
\end{figure}

\begin{figure}[htbp]
  \figconts[alt=Image of a terminal where user has typed the command
   to run LaTeX]
  {pictures/terminal7}
  {\caption{Running \LaTeX}}
  {fig:terminal7}
\end{figure}

\begin{figure}[htbp]
  \figconts[alt=Image of a terminal showing normal LaTeX messages]
  {pictures/terminal8}
  {\caption{Running \LaTeX}}
  {fig:terminal8}
\end{figure}

\refstepcounter{object}\label{obj:transcript}
Numbers appearing in square brackets, e.g.\ \texttt{[1]}, indicate
which page \LaTeX\ is currently processing.  In this case, there is
only one page.  The last line to appear on screen
indicates that information about this \LaTeX\ run has been written
to the log file \texttt{sample1.log}, which you can look at using 
your text editor.

The most important thing to note is the penultimate line\footnote{if
you are using \PDFLaTeX, it will have \texttt{sample1.pdf} instead of
\texttt{sample1.dvi}}:
\begin{verbatim}
Output written on sample1.dvi (1 page, 248 bytes).
\end{verbatim}
This means that the document has been successfully created, and
is one page long.

If you have made a mistake in your \htmlref{source code}{sec:source},
for example suppose you have missed the starting backslash in
\cmdname{documentclass}, then the output will look something like:
\begin{verbatim}
! LaTeX Error: Missing \begin{document}.

See the LaTeX manual or LaTeX Companion for explanation.
Type  H <return>  for immediate help.
 ...

l.1 d
     ocumentclass[a4paper]{article}
?
\end{verbatim}
There are several things you can do at this point, but the easiest
thing to do is to exit \LaTeX\ by typing \texttt{X} followed by the
return key. Go back to your editor, fix the mistake, save the
document, and then try again. (If you do get an error message, check
\chapterref{ch:errors}.) Note that it is important to always save your
document before running \LaTeX.

\refstepcounter{object}\label{obj:xdvi}
You can view the typeset output by loading the file\indexDVI\  
\texttt{sample1.dvi} into a DVI viewer, such as \iappname{xdvi}
or \iappname{kdvi}. To do this type
\begin{verbatim}
xdvi sample1.dvi
\end{verbatim}
or
\begin{verbatim}
kdvi sample1.dvi
\end{verbatim}
at the command prompt (see
\figureref{fig:terminal9}). You will then see the final output,
as shown in \figureref{fig:terminal10}.

\begin{figure}[htbp]
 \figconts[alt=Image of a terminal where user has typed command to
  load DVI file into kdvi]
  {pictures/terminal9}
  {\caption{Load a DVI file into a DVI viewer}}
  {fig:terminal9}
\end{figure}

\begin{figure}[htbp]
  \figconts[alt=Image of DVI file being viewed using kdvi]
  {pictures/terminal10}
  {\caption{Viewing a DVI file in kdvi}}
  {fig:terminal10}
\end{figure}

If you have used \iPDFLaTeX\ instead of \LaTeX, you should have a
file called \texttt{sample1.pdf} instead of \texttt{sample1.dvi}.
You can view this using a PDF viewer, such as \iappname{acroread}
or \iappname{kpdf}.

Some viewers, such as \iappname{kdvi} and \iappname{kpdf} will
automatically reload the file whenever it is modified, in which case
you may like to keep the viewer open, and as you keep editing and
\LaTeX{}ing your document, the viewer will automatically reload the
new versions. Some viewers, such as \iappname{xpdf} don't
automatically reload, but have a reload facility, which you can use
whenever you \htmlref{\LaTeX\ your document}{itm:step2}.

\refstepcounter{object}\label{obj:dvips}
If you like, you can convert your DVI file\indexDVI\ to PostScript 
using \iappname{dvips}. To do this, type the following at the command
prompt in your terminal:
\begin{verbatim}
dvips -o sample1.ps sample1.dvi
\end{verbatim}
(The \texttt{.dvi} extension may be omitted.)
You can then view the PostScript file using \iappname{ghostscript}
or one of its associated applications, such as \iappname{ghostview}
or \iappname{GSview}. I have \iappname{kghostview} installed on
my laptop, so to view the PostScript file, \texttt{sample1.ps},
I would need to type:
\begin{verbatim}
kghostview sample1.ps
\end{verbatim}
(See \figureref{fig:terminal11}.)

\begin{figure}[htbp]
  \figconts[alt=Image of terminal where user has typed command to
   load PostScript file into kghostview]
   {pictures/terminal11}
  {\caption{Loading a PostScript file}}
  {fig:terminal11}
\end{figure}

%%%%%%%%%%%%%%%%% TeXnicCenter %%%%%%%%%%%%%%%%%%%%%%%%%%%%

\setnode{TeXnicCenter}
\section{TeXnicCenter}
\label{sec:TeXnicCenter}

\iappname{TeXnicCenter} is an application that enables you to edit
\LaTeX\ source code, and simply click on a button to pass the source
code to \LaTeX, and then click on another button to view the
resulting typeset document.  Many people prefer this approach to the
\latexhtml{text editor and terminal approach described in the
previous section}{\htmlref{text editor and terminal}{sec:editorandterminal}
approach}. This section gives a brief overview of 
\appname{TeXnicCenter}, however it has been several years since
I last used it\footnote{I used to use TeXnicCenter and
MiKTeX when I was teaching \LaTeX, but that was my limit of using
\TeX\ under Windows}, so this information may be dated.

\appname{TeXnicCenter} is free and can be downloaded from \gls{ctan} in the
\htmladdnormallink{\texttt{systems/win32/TeXnicCenter/}}{http://www.tex.ac.uk/tex-archive/systems/win32/TeXnicCenter/}
directory or from \url{http://www.toolscenter.org/}.  Note that you
must have a \htmlref{\TeX/\LaTeX\ distribution}{obj:MiKTeX} installed
before you install \appname{TeXnicCenter}. If you installed
\itexdistrolink{proTeXt}{http://www.tug.org/protext}, you should
already have \appname{TeXnicCenter} installed. If you have any
problems with installing or running \appname{TeXnicCenter}, go to
their \latexhtml{help page at
\url{http://www.texniccenter.org/help.html}}{\htmladdnormallink{help
page}{http://www.texniccenter.org/help.html}}.

Once the installation is complete, you can then run
\appname{TeXnicCenter} from the Start Menu: \startmenu{TeXnicCenter
\menuto\ TeXnicCenter} Firstly you should see the tip of the day
window (\figureref{fig:TXCtip}.)

\begin{figure}[hbtp]
  \figconts[alt=Image of TeXnicCenter's tip of the day window]
  {pictures/texniccenter/tip}
  {\caption{TeXnicCenter Tip of the Day Window}}
  {fig:TXCtip}
\end{figure}

You can close this window, and then, if this is the first time you
are using \appname{TeXnicCenter} you will have to use the
configuration wizard to set up \appname{TeXnicCenter} correctly.  I
would recommend that you choose the default settings. (Select
\menu{\underline{N}ext}, \menu{\underline{N}ext} and then
\menu{Finish}.)

\begin{figure}[hbtp]
  \figconts[alt=Image of TeXnicCenter's configuration wizard: welcome
   screen]
  {pictures/texniccenter/setup01}
  {\caption{TeXnicCenter Configuration Wizard}}
  {fig:TXCsetup1}
\end{figure}

\begin{figure}[hbtp]
  \figconts[alt=Image of TeXnicCenter's configuration wizard: use
MiKTeX option selected]
  {pictures/texniccenter/setup02}
  {\caption{TeXnicCenter Configuration Wizard}}
  {fig:TXCsetup2}
\end{figure}

\begin{figure}[hbtp]
\figconts[alt=Image of TeXnicCenter's configuration wizard: finish
configuration]
  {pictures/texniccenter/setup03}
  {\caption{TeXnicCenter Configuration Wizard}}
  {fig:TXCsetup3}
\end{figure}

Now you are ready to use \appname{TeXnicCenter}.  It should look like
\figureref{fig:TXCuse1}.

\begin{figure}[hbtp]
\figconts[alt=Image of TeXnicCenter: no files loaded]
  {pictures/texniccenter/use01}
  {\caption{TeXnicCenter}}
  {fig:TXCuse1}
\end{figure}

\mbox{}\refstepcounter{object}\label{TXCnewproject}%
To start a new project select \menu{File \menuto\ New Project}.
This will open the window shown in \figureref{fig:TXCuse2}.

\begin{figure}[hbtp]
\figconts[alt=Image of TeXnicCenter's New Project dialog box]
  {pictures/texniccenter/use02}
  {\caption{New Project Dialog Box}}
  {fig:TXCuse2}
\end{figure}

Enter a name for your project, and specify the directory where you
want to save your work.  For example, I shall call my project
\dq{example} and I want to save it in 
\verb|c:\My Documents\Nicky\example| (see \figureref{fig:TXCuse3}.)

\begin{figure}[hbtp]
\figconts[alt=Image of TeXnicCenter's New Project dialog box:
project name has been entered]
  {pictures/texniccenter/use03}
  {\caption{New Project Dialog Box}}
  {fig:TXCuse3}
\end{figure}

Select the \dq{Empty Project} icon, and click on \dq{Okay}.  You
should now see something like \figureref{fig:TXCuse4}.

\begin{figure}[hbtp]
\figconts[alt=Image of TeXnicCenter: new project on view]
  {pictures/texniccenter/use04}
  {\caption{TeXnicCenter --- New Project Started}}
  {fig:TXCuse4}
\end{figure}

You can now start typing the source code (we'll cover this
\htmlref{later}{ch:simpledoc}).  See \figureref{fig:TXCuse5}.

\begin{figure}[hbtp]
\figconts[alt=Image of TeXnicCenter: user has typed source code in
the edit panel]
  {pictures/texniccenter/use05}
  {\caption{TeXnicCenter --- Typing in Source Code}}
  {fig:TXCuse5}
\end{figure}

Save it by either clicking on the save icon 
\incGraphics[alt=Image of save icon]{pictures/texniccenter/saveicon} or 
select \menu{File \menuto\ Save}

Now select what type of output you want (DVI, PDF or
\Index{PostScript}\refstepcounter{object}\label{obj:txcdvips}) see
\figureref{fig:TXCuse6}.  If this box is blank, then it's possible
that you didn't complete all the steps in the configuration wizard
described above.

\begin{figure}[hbtp]
\figconts[alt=Image of TeXnicCenter: drop-down menu showing]
  {pictures/texniccenter/use06}
  {\caption{TeXnicCenter --- Selecting Output Type}}
  {fig:TXCuse6}
\end{figure}

\refstepcounter{object}\label{TXC:latex}
Now click on the build output icon 
\incGraphics[alt=Image of build icon]{pictures/texniccenter/build} or 
select \menu{Build \menuto\ Build Output}.  The transcript will be
written in the window at the bottom (see \xfigureref{fig:TXCuse7})
This transcript should be the same as described \latexhtml{on
page~\pageref{obj:transcript}
onwards}{\htmlref{earlier}{obj:transcript}}.  If you have selected
\verb|LaTeX => PDF|, then \appname{TeXnicCenter} will use \iPDFLaTeX\
instead of \LaTeX.  If you have selected \verb|LaTeX => PS|, then
\appname{TeXnicCenter} will use \LaTeX\ followed by \iappname{dvips}
(as in \xfigureref{fig:TXCuse7}).  The \appname{dvips} messages will
follow on from the \LaTeX\ messages.  (If you selected the BibTeX or
MakeIndex features when you initialised the project,
\figureref{fig:TXCuse3}, then \iappname{TeXnicCenter} will also use
the \iBiBTeX\ and \appname{MakeIndex} applications.)

\begin{figure}[hbtp]
\figconts[alt=Image of TeXnicCenter: messages have been printed on
the build panel]
  {pictures/texniccenter/use07}
  {\caption[TeXnicCenter (using LaTeX and dvips)]{TeXnicCenter (using \LaTeX\ and \appname{dvips})}}
  {fig:TXCuse7}
\end{figure}

\refstepcounter{object}\label{TXC:yap}
To view the document, click the View Output button 
\incGraphics[alt=Image of view icon]{pictures/texniccenter/view}.
(Note that if you have selected \verb|LaTeX => PDF| or
\verb|LaTeX => PS| you will need \iappname{Adobe Reader}
or \iappname{GSview}, respectively, to view the output file.)

\refstepcounter{object}\label{TXCerrors}
If there are any errors, you can select \menu{Build \menuto\ Next
Error} and it will show you where the error has occured (See
\figureref{fig:TXCuse8}).  If you do have any errors, check
\chapterref{ch:errors}.

\begin{figure}[hbtp]
\figconts[alt=Image of TeXnicCenter: error message displayed in
build panel. A red arrow points to line containing the error in the
edit panel.]
  {pictures/texniccenter/use08}
  {\caption{TeXnicCenter --- Showing Error}}
  {fig:TXCuse8}
\end{figure}


%%%%%%%%%%%%%%%%%%%%%%%%%%%%% WinEdt %%%%%%%%%%%%%%%%%%%%%%%%%%%%%

\setnode{winedt}
\section{WinEdt}
\label{sec:winedt}

\iappname{WinEdt} (not to be confused with \appname{WinEdit} which is
a completely different application) is an application that enables
you to edit \LaTeX\ source code, and simply click on a button to pass
the source code to \LaTeX, and then click on another button to view
the resulting typeset document.  This section gives a brief overview
of \iappname{WinEdt}, however it has been several years since I
last used it, so this information may be dated.

\iappname{WinEdt} is shareware: it can be downloaded from \gls{ctan} in the
\htmladdnormallink{\texttt{systems/win32/winedt}}{http://www.tex.ac.uk/tex-archive/systems/win32/winedt/}
directory or from \url{http://www.winedt.com/} and evaluated for a
trial period of 31 days, after which, if you want to continue to use
it, you must pay the registration fee.  Details of prices and types
of licence available can be found at \url{http://www.winedt.com/}.
Note that members of the \gls{uktug} can benefit from their 
\iappname{WinEdt} licence
scheme (see \url{http://uk.tug.org/membership} for further details).

Again, you must have a \htmlref{\TeX/\LaTeX\
distribution}{obj:MiKTeX} installed before you start.
\appname{WinEdt} is fairly easy to install.  First unpack all the
files, and then run the \appname{setup.exe} application.  I recommend
that you use the default settings.  If you have any problems
installing or using \appname{WinEdt}, go to
\url{http://www.winedt.com/support.html}.

To run \iappname{WinEdt}, select \appname{WinEdt} from the start menu:
\startmenu{WinEdt \menuto\ WinEdt}
It should look like \figureref{fig:winedt1}.

\begin{figure}[hbtp]
\figconts[alt=Image of WinEdt: no files loaded]
  {pictures/winedt/01}%
  {\caption{WinEdt}}
  {fig:winedt1}
\end{figure}

Click on the \dq{New Document} button or select \menu{File \menuto\
New}.  You can now start typing your \htmlref{source
code}{sec:source} into the \appname{WinEdt} window, as shown in
\figureref{fig:winedt3}.

\begin{figure}[htbp]
\figconts[alt=Image of WinEdt: user has typed source code]
  {pictures/winedt/03}
  {\caption{WinEdt}}
  {fig:winedt3}
\end{figure}

You can now save your document using the \menu{File \menuto\ Save as}
menu.  Select the file type to be \texttt{TeX}, and type in the name
of your file, e.g.\ \texttt{sample1.tex}.  See
\figureref{fig:winedt4}.

\begin{figure}[htbp]
  \figconts[alt=Image save file dialog]
  {pictures/winedt/04}
  {\caption{WinEdt --- Saving the File}}
  {fig:winedt4}
\end{figure}

\refstepcounter{object}\label{winedt:latex}
To \LaTeX\ your document, simply click on the \LaTeX\ button
\incGraphics[alt=Image of LaTeX icon]{pictures/winedt/latex}.
The output will appear in an MSDOS Prompt window (see
\figureref{fig:winedt5}).

\begin{figure}[htbp]
\figconts[alt=Image of MSDOS prompt showing standard LaTeX messages]
  {pictures/winedt/05}%
  {\caption[WinEdt --- LaTeX Output]{WinEdt --- \LaTeX\ Output}}
  {fig:winedt5}
\end{figure}

\refstepcounter{object}\label{winedt:yap}
To view your typeset document, click on the \dq{view DVI}\indexDVI\ 
button \incGraphics[alt=Image of previewer icon]{pictures/winedt/yap}.

\refstepcounter{object}\label{obj:winedtdvips}
You can convert your DVI file\indexDVI\ to \Index{PostScript} by clicking 
the \incGraphics[alt=Image of dvips icon]{pictures/winedt/dvips} button.  
If you have \iappname{GSview} installed, you can then view
the PostScript file by clicking on the 
\incGraphics[alt=Image of GSView icon]{pictures/winedt/gsview2} button.  

Depending on which version of \iappname{WinEdt} you have installed,
there may also be a \iPDFLaTeX\ button which you can click on to
create a Portable Document Format (\texttt{.pdf}) document.  If not,
you can click on the 
\incGraphics[alt=Image of MSDOS icon]{pictures/winedt/msdos} button to 
open up an MS-DOS Prompt window, and type \texttt{pdflatex} followed by 
the filename. For example:
\begin{verbatim}
pdflatex sample1.tex
\end{verbatim}
Note that if the filename contains a
space, you will need to use double quotes:
\begin{verbatim}
pdflatex "my file.tex"
\end{verbatim}


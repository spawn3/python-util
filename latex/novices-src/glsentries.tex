\renewcommand{\summarypreamble}{%
Commands or environments defined in the \LaTeX\ kernel are always
available.%
}

\defgactivechar
 {backslashchar}
 {\backslashsym}
 {}
 {\LaTeX\ Kernel}
 {Escape character (indicates a command).}
 {}

\defgactivechar
 {dbbackslashchar}
 {\dbbackslashsym}
 {\oarg{\meta{height}}}
 {\LaTeX\ Kernel}
 {\nopostdesc}
 {%
   \BeginArgList
    \csentryargitem{height} Extra vertical space.
   \EndArgList
 }

\defgchildactivechar
 {newline.dbbackslashchar}
 {dbbackslashchar}
 {\dbbackslashsym}
 {Breaks a line without justification (\protect\htmlref{starred form}{itm:starredcommand}
   forbids a page break)}

\defgchildactivechar
 {tab.dbbackslashchar}
 {dbbackslashchar}
 {\dbbackslashsym}
 {Starts a new row in tabular-style environments}

\defgcs{tabularnewline}%
 {}%
 {\LaTeX\ Kernel}%
 {Behaves like \nxglsi{tab.dbbackslashchar} in a
   \nxglsni{env-tabular}-like environment but helps to disambiguate a
    \nxglslink{newline.dbbackslashchar}{line break} in a paragraph
    cell from a \nxglslink{tab.dbbackslashchar}{row separator}.}%
 {}

\defgidxactivechar
 {leftbracechar}
 {\leftbracesym}
 {}
 {\LaTeX\ Kernel}
 {Marks the beginning of a \nxglsi{group}.}
 {}

\defgidxactivechar
 {rightbracechar}
 {\rightbracesym}
 {}
 {\LaTeX\ Kernel}
 {Marks the end of a \nxglsi{group}.}
 {}

\defgidxactivecharcs
 {leftbrace}
 {\leftbracesym}
 {}
 {\LaTeX\ Kernel}
 {Left brace \{ character. In math mode may be used as a delimiter.}
 {}

\defgidxactivecharcs
 {rightbrace}
 {\rightbracesym}
 {}
 {\LaTeX\ Kernel}
 {Right brace \{ character. In math mode may be used as a delimiter.}
 {}

\defgidxactivechar
 {emdash}%
 {\emDashcs}%
 {}%
 {\LaTeX\ Kernel}%
 {Em-dash \textemdash\ symbol. (Normally used to
  indicate omissions or interruptions or to highlight a parenthetical element.)
  See also \nxglsni{textemdash}.}%
 {}

\defgidxactivechar
 {endash}%
 {\enDashcs}%
 {}%
 {\LaTeX\ Kernel}%
 {En-dash \textendash\ symbol.  (Normally used for number ranges.)
  See also \nxglsni{textendash}.}%
 {}

\defgidxactivechar
 {questiondown}%
 {\questiondowncs}%
 {}%
 {\LaTeX\ Kernel}%
 {Upside-down question mark \textquestiondown\ symbol. See also \nxglsni{textquestiondown}.}%
 {}

\defgidxactivechar
 [!Y]
 {exclamdown}%
 {\exclamdowncs}%
 {}%
 {\LaTeX\ Kernel}%
 {Upside-down exclamation mark \textexclamdown\ symbol. See also \nxglsni{textexclamdown}.}%
 {}

\defgchar
 {quoteleft}%
 {\quoteleftcs}%
 {}%
 {\LaTeX\ Kernel}%
 {Open quote \textquoteleft\ symbol. See also \nxglsni{textquoteleft}.}%
 {}

\defgchar
 {quotedblleft}%
 {\quotedblleftcs}%
 {}%
 {\LaTeX\ Kernel}%
 {Open double quote \textquotedblleft\ symbol. See also \nxglsni{textquotedblleft}.}%
 {}

\defgchar
 {quoteright}%
 {\quoterightcs}%
 {}%
 {\LaTeX\ Kernel}%
 {Closing quote or apostrophe \textquoteright\ symbol in text mode 
  or prime symbol \ensuremath{'} in math mode. See also \nxglsni{textquoteright}.}%
 {}

\defgchar
 {quotedblright}%
 {\quotedblrightcs}%
 {}%
 {\LaTeX\ Kernel}%
 {Closing double quote \textquotedblright\ symbol in text mode
  or double prime \ensuremath{''} in math mode. See also \nxglsni{textquotedblright}.}%
 {}

\defgchar
 {dash}%
 {\dashcs}%
 {}%
 {\LaTeX\ Kernel}%
 {Hyphen - in text mode or minus sign $-$ in math mode.}%
 {}

\defgidxactivecharcs
 {exclam}
 {\exclamsym}
 {}
 {\LaTeX\ Kernel (Math Mode)}
 {Negative thin space.}
 {}

\defgidxactivechar
 {exclamchar}
 {\exclamsym}
 {}
 {}
 {\nopostdesc}
 {}

\defgchildidxactivechar
 {sentence.exclamchar}
 {exclamchar}
 {\exclamsym}
 {Exclamation symbol (end of sentence marker)}
 

\defgchildidxactivechar
 {makeindex.exclamchar}
 {exclamchar}
 {\exclamsym}
 {\nxiappname{makeindex} sublevel special character}

\defgchildidxactivechar
 {resizebox.exclamchar}
 {exclamchar}
 {\exclamsym}
 {Used in \nxglsi{resizebox} to maintain aspect ratio}

\defgidxactivecharcs
 {csbar}
 {\vbarsym}
 {}
 {\LaTeX\ Kernel (Math Mode)}
 {Double vertical bar \doublebar{} delimiter}
 {}

\defgidxactivechar
 {barchar}
 {\vbarsym}
 {}
 {\LaTeX\ Kernel}
 {\nopostdesc}
 {}

\defgchildidxactivechar
 {delim.barchar}
 {barchar}
 {\vbarsym}
 {Delimiter. (Math mode only. Use \nxglsni{textbar} in text mode.)}

\defgchildidxactivechar
 {array.barchar}
 {barchar}
 {\vbarsym}
 {Vertical rule specifier (\nxglsi{env-tabular} or
   \nxglsi{env-array})}

\defgchar
 {slashchar}
 {\slashsym}
 {}
 {}
 {\nopostdesc}
 {}

\defgchildchar
 {text.slash}
 {slashchar}
 {\slashsym}
 {Forward slash symbol (see also \nxglsi{slash})}

\defgxchildchar
 {dir.slash}
 {slashchar}
 {\slashsym}
 {Directory divider}
 {directory divider}

\defgchildchar
 {delim.slash}
 {slashchar}
 {\slashsym}
 {Forward slash delimiter (math mode)}

\defgidxactivechar
 [!Z]
 {visiblespace}
 {\textvisiblespace}
 {}
 {}
 {A visual indication of a space in the code. When you type up 
   the code, replace all instances of this symbol with a space via the space bar on your 
   keyboard.}
 {}

\defgidxactivecharcs
 {at}
 {\atsym}
 {}
 {\LaTeX\ Kernel}
 {Used when a sentence ends with a capital letter.
   This command should be placed after the letter and before the
   punctuation mark.}
 {}

\defgidxactivechar
 {atchar}
 {\atsym}
 {\marg{\meta{text}}}
 {\LaTeX\ Kernel}
 {%
   Used in the argument of \nxglsi{env-tabular} or 
   \nxglsi{env-array} like environments to specify text to insert between columns.%
 }
 {%
     \BeginArgList
      \csentryargitem{text} Text to insert between columns
     \EndArgList
 }
  
\defgidxactivechar
 {questionchar}
 {\questionsym}
 {}
 {}
 {Question mark (end of sentence marker).}
 {}

\defgchar
 {ltchar}
 {\lesssym}
 {}
 {\LaTeX\ Kernel (Math Mode)}
 {Less than symbol. (Use \nxglsni{textless} in text mode.)}
 {}

\defgchar
 {gtchar}
 {\greatersym}
 {}
 {\LaTeX\ Kernel (Math Mode)}
 {Greater than symbol. (Use \nxglsni{textgreater} in text mode.)}
 {}

\defgchar
 {gtcol}
 {\greatersym}
 {\marg{\meta{decl}}}
 {\nxisty{array} package}
 {%
   Used in \nxglsi{env-tabular} or \nxglsi{env-array} column specifiers before
   \texttt{l}, \texttt{r}, \texttt{c}, \texttt{p}, \texttt{m} or
   \texttt{b} to insert \meta{decl} directly in front of the entry
   for that column.%
 }
 {%
     \BeginArgList
       \csentryargitem{decl} The code to insert at the start of the
         column.
     \EndArgList
 }

\defgchar
 {ltcol}
 {\lesssym}
 {\marg{\meta{decl}}}
 {\nxisty{array} package}
 {%
   Used in \nxglsi{env-tabular} or \nxglsi{env-array} column specifiers after
   \texttt{l}, \texttt{r}, \texttt{c}, \texttt{p}, \texttt{m} or
   \texttt{b} to insert \meta{decl} directly after the entry
   for that column.%
 }
 {%
     \BeginArgList
       \csentryargitem{decl} The code to insert at the end of the
         column.
     \EndArgList
 }

\defgcs{begin}%
 {\marg{\meta{env-name}}\oarg{\meta{env-option}}\marg{\meta{env-arg-1}}\ldots\marg{\meta{env-arg-n}}}%
 {\LaTeX\ Kernel}%
 {%
   Starts an environment. (Must have a matching \nxglsi{end}.)
 }%
 {%
   \BeginArgList
    \csentryargitem{env-name} The name of the environment. (\emph{No
      backslash.})
    \csentryargitem{env-option} An optional argument that may be 
      passed to the environment. Not all environments have optional
      arguments.
    \csentryargitem{env-arg-1}\ldots\meta{env-arg-n} Any mandatory
      arguments required by the environment. Not all environments
      require arguments.
   \EndArgList
 }

\defgcs{end}%
  {\marg{\meta{env-name}}}%
  {\LaTeX\ Kernel}%
  {Ends an environment. (Must have a matching \nxglsi{begin}.)}
  {%
    \BeginArgList
      \csentryargitem{env-name} The name of the environment.
    \EndArgList
  }

\defgcs{documentclass}%
 {\oarg{\meta{option-list}}\marg{\meta{class-name}}}%
 {\LaTeX\ Kernel}%
 {%
   Loads the document class file, which sets up the type of document
   you wish to write.%
 }%
 {%
   \BeginArgList
    \csentryargitem{option-list} A comma-separated list of options to
     pass to the class file or any packages that will later be
     loaded.
    \csentryargitem{class-name} The name of the document class. This
    corresponds to a file called \meta{class-name}\texttt{.cls},
    which must be installed. 
   \EndArgList
 }

\defgcs{usepackage}%
 {\oarg{\meta{option-list}}\marg{\meta{package-list}}}%
 {\LaTeX\ Kernel}%
 {%
   Loads the named packages.%
 }%
 {%
   \BeginArgList
    \csentryargitem{option-list} A comma-separated list of options to
     pass to the package.
    \csentryargitem{package-list} A comma-separated list of package
     names (without the \texttt{.sty} extension).
   \EndArgList
 }

\defgcs{footnote}%
 {\oarg{\meta{number}}\marg{\meta{text}}}%
 {\LaTeX\ Kernel}%
 {Inserts a footnote.}%
 {%
  \BeginArgList
    \csentryargitem{number} Overrides the default footnote number with
    the specified \meta{number}.
    \csentryargitem{text} The footnote text.
  \EndArgList
 }%

\defgcs{textbackslash}%
 {}%
 {\LaTeX\ Kernel (Text Mode)}%
 {Backlash \textbackslash\ symbol. (Use
  \nxglsi{backslash} for math mode.)}%
 {}

\defgcs{backslash}%
 {}%
 {\LaTeX\ Kernel (Math Mode)}%
 {Backslash \protect\ensuremath{\protect\backslash} symbol, which may be used
  as a delimiter. (Use
 \nxglsi{textbackslash} for text mode.)}%
 {}

\defgactivecharcs
 [\underscoresym]
 {underscore}
 {\textunderscore}
 {}
 {\LaTeX\ Kernel}
 {Underscore \_ symbol (see also \nxglsni{textunderscore}).}
 {}

\defgcs{textunderscore}
 {}
 {\LaTeX\ Kernel}
 {Underscore \textunderscore\ symbol (see also \nxglsni{textunderscore}).}
 {}

\defgactivechar
 [\underscoresym]
 {underscorechar}
 {\_}
 {\marg{\meta{maths}}}
 {\LaTeX\ Kernel (Math Mode)}
 {Displays its argument as a subscript.}
 {%
   \BeginArgList
     \csentryargitem{maths} The subscript.
   \EndArgList
 }

\defgcs{textgreater}%
 {}%
 {\LaTeX\ Kernel (Text Mode)}%
 {Greater than \textgreater\ symbol. (Just use \texttt{\textgreater} in math mode.)}%
 {}

\defgcs{textless}%
 {}%
 {\LaTeX\ Kernel (Text Mode)}%
 {Less than \textless\ symbol. (Just use \texttt{\textless} in math mode.)}%
 {}

\defgcs{textasciicircum}%
 {}%
 {\LaTeX\ Kernel}%
 {Circumflex \textasciicircum\ symbol.}%
 {}%

\defgcs[][cshyphenation]{hyphenation}
 {\marg{\meta{word}}}
 {\LaTeX\ Kernel}
 {Specifies hyphenation points.}
 {%
   \BeginArgList
    \csentryargitem{word} Word with hyphen points indicated with a
     dash (\texttt{-}).
   \EndArgList
 }

\defgsymcs[hyphen]{\dashcs}
 {}
 {\LaTeX\ Kernel}
 {Insert discretionary hyphen.}
 {}

\defgactivecharcs
 {dollar}
 {\dollarsym}
 {}
 {\LaTeX\ Kernel}
 {Dollar \$ symbol.}
 {}

\defgactivechar
 {dollarchar}
 {\dollarsym}
 {}
 {\LaTeX\ Kernel}
 {Switches in and out of in-line math mode.}
 {}

\defgcs{textasciitilde}%
 {}%
 {\LaTeX\ Kernel}%
 {Tilde \textasciitilde\ symbol. (If you are typing an
  URL, use the \nxisty{url} package, which provides \nxglsi{url}\marg{\meta{address}}
  that allows you to directly type \textasciitilde\ in the address.)}%
 {}%

\defgcs{url}%
 {\marg{\meta{address}}}%
 {\nxisty{url} package}%
 {Typesets an URL in a typewriter font and allows you to use
  characters such as \textasciitilde.}%
 {%
   \BeginArgList
    \csentryargitem{address} The web address.
   \EndArgList
 }%

\defgcs{dag}%
 {}%
 {\LaTeX\ Kernel}%
 {Dagger \dag\ symbol.}%
 {}

\defgcs{ddag}%
 {}%
 {\LaTeX\ Kernel}%
 {Double-dagger \ddag\ symbol.}%
 {}

\defgcs{textregistered}%
 {}%
 {\LaTeX\ Kernel}%
 {Registered \textregistered\ symbol.}%
 {}

\defgcs{texttrademark}%
 {}%
 {\LaTeX\ Kernel}%
 {Trademark \texttrademark\ symbol.}%
 {}

\defgcs{copyright}%
 {}%
 {\LaTeX\ Kernel}%
 {Copyright \copyright\ symbol.}%
 {}

\defgcs{textquoteright}%
 {}%
 {\LaTeX\ Kernel}%
 {Closing single quote (or apostrophe) \textquoteright\ symbol.}%
 {}

\defgcs{textquoteleft}%
 {}%
 {\LaTeX\ Kernel}%
 {Opening single quote \textquoteleft\ symbol.}%
 {}

\defgcs{textquotedblright}%
 {}%
 {\LaTeX\ Kernel}%
 {Closing double quote \textquotedblright\ symbol.}%
 {}

\defgcs{textquotedblleft}%
 {}%
 {\LaTeX\ Kernel}%
 {Opening double quote \textquotedblleft\ symbol.}%
 {}

\defgcs{textbullet}%
 {}%
 {\LaTeX\ Kernel (Text Mode)}%
 {Bullet \textbullet\ symbol.}%
 {}

\defgcs{textquestiondown}%
 {}%
 {\LaTeX\ Kernel}%
 {Upside-down question mark \textquestiondown\ symbol.}%
 {}

\defgcs{textexclamdown}%
 {}%
 {\LaTeX\ Kernel}%
 {Upside-down exclamation mark \textexclamdown\ symbol.}%
 {}

\defgcs{textendash}%
 {}%
 {\LaTeX\ Kernel}%
 {En-dash \textendash\ symbol. (Normally used for number ranges.)
  See also \nxglsni{endash}.}%
 {}

\defgcs{textemdash}%
 {}%
 {\LaTeX\ Kernel}%
 {Em-dash \textemdash\ symbol. (Normally used to
  indicate omissions or interruptions or to highlight a parenthetical element.)
  See also \nxglsni{emdash}.}%
 {}

\defgcs{textperiodcentered}%
 {}%
 {\LaTeX\ Kernel (Text Mode)}%
 {Centred period \textperiodcentered\ symbol.}%
 {}

\defgcs{i}%
 {}%
 {\LaTeX\ Kernel}%
 {Dotless i character: \i.}%
 {}

\defgcs{j}%
 {}%
 {\LaTeX\ Kernel}%
 {Dotless j character: \dotlessj.}%
 {}

\defgactivecharcs
 {hash}
 {\hashsym}
 {}
 {\LaTeX\ Kernel}
 {Hash \# symbol.}
 {}

\defgactivechar
 {hashchar}
 {\hashsym}
 {\meta{digit}}
 {\LaTeX\ Kernel}
 {Replacement text for argument \meta{digit}.}
 {}

\defgactivecharcs
 {percent}
 {\percentsym}
 {}
 {\LaTeX\ Kernel}
 {Percent \% symbol}
 {}

\defgactivechar
 {percentchar}
 {\percentsym}
 {}
 {\LaTeX\ Kernel}
 {Comment character used to ignore everything up to and including
  the newline character in the \nxgls{source}.}
 {}

\defgactivecharcs
 {amp}
 {\ampsym}
 {}
 {\LaTeX\ Kernel}
 {Ampersand \& symbol}
 {}

\defgactivechar
 {ampchar}
 {\ampsym}
 {}
 {\LaTeX\ Kernel}
 {Alignment tab.}
 {}

\defgcs{S}%
 {}%
 {\LaTeX\ Kernel}%
 {Sectional \S\ symbol.}%
 {}

\defgcs{P}%
 {}%
 {\LaTeX\ Kernel}%
 {Paragraph \P\ symbol.}%
 {}

\defgcs{slash}%
 {}%
 {\LaTeX\ Kernel}%
 {Forward slash \slash\ symbol.}%
 {}

\defgcs{AE}%
 {}%
 {\LaTeX\ Kernel}%
 {\protect\AE\ ligature.}%
 {}

\defgcs{ae}%
 {}%
 {\LaTeX\ Kernel}%
 {\protect\ae\ ligature.}%
 {}

\defgcs{OE}%
 {}%
 {\LaTeX\ Kernel}%
 {\protect\OE\ ligature.}%
 {}

\defgcs{oe}%
 {}%
 {\LaTeX\ Kernel}%
 {\protect\oe\ ligature.}%
 {}

\defgcs{AA}%
 {}%
 {\LaTeX\ Kernel}%
 {Upper case A-ring \AA\ character.}%
 {}

\defgcs{aa}%
 {}%
 {\LaTeX\ Kernel}%
 {Lower case a-ring \aa\ character.}%
 {}

\defgcs{L}%
 {}%
 {\LaTeX\ Kernel}%
 {Upper case L-bar \L\ character.}%
 {}

\defgcs{l}%
 {}%
 {\LaTeX\ Kernel}%
 {Lower case l-bar \l\ character.}%
 {}

\defgcs{O}%
 {}%
 {\LaTeX\ Kernel}%
 {Upper case slashed-O \O\ character.}%
 {}

\defgcs{o}%
 {}%
 {\LaTeX\ Kernel}%
 {Lower case slashed-o \o\ character.}%
 {}

\defgcs{ss}%
 {}%
 {\LaTeX\ Kernel}%
 {Eszett \ss\ character.}%
 {}

\defgcs{SS}%
 {}%
 {\LaTeX\ Kernel}%
 {SS (upper case \ss).}%
 {}

\defgcs{left}
 {\meta{delimiter}}
 {\LaTeX\ Kernel (Math Mode)}
 {Indicates a left stretchable delimiter. Must have a matching
  \nxglsi{right}.}
 {%
   \BeginArgList
    \csentryargitem{delimiter} A delimiter symbol, such as
    \nxglsni{openparen}, or a delimiter command, such as \nxglsni{langle}.
   \EndArgList
 }

\defgcs{right}
 {\meta{delimiter}}
 {\LaTeX\ Kernel (Math Mode)}
 {Indicates a right stretchable delimiter. Must have a matching
  \nxglsi{left}.}
 {%
   \BeginArgList
    \csentryargitem{delimiter} A delimiter symbol, such as
    \nxglsni{closeparen}, or a delimiter command, such as \nxglsni{rangle}.
   \EndArgList
 }

\defgcs{bigl}%
 {\meta{delimiter}}%
 {\nxisty{amsmath} package (Math Mode)}%
 {Left delimiter sizing.}%
 {%
   \BeginArgList
    \csentryargitem{delimiter} A delimiter symbol, such as
    \nxglsni{openparen}, or a delimiter command, such as \nxglsni{langle}.
   \EndArgList
 }

\defgcs{bigr}%
 {\meta{delimiter}}%
 {\nxisty{amsmath} package (Math Mode)}%
 {Right delimiter sizing.}%
 {%
   \BeginArgList
    \csentryargitem{delimiter} A delimiter symbol, such as
    \nxglsni{closeparen}, or a delimiter command, such as \nxglsni{rangle}.
   \EndArgList
 }

\defgcs{Bigl}%
 {\meta{delimiter}}%
 {\nxisty{amsmath} package (Math Mode)}%
 {Left delimiter sizing.}%
 {%
   \BeginArgList
    \csentryargitem{delimiter} A delimiter symbol, such as
    \nxglsni{openparen}, or a delimiter command, such as \nxglsni{langle}.
   \EndArgList
 }

\defgcs{Bigr}%
 {\meta{delimiter}}%
 {\nxisty{amsmath} package (Math Mode)}%
 {Right delimiter sizing.}%
 {%
   \BeginArgList
    \csentryargitem{delimiter} A delimiter symbol, such as
    \nxglsni{closeparen}, or a delimiter command, such as \nxglsni{rangle}.
   \EndArgList
 }

\defgcs{biggl}%
 {\meta{delimiter}}%
 {\nxisty{amsmath} package (Math Mode)}%
 {Left delimiter sizing.}%
 {%
   \BeginArgList
    \csentryargitem{delimiter} A delimiter symbol, such as
    \nxglsni{openparen}, or a delimiter command, such as \nxglsni{langle}.
   \EndArgList
 }

\defgcs{biggr}%
 {\meta{delimiter}}%
 {\nxisty{amsmath} package (Math Mode)}%
 {Right delimiter sizing.}%
 {%
   \BeginArgList
    \csentryargitem{delimiter} A delimiter symbol, such as
    \nxglsni{closeparen}, or a delimiter command, such as \nxglsni{rangle}.
   \EndArgList
 }

\defgcs{Biggl}%
 {\meta{delimiter}}%
 {\nxisty{amsmath} package (Math Mode)}%
 {Left delimiter sizing.}%
 {%
   \BeginArgList
    \csentryargitem{delimiter} A delimiter symbol, such as
    \nxglsni{openparen}, or a delimiter command, such as \nxglsni{langle}.
   \EndArgList
 }

\defgcs{Biggr}%
 {\meta{delimiter}}%
 {\nxisty{amsmath} package (Math Mode)}%
 {Right delimiter sizing.}%
 {%
   \BeginArgList
    \csentryargitem{delimiter} A delimiter symbol, such as
    \nxglsni{closeparen}, or a delimiter command, such as \nxglsni{rangle}.
   \EndArgList
 }

\defgcs{pounds}%
 {}%
 {\LaTeX\ Kernel}%
 {Pound \pounds\ symbol.}%
 {}%


\defgcs{v}%
 {\marg{\meta{c}}}%
 {\LaTeX\ Kernel}%
 {Caron diacritic over \meta{c}. Example: 
  \cmdname{v}\marg{o} produces \ocaron.}%
 {%
   \BeginArgList
    \csentryargitem{c} The character that requires the caron
      diacritic.
   \EndArgList
 }

\defgcs{u}%
 {\marg{\meta{c}}}%
 {\LaTeX\ Kernel}%
 {Breve diacritic over \meta{c}. Example: 
  \cmdname{u}\marg{o} produces \obreve.%
 }%
 {%
   \BeginArgList
    \csentryargitem{c} The character that requires the breve
      diacritic.
   \EndArgList
 }

\defgcs{H}%
 {\marg{\meta{c}}}%
 {\LaTeX\ Kernel}%
 {Double acute diacritic over \meta{c}. Example: 
  \cmdname{H}\marg{o} produces \odoubleacute.%
 }%
 {%
   \BeginArgList
    \csentryargitem{c} The character that requires the double
      acute diacritic.
   \EndArgList
 }

\defgcs{t}%
 {\marg{\meta{characters}}}%
 {\LaTeX\ Kernel}%
 {Tie over \meta{characters}. Example: 
  \cmdname{t}\marg{xy} produces \xytie.%
 }%
 {%
   \BeginArgList
    \csentryargitem{characters} The characters that require the tie.
   \EndArgList
 }

\defgcs{c}%
 {\marg{\meta{c}}}%
 {\LaTeX\ Kernel}%
 {Cedilla under \meta{c}. Example: 
  \cmdname{c}\marg{o} produces \ocedilla.%
 }%
 {%
   \BeginArgList
    \csentryargitem{c} The character that requires the cedilla.
   \EndArgList
 }

\defgcs{d}%
 {\marg{\meta{c}}}%
 {\LaTeX\ Kernel}%
 {Dot under \meta{c}. Example: 
  \cmdname{d}\marg{o} produces \odotunder.%
 }%
 {%
   \BeginArgList
    \csentryargitem{c} The character that requires the dot
     under it.
   \EndArgList
 }

\defgcs{b}%
 {\marg{\meta{c}}}%
 {\LaTeX\ Kernel}%
 {Bar under \meta{c}. Example: 
  \cmdname{b}\marg{r} produces \rbarunder.%
 }%
 {%
   \BeginArgList
    \csentryargitem{c} The character that requires the bar
     under it.
   \EndArgList
 }

\defgcs{r}%
 {\marg{\meta{c}}}%
 {\LaTeX\ Kernel}%
 {Ring over \meta{c}. Example: 
  \cmdname{r}\marg{u} produces \uring.%
 }%
 {%
   \BeginArgList
    \csentryargitem{c} The character that requires the ring
      over it.
   \EndArgList
 }

\defgsymcs[acute]{\quoterightcs}%
 {\marg{\meta{c}}}%
 {\LaTeX\ Kernel}%
 {Acute accent over \meta{c}. Example:
   \cmdname{'}\marg{o} produces \protect\'{o}.%
 }%
 {%
   \BeginArgList
    \csentryargitem{c} The character that requires an acute
     accent over it.
   \EndArgList
 }

\defgsymcs[grave]{\quoteleftcs}%
 {\marg{\meta{c}}}%
 {\LaTeX\ Kernel}%
 {Grave accent over \meta{c}. Example:
   \cmdname{`}\marg{o} produces \protect\`{o}.}%
 {%
   \BeginArgList
    \csentryargitem{c} The character that requires a grave
     accent over it.
   \EndArgList
 }

\defgsymcs[period]{\periodsym}%
 {\marg{\meta{c}}}%
 {\LaTeX\ Kernel}%
 {Dot over \meta{c}. Example:
   \cmdname{.}\marg{o} produces \odotover.%
 }%
 {%
   \BeginArgList
    \csentryargitem{c} The character that requires a dot
     over it.
   \EndArgList
 }

\defgchar
 {periodchar}
 {\periodsym}
 {}
 {\LaTeX\ Kernel}
 {\nopostdesc}
 {}

\defgchildchar
 {sentence.periodchar}
 {periodchar}
 {\periodsym}
 {period (full stop) or decimal point}

\defgchildchar
 {delimiter.periodchar}
 {periodchar}
 {\periodsym}
 {invisible delimiter}

\defgsymcs[macron]{\equalsym}%
 {\marg{\meta{c}}}%
 {\LaTeX\ Kernel}%
 {Macron accent over \meta{c}. Example:
   \cmdname{=}\marg{o} produces \omacron.%
 }%
 {%
   \BeginArgList
    \csentryargitem{c} The character that requires a macron
     accent over it.
   \EndArgList
 }

\defgactivecharcs
 [\circumsym\space]
 {circum}
 {\textasciicircum}
 {\marg{\meta{c}}}
 {\LaTeX\ Kernel}
 {%
   Circumflex accent over \meta{c}. Example:
   \cmdname{\textasciicircum}\marg{o} produces \ocircum.%
 }
 {%
   \BeginArgList
    \csentryargitem{c} The character that requires a
     circumflex accent over it.
   \EndArgList
 }

\defgactivechar
 [\circumsym]
 {circumchar}
 {\textasciicircum}
 {\marg{\meta{maths}}}
 {\LaTeX\ Kernel (Math Mode)}
 {Displays its argument as a superscript.}
 {%
   \BeginArgList
     \csentryargitem{maths} The superscript.
   \EndArgList
 }

\defgactivecharcs
 [\tildesym\space]
 {tilde}
 {\textasciitilde}
 {\marg{\meta{c}}}
 {\LaTeX\ Kernel}
 {%
   Tilde accent over \meta{c}. Example:
   \cmdname{\textasciitilde}\marg{o} produces \otilde.%
 }
 {%
   \BeginArgList
    \csentryargitem{c} The character that requires a
     tilde accent over it.
   \EndArgList
 }

\defgactivechar
 [\tildesym]
 {tildechar}
 {\textasciitilde}
 {}
 {\LaTeX\ Kernel}
 {Unbreakable space.}
 {}

\defgidxactivecharcs
 {doublequote}
 {\doublequotesym}
 {\marg{\meta{c}}}
 {\LaTeX\ Kernel}%
 {%
   Umlaut over \meta{c}. Example:
   \quotecs\marg{o} produces \oumlaut.%
 }%
 {%
   \BeginArgList
    \csentryargitem{c} The character that requires an umlaut
     over it.
   \EndArgList
 }

\defgcs{item}%
 {\oarg{\meta{marker}}}%
 {\LaTeX\ Kernel}%
 {Specifies the start of an item in a list. (Only allowed inside one of
  the list making environments.)}%
 {%
   \BeginArgList
    \csentryargitem{marker} If specified, the default item marker is
     replaced with \meta{marker}.
   \EndArgList
 }

\defgsymcs[itcorr]{\slashsym}%
 {}%
 {\LaTeX\ Kernel}%
 {Italic correction.}%
 {}

\defgcs{textrm}%
 {\marg{\meta{text}}}%
 {\LaTeX\ Kernel}%
 {Renders \meta{text} in the predefined serif font. (Defaults to
  Computer Modern Roman.)}%
 {%
   \BeginArgList
     \csentryargitem{text} The text on which to apply the font change.
   \EndArgList
 }

\defgcs{textsf}%
 {\marg{\meta{text}}}%
 {\LaTeX\ Kernel}%
 {Renders \meta{text} in the predefined sans-serif font. (Defaults to
  Computer Modern Sans.)}%
 {%
   \BeginArgList
     \csentryargitem{text} The text on which to apply the font change.
   \EndArgList
 }

\defgcs{texttt}%
 {\marg{\meta{text}}}%
 {\LaTeX\ Kernel}%
 {Renders \meta{text} in the predefined monospaced font. (Defaults to
  Computer Modern Typewriter.)}%
 {%
   \BeginArgList
     \csentryargitem{text} The text on which to apply the font change.
   \EndArgList
 }

\defgcs{textmd}%
 {\marg{\meta{text}}}%
 {\LaTeX\ Kernel}%
 {Renders \meta{text} with a medium weight in the current font
  family.}%
 {%
   \BeginArgList
     \csentryargitem{text} The text on which to apply the font change.
   \EndArgList
 }

\defgcs{textbf}%
 {\marg{\meta{text}}}%
 {\LaTeX\ Kernel}%
 {Renders \meta{text} with a bold weight in the current font family, if it exists.}%
 {%
   \BeginArgList
     \csentryargitem{text} The text on which to apply the font change.
   \EndArgList
 }

\defgcs{textup}%
 {\marg{\meta{text}}}%
 {\LaTeX\ Kernel}%
 {Renders \meta{text} with the upright form of the current font family.}%
 {%
   \BeginArgList
     \csentryargitem{text} The text on which to apply the font change.
   \EndArgList
 }

\defgcs{textit}%
 {\marg{\meta{text}}}%
 {\LaTeX\ Kernel}%
 {Renders \meta{text} with the italic form of the current font family, if it exists.}%
 {%
   \BeginArgList
     \csentryargitem{text} The text on which to apply the font change.
   \EndArgList
 }

\defgcs{textsl}%
 {\marg{\meta{text}}}%
 {\LaTeX\ Kernel}%
 {Renders \meta{text} with the slanted form of the current font family, if it exists.}%
 {%
   \BeginArgList
     \csentryargitem{text} The text on which to apply the font change.
   \EndArgList
 }

\defgcs{textsc}%
 {\marg{\meta{text}}}%
 {\LaTeX\ Kernel}%
 {Renders \meta{text} with the small-caps form of the current font family, if it exists.}%
 {%
   \BeginArgList
     \csentryargitem{text} The text on which to apply the font change.
   \EndArgList
 }

\defgcs{emph}%
 {\marg{\meta{text}}}%
 {\LaTeX\ Kernel}%
 {Toggles the upright and italic\slash slanted rendering of \meta{text}.}%
 {%
   \BeginArgList
     \csentryargitem{text} The text on which to apply the font change.
   \EndArgList
 }

\defgcs{textnormal}%
 {\marg{\meta{text}}}%
 {\LaTeX\ Kernel}%
 {Renders \meta{text} in the default font style.}%
 {%
   \BeginArgList
     \csentryargitem{text} The text on which to apply the font change.
   \EndArgList
 }

\defgcs{ttdefault}%
 {}%
 {\LaTeX\ Kernel}%
 {The name of the default typewriter family as used by \nxglsni{ttfamily}. 
  Defaults to \texttt{cmtt} (Computer Modern Typewriter).}%
 {}

\defgcs{rmdefault}%
 {}%
 {\LaTeX\ Kernel}%
 {The name of the default serif family as used by \nxglsni{rmfamily}. 
  Defaults to \texttt{cmr} (Computer Modern Roman).}%
 {}

\defgcs{sfdefault}%
 {}%
 {\LaTeX\ Kernel}%
 {The name of the default sans-serif family as used by \nxglsni{sffamily}. 
  Defaults to \texttt{cmss} (Computer Modern Sans-serif).}%
 {}

\defgcs{familydefault}%
 {}%
 {\LaTeX\ Kernel}%
 {Specifies the default font family. Defaults to
   \nxglsi{rmdefault} but may be redefined by certain classes.}%
 {}

\defgcs{rmfamily}%
 {}%
 {\LaTeX\ Kernel}%
 {Switches to the predefined serif font. (Defaults to
  Computer Modern Roman.)}%
 {}

\defgcs{sffamily}%
 {}%
 {\LaTeX\ Kernel}%
 {Switches to the predefined sans-serif font. (Defaults to
  Computer Modern Sans.)}%
 {}

\defgcs{ttfamily}%
 {}%
 {\LaTeX\ Kernel}%
 {Switches to the predefined monospaced font. (Defaults to
  Computer Modern Typewriter.)}%
 {}

\defgcs{mdseries}%
 {}%
 {\LaTeX\ Kernel}%
 {Switches to the medium weight in the current font
  family.}%
 {}

\defgcs{bfseries}%
 {}%
 {\LaTeX\ Kernel}%
 {Switches to the bold weight in the current font
  family.}%
 {}


\defgcs{upshape}%
 {}%
 {\LaTeX\ Kernel}%
 {Switches to the upright form of the current font family.}%
 {}

\defgcs{itshape}%
 {}%
 {\LaTeX\ Kernel}%
 {Switches to the italic form of the current font family, if it exists.}%
 {}

\defgcs{slshape}%
 {}%
 {\LaTeX\ Kernel}%
 {Switches to the slanted form of the current font family, if it exists.}%
 {}

\defgcs{scshape}%
 {}%
 {\LaTeX\ Kernel}%
 {Switches to the small-caps form of the current font family, if it exists.}%
 {}

\defgcs{em}%
 {}%
 {\LaTeX\ Kernel}%
 {Toggles the upright and italic\slash slanted form of the current font family.}%
 {}

\defgcs{normalfont}%
 {}%
 {\LaTeX\ Kernel}%
 {Switches to the default font style.}%
 {}

\defgcs{tiny}%
 {}%
 {Most document classes}%
 {Switches to tiny sized text.}%
 {}

\defgcs{scriptsize}%
 {}%
 {Most document classes}%
 {Switches to sub- or superscript sized text.}%
 {}

\defgcs{footnotesize}%
 {}%
 {Most document classes}%
 {Switches to footnote sized text.}%
 {}

\defgcs{small}%
 {}%
 {Most document classes}%
 {Switches to small sized text.}%
 {}

\defgcs{normalsize}%
 {}%
 {\LaTeX\ Kernel}%
 {Switches to normal sized text.}%
 {%
 }

\defgcs{large}%
 {}%
 {Most document classes}%
 {Switches to large sized text.}%
 {%
 }

\defgcs{Large}%
 {}%
 {Most document classes}%
 {Switches to extra-large sized text.}%
 {%
 }

\defgcs{LARGE}%
 {}%
 {Most document classes}%
 {Switches to extra-extra-large sized text.}%
 {%
 }

\defgcs{huge}%
 {}%
 {Most document classes}%
 {Switches to huge sized text.}%
 {%
 }

\defgcs{Huge}%
 {}%
 {Most document classes}%
 {Switches to extra-huge sized text.}%
 {%
 }

\defgcs{multicolumn}%
 {\marg{\meta{cols spanned}}\marg{\meta{col specifier}}\marg{\meta{text}}}%
 {\LaTeX\ Kernel}%
 {Spans multiple columns in a tabular-style environment.}%
 {%
   \BeginArgList
    \csentryargitem{cols spanned} The number of columns to span.
    \csentryargitem{col specifier} The alignment of this column-spanning entry.
    \csentryargitem{text} The contents of this column-spanning entry.
   \EndArgList
 }

\defgcs{toprule}%
 {\oarg{\meta{wd}}}%
 {\nxisty{booktabs} package}%
 {Horizontal rule for the top of a \nxglsi{env-tabular} environment.}%
 {%
   \BeginArgList
    \csentryargitem{wd} Thickness of the rule (dimension).
   \EndArgList
 }

\defgcs{bottomrule}%
 {\oarg{\meta{wd}}}%
 {\nxisty{booktabs} package}%
 {Horizontal rule for the bottom of a \nxglsi{env-tabular} environment.}%
 {%
   \BeginArgList
    \csentryargitem{wd} Thickness of the rule (dimension).
   \EndArgList
 }

\defgcs{midrule}%
 {\oarg{\meta{wd}}}%
 {\nxisty{booktabs} package}%
 {Horizontal rule to go below headings row of a \nxglsi{env-tabular} environment.}%
 {%
   \BeginArgList
    \csentryargitem{wd} Thickness of the rule (dimension).
   \EndArgList
 }

\defgcs{cmidrule}%
 {\oarg{\meta{wd}}\parg{\meta{trim}}\marg{\meta{n}-\meta{m}}}%
 {\nxisty{booktabs} package}%
 {Like \nxglsi{midrule} but only spans column \meta{n} to column
  \meta{m}.}%
 {%
   \BeginArgList
    \csentryargitem{wd} Thickness of the rule (dimension).
    \csentryargitem{trim} Trimming commands. Specifications:
     \texttt{r}, \texttt{r}\marg{\meta{wd}}, \texttt{l} and
      \texttt{l}\marg{\meta{wd}}
    \csentryargitem{n} Index of start column.
    \csentryargitem{m} Index of end column.
   \EndArgList
 }

\defgcs{heavyrulewidth}%
 {}%
 {\nxisty{booktabs} package}%
 {\nxGls{length} register specifying the thickness of \nxglsi{toprule} and
 \nxglsi{bottomrule}.}%
 {}

\defgcs{lightrulewidth}%
 {}%
 {\nxisty{booktabs} package}%
 {\nxGls{length} register specifying the thickness of \nxglsi{midrule}.}%
 {}

\defgcs{addlinespace}%
 {\oarg{\meta{wd}}}%
 {\nxisty{booktabs} package}%
 {Adds extra vertical space (\meta{wd} if specified, otherwise
  \nxglsi{defaultaddspace}) in a row of 
  a \nxglsi{env-tabular} environment. (Must be inserted after
  \nxglsi{dbbackslashchar} marker.)}%
 {%
   \BeginArgList
    \csentryargitem{wd} Height of space (dimension). Default is
     \nxglsi{defaultaddspace}.
   \EndArgList
 }

\defgcs{defaultaddspace}%
 {}%
 {\nxisty{booktabs} package}%
 {Default extra space before or after an adjacent rule.}%
 {}

\defgcs{parbox}%
 {\oarg{\meta{pos}}\oarg{\meta{height}}\marg{\meta{width}}\marg{\meta{text}}}%
 {\LaTeX\ Kernel}%
 {Makes a box with line-wrapped contents. (More restrictive than
  \nxglsi{env-minipage}.)}%
 {%
   \BeginArgList
    \csentryargitem{pos} The vertical alignment of the box relative to
     the surrounding text. (Centred if omitted.)
    \csentryargitem{height} The height of the box. (If omitted the
     height is the natural height of the contents of the box.)
    \csentryargitem{width} The width of the box.
    \csentryargitem{text} The contents of the box.
   \EndArgList
 }

\defgcs{author}%
 {\marg{\meta{name}}}%
 {Most classes that have the concept of a title page}%
 {Specifies the document author (or authors). This command doesn't
  display any text so may be used in the preamble, but if it's not
  in the preamble it must be placed before \nxglsi{maketitle}.}%
 {%
   \BeginArgList
    \csentryargitem{name} The name (or names) of the document author
     (or authors).
   \EndArgList
   Note that some classes, such as those supplied by journals or
   conference proceedings, may also define an optional argument
   that can be used to specify an abbreviated author list for the
   page headers.
 }

\defgcs{title}%
 {\marg{\meta{text}}}%
 {Most classes that have the concept of a title page}%
 {Specifies the document title. This command doesn't
  display any text so may be used in the preamble, but if it's not
  in the preamble it must be placed before \nxglsi{maketitle}.}%
 {%
   \BeginArgList
    \csentryargitem{text} The title of the document.
   \EndArgList
   Note that some classes, such as those supplied by journals or
   conference proceedings, may also define an optional argument
   that can be used to specify an abbreviated title for the page
   headers.
 }

\defgcs{date}%
 {\marg{\meta{text}}}%
 {Most classes that have the concept of a title page}%
 {Specifies the document date. This command doesn't
  display any text so may be used in the preamble, but if it's not
  in the preamble it must be placed before \nxglsi{maketitle}. If omitted, most
  classes assume the current date (as provided by \nxglsi{today}).}%
 {%
   \BeginArgList
    \csentryargitem{text} The document date.
   \EndArgList
 }

\defgcs{titlehead}%
 {\marg{\meta{text}}}%
 {\nxicls{scrartcl}, \nxicls{scrreprt}, \nxicls{scrbook} classes}%
 {Specifies the title header (typeset at the top of the title page).}%
 {%
   \BeginArgList
    \csentryargitem{text} The title header text.
   \EndArgList
 }

\defgcs{subtitle}%
 {\marg{\meta{text}}}%
 {\nxicls{scrartcl}, \nxicls{scrreprt}, \nxicls{scrbook} classes}%
 {Specifies the subtitle (typeset just below the title).}%
 {%
   \BeginArgList
    \csentryargitem{text} The subtitle text.
   \EndArgList
 }

\defgcs{subject}%
 {\marg{\meta{text}}}%
 {\nxicls{scrartcl}, \nxicls{scrreprt}, \nxicls{scrbook} classes}%
 {Specifies the subject (typeset just above the title).}%
 {%
   \BeginArgList
    \csentryargitem{text} The subject.
   \EndArgList
 }

\defgcs{publishers}%
 {\marg{\meta{text}}}%
 {\nxicls{scrartcl}, \nxicls{scrreprt}, \nxicls{scrbook} classes}%
 {Specifies the publisher (typeset after all the other titling
   information).}%
 {%
   \BeginArgList
    \csentryargitem{text} The publisher text.
   \EndArgList
 }

\defgcs{maketitle}%
 {}%
 {Most classes that have the concept of a title page}%
 {Generates the title page (or title block). This command is usually
  placed at the beginning of the document environment.}%
 {}

\defgcs{thanks}%
 {\marg{\meta{text}}}%
 {Most classes that have the concept of a title page}%
 {Inserts a special type of footnote in one of the titling fields,
  such as \nxglsi{author} or \nxglsni{title}. Usually used for some form of
  acknowledgement or affiliation.}%
 {}

\defgcs{today}%
 {}%
 {Most of the commonly-used classes}%
 {Inserts into the output file the date when the \LaTeX\ 
  application created it from the source code.}%
 {}

\defgcs{part}%
 {\oarg{\meta{short title}}\marg{\meta{title}}}%
 {Most classes that have the concept of document structure}%
 {Inserts a part sectional unit. \textbf{This command has a \htmlref{moving
  argument}{sec:fragile}.}}%
 {%
   \BeginArgList
     \csentryargitem{short title} An abbreviated form of the title to
       go in the table of contents or the page header.
     \csentryargitem{title} The title.
   \EndArgList
   The starred form of this command doesn't have an optional
   argument and doesn't increment or display the part counter.
 }

\defgcs{chapter}%
 {\oarg{\meta{short title}}\marg{\meta{title}}}%
 {Book-style classes (such as \nxicls{scrbook} or \nxicls{scrreprt}) that have 
  the concept of chapters}%
 {Inserts a chapter heading. \textbf{This command has a \htmlref{moving
  argument}{sec:fragile}.}}%
 {%
   \BeginArgList
     \csentryargitem{short title} An abbreviated form of the title to
       go in the table of contents or the page header.
     \csentryargitem{title} The title.
   \EndArgList
   The starred form of this command doesn't have an optional
   argument and doesn't increment or display the chapter counter.
 }

\defgcs{section}%
 {\oarg{\meta{short title}}\marg{\meta{title}}}%
 {Most classes that have the concept of document structure}%
 {Inserts a section header. \textbf{This command has a \htmlref{moving
  argument}{sec:fragile}.}}%
 {%
   \BeginArgList
     \csentryargitem{short title} An abbreviated form of the title to
       go in the table of contents or the page header.
     \csentryargitem{title} The title.
   \EndArgList
   The starred form of this command doesn't have an optional
   argument and doesn't increment or display the section counter.
 }

\defgcs{subsection}%
 {\oarg{\meta{short title}}\marg{\meta{title}}}%
 {Most classes that have the concept of document structure}%
 {Inserts a subsection header. \textbf{This command has a \htmlref{moving
  argument}{sec:fragile}.}}%
 {%
   \BeginArgList
     \csentryargitem{short title} An abbreviated form of the title to
       go in the table of contents or the page header.
     \csentryargitem{title} The title.
   \EndArgList
   The starred form of this command doesn't have an optional
   argument and doesn't increment or display the subsection counter.
 }

\defgcs{subsubsection}%
 {\oarg{\meta{short title}}\marg{\meta{title}}}%
 {Most classes that have the concept of document structure}%
 {Inserts a subsubsection header. \textbf{This command has a \htmlref{moving
  argument}{sec:fragile}.}}%
 {%
   \BeginArgList
     \csentryargitem{short title} An abbreviated form of the title to
       go in the table of contents or the page header.
     \csentryargitem{title} The title.
   \EndArgList
   The starred form of this command doesn't have an optional
   argument and doesn't increment or display the subsubsection counter.
 }

\defgcs{paragraph}%
 {\oarg{\meta{short title}}\marg{\meta{title}}}%
 {Most classes that have the concept of document structure}%
 {Inserts a subsubsubsection header. Most classes default to an 
  unnumbered running header for
  this sectional unit. \textbf{This command has a \htmlref{moving
  argument}{sec:fragile}.}}%
 {%
   \BeginArgList
     \csentryargitem{short title} An abbreviated form of the title to
       go in the table of contents or the page header.
     \csentryargitem{title} The title.
   \EndArgList
   The starred form of this command doesn't have an optional
   argument and doesn't increment or display the associated counter.
 }

\defgcs{subparagraph}%
 {\oarg{\meta{short title}}\marg{\meta{title}}}%
 {Most classes that have the concept of document structure}%
 {Inserts a subsubsubsubsection header. Most classes default 
  to an unnumbered running header for
  this sectional unit. \textbf{This command has a \htmlref{moving
  argument}{sec:fragile}.}}%
 {%
   \BeginArgList
     \csentryargitem{short title} An abbreviated form of the title to
       go in the table of contents or the page header.
     \csentryargitem{title} The title.
   \EndArgList
   The starred form of this command doesn't have an optional
   argument and doesn't increment or display the associated counter.
 }

\defgcs{minisec}%
 {\marg{\meta{heading}}}%
 {\nxicls{scrartcl}, \nxicls{scrreprt} and
  \nxicls{scrbook} classes}%
 {An unnumbered heading not associated with any structuring level.}%
 {%
   \BeginArgList
    \csentryargitem{heading} The heading text.
   \EndArgList
 }

\defgcs{addtokomafont}%
 {\marg{\meta{element name}}\marg{\meta{commands}}}%
 {\nxicls{scrartcl}, \nxicls{scrreprt} and
  \nxicls{scrbook} classes}%
 {Sets the font characteristics for the given \koma\ element.}%
 {%
   \BeginArgList
     \csentryargitem{element name} The element's name, for example
      \texttt{chapter}. See the \koma\ manual for a full list
      of defined elements.
     \csentryargitem{commands} The font changing commands to apply
      to that element.
   \EndArgList
 }

\defgcs{appendix}%
 {}%
 {Most classes that have the concept of document structure}%
 {Indicates (but doesn't print anything) that the document is
  switching to the appendices. If chapters exist, the chapter
  numbering is reset and switched to a different format
  (usually upper case letters) otherwise the section numbering
  is reset and switched to a different format.}%
 {}

\defgcs{tableofcontents}
 {}%
 {Most classes that have the concept of document structure}%
 {Inserts the table of contents. A second (possibly third) run
  is required to ensure the page numbering is correct.}%
 {}

\defgcs{addcontentsline}%
 {\marg{\meta{toc}}\marg{\meta{section unit}}\marg{\meta{text}}}%
 {\LaTeX\ Kernel}%
 {Adds a sectional unit header to the contents list.}%
 {%
   \BeginArgList
    \csentryargitem{toc} The extension of the required external file
    without the dot (for example, \texttt{toc} for the table of
     contents or \texttt{lof} for the list of figures).
    \csentryargitem{section unit} The name of the sectional unit (for
     example, \texttt{chapter}).
    \csentryargitem{text} The text to be added to the contents list.
   \EndArgList
 }

\defgcs{label}%
 {\marg{\meta{string}}}%
 {\LaTeX\ Kernel}%
 {Assigns a unique textual label linked to the most recently
  incremented cross-referencing counter in the current \nxglslink{group}{scope} (see
  also \nxglsi{ref}).}%
 {%
   \BeginArgList
    \csentryargitem{string} A unique label that can be referenced
     elsewhere in the document with \nxglsi{ref}. (It's best to 
     just use alphanumerics and non-active punctuation characters in
     the label.)
   \EndArgList
 }

\defgcs{ref}%
 {\marg{\meta{string}}}%
 {\LaTeX\ Kernel}%
 {References the value of the counter linked to the given label.
  A second (possibly third) run of \LaTeX\ is required to ensure the cross-references
  are up-to-date.
 }%
 {%
   \BeginArgList
    \csentryargitem{string} The required label that was used in the
     corresponding \nxglsi{label}.
   \EndArgList
 }

\defgcs{pageref}%
 {\marg{\meta{string}}}%
 {\LaTeX\ Kernel}%
 {Similar to \nxglsi{ref} but inserts the page number where the given
  label was defined.
  A second (possibly third) run of \LaTeX\ is required to ensure the cross-references
  are up-to-date.
 }%
 {%
   \BeginArgList
    \csentryargitem{string} The required label that was used in the
     corresponding \nxglsi{label}.
   \EndArgList
 }

\defgcs{vref}%
 {\marg{\meta{string}}}%
 {\nxisty{varioref} package}%
 {Like \nxglsi{ref} but also adds information about the location, such
  as \dq{on page~\meta{n}} or \dq{on the following page}.
 }%
 {%
   \BeginArgList
    \csentryargitem{string} The required label that was used in the
     corresponding \nxglsi{label}.
   \EndArgList
 }

\defgcs{bibitem}%
 {\oarg{\meta{tag}}\marg{\meta{key}}}%
 {\LaTeX\ Kernel}%
 {Indicates the start of a new reference in the bibliography. May
  only be used inside the contents of \nxglsi{env-thebibliography} environment}%
 {%
   \BeginArgList
    \csentryargitem{tag} If present, overrides the marker at the
    start of the reference.
    \csentryargitem{key} A unique key that identifies this reference so
    it can be cited elsewhere in the document using \nxglsi{cite}.
   \EndArgList
 }

\defgcs{cite}%
 {\oarg{\meta{text}}\marg{\meta{key list}}}%
 {\LaTeX\ Kernel}%
 {Inserts the citation markers of each reference identified in the
  key list. A second run is required to ensure the reference is
  correct.}%
 {%
    \BeginArgList
      \csentryargitem{text} Additional text to insert into the citation
       (such as the chapter number, or a particular page or
       page range within the citation).
      \csentryargitem{key list} A comma-separated list of keys used in
      the corresponding \nxglsi{cite}.
    \EndArgList
 }

\defgcs{pagenumbering}%
 {\marg{\meta{style}}}%
 {\LaTeX\ Kernel}%
 {Sets the style of the page numbers.}%
 {%
   \BeginArgList
     \csentryargitem{style} The page numbering style (e.g.\
      \texttt{roman} for lower case Roman numerals).
   \EndArgList
 }

\defgcs{pagestyle}%
 {\marg{\meta{style}}}%
 {\LaTeX\ Kernel}%
 {Sets the style of the headers and footers.}%
 {%
   \BeginArgList
     \csentryargitem{style} The name of the page style. The \LaTeX\
     kernel defines only two styles: \nxipagestyle{empty} and
     \nxipagestyle{plain}. Most of the standard classes also provide
     the \nxipagestyle{headings} style.
   \EndArgList
 }

\defgcs{thispagestyle}%
 {\marg{\meta{style}}}%
 {\LaTeX\ Kernel}%
 {Like \nxglsi{pagestyle} but only affects the current page.}%
 {%
   \BeginArgList
     \csentryargitem{style} The name of the page style.
   \EndArgList
 }

\defgcs{markboth}%
 {\marg{\meta{left head}}\marg{\meta{right head}}}%
 {\LaTeX\ Kernel}%
 {Specifies information for the left and right page headers. Not all
   page styles use this information, in which case the arguments are
   ignored.}%
 {%
   \BeginArgList
     \csentryargitem{left head} Text for the left (even) page header,
     if the document has been identified as a double-sided document
     (usually with the \nxclsopt{twoside} class option).

     \csentryargitem{right head} Text for the right page header,
     if the \nxclsopt{twoside} option has been used. Otherwise used
     for both odd and even page headers.
   \EndArgList
 }

\defgcs{markright}%
 {\marg{\meta{right head}}}%
 {\LaTeX\ Kernel}%
 {Specifies information for the right (odd) page header. Not all
  page styles use this information, in which case the argument is
  ignored.}%
 {%
   \BeginArgList
     \csentryargitem{right head} Text for the right page header,
     if the \nxclsopt{twoside} option has been used. Otherwise used
     for both odd and even page headers.
   \EndArgList
 }

\defgcs{includegraphics}%
 {\oarg{\meta{key vals}}\marg{\meta{filename}}}%
 {\nxisty{graphicx} package}%
 {Inserts a graphics file into the document.}%
 {%
   \BeginArgList
     \csentryargitem{key vals} Comma-separated list of options that can
      be used to change the way the image is displayed.
   \EndArgList
 }

\defgcs{DeclareGraphicsExtensions}%
 {\marg{\meta{ext-list}}}%
 {\nxisty{graphicx} package}%
 {Specify the file extensions to look for if no extension is used in
 \nxglsi{includegraphics}}%
 {%
   \BeginArgList
     \csentryargitem{ext-list} A comma-separated list of extensions.
   \EndArgList
 }

\defgcs{selectlanguage}
 {\marg{\meta{language name}}}%
 {\nxisty{babel} package}%
 {Switches to the named language. Predefined textual elements, such
  as the date given by \nxglsi{today} or prefixes like \dq{Chapter},
  are redefined to those supplied by the given language.}%
 {%
   \BeginArgList
    \csentryargitem{language} The language identifier.
   \EndArgList
 }

\defgcs{foreignlanguage}%
 {\marg{\meta{language name}}\marg{\meta{text}}}%
 {\nxisty{babel} package}%
 {Typesets the given text using any predefined names or date formats supplied by the
  given language.
 }%
 {%
   \BeginArgList
    \csentryargitem{language} The language identifier.
    \csentryargitem{text} The foreign text.
   \EndArgList
 }

\defgcs{iflanguage}%
 {\marg{\meta{language name}}\marg{\meta{true text}}\marg{\meta{false
text}}}%
 {\nxisty{babel} package}%
 {Tests the current language.}%
 {%
   \BeginArgList
    \csentryargitem{language} The language identifier.
    \csentryargitem{true text} The text to typeset if true.
    \csentryargitem{false text} The text to typeset if not true.
   \EndArgList
 }

\defgcs{caption}%
 {\oarg{\meta{short caption}}\marg{\meta{caption text}}}%
 {\LaTeX\ Kernel}%
 {Inserts the caption for a float such as a figure or table. 
  \textbf{This command has a \htmlref{moving argument}{sec:fragile}.}}%
 {%
   \BeginArgList
    \csentryargitem{short caption} If provided, an abbreviated caption
      to go in the list of figures\slash tables etc.
    \csentryargitem{caption text} The caption contents.
   \EndArgList
 }

\defgcs{listoffigures}%
 {}%
 {Most classes that have the concept of document structure}%
 {Inserts the list of figures. A second (possibly third) run
  is required to ensure the page numbering is correct.}%
 {}

\defgcs{listoftables}%
 {}%
 {Most classes that have the concept of document structure}%
 {Inserts the list of tables. A second (possibly third) run
  is required to ensure the page numbering is correct.}%
 {}

\defgcs{hspace}%
 {\marg{\meta{length}}}%
 {\LaTeX\ Kernel}%
 {Inserts a horizontal gap of the given width.}%
 {%
   \BeginArgList
    \csentryargitem{length} The width of the horizontal gap.
   \EndArgList
 }

\defgcs{vspace}%
 {\marg{\meta{length}}}%
 {\LaTeX\ Kernel}%
 {Inserts a vertical gap of the given height.}%
 {%
   \BeginArgList
    \csentryargitem{length} The \protect\htmlref{height}{sec:length} of the vertical gap.
   \EndArgList
 }

\defgcs{baselineskip}%
 {}%
 {\LaTeX\ Kernel}%
 {A \nxglsi{length} register that stores the current interline spacing.
  This is recalculated whenever the font changes.}%
 {}

\defgcs{parskip}%
 {}%
 {\LaTeX\ Kernel}%
 {A \nxglsi{length} register that stores the spacing between paragraphs.
  (If you're using one of the \koma\ classes, use the \nxscrclsopt{parskip} 
   option to set it to full or half line height.)}%
 {}

\defgcs{twocolumn}%
 {\oarg{\meta{header text}}}%
 {\LaTeX\ Kernel}%
 {Starts a new page and switches to two column mode.}%
 {%
   \BeginArgList
    \csentryargitem{header text} If present, this header text will be
    placed at the top of the page spanning both columns. (For
    example, as used by \nxglsi{maketitle} in two column articles.)
   \EndArgList
 }

\defgcs{onecolumn}%
 {}%
 {\LaTeX\ Kernel}%
 {Starts a new page and switches to one column mode.}%
 {}

\defgcs{TeX}%
 {}%
 {\LaTeX\ Kernel}%
 {Typesets the \TeX\ logo.}%
 {}

\defgcs{LaTeX}%
 {}%
 {\LaTeX\ Kernel}%
 {Typesets the \makeimg{LaTeX}{\LaTeX}\ logo.}%
 {}

\defgcs{LaTeXe}%
 {}%
 {\LaTeX\ Kernel}%
 {Typesets the \makeimg{LaTeX2e}{\LaTeXe}\ logo.}%
 {}

\defgcs{framebox}%
 {\oarg{\meta{width}}\oarg{\meta{align}}\marg{\meta{text}}}%
 {\LaTeX\ Kernel}%
 {Puts a frame around its contents, prohibiting a line break in the
  contents.}%
 {%
   \BeginArgList
     \csentryargitem{width} The width of the box. (If omitted, this value is
      the natural width of the box contents.)
     \csentryargitem{align} Horizontal alignment of box contents
      (\texttt{c} if omitted).
     \csentryargitem{text} The contents of the box.
   \EndArgList
 }

\defgcs{fbox}%
 {\marg{\meta{text}}}%
 {\LaTeX\ Kernel}%
 {Puts a frame around its contents, prohibiting a line break in the
  contents.}%
 {%
   \BeginArgList
     \csentryargitem{text} The contents of the box.
   \EndArgList
 }

\defgcs{shadowbox}%
 {\marg{\meta{text}}}%
 {\nxisty{fancybox} package}%
 {Puts a shadow frame around its contents, prohibiting a line break in the
  contents.}%
 {%
   \BeginArgList
     \csentryargitem{text} The contents of the box.
   \EndArgList
 }

\defgcs{doublebox}%
 {\marg{\meta{text}}}%
 {\nxisty{fancybox} package}%
 {Puts a double-lined frame around its contents, prohibiting a line break in the
  contents.}%
 {%
   \BeginArgList
     \csentryargitem{text} The contents of the box.
   \EndArgList
 }

\defgcs{ovalbox}%
 {\marg{\meta{text}}}%
 {\nxisty{fancybox} package}%
 {Puts a thin-lined oval frame around its contents, prohibiting a line break in the
  contents.}%
 {%
   \BeginArgList
     \csentryargitem{text} The contents of the box.
   \EndArgList
 }

\defgcs{Ovalbox}%
 {\marg{\meta{text}}}%
 {\nxisty{fancybox} package}%
 {Puts a thick-lined oval frame around its contents, prohibiting a line break in the
  contents.}%
 {%
   \BeginArgList
     \csentryargitem{text} The contents of the box.
   \EndArgList
 }

\defgcs{protect}%
 {\meta{command}}%
 {\LaTeX\ Kernel}%
 {Used in a moving argument to prevent a fragile command from expanding.}%
 {%
   \BeginArgList
    \csentryargitem{command} The fragile command that needs protecting.
   \EndArgList
 }

\defgcs{par}%
 {}%
 {\LaTeX\ Kernel}%
 {Insert a paragraph break.}%
 {}

\defgcs{linewidth}%
 {}%
 {\LaTeX\ Kernel}%
 {A length containing the desired current line width. This is
  usually the width of the typeblock, but inside a \nxglsi{env-minipage}
  or \nxglsi{parbox} it will be the width the box. Note that
  the actual contents of the line may fall short of the line width
  (underfull hbox) or extend beyond it (overfull hbox).}%
 {}

\defgcs{textwidth}%
 {}%
 {\LaTeX\ Kernel}%
 {A length containing the width of the typeblock. Note that
  the actual contents of the line may fall short of the line width
  (underfull hbox) or extend beyond it (overfull hbox). This width
  does not include the area for marginal notes.}%
 {}

\defgcs{textheight}%
 {}%
 {\LaTeX\ Kernel}%
 {A length containing the height of the typeblock. Note that
  the actual contents of the page may fall short of the text height
  (underfull vbox) or extend beyond it (overfull vbox). This
  measurement does not include the header and footer areas.}%
 {}

\defgcs{and}%
 {}%
 {\LaTeX\ Kernel}%
 {Used to separate authors in \nxglsi{author}}%
 {}

\defgcs{clearpage}%
 {}%
 {\LaTeX\ Kernel}%
 {Inserts a page break and processes any unprocessed floats}%
 {}

\defgcs{rotatebox}%
 {\oarg{\meta{option list}}\marg{\meta{angle}}\marg{\meta{text}}}%
 {\nxisty{graphicx} package}%
 {Rotates the given contents by the given angle.}%
 {%
   \BeginArgList
     \csentryargitem{option list} A comma-separated list of options:
      \protect\begin{itemize}
       \protect\item \nxikeyvalopt{rotatebox}{origin}=\meta{label}
       \protect\item \nxikeyvalopt{rotatebox}{x}=\meta{dimen}
       \protect\item \nxikeyvalopt{rotatebox}{y}=\meta{dimen}
       \protect\item \nxikeyvalopt{rotatebox}{units}=\meta{number}
      \protect\end{itemize}

     \csentryargitem{angle} The angle of rotation.

     \csentryargitem{text} The text to be rotated.
   \EndArgList
 }

\defgcs{scalebox}%
 {\marg{\meta{h scale}}\oarg{\meta{v scale}}\marg{\meta{text}}}%
 {\nxisty{graphicx} package}%
 {Scales the specified contents.}%
 {%
    \BeginArgList
      \csentryargitem{h scale} The horizontal scale factor.
      \csentryargitem{v scale} The vertical scale factor. If omitted,
       the same as \meta{h scale}.
      \csentryargitem{text} The text to be scaled.
    \EndArgList
 }

\defgcs{reflectbox}%
 {\marg{\meta{text}}}%
 {\nxisty{graphicx} package}%
 {Reflects the specified contents in the \makeimg{y}{\ensuremath{y}}-axis.)}%
 {%
    \BeginArgList
      \csentryargitem{text} The text to be reflected.
    \EndArgList
 }

\defgcs{resizebox}%
 {\marg{\meta{h length}}\marg{\meta{v length}}\marg{\meta{text}}}%
 {\nxisty{graphicx} package}%
 {Scales the specified contents to the given dimensions.}%
 {%
   \BeginArgList
      \csentryargitem{h length} The required width or \texttt{!} to keep
      the aspect ratio.
      \csentryargitem{v length} The required height or \texttt{!} to keep
      the aspect ratio.
      \csentryargitem{text} The text to be scaled.
   \EndArgList
 }

\defgcs{centering}%
 {}%
 {\LaTeX\ Kernel}%
 {Switches the paragraph alignment to centred.}%
 {}

\defgcs{newcommand}%
 {\marg{\meta{cmd}}\oarg{\meta{n-args}}\oarg{\meta{default}}\marg{\meta{text}}}%
 {\LaTeX\ Kernel}%
 {Defines a new command.}%
 {%
   \BeginArgList
    \csentryargitem{cmd} The new command name (including initial backslash).
    \csentryargitem{n-args} The number of arguments this new command
     should have.
    \csentryargitem{default} If the first argument should be optional,
     the default value if omitted.
    \csentryargitem{text} What actions the command should perform.
   \EndArgList
 }

\defgcs{renewcommand}%
 {\marg{\meta{cmd}}\oarg{\meta{n-args}}\oarg{\meta{default}}\marg{\meta{text}}}%
 {\LaTeX\ Kernel}%
 {Redefines an existing command.}%
 {%
   \BeginArgList
    \csentryargitem{cmd} The command name (including initial backslash).
    \csentryargitem{n-args} The number of arguments this command
     should have.
    \csentryargitem{default} If the first argument should be optional,
     the default value if omitted.
    \csentryargitem{text} What actions the command should perform.
   \EndArgList
 }

\defgcs{newenvironment}%
 {\marg{\meta{env-name}}\oarg{\meta{n-args}}\oarg{\meta{default}}\marg{\meta{begin-code}}\marg{\meta{end-code}}}%
 {\LaTeX\ Kernel}%
 {Defines a new environment.}%
 {%
   \BeginArgList
    \csentryargitem{env-name} The new environment name (\emph{no} backslash).
    \csentryargitem{n-args} The number of arguments this new
     environment should have.
    \csentryargitem{default} If the first argument should be optional,
     the default value if omitted.
    \csentryargitem{begin-code} Actions to perform at the start of the
      environment.
    \csentryargitem{end-code} Actions to perform at the end of the
      environment.
   \EndArgList
 }

\defgcs{renewenvironment}%
 {\marg{\meta{env-name}}\oarg{\meta{n-args}}\oarg{\meta{default}}\marg{\meta{begin-code}}\marg{end-code}}%
 {\LaTeX\ Kernel}%
 {Redefines an existing environment.}%
 {%
   \BeginArgList
    \csentryargitem{env-name} The environment name (\emph{no} backslash).
    \csentryargitem{n-args} The number of arguments this new
     environment should have.
    \csentryargitem{default} If the first argument should be optional,
     the default value if omitted.
    \csentryargitem{begin-code} Actions to perform at the start of the
      environment.
    \csentryargitem{end-code} Actions to perform at the end of the
      environment.
   \EndArgList
 }

\defgcs{labelitemi}%
 {}%
 {Classes that define the \nxglsni{env-itemize} environment}%
 {The default label for the first level \nxglsni{env-itemize}.}%
 {}

\defgcs{labelitemii}%
 {}%
 {Classes that define the \nxglsni{env-itemize} environment}%
 {The default label for the second level \nxglsni{env-itemize}.}%
 {}

\defgcs{labelitemiii}%
 {}%
 {Classes that define the \nxglsni{env-itemize} environment}%
 {The default label for the third level \nxglsni{env-itemize}.}%
 {}

\defgcs{labelitemiv}%
 {}%
 {Classes that define the \nxglsni{env-itemize} environment}%
 {The default label for the fourth level \nxglsni{env-itemize}.}%
 {}

\defgcs{mbox}%
 {\marg{\meta{text}}}%
 {\LaTeX\ Kernel}%
 {Ensures that the given text doesn't contain a line break.}%
 {}

\defgcs{text}%
 {\marg{\meta{text}}}%
 {\nxisty{amsmath} package (Math Mode)}%
 {Displays its argument in the normal text font (as opposed to the
  current maths font).}%
 {%
   \BeginArgList
     \csentryargitem{text} The text to be displayed in the normal text font.
   \EndArgList
 }

\defgcs{intertext}%
 {\marg{\meta{text}}}%
 {\nxisty{amsmath} package (Math Mode)}%
 {Used for a short interjection in the middle of a multi-line
  displayed maths, such as in an \nxglsi{env-align} environment.
  May only appear right after \nxglsi{dbbackslashchar}.}%
 {%
   \BeginArgList
     \csentryargitem{text} The interjection text.
   \EndArgList
 }

\defgcs{overleftarrow}%
 {\marg{\meta{maths}}}%
 {\LaTeX\ Kernel (Math Mode)}%
 {Puts an extendible left arrow over \meta{maths}}%
 {%
   \BeginArgList
     \csentryargitem{maths} The maths over which to put the arrow.
   \EndArgList
 }

\defgcs{overrightarrow}%
 {\marg{\meta{maths}}}%
 {\LaTeX\ Kernel (Math Mode)}%
 {Puts an extendible right arrow over \meta{maths}}%
 {%
   \BeginArgList
     \csentryargitem{maths} The maths over which to put the arrow.
   \EndArgList
 }

\defgcs{overleftrightarrow}%
 {\marg{\meta{maths}}}%
 {\nxisty{amsmath} package (Math Mode)}%
 {Puts an extendible left-right arrow over \meta{maths}}%
 {%
   \BeginArgList
     \csentryargitem{maths} The maths over which to put the arrow.
   \EndArgList
 }

\defgcs{underleftarrow}%
 {\marg{\meta{maths}}}%
 {\nxisty{amsmath} package (Math Mode)}%
 {Puts an extendible left arrow under \meta{maths}}%
 {%
   \BeginArgList
     \csentryargitem{maths} The maths under which to put the arrow.
   \EndArgList
 }

\defgcs{underrightarrow}%
 {\marg{\meta{maths}}}%
 {\nxisty{amsmath} package (Math Mode)}%
 {Puts an extendible right arrow under \meta{maths}}%
 {%
   \BeginArgList
     \csentryargitem{maths} The maths under which to put the arrow.
   \EndArgList
 }

\defgcs{underleftrightarrow}%
 {\marg{\meta{maths}}}%
 {\nxisty{amsmath} package (Math Mode)}%
 {Puts an extendible left-right arrow under \meta{maths}}%
 {%
   \BeginArgList
     \csentryargitem{maths} The maths under which to put the arrow.
   \EndArgList
 }

\defgcs{xrightarrow}%
 {\oarg{\meta{subscript}}\marg{\meta{superscript}}}%
 {\nxisty{amsmath} package (Math Mode)}%
 {An extendible right arrow with a superscript and optionally a
  subscript.}%
 {%
   \BeginArgList
     \csentryargitem{subscript} The subscript to go under the arrow.
     \csentryargitem{superscript} The superscript to go over the arrow.
   \EndArgList
 }

\defgcs{xleftarrow}%
 {\oarg{\meta{subscript}}\marg{\meta{superscript}}}%
 {\nxisty{amsmath} package (Math Mode)}%
 {An extendible left arrow with a superscript and optionally a
  subscript.}%
 {%
   \BeginArgList
     \csentryargitem{subscript} The subscript to go under the arrow.
     \csentryargitem{superscript} The superscript to go over the arrow.
   \EndArgList
 }

\defgcs{rightarrow}%
 {}%
 {\LaTeX\ Kernel (Math Mode)}%
 {Right arrow \ensuremath{\rightarrow}.}%
 {}

\defgcs{to}%
 {}%
 {\LaTeX\ Kernel (Math Mode)}%
 {Right arrow \ensuremath{\to}.}%
 {}

\defgcs{Rightarrow}%
 {}%
 {\LaTeX\ Kernel (Math Mode)}%
 {Double-lined right arrow \ensuremath{\Rightarrow}.}%
 {}

\defgcs{leftarrow}%
 {}%
 {\LaTeX\ Kernel (Math Mode)}%
 {Left arrow \ensuremath{\leftarrow}.}%
 {}

\defgcs{gets}%
 {}%
 {\LaTeX\ Kernel (Math Mode)}%
 {Left arrow \ensuremath{\gets}.}%
 {}

\defgcs{Leftarrow}%
 {}%
 {\LaTeX\ Kernel (Math Mode)}%
 {Double-lined left arrow \ensuremath{\Leftarrow}.}%
 {}

\defgcs{downarrow}%
 {}%
 {\LaTeX\ Kernel (Math Mode)}%
 {Down arrow \ensuremath{\downarrow}. (May be used as a delimiter.)}%
 {}

\defgcs{Downarrow}%
 {}%
 {\LaTeX\ Kernel (Math Mode)}%
 {Double-lined down arrow \ensuremath{\Downarrow}. (May be used as a delimiter.)}%
 {}

\defgcs{uparrow}%
 {}%
 {\LaTeX\ Kernel (Math Mode)}%
 {Up arrow \ensuremath{\uparrow}. (May be used as a delimiter.)}%
 {}

\defgcs{Uparrow}%
 {}%
 {\LaTeX\ Kernel (Math Mode)}%
 {Double-lined up arrow \ensuremath{\Uparrow}. (May be used as a delimiter.)}%
 {}

\defgcs{updownarrow}%
 {}%
 {\LaTeX\ Kernel (Math Mode)}%
 {Double-ended vertical arrow \ensuremath{\updownarrow}. (May be used as a delimiter.)}%
 {}

\defgcs{Updownarrow}%
 {}%
 {\LaTeX\ Kernel (Math Mode)}%
 {Double-ended double-lined vertical arrow \ensuremath{\Updownarrow}. (May be used as a delimiter.)}%
 {}

\defgcs{hookrightarrow}%
 {}%
 {\LaTeX\ Kernel (Math Mode)}%
 {Hooked right arrow \ensuremath{\hookrightarrow}.}%
 {}

\defgcs{hookleftarrow}%
 {}%
 {\LaTeX\ Kernel (Math Mode)}%
 {Hooked left arrow \ensuremath{\hookleftarrow}.}%
 {}

\defgcs{leftharpoondown}%
 {}%
 {\LaTeX\ Kernel (Math Mode)}%
 {Left down harpoon \ensuremath{\leftharpoondown}.}%
 {}

\defgcs{leftharpoonup}%
 {}%
 {\LaTeX\ Kernel (Math Mode)}%
 {Left up harpoon \ensuremath{\leftharpoonup}.}%
 {}

\defgcs{rightharpoondown}%
 {}%
 {\LaTeX\ Kernel (Math Mode)}%
 {Right down harpoon \ensuremath{\rightharpoondown}.}%
 {}

\defgcs{rightharpoonup}%
 {}%
 {\LaTeX\ Kernel (Math Mode)}%
 {Right up harpoon \ensuremath{\rightharpoonup}.}%
 {}

\defgcs{rightleftharpoons}%
 {}%
 {\LaTeX\ Kernel (Math Mode)}%
 {Right-left harpoons \ensuremath{\rightleftharpoons}.}%
 {}

\defgcs{leftrightarrow}%
 {}%
 {\LaTeX\ Kernel (Math Mode)}%
 {Double-ended horizontal arrow \ensuremath{\leftrightarrow}.}%
 {}

\defgcs{Leftrightarrow}%
 {}%
 {\LaTeX\ Kernel (Math Mode)}%
 {Double-ended double-lined horizontal arrow \ensuremath{\Leftrightarrow}.}%
 {}

\defgcs{longrightarrow}%
 {}%
 {\LaTeX\ Kernel (Math Mode)}%
 {Long right arrow \ensuremath{\longrightarrow}.}%
 {}

\defgcs{longleftarrow}%
 {}%
 {\LaTeX\ Kernel (Math Mode)}%
 {Long left arrow \ensuremath{\longleftarrow}.}%
 {}

\defgcs{longleftrightarrow}%
 {}%
 {\LaTeX\ Kernel (Math Mode)}%
 {Long double-ended horizontal arrow \ensuremath{\longleftrightarrow}.}%
 {}

\defgcs{Longrightarrow}%
 {}%
 {\LaTeX\ Kernel (Math Mode)}%
 {Long double-lined right arrow \ensuremath{\Longrightarrow}.}%
 {}

\defgcs{Longleftarrow}%
 {}%
 {\LaTeX\ Kernel (Math Mode)}%
 {Long double-lined left arrow \ensuremath{\Longleftarrow}.}%
 {}

\defgcs{Longleftrightarrow}%
 {}%
 {\LaTeX\ Kernel (Math Mode)}%
 {Long double-lined double-ended horizontal arrow \ensuremath{\Longleftrightarrow}.}%
 {}

\defgcs{longmapsto}%
 {}%
 {\LaTeX\ Kernel (Math Mode)}%
 {Long mapping arrow \ensuremath{\longmapsto}.}%
 {}

\defgcs{mapsto}%
 {}%
 {\LaTeX\ Kernel (Math Mode)}%
 {Mapping arrow \ensuremath{\mapsto}.}%
 {}

\defgcs{nearrow}%
 {}%
 {\LaTeX\ Kernel (Math Mode)}%
 {North-East arrow \ensuremath{\nearrow}.}%
 {}

\defgcs{nwarrow}%
 {}%
 {\LaTeX\ Kernel (Math Mode)}%
 {North-West arrow \ensuremath{\nwarrow}.}%
 {}

\defgcs{searrow}%
 {}%
 {\LaTeX\ Kernel (Math Mode)}%
 {South-East arrow \ensuremath{\searrow}.}%
 {}

\defgcs{swarrow}%
 {}%
 {\LaTeX\ Kernel (Math Mode)}%
 {South-West arrow \ensuremath{\swarrow}.}%
 {}

\defgcs{makeindex}%
 {}%
 {\LaTeX\ Kernel (Preamble Only)}%
 {Enables \nxglsi{index}.}%
 {}

\defgcs{index}%
 {\marg{\meta{text}}}%
 {\LaTeX\ Kernel}%
 {Adds indexing information to an external index file. The command
   \nxglsi{makeindex} must be used in the preamble to enable
   this command. The external index file must be post-processed with
   an indexing application, such as \nxiappname{makeindex}.}%
 {%
   \BeginArgList
     \csentryargitem{text} The text to go in the index.
   \EndArgList
 }

\defgcs{printindex}%
 {}%
 {\nxisty{makeidx} package}%
 {Prints the index. Must be used with \nxglsi{makeindex} and \nxglsi{index}.
 (The external index file must first be processed by an indexing application.)}%
 {}

\defgcs{alpha}%
 {}%
 {\LaTeX\ Kernel (Math Mode)}%
 {Greek lower case alpha \ensuremath{\alpha}.}%
 {}

\defgcs{beta}%
 {}%
 {\LaTeX\ Kernel (Math Mode)}%
 {Greek lower case beta \ensuremath{\beta}.}%
 {}

\defgcs{gamma}%
 {}%
 {\LaTeX\ Kernel (Math Mode)}%
 {Greek lower case gamma \ensuremath{\gamma}.}%
 {}

\defgcs{delta}%
 {}%
 {\LaTeX\ Kernel (Math Mode)}%
 {Greek lower case delta \ensuremath{\delta}.}%
 {}

\defgcs{epsilon}%
 {}%
 {\LaTeX\ Kernel (Math Mode)}%
 {Greek lower case epsilon \ensuremath{\epsilon}.}%
 {}

\defgcs{varepsilon}%
 {}%
 {\LaTeX\ Kernel (Math Mode)}%
 {Variant Greek lower case alpha \ensuremath{\varepsilon}.}%
 {}

\defgcs{zeta}%
 {}%
 {\LaTeX\ Kernel (Math Mode)}%
 {Greek lower case zeta \ensuremath{\zeta}.}%
 {}

\defgcs{eta}%
 {}%
 {\LaTeX\ Kernel (Math Mode)}%
 {Greek lower case eta \ensuremath{\eta}.}%
 {}

\defgcs{theta}%
 {}%
 {\LaTeX\ Kernel (Math Mode)}%
 {Greek lower case theta \ensuremath{\theta}.}%
 {}

\defgcs{vartheta}%
 {}%
 {\LaTeX\ Kernel (Math Mode)}%
 {A variant Greek lower case theta \ensuremath{\vartheta}.}%
 {}

\defgcs{iota}%
 {}%
 {\LaTeX\ Kernel (Math Mode)}%
 {Greek lower case iota \ensuremath{\iota}.}%
 {}

\defgcs{kappa}%
 {}%
 {\LaTeX\ Kernel (Math Mode)}%
 {Greek lower case kappa \ensuremath{\kappa}.}%
 {}

\defgcs{lambda}%
 {}%
 {\LaTeX\ Kernel (Math Mode)}%
 {Greek lower case lambda \ensuremath{\lambda}.}%
 {}

\defgcs{mu}%
 {}%
 {\LaTeX\ Kernel (Math Mode)}%
 {Greek lower case mu \ensuremath{\mu}.}%
 {}

\defgcs{nu}%
 {}%
 {\LaTeX\ Kernel (Math Mode)}%
 {Greek lower case nu \ensuremath{\nu}.}%
 {}

\defgcs{xi}%
 {}%
 {\LaTeX\ Kernel (Math Mode)}%
 {Greek lower case xi \ensuremath{\xi}.}%
 {}

\defgcs{pi}%
 {}%
 {\LaTeX\ Kernel (Math Mode)}%
 {Greek lower case pi \ensuremath{\pi}.}%
 {}

\defgcs{varpi}%
 {}%
 {\LaTeX\ Kernel (Math Mode)}%
 {Variant Greek lower case pi \ensuremath{\varpi}.}%
 {}

\defgcs{rho}%
 {}%
 {\LaTeX\ Kernel (Math Mode)}%
 {Greek lower case rho \ensuremath{\rho}.}%
 {}

\defgcs{varrho}%
 {}%
 {\LaTeX\ Kernel (Math Mode)}%
 {Variant Greek lower case rho \ensuremath{\varrho}.}%
 {}

\defgcs{sigma}%
 {}%
 {\LaTeX\ Kernel (Math Mode)}%
 {Greek lower case sigma \ensuremath{\sigma}.}%
 {}

\defgcs{varsigma}%
 {}%
 {\LaTeX\ Kernel (Math Mode)}%
 {Variant Greek lower case sigma \ensuremath{\varsigma}.}%
 {}

\defgcs{tau}%
 {}%
 {\LaTeX\ Kernel (Math Mode)}%
 {Greek lower case tau \ensuremath{\tau}.}%
 {}

\defgcs{upsilon}%
 {}%
 {\LaTeX\ Kernel (Math Mode)}%
 {Greek lower case upsilon \ensuremath{\upsilon}.}%
 {}

\defgcs{phi}%
 {}%
 {\LaTeX\ Kernel (Math Mode)}%
 {Greek lower case phi \ensuremath{\phi}.}%
 {}

\defgcs{varphi}%
 {}%
 {\LaTeX\ Kernel (Math Mode)}%
 {Variant Greek lower case phi \ensuremath{\varphi}.}%
 {}

\defgcs{chi}%
 {}%
 {\LaTeX\ Kernel (Math Mode)}%
 {Greek lower case chi \ensuremath{\chi}.}%
 {}

\defgcs{psi}%
 {}%
 {\LaTeX\ Kernel (Math Mode)}%
 {Greek lower case psi \ensuremath{\psi}.}%
 {}

\defgcs{omega}%
 {}%
 {\LaTeX\ Kernel (Math Mode)}%
 {Greek lower case omega \ensuremath{\omega}.}%
 {}

\defgcs{Gamma}%
 {}%
 {\LaTeX\ Kernel (Math Mode)}%
 {Greek upper case gamma \ensuremath{\Gamma}.}%
 {}

\defgcs{Delta}%
 {}%
 {\LaTeX\ Kernel (Math Mode)}%
 {Greek upper case delta \ensuremath{\Delta}.}%
 {}

\defgcs{Theta}%
 {}%
 {\LaTeX\ Kernel (Math Mode)}%
 {Greek upper case theta \ensuremath{\Theta}.}%
 {}

\defgcs{Lambda}%
 {}%
 {\LaTeX\ Kernel (Math Mode)}%
 {Greek upper case lambda \ensuremath{\Lambda}.}%
 {}

\defgcs{Xi}%
 {}%
 {\LaTeX\ Kernel (Math Mode)}%
 {Greek upper case xi \ensuremath{\Xi}.}%
 {}

\defgcs{Pi}%
 {}%
 {\LaTeX\ Kernel (Math Mode)}%
 {Greek upper case pi \ensuremath{\Pi}.}%
 {}

\defgcs{Sigma}%
 {}%
 {\LaTeX\ Kernel (Math Mode)}%
 {Greek upper case sigma \ensuremath{\Sigma}.}%
 {}

\defgcs{Upsilon}%
 {}%
 {\LaTeX\ Kernel (Math Mode)}%
 {Greek upper case upsilon \ensuremath{\Upsilon}.}%
 {}

\defgcs{Phi}%
 {}%
 {\LaTeX\ Kernel (Math Mode)}%
 {Greek upper case phi \ensuremath{\Phi}.}%
 {}

\defgcs{Psi}%
 {}%
 {\LaTeX\ Kernel (Math Mode)}%
 {Greek upper case psi \ensuremath{\Psi}.}%
 {}

\defgcs{Omega}%
 {}%
 {\LaTeX\ Kernel (Math Mode)}%
 {Greek upper case omega \ensuremath{\Omega}.}%
 {}

\defgcs{sb}%
 {\marg{\meta{maths}}}%
 {\LaTeX\ Kernel (Math Mode)}%
 {Displays its argument as a subscript.}%
 {%
   \BeginArgList
     \csentryargitem{maths} The subscript.
   \EndArgList
 }

\defgcs{sp}%
 {\marg{\meta{maths}}}%
 {\LaTeX\ Kernel (Math Mode)}%
 {Displays its argument as a superscript.}%
 {%
   \BeginArgList
     \csentryargitem{maths} The superscript.
   \EndArgList
 }

\defgcs{pmod}%
 {\marg{\meta{maths}}}%
 {\LaTeX\ Kernel (Math Mode)}%
 {Modulo operator with parentheses.}%
 {%
   \BeginArgList
     \csentryargitem{maths} The modulo (placed in parentheses).
   \EndArgList
 }

\defgcs{mod}%
 {\marg{\meta{maths}}}%
 {\nxisty{amsmath} (Math Mode)}%
 {Modulo operator without parentheses.}%
 {%
   \BeginArgList
     \csentryargitem{maths} The modulo (placed in parentheses).
   \EndArgList
 }

\defgcs{pod}%
 {\marg{\meta{maths}}}%
 {\nxisty{amsmath} (Math Mode)}%
 {Modulo operator with parentheses but no \dq{mod}.}%
 {%
   \BeginArgList
     \csentryargitem{maths} The modulo (placed in parentheses).
   \EndArgList
 }

\defgcs{bmod}%
 {}%
 {\LaTeX\ Kernel (Math Mode)}%
 {Modulo operator.}%
 {}

\defgcs{arccos}%
 {}%
 {\LaTeX\ Kernel (Math Mode)}%
 {Typesets \ensuremath{\arccos} function name.}%
 {}

\defgcs{arcsin}%
 {}%
 {\LaTeX\ Kernel (Math Mode)}%
 {Typesets \ensuremath{\arcsin} function name.}%
 {}

\defgcs{arctan}%
 {}%
 {\LaTeX\ Kernel (Math Mode)}%
 {Typesets \ensuremath{\arctan} function name.}%
 {}

\defgcs{arg}%
 {}%
 {\LaTeX\ Kernel (Math Mode)}%
 {Typesets \ensuremath{\arg} function name.}%
 {}

\defgcs{cos}%
 {}%
 {\LaTeX\ Kernel (Math Mode)}%
 {Typesets \ensuremath{\cos} function name.}%
 {}

\defgcs{cosh}%
 {}%
 {\LaTeX\ Kernel (Math Mode)}%
 {Typesets \ensuremath{\cosh} function name.}%
 {}

\defgcs{cot}%
 {}%
 {\LaTeX\ Kernel (Math Mode)}%
 {Typesets \ensuremath{\cot} function name.}%
 {}

\defgcs{coth}%
 {}%
 {\LaTeX\ Kernel (Math Mode)}%
 {Typesets \ensuremath{\coth} function name.}%
 {}

\defgcs{csc}%
 {}%
 {\LaTeX\ Kernel (Math Mode)}%
 {Typesets \ensuremath{\csc} function name.}%
 {}

\defgcs{deg}%
 {}%
 {\LaTeX\ Kernel (Math Mode)}%
 {Typesets \ensuremath{\deg} function name.}%
 {}

\defgcs{det}%
 {}%
 {\LaTeX\ Kernel (Math Mode)}%
 {Typesets \ensuremath{\det} function name (may have limits via \nxglsi{underscorechar} or \nxglsi{circumchar}).}%
 {}

\defgcs{dim}%
 {}%
 {\LaTeX\ Kernel (Math Mode)}%
 {Typesets \ensuremath{\dim} function name.}%
 {}

\defgcs{exp}%
 {}%
 {\LaTeX\ Kernel (Math Mode)}%
 {Typesets \ensuremath{\exp} function name.}%
 {}

\defgcs{gcd}%
 {}%
 {\LaTeX\ Kernel (Math Mode)}%
 {Typesets \ensuremath{\gcd} function name (may have limits via \nxglsi{underscorechar} or \nxglsi{circumchar}).}%
 {}

\defgcs{hom}%
 {}%
 {\LaTeX\ Kernel (Math Mode)}%
 {Typesets \ensuremath{\hom} function name.}%
 {}

\defgcs{inf}%
 {}%
 {\LaTeX\ Kernel (Math Mode)}%
 {Typesets \ensuremath{\inf} function name (may have limits via \nxglsi{underscorechar} or \nxglsi{circumchar}).}%
 {}

\defgcs{injlim}%
 {}%
 {\nxisty{amsmath} (Math Mode)}%
 {Typesets \ensuremath{\injlim} function name (may have limits via \nxglsi{underscorechar} or \nxglsi{circumchar}).}%
 {}

\defgcs{projlim}%
 {}%
 {\nxisty{amsmath} (Math Mode)}%
 {Typesets \ensuremath{\projlim} function name (may have limits via \nxglsi{underscorechar} or \nxglsi{circumchar}).}%
 {}

\defgcs{varlimsup}%
 {}%
 {\nxisty{amsmath} (Math Mode)}%
 {Typesets \ensuremath{\varlimsup} function name (may have limits via \nxglsi{underscorechar} or \nxglsi{circumchar}).}%
 {}

\defgcs{varliminf}%
 {}%
 {\nxisty{amsmath} (Math Mode)}%
 {Typesets \ensuremath{\varliminf} function name (may have limits via \nxglsi{underscorechar} or \nxglsi{circumchar}).}%
 {}

\defgcs{varinjlim}%
 {}%
 {\nxisty{amsmath} (Math Mode)}%
 {Typesets \ensuremath{\varinjlim} function name (may have limits via \nxglsi{underscorechar} or \nxglsi{circumchar}).}%
 {}

\defgcs{varprojlim}%
 {}%
 {\nxisty{amsmath} (Math Mode)}%
 {Typesets \ensuremath{\varprojlim} function name (may have limits via \nxglsi{underscorechar} or \nxglsi{circumchar}).}%
 {}

\defgcs{ker}%
 {}%
 {\LaTeX\ Kernel (Math Mode)}%
 {Typesets \ensuremath{\ker} function name.}%
 {}

\defgcs{lg}%
 {}%
 {\LaTeX\ Kernel (Math Mode)}%
 {Typesets \ensuremath{\lg} function name.}%
 {}

\defgcs{lim}%
 {}%
 {\LaTeX\ Kernel (Math Mode)}%
 {Typesets \ensuremath{\lim} function name (may have limits via \nxglsi{underscorechar} or \nxglsi{circumchar}).}%
 {}

\defgcs{liminf}%
 {}%
 {\LaTeX\ Kernel (Math Mode)}%
 {Typesets \ensuremath{\liminf} function name (may have limits via \nxglsi{underscorechar} or \nxglsi{circumchar}).}%
 {}

\defgcs{limsup}%
 {}%
 {\LaTeX\ Kernel (Math Mode)}%
 {Typesets \ensuremath{\limsup} function name (may have limits via \nxglsi{underscorechar} or \nxglsi{circumchar}).}%
 {}

\defgcs{ln}%
 {}%
 {\LaTeX\ Kernel (Math Mode)}%
 {Typesets \ensuremath{\ln} function name.}%
 {}

\defgcs{log}%
 {}%
 {\LaTeX\ Kernel (Math Mode)}%
 {Typesets \ensuremath{\log} function name.}%
 {}

\defgcs{max}%
 {}%
 {\LaTeX\ Kernel (Math Mode)}%
 {Typesets \ensuremath{\max} function name (may have limits via \nxglsi{underscorechar} or \nxglsi{circumchar}).}%
 {}

\defgcs{min}%
 {}%
 {\LaTeX\ Kernel (Math Mode)}%
 {Typesets \ensuremath{\min} function name (may have limits via \nxglsi{underscorechar} or \nxglsi{circumchar}).}%
 {}

\defgcs{Pr}%
 {}%
 {\LaTeX\ Kernel (Math Mode)}%
 {Typesets \ensuremath{\Pr} function name (may have limits via \nxglsi{underscorechar} or \nxglsi{circumchar}).}%
 {}

\defgcs{sec}%
 {}%
 {\LaTeX\ Kernel (Math Mode)}%
 {Typesets \ensuremath{\sec} function name.}%
 {}

\defgcs{sin}%
 {}%
 {\LaTeX\ Kernel (Math Mode)}%
 {Typesets \ensuremath{\sin} function name.}%
 {}

\defgcs{sinh}%
 {}%
 {\LaTeX\ Kernel (Math Mode)}%
 {Typesets \ensuremath{\sinh} function name.}%
 {}

\defgcs{sup}%
 {}%
 {\LaTeX\ Kernel (Math Mode)}%
 {Typesets \ensuremath{\sup} function name (may have limits via \nxglsi{underscorechar} or \nxglsi{circumchar}).}%
 {}

\defgcs{tan}%
 {}%
 {\LaTeX\ Kernel (Math Mode)}%
 {Typesets \ensuremath{\tan} function name.}%
 {}

\defgcs{tanh}%
 {}%
 {\LaTeX\ Kernel (Math Mode)}%
 {Typesets \ensuremath{\tanh} function name.}%
 {}

\defgcs{infty}%
 {}%
 {\LaTeX\ Kernel (Math Mode)}%
 {Infinity \ensuremath{\infty} symbol.}%
 {}

\defgcs{partial}%
 {}%
 {\LaTeX\ Kernel (Math Mode)}%
 {Partial \ensuremath{\partial} symbol.}%
 {}

\defgcs{subset}%
 {}%
 {\LaTeX\ Kernel (Math Mode)}%
 {Subset \ensuremath{\subset} symbol.}%
 {}

\defgcs{approx}%
 {}%
 {\LaTeX\ Kernel (Math Mode)}%
 {Relational \ensuremath{\approx} symbol.}%
 {}

\defgcs{asymp}%
 {}%
 {\LaTeX\ Kernel (Math Mode)}%
 {Relational \ensuremath{\asymp} symbol.}%
 {}

\defgcs{bowtie}%
 {}%
 {\LaTeX\ Kernel (Math Mode)}%
 {Relational \ensuremath{\bowtie} symbol.}%
 {}

\defgcs{cong}%
 {}%
 {\LaTeX\ Kernel (Math Mode)}%
 {Relational \ensuremath{\cong} symbol.}%
 {}

\defgcs{dashv}%
 {}%
 {\LaTeX\ Kernel (Math Mode)}%
 {Relational \ensuremath{\dashv} symbol.}%
 {}

\defgcs{doteq}%
 {}%
 {\LaTeX\ Kernel (Math Mode)}%
 {Relational \makeimg{equals symbol with dot on top}{\ensuremath{\doteq}} symbol.}%
 {}

\defgcs{equiv}%
 {}%
 {\LaTeX\ Kernel (Math Mode)}%
 {Relational \makeimg{equivalent}{\ensuremath{\equiv}} symbol.}%
 {}

\defgcs{frown}%
 {}%
 {\LaTeX\ Kernel (Math Mode)}%
 {Relational \ensuremath{\frown} symbol.}%
 {}

\defgcs{ge}%
 {}%
 {\LaTeX\ Kernel (Math Mode)}%
 {Relational \makeimg{greater than or equal to}{\ensuremath{\ge}} symbol.}%
 {}

\defgcs{geq}%
 {}%
 {\LaTeX\ Kernel (Math Mode)}%
 {Relational \makeimg{greater than or equal to}{\ensuremath{\geq}} symbol.}%
 {}

\defgcs{gg}%
 {}%
 {\LaTeX\ Kernel (Math Mode)}%
 {Relational \makeimg{much greater than}{\ensuremath{\gg}} symbol.}%
 {}

\defgcs{in}%
 {}%
 {\LaTeX\ Kernel (Math Mode)}%
 {Relational \ensuremath{\in} symbol.}%
 {}

\defgcs{le}%
 {}%
 {\LaTeX\ Kernel (Math Mode)}%
 {Relational \makeimg{less than or equal to}{\ensuremath{\le}} symbol.}%
 {}

\defgcs{leq}%
 {}%
 {\LaTeX\ Kernel (Math Mode)}%
 {Relational \makeimg{less than or equal to}{\ensuremath{\leq}} symbol.}%
 {}

\defgcs{ll}%
 {}%
 {\LaTeX\ Kernel (Math Mode)}%
 {Relational \makeimg{much less than}{\ensuremath{\ll}} symbol.}%
 {}

\defgcs{mid}%
 {}%
 {\LaTeX\ Kernel (Math Mode)}%
 {Relational \ensuremath{\mid} symbol.}%
 {}

\defgcs{models}%
 {}%
 {\LaTeX\ Kernel (Math Mode)}%
 {Relational \ensuremath{\models} symbol.}%
 {}

\defgcs{neq}%
 {}%
 {\LaTeX\ Kernel (Math Mode)}%
 {Relational \makeimg{not equal to}{\ensuremath{\neq}} symbol.}%
 {}

\defgcs{ni}%
 {}%
 {\LaTeX\ Kernel (Math Mode)}%
 {Relational \makeimg{contains element}{\ensuremath{\ni}} symbol.}%
 {}

\defgcs{notin}%
 {}%
 {\LaTeX\ Kernel (Math Mode)}%
 {Relational \makeimg{not element of}{\ensuremath{\notin}} symbol.}%
 {}

\defgcs{parallel}%
 {}%
 {\LaTeX\ Kernel (Math Mode)}%
 {Relational \ensuremath{\parallel} symbol.}%
 {}

\defgcs{prec}%
 {}%
 {\LaTeX\ Kernel (Math Mode)}%
 {Relational \ensuremath{\prec} symbol.}%
 {}

\defgcs{preceq}%
 {}%
 {\LaTeX\ Kernel (Math Mode)}%
 {Relational \ensuremath{\preceq} symbol.}%
 {}

\defgcs{perp}%
 {}%
 {\LaTeX\ Kernel (Math Mode)}%
 {Relational \makeimg{perpendicular to}{\ensuremath{\perp}} symbol.}%
 {}

\defgcs{propto}%
 {}%
 {\LaTeX\ Kernel (Math Mode)}%
 {Relational \makeimg{proportional to}{\ensuremath{\propto}} symbol.}%
 {}

\defgcs{sim}%
 {}%
 {\LaTeX\ Kernel (Math Mode)}%
 {Relational \ensuremath{\sim} symbol.}%
 {}

\defgcs{simeq}%
 {}%
 {\LaTeX\ Kernel (Math Mode)}%
 {Relational \ensuremath{\simeq} symbol.}%
 {}

\defgcs{smile}%
 {}%
 {\LaTeX\ Kernel (Math Mode)}%
 {Relational \ensuremath{\smile} symbol.}%
 {}

\defgcs{sqsubseteq}%
 {}%
 {\LaTeX\ Kernel (Math Mode)}%
 {Relational \ensuremath{\sqsubseteq} symbol.}%
 {}

\defgcs{sqsupseteq}%
 {}%
 {\LaTeX\ Kernel (Math Mode)}%
 {Relational \ensuremath{\sqsupseteq} symbol.}%
 {}

\defgcs{subseteq}%
 {}%
 {\LaTeX\ Kernel (Math Mode)}%
 {Relational \ensuremath{\subseteq} symbol.}%
 {}

\defgcs{succ}%
 {}%
 {\LaTeX\ Kernel (Math Mode)}%
 {Relational \ensuremath{\succ} symbol.}%
 {}

\defgcs{succeq}%
 {}%
 {\LaTeX\ Kernel (Math Mode)}%
 {Relational \ensuremath{\succeq} symbol.}%
 {}

\defgcs{supset}%
 {}%
 {\LaTeX\ Kernel (Math Mode)}%
 {Relational \ensuremath{\supset} symbol.}%
 {}

\defgcs{supseteq}%
 {}%
 {\LaTeX\ Kernel (Math Mode)}%
 {Relational \ensuremath{\supseteq} symbol.}%
 {}

\defgcs{vdash}%
 {}%
 {\LaTeX\ Kernel (Math Mode)}%
 {Relational \ensuremath{\vdash} symbol.}%
 {}

\defgcs{amalg}%
 {}%
 {\LaTeX\ Kernel (Math Mode)}%
 {Binary operator \ensuremath{\amalg} symbol.}%
 {}

\defgcs{ast}%
 {}%
 {\LaTeX\ Kernel (Math Mode)}%
 {Binary operator \makeimg{asterisk}{\ensuremath{\ast}} symbol.}%
 {}

\defgcs{bullet}%
 {}%
 {\LaTeX\ Kernel (Math Mode)}%
 {Binary operator \makeimg{bullet point}{\ensuremath{\bullet}} symbol.}%
 {}

\defgcs{bigcirc}%
 {}%
 {\LaTeX\ Kernel (Math Mode)}%
 {Binary operator \ensuremath{\bigcirc} symbol.}%
 {}

\defgcs{bigtriangledown}%
 {}%
 {\LaTeX\ Kernel (Math Mode)}%
 {Binary operator \ensuremath{\bigtriangledown} symbol.}%
 {}

\defgcs{bigtriangleup}%
 {}%
 {\LaTeX\ Kernel (Math Mode)}%
 {Binary operator \ensuremath{\bigtriangleup} symbol.}%
 {}

\defgcs{cap}%
 {}%
 {\LaTeX\ Kernel (Math Mode)}%
 {Binary operator \ensuremath{\cap} symbol.}%
 {}

\defgcs{cdot}%
 {}%
 {\LaTeX\ Kernel (Math Mode)}%
 {Centred dot \ensuremath{\cdot} symbol.}%
 {}

\defgcs{circ}%
 {}%
 {\LaTeX\ Kernel (Math Mode)}%
 {Circle \ensuremath{\circ} symbol.}%
 {}

\defgcs{cup}%
 {}%
 {\LaTeX\ Kernel (Math Mode)}%
 {Operator \ensuremath{\cup} symbol.}%
 {}

\defgcs{dagger}%
 {}%
 {\LaTeX\ Kernel (Math Mode)}%
 {Binary operator \ensuremath{\dagger} symbol.}%
 {}

\defgcs{ddagger}%
 {}%
 {\LaTeX\ Kernel (Math Mode)}%
 {Binary operator \ensuremath{\ddagger} symbol.}%
 {}

\defgcs{diamond}%
 {}%
 {\LaTeX\ Kernel (Math Mode)}%
 {Binary operator \ensuremath{\diamond} symbol.}%
 {}

\defgcs{div}%
 {}%
 {\LaTeX\ Kernel (Math Mode)}%
 {Division operator \ensuremath{\div} symbol.}%
 {}

\defgcs{mp}%
 {}%
 {\LaTeX\ Kernel (Math Mode)}%
 {Minus or plus operator \makeimg{minus or plus}{\ensuremath{\mp}} symbol.}%
 {}

\defgcs{odot}%
 {}%
 {\LaTeX\ Kernel (Math Mode)}%
 {Operator \ensuremath{\odot} symbol.}%
 {}

\defgcs{ominus}%
 {}%
 {\LaTeX\ Kernel (Math Mode)}%
 {Operator \ensuremath{\ominus} symbol.}%
 {}

\defgcs{oplus}%
 {}%
 {\LaTeX\ Kernel (Math Mode)}%
 {Operator \ensuremath{\oplus} symbol.}%
 {}

\defgcs{oslash}%
 {}%
 {\LaTeX\ Kernel (Math Mode)}%
 {Operator \ensuremath{\oslash} symbol.}%
 {}

\defgcs{otimes}%
 {}%
 {\LaTeX\ Kernel (Math Mode)}%
 {Operator \ensuremath{\otimes} symbol.}%
 {}

\defgcs{pm}%
 {}%
 {\LaTeX\ Kernel (Math Mode)}%
 {Operator \makeimg{plus or minus}{\ensuremath{\pm}} symbol.}%
 {}

\defgcs{setminus}%
 {}%
 {\LaTeX\ Kernel (Math Mode)}%
 {Operator \ensuremath{\setminus} symbol.}%
 {}

\defgcs{sqcap}%
 {}%
 {\LaTeX\ Kernel (Math Mode)}%
 {Operator \ensuremath{\sqcap} symbol.}%
 {}

\defgcs{sqcup}%
 {}%
 {\LaTeX\ Kernel (Math Mode)}%
 {Operator \ensuremath{\sqcup} symbol.}%
 {}

\defgcs{star}%
 {}%
 {\LaTeX\ Kernel (Math Mode)}%
 {Operator \ensuremath{\star} symbol.}%
 {}

\defgcs{times}%
 {}%
 {\LaTeX\ Kernel (Math Mode)}%
 {Operator \ensuremath{\times} symbol.}%
 {}

\defgcs{triangleleft}%
 {}%
 {\LaTeX\ Kernel (Math Mode)}%
 {Binary operator \ensuremath{\triangleleft} symbol.}%
 {}

\defgcs{triangleright}%
 {}%
 {\LaTeX\ Kernel (Math Mode)}%
 {Binary operator \ensuremath{\triangleright} symbol.}%
 {}

\defgcs{uplus}%
 {}%
 {\LaTeX\ Kernel (Math Mode)}%
 {Operator \ensuremath{\uplus} symbol.}%
 {}

\defgcs{vee}%
 {}%
 {\LaTeX\ Kernel (Math Mode)}%
 {Operator \ensuremath{\vee} symbol.}%
 {}

\defgcs{wedge}%
 {}%
 {\LaTeX\ Kernel (Math Mode)}%
 {Operator \ensuremath{\wedge} symbol.}%
 {}

\defgcs{wr}%
 {}%
 {\LaTeX\ Kernel (Math Mode)}%
 {Operator \ensuremath{\wr} symbol.}%
 {}

\defgcs{sum}%
 {}%
 {\LaTeX\ Kernel (Math Mode)}%
 {Summation \ensuremath{\sum} symbol (may take limits).}%
 {}

\defgcs{int}%
 {}%
 {\LaTeX\ Kernel (Math Mode)}%
 {Integral \ensuremath{\int} symbol (may take limits).}%
 {}

\defgcs{oint}%
 {}%
 {\LaTeX\ Kernel (Math Mode)}%
 {Closed path integral \ensuremath{\oint} symbol (may take limits).}%
 {}

\defgcs{prod}%
 {}%
 {\LaTeX\ Kernel (Math Mode)}%
 {Product \ensuremath{\prod} symbol (may take limits).}%
 {}

\defgcs{coprod}%
 {}%
 {\LaTeX\ Kernel (Math Mode)}%
 {Co-product \ensuremath{\coprod} symbol (may take limits).}%
 {}

\defgcs{bigcap}%
 {}%
 {\LaTeX\ Kernel (Math Mode)}%
 {Collection intersection \ensuremath{\bigcap} symbol (may take limits).}%
 {}

\defgcs{bigcup}%
 {}%
 {\LaTeX\ Kernel (Math Mode)}%
 {Collection union \ensuremath{\bigcup} symbol (may take limits).}%
 {}

\defgcs{bigsqcup}%
 {}%
 {\LaTeX\ Kernel (Math Mode)}%
 {Big operator \ensuremath{\bigsqcup} (may take limits).}%
 {}

\defgcs{bigvee}%
 {}%
 {\LaTeX\ Kernel (Math Mode)}%
 {Big operator \ensuremath{\bigvee} (may take limits).}%
 {}

\defgcs{bigwedge}%
 {}%
 {\LaTeX\ Kernel (Math Mode)}%
 {Big operator \ensuremath{\bigwedge} (may take limits).}%
 {}

\defgcs{bigodot}%
 {}%
 {\LaTeX\ Kernel (Math Mode)}%
 {Big operator \ensuremath{\bigodot} (may take limits).}%
 {}

\defgcs{bigotimes}%
 {}%
 {\LaTeX\ Kernel (Math Mode)}%
 {Big operator \ensuremath{\bigotimes} (may take limits).}%
 {}

\defgcs{bigoplus}%
 {}%
 {\LaTeX\ Kernel (Math Mode)}%
 {Big operator \ensuremath{\bigoplus} (may take limits).}%
 {}

\defgcs{biguplus}%
 {}%
 {\LaTeX\ Kernel (Math Mode)}%
 {Big operator \ensuremath{\biguplus} (may take limits).}%
 {}

\defgcs{not}%
 {\meta{symbol command}}%
 {\LaTeX\ Kernel (Math Mode)}%
 {Negates the following symbol. Example: \cmdname{not}\nxglsi{subset} produces
   \ensuremath{\not\subset}.}%
 {%
   \BeginArgList
    \csentryargitem{symbol command} The symbol to be negated.
   \EndArgList
 }

\defgcs{ldots}%
 {}%
 {\LaTeX\ Kernel}%
 {Ellipses \ldots\ symbol.}%
 {}

\defgcs{cdots}%
 {}%
 {\LaTeX\ Kernel (Math Mode)}%
 {Centred ellipses \ensuremath{\cdots} symbol.}%
 {}

\defgcs{vdots}%
 {}%
 {\LaTeX\ Kernel (Math Mode)}%
 {Vertical ellipses \ensuremath{\vdots} symbol.}%
 {}

\defgcs{ddots}%
 {}%
 {\LaTeX\ Kernel (Math Mode)}%
 {Diagonal ellipses \ensuremath{\ddots} symbol.}%
 {}

\defgcs{dotsc}%
 {}%
 {\nxisty{amsmath} (Math Mode)}%
 {Ellipses \ensuremath{\dotsc} for dots with commas.}%
 {}

\defgcs{dotsb}%
 {}%
 {\nxisty{amsmath} (Math Mode)}%
 {Ellipses \ensuremath{\dotsb} for dots with binary
  operators\slash relations.}%
 {}

\defgcs{dotsm}%
 {}%
 {\nxisty{amsmath} (Math Mode)}%
 {Ellipses \ensuremath{\dotsm} for dots with multiplications.}%
 {}

\defgcs{dotsi}%
 {}%
 {\nxisty{amsmath} (Math Mode)}%
 {Ellipses \ensuremath{\dotsi} for dots with integrals.}%
 {}

\defgcs{dotso}%
 {}%
 {\nxisty{amsmath} (Math Mode)}%
 {Ellipses \ensuremath{\dotso} for general dots.}%
 {}

\defgcs{langle}%
 {}%
 {\LaTeX\ Kernel (Math Mode)}%
 {Left-angled \ensuremath{\langle} delimiter.}%
 {}

\defgcs{rangle}%
 {}%
 {\LaTeX\ Kernel (Math Mode)}%
 {Right-angled \ensuremath{\rangle} delimiter.}%
 {}

\defgcs{lfloor}%
 {}%
 {\LaTeX\ Kernel (Math Mode)}%
 {Left floor \ensuremath{\lfloor} delimiter.}%
 {}

\defgcs{rfloor}%
 {}%
 {\LaTeX\ Kernel (Math Mode)}%
 {Right floor \ensuremath{\rfloor} delimiter.}%
 {}

\defgcs{lceil}%
 {}%
 {\LaTeX\ Kernel (Math Mode)}%
 {Left ceil \ensuremath{\lceil} delimiter.}%
 {}

\defgcs{rceil}%
 {}%
 {\LaTeX\ Kernel (Math Mode)}%
 {Right ceil \ensuremath{\rceil} delimiter.}%
 {}

\defgcs{lvert}%
 {}%
 {\nxisty{amsmath} (Math Mode)}%
 {Left vertical bar \ensuremath{\lvert} delimiter.}%
 {}

\defgcs{rvert}%
 {}%
 {\nxisty{amsmath} (Math Mode)}%
 {Right vertical bar \ensuremath{\rvert} delimiter.}%
 {}

\defgcs{lVert}%
 {}%
 {\nxisty{amsmath} (Math Mode)}%
 {Left double vertical bar \ensuremath{\lVert} delimiter.}%
 {}

\defgcs{rVert}%
 {}%
 {\nxisty{amsmath} (Math Mode)}%
 {Right double vertical bar \ensuremath{\rVert} delimiter.}%
 {}

\defgcs{vec}%
 {\marg{\meta{c}}}%
 {\LaTeX\ Kernel (Math Mode)}%
 {Typesets its argument as a vector (defaults to a right arrow
   accent).}%
 {%
   \BeginArgList
    \csentryargitem{c} The character or symbol that represents a vector.
   \EndArgList
 }

\defgcs{forall}%
 {}%
 {\LaTeX\ Kernel (Math Mode)}%
 {\dq{For all} \ensuremath{\forall} symbol.}%
 {}


\defgcs{noindent}%
 {}%
 {\LaTeX\ Kernel}%
 {Suppress the indentation that would usually occur at the start of
  the next paragraph.}%
 {}

\defgcs{refstepcounter}%
 {\marg{\meta{counter}}}%
 {\LaTeX\ Kernel}%
 {Increments the value of the given counter by one and allows the
  counter to be cross-referenced using \nxglsni{ref} and \nxglsi{label}.}%
 {%
   \BeginArgList
    \csentryargitem{counter} The name of the counter that needs
     incrementing.
   \EndArgList
 }

\defgcs{stepcounter}%
 {\marg{\meta{counter}}}%
 {\LaTeX\ Kernel}%
 {Increments the value of the given counter by one.}%
 {%
   \BeginArgList
    \csentryargitem{counter} The name of the counter that needs
     incrementing.
   \EndArgList
 }

\defgcs{setcounter}%
 {\marg{\meta{counter}}\marg{\meta{number}}}%
 {\LaTeX\ Kernel}%
 {Sets the value of a counter.}%
 {%
   \BeginArgList
    \csentryargitem{counter} The name of the counter that needs
     changing.
    \csentryargitem{number} The new value. (Must be an integer.)
   \EndArgList
 }

\defgcs{addtocounter}%
 {\marg{\meta{counter}}\marg{\meta{increment}}}%
 {\LaTeX\ Kernel}%
 {Increments the value of a counter by the given amount.}%
 {%
   \BeginArgList
    \csentryargitem{counter} The name of the counter that needs
     changing.
    \csentryargitem{number} The amount by which to increment the counter. (Must be an integer.)
   \EndArgList
 }

\defgcs{newcounter}%
 {\marg{\meta{counter}}\oarg{\meta{outer counter}}}%
 {\LaTeX\ Kernel}%
 {Defines a new counter.}%
 {%
   \BeginArgList
    \csentryargitem{counter} The name of the new counter.
    \csentryargitem{outer counter} The name of the parent counter.
   \EndArgList
 }


\defgcs{value}%
 {\marg{\meta{counter}}}%
 {\LaTeX\ Kernel}%
 {References the value of the given counter where a number rather
   than a counter name is required.}%
 {%
   \BeginArgList
    \csentryargitem{counter} The name of the counter.
   \EndArgList
 }

\defgcs{parindent}%
 {}%
 {\LaTeX\ Kernel}%
 {A \nxglsi{length} register that stores the indentation at the start of paragraphs.}%
 {}


\defgcs{setlength}%
 {\marg{\meta{register}}\marg{\meta{dimension}}}%
 {\LaTeX\ Kernel}%
 {Sets the value of a length register.}%
 {%
   \BeginArgList
     \csentryargitem{register} The name of the length register (for
       example \nxglsi{parindent}).
     \csentryargitem{dimension} The new dimension.
   \EndArgList
 }

\defgcs{addtolength}%
 {\marg{\meta{register}}\marg{\meta{dimension}}}%
 {\LaTeX\ Kernel}%
 {Adds \meta{dimension} to the value of the given length register.}%
 {%
   \BeginArgList
     \csentryargitem{register} The name of the length register (for
       example \nxglsi{parindent}).
     \csentryargitem{dimension} The dimension to add to the length.
   \EndArgList
 }

\defgcs{currenttime}%
 {}%
 {\nxisty{datetime} package}%
 {Inserts into the output file the time when the \LaTeX\ 
  application created it from the source code.}%
 {}

\defgcs{ddmmyyyydate}%
 {}%
 {\nxisty{datetime} package}%
 {Changes the format of \nxglsi{today} so that it displays the date in
  the form {\ddmmyyyydate\today} (day\slash month\slash year in digits).}%
 {}

\defgcs{subref}%
 {\marg{\meta{label}}}%
 {\nxisty{subcaption} package}%
 {Analogous to \nxglsi{ref} but only references the subfigure
  or subtable caption.}%
 {%
   \BeginArgList
    \csentryargitem{label} The required label that was used in the
     corresponding \nxglsi{label}.
   \EndArgList
 }

\defgcs{eqref}%
 {\marg{\meta{label}}}%
 {\nxisty{amsmath} package}%
 {Short cut for \texttt{(\nxglsi{ref}\marg{\meta{label}})} for
  referencing equations.}%
 {%
   \BeginArgList
    \csentryargitem{label} The required label that was used in the
     corresponding \nxglsi{label}.
   \EndArgList
 }

\defgcs{frenchspacing}%
 {}%
 {\LaTeX\ Kernel}%
 {Switch to French spacing.}%
 {}

\defgcs{nonfrenchspacing}%
 {}%
 {\LaTeX\ Kernel}%
 {Switch to English spacing.}%
 {}

\defgcs{newline}%
 {}%
 {\LaTeX\ Kernel}%
 {Forces a line break.}%
 {}

\defgcs{linebreak}%
 {\oarg{\meta{n}}}%
 {\LaTeX\ Kernel}%
 {Requests a line break, ensuring the paragraph remains justified.
  This may cause excess white space in the paragraph.}%
 {%
   \BeginArgList
    \csentryargitem{n} An integer from~0 to~4 indicating how strongly
    you want a line break to occur. The higher the number, the
    stronger the request to break the line.
   \EndArgList
 }

\defgcs{displaybreak}%
 {\oarg{\meta{n}}}%
 {\nxisty{amsmath} package}%
 {Allows a page break in multi-lined maths environments, such as
  \nxglsi{env-align}.}%
 {%
   \BeginArgList
    \csentryargitem{n} An integer from~0 to~4 indicating how strongly
    you want a page break to occur. The higher the number, the
    stronger the request.
   \EndArgList
 }

\defgsymcs[beginmath]{\openparensym}%
 {}%
 {\LaTeX\ Kernel}%
 {Equivalent to \nxglsni{begin}\marg{math}.}%
 {}

\defgsymcs[endmath]{\closeparensym}%
 {}%
 {\LaTeX\ Kernel}%
 {Equivalent to \nxglsni{end}\marg{math}.}%
 {}

\defgchar
 {openparen}
 {\openparensym}
 {}
 {\LaTeX\ Kernel}
 {%
   Opening parenthesis in text mode or left round bracket
   delimiter in math mode.%
 }
 {}

\defgchar
 {closeparen}
 {\closeparensym}
 {}
 {\LaTeX\ Kernel}
 {%
   Closing parenthesis in text mode or right round bracket
   delimiter in math mode.%
 }
 {}

\defgsymcs[begindispmath]{\opensqsym}%
 {}%
 {\LaTeX\ Kernel (inconsistency corrected in \nxisty{amsmath})}%
 {Starts an unnumbered single-line of displayed maths.}%
 {}

\defgsymcs[enddispmath]{\closesqsym}%
 {}%
 {\LaTeX\ Kernel (inconsistency corrected in \nxisty{amsmath})}%
 {Ends an unnumbered single-line of displayed maths.}%
 {}

\defgchar
 {opensq}
 {\opensqsym}
 {}
 {\LaTeX\ Kernel}
 {\nopostdesc}
 {}

\defgchildchar
 {opt.opensq}
 {opensq}
 {\opensqsym}
 {Open delimiter of an \nxglsi{optional}}

\defgchildchar
 {text.opensq}
 {opensq}
 {\opensqsym}
 {Open square bracket in text mode}

\defgchildchar
 {delimiter.opensq}
 {opensq}
 {\opensqsym}
 {Left square bracket delimiter in math mode}

\defgchar
 {closesq}
 {\closesqsym}
 {}
 {\LaTeX\ Kernel}
 {\nopostdesc}
 {}

\defgchildchar
 {opt.closesq}
 {closesq}
 {\closesqsym}
 {Closing delimiter of an \nxglsi{optional}}

\defgchildchar
 {text.closesq}
 {closesq}
 {\closesqsym}
 {Closing square bracket in text mode}

\defgchildchar
 {delimiter.closesq}
 {closesq}
 {\closesqsym}
 {Right square bracket delimiter in math mode}

\defgsymcs[comma]{\commasym}%
 {}%
 {\LaTeX\ Kernel}%
 {Thin space.}%
 {}

\defgcs{thinspace}%
 {}%
 {\LaTeX\ Kernel}%
 {Thin space.}%
 {}

\defgsymcs[colon]{\colonsym}%
 {}%
 {\LaTeX\ Kernel (Math Mode)}%
 {Medium space.}%
 {}

\defgcs{medspace}%
 {}%
 {\nxisty{amsmath} package}%
 {Medium space.}%
 {}

\defgsymcs[semicolon]{\semicolonsym}%
 {}%
 {\LaTeX\ Kernel (Math Mode)}%
 {Thick space.}%
 {}

\defgcs{thickspace}%
 {}%
 {\nxisty{amsmath} package}%
 {Thick space.}%
 {}

\defgcs{qquad}%
 {}%
 {\LaTeX\ Kernel}%
 {Horizontal spacing command (twice as wide as \nxglsni{quad}).}%
 {}

\defgsymcs[spacesym]{\textvisiblespace}%
 {}%
 {\LaTeX\ Kernel}%
 {(Backslash followed by space character.) Horizontal spacing command.}%
 {}

\defgcs{enspace}%
 {}%
 {\LaTeX\ Kernel}%
 {Horizontal spacing command (half as wide as \nxglsi{quad}).}%
 {}

\defgcs{quad}%
 {}%
 {\LaTeX\ Kernel}%
 {Horizontal spacing command equal to the current font's \protect\htmlref{em}{sec:length} value.}%
 {}

\defgcs{negthinspace}%
 {}%
 {\LaTeX\ Kernel}%
 {Negative thin space.}%
 {}

\defgcs{negmedspace}%
 {}%
 {\nxisty{amsmath} package}%
 {Negative medium space.}%
 {}

\defgcs{negthickspace}%
 {}%
 {\nxisty{amsmath} package}%
 {Negative thick space.}%
 {}

\defgcs{contentsname}%
 {}%
 {Classes or packages that define a table of contents}%
 {Text used for table of contents heading.}%
 {}

\defgcs{listfigurename}%
 {}%
 {Classes or packages that define a list of figures}%
 {Text used for list of figures heading.}%
 {}

\defgcs{listtablename}%
 {}%
 {Classes or packages that define a list of tables}%
 {Text used for list of tables heading.}%
 {}

\defgcs{bibname}%
 {}%
 {Report or book style classes that define a bibliography chapter}%
 {Text used for bibliography chapter heading. (See also
  \nxglsi{refname}.)}%
 {}

\defgcs{refname}%
 {}%
 {Article style classes that define a bibliography section}%
 {Text used for bibliography section heading. (See also
   \nxglsi{bibname}.)}%
 {}

\defgcs{indexname}%
 {}%
 {Classes or packages that define an index section}%
 {Text used for index heading.}%
 {}

\defgcs{figurename}%
 {}%
 {Classes or packages that define figures}%
 {Number prefix used in figure captions.}%
 {}

\defgcs{tablename}%
 {}%
 {Classes or packages that define tables}%
 {Number prefix used in table captions.}%
 {}

\defgcs{partname}%
 {}%
 {Classes or packages that define parts with a number prefix}%
 {Number prefix used in part headings.}%
 {}

\defgcs{chaptername}%
 {}%
 {Classes or packages that define chapters}%
 {Number prefix used in chapter headings.}%
 {}

\defgcs{appendixname}%
 {}%
 {Classes or packages that define chapters}%
 {Number prefix used in appendix headings.}%
 {}

\defgcs{abstractname}%
 {}%
 {Classes or packages that define an \nxglsni{env-abstract}
  environment}%
 {Text used in abstract heading.}%
 {}

\defgcs{addto}%
 {\marg{\meta{command}}\marg{\meta{code}}}
 {\nxisty{babel} package}
 {Adds \meta{code} to the definition of \meta{command}. (See also \nxglsni{appto}.)}
 {%
   \BeginArgList
     \csentryargitem{command} The command that needs patching.
     \csentryargitem{code} The additional code to add to the command
definition.
   \EndArgList
 }

\defgcs{appto}%
 {\marg{\meta{command}}\marg{\meta{code}}}
 {\nxisty{etoolbox} package}
 {Adds \meta{code} to the definition of \meta{command}.}
 {%
   \BeginArgList
     \csentryargitem{command} The command that needs patching.
     \csentryargitem{code} The additional code to add to the command definition.
   \EndArgList
 }

\defgcs{mathrm}%
 {\marg{\meta{maths}}}%
 {\LaTeX\ Kernel (Math Mode)}%
 {Renders \meta{maths} in the predefined maths serif font.}%
 {%
   \BeginArgList
     \csentryargitem{maths} The text on which to apply the font change.
   \EndArgList
 }

\defgcs{mathsf}%
 {\marg{\meta{maths}}}%
 {\LaTeX\ Kernel (Math Mode)}%
 {Renders \meta{maths} in the predefined maths sans-serif font.}%
 {%
   \BeginArgList
     \csentryargitem{maths} The text on which to apply the font change.
   \EndArgList
 }

\defgcs{mathtt}%
 {\marg{\meta{maths}}}%
 {\LaTeX\ Kernel (Math Mode)}%
 {Renders \meta{maths} in the predefined maths typewriter font.}%
 {%
   \BeginArgList
     \csentryargitem{maths} The text on which to apply the font change.
   \EndArgList
 }

\defgcs{mathit}%
 {\marg{\meta{maths}}}%
 {\LaTeX\ Kernel (Math Mode)}%
 {Renders \meta{maths} in the predefined maths italic font.}%
 {%
   \BeginArgList
     \csentryargitem{maths} The text on which to apply the font change.
   \EndArgList
 }

\defgcs{mathbf}%
 {\marg{\meta{maths}}}%
 {\LaTeX\ Kernel (Math Mode)}%
 {Renders \meta{maths} in the predefined maths bold font. (Doesn't
  work with numbers and nonalphabetical symbols.) See also \nxglsi{boldsymbol}.}%
 {%
   \BeginArgList
     \csentryargitem{maths} The text on which to apply the font change.
   \EndArgList
 }

\defgcs{boldsymbol}%
 {\marg{\meta{symbol}}}%
 {\nxisty{amsmath} package (Math Mode)}%
 {Like \nxglsi{mathbf} but also works for numbers and
   many nonalphabetical symbols. (See also \nxglsi{pmb}.)}%
 {%
   \BeginArgList
     \csentryargitem{symbol} The character or symbol on which to apply the font change.
   \EndArgList
 }

\defgcs{pmb}%
 {\marg{\meta{symbol}}}%
 {\nxisty{amsmath} package (Math Mode)}%
 {\dq{Poor man's bold.} Overlays multiple copies of the symbol to
   produce a bold effect for symbols that don't work with
   \nxglsi{boldsymbol}.}%
 {%
   \BeginArgList
     \csentryargitem{symbol} The symbol on which to apply the bold
      effect.
   \EndArgList
 }

\defgcs{mathcal}%
 {\marg{\meta{maths}}}%
 {\LaTeX\ Kernel (Math Mode)}%
 {Typesets its argument in the maths calligraphic font.
  Example: \nxglsni{beginmath}\protect\cmdname{mathcal}\marg{S}\nxglsni{endmath} produces
 \ensuremath{\mathcal{S}}.}%
 {%
   \BeginArgList
    \csentryargitem{maths} The maths to be displayed in a calligraphic
      font.
   \EndArgList
 }

\defgcs{mathbb}%
 {\marg{\meta{maths}}}%
 {\nxisty{amsfonts} package (Math Mode)}%
 {Typesets its argument in the blackboard bold font.
  Example: \nxglsni{beginmath}\protect\cmdname{mathbb}\marg{R}\nxglsni{endmath} produces
  \ensuremath{\mathbb{R}}.}%
 {%
   \BeginArgList
    \csentryargitem{maths} The maths to be displayed in a blackboard
      bold font.
   \EndArgList
 }

\defgcs{mathfrak}%
 {\marg{\meta{maths}}}%
 {\nxisty{amsfonts} package (Math Mode)}%
 {Typesets its argument in Euler Fraktur letters.
  Example: \nxglsni{beginmath}\protect\cmdname{mathfrak}\marg{U}\nxglsni{endmath} produces
  \ensuremath{\mathfrak{U}}.}%
 {%
   \BeginArgList
    \csentryargitem{maths} The maths to be displayed in Euler Fraktur
      letters.
   \EndArgList
 }

\defgcs{DeclareMathOperator}%
 {\marg{\meta{cmd}}\marg{\meta{operator-name}}}%
 {\nxisty{amsmath} package (Preamble Only)}%
 {Defines a new maths operator. The starred version allows limits.}%
 {%
   \BeginArgList
    \csentryargitem{cmd} The name of the new operator command (must
      begin with a backslash).
    \csentryargitem{operator name} The name of the maths operator.
   \EndArgList
 }

\defgcs{frac}%
 {\marg{\meta{numerator}}\marg{\meta{denominator}}}%
 {\LaTeX\ Kernel (Math Mode)}%
 {Displays a fraction.}%
 {%
    \BeginArgList
     \csentryargitem{numerator} The numerator (above the line).
     \csentryargitem{denominator} The denominator (below the line).
    \EndArgList
 }

\defgcs{cfrac}%
 {\oarg{\meta{pos}}\marg{\meta{numerator}}\marg{\meta{denominator}}}%
 {\nxisty{amsmath} (Math Mode)}%
 {Displays a continued fraction.}%
 {%
    \BeginArgList
     \csentryargitem{pos} May be \texttt{l} (left) or \texttt{r}
      (right).
     \csentryargitem{numerator} The numerator (above the line).
     \csentryargitem{denominator} The denominator (below the line).
    \EndArgList
 }

\defgcs{sqrt}%
 {\oarg{\meta{order}}\marg{\meta{operand}}}%
 {\LaTeX\ Kernel (Math Mode)}%
 {Displays a root.}%
 {%
    \BeginArgList
     \csentryargitem{order} The order of the root. (If omitted, a
       square root).
     \csentryargitem{operand} The operand.
    \EndArgList
 }

\defgcs{arabic}%
 {\marg{\meta{counter}}}%
 {\LaTeX\ Kernel}%
 {Displays counter value as an Arabic number. (1, 2, 3, \ldots)}%
 {%
    \BeginArgList
     \csentryargitem{counter} The name of the counter.
    \EndArgList
 }

\defgcs{roman}%
 {\marg{\meta{counter}}}%
 {\LaTeX\ Kernel}%
 {Displays counter value as a lower case Roman number. (i, ii, iii, \ldots)}%
 {%
    \BeginArgList
     \csentryargitem{counter} The name of the counter.
    \EndArgList
 }

\defgcs{Roman}%
 {\marg{\meta{counter}}}%
 {\LaTeX\ Kernel}%
 {Displays counter value as an upper case Roman number. (I, II, III, \ldots)}%
 {%
    \BeginArgList
     \csentryargitem{counter} The name of the counter.
    \EndArgList
 }

\defgcs{alph}%
 {\marg{\meta{counter}}}%
 {\LaTeX\ Kernel}%
 {Displays counter value as a lower case letter. (a, b, c, \ldots, z)}%
 {%
    \BeginArgList
     \csentryargitem{counter} The name of the counter.
    \EndArgList
 }

\defgcs{Alph}%
 {\marg{\meta{counter}}}%
 {\LaTeX\ Kernel}%
 {Displays counter value as an upper case letter. (A, B, C, \ldots, Z)}%
 {%
    \BeginArgList
     \csentryargitem{counter} The name of the counter.
    \EndArgList
 }

\defgcs{fnsymbol}%
 {\marg{\meta{counter}}}%
 {\LaTeX\ Kernel}%
 {Displays counter value as footnote symbol. (\makeimg{asterisk
  dagger double dagger section mark
  paragraph mark double bar double asterisk two single daggers
  two double daggers}{\footnotesymbols})}%
 {%
    \BeginArgList
     \csentryargitem{counter} The name of the counter.
    \EndArgList
 }

\defgenv{bfseries}%
 {}%
 {\LaTeX\ Kernel}%
 {Typesets the environment contents in a bold font.}%
 {}

\defgenv{itshape}%
 {}%
 {\LaTeX\ Kernel}%
 {Typesets the environment contents in an italic font.}%
 {}

\defgenv{em}%
 {}%
 {\LaTeX\ Kernel}%
 {Typesets the environment contents in an emphasized font. (Switches
  to italic\slash slanted if the surrounding font is upright, or switches to
  upright if the surrounding font is italic\slash slanted.)}%
 {}

\defgenv{itemize}%
 {}%
 {\LaTeX\ Kernel}%
 {Unordered list.}%
 {}

\defgenv{enumerate}%
 {}%
 {\LaTeX\ Kernel}%
 {Ordered list.}%
 {}

\defgenv{description}%
 {}%
 {Most class files}%
 {Labelled list.}%
 {}

\defgenv{document}%
 {}%
 {\LaTeX\ Kernel}%
 {The body of the document.}%
 {}

\defgenv{tabular}%
 {\oarg{\meta{v-pos}}\marg{\meta{column specifiers}}}%
 {\LaTeX\ Kernel (Text Mode)}%
 {%
   Environment for lining things up in rows and columns.
   Use \nxglsi{env-array} for math mode.
 }%
 {%
    \BeginArgList
     \csentryargitem{v-pos} Vertical alignment of the entire
      environment with respect to the surrounding baseline. May be
      one of \texttt{t} (top), \texttt{b} (bottom) or \texttt{c}
      (centered).
     \csentryargitem{column specifiers} Indicates how to align the
     columns.
    \EndArgList
 }

\defgenv{array}%
 {\oarg{\meta{v-pos}}\marg{\meta{column specifiers}}}%
 {\LaTeX\ Kernel (Math Mode)}%
 {%
   Environment for lining things up in rows and columns.
   Use \nxglsi{env-tabular} for text mode.
 }%
 {%
    \BeginArgList
     \csentryargitem{v-pos} Vertical alignment of the entire
      environment with respect to the surrounding baseline. May be
      one of \texttt{t} (top), \texttt{b} (bottom) or \texttt{c}
      (centered).
     \csentryargitem{column specifiers} Indicates how to align the
     columns.
    \EndArgList
 }

\defgenv{align}
 {}
 {\nxisty{amsmath} package}
 {Used for numbered aligned equations.}
 {}

\defgenv{align*}
 {}
 {\nxisty{amsmath} package}
 {Used for unnumbered aligned equations.}
 {}

\defgenv{minipage}%
 {\oarg{\meta{pos}}\oarg{\meta{height}}\marg{\meta{width}}}%
 {\LaTeX\ Kernel}%
 {Makes a box with line-wrapped contents. (See also \nxglsi{parbox}.)}%
 {%
   \BeginArgList
    \csentryargitem{pos} The vertical alignment of the box relative to
     the surrounding text. (Centred if omitted.)
    \csentryargitem{height} The height of the box. (If omitted the
     height is the natural height of the contents of the box.)
    \csentryargitem{width} The width of the box.
   \EndArgList
 }

\defgenv{abstract}%
 {}%
 {Most article- or report-style classes, such as \nxicls{scrartcl} or
  \nxicls{scrreprt}. Not usually defined in book-style classes, such
  as \nxicls{scrbook}, but is defined in \nxicls{memoir}}%
 {Displays its contents as an abstract.}%
 {}

\defgenv{thebibliography}%
 {\marg{\meta{widest entry label}}}%
 {Most classes that define sectioning commands}%
 {Bibliographic list. (See also \nxglsi{bibitem} and
  \nxglsi{cite}).}%
 {%
   \BeginArgList
    \csentryargitem{widest entry label} The widest label in the
     bibliography list.
   \EndArgList
 }

\defgenv{otherlanguage}
 {\marg{\meta{language name}}}%
 {\nxisty{babel} package}%
 {Within the environment contents, predefined textual elements, such
  as the date given by \nxglsi{today} or prefixes like \dq{Chapter},
  are set to those supplied by the given language.}%
 {%
   \BeginArgList
    \csentryargitem{language} The language identifier.
   \EndArgList
 }

\defgenv{figure}
 {\oarg{\meta{placement}}}%
 {Most classes that define sectioning commands}%
 {%
   Floats the contents to the nearest location according to the
   preferred placement options, if possible. Within the environment,
   \nxglsi{caption} may be used one or more times, as required.
   The caption will usually include the prefix given by
   \nxglsni{figurename}.
 }%
 {%
   \BeginArgList
    \csentryargitem{placement} The preferred placement.
   \EndArgList
 }

\defgenv{sidewaysfigure}%
 {}%
 {\nxisty{rotating} package}%
 {Like the \nxglsi{env-figure} environment but rotates the entire
  figure (including caption) sideways.}%
 {}

\defgenv{table}
 {\oarg{\meta{placement}}}%
 {Most classes that define sectioning commands}%
 {%
   Floats the contents to the nearest location according to the
   preferred placement options, if possible. Within the environment,
   \nxglsi{caption} may be used one or more times, as required.
   The caption will usually include the prefix given by
   \nxglsni{tablename}.
 }%
 {%
   \BeginArgList
    \csentryargitem{placement} The preferred placement.
   \EndArgList
 }

\defgenv{sidewaystable}%
 {}%
 {\nxisty{rotating} package}%
 {Like the \nxglsi{env-table} environment but rotates the entire
  table (including caption) sideways.}%
 {}

\defgenv{subfigure}%
 {\oarg{\meta{pos}}\marg{\meta{width}}}%
 {\nxisty{subcaption} package}%
 {Used to form a subfigure within a \nxglsi{env-figure}
  environment. The \nxglsi{caption} command may be used in this
  environment to produce a subcaption.}%
 {%
   \BeginArgList
    \csentryargitem{pos} The vertical alignment of the box relative to
     the surrounding text. (Centred if omitted.)
    \csentryargitem{width} The width of the box.
   \EndArgList
 }

\defgenv{subtable}%
 {\oarg{\meta{pos}}\marg{\meta{width}}}%
 {\nxisty{subcaption} package}%
 {Used to form a subtable within a \nxglsni{env-table}
  environment. The \nxglsi{caption} command may be used in this
  environment to produce a subcaption.}%
 {%
   \BeginArgList
    \csentryargitem{pos} The vertical alignment of the box relative to
     the surrounding text. (Centred if omitted.)
    \csentryargitem{width} The width of the box.
   \EndArgList
 }

\defgenv{equation}%
 {}%
 {\LaTeX\ Kernel}%
 {Displays its contents as a single-lined numbered equation.}%
 {}

\defgenv{displaymath}%
 {}%
 {\LaTeX\ Kernel}%
 {Displays its contents as a single-lined unnumbered equation.}%
 {}

\defgenv{math}%
 {}%
 {\LaTeX\ Kernel}%
 {Sets its contents in in-line math mode.}%
 {}

\defgenv{cases}%
 {}%
 {\nxisty{amsmath} package (Math Mode)}%
 {%
   Like the \nxglsi{env-array} environment, but adds a left
   brace start delimiter and an invisible end delimiter.
 }%
 {}

\defgenv{matrix}%
 {}%
 {\nxisty{amsmath} package (Math Mode)}%
 {%
   Like the \nxglsi{env-array} environment, but doesn't have an
   argument.
 }%
 {}

\defgenv{pmatrix}%
 {}%
 {\nxisty{amsmath} package (Math Mode)}%
 {%
   Like the \nxglsi{env-array} environment, but doesn't have an
   argument and adds round bracket delimiters.
 }%
 {}

\defgenv{smallmatrix}%
 {}%
 {\nxisty{amsmath} package (Math Mode)}%
 {%
   Like the \nxglsi{env-array} environment but doesn't have an
   argument and is designed for in-line maths.
 }%
 {}

\defgenv{bmatrix}%
 {}%
 {\nxisty{amsmath} package (Math Mode)}%
 {%
   Like the \nxglsi{env-array} environment, but doesn't have an
   argument and adds square bracket delimiters.
 }%
 {}

\defgenv{Bmatrix}%
 {}%
 {\nxisty{amsmath} package (Math Mode)}%
 {%
   Like the \nxglsi{env-array} environment, but doesn't have an
   argument and adds curly brace delimiters.
 }%
 {}

\defgenv{vmatrix}%
 {}%
 {\nxisty{amsmath} package (Math Mode)}%
 {%
   Like the \nxglsi{env-array} environment, but doesn't have an
   argument and adds single vertical bar delimiters.
 }%
 {}

\defgenv{Vmatrix}%
 {}%
 {\nxisty{amsmath} package (Math Mode)}%
 {%
   Like the \nxglsi{env-array} environment, but doesn't have an
   argument and adds double vertical bar delimiters.
 }%
 {}

\defgcs{raggedright}%
 {}%
 {\LaTeX\ Kernel}%
 {Ragged-right paragraph justification.}%
 {}

\defgcs{raggedleft}%
 {}%
 {\LaTeX\ Kernel}%
 {Ragged-left paragraph justification.}%
 {}

\defgcs{frontmatter}
 {}%
 {Most book-style classes, such as \nxicls{scrbook}}%
 {Switches to lower case Roman numeral page numbering. Also suppresses
   chapter and section numbering, but still adds unstarred sectional
   units to the table of contents. (See also \nxglsi{mainmatter}
   and \nxglsi{backmatter}.)}%
 {}

\defgcs{mainmatter}
 {}%
 {Most book-style classes, such as \nxicls{scrbook}}%
 {Switches to Arabic page numbering and enables
   chapter and section numbering. (See also
   \nxglsi{frontmatter} and \nxglsi{backmatter}.)}%
 {}

\defgcs{backmatter}
 {}%
 {Most book-style classes, such as \nxicls{scrbook}}%
 {Suppresses chapter and section numbering, but still adds unstarred
  sectional units to the table of contents. (See also \nxglsi{frontmatter}
  and \nxglsi{mainmatter}.)}%
 {}

\defgenv{dinglist}
 {\marg{\meta{number}}}
 {\nxisty{pifont} package}
 {A list where the item marker is given by character \meta{number}
  in the Zapf Dingbats font.}
 {%
   \BeginArgList
    \csentryargitem{number} The character code for the item marker.
   \EndArgList
 }

\defgcs{notag}%
 {}%
 {\nxisty{amsmath} package}%
 {Suppresses equation numbering for the current row in environments such as
   \nxglsi{env-align}.}%
 {}

\defgcs{tag}%
 {\marg{\meta{tag}}}%
 {\nxisty{amsmath} package}%
 {Overrides equation numbering for the current row in environments such as
   \nxglsi{env-align}.}%
 {%
    \BeginArgList
     \csentryargitem{tag} The replacement for the equation number.
    \EndArgList
 }

\defgcs{ding}%
 {\marg{\meta{n}}}%
 {\nxisty{pifont} package}%
 {Inserts PostScript ZapfDingbats character with code \meta{n},
   which must be an integer.}%
 {%
   \BeginArgList
    \csentryargitem{n} The decimal code of the required character.
   \EndArgList
 }

\defgcs{color}%
 {\oarg{\meta{model}}\marg{\meta{specs}}}%
 {\nxisty{color} and \nxisty{xcolor} packages}%
 {A declaration that switches the current foreground colour to the given specification.}%
 {%
    \BeginArgList
      \csentryargitem{model} The colour model, for example
      \texttt{rgb} or \texttt{named}.
      \csentryargitem{specs} The colour specification for the
      given model. For example, if the \texttt{rgb} model is chosen,
      \meta{specs} must be a comma-separated list of three numbers
      each between~0 and~1.
    \EndArgList
 }

\defgcs{textcolor}%
 {\oarg{\meta{model}}\marg{\meta{specs}}\marg{\meta{text}}}%
 {\nxisty{color} and \nxisty{xcolor} packages}%
 {Sets \meta{text} with the foreground colour according to the given \meta{specs}.}%
 {%
    \BeginArgList
      \csentryargitem{\meta{model}} The colour model, for example
      \texttt{rgb} or \texttt{named}.
      \csentryargitem{\meta{specs}} The colour specification for the
      given model. For example, if the \texttt{rgb} model is chosen,
      \meta{specs} must be a comma-separated list of three numbers
      each between~0 and~1.
      \csentryargitem{\meta{text}} The text to be displayed in the
      given colour.
    \EndArgList
 }

\defgcs{tabcolsep}
 {}%
 {\LaTeX\ Kernel}%
 {\nxGls{length} register specifying half the gap between
   columns in a \nxglsni{env-tabular} environment.}%
 {}

\defgcs{arraycolsep}
 {}%
 {\LaTeX\ Kernel}%
 {\nxGls{length} register specifying half the gap between
   columns in an \nxglsni{env-array} environment.}%
 {}


\defgcs{labelformat}%
 {\marg{\meta{ctr}}\marg{\meta{defn}}}%
 {\nxisty{fncylab} package}%
 {Defines how the label for the counter \meta{ctr} should be
 formatted. The definition \meta{defn} should use \texttt{\nxglsi{hashchar}1} to
 indicate the label value.}%
 {%
   \BeginArgList
     \csentryargitem{ctr} The name of the counter.
     \csentryargitem{defn} How to display the value of the counter.
   \EndArgList
 }

\defgcs{thefootnote}%
 {}%
 {\LaTeX\ Kernel}%
 {Displays the current value of the \nxicounter{footnote} counter}%
 {}

\defgcs{thechapter}%
 {}%
 {\LaTeX\ Kernel}%
 {Displays the current value of the \nxicounter{chapter} counter}%
 {}

\defgcs{thesection}%
 {}%
 {\LaTeX\ Kernel}%
 {Displays the current value of the \nxicounter{section} counter}%
 {}

\defgcs{thepage}%
 {}%
 {\LaTeX\ Kernel}%
 {Displays the current value of the \nxicounter{page} counter}%
 {}

\defgcs{thefigure}%
 {}%
 {\LaTeX\ Kernel}%
 {Displays the current value of the \nxicounter{figure} counter}%
 {}

\defgcs{thetable}%
 {}%
 {\LaTeX\ Kernel}%
 {Displays the current value of the \nxicounter{table} counter}%
 {}

\defgcs{ignorespaces}
 {}
 {\LaTeX\ Kernel}
 {Used in begin environment code to suppress any spaces occurring 
  at the start of the environment (see also \nxglsni{ignorespacesafterend}).}
 {}

\defgcs{ignorespacesafterend}%
 {}%
 {\LaTeX\ Kernel}%
 {Used in end environment code to suppress any spaces following the
  end of the environment.}%
 {}

\defgcs{textbar}%
 {}%
 {\LaTeX\ Kernel}%
 {Vertical bar \textbar\ symbol.}%
 {}

\defgcs{the}%
 {\meta{register}}%
 {\LaTeX\ Kernel}%
 {Displays the value of the given register (such as a \nxglsi{length} register).
  Not to be confused with \cmdname{the}\meta{ctr} commands, such as \nxglsni{thefigure}.}%
 {%
   \BeginArgList
     \csentryargitem{register} The name of the register.
   \EndArgList
 }

\defgcs{substack}%
 {\marg{\meta{maths}}}%
 {\nxisty{amsmath} package}%
 {Can be used to produce a multiline subscript or superscript. Lines
  are separated using \nxglsi{tab.dbbackslashchar}.}%
 {%
   \BeginArgList
    \csentryargitem{maths} The subscript or superscript maths with rows
      ended using \nxglsi{tab.dbbackslashchar}
   \EndArgList
 }

\defgcs{DeclareCaptionLabelFormat}
 {\marg{\meta{name}}\marg{\meta{code}}}
 {\nxisty{caption}}
 {Used to defined your own caption label formats.}
 {%
   \BeginArgList
     \csentryargitem{name} The name used to identify this new format
     \csentryargitem{code} The code that defines this format
     (\nxglsi{hashchar}1 gets replaced with the name and \nxglsni{hashchar}2
     gets replaced with the reference number.)
   \EndArgList
 }

\defgcs{captionsetup}
 {\oarg{\meta{float type}}\marg{\meta{options}}}
 {\nxisty{caption} package}
 {Used to set up the options affecting float captions.}
 {%
   \BeginArgList
     \csentryargitem{float type} The float type.
     \csentryargitem{options} A \meta{key}=\meta{value}
     comma-separated list of options.
   \EndArgList
 }


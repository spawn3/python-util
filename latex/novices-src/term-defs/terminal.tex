Sometimes you may find that you need to use a command-line
application. This is an application that doesn't have a graphical
user interface. This isn't specific to \glsterm{tex}, but the \TeX\
distribution comes with a number of them. In fact, front-ends (such
as TeXWorks) run some of these applications for you when click on
the typeset or build button.

Most operating systems provide a \gls*{terminal} or command prompt where
you can type the command-line application name and any associated
information. For example, \xfigureref{fig:terminal} shows a terminal
running under Fedora on Linux.

\begin{description}
\item[Windows] To open the MSDOS Prompt, go to the Start menu, then
\dq{All Programs}, then \dq{Accessories} and click on \dq{MSDOS
Prompt}.

\item[Mac OSX] To open the Mac Terminal, go to your
\dq{Applications} folder, open \dq{Utilities} and double click on \dq{Terminal}.

\item[Unix etc] The Terminal is usually located either in the \dq{Applications}
menu or in the \dq{System Tools} subdirectory of the
\dq{Applications} menu.
\end{description}

\begin{figure}[htbp]
\figconts
 {pictures/terminal}
 {%
   \caption{A Terminal}
 }
 {fig:terminal}
\end{figure}

\xminisec{Example:}

One such command-line application you are likely to need is
\iappname{texdoc}. This is mentioned in more detail in
\sectionref{sec:texdoc}, but to use \iappname{texdoc} you need to
open the terminal or command prompt as described above and type
\texttt{texdoc} followed by a package or class name, for example:
\begin{verbatim}
texdoc scrbook
\end{verbatim}
(see \figureref{fig:terminal-texdoc})
then press the Enter or Return \enter\ key.

\begin{figure}[htbp]
\figconts
 {pictures/terminal-texdoc}
 {%
   \caption{Running texdoc From a Terminal}
 }
 {fig:terminal-texdoc}
\end{figure}

Other \glsterm{tex}-related command-line applications include
\iappname{pdflatex}, \iappname{bibtex}, \iappname{makeindex},
\iappname{xindy} and \iappname{kpsewhich}.

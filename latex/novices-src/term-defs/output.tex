  The \LaTeX\ application reads in your \glsterm{source} 
  and creates the typeset document, the
  \gls*{output}. This book assumes that you will be using the
  version of \LaTeX\ that produces PDF files (PDF\LaTeX). If you are
  using TeXWorks (see \chapterref{ch:tex2pdf}), you need to select the
  \mbox{\dq{PDFLaTeX}} item from the drop-down list. If you are using
  TeXnicCenter, select the \mbox{\dq{LaTeX\TO PDF}} build
  profile. If you are
  using WinEdt, when you want to build your document click on the
  button marked \dq{PDFLaTeX} rather than the one marked \dq{LaTeX}.
  If you are using a terminal or command prompt, use the command
  \texttt{pdflatex} rather than \texttt{latex}. (TeXnicCenter, WinEdt
  and using the terminal or command prompt approach are described in
  the \suppmaterial[.)]{supplemental material}

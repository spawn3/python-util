\label{ch:def}%

As mentioned in \chapterref[the introduction]{ch:intro}, \LaTeX\ is a
language, so you can't simply start typing and expect to see your
document appear before your very eyes\faq{Why is TeX not a WYSIWYG system?}{notWYSIWYG}.  You need to know a few things
before you can get started, so it's best to define a few terms first.
Don't worry if there seems a lot to take in, there will be some
practical examples later, which should hopefully make things a little
clearer.

\begin{latexonly}
Throughout this book, \glstext{source} is
illustrated in a typewriter font with the word
\inputlabelformat{Input} placed in the margin, and the corresponding
output (how it will appear in the PDF document) is typeset with the word \outputlabelformat{Output} in the
margin.  

\xminisec{Example:}
A single line of code is displayed like this:
\end{latexonly}
\html{Throughout this document, \glstext{source} is
illustrated in the form:}
\begin{codeS}
This is an \glsni{textbf}\marg{example}.
\end{codeS}%
The corresponding 
\glslink{output}{output}\html{ (how it will appear in the PDF document)\footnote{This HTML version of the book uses bitmaps to illustrate the output, which doesn't look as good as the actual PDF version.}}\ is illustrated like this:
\begin{resultS}[exampleoutput.html]
This is an \textbf{example}.
\end{resultS}%
\begin{latexonly}%
Segments of code that are longer than one line are bounded above and below, illustrated as follows:
\begin{code}
\begin{alltt}
Line one\glsni{par}
Line two\glsni{par}
Line three.
\end{alltt}
\end{code}%
with corresponding output:
\begin{result}
Line one\par
Line two\par
Line three.
\end{result}%
\end{latexonly}%
Take care not to confuse a backslash~\glsni{backslashchar}
with a forward slash~\gls{text.slash} as they have different meanings.
\doifnotbook
{%
  (Commands typeset in blue, such as \glsni{par}, indicate a hyperlink to
  the command definition in the \htmlref{summary}{ch:glossary}.)
}

\Glstext{command} definitions are shown in a typewriter font in the form:
\begin{definition}
\glsni{documentclass}\oarg{\meta{options}}\marg{\meta{class file}}
\end{definition}%
In this case the command being defined is called
\cmdname{documentclass} and text typed \meta{like this} (such as
\meta{options} and \meta{class file}) indicates the type of thing you
need to substitute. (Don't type the angle brackets!) For example, if you want the \icls{scrartcl}
\glstext{cls} you would substitute \meta{class file} with
\texttt{scrartcl} and if you want the \clsopt{letterpaper} option you
would substitute \meta{options} with \texttt{letterpaper}, like this:
\begin{codeS}
\glsnl{documentclass}\oarg*{letterpaper}\marg*{scrartcl}
\end{codeS}
But more on that later.

\refstepcounter{object}\label{obj:visiblespace}%
Sometimes it can be easy to miss a space character when you're
reading this kind of \latexhtml{book}{document}. When it's important
to indicate a space, the visible space symbol~\gls{visiblespace} is used. For
example:
\begin{codeS}
A\glsni{visiblespace}sentence\glsni{visiblespace}consisting\glsni{visiblespace}of\glsni{visiblespace}six\glsni{visiblespace}words.
\end{codeS}
When you type up the code, replace any occurrence of
\glsni{visiblespace} with a space.

\refstepcounter{object}\label{obj:comment}
One other thing to mention is the comment character~\gls{percentchar}
(the percent symbol). Anything from the percent symbol up to, and
including, the end of line character is ignored by \LaTeX. Thus
\begin{code}
A simple \glsni{percentchar} next comes a command to make some bold text\newline
\glsni{textbf}\marg{example} 
\end{code}%
will produce the output
\begin{resultS}[commentexample.html]
A simple % next comes a command to make some bold text
\textbf{example} 
\end{resultS}%
\refstepcounter{object}\label{obj:suppresseol}%
The percent symbol is often used to suppress unwanted space 
resulting from line breaks\footnote{\LaTeX\ treats the end-of-line
character as a space.}\ in the \glstext{source}.
For example, the following code
\begin{code}
Foo\glsni{percentchar}\newline
Bar
\end{code}%
will produce the output:
\begin{resultS}[FooBar (single word)]
Foo%
Bar
\end{resultS}%
as opposed to
\begin{code}
Foo\newline
Bar
\end{code}%
which will produce the output:
\begin{resultS}[Foo Bar (two words)]
Foo\ Bar\relax% LaTeX2HTML is getting confused!
\end{resultS}%
On the other hand, spaces at the start of a line of input are
ignored, so
\begin{code}
Foo\glsni{percentchar}\newline
\strut~~Bar
\end{code}
still produces:
\begin{resultS}[FooBar]
Foo%
  Bar
\end{resultS}

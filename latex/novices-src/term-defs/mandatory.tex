\Glspl*{mandatory} are arguments that \emph{have} to be
specified.
\glspar
\xminisec{Examples:}
\begin{enumerate}
\item If you want a footnote, you need to use the
\glsi{footnote} command, which has a mandatory argument that
specifies the contents of the footnote. Like this:
\begin{codeS}
Here is a footnote.\glsni{footnote}\marg{This is the
footnote
text.}
\end{codeS}
\begin{latexonly}
\begin{resultS}
Here is a footnote.\footnotemark
\end{resultS}
\footnotetext{This is the footnote text.}%
\doifnotbook{(By default you
won't get a hyperlink. This book uses the \sty{hyperref} package,
which generates the hyperlinks.)}
\end{latexonly}

\item If you want to start a new chapter, you need to use the
\glsi{chapter} command, but you also need to tell \LaTeX\
the
title of this new chapter.  So the \glsi{chapter} command
takes
one mandatory argument that specifies the title.

For example, the following code:
\begin{codeS}
\glsni{chapter}\marg{Some Definitions}
\end{codeS}%
was used to generate the heading for
\latexhtml{the current chapter (at the top of
page~\pageref{ch:def})}{\chapterref[the definitions chapter]{ch:def}
of this document}.
\item The command \glsi{textbf} typesets its argument in a
bold
font (as opposed to the
\glsterm{declaration}
\glsi{bfseries} which switches to a bold font.)

For example, the following code:
\begin{codeS}
\glsni{textbf}\marg{Some bold text.}
\end{codeS}%
produces the output
\begin{resultS}[Some bold text. (All the text is in bold.)]
\textbf{Some bold text.}
\end{resultS}
\end{enumerate}

\xminisec{Notes:}
\begin{enumerate}
\item\label{itm:argnogrp} \LaTeX\ takes the first non-space object
following the command name as the argument, which is why in the
above
examples the arguments have to be \glsd{group}.

Suppose the last example above didn't have a group, so instead the
code was:
\begin{codeS}
\glsni{textbf} Some bold text.
\end{codeS}%
then only the \dq{S} would be the argument because it's the first
object
following the command, in which case the output would look like:
\begin{resultS}[Some bold text. (Only the letter S is bold.)]
\textbf Some bold text.
\end{resultS}

\item If you want the argument to be blank, use empty braces:
\gls{leftbracechar}\gls{rightbracechar}.  For example, suppose you
want to have a
chapter without a title\protect\footnote{The numbers for chapters, sections
etc are automatically inserted by \LaTeX, so this example would
produce a numbered chapter without a title.}\ you would need to do:
\begin{codeS}
\glsnl{chapter}\marg{}
\end{codeS}
\end{enumerate}

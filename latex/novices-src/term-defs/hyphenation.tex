Words sometimes require \gls*{hyphenation} to help justify paragraphs
and prevent overly large areas of white space or protrusions into
the right margin. Some word processors by default don't hyphenate
words in fully-justified paragraphs, which has led some people to
believe that hyphenation is bad. Just because word processors do
something a certain way, doesn't mean that it's the correct way.
\TeX\ has an excellent hyphenation algorithm, but the default
hyphenation pattern is designed for English. If you are writing in
another language, use the \isty{babel} package to switch the
hyphenation pattern (see \sectionref{sec:babel}).

Despite using an excellent algorithm, \TeX\ occasionally gets the
hyphenation wrong, particularly where the hyphenation is context
sensitive. There are two ways of setting the hyphenation for a given
word.

\begin{enumerate}
\item For all occurrences of the word, use
\begin{definition}
\gls{cshyphenation}\marg{\meta{hyphenated word}}
\end{definition}
inserting a hyphen \Indextt{-} at all possible hyphenation
points. For example:
\begin{codeS}
\glsni{cshyphenation}\marg{gal-axy}
\end{codeS}

\item For a particular instance of a word, use \gls{hyphen} at the
hyphenation point within the word. For example:
\begin{code}
There once was a little alien called Uiop who lived in the faraway gal\glsni{hyphen}axy of
Zxcv.
\end{code} 
\end{enumerate}

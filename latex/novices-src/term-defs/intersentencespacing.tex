\Gls*{intersentencespacing} refers to the default type of space
to be inserted between adjacent sentences within a paragraph. There is
disagreement over what size this space should be. French spacing
uses the same space as used between words. English spacing uses an
\Index{en-space} (half an \htmlref{em}{tab:units}\doifbook
 {\csname footnote\endcsname{see \sectionref{sec:length}}}-space). With proportional
fonts\latex{ (such as this one)}, the en-space is slightly larger than a
single space. The difference is shown here:
\begin{code}
\begin{alltt}
X X

X\gls{enspace} X
\end{alltt}
\end{code}
\begin{result}
X X
\glspar
X\enspace X
\end{result}
(Note that with fully-justified paragraphs, in both cases the spaces
may be stretched to ensure the sides of the paragraph are flushed.)

\TeX\ (and \LaTeX) defaults to English spacing, but you can switch
to French spacing using the declaration:
\begin{definition}
\gls{frenchspacing}
\end{definition}
and switch back again using
\begin{definition}
\gls{nonfrenchspacing}
\end{definition}

There was no en-space on a typewriter, so typists started using two
spaces in an attempt to emulate that slightly larger than one space
look. This habit has spread to word-processor users as well, and now
many people incorrectly assume English spacing means adding two
spaces after a full stop, which is too wide and looks ugly, but this
error shouldn't be used as a criticism against English spacing.

There has been a gradual trend towards French spacing over the last
century, and some publishers insist on it. I~think this may in part
be due to a backlash against the ugliness of two spaces in
typewritten and word-processed documents. In fact the Oxford Style
Manual~\cite{oxford} simply states, \dq{In text, only use a single
space after all sentence punctuation.}

Personally, I~prefer English spacing, particularly in reference
books. I~have many reference books on my shelf, but I~haven't read
any of them from cover-to-cover. I~flick to a particular section and
skim through the paragraphs until I~reach the desired bit of
information. Sometimes I've already looked something up, so I~have a vague
idea as to where to find the information. The extra space between
sentences makes it easier to locate a particular sentence.

This isn't so much of an issue with books designed to be read from
beginning to end, such as a novel. However, I~have read one such
book that used a font where the commas had tiny tails and most of
the sentences contained multiple proper nouns, which made it very
difficult to read as it wasn't clear where the sentences ended. Is
that a full stop followed by a new sentence that happens to start
with a proper noun, or is it a comma whose tiny tail is blurred by
my short-sighted eyes followed by a clause that happens to start
with a proper noun? A~well-written, well-presented document should
not interrupt the reader, forcing them to continually go back to
re-parse a sentence.

However, if you are writing a document, whether prose or technical,
with the intention of having it published you must check with the
publisher's guidelines to see if they insist on a particular style.
\xminisec{Notes:}
An end of sentence punctuation mark can be one of: a full stop
(\gls{sentence.periodchar}), exclamation mark
(\gls{sentence.exclamchar}) or question mark (\gls{questionchar}).

\begin{enumerate}
\item If an end of sentence punctuation mark follows a lower case
character, \TeX\ assumes the punctuation mark indicates the end of
the sentence. For example, as in:
\begin{codeS}
Did you see that? I certainly did.
\end{codeS}

\item Where this isn't the case, use \gls{spacesym} (backslash
followed by a space).
\begin{code}
This can happen when a sentence contains
a lower case abbreviation, e.g.\glsni{spacesym}like this
one.
\end{code}

\item If an end of sentence punctuation mark follows an upper case
character, \TeX\ assumes the sentence hasn't ended at that point.
For example, as in:
\begin{codeS}
The G.P. said it was only hypochondria.
\end{codeS}

\item Where the sentence actually ends with an upper case letter,
add \gls{at} after the letter and before the punctuation mark.
\begin{code}
Yesterday, I saw my G.P\glsni{at}. Tomorrow I'm going to see the
specialist.
\end{code}

\end{enumerate}

\xminisec{Note on Typewriter Fonts}

Note that \gls{nonfrenchspacing} in a monospaced font will insert
two spaces between sentences, emulating a typewritten document.
\begin{code}
\begin{alltt}
\glsni{ttfamily}
\glsni{nonfrenchspacing} x. x.

\glsni{frenchspacing} x. x.
\end{alltt}
\end{code}
\begin{result}\ttfamily
\nonfrenchspacing x. x.

\frenchspacing x. x.
\end{result}

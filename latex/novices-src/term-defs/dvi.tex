\glsterm{tex} (and subsequently \LaTeX) originally created \glspl*{dvi} 
instead of PDF files.\footnote{There was no PDF
back then.} However, although there are free DVI viewers, not many
people have them installed, so it's really only \TeX\ users who
can read them. Also, you can't embed image files in a DVI file or
have fancy effects, such as rotation. Instead, people can use
\TeX\slash \LaTeX\ to create
a DVI file and then use an application to convert the DVI file to
PostScript.

These days PDF is the preferred platform-independent format, and
with the advent of PDF\TeX, modern \TeX\slash \LaTeX\ users can
directly create PDF documents rather than going through the DVI
route. Some people still prefer to create DVI files as an
intermediate step, particularly if they want to embed PostScript
instructions (as is done by the \isty{pstricks} package). For
simplicity, this book assumes that you have a modern \TeX\
distribution and are using PDF\LaTeX\ rather than \LaTeX\TO DVI.

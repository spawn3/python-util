When \LaTeX\ creates your \gls{output}, it not only creates a PDF
file but also creates other associated files. The most common of these are
the log file\indexLOG, which has the extension \texttt{.log}, and
the \gls*{auxiliary}, which has the extension \texttt{.aux}.

The log file contains a transcript of the most recent \LaTeX\ run.
It lists all the files that have been loaded, including the \glsterm{cls} 
and any \htmlref{packages}{sec:packages} that your document has used.
There should also be the class or package version number and date,
although this is dependent on the class or package author. If you
ever want \htmlref{to ask for help}{ch:help}, you need to say what
version you are using.

For example, this book uses the \icls{scrbook}
class, so the log file includes the lines:
\begin{flushleft}\ttfamily
\noindent
(\slash usr\slash local\slash texlive\slash 2010\slash texmf-dist\slash tex\slash latex\slash koma-script\slash scrbook.cls\par
\noindent
Document Class:\ scrbook 2010\slash 06\slash 17 v3.06 KOMA-Script document class (book)\par
\end{flushleft}
(This is actually now out-of-date as the latest version at the time
of writing this is version~3.11a dated 2012/07/05.)

Error messages, warnings and general information messages are also
written to the log file as well as the document statistics. You can
delete this log file if you like. It will be created again the next
time you run \LaTeX.

The auxiliary file contains all the information needed for
cross-referencing (covered in \sectionref{sec:crossref}). This is
needed to ensure all your cross-references are up-to-date. You can
delete this file, but you will need at least two \LaTeX\ runs to
ensure your cross-references are correct the next time you create
your \gls{output}.

\htmlref{TeXWorks}{sec:texworks} also creates a file with the extension
\texttt{.synctex.gz}\indexSYNCTEX. This file allows you to jump to
and from the \gls{source} and the appropriate part of the
\gls{output}. If you delete this file, you will have to run \LaTeX\
again before you can use this function.

Other files that may be created include the table of contents file
(\texttt{.toc})\indexTOC, the list of figures file
(\texttt{.lof})\indexLOF\ and the list of tables file (\texttt{.lot})\indexLOT.
Some \glsunset{cls}\glspl{cls}\glsreset{cls} or \htmlref{packages}{sec:packages} create 
additional files. If your operating system
hides file extensions, you might want to switch off this behaviour,
if possible, to make it easier to distinguish between all the
various files.

TeXWorks has a menu item \menu{File}\menuto\menu{Remove AUX Files}
that will remove the auxiliary files.

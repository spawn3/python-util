\glsterm{tex}-distributions such as \itexdistro{TeX~Live} and \itexdistro{MiKTeX} also 
include some helper applications that you may find useful. For example,
\iappname{texdoc} (\sectionref{sec:texdoc}) helps you access
installed documentation and \iappname{makeindex} helps generate an
index for your document. Some of the helper applications are
written in a scripting language called \gls*{perl}, and you must
have the \iappname{perl} application installed to be able to use
them. Unix-like operating systems should already have it installed.
Windows users can choose between several Perl distributions. The
most popular seem to be \iperldistro{Strawberry Perl}{http://strawberryperl.com/} and 
\iperldistro[.]{Active Perl}{http://www.activestate.com/activeperl} Perl scripts that come
with \glsterm{tex} include: \iappname{epstopdf} (converts Encapsulated
PostScript (EPS)\indexEPS\ files to PDF), \iappname{pdfcrop} (crops
a PDF file), \iappname{xindy} (a more flexible indexing application
than \iappname{makeindex}), \iappname{texcount} (counts the number
of words in a \LaTeX\ document) and \iappname{latexmk} (runs \LaTeX\
and any associated applications, such as \iappname{bibtex}, the
required number of times to ensure the document is fully
up-to-date).


 A segment of code may be \glsd[hyper=false]{group} by placing it
 within \gls{leftbracechar} and \gls{rightbracechar} (curly braces).
 Most \glspl{command} that occur within a group will be
 local to that group.  For example, \glsi{bfseries} changes
 the font weight to bold, so the following segment of code:
\begin{code}
Here is some text. \marg{This text \glsi{bfseries} is in a 
group.} Here is some more text.
\end{code}%
will appear in the typeset document looking like:
\begin{resultS}[grouping.html]
Here is some text. {This text \bfseries is in a group.}
Here is some more text.
\end{resultS}

As can be seen, the font change only stays in effect until it
reaches the end of the group (signified by the closing curly brace
\glsni{rightbracechar}.) For a command to be in the same
\keywordfmt{scope} as another command, both commands must be within the same group.
For example, in the following, \glsi{bfseries} and
\glsi{itshape} are in the same scope:
\begin{codeS}
\marg{\glsi{bfseries} Some bold \glsi{itshape}
and italic text}
\end{codeS}
But below, they are in different scopes:
\begin{codeS}
\marg{\glsni{bfseries} Some bold text}
\marg{\glsni{itshape} and some italic text}
\end{codeS}

\Glsplural{environment} form an implicit scope.

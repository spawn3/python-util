An \gls*{environment} is a block of code contained within the
\glspl{command}
\begin{definition}
\gls{begin}\marg{\meta{env-name}}
\end{definition}%
and
\begin{definition}
\gls{end}\marg{\meta{env-name}}
\end{definition}%
where \meta{env-name} is the
name of the environment.  The block of code is then formatted in a
method specific to that environment.  For example, the
\gls{env-bfseries}\footnote{Note there is no backslash in
the
environment name.}\ environment will typeset the contents of the
environment in a bold font. The following code:
\begin{codeS}
\glsni{begin}\marg{bfseries}Here is some bold text.\glsni{end}\marg{bfseries}
\end{codeS}%
will appear in the typeset document looking like:
\begin{resultS}[Here is some bold text. (The entire sentence is in a
bold font.)]
\begin{bfseries}Here is some bold text.\end{bfseries}
\end{resultS}

Some environments also supply \glspl{command} that may only be used
within that environment. 

\xminisec{Example:}
The \glsni{env-itemize} environment provides a command called
\glsni{item} so that you can specify individual items within
an unordered list:
\begin{code}
\begin{alltt}
Shopping List:
\glsni{begin}\marg{itemize}
  \glsni{item} Cabbages
  \glsni{item} Bananas
  \glsni{item} Apples
\glsni{end}\marg{itemize}
\end{alltt}
\end{code}%
The above will produce the following output:
\begin{result}[shoppinglist.html]
Shopping List:
\begin{itemize}
\item Cabbages
\item Bananas
\item Apples
\end{itemize}
\end{result}

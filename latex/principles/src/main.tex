% -*- coding: UTF-8 -*-
% hello.tex

\documentclass[UTF8,oneside]{ctexbook}

% \usepackage{xeCJK}
\usepackage[utf8]{inputenc}

% load paralist before enumitem
\usepackage{paralist}

\usepackage{hyperref}
\hypersetup{pdftex,colorlinks=true,allcolors=blue}
\usepackage{hypcap}

\usepackage{color}
\usepackage[usenames, dvipsnames, svgnames, table]{xcolor}
% \pagecolor{gray}

\usepackage{makeidx}
\makeindex

\usepackage{amsmath}
\usepackage{mathtools}

\usepackage{listings}
\usepackage{multicol}
\usepackage{fancybox}
\usepackage{tcolorbox}
\usepackage{enumitem}

\usepackage{indentfirst}

\newenvironment{enumbox}[0]{
    \begin{tcolorbox}
    \begin{compactenum}
} {
    \end{compactenum}
    \end{tcolorbox}
}

\newenvironment{itembox}[0]{
    \begin{tcolorbox}
    \begin{compactitem}
} {
    \end{compactitem}
    \end{tcolorbox}
}

\usepackage[ampersand]{easylist}

\tcbset{colback=red!5!white,colframe=blue!75!black,boxrule=0.1mm}

\newenvironment{myeasylist}[1]{
    \Activate
    \begin{tcolorbox}
    \begin{easylist}[#1]

} {
    \end{easylist}
    \end{tcolorbox}
    \Deactivate
}

\newcommand{\mygraphics}[1] 
{
    \begin{center}
        \includegraphics[width=10cm]{#1}
    \end{center}
}

\newcommand{\mygraphicsh}[1]
{
    \begin{center}
        \includegraphics[height=11cm]{#1}
    \end{center}
}


% table
\setlength{\arrayrulewidth}{1pt}
\setlength{\tabcolsep}{16pt}
\renewcommand{\arraystretch}{2.5}
\newcolumntype{s}{>{\columncolor[HTML]{AAACED}} p{3cm}}

\arrayrulecolor[HTML]{DB5800}

\usepackage{tikz,mathpazo}
\usetikzlibrary{positioning, fit, matrix, shapes, arrows, chains, trees, arrows.meta}

% \bibliographystyle{plain}
% \bibliography{math}

\tikzset{%
  >={Latex[width=2mm,length=2mm]},
  % Specifications for style of nodes:
            base/.style = {rectangle, rounded corners, draw=black,
                           minimum width=4cm, minimum height=1cm,
                           text centered, font=\sffamily},
  activityStarts/.style = {base, fill=blue!30},
       startstop/.style = {base, fill=red!30},
    activityRuns/.style = {base, fill=green!30},
         process/.style = {base, minimum width=2.5cm, fill=orange!15,
                           font=\ttfamily},
}

% 摘录
\usepackage{verbatim}
\usepackage{libertine}
\usepackage{graphicx}
\usepackage{framed}

\newcommand*\openquote{\makebox(25,-22){\scalebox{5}{``}}}
\newcommand*\closequote{\makebox(25,-22){\scalebox{5}{''}}}
\colorlet{shadecolor}{Azure}

\makeatletter
\newif\if@right
\def\shadequote{\@righttrue\shadequote@i}
\def\shadequote@i{\begin{snugshade}\begin{quote}\openquote}
\def\endshadequote{%
\if@right\hfill\fi\closequote\end{quote}\end{snugshade}}
\@namedef{shadequote*}{\@rightfalse\shadequote@i}
\@namedef{endshadequote*}{\endshadequote}
\makeatother

\usepackage[normalem]{ulem}

\newcommand{\hl}{\bgroup\markoverwith
  {\textcolor{yellow}{\rule[-.5ex]{2pt}{2.5ex}}}\ULon}

%\usepackage{soul}

%\newcommand{\hlc}[2][yellow]{{%
%    \colorlet{foo}{#1}%
%    \sethlcolor{foo}\hl{#2}}%
%}

% todonode
\usepackage{lipsum}                     % Dummytext
\usepackage{xargs}                      % Use more than one optional parameter in a new commands
% 
\usepackage[colorinlistoftodos,prependcaption,textsize=tiny]{todonotes}
\newcommandx{\unsure}[2][1=]{\todo[linecolor=red,backgroundcolor=red!25,bordercolor=red,#1]{#2}}
\newcommandx{\change}[2][1=]{\todo[linecolor=blue,backgroundcolor=blue!25,bordercolor=blue,#1]{#2}}
\newcommandx{\info}[2][1=]{\todo[linecolor=OliveGreen,backgroundcolor=OliveGreen!25,bordercolor=OliveGreen,#1]{#2}}
\newcommandx{\improvement}[2][1=]{\todo[linecolor=Plum,backgroundcolor=Plum!25,bordercolor=Plum,#1]{#2}}
\newcommandx{\thiswillnotshow}[2][1=]{\todo[disable,#1]{#2}}
%

\usepackage[simplified]{pgf-umlcd}

\title{SUZAKU开发者手册}
\author{董冠军}
\date{\today}

\begin{document}

\maketitle
\tableofcontents

\listoftodos[Notes]

\chapter{导言}

\section{导言}

研判形势,淬炼心法,有所为,有所不为,乃至无为而无不为。

修道而保法,故能为胜败之政。

\begin{shadequote}

    道生一,一生二,二生三,三生万物。\\
    道生之,德蓄之,物形之,势成之。
\end{shadequote}

精一之学,体用兼备。
\begin{shadequote}

    天地之道,可一言而尽也:其为物不二,则其生物不测。\\
    天下之动,贞夫一者也。\\
    圣人抱一以为天下式。\\
    恒以一德。
\end{shadequote}

太极哲学,双线法则,圆点哲学,一分为三,提供了诸多值得反复体味的命题。

一,切己言之,就是事业,须更上一层楼。一是整体,是根据地,是不间断,也是突破点。

博厚,高明,悠久。

空灵之境,有无相生,有生于无。空非空寂,众缘所起,云行雨施,品物流行。

太极本无极。

上溯,万法归一,一归空。

五轮书,地水火风空。

建立自我,追求无我,是逆向工程。下学而上达。

\section{战略,或道}

\section{方法谈}

爱因斯坦说过这句话:我们不能用制造问题时同一水平的思维来解决问题。也许他意味着我们需要摆脱与我们对一个问题有关的消极的看法。如果我们对问题本身太投入,那么我们永远无法越过这个局面。

在一本叫“治愈与复原”中,David R. Hawkins详细阐述了这一点。他说,“问题最好不要在他们发生的同一水平上解决,而是在他们的上一个阶级上解决...通过超越他们,从更高的角度看待问题,问题很容易迎刃而解。
较高层次上,由于这种观点的转变,问题会自动解决,否则人们可能会看不到任何的问题。”

很多时候,我们面对一个问题时,总会把精力集中在问题上,一直问怎么“解决”呢?我们可能最终会走入死角,沮丧。
因为我们似乎找不到很好的解决办法。无论如何,不要把精力集中在问题本身上。花几分钟时间,花费你的时间和精力来正面地解析。
我们无法控制经常会有事情出现的,不要浪费时间担心这些事情;只花时间在你可以改变或控制的事情上。

\subsection{中庸}

\subsection{圆点哲学}
\subsection{双线法则}
\subsection{黄金分割率}
\subsection{80/20规则}
\subsection{黄金圈法则}

\subsection{达里奥的原则}

欲达到我们的目标,必须实事求是,客观公正地面对现实,正视自身的缺点和不足,而有以克服之。

这是真的吗?求真是第一位的,吾爱吾师,吾更爱真理。对道听途说的观念,我们固然要保持警觉和必要的批判精神。
对自我意识,也要慎思明辨。保持开放之心和专注之念,对自己的观念做压力测试,力求准确更准确。
而不能陷入先入之见,或自欺欺人,没有荣辱,只有是非。不当的虚荣心和自尊心会妨碍通向真正的目标。

对我们不知之物,保持谦卑,保持饥饿,保持愚蠢。

选择至关重要,我们必须承担选择的后果,为选择负起责任。
弱点,由弱点导致错误,皆在所难免。
但由此错误,吃一堑长一智,如果能通过反思而增强了自己,就是有益的。
从错误中学习,进步,进化,是通向成功的捷径。

任一选择,都带来其效应和影响。一阶效应也许不错,但二三阶效应可能已变形,
祸福相依,需要更多的洞见。关键的选择,决定了我们人生的质量。

成长,或曰进化,是唯一的目的。财富,名利皆是果,而不是因。
当我们围绕成长,而动心忍性,增益其所不能的时候,就是走在自我进化的路上。

自我进化,有一五步法可资遵循:
\begin{enumbox}
\item 设定清晰的目标
\item 觉知问题
\item 诊断问题
\item 设计方案
\item 执行方案
\end{enumbox}

五步法是迭代过程。每一步都需要投入必要的资源,做选择,做策划。

对比目标和输出的不同,找到不足,做出适当的调整,类似于PDCD。不妨想象,有台巨大的机器,作为输入输出的中介。
我们的核心任务,就是维持机器的良好运行和高效产出。

资源调度,采取开放的视角,并非一定需要我们亲力亲为。
我,即是设计者,也是执行者,主要作为设计者而存在。
任何人都非全知全能,而是有长有短,管理者的职责,在于知人善任。

不必为自己的弱点而沮丧,君子性非异也,善假于物。

唯一的目的,就是自我进化。唯一的事,就是打造机器。
机器是我们拥有的容器,即是心法,也是产品。我们是机器的架构师。

单纯观念,不足以动人。做出作品,持续产出,才能实现自我价值,立于不败之地。

实有诸己之谓德。默默地完成进化,是最明智的选择。围绕选定的一,厚积薄发,静水深流。

道生一,一即是战略,也是方法。

原则,架构起了价值和行动的桥梁。让我们有所遵循,持续积累,而不是茫然无措,本末倒置。

佛陀的教导

以戒为师。戒可释为原则,或良好习惯。

大乘起信论的一心二门的义理架构,予人深刻启示。

达里奥与王阳明

良知是比原则更基础的范畴,良知是一种元认知能力。
致吾心良知于事事物物,则事事物物皆得其理。

达里奥求真的意志和可操作性,较阳明为突出。
资本主义的熏陶,更适应于现实人生。

毛主席在其著作中,深入分析了认识的各种问题,如主观主义,教条主义,经验主义,
统称为主观主义,即主客观的分裂和不一致。以此指导行动,则误导行动。

马利克的管理学,采用系统论,控制论和仿生学等知识,以应对现实世界的复杂性。


\chapter{设计}

设计在解决关键问题的同时,要降低实现,测试和维护的复杂度。

加强测试,通过重构降低复杂度。

分解问题,界定边界,降低复杂度

设计的基本原则

分离机制和策略,接口和实现

性能依赖于设计,在一定的设计下,取决于实现。

性能优化手段:并行,聚合,缓存等,根本在于设计,控制复杂度。

识别实体和关系,ERD,DFD等,FSM是机器语言。

核心概念:
\begin{compactenum}
\item 存储池/目录
\item 卷
\item 快照
\item 主机映射
\end{compactenum}

core thread边界,core\_request进入。

分布式副本一致性:clock版本机制,msgqueue离线消息处理。

性能:并发,聚合和cache等

元数据管理,非计算而来

快照树的实现

后台任务统一管理,包括:
\begin{compactenum}
\item recovery
\item balance
\item vol rm
\item snap rm
\item snap rollback
\item snap flat
\end{compactenum}

架构问题:
\begin{compactenum}
\item 元数据管理成本
\item 支持大容量卷
\item 支持ROW快照树
\item 诊断流程和工具
\item 性能profile
\end{compactenum}

\section{故障域}

对任一存储池,设故障域数为M,副本数为N,

当M>=N时,每个故障域内一个副本,随机分布;
当M<N时,
- 策略1,每个故障域内一个副本
- 策略2a,剩余的副本按策略1进行,直到写完所有副本数
- 策略2b,不写剩余副本

按策略2a:

case 1:故障域为2,副本数为3,则副本在故障域的分布为(2,1)或(1,2)
case 2:故障域为1,副本数为3,则副本在故障域的分布为 3

按策略2b:

case 1:故障域为2,副本数为3,则副本在故障域的分布为(1,1)
case 2:故障域为1,副本数为3,则副本在故障域的分布为(1) (副本数不能少于2个,分配失败)

同时,恢复过程须按以上故障域规则进行自动校正!!!

\section{存储池状态}

\begin{compactenum}
\item 不可用
\item 磁盘空间不足/READ ONLY
\item 降级
\item 正常/健康
\end{compactenum}

\section{诊断方法}

对需要改进的流程进行区分,找到最有潜力的改进机会,优先对需要改进的流程实施改进。如果不确定优先次序,企业多方面出手,就可能分散精力,
影响6σ管理的实施效果。业务流程改进遵循五步循环改进法,即DMAIC模式:

\begin{compactenum}
\item 定义[Define]——辨认需改进的产品或过程,确定项目所需的资源。
\item 测量[Measure]——定义缺陷,收集此产品或过程的表现作底线,建立改进目标。
\item 分析[Analyze]——分析在测量阶段所收集的数据,以确定一组按重要程度排列的影响质量的变量。
\item 改进[Improve]——优化解决方案,并确认该方案能够满足或超过项目质量改进目标。
\item 控制[Control]——确保过程改进一旦完成能继续保持下去,而不会返回到先前的状态。
\end{compactenum}

信息有多级:USE。诊断问题依赖于结构化的诊断方法PAT,解决问题也是,构建知识图谱。

欲分析问题,必分析事物发展的完整过程,包括每个参与者的生命周期模型,参与者之间的相互作用。

\section{检查清单checklist}

先宏观,后微观,致广大而尽精微

\begin{compactenum}
\item 集群健康情况
\item 数据一致性检查
\item 硬件
    \begin{compactenum}
    \item 磁盘
    \item 网络
    \item 内存
    \item CPU
    \item 操作系统
    \end{compactenum}
\item LICH
    \begin{compactenum}
    \item 后台任务,包括恢复,删除,快照后台任务等
    \item 日志
    \item core
    \end{compactenum}
\end{compactenum}

\chapter{代码}

集中兵力,各个击破

管理和技术,管理,不仅是运营管理,还有技术管理

可重用性

可测试性

\section{Reading Code}

RDMA从poll开始,深度优先的遍历策略。收到消息后,按消息类型派遣到不同的handler去处理。

\begin{enumbox}
\item 每个组件有rpc,导出接口
\end{enumbox}

结构
\begin{enumbox}
\item 命名规则
\end{enumbox}

函数
\begin{enumbox}
\item 行数
\end{enumbox}

MM
\begin{enumbox}
\item buffer\_t
\item coroutine stack
\item 小对象
\end{enumbox}

\section{Debug}

调试代码,要跟踪backtrace,要跟踪消息流向,即消息的生命周期,要比对时间线。
\begin{myeasylist}{itemize}
& module
& assert
& log
&& message flow
&& timeline
&& backtrace
\end{myeasylist}

\hl{按时间线trace消息流向}是强有力的跟踪法。

\chapter{测试}

\lstset{numbers=left,
    frame=shadowbox,
    numberstyle= \tiny,
    keywordstyle= \color{ blue!70},commentstyle=\color{red!50!green!50!blue!50}, 
    rulesepcolor= \color{ red!20!green!20!blue!20} 
}

\section{已知问题}

\begin{enumbox}
\item vol resize会产生死锁
\item vol copy的提示
\item flat后保护快照
\end{enumbox}

\section{部署}

基本步骤:
\begin{enumbox}
\item 创建集群
\item 创建存储池
\item 向存储池添加磁盘(Tier, SSD Cache)
\item 创建卷
\item 创建快照
\end{enumbox}

\subsection{创建集群}

\begin{lstlisting}[language=bash]
lich prep t151 t152 t153
lich create t151 t152 t153
\end{lstlisting}

\hl{注意事项}:
\begin{compactenum}
\item 检查IP是否重复
\item 检查子网mask是否匹配
\item ...
\end{compactenum}

\subsection{创建存储池}

\begin{lstlisting}[language=bash]
lichbd pool create p1
\end{lstlisting}

\subsection{向存储池添加磁盘}

\begin{lstlisting}[language=bash]
lich.node --disk_add all --force --pool p1
\end{lstlisting}

\hl{注意事项}:
\begin{compactenum}
\item 存储池内每个节点上需要有SSD,支持tier功能
\item 存储池内每个节点上需要有SSD,支持SSD cache功能
\end{compactenum}

\subsection{创建卷}

\begin{lstlisting}[language=bash]
# 卷路径规范:<pool>/<protocol>/<volume>
# 三副本
# row2格式
lichbd vol create p1/iscsi/v1 --size 4096Gi --repnum 3 -F row2
lich.inspect --localize /iscsi/v1 0 --pool p1
\end{lstlisting}

\hl{注意事项}:
\begin{compactenum}
\item 卷格式:row2 or raw (default)
\item 三副本 (default: 2)
\item 关闭localize
\end{compactenum}

\subsection{创建快照}

\begin{lstlisting}[language=bash]
# 快照路径规范:<pool>/<protocol>/<volume>@<snap>
lichbd snap create p1/iscsi/v1@snap1
\end{lstlisting}

\section{工具}

省略...

\begin{lstlisting}[language=bash]
iscsiadm -m discovery -t st -p 192.168.251.202
\end{lstlisting}

\section{故障测试}

每类节点故障行为不同。除选举过程外,还有vip,iscsi连接,controller的切换,lease,io,恢复过程等。
评价可靠性的指标,主要是vdbench测试中,各种故障条件下io无中断。

另外,故障点还会破坏事务执行的原子性,如allocte过程,创建snapshot过程,
导致严重后果,如造成垃圾,数据状态不一致。如何通过可重入性,或事务解决此类问题?

快照的rollback,delete,flat都设计为可重入过程。如果任务执行失败,可以重新调度。
各种持久化状态之间,保持一致性。

\subsection{单磁盘故障}

磁盘有两种角色:数据盘和cache盘。拔cache盘等同于节点故障?

\subsection{节点故障}

节点有多种角色:
\begin{compactenum}
\item etcd master
\item lich admin
\item lich normal
\end{compactenum}

受VIP机制影响,arp协议会影响客户端到iscsi target的网络连接。
需要注意的是,大部分网络会禁用arp广播,单播则可以。

控制器的加载,lease获取等需要一定时间。

\chapter{交付}


\end{document}

\chapter{软件架构}

meta管理采用对称的中心化架构。在节点中选举出admin节点,
管理全局状态和数据分配工作。当admin节点发生故障时,会发生
failover过程,选举出新的admin节点来。

lichd进程内嵌各种server,包括iscsi等。

作为存储系统,主要考虑是元数据组织和IO,恢复等关键过程。
功能之外,性能,可靠性,可扩展性至关重要。

client可以和每个节点进行通信,推荐采用VIP机制,简化连接管理。

\section{VIP}

\section{分布}

数据分布首先要满足规则要求,其次则需做到均衡和局部性。
规则是强制的,极端条件下可能退化。均衡和局部性则影响系统性能。

数据分布规则有:
\begin{itemize}
\item 保护域规则
\item 存储池规则
\item 故障域规则(\ref{rule:faultset})
\end{itemize}

负载均衡和本地化两方面考虑,平衡包括数据平衡和任务平衡。

chunk在节点上的分布,节点内chunk在磁盘上的分布(包括分层)

controller在节点上的分布,controller在core上的分布

进一步,需要考虑保护域,存储池,故障域各自的作用和特性。

\section{复制}

\begin{tabular}{|s|p{6cm}|}
    %\hline
    %\rowcolor{lightgray} \multicolumn{5}{|c|} {Snapshot} \\
    \hline
    pool & LICH\_REPLICA\_METADATA  \\
    \hline
    subpool & LICH\_REPLICA\_METADATA  \\
    \hline
    vol & fileinfo \\
    \hline
    subvol & fileinfo  \\
    \hline
    raw & fileinfo  \\
    \hline
\end{tabular}

恢复和再平衡过程,是怎么工作的?

\section{分配和回收}

\subsection{分配一个chunk的过程}

函数:
\begin{itemize}
    \item \verb|__table2_chunk_create|
    \item \verb|replica_srv_create|
    \item \verb|disk_create|
\end{itemize}

与admin交互,返回节点列表,即各副本所在节点。

需要持久化的信息:
\begin{itemize}
    \item table2 meta,记录chunk info(副本位置)
    \item sqlite3,记录chkid(副本)到物理地址的映射关系
    \item disk bitmap,记录磁盘上每个chunk的分配状态
    \item 填充chunk内容为全0?
\end{itemize}

分配chunk的过程,会影响到若干特性,如精简配置,恢复,再平衡,写入等,都产生新chunk。

优化allocate的性能:
\begin{itemize}
    \item 异步化sqlite,每个db一个工作线程
    \item 加大lich.inspect线程数到20
    \item table2 lock粒度 \change{dynamical lock table}
\end{itemize}

\subsection{磁盘管理}

每块磁盘对应一个bitmap,用于该盘的空间管理。

磁盘有分层属性,通常0表示SSD,1表示HDD。有三种分层策略:tier==0,表示写入SSD,tier==1表示写入HDD,
tier==-1表示自动分层,先写入SSD,通过异步后台线程flush不活跃的数据到HDD。

在分配每一个chunk的时候,可以指定tier。没有指定的情况下,默认为卷的priority设定。

chunk\_id到磁盘物理地址的映射,是一随机过程,定位到空闲的bitmap上。

如何确定磁盘的分层?

RAID管理,disk和raid都有cache,需要注意掉电情况下是否丢数据。

\section{IO过程}

写过程,可能内在地包含了分配chunk的过程,缺页分配。当在末尾写入时,还可能扩展了卷的大小。

大范围内的随机写入,造成很多的缺页分配,分配过程会成为性能瓶颈。

\section{负载均衡}

\section{本地化}

卷控制器所在节点,具有所有chunk的副本。

当切换控制器的时候,需要控制本地化过程的QoS。

\section{分层}

\section{SSD缓存}

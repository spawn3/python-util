\chapter{实现相关}

\begin{itemize}
    \item polling \change{polling}
    \item coroutine and scheduler
    \item mbuffer
    \item kernel bypass
\end{itemize}

\section{Safe Mode}

卷级进行检查,处在保护模式的卷,不允许iscsi连接,返回错误码。

加载时间较长的模块,用half sync/half async模式来处理。分为两阶段:同步+异步。

一个卷,加载成功,依赖于几个条件:
\begin{itemize}
    \item raw/lsv
    \item module load
\end{itemize}

\section{Coroutine}

使用规则:
\begin{itemize}
    \item \verb|schedule_task_get|和\verb|schedule_yield|要匹配
    \item task有数量上的限制:1024,超出后容易引起死锁
\end{itemize}

\section{Memory Management}

\subsection{ymalloc}

\subsection{mbuffer}

\subsection{huge page}

2M

\section{AIO}

生产者-消费者模型。

\section{sqlite3}

异步sqlite3。

\begin{itemize}
    \item 共10个db,每个一个线程,每个线程管理一个队列
    \item 所有sqlite3操作,泛化为统一的结构,放入线程队列
    \item 消费者线程批量处理队列中的任务
    \item 生产者线程和消费者线程通过sem进行通信
    \item 采用协程机制(yield/resume)同步任务执行顺序
\end{itemize}

消费者线程wait在sem上,生产者线程有消息的时候,调用\verb|sem_post|。

\section{RPC}

\begin{itemize}
    \item minirpc
    \item rpc
    \item corerpc
\end{itemize}

\section{Algorithm}

\begin{itemize}
    \item paxos/raft 选举admin和meta节点
    \item vector clock 副本一致性
    \item lease controller的唯一性
\end{itemize}

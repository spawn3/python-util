\chapter{阴符经}

\section{原文}

观天之道,执天之行,尽矣。
故天有五贼,见之者昌。
五贼在乎心,施行于天。宇宙在乎手,万化生乎身。
天性,人也;人心,机也。立天之道,以定人也。
天发杀机,移星易宿;地发杀机,龙蛇起陆;人发杀机,天地反覆;天人合发,万变定基。
性有巧拙,可以伏藏。九窍之邪,在乎三要,可以动静。
火生于木,祸发必克;奸生于国,时动必溃。知之修炼,谓之圣人。

天生天杀,道之理也。
天地,万物之盗;万物,人之盗;人,万物之盗。
三盗既宜,三才既安。故曰:食其时,百骸理;动其机,万化安。
人知其神而神,不知其不神之所以神也。
日月有数,大小有定,圣功生焉,神明出焉。
其盗机也,天下莫能见,莫能知也。君子得之固躬,小人得之轻命。 

瞽者善听,聋者善视。绝利一源,用师十倍。三返昼夜,用师万倍。
心生于物,死于物,机在于目。
天之无恩而大恩生。迅雷烈风,莫不蠢然。
至乐性余,至静性廉。天之至私,用之至公。禽之制在炁。
生者死之根,死者生之根。恩生于害,害生于恩。
愚人以天地文理圣,我以时物文理哲。人以愚虞圣,我以不愚虞圣;人以奇期圣,我以不奇期圣。
故曰:沉水入火,自取灭亡。
自然之道静,故天地万物生。
天地之道浸,故阴阳胜。
阴阳相推,而变化顺矣。是故圣人知自然之道不可违,因而制之至静之道。律历所不能契。
爰有奇器,是生万象,八卦甲子,神机鬼藏。阴阳相胜之术,昭昭乎进于象矣。 

\section{释义}

以道心物三合之道来诠释。

观天之道:由心出发,体察天地之道,而后可以循道而行,此为道知,道尽为学处世之道。

天非茫茫之天,内蕴五行,能体察五行之气运,则可以昌盛。

心为能动的一方,以心受道体道,就可以立其环中,以应无穷,包括领导统御之术。

天人合发,万变定基:心与道合、与天合,这是做一切事的根基。

明了五行生克的结构与动态关系,进而内化于心,正心诚意,可称为圣人。

一事或成或败,皆有道理蕴含其中。天地-人-万物合乎三合之道,尽心知性则知天矣,格物致知穷理,
尊德性而道问学,此三者相生相克,转圆而求其合。藏器于身待时而动,则万事如意,臻于中道。

绝利一源,一者何?道也,进乎技也。三返昼夜,循环至三,如昼夜交替,运行不废。
其功效甚大,有事半功倍之效果。一不能理解为具体的事,如此则器,君子不器,本立道生。
若心能体道,秉道御物,乘物游心,则三合之道可以大成。以道控势,顺势而为,与道浮沉。

口目耳,此身之三要,心能制之。微信控,游戏控,则失心之所以为主,惑矣。

气韵生动

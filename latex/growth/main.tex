% -*- coding: UTF-8 -*-
% hello.tex

\documentclass[UTF8]{ctexart}

\usepackage{xeCJK}
\usepackage[utf8]{inputenc}

\usepackage{hyperref}
\hypersetup{pdftex,colorlinks=true,allcolors=blue}
\usepackage{hypcap}

\usepackage{color}
\usepackage[usenames, dvipsnames, svgnames, table]{xcolor}
% \pagecolor{gray}

\usepackage{makeidx}
\makeindex

\usepackage{amsmath}
\usepackage{mathtools}

\usepackage{listings}
\usepackage{multicol}
\usepackage{fancybox}
\usepackage{tcolorbox}
\usepackage{enumitem}

\usepackage{indentfirst}

% table
\setlength{\arrayrulewidth}{1pt}
\setlength{\tabcolsep}{16pt}
\renewcommand{\arraystretch}{2.5}
\newcolumntype{s}{>{\columncolor[HTML]{AAACED}} p{3cm}}

\arrayrulecolor[HTML]{DB5800}

% 摘录
\usepackage{verbatim}
\usepackage{libertine}
\usepackage{graphicx}
\usepackage{framed}

\newcommand*\openquote{\makebox(25,-22){\scalebox{5}{``}}}
\newcommand*\closequote{\makebox(25,-22){\scalebox{5}{''}}}
\colorlet{shadecolor}{Azure}

\makeatletter
\newif\if@right
\def\shadequote{\@righttrue\shadequote@i}
\def\shadequote@i{\begin{snugshade}\begin{quote}\openquote}
\def\endshadequote{%
\if@right\hfill\fi\closequote\end{quote}\end{snugshade}}
\@namedef{shadequote*}{\@rightfalse\shadequote@i}
\@namedef{endshadequote*}{\endshadequote}
\makeatother


\title{成长}
\author{董冠军}
\date{\today}

% \bibliographystyle{plain}
% \bibliography{math}

\begin{document}

\maketitle
\tableofcontents

\section{导言}

研判形势,淬炼心法,有所为,有所不为,乃至无为而无不为。

\begin{shadequote}

    道生一,一生二,二生三,三生万物。\\
    道生之,德蓄之,物形之,势成之。
\end{shadequote}

精一之学,体用兼备。
\begin{shadequote}

    天地之道,可一言而尽也:其为物不二,则其生物不测。\\
    天下之动,贞夫一者也。\\
    圣人抱一以为天下式。\\
    恒以一德。
\end{shadequote}

太极哲学,双线法则,圆点哲学,一分为三,提供了诸多值得反复体味的命题。

一,切己言之,就是事业,须更上一层楼。一是整体,是根据地,是不间断,也是突破点。

博厚,高明,悠久。

空灵之境,有无相生,有生于无。空非空寂,众缘所起,云行雨施,品物流行。

太极本无极。

上溯,万法归一,一归空。

五轮书,地水火风空。

建立自我,追求无我,是逆向工程。下学而上达。

\section{战略,或道}

\section{方法谈}

爱因斯坦说过这句话:我们不能用制造问题时同一水平的思维来解决问题。也许他意味着我们需要摆脱与我们对一个问题有关的消极的看法。如果我们对问题本身太投入,那么我们永远无法越过这个局面。

在一本叫“治愈与复原”中,David R. Hawkins详细阐述了这一点。他说,“问题最好不要在他们发生的同一水平上解决,而是在他们的上一个阶级上解决...通过超越他们,从更高的角度看待问题,问题很容易迎刃而解。
较高层次上,由于这种观点的转变,问题会自动解决,否则人们可能会看不到任何的问题。”

很多时候,我们面对一个问题时,总会把精力集中在问题上,一直问怎么“解决”呢?我们可能最终会走入死角,沮丧。
因为我们似乎找不到很好的解决办法。无论如何,不要把精力集中在问题本身上。花几分钟时间,花费你的时间和精力来正面地解析。
我们无法控制经常会有事情出现的,不要浪费时间担心这些事情;只花时间在你可以改变或控制的事情上。

\subsection{80/20规则}
\subsection{双线法则}
\subsection{黄金圈法则}

\section{产品的世界}

\end{document}

% -*- coding: UTF-8 -*-
% hello.tex

\documentclass[UTF8]{ctexbook}

\usepackage{xeCJK}
\usepackage[utf8]{inputenc}

% load paralist before enumitem
\usepackage{paralist}

\usepackage{hyperref}
\hypersetup{pdftex,colorlinks=true,allcolors=blue}
\usepackage{hypcap}

\usepackage{color}
\usepackage[usenames, dvipsnames, svgnames, table]{xcolor}
% \pagecolor{gray}

\usepackage{makeidx}
\makeindex

\usepackage{amsmath}
\usepackage{mathtools}

\usepackage{listings}
\usepackage{multicol}
\usepackage{fancybox}
\usepackage{tcolorbox}
\usepackage{enumitem}
\usepackage{multirow}
\usepackage{longtable}

\usepackage{indentfirst}

% table
\setlength{\arrayrulewidth}{1pt}
\setlength{\tabcolsep}{16pt}
\renewcommand{\arraystretch}{2.5}
\newcolumntype{s}{>{\columncolor[HTML]{AAACED}} p{3cm}}

\arrayrulecolor[HTML]{DB5800}

% 摘录
\usepackage{verbatim}
\usepackage{libertine}
\usepackage{graphicx}
\usepackage{framed}

\lstset{%
    %alsolanguage=Java,
    %language={[ISO]C++}, %language为,还有{[Visual]C++}
    %alsolanguage=[ANSI]C, %可以添加很多个alsolanguage,如alsolanguage=matlab,alsolanguage=VHDL等
    %alsolanguage=tcl,
    %alsolanguage=XML,
    %alsolanguage=bash,
    tabsize=4, %
    frame=shadowbox, %把代码用带有阴影的框圈起来
    commentstyle=\color{red!50!green!50!blue!50},%浅灰色的注释
    rulesepcolor=\color{red!20!green!20!blue!20},%代码块边框为淡青色
    keywordstyle=\color{blue!90}\bfseries, %代码关键字的颜色为蓝色,粗体
    showstringspaces=false,%不显示代码字符串中间的空格标记
    stringstyle=\ttfamily, % 代码字符串的特殊格式
    keepspaces=true, %
    breakindent=22pt, %
    numbers=left,%左侧显示行号 往左靠,还可以为right,或none,即不加行号
    stepnumber=1,%若设置为2,则显示行号为1,3,5,即stepnumber为公差,默认stepnumber=1
    %numberstyle=\tiny, %行号字体用小号
    numberstyle={\color[RGB]{0,192,192}\tiny} ,%设置行号的大小,大小有tiny,scriptsize,footnotesize,small,normalsize,large等
    numbersep=8pt, %设置行号与代码的距离,默认是5pt
    basicstyle=\footnotesize, % 这句设置代码的大小
    showspaces=false, %
    flexiblecolumns=true, %
    breaklines=true, %对过长的代码自动换行
    breakautoindent=true,%
    breakindent=4em, %
    escapebegin=\begin{CJK*}{GBK}{hei},escapeend=\end{CJK*},
    aboveskip=1em, %代码块边框
    tabsize=2,
    showstringspaces=false, %不显示字符串中的空格
    backgroundcolor=\color[RGB]{245,245,244}, %代码背景色
    %backgroundcolor=\color[rgb]{0.91,0.91,0.91} %添加背景色
    escapeinside=``, %在``里显示中文
    %% added by http://bbs.ctex.org/viewthread.php?tid=53451
    fontadjust,
    captionpos=t,
    framextopmargin=2pt,framexbottommargin=2pt,abovecaptionskip=-3pt,belowcaptionskip=3pt,
    xleftmargin=4em,xrightmargin=4em, % 设定listing左右的空白
    texcl=true,
    % 设定中文冲突,断行,列模式,数学环境输入,listing数字的样式
    extendedchars=false,columns=flexible,mathescape=false
    % numbersep=-1em
}


\newenvironment{enumbox}[0]{
    \begin{tcolorbox}
    \begin{compactenum}
} {
    \end{compactenum}
    \end{tcolorbox}
}

\newenvironment{itembox}[0]{
    \begin{tcolorbox}
    \begin{compactitem}
} {
    \end{compactitem}
    \end{tcolorbox}
}

\newcommand*\openquote{\makebox(25,-22){\scalebox{5}{``}}}
\newcommand*\closequote{\makebox(25,-22){\scalebox{5}{''}}}
\colorlet{shadecolor}{Azure}

\makeatletter
\newif\if@right
\def\shadequote{\@righttrue\shadequote@i}
\def\shadequote@i{\begin{snugshade}\begin{quote}\openquote}
\def\endshadequote{%
\if@right\hfill\fi\closequote\end{quote}\end{snugshade}}
\@namedef{shadequote*}{\@rightfalse\shadequote@i}
\@namedef{endshadequote*}{\endshadequote}
\makeatother


\title{成长}
\author{炼金术士}
\date{\today}

% \bibliographystyle{plain}
% \bibliography{math}

\begin{document}

\maketitle
\tableofcontents

\chapter{导言}

\section{导言}

研判形势,淬炼心法,有所为,有所不为,乃至无为而无不为。

修道而保法,故能为胜败之政。

\begin{shadequote}

    道生一,一生二,二生三,三生万物。\\
    道生之,德蓄之,物形之,势成之。
\end{shadequote}

精一之学,体用兼备。
\begin{shadequote}

    天地之道,可一言而尽也:其为物不二,则其生物不测。\\
    天下之动,贞夫一者也。\\
    圣人抱一以为天下式。\\
    恒以一德。
\end{shadequote}

太极哲学,双线法则,圆点哲学,一分为三,提供了诸多值得反复体味的命题。

一,切己言之,就是事业,须更上一层楼。一是整体,是根据地,是不间断,也是突破点。

博厚,高明,悠久。

空灵之境,有无相生,有生于无。空非空寂,众缘所起,云行雨施,品物流行。

太极本无极。

上溯,万法归一,一归空。

五轮书,地水火风空。

建立自我,追求无我,是逆向工程。下学而上达。

\section{战略,或道}

\section{方法谈}

爱因斯坦说过这句话:我们不能用制造问题时同一水平的思维来解决问题。也许他意味着我们需要摆脱与我们对一个问题有关的消极的看法。如果我们对问题本身太投入,那么我们永远无法越过这个局面。

在一本叫“治愈与复原”中,David R. Hawkins详细阐述了这一点。他说,“问题最好不要在他们发生的同一水平上解决,而是在他们的上一个阶级上解决...通过超越他们,从更高的角度看待问题,问题很容易迎刃而解。
较高层次上,由于这种观点的转变,问题会自动解决,否则人们可能会看不到任何的问题。”

很多时候,我们面对一个问题时,总会把精力集中在问题上,一直问怎么“解决”呢?我们可能最终会走入死角,沮丧。
因为我们似乎找不到很好的解决办法。无论如何,不要把精力集中在问题本身上。花几分钟时间,花费你的时间和精力来正面地解析。
我们无法控制经常会有事情出现的,不要浪费时间担心这些事情;只花时间在你可以改变或控制的事情上。

\subsection{中庸}

\subsection{圆点哲学}
\subsection{双线法则}
\subsection{黄金分割率}
\subsection{80/20规则}
\subsection{黄金圈法则}

\subsection{达里奥的原则}

欲达到我们的目标,必须实事求是,客观公正地面对现实,正视自身的缺点和不足,而有以克服之。

这是真的吗?求真是第一位的,吾爱吾师,吾更爱真理。对道听途说的观念,我们固然要保持警觉和必要的批判精神。
对自我意识,也要慎思明辨。保持开放之心和专注之念,对自己的观念做压力测试,力求准确更准确。
而不能陷入先入之见,或自欺欺人,没有荣辱,只有是非。不当的虚荣心和自尊心会妨碍通向真正的目标。

对我们不知之物,保持谦卑,保持饥饿,保持愚蠢。

选择至关重要,我们必须承担选择的后果,为选择负起责任。
弱点,由弱点导致错误,皆在所难免。
但由此错误,吃一堑长一智,如果能通过反思而增强了自己,就是有益的。
从错误中学习,进步,进化,是通向成功的捷径。

任一选择,都带来其效应和影响。一阶效应也许不错,但二三阶效应可能已变形,
祸福相依,需要更多的洞见。关键的选择,决定了我们人生的质量。

成长,或曰进化,是唯一的目的。财富,名利皆是果,而不是因。
当我们围绕成长,而动心忍性,增益其所不能的时候,就是走在自我进化的路上。

自我进化,有一五步法可资遵循:
\begin{enumbox}
\item 设定清晰的目标
\item 觉知问题
\item 诊断问题
\item 设计方案
\item 执行方案
\end{enumbox}

五步法是迭代过程。每一步都需要投入必要的资源,做选择,做策划。

对比目标和输出的不同,找到不足,做出适当的调整,类似于PDCD。不妨想象,有台巨大的机器,作为输入输出的中介。
我们的核心任务,就是维持机器的良好运行和高效产出。

资源调度,采取开放的视角,并非一定需要我们亲力亲为。
我,即是设计者,也是执行者,主要作为设计者而存在。
任何人都非全知全能,而是有长有短,管理者的职责,在于知人善任。

不必为自己的弱点而沮丧,君子性非异也,善假于物。

唯一的目的,就是自我进化。唯一的事,就是打造机器。
机器是我们拥有的容器,即是心法,也是产品。我们是机器的架构师。

单纯观念,不足以动人。做出作品,持续产出,才能实现自我价值,立于不败之地。

实有诸己之谓德。默默地完成进化,是最明智的选择。围绕选定的一,厚积薄发,静水深流。

道生一,一即是战略,也是方法。

原则,架构起了价值和行动的桥梁。让我们有所遵循,持续积累,而不是茫然无措,本末倒置。

佛陀的教导

以戒为师。戒可释为原则,或良好习惯。

大乘起信论的一心二门的义理架构,予人深刻启示。

达里奥与王阳明

良知是比原则更基础的范畴,良知是一种元认知能力。
致吾心良知于事事物物,则事事物物皆得其理。

达里奥求真的意志和可操作性,较阳明为突出。
资本主义的熏陶,更适应于现实人生。

毛主席在其著作中,深入分析了认识的各种问题,如主观主义,教条主义,经验主义,
统称为主观主义,即主客观的分裂和不一致。以此指导行动,则误导行动。

马利克的管理学,采用系统论,控制论和仿生学等知识,以应对现实世界的复杂性。


\chapter{算法}

\section{排序算法}

\subsection{Insertion Sort}

\subsection{Shell Sort}

\subsection{Select Sort}

\subsection{Heap Sort}

\subsection{Bubble Sort}

\subsection{Quick Sort}

\subsection{Merge Sort}

归并,多路归并,外排序

\subsection{Counting Sort}

\subsection{Bucket Sort}

\subsection{Radix Sort}

\section{查找算法}

\subsection{二分查找}

\subsection{TOP N}

\chapter{原则}

原则近似于我之所谓道,人何以之道?曰:心。心何以知?曰:虚一而静。
精于道者兼物物,一于道而以赞稽之,则万物官矣。

\section{尊道}

道德仁义礼,五者一体也,而道为之主,故第0原则即是尊道。
然何以尊之?需明法。

老子七善,三学六度

\subsection{至诚不息}

君子养心莫善乎诚,致诚则无它事也。

实事求是,拥抱现实,超越现实。

\subsection{虚壹而静}

是大原则,每临大事有静气,不信今时无古贤。

心善渊

将军之事,静以幽,正以治。

\subsection{圆点哲学}

最小最大模型,立其环中,以应无穷。

\subsection{双线法则}

\subsection{123哲学}

\subsection{机器之喻}

欲收无为而治之效,不能不着重在打磨机器、系统上,建立系统思维。
自组织、自进化的系统是工作的产物。

用系统来工作

\section{工作之道}

以终为始

要事第一

全局优化(统合)

\section{生活之道}

闲居静思则通

\chapter{战略}

专业学习之外,把战略研究列为今后几年的重点。

更喜欢演绎法,如欧几里得几何的那种公理化方法,斯宾诺莎的伦理学,达里奥的原则,金岳霖的论道都采用了该方法。
道生一,一生二,二生三,三生万物。一为取势,二为明道和优术,1+2=3,三者匹配,可运用于每一领域。


为什么?

因为战略很重要,战略赢是大赢,战略输是大输。孙子:兵者,国之大事,死生之地,存亡之道,不可不察也。
不具备强大的战略思维能力,就很难实现远期的发展目标。

多聚变为一,一裂变为多,分合为变。以正治国,以奇用兵,以无事取天下。战略明,则可无事。否则,陷入事务之中,而无结果,可悲。

欲研习战略,需读经典,多实践,请教高人,开阔视野,放大格局。在解决现实问题中,融会贯通,知行合一。持续优化,不断把认知引向深入。
道德经,孙子兵法,商君书等,皆为经典。西方亦有经典,然不够精炼。待心有所主,则可进一步泛观博览。第一步,则在知止,懂取舍,有所不为。

战略罗盘之喻,精当。战略几何学,形象生动。柏拉图言:不懂几何者不得入内。以几何去研习战略,可收简洁精当,生动形象之效。

专业学习和战略研究,可谓一文一武,一阴一阳,一张一弛,相互促进,相得益彰。
战略研究要有更多的问题意识和进取精神,不仅仅是知识的获取,而为我所用,服务于最终目标的实现。
战略研究的境界,可用中庸的致广大而尽精微,极高明而道中庸形容之。
层次则有历史,科学,艺术,哲学。

\section{问题}

大学:物有本末,事有终始,知所先后,则近道矣。认识事物的轻重缓急,按其行动,就接近道了。按重要性和必要性维度进行分解,是柯维几本书的一个重要内容。
李笑来在财富自由之路中,一语中的,作为终极问题。对该问题的反复审问,是磨练价值观的利器,有助于提高选择和决策能力。别的问题都是术,这个问题近乎道。

所谓选择,即是增加必要的条件。尽量必要,尽量充分,最小完备集,奥卡姆剃刀原则在决策问题上的应用。李笑来所说的万能钥匙,即是NLP的换框法,转换视角。

\section{维度}

何为维度?升维思考,降维攻击。把重要维度都列出来,从中选择优势维度,扬长避短,有所取舍,特色组合,从而构建核心竞争力。
价值链分析如此,蓝海战略也如此。互联网,成本等都可以成为分析的重要维度,如差异化竞争,互联网对+,免费等常用的竞争策略。

维度,或者说条件,要素,李笑来有个认识:增加条件。

整合,跨界,爆裂,裂变都是这个核心思想的变形。借助技术手段,多维整合为一维,或一维细分为多维。
技术,市场和自然所谓3M力量,在塑造未来世界。

升维思考,降维贯通。维度一概念,用代数方式研究,线性空间。在数学和战略研究之间,架起会通的桥梁。数学和战略之间,有着深刻的联系。
构造结构,识别模式和关系,掌握变化的趋势。

蓝海战略的价值链分析,波特的五力分析着眼于产业竞争分析。价值链分析通过加减乘除来构建自己的优势维度,有所为有所不为。

维度是立体的,层次的,从分形的角度看,不仅仅有整数维,也有分数维,涌现出奇异的系统特性。

当代科学进展,描述了令人惊奇的时空结构。按照量子力学和相对论的认识,物质,能量,时空都呈现了超出直觉的功能和特性。
对微观粒子结构的认识在深入,对宏观时空结构的认识也在深入,在极小和极大的时空尺度上,都存在一些真正的大问题。

所谓认知,就是对维度的认识。志如其量,量如其识,三位一体,相互影响。量,开放心态,识,见识,认知水平。

\section{老子之道}

战略研究有层次境界之别,重点是理解并运用老子的一句话:道生一,一生二,二生三,三生万物。
其中一是重点的重点。侯王得一以为天下正,得一,则万事毕。

一生二,因二以济民行。一个模型是双线法则,守正出奇。守底线,抓关键。修道保法,故能为胜败之政。

老子之道,博大精深,内涵基本原则。马利克的管理,植根于系统论、控制论和仿生学等现代科学,耗散结构是另一值得注意的。
道生一,一即战略;精心守一,参悟商道。以道为中心,通达战略和管理,促进成长和进化。

半部老子治天下,围绕老子,通达道家思想。没有厚此薄彼,抱一而为天下式。老子文约义丰,且较为熟悉。

为自己打造一口深井,由老子承载大学之道,修齐治平,尽在其中。

从认知升级的维度读老庄,会有趣得多。

\section{圆点战略}

圆是格局,点是破局。圆是调查分析,点是指导方针和行动序列。

\section{双线法则}

\section{西方战略管理思想}

战略始于问题。战略七问,德鲁克五问,如何回答这些问题需要深入分析。黄金圈法则,明确了问题的顺序。

受到商业机构成功的启发,比尔·盖茨在今年的公开信中提出了一套“成功法则”:量化目标―选择策略―考量结果―调整策略―实现目标。
比尔·盖茨认为,这不仅仅是商业机构成功的秘诀,致力于扶贫帮困解决社会问题的非营利机构同样应遵循这一法则。

PDCA是唯一的管理方法,法尔科尼管理方法。

双环学习,前提批判,在表面原因之外,有更深层的原因或约束。
在尝试解决问题之前,需要更深入的调查分析,找到问题的根本原因,而不是流于表面,治标不治本。这引向了系统动力学的视野。

战略是可行性的假设,需要持续的压力测试,来检验其正确性和有效性。

机器的隐喻,有机体的正常运转。可以把组织看作一台机器或有机体,机器是在正常运转吗?
用控制论去理解,控制论适用于机器和生命领域。

\section{价值链分析}

场景,价值链,商业模式。价值链拆分为要素和连接,归核,裂变,融合。效率,成本和差异化。

每一要素是一维度,如孙子兵法的五事七计,就是五个维度及其量化。五个维度构成一向量,或矩阵。
这是认识事物的一般方法。

德鲁克五问:


\section{11}

\subsection{01}

编程注意事项
\begin{enumbox}
\item long time操作之前与之后,需要renew lease,防止lease失效,导致vctl切换
\item 读写、vfm\_get、vfm\_stat、stat等操作之前,必须chunk check。\hl{如果不检查,可能会返回ESTALE, see \_\_chunk\_read\_getnode}
\item chunk组织成tree,就意味着依赖性,需要从上到下依次保护,check
\item chunk上有lock保护
\item alloc/discard与io分离,故io加rdlock
\item \hl{困难之处在并发,tree结构上的并发,并且涉及到持久化、lease}
\end{enumbox}

诊断工具
\begin{enumbox}
\item 怎么快速得到各个controller的状态,包括pool、volume、snapshot等?
\item 怎么检查底层数据一致性?
\item 打印集群chunk tree,以及每个chunk的详细信息
\end{enumbox}

写到LUN结尾处,会自动扩容。ROW3这里有问题?

垃圾数据,干扰恢复,gc replica目前不能打开?

\subsection{02}

尽可能通过\hl{DBUG、GOTO}保留可跟踪线索,以方便线上调试。

mellanox交换机编程:SDK?


\section{12}

\subsection{03}

手腕扭伤快好了。真是旷日持久,一点点小问题,竟然如此,不可不慎重。病向浅中医,养生须趁早。

最重要的是下一阶段的工作问题。内壮则外强,不论何时何地,不要心怀幻想。\hl{放弃幻想,准备战斗}是战士的必然选择。
枕戈待旦是为将者的核心素养。

学习中医、太极,一是道,二是术,道术并重,前期以道为主,明理,后期深入术中,以道卫术。
有所偏重,交相互养。从更大的视野去看,道一而术多,都可以看作道的应用,博观约取。

对学习计算机技术也有好处,相互发明,在交叉地带有所领悟。为学日益为道日损。

徐特立老读书学习之法,定量、有恒,不可自乱方寸。

\subsection{04}

以心行气,以气运身。气是沟通身心的桥梁,区分先天气和后天气,呼吸是后天气,内气、真气从之的是先天气。
随外在呼吸,引领先天气之循环:沿着任脉而沉至丹田,沿着督脉而上达百会,形成闭环。任督通,则成小周天。
吸升呼降,升级的圆运动。医理拳理通。只有丹田聚气养气充沛,自然能打通任督二脉。

活动关节,找到途径各关节的主要穴位进行按摩,以起到拉伸的效果。循经找穴,进而通过经络上的关键穴位进行按摩。

\subsection{05}

洪范五福:《尚书》上所记载的五福\hl{一曰寿、二曰富、三曰康宁、四曰攸好德、五曰考终命},
东汉桓谭于《新论·辨惑第十三》把“考终命”更改,将五福改为“寿、富、贵、安乐、子孙众多”。
现代常把“五福临门”当作新春祝福使用。

六极:一曰凶、短、折, 二曰疾,三曰忧,四曰贫,五曰恶,六曰弱。

积贫积弱,富强的现代化中国。五福有可为有可为而不可期,如康宁、攸好德、富,一定程度上是可以做的。
这里有因果业力法则在起作用。

\begin{shadequote}
兴五福销六极

问:昔周著《九畴》之书,汉述《五行》之志,皆所以精究天人之际,穷探政化之源。然则五福之祥,何从而作;六极之沴,何故而生?
将欲辨行,可明本末。又今人财耗费,既贫且忧,时沴流行,或疾而夭。思欲销六极,致五福,殴一代于富寿,纳万人于康宁。何所施为,可致于此?

臣闻圣人兴五福销六极者,在乎\hl{立大中致大和}也。至哉中和之为德,不动而感,不劳而化,以之守则仁,以之用则神,卷之可以理一身,舒之可以济万物。
然则和者生于中也,中者生于不偏也,不邪也,不过也,不及也。若人君内非中勿思,外非中勿动,动静进退,皆得其中,故君得其中,则人得其所,人得其所,则和乐生焉。
是以君人之心和,则天地之气和,天地之气和,则万物之生和。于是乎三和之气,䜣合絪缊,积为寿,蓄为富,舒为康宁,敷为攸好德,益为考终命。
其羡者则融为甘露,凝为庆云,垂为德星,散为景风,流为醴泉。六气叶乎时,七曜顺乎轨,迨于巢穴羽毛之物,皆煦妪而自蕃,草木鳞介之祥,皆丛萃而继出。
夫然者,中和之气所致也。若人君内非中是思,外非中是动,动静进退,不得其中,故君不得其中,则人不得其所,人不得其所,则怨叹兴焉。
是以君人之心不和,则天地之气不和,天地之气不和,则万物之生不和。于是乎三不和之气,交错堙郁,伐为凶短折,攻为疾,聚为忧,损为贫,结为恶,耗为弱。
其羡者潜为伏阴,淫为愆阳,守为彗星,发为暴风,降为苦雨。四序失其节,三辰乱其行,迨乎襁褓卵胎之生,皆夭阏而不遂,木石华虫之怪,皆糅杂而毕呈。
夫然者,不中不和之气所致也。则天人交感之际,五福六极之来,岂不昭昭然哉。臣伏见比者兵赋未减,人鲜无忧,时沴所加,众或有疾。
德宗皇帝病人之病,忧人之忧,于是救之以广利之方,悦之以中和之乐,将使易忧为乐,变病为和,惠化之恩,莫斯甚也。

然臣窃闻善除害者察其本,善理疾者绝其源。伏惟陛下欲纾人之忧,先念忧之所自;欲救人之病,先思病之所由。知所自以绝之,则人忧自弭也;知所由以去之,则人病自瘳也。
然后申之以救疗之术,则人易康宁;鼓之以安乐之音,则人易和悦。斯必应疾而化速,利倍而功兼。六极待此而销,五福待此而作。
如是,可以陶三才缪滥之气,发为休祥;殴一代鄙夭之人,臻乎仁寿。中和之化,夫何远哉!
\end{shadequote}

东方治理学中阐述的黄帝四经基本方法论,作为基石。由此而展开为完整体系。理论核心在于体味阴阳中和的奥义。

撞丹田有上中下三个高度,每种高度20次为一组。完成5组为一轮。暂定如此,必须数量化。
\hl{任何活动都考虑量化},包括标准和验收。按由轻到重的原则,逐步加大运动量到一最大值,然后保持,允许有一定波动。
度数信,即是围绕中心轴的圆周运动,弦动方式。务必重视周期和节律。

张三丰打坐歌需要背诵。

意气运动要贯彻到一切运动中,这就意味着尽量慢,体会其中意气运行的节律。
数量并非第一重要,重要的是内在质量。

\subsection{06}

\subsection{07}

本周基本理解了RDMA,下周需要继续,列出以学习计划。

修炼也渐渐进入状态,坚持的并不算好。还需要加大力度和决心,通过一定手段调理身心是一辈子要进行的工作,不能轻视。
核心理念就是通经络,调气血,致中和。这也是世界运行的方式。

张首晟去世事件令人震惊,跨界资本失败造成的?也发人深思,\hl{专业是立身之本},下一步需要进一步深入去学习。
丹华资本是一步错棋?投资决策过于乐观?默默地做好自己的专业,再考虑趁势而起。不要给自己太大压力。\hl{从容中道乃最佳策略}。

\hl{太极混元桩站起},静中有动,气息流动,一刻不停。

每次集中攻克一个问题,目前静坐中,发现一些痛点。按瓶颈理论去处理。

\subsection{10}

站桩,太极混元桩、三体桩。为什么说万法出自三体桩。从无极桩、混元桩练习开始。
有足够的时间慢慢练习、体悟其原理。

动静交养,静坐、站桩、内家拳,都需要\hl{调心、调息、调身}。一呼一吸为一息,呼吸于生命而言,极为重要。
此中反思,对今后若干年具有重要价值。生活习惯、对生命的理解在这个过程中得以深化。
静坐、站桩、丹田功这些基本的内功心法,需要坚持下去,\hl{循序渐进、持之以恒}。信解行证,度数信。
基本方法无它,就是围绕一中心点,日积月累,以达豁然贯通之境。

\hl{脏腑、经络都可以看作中医的象数模型}。取象比类,就是一种模型思维。
至于模型是否反映真实情况,需要在实践中进行调节。

HY下一步会很艰难,是否陪着走下去是一个重要的抉择。是否有利于今后长期发展?是否有大的突破?
当断不断,必受其乱。选择的最重要标准就是下一步的发展。平台、领导、行业都极其重要。

简化简化再简化,放弃幻想、准备战斗,机会主义要不得。

\subsection{11}

平时行坐站卧都要注意姿势,功夫要下在平时,用正确的理念塑造良好的生活习惯。这个才是细水长流之道。
功夫怎么才能上身?不是机械地去练习,而是全身心地投入,用心领悟其精义。

体育运动与专业学习一样,遵循相同原则,如循序、有恒。刻意练习的理念,怎么用起来?
专项训练,全面提升。如何设计一套切实可行的健身法?应该包括:
\begin{enumbox}
\item 站桩
\item 静坐
\item 八段锦
\item 太极
\end{enumbox}

一些小敲门:
\begin{enumbox}
\item 腹式呼吸
\item 撞丹田
\item 跪式
\item 刷牙
\item 扣齿
\item 梳头
\item 提肛
\end{enumbox}

禅坐先不求速效,每天盘盘腿,重点关注一些痛点,就会有进步。日拱一卒的精神。
每一种功法都需要大量的时间积累,才能见效果。一个时期最好只有一种重点项目,待步上正轨时,再开始下一项。
采取一主多辅的架构。比如本阶段一撞丹田为主,以站桩、静坐为辅等等。
最后把所有的功法九九归一,抱元归一,回到旋极图的象征。

老子四十二章是最高哲学。\hl{道生一、一生二、二生三、三生万物。万物负阴而抱阳,中气以为和}。
这几句话包含了终极真理。要从中出发,展开为现实的力量。

近期的沉淀期,有其价值,沉下来,再出发。收敛到中间一点,收放自如。潜龙勿用,阳在下也。复其见天地之心。
最重要是保养此一团阳气,以直养而无害,则塞乎天地之间。越养越精神,以做持久战。

悟中道之理,成炼金术士。炼金术士者,能化腐朽为神奇。河洛五行,中土最为贵,乃调理转化之器。

放弃吧,现在变成很大的负能量,就这样不明不白的,有什么意思?现在的主要精力,应放在下一步的健康发展上。
真是内忧外患,有陷在泥潭里的感觉,世界那么大,为什么不走出去看看?

可以了解很多,最重要的确实基本功。

\subsection{14}

书本知识往往已经过时,紧跟会议、论文、各公司的实践活动。

\subsection{17}

\hl{大动不如小动,小动不如不动,不动之动乃生生不已之动}。此语精妙,极有启发。让人回归本源,无为而无不为,故大成拳是无为法。
剔除枝叶,一意本源。

在技术上事业上都有很好的启迪。处当今之际,HY可谓内忧外患,风雨飘摇。更有追问本质,以图活下去。
在技术上,一切围绕ABC,又分主次。主为linux、ceph等。通其一,万事毕。

大成拳心法功法俱为上佳,可以为一段时间的探索画上一个句号了。医武同源,同臻于道。道零德一而万物化生。
信解行证是一循环,转动不已。进而,立禅即意,此中禅意,渐渐融入生活中,行住坐卧,无往不在禅意中。
则与道合一。往日学习,偏于知解,缺乏体证,遂茫茫然不知所归,没有受用处。

大成拳与阳明心学都在唤醒自然本具之良知良能,栽培涵养,心中分别心生,则离道日远,佛理深邃,不可思议。
水性太极,妙悟圆觉,大成立禅,归心金刚。都需用心体证,勿自限于文字知解,何况不能达于文字般若,
妄生议论,枉费口舌,大可不必。

至此,各方面均有妥当安排,可以心无旁骛,尽心驰骋了。艺宗AB,拳归大成。

不管HY如何,这次一定要走。初步定为xsky,离家距离不算太远,发展势头蒸蒸日上。
最重要的是,与手头做的最为接近,可以全力以赴投入技术的进一步深造。

也需要安排几个候选,如京东云、联想、首都在线。至此,对下一步的职业规划基本定位清晰。

\subsection{18}

想不到华云窘迫至此,真是可叹!

贪多嚼不烂,全闪是唯一机会。

放弃幻想,准备战斗。积极准备找工作,静观其变吧。损失云云,不作为主要考虑项。
永远往前看,修之于身,其德乃真。

全用户态SDS,就照着这个目标努力。把握好AFA这个风口。
怎么建立相关知识体系呢?

运用心的综合能力,不要被细枝末节所遮蔽。

\subsection{19}

\subsection{20}

数据为王,存储是关键,必须坚持深入,建立完整的知识体系,同时有一门深入的定力和决心。
致广大而尽精微。

领域
\begin{enumbox}
\item 操作系统
\item 数据结构和算法
\item C/C++
\item RDMA
\item DPDK/SPDK
\item NVMe
\item FusionStor/Ceph
\item 其它分布式存储系统 (FS and DB etc)
\end{enumbox}

现在是一个重要的发展阶段。静下心来,好好转动PDCA循环。

linux io path非常重要,direct io绕过page cache,故需要对齐。aio依赖于direct io,也需要对齐。

块设备驱动的分层架构:\hl{vfs,page cache,general block,scsi},主机通过scsi协议连接到磁盘。
主机通过pcie的NVMe连接到NVMe SSD。

为什么1M块的读写如此慢?

latency和iops的关系:拿行驶在路上的车为例,没有饱和时,latency不受iops的影响;达到饱和后,如何维持稳定的iops就是一个大问题。
另外,故障对iops的影响也是设计上的大问题。故障后,进入降级运行状态,需要标记,以利恢复。\hl{论述并发和故障对iops的影响}。

\hl{理解异步操作},异步操作不同于非阻塞操作,依赖于非阻塞操作。aio、NVMe、RDMA、DPDK都采用了异步通信机制。
提交任务到队列,在polling线程里处理完成事件:
\begin{enumbox}
\item 提交
\item 完成(事件驱动、或PMD, callback)
\item 保持上下文
\end{enumbox}

提交和完成可以在同一线程里。

在异步操作之上可以构建同步操作。如rpc,提交请求后,进入yield状态;完成请求后,resume。
用状态机、或协程实现,原理都一样。

\hl{线程+队列}是中异常强大的模型。

\begin{enumbox}
\item aio syscall与eventfd结合,可以纳入epool/pool/select机制之中。
\item NUMA/cpuset和hugepage
\item 增删节点(异步化)
\end{enumbox}

其它如timerfd,signalfd都可以纳入epool机制,提供事件等待和通知机制。

启动若干aio polling线程,提交或检查完成状态。在aio api之上,用poll作为事件通知机制。
用到了两个eventfd,一个用于提交请求的通知,一个用于检查完成状态。

线程加一控制对象,对象包括上下文信息、队列等。

sqlite操作采用了类似机制。

\subsection{21}

linux内核分析和应用、大规模分布式存储系统是两本好书,按此知识体系按图索骥,\hl{致广大而尽精微}。
但这里的主题不够全面、深刻,更新更深的主题和知识点通过微信公众号、论文、blog等加以补充。

把握行动学习两原则和刻意练习、PDCA、黄金圈发展等基本理念,来指导自己的学习过程。

怎么高效地利用资源?

NUMA架构的cpu资源如何利用?线程core binding,私有hugepage分配器。
上层网络连接等任务hash到不同的core thread里。在一个core thread里处理提交、完成检查等基本操作。
同步操作,如io、网络、数据库访问,派生独立的线程池去处理。
如何做到内存本地化?如何做到thread均匀地映射到NUMA节点上。

如何减少上下文切换和数据copy的开销。

可抽象出独立的内存分配器,每个core有一个内存分配器的实例。

从线程的角度看,分为事件循环线程和工作线程,通过队列把各个线程连接起来,actor and channel模式、csp模型。

分析烧开水这个过程。打开开关后,我就可以走开去处理别的事情。由电水壶独立处理烧开水这个事,等它处理完成后,会鸣笛通知。
我得到通知,就可以使用其中的开水了。如果有多个电水壶,它们就可以并行工作。(\hl{用UML序列图来描述})

领导安排任务也遵循同样的逻辑。通过比类取象的方法去理解,并不难。

\hl{网中网}:主机与磁盘的连接,也可以看作是网络。如NVMe SSD,就是NVMe over PCIe,运行在PCIe上的NVMe。
普通的disk也是如此。PCIe采用串口技术,比并口更快是因为具有更高的频率。
从协议上看,NVMe对SCSI的优化体现在哪些地方? NVMe标准依然在快速演进。

与网络连接不同的是,主机与设备的连接是本地的,不是远程跨主机的。

从架构上,本地存储引擎这部分代码应该是简单的。本地磁盘管理:
\begin{enumbox}
\item pool与disk具有一对多的映射关系
\item 每个盘是个状态机
\item 磁盘多态:支持normal and NVMe disks
\item 分配磁盘位置,首先选择盘,再选择盘上的某个位置
\item 检查磁盘故障,并修复数据
\item 引用计数技术
\end{enumbox}

每个盘有属性和方法,包括分配、读写。每个disk有独立的分配工作线程。
一次分配请求,大部分情况下,是一块盘去满足。如果一块盘空间不足,就需要多盘。
要能够表达非连续的离散空间。请求端是同步操作,用一个lock来实现。由工作线程unlock。wuw序列。

normal disks采用aio方式,NVMe才有自己的方式(kernel bypass)。

chunk位置映射,有cache,读多写少。数量大,cache采用LRU置换算法。
\hl{要充分理解各种各样cache的重要性。作为一个重要的主题去掌握}。

参考lich架构,用c++写一个分布式全闪产品,是一个重要的想法。
这样,既可以从设计者的角度去理解lich,也可以引入一些优化项。
最重要是精简,不是一日之功,需要持久战,长期专注思考和coding。
就是她了。

\subsection{22}

lich采用了epoll+aio的io模型,epoll+同步io会引起线程主循环的效率。可以从两个层面看这个模型,
堵塞在epool上,就绪时进一步引入aio。事件和实际的io操作。aio是通过工作线程和多个队列共同完成的。

transfer,tcp和rdma,是可以共存的,都可以理解为条条大路通罗马,信息高速公路。在c/s之间建立了一条虚拟通路。
这条通路是在第三个对象的辅助之下建立的,监听socket。监听socket和连接socket可以统一地在一个事件处理框架内去处理。
针对每个socket,有独立的处理过程。

自转、公转两个环,polling处就是中心所在。消息有自转,也围绕中心公转,是双环,两个层面的事情。

提交和完成是两个过程,可以放在两个独立的线程里,也可以合二为一,用一个独立线程去做。队列需要各自独立。
队列+线程是强大的编程模型。如SEDA架构所描述的,可以有很多的变体。可以用来实现异步操作。

中断指的是cpu处理的中断,有外部事件发生时触发,cpu收到后,进入中断处理程序。与进程调度一起来考量。
中断时理解问题的一个关键概念。线程与协程的主要不同就在于此。线程可抢占,协程不可抢占。进程进入wait态后,需要中断去唤醒。

\hl{等待事件的线程,被中断唤醒后,重新加入ready队列,可以被调度执行}。

进一步,要理解mmap和sendfile的工作机制和带来的好处,主要从syscall讲起,如何减少上下文切换和内存copy的成本。

取象比类的方法理解计算的世界。

多存储网段,有多个port互联,若做bond,就是合成一条。
异步操作,烧开水。性能指标,汽车在路上行驶。确实有生动形象、易于理解的效果。中医的经络、藏象,方法上是相通的。
这是费曼提倡的方法。操作系统里的关键问题和技术,都是通过故事的形式引入的,如生产者消费者,哲学家就餐,背包等。

这比一味地在api之间晕头转向要好得多,并不是说api不需要掌握,而是明理后api的设计就是自热而然的过程。
这样设计出来的api自然贴切,持久稳定。

用解剖麻雀的方式解剖lich,认识其亮点和不足之处。在与ceph对比的过程中,进行创造性综合,集大成。

本周通过分析异步操作,终于有了贯通感,网络编程,包括TCP和RDMA、NVMe、iSER等,都采用了相似的设计模式。
深入理解进程、线程和协程,加上对队列的理解,就可以理解大部分问题了。

进而理解lua、erlang、go的协程模型。

道法术器,器是产品、系统,术是实现技术,法是架构和算法,道是原理。明理、善学,在术的层面要多进行刻意练习。

渐渐觉得,单纯看书不是学习的最佳方法。看书是第二位的事情,第一位的是在头脑里进行的综合分析和判断,就是用意不用力。
比如学习网络编程,大部分的书籍,都是在展现一个知识体系,至于最佳实践,往往显得过时而且没有针对性。

边想边做边读书,效果会更好。一直以来,有过于偏重读书的习惯,反而迷失了宗旨大义所在。
每本书,在知识的程序上,有重点和等级的不同。如linux环境编程更多讲一个一个api,以及api在linux kernel的实现方式。
这一层面是深入的知识,但从应用的角度去看,却远远不够直接。

道法更多是理念层面的,术器是实物层面,术是建筑系统这一大厦的一砖一瓦。道法则是指导建筑的原理和方法。

投资护城河,养生之理明,则进入专业和财富的领域,大道至简、一以贯之。

\subsection{23}

专业能力就是目前的一,有这个才有谈得上一生二后续过程,得一以为天下式。万不可再不重视了。全力以赴即可,至诚无息。
中庸是一部经典。

易经、道德经、中庸统摄黄帝内经、四经,一内一外,内圣外王之道备。

先从单机和分布式开始。

不应该开启超线程,会降低单个core的性能。NUMA节点内多核与内存是同一距离。在malloc时,可以注册RDMA。

完成一个任务可以有多种多样的方式。自己处理,或委托给别的线程处理。
自己处理比较简单,就是block,或非block的同步调用。

引入\hl{io multiplexing}的意义:监听多个描述符,只处理ready的描述符。
这就多了一层,成为两层,带来了非常强大的能力。

\subsection{24}

总是太乐观,HY处境一至于此。内忧外患,真是举步维艰。
抓紧呀,抓紧,真是危在旦夕,不能不做最坏打算。

下一步学习重点:\hl{操作系统、编程语言、算法和数据结构}。
这是最基础的,从这里引申出来,如存储、网络、文件系统、数据库等等。
CBA也离不开这个基础。勤练基本功,是做好任何事情的秘诀。

进程、地址空间、文件是OS的重要抽象,务必深入去理解。
磁盘等外设也是文件。io路径是个重点。ioctl能做什么?

线程是可调度的最小单元,同一进程的多个线程共享\hl{进程地址空间}。

外设中断cpu执行,进入中断处理程序,重新调度。
调度器具有最高权限,所以进程是可抢占的。

外设具有控制器,包括寄存器、数据缓冲器和调度算法。主机与外设如何通信?外设的数据缓冲器独立于进程地址空间。
direct io是越过page cache层,直接进入设备缓冲器,所以需要对齐到块边界。

删除卷或快照,分摊到各个节点上并行执行,如果引入回收站的功能则最好。
在存在离线节点的情况下,不应该违反\hl{safety和liveness}性质。

timer如何实现?

提交:直接提交,或提交到队列,然后通知负责提交队列的线程。所有的事件,都可以\hl{统一到epoll}框架内,
如enentfd,signalfd,timerfd,以及aio、socket fd等。

分解事件检查和实际处理过程。

线程
\begin{enumbox}
\item 一个core对应若干aio线程,
\item 一个disk对应一个allocator线程,
\item 一个sqlite db对应一个线程。
\end{enumbox}

都涉及到队列和多线程同步。队列的位置有所不同,有的归工作线程,有的归提交线程。
这存在非常大的灵活性。两种模式:\hl{hash到工作线程,工作线程polling task}。

协程是用户空间线程,依托于内核线程对象,不可被抢占。
因为OS感知不到,内核线程对象才是调度的最小单元。

协程维护1+N个上下文对象,N是协程的上下文。

\subsection{25}

VM failover,源vm自杀,目标vm才能启动,lease机制。

着重\hl{思考痛点和难点}。lich的难点在于故障下io无中断,数据副本一致性。

虚拟化和容器,都是需要探索的技术领域。要在\hl{操作系统、编程语言、算法和数据结构}的基础上,做出技术的Y型知识结构。

smartx的技术要求:分布式paxos、raft、etcd、zookeeper等。本地存储、存储协议(iSCSI、NVMf、SPDK等)。

基本功
\begin{enumbox}
\item 操作系统:io path, vfs、aio
\item 编程语言:c/c++,python、go、erlang等。
\item ext2/3/4, xfs, btrfs, f2fs, block layer
\item LevelDB/RocksDB
\item iSCSI, NFS, samba以及其它存储协议
\item HDFS/Ceph/Sheepdog/GlusterFS
\item SSD IO性能优化,有FTL开发经验
\end{enumbox}

每个core thread有线程本地变量,指向aio指针数组,共4个,前两个用于direct io,后两个用于元数据的sync io。
同一磁盘设备文件,以不同模式打开多次。这是fd与inode的多对一的关系。fd对应的对象一定存有状态信息。

redis的数据模式:每节点若干redis实例,每个实例保存:disk,metadata,raw记录。raw记录按vol组织成hash。
支持的操作:
\begin{enumbox}
\item 分配chunk
\item 回收chunk
\item 删除卷
\end{enumbox}

\hl{NUMA架构的拓扑结构},如何分配core和aio线程,如何分配内存?
接着看polling线程如何管理私有内存。

cat /proc/cpuinfo, cores与sibling相等,则没有开启hyper threading。若sibling是cores的二倍,则开启了超线程。
还有别的可能吗?

polling线程在NUMA节点之间均匀分布,aio线程优先选取超线程,其次选取同一物理cpu上的其它core。总的原则是局部性。

\hl{从core的初始化过程看起,接着重点关注polling线程的工作}。
外部线程如何与core线程通信,各类事件如何注册到core线程的?

统计内存使用量,统计各类资源的用法

怎么理解syscall,怎么理解内核空间和用户态?上下文切换和内存copy。如read过程,如何分析?如何优化?

\subsection{26}

从主体出发,从进程出发。进程调度、内存、文件和IO等。中断,异步机制。

每个cpu都有自己的线程队列,亲和力是进程的一个属性,指定可以运行在哪些cpu上。就是可以对应cpu的调度队列。

NUMA节点对应的内存,\hl{内存条应该在NUMA节点上对称插入}。否则,没有内存的NUMA节点需要访问远程内存,影响性能。

cpu拓扑是一棵树,叶子节点是逻辑cpu。polling线程binding到逻辑cpu上,
相应的aio线程选择与polling线程\hl{所在cpu最近的那个逻辑cpu}。
\hl{NUMA节点,多处理器,多core,超线程}会增加这棵树的层次。

NVMe设备自动发现会成为一个设备文件,通过kernel io路径进行存取。unbind设备驱动程序后,可以通过pci进行访问。
设备接入bus,通过port空间或内存映射访问设备。

在第一阶段,从各个NUMA节点上平均分配hugepage(mbind)。 hugepage与NUMA节点的这种关系得到保持。
mmap: posix\_memalign虚拟地址空间,然后mmap到对应的hugepage。然后初始化管理元数据,包括pages和buddy。
这个地方的代码可以优化,用统一方式管理global和private内存区。

为什么要用虚拟地址获取物理地址?方法是什么?物理地址在什么情况下被用到?

可以抽象出hugepage region这样的概念,用于管理多个hugepage,需要处理如下需求:
\begin{enumbox}
\item 指定NUMA节点
\item 用buddy算法管理多个hugepage的分配
\item 管理虚拟地址到物理地址的映射
\item 用统一方式管理global和private内存区
\item 统计内存区使用情况
\item 不需要借助hugetlbfs
\end{enumbox}

恢复策略:需要增加维护模式,在此期间,不进行数据修复。

学习路线图:
\begin{enumbox}
\item core
\item mm
\item rpc (tcp and rdma)
\item aio and nvme
\item iscsi/iser
\item NVMf
\item etcd
\item ***
\item snapshot
\item tier
\item bcache
\item EC
\end{enumbox}

还有什么可以改进的?

epoll机制的通用性
\begin{enumbox}
\item fd and socket
\item eventfd
\item signalfd
\item timerfd
\item pipe
\end{enumbox}

作为\hl{异步事件的监听机制},具有广泛的适用性。其它类似sem wait and post, \hl{wuw加锁机制}等。都可以用epoll来实现。
真正体现了event driven的特征。

\subsection{27}

六种同步方法:\hl{mutex, cond, sem, flock, spin, rwlock}。sem可以用于共享内存的同步,flock用于文件同步。
mutex, cond, spin, rwlock多多于多线程的同步。注意它们自己的不同。

mutex, spin, rwlock使用场景相似,性能有差别。涉及临界区和对共享变量的访问。
cond与mutex结合起来使用,是否一定要结合mutex?wait在一个外部条件上,这个外部条件/变量被多线程改变。
sem的使用场景又有所不同。如同步线程创建过程。串行化多个线程的创建过程等等。另外\hl{支持timedwait和try语义}。
至于文件锁,使用场景易于确认。

sem的wait和post是由不同线程执行的,一般的mutex、spin、rwlock则最好是由同一线程执行。否则会复杂化执行流。

简单的同步比较容易实施,比如多线程下保护一个全局变量。但要保护一个大型的数据结构,就比较困难,需要遵循一定的加锁、解锁协议/约定。
\hl{2PL,tree protocol}都是这样的协议。

场景一,一线程wait,另一线程需要唤醒它。用sem wait and post可以实现。

场景二,\hl{生产者在队列满时block,消费者在队列空时block}。队列满是一个条件,队列空是一个条件,所以需要两个sem。
同时对于队列长度的变化,需要加以并发同步。

\hl{所谓条件,只是可能性,而不是确定性的}。收到通知后,再次检查条件,进行相应处理。所以,这里通常是一个循环。

卷chunk树的cc比较复杂,每一个chunk都是一个cc的保护单元,又形成了层次结构。

\hrulefill

WAL和db共同构成完成的数据。这是大数据的alpha架构,什么是完整数据?如何构建完整数据。
为什么要先写入WAL?因为写入WAL相对性能高。WAL要支持UNDO和REDO操作。\hl{ARIES事务过程}。

fusionstor是没有commit log的,如何保障事务性?需要保障事务性吗?什么是顺序一致性。
严格\hl{区分提交的数据和未提交的数据},对两者的要求截然不同。
采用的clock机制,带来了系列问题,如周期性合并clock导致iops抖动,掉电情况下clock丢失,引起大量的恢复流量。unsafe\_clock机制的不自然。

\dotfill

一旦理解\hl{线程+队列}这种模型的强大威力,则scheduler、SEDA、alpha都比较容易理解了。
本质上是线程以及线程通信方式,有很多种组合情况。

线程模型有多种:M/1,1/1,M/N。按线程与KSE的比例。\hl{协程是M/1实现,pthread是1/1模型}。

scheduler resume当然可以由外部线程调用,只需要传入taskid参数、返回值和返回数据即可。
所谓resume就是重新加入runnable队列,进行重新调度。
被调度器选中后,继续执行schedule\_yield1的后半部分代码,返回值和数据也跟着传出。

\hl{scheduler和task的关系是1+N的关系。task可以用状态机来描述}。\hl{现代操作系统}一书对进程的描述,采用了这种方式。

\dotfill

用户和组是个正交的话题,留待最后再看。

进程和线程是活动的实体,即主体。内存分配、文件io等等都是在进程内执行的。
从一个到多个主体,引入了\hl{并发情况下通信和同步}的需求。

先理解同步机制,再理解IPC机制。最重要的IPC机制包括file和socket。
通信有\hl{共享内存和消息传递}两种基本形式。结合\hl{erlang和go语言的运行时模型}去理解这些概念。

服务端编程不能不深入理解epoll、aio等高级io特性。进一步理解其kernel实现原理。

\hrulefill

\hl{md5sum的计算},要合理安排\hl{存储和计算的位置}。计算放在控制器上去做是否更高效?
md5sum的计算是一个序列化的过程,只能一个完成后,进行下一个。如果vc切换,能否接着算?

按SEDA架构,组织成流水线计算,分两阶段:读取、计算,第一阶段可以并行,第二节点需要串行处理。
即fork and join并行模式。无法采用MapReduce计算框架。

\hl{卷在存储池间的离线复制和迁移},在线方式呢?会更复杂。
如果卷上有快照呢?从产品角度如何定义?

\hrulefill

\hl{研究nutanix的产品和技术}。

\subsection{28}

\subsection{29}

\subsection{30}


\chapter{2019}

\section{01}

\subsection{08}

用本文档记录日志,只维护一个即可。growth记录更高层次的指导原则,一般来说相对比较稳定。
如此就形成了一个相互促进的双环架构。

不要自我设限,一定如何如何?打基础,看机遇。

一些指导原则
\begin{enumbox}
\item 概念图、重要的是理解核心概念,费曼学习法,用自己的语言描述出来,取象比类,借助类比、联想等思维方式。
\item 在默认模式/执行模式,走神/专注等进行切换,\hl{文武之道、一张一弛},有助于激发出创造力。
\item 强调综合判断能力。
\item 慢练
\end{enumbox}

体系架构、操作系统和汇编语言是底层逻辑。数据结构、算法、编程语言结合起来进行学习。

坚持专业技术方面的发展,\hl{CBA都要有所涉猎}。现在正是最佳时机,不能再没有重点了。

重点应放在做过的项目和全闪上。\hl{多路径?多存储网段?SPDK、NVMe}等。
比如单机存储引擎,用sqlite3、redis、RocksDB,为什么?

从势、道、法、术、器等维度进行梳理。

\subsection{09}

clone后,没把源快照置为protected状态。

\hl{software pipelines和排队论是性能分析利器}。SEDA、actor和lambda是架构方法。
图论、网络流,老三论。按中医理论,人体由藏象经络组成,运行气血。与网络系统有很强的相似性。
图是最复杂的数据结构,降维去看其它数据结构会如何?petri net和有限状态机。
把以前生活中经历的点连成线,会更明晰。

专业的力量、能力变现、价值规律、写作是最好的自我投资、自我进化、放大核心优势都是很好的理念,关键在落实。

本质上是核模式的应用。先在一个小的专业领域建立根据地,再顺势裂变扩张。

眼观六路耳听八方,行业格局分析至关重要,技术加商业两条腿走路。

自己要变得强大,这是积极心态,接下来是如何才能变得强大,这是现实,
要做现实的造梦者,不做桃花源中人。

lich的io 路径需要控制器中转,是否影响性能?client直接读写数据会如何?ceph client与osd primary直接通信。

只有一个RDMA网卡,存在硬件上限。如何使用多个RDMA网卡呢?8k 100w iops。

接触斯多葛主义。内心的自由和平静。

Oracle asm?

\subsection{10}

向身边的人学习,虚心,不要老端着,自认为了不起,其实需要理解的东西很多。
差不多先生,含混不清是极大的弊端和缺陷,一定要清晰,清晰才有力量。
表达要简洁清澈清晰。空性不代表含混,而是包容。

写作是最好的自我投资,要深刻理解这句话,并每日打磨写作技能。
先记下流水账,固定时间进行归纳整理,提炼升华。

做事情太保守,\hl{视野不够高,格局不够大},现在呢?更多自以为是,
需要沉下心来,好好观察、琢磨。看清趋势。正心诚意,取势、明道、优术,然后是利器、举例。

集中一段时间,专攻技术。围绕关键问题,从广度和深度,\hl{致广大而尽精微}。
技术综述好做,选几本书多读几遍即可,难的还是深度。另外一条途径,就是多与人聊技术。

围绕全闪进行写作。全力以赴吧。

\hrulefill

\hl{重点整理在HY做过的工作}:
\begin{enumbox}
\item snapshot and LSV
\item Recovery
\item Balance
\item QoS
\item ***
\item ETCD
\item Network(TCP and RDMA)
\item ***
\item Scheduler
\item Memory Allocator
\item Disk Management (Local Storage Engine, +NVMe)
\item ***
\item iSCSI/iSER/NVMf
\end{enumbox}

\hl{恢复、平衡、删卷/snapshot、rollback、clone、flat}等操作,如何用统一的任务管理系统进行管理。
可以借鉴k8s等集群管理系统的经验。有了总控,可以更好地加入策略。

\hl{恢复和平衡}有些逻辑是通用的,如按pool处理,检测本地vctl变化,扫描本地vctl。
可以放到一个处理框架内进行处理。ceph如何进行数据平衡呢?

太大的调度粒度(vctl)不利于任务调度和负载均衡。qos如同节流阀。可以放到不同的位置上。

\hl{if-what}当前的恢复处理逻辑,如果发生\hl{网络分区故障},会如何?

为了支持多网卡,需要做什么?

为了支持多路径,需要做什么?

\hrulefill

研读google发表的系统相关论文,看真实需求以及技术演进方向。
性能、稳定可靠、容错高可用、可扩展是几大非功能属性,决定产品品质。

理解google的技术体系,不能老是泛泛而谈。borg衍生出k8s。
\hl{hdfs、bigtable、megastore、mapreduce}都已经实现了容器化和统一管理。

有bigtable的sstable演化出leveldb和rocksdb。redis可以与rocksdb进行组合,改变全内存数据库的局限性。

催生了hadoop生态,进一步衍生了spark等lambda架构的统一计算平台。

\hrulefill

光点图灵

RAID之上有LVM,都是单机环境下的产物。虚拟化程序越来越高。进而是盘阵,然后是分布式架构,
scale-out越来越强,灵活性越来越高,软件定义和超融合由此而来。

MongoDB直接做RAID0,然后利用其自身的集群能力,master/slave replication架构,
可以failover and failback。这相当于内置存储。当时对存储层面的理解相当有限。

对gridfs扩容是如何做的?通过多个目录做的,也不设计磁盘虚拟化方面,如LVM。
面对的就是一个扩展性问题。

\hl{迁入阿里云、采用七牛对象存储后},不自己维护服务器,避免了很大的麻烦。

\hrulefill

\hl{美地森的工作和成长经历}:

很长时间都理解不离aio、事件驱动的实质所在。
做技术确实死磕精神,重视贪多求大,结果反而欲速不达。

当时做edog又是如何做的呢?有什么可以反思的?
集群管理采用了Erlang,通过libvirt对接qemu,改了qemu的驱动,接入yfs作为底层存储。

对网络理解不够透彻,接触也少。前端时间获取交换机管理信息,技术收到都是类似的。
不过关于交换机,很多技术细节理解不到位。\hl{天马行空惯了,细节把握不到位}。
需要改进,做技术要认真再认真,程序化,每日记录、反思、总结、提炼。
既要有广度,深度更是必不可少。\hl{专精一件事,就可以立于不败之地}。
找到了自己的哲学、守中致和,核模式,就要学以致用,一以贯之,用来指导自己的言行。

接触了hadoop,为什么没有坚持呢?放着这么好的东西,不去跟进,
只是习惯于做些乱七八糟的东西,实在是有眼无珠呀。

虽然也有跟进,\hl{不坚定,不深入},当然更谈不上真的掌握。

\hrulefill

华胜天成:法国电信做手机的SIP协议,管理软件,一头雾水。

包括以前在\hl{天地伟业、鸿业科技},都没有深入业务,总是局限在一个小小的技术视角。
现在软硬件的变化,真是日新月异,不进则退。\hl{没有一个好的平台和机遇},就很难崛起。
如何把握好的平台和机遇,当然要靠本日一点一滴的积累,磨炼基本功,把握大趋势。

鸿业科技做的\hl{官网计算}回忆起来,倒是给了很好的启示,说明了管道理论的普遍性。

一是个人成长放在首位,接着就是睁大眼睛,静静地坚定地去找团队、平台和机遇。(\hl{人和事的胜任度})
两者匹配了,能力胜任度高,自然一切水到渠成。所以也不必着急焦虑,反反复复磨炼自己就够了,
金子总会发光的,好酒也怕巷子深。这些看似对立的格言,其实包含了深刻智慧。

这几年,技术发展真是太快了。风口一个也没把握住,原因何在呢?

\hl{接下来干什么呢}?

转移关注点,放低姿态,学习关键知识点,并记录在案。从整体系统中抽离,聚焦于更小的模块。

最近老王他们进行的性能优化工作是值得关注的架构调整。支持多路径,多网卡,甚至插到不同的交换机port上对性能都有影响(8x or 16x),以提升性能。
性能的理论上限如何预估?

更多地转变到SPDK(NVMe, NVMf)、RDMA上面来。全NVMe方案如何?怎么构建全局共享,client与数据不同于控制器直接存取?
可以参考的资源有什么?

纪翠叶问及fio的部分参数什么含义?这是器层面的知识点。

\subsection{11}

中庸是我的圣经。从心道物等诸多维度确定了必要的理念。

\hrulefill

不深入理解传统存储的高度和局限,就不能理解很多核心概念,比如多路径、共享、全局缓存等。
分布式架构在\hl{扩展性、灵活性}方面胜出。

通过真假latency,可以估计等效并发。方法如下,单并发测量固定时延。多并发测量iops,取倒数即为假时延。
固定时延和假时延的比值就是该配置下的等效并发度。

io path也有串行和并行两部分,从而引入加速比的概念,上面的等效并发度就是加速比。
可以评估整个体系结构的并行执行效果。

按pipelines理解体系结构,体系结构也是网中网的分形结构。
官网、电路等等可以作为分析模型,从流体动力学和物理学里吸收能量。
但也不必把问题复杂化。从最少的元素推导出更多的规则,解释更多的现象,是科学的研究方法。

从体系结构、操作系统的原理和知识,去深入存储子系统的相关理解。

\hrulefill

\hl{网络分区}下,恢复、平衡和一般的IO是如何进行的?

比如一个5节点集群,采用2副本策略。分区意味着什么呢?怎么理解多数派这一要求。
5节点2副本,也是只能容忍单个节点故障,\hl{故障容错度是有副本数决定的},与集群规模无关。
比如一个10节点的集群,2副本的情况下也只能容忍一个节点故障?

可以通过划分故障域改进这一点。副本数据分布到不同的故障域,容错按故障域来定义。

保护域和pool是等价的概念,保护域是一种物理隔离机制。把一个大集群分拆成小集群。所以基本的嵌套关系是\hl{集群-保护域-故障域-节点-磁盘}。

网络的VLAN是否也是一样的分区机制?

这一点与zk、etcd等协同共识系统存在很大的不同。

MySQL master/slave架构,能否自动切换?

\hrulefill

client直接与数据块通信,lich需要通过vctl中转,如何克服这一点?

多网卡支持,通过多网卡连接控制器,MP软件用来管理导出的LUN,确保只有一个。

SSD FTL中心任务是映射管理,与OS的页表、LSV的快照管理是一样的,关键是维护一个虚拟地址到物理地址的映射。
可以是一级,也可以是多级。\hl{伟大的计算原理}里面,讨论存储的时候提到四点:\hl{命名,映射,定位,授权}。
可以按此四个维度去思考林林总总的存储系统。

\hl{按主体-对象模型,授权有两种}:CL和ACL。CL以主体为中心,ACL以对象为中心(如文件系统的ACL)。

\hl{全力投入AFA},大量阅读,深度思考。知识就是力量,这点认识还是不够深刻。

地址空间管理,抽象成了一切皆文件。内存、磁盘、文件都是地址空间。文件系统的设计需要充分地认识到这一点。
SSD最核心的功能也是这一点。SSD FTL很类似于LSV,很多可类比的地方,如映射、GC。

数据结构主要也是映射,如hash、map、graph,graph是最通用的映射关系。
再如,函数也是映射。lambda架构也提及此,map-reduce更成了并行计算模式之一。

\hrulefill

多快好省,快和好是当前重点,多和省在此基础上,迭代优化。\hl{性能、负载、高可用}是几种集群形式,
bonding,多网卡支持。bonding具有什么特征?只是HA?而不能同时工作,聚合性能和均分负载?

\hl{全力投入AFA},大量阅读,深度思考。知识就是力量,这点认识还是不够深刻。

\subsection{12}

地址空间,如内存,文件都采用地址空间概念。多级页表可以支持稀疏地址空间,页表与inode采用的radix tree什么区别?
lich卷的chunk tree也采用了同样的结构。这样,需要维护虚拟地址到物理地址的映射。

对计算机体系结构、指令集、操作系统、汇编、编程语言有了初步的理解。对存储、网络、数据库也有一定的理解。
接下来就是围绕算法和数据结构为中心,\hl{以问题为主线}, 穿起来进行思考。伟大的计算原理给出了很好的总结。

分层去理解更透彻,如计算机组成结构化方法里,分为七层,网络也是分为若干层。

大的问题主要有调度、内存管理、地址空间管理、通信和同步、事务、分布式算法等。

以数据结构和算法为核心,去解决各类系统的核心问题。问题具有普遍性,算法也应理解为通用的算法。问题和算法是对应的。

提出问题,然后看应该如何去解题。解决问题是价值输出的重要方式。

分析io path就可以把一切知识点贯通了。一个写、一个读。集中在数据平面的相关问题。

\subsection{13}

在华云和光点的工作具有互补性,麻雀虽小五脏俱全,需要得到更好的总结、提炼和升华,一花一世界,由此去理解更广阔的技术世界。
譬如一个同心圆,虽有规模效应,规模引起一些质变,但稳固的核心,扎实的基本功,确是最为重要的,如此就拥有了强大的迁移能力。

不能不重视写作。

\hl{单一要素最大化,其它要素最小化}。算法可以作为下一阶段的单一要素,是不变的一。
算法的学习和理解脱离不了具体系统,一以贯之,就是以算法贯穿技术学习活动。

这里的算法是广义上的算法,不仅包括教科书里常常提到的算法,也包括协议、调度、事务等重要概念相关的算法。

\subsection{14}

\hl{化繁为简,打破学科边界},从实际问题及其方案开始理解。深入理解计算机系统即是采用这种叙述法。

计算机系统可分为cpu(\hl{进程、并发、中断、隔离})、memory、io子系统(storage/fs and network)等,按此结构循序展开。
io又分存储和网络等,组成tree架构。关键组件:\hl{调度器(thread、io),空间分配器(cache、memory、disk)}。

采用流一元论(flow dynamics)的描述框架,cpu居中调度,上下文切换和内存copy,可称为指令分配器。
\hl{flow由节点和边组成},节点是处理单元或节流阀,边是管道。处理方式可以有多种:转发、映射、压缩、消重等。

衡量流效率的指标是latency、iops、throughput等。流的平衡态由平衡方程描述,输入过多,会导致处理不过来,
发生拥塞、溢出等故障,故需要限流/节流。

memory copy消耗总线带宽,bus是如何利用的?

先理解一般case,加入cc和故障恢复,系统就变得复杂了。虚拟化、cc、持久性是操作系统研究的三大块。

cpu亲和性需要通盘考虑,不仅仅是内存,还有disk、network,读设计线程和buffer,是否与core线程处于一个NUMA节点?
还是跨NUMA节点进行内存copy? aio线程尽量与core线程距离近。\hl{如果只要一块网卡,则网卡属于哪个NUMA节点呢}?

\hrulefill

\hl{数据结构的内存layout},比如一个int应该怎么存?大端序、还是小端序?浮点数呢?结构体呢?
什么是补码表示?

任何一个变量、常量都占据一定的内存区域(addr,len)。代码也是,cpu从内存中取指、解码后执行。
一个函数是一个代码端,微观上是一个指令序列,符号表。一个对象呢?也是如此。内存布局,成员函数指向函数代码的指针。
模板在编译阶段实例化,用具体类型代替。最后都转化为变量和过程。\hl{class引入了对象作用域和类作用域}。构成了符号的层次结构。
可以假设最外层有一个global namespace,作为ns树的根。

把内存看做一维向量空间,就容易理解这个问题了。
最小地址单元是byte。指针运算,与数据类型有关,void *以byte为单元。

磁盘空间也是一维向量空间,分区、文件系统、文件就构成了层次结构,靠mapping建立虚拟地址到物理地址的映射,完成\hl{存储虚拟化}的功能。
RAID/LVM都是映射,需要\hl{计算或metadata}去实现这种映射。

snapshot、LSV也是meta+data,通过引入mapping实现特性。

\hrulefill

梳理lich中用到的数据结构和算法
\begin{enumbox}
\item array
\item list
\item stack
\item queue
\item heap
\item ***
\item bitmap
\item string
\item set
\item ***
\item skiplist
\item hash
\item tree
\item graph
\item ***
\item token bucket and leaky bucket
\end{enumbox}

\hl{key的结构和分布}非常重要。索引结构不仅要考虑内存结构,还有考虑磁盘结构。
一个线性空间中key的分布有规律可循,且有序。如果key是string,则分布是多样的。

为什么数据库的索引结构用hash和B+,而不是别的?

\hrulefill

bitmap也是用来管理1D线性空间,如磁盘空间管理,没有引入中间结构,key是offset。
bloom filter是一种特殊的存在性查询结构,不能用来检索,key是字符串。

页表、inode address space,都是trie(radix tree)结构。与邮政编码、url、dns域名类似。
name是有结构的,构成prefix tree。

trie与hash、RBtree、B+皆不同,它的key是有结构的,从而形成一定的层次。用来检索一个稀疏的线性空间。
lich volume即采用了该索引结构,定为三层。路由表采用radix tree设计。

array and list是基础存储结构,其它结构都是建立在此二者之上。

\hrulefill

以上是基础数据结构和算法,加上\hl{cc、事务、分布式、机器学习}等要求后,演化出更加多彩的算法世界。
分布式算法更多是交互协议设计。

\hl{墨菲法则},事情总比想象的要坏,真的不能再嘚瑟,认真考虑下一站是当务之急,损失什么的可以不用考虑太多。
重要的是,把握当下,面向未来。

可以参考的开源软件:
\begin{enumbox}
\item nginx
\item memcached
\item redis
\item mongodb
\item ***
\item LevelDB
\item RocksDB
\item Ceph
\item Sheepdog
\item ***
\item SPDK
\item DPDK
\item RDMA
\end{enumbox}

\hrulefill

由设备文件,可以\hl{分区、格式化、mount}。NVMe采用pci号实现kernel bypass。
读写设备文件和普通文件,采用同一套接口和语义。

io路径上\hl{page和block dismatch},page cache和buffer cache,buffer用指针引用page的区域。
故\hl{不经过page cache的io需要block对齐,包括direct io和aio}。

bcache? \hl{1+N虚拟出N个设备}。

bit + context,要深刻理解这一点。

\hrulefill

调度器:cpu调度器和io调度。多队列。

分配器:内存allocator和fs 空间管理,\hl{命名、映射}。

定位包括cache和replication。流动的数据。

\subsection{15}

\hl{行路难,行路难,多歧路,今安在}?好好沉淀一下吧,路真的不好走,一不留神就万劫不复。

忠于专业,这是安身立命之本。进一步的学习计划:
\begin{enumbox}
\item 数据结构和算法
\item software pipelines
\item SEDA and actor
\item lambda
\item queueing theory
\item network flow
\item *** 系统
\item \hl{操作系统(包括体系结构、OS和编程语言)}
\item 文件系统
\item 网络
\item 分布式系统
\item 数据库系统
\item *** 应用
\end{enumbox}

从\hl{scheduler、allocator和cache}几个维度去研究。

先研究scheduler,有\hl{线程调度、协程调度、io调度、事务调度器},甚至k8s一样的全局资源调度。
比如erlang、go的调度也很有特点。

\hl{任务状态,任务队列}。引入任务优先级,就是多队列。进程调度的任务就是选出下一个要运行的进程、线程。

任务上下文切换,scheduler和tasks之间是1+N的关系,交错执行。

调度分抢占式和非抢占式。线程相对于协程,多了中断方式的切换,包括硬中断和软中断。

\hl{调度器就是software pipelines里的分配器},影响系统整体性能。

scheduler采用什么数据结构来组织任务队列呢?

scheduler采用什么QoS策略?

多调度器下的任务漂移/窃取

调度器自身是不能有长时的block任务的,对这类任务采用异步方式处理。
线程调度是抢占式,\hl{协程任务是非抢占式,要杜绝出现block操作},由CQ来完成。

oracle asm架构

三层:\hl{并行、polling、事件处理},并行线程的cache、memory、pci bus要local。
尽量避免并发访问共享资源。share nothing是最好的,但有时共享、通信难以避免。

polling和事件处理研究单个scheduler的情况,并行研究多个scheduler的情况。

\hrulefill

所谓异步就是不等待任务处理结果,当任务完成时,会触发事件,进入cq。

\hl{以FusionStor为例},io有两种情况,\hl{普通disk方式和NVMe方式}。
普通disk采用aio,NVMe采用其自身的异步机制。

网络也有两种方式:TCP and RDMA。采用epoll多路机制,polling请求或完成事件。

\hl{RDMA和NVMe都是基于事件和多队列}的异步通信方式。

NVMe也很简单,用pci打开后,可以异步读写,\hl{poll时执行callback}。
Lich里主要处理了core线程和buffer的适配性。

NVMe没有使用aio线程,自身的异步机制,与coroutine结合很好。
aio还需要结合thread+epoll机制。

allocate时考虑数据块与disk的对应关系。
NVMe disk需要考虑NUMA亲和性?
普通盘连接南桥,不需要考虑?

sqlite3、redis相关的block操作,用工作线程pool。\hl{不同于aio,没有CQ这种机制}。
可以聚合起来进行批处理。因为缺少异步原生支持,可以称之为模拟aio。

网络io怎么做的?\hl{TCP和RDMA不同},RDMA是异步机制,TCP用得是recv和send。

统一的polling框架,包括aio and NVMe,TCP and RDMA,scheduler等。
通过eventfd block到一个block点上。

\hrulefill

状态机相对协程,是否更高效?对象的队列是少不了的,也需要schedule。
每个对象的状态与task不同,需要按每种类型进行分解。而task则与进程一样,有三种基本状态。

\hrulefill

\hl{总结性能优化方法}:NUMA和CPU亲和性,Hugepage,并行、流水线、聚合等方法。
locality考虑。

virtio, IOMMU,安装在client端,感知虚拟化。

\hl{硬件虚拟化}是指处理器支持虚拟化相关指令集,
\hl{VAAI}扩展了SCSI语义,加速数据copy速度,与sendfile类似。

\subsection{16}

每个逻辑Core有一个调度队列(RBtree),所谓NUMA和CPU亲和性,就是通过mask,指定线程可以进入的队列。
调度器是一个函数,从队列或多队列里取出一个或多个要处理的元素。调度器有控制线程,控制线程可以做其它的事情。
从这个角度说,把调度器理解为一个函数为合适。线程调度、io调度和事务调度,都是这种方式。

调度包括队列、调度entity和class。用class定制调度策略。

LICH中的coroutine调度,采用了简单的FIFO方式,有优化空间。

\hl{从底层物理资源去考虑},kernel起到资源管理的作用,即资源虚拟化。cpu、memory、devices。理解devices更复杂,
接入bus,被scan到,加载driver,然后就可以通过一定的方式进行通信,有外部事件时也可以中断方式通知cpu。
中断通知,必然是先准备好上下文。这个上下文的切换是个重点。

block用物理disk作为backing device,就是\hl{利用disk的物理特性},而不是虚拟分区。

\hl{时间片,优先级,可抢占}是内核线程调度的几大特征。

协程与状态机方式,都需要分两层看。调度器和状态推进。状态机选出下一个要处理的任务,执行任务状态机一步或多步后退出。

调度器相当于软件pipelines中的分配器。按epoll,需要把fd加入epoll的兴趣列表,
同时维护分离表/routor:fd到node(event handler以及context)的映射关系。

RDMA用了两层分离机制:epoll用于连接管理,自身的polling机制。

\subsection{17}

\hl{在lich里引入pg如何}?object-pg-osd的关系由两级映射完成,一是降低了metadata量,二是恢复平衡更容易做。
这融合了ceph和lich两者的优点。

如osd故障,pg处在降级状态。与lich引入vfm要解决的问题类似。\hl{引入最小副本数},会使得方案更优雅。
从osd故障到数据恢复,可以定制恢复策略,如时长等。

lich的chunk和replication,类似于RAID的条带化和mirroring。

传统盘阵的控制器结构,有双控、多控。双控关系有A/A,A/P。横向扩展。LUN的归属。
\hl{理解MPIO在多控架构下的工作原理}。

单处理器的false sharing。\hl{同一cache line}上有多个数据项,被不同线程更新。

多处理器的缓存一致性(同一存储位置),存储一致性(不同存储位置)。NAS的client 缓存一致性

ABA问题

本地文件系统
\begin{enumbox}
\item \hl{RAID/LVM}
\item ext2/3/4
\item XFS
\item ZFS
\item btrfs
\item ReiserFS
\item ***
\item Fuse
\end{enumbox}

分布式文件系统
\begin{enumbox}
\item GFS/HDFS
\item GlusterFS
\item Lustre
\end{enumbox}

与ceph对比
\begin{enumbox}
\item 数据分布(元数据管理)
\item 副本分布(节点为最小故障域粒度)
\item TP
\item 卷控(无日志,顺序一致性语义)
\item 恢复、平衡、流控
\item ***
\item \hl{RDMA/DPDK/SPDK}
\item \hl{iSER/NVMf}
\item Cache
\item Tier
\item ***
\item \hl{Snapshot}
\item Remote Replication
\item ***
\item EC
\item Dedup/Compress
\end{enumbox}

Lich优化项
\begin{enumbox}
\item pool下flat namespace,不需要支持多级目录
\item 引入类似pg的结构,二级映射
\item Node内管理磁盘,而没有把disk暴露
\item ***
\item 卷控平衡
\item 查找卷控位置采用了multicast机制,为何不去admin上根据lease情况找?
\item 数据访问需要vctl中转,client不能存取chunk (与array的双控架构做对比)
\item 卷控拆分成子卷,支持大容量卷
\item 恢复可以由卷控自己执行(智能卷控)
\item 停止恢复
\end{enumbox}

\hrulefill

Amazon EBS挂载到一个EC2实例上,EBS可以打快照,快照保存到S3上。
第一个快照是全量的,后续快照是增量的。

对一个pool,容错级别与副本数有关。故障域要大于等于副本数。\hl{EC能改进这个问题}。
降低同等或更高容错等级下的成本。
同等品质的硬件配置,pool越大,硬件发生故障的概率就越高。所以pool的规模也有上限。

ceph数据一致性检测和恢复是以PG为基本单元进行的。PG是对象的集合,PG对应的OSD存储了对象的副本。
副本一致性检测:\hl{peering, recovery, backfill, scrub}等过程。

PG上所有对象的写入,有primary OSD调度,串行化并记录有日志。

故障情况下,OSD(include primary and replica OSD)出现降级对象。

\hrulefill

RAID、副本、EC放在一块去理解。RAID有条带化、镜像、RAID5,在分布式系统中,对应\hl{分块,每块数据做副本orEC等机制}。
一是数据分块,放到多个节点上,提升并行度。再一个是容错等级,副本、EC是两种标准方式。
\hl{副本一致性和条带一致性是难点}。副本一致性通过RAFT协议去保障。

scrub可以发现静默错误,依赖的是checksum。

存储处在新旧之交,在快速演进。硬件和软件架构都在变革之中。
硬件引入了RDMA、SPDK、DPDK、NVMe等快速网络和存储介质、协议。
软件架构从传统阵列的SCSI、SAN、NAS到新的分布式架构。

新型存储SDS,包括Ceph、Amazon EC2、阿里的盘古等。

\hrulefill

快照,从COW到ROW到LSV,ROW用户数据共享,LSV不仅用户数据共享,而且数据出现多版本,所以需要GC。

ROW2和ROW3,都是带两级bitmap index。不同在于ROW2在新写时,按chunk发生了COW,
ROW3在新写入时不发生COW,但导致更严重的碎片化,影响读性能。

写有对齐写、不对齐写、overwrite覆盖写。读有对齐读和不对齐读。\hl{不对齐的写和读需要特别处理}。

\hrulefill

\hl{按对象、块、文件遍历本地和分布式系统},更复杂的NoSQL、NewSQL,暂时没有精力照顾到,后期也需要整理。
从哪些维度着眼?\hl{分布、副本/EC、sharding、快照}等。

先有disk,再有fs,fs里继续有disk(设备文件),我中有你,你中有我的关系。

怎么理解object、block、file之间的关系?\hl{NAS和SAN}在实际的应用场景中,有什么特别之处?
大家又是怎么使用的?

\hrulefill

\hl{纸上得来终觉浅,绝知此事要躬行}。看书、源码的时间要慢慢过渡到分析问题、解决问题。
头脑中要有big picture,在此基础上深化。如果没有这个,很容易迷失在技术的细枝末节里无力自拔。

更基础的知识放在长期的学习计划里,定量有恒,日积月累,定有所获。

\hl{戒定慧,此三学六度},实在是度人的船阀,还有很深刻简练的吗?
故称三无漏学,其道甚大,百物不废。何必舍近求远?实乃没有甚深定力和慧解的表现。

应渐渐地少看书了,多动脑、多动手,在解决问题的过程中,扩展知识和能力的边界。
(这段描述,有所不为而后可以有为,然后有所得) 知止是高级智慧。

AFA是方向,相关技术包括\hl{RDMA/DPDK/SPDK,NVMe、iSER、NVMf}等。
这些方面应该是下一步的重中之重,道不远人,如果舍近求远,恐有缘木求鱼的危害。

\subsection{18}

cache一致性与副本一致性存在很大的不同。cache一致性出现在共享内存多处理器、分布式文件系统client cache等场景。
副本一致性出现在replication和EC等场景。

cache一致性关注的是一个数据源,多个cache时的行为。副本一致性表示多个副本之间的一致性。
cache一致性可以用lease、oplock,可靠多播,MESI协议解决,副本一致性用RAFT协议解决。

\hl{完善性能分析和故障诊断toolbox}。

按\hl{主体-对象模型},一致性可分为观测一致性和数据一致性。

\hl{CAP和ACID},深入理解种种细分情形。

\hrulefill

lich meta里记录的是chkid到nids的映射,每个nid通过db查询到实际的disk loc。
相当于两级映射。

\hl{chunk tree过于复杂},难以做到事务性的要求。
allocate的时候,有可能会连续更新多级chunk。
迁移或故障的时候,也会引发级联更新。

lich无日志,如何保证一致性的?

\hl{lich chunk内并发},chunk内非覆盖区域,clock的连续性,理论上可以或并发或批量写入。
有一次排序、精简、聚合、并发的机会。覆盖区域,按fifo顺序提交。
如clock不连续,则需要等待。(\hl{paxos的日志写入,放松了该要求})

每个term包括\hl{选举、恢复、正常操作}诸阶段。

\hl{如何减少故障下的io中断时长}?admin、vctl可能发生切换、或reload。需要case by case分析。

实现LSV时采用了该方案。\hl{如果io size不规则,当如何}?
log和bitmap的写入,采用了pipeline模式。以确保log的并发提交,bitmap的顺序提交。

rcache比较复杂,开始按不同size建立多个cache是不明智的,应统一采用page cache,易于管理。
不过可以加入readahead等策略。

EC/dedup/compress/都需要建立新的映射。share底层数据块,但造成严重的碎片,对顺序读不友好。

\hl{回顾open vstorage的映射关系}。zerocopy snapshot?索引、共享、多版本?
多版本比共享有更复杂的引用关系。

\hrulefill

建立知识体系,依次scan各个分支领域。
\begin{enumbox}
\item \hl{CAP consensus, include paxos, RAFT and lease}
\item ACID transaction ARIES
\item metadata管理,如GFS、BigTable等。
\end{enumbox}

\hl{这次彻底把Paxos研究透彻,不能在一个地方跌倒两次。
否则就是有勇无谋,把握不住重点所在}。

戒定慧三字深邃,作为下一阶段的座右铭。

\subsection{19}

庄子内七篇、六祖坛经,从专业的角度去解读,很有意味。工匠精神、企业家精神是一而二、二而一。
回归专业是正确选择,一直知道、一直不能很好执行,知行不能合一。这次不能再偏离。

从分布式存储开始,建立问题和知识联动的体系,以助力其后的职业发展。

\subsection{21}

怎么查看cs和中断信息?

如何处理slow disk?

性能之颠:\hl{可观测指标=f(资源,workload)}。资源有使用率,基于时间(排队论?)、或容量。
在software pipelines里,workload是上游,可观测指标是下游,架构影响性能(化学)。

资源:\hl{CPU,memory,fs,disk and network}。

多网卡,MPIO。如何利用多块网卡?

面试经验谈
\begin{enumbox}
\item BAT,TMD,浪潮,联想等。
\item 仔细整理做过的东西,要更深入。不熟悉的东西宁可不写。要非常严谨
\item 用的什么型号的设备,测试的性能。
\item RDMA是重点,建立连接的流程,遇到的坑。
\item 基础编程题,本质上都是工程师
\item 要知道lich的不足之处
\end{enumbox}

引入kv用于元数据管理,加强client功能,client可以与object直接通信,更大的chunksize。
rich client,目前vctl就好像是这样的rich client。

恢复和数据平衡都依赖于控制器平衡,才能有更好的并发度?

\subsection{22}

CAP平衡ACID和BASE,进而引入单副本一致性问题,RSM的解决引入paxos、raft。

独立出metadata(KV)后,rich client,相当于controller前移到client,
如果允许多个client并发open一个LUN,会涉及一致性问题。

sheepdog采用DHT,不方便scan,所以recovery不好弄?

\subsection{24}

偏爱非对称架构,如有元数据服务器。再如SMP和NUMA。

控制器的粒度,从传统array、卷控制器,到primary osd。

io size影响到怎么处理,如果两个并发io发生了覆盖,则必须排序,然后在多个副本上按同一顺序apply。
\hl{如果不发生覆盖,则怎么执行都不会影响到最终结果}?元数据更新的io size通常较小,如64B。

ceph中pool不能过大,\hl{把pool限定在一个较小的节点集},避免过多的节点导致故障概率增加。

存储索引结构,\hl{btree,lsm},hash,bitmap、bloom filter。
bloom filter用在允许误判的情况下判定一个元素的存在性。

bcache用btree做什么?
bLSM是LSM和btree的组合体。

学习数据结构和算法,要自己动手推导,三部曲:\hl{猜想、重建、证实}。

\hrulefill

\hl{edog架构}。通过mnesia数据库管理vm和用户数据。
libvirt用于控制qemu-img,可以用xml配置文件。云管理平台通过libvirt来创建、启动、关闭、销毁vm。
qemu通过driver请求各种后台存储,如\hl{rbd、sheepdog、fusionstor}等。

\hrulefill

vfs不仅对理解文件系统有帮助,对理解任何存储系统都是有帮助的,包括对象、块等。
superblock是引导信息。inode维护着dir/file/volume到数据块的映射,dentry建立了global namespace。

zfs的pool化,\hl{FS/Volume}共存的架构令人耳目一新。传统的\hl{Volume/FS架构}。

存储有几级映射:
\begin{enumbox}
\item namespace (tree)
\item file to object mapping
\item object to location mapping
\end{enumbox}

比如ceph引入了pg,又加入了一层。bcache虚拟地址到物理地址的mapping。

ext2/3/4的间接块、extent tree数据结构。ext4区分inode为dir和file,
分别用htree和extent btree来实现。由此可见\hl{btree作为通用index结构}的强大之处。

\hrulefill

现在可认为建立了相对完善的知识体系。接下来就要围绕几个核心问题,
查漏补缺,\hl{致广大而尽精微},打造T型知识结构
\begin{enumbox}
\item 体系结构、操作系统、编程语言、数据结构和算法
\item 网络和分布式系统
\item 存储
\item TRIZ
\end{enumbox}

Lich需要进一步加入理解的地方
\begin{enumbox}
\item 数据模型
\item 数据分布
\item \hl{IO PATH}
\item 控制器架构
\item 本地磁盘空间管理
\item 副本一致性协议 (vctl, clock)
\item 恢复( +VFM )
\item 平衡
\item QoS(token bucket and leaky bucket)
\item 快照:COW、ROW、LSV
\item Slow and Broken Disk Detecter
\item ***
\item Redis Engine
\item BCache
\item Tier
\item EC
\item VDO
\item RR (sync and async)
\item ***
\item SCSI/iSCSI/iSER
\item NVMf
\item ***
\item Scheduler
\item Memory Allocator and Hugepage
\item RPC and corerpc (over TCP or RDMA)
\item RDMA/DPDK/SPDK
\item Multi NIC
\item MPIO
\end{enumbox}

\hrulefill

\hl{把操作系统作为主要隐喻},按操作系统的视角去看待形形色色的系统。
操作系统管理物理资源,提供syscall供上层应用使用。
在操作系统的层面有最好的抽象,如file、virtual memory and process。

\hl{分布式操作系统}会更强大,在OS之上加入分布式,则形成\hl{两层的知识体系},有分有总。
网络流,节点和边构成分布式,细化节点形成现代OS的设计。

\hl{分布式操作系统,简称操作系统},是看待世界的系统方法。

\hl{深耕操作系统这块良田},\hl{一分耕耘一分收获}。
这里的OS,既指实际的操作系统,也是一个更为广泛的隐喻。
坎贝尔神话隐喻予人以重大启示,终于找到了现阶段最重要的隐喻。
任正非是管道,OS是虚实结合的产物。

\hl{操作系统是软硬件协同设计的典范},体系结构、编程语言、存储、数据库和网络都是这个生态的积极参与者。
数据结构和算法。

用OS这个隐喻具有最大的开放性,同时又是专注的。
\hl{炼金术士手中的操作系统}整合了资源,且有着预定价值输出。
\hl{OS具有一即一切一切即一的大气魄}。

\subsection{25}

操作系统包括\hl{计算、存储、网络、协作}。从协作的视角来看,分布式系统算法。

卷管理 disk, raid and LVM,ZFS

卷的两种使用方式,一是Filesystem,二是iSCSI,分别对应NAS和SAN。
NAS文件系统在内部,SAN文件系统在外部。\hl{SAN通过NAS网关即可把NAS包含在内}。

体系结构、编程语言都是OS应有之义。OS封装体系结构供PL使用它的服务,OS又是PL构建的,\hl{Arch、OS、PL构成三连环}。
OS提供了最有用的抽象,File、Virtual Memory、Process and VM。

\hrulefill

\hl{删除卷}不同于一个一个free object,整体删除应有更便捷的方式。
在etcd上产生一个删除卷的JOB,包括每个节点上的task。由每个节点调度执行。

每个节点上,周期性地polling相关任务,加入本地任务队列。由独立线程或线程池处理该队列。
\hl{每个task建模为状态机}。处理完一个task后,从etcd上删除该task。

从replica cache、db和disk里删除vol相关记录。
注意操作顺序,应先解除ref关系,再回收disk空间。

\hl{删除pool}也应采用batch方式。保证后置条件,执行后不再有该pool数据。
因为pool id采用了name,\hl{what if}如果在此过程中,又创建了同名pool,会如何?
最简单的做法是禁止这样的行为。

\hrulefill

能用int表示的,会方便很多,如nid、diskid,需要保证在作用域内的唯一性。

CAS lock?

抽象,庄子齐物论

\hrulefill

bcache,拔盘后虚拟盘依然可写?

\hl{EIO下踢盘},这样做会引起什么问题?合适吗?
踢盘后,如果引起部分对象副本不可用,应该如何处理?

画草图,保留自己的笔记!

\subsection{26}

单核、SMP和NUMA,共享内存多处理器

先研究各类调度器,如linux进程调度、io调度,erlang/go协程调度,k8s资源调度等。

linux进程调度,要处理MP架构,减少上下文切换和cache miss等问题。

erlang/go的协程调度,是建立在内核线程上的非抢占式调度,支持大量协程调度,还要处理堵塞io带来的问题。
即\hl{调度与内存和io}要结合起来考量。io是通过异步提交和事件驱动的方式做的。
coroutine虽然是非抢占式的,但可以通过\hl{yield和resume机制}主动让出,由外部事件来唤醒。

allocator包括内存管理、磁盘、卷管理,乃至文件系统。

\subsection{28}

协议了解少,包括\hl{SCSI、NFS、CIFS}等,须强化。
RDMA/DPDK/SPDK、NVMe等慢慢成为主流

\hrulefill

分布式系统概念与设计。

分布式系统和操作系统是基础,需要专精。\hl{一门深入、长时薰修}。

理解分布式系统的起点是CAP定理,为了HA,引入副本,为了性能、引入cache。如此就要进一步解决带来的一致性问题。

\hl{CAP是思考原点},引入一致性、容错、可用性、性能,弹性、可伸缩性等等。
引入safety and liveness、ACID and BASE。
引入replication and cache,进而引入consensus (zk、etcd等)。paxos是核心算法。

分布式系统的很多手段在RAID里已经采用,如条带化、镜像、校验码。分别对应\hl{分块、复制/EC}。

Erlang/OTP在解决分布式系统的痛点上是怎么做的?如何拥抱失败?
协程调度,适应MP体系结构,同时把mm和io纳入考虑。进程有mailbox,异步通信。
\hl{go与此有所不同,采用了csp范式},把channel显式地提了出来,增加了灵活性(Q)。

分析磁盘故障的处理策略:如何检测?恢复和io策略?
\begin{enumbox}
\item 出现EIO应该马上踢盘吗?
\item 检测慢盘坏盘的方式是什么?
\item 无clean副本时,下一步该如何处理?
\item 一次恢复失败,接下来怎么办? retry, redo(all of disk)
\end{enumbox}

两种模式:强一致性和降级。如果采用强一致性,会堵塞io直到恢复完成,iops下降严重。
如果指定最小副本数,\hl{运行于降级模式,采用异步恢复策略}。就影响一致性,无法容忍连续故障。
\hl{性能是可用性的一个体现,这是CAP认识的深化}。
何况CA不是铁板一块,而是分层分模块有结构的,每个接口都需要单独分析,
在总的指导思想下,case by case地去处理。

节点故障:不同于磁盘故障,需要全面scan发现待修复的chunk。

两层:分布式和节点。

熟悉lich模块
\begin{enumbox}
\item scheduler
\item memory
\item io (normal, nvme, spdk)
\item rpc (including \hl{TCP/IP/RDMA/DPDK} etc)
\item ***
\item application protocols, iSCSI/NFS/CIFS
\end{enumbox}

每个core线程整合了scheduler和io,统一用epoll方式驱动(eventfd, timerfd)。
不同core线程通过corerpc进行通信。

为什么要依赖于物理时间戳?每个host时间相差要在5s以内?

知识体系:
\begin{enumbox}
\item 横(\hl{编程语言、算法、设计模式、架构})
\item 存储协议
\item 分布式系统
\item 本地引擎
\end{enumbox}

\subsection{29}

\hl{RDMA对NIC或switch有要求}。

DPDK是用户态TCP,用在以太网环境下。SPDK也是用户态NVMe驱动,并提供了NVMf。
如果不引入支持RDMA的交换机,在all以太网环境下,可以用SPDK/DPDK分别实现\hl{全用户态的存储和网络}。

输出价值的能力是最重要的能力。

没有时间深入研读DPDK/SPDK,只能做大体上的把握。
第一个问题就是\hl{如何在用户空间实现设备驱动}?

每个进程的地址空间分两部分:用户态和内核态。内核态只有一个地址空间,映射到不同的进程。
\hl{不同进程share一份内核空间},多进程trap后并发执行。

pci设备,通过mmap可以让用户态代码访问。\hl{每个pci设备都是映射到内核的一块内存区域},可以把该内存区域映射到进程地址空间。

能否在polling模式和中断模式之间动态切换?\hl{用户态接收不到中断,由uio接管}。polling模式检查状态寄存器的值。

缓冲区溢出攻击

\subsection{30}

\subsection{31}

\chapter{2018}

\section{书单}

古籍
\begin{enumbox} 
\item 大学中庸
\item 黄帝阴符经
\item 鬼谷子
\item 大乘起信论
\item 曾国藩家书
\end{enumbox} 

学会学习
\begin{enumbox}
\item 东尼·博赞学习系列
\item NLP: 复制卓越的艺术
\item NLP执行师
\item 概念地图在教学和学习中的应用
\end{enumbox}

专业
\begin{enumbox}
\item Paxos/RAFT
\item 分布式系统: 概念与设计
\item Linux/UNIX系统编程手册
\item 性能之巅
\item 深度学习
\end{enumbox}

个人成长
\begin{enumbox}
\item 坎贝尔千面英雄
\item 卓越元素
\item 洛克菲勒家信
\item 摩根家信
\end{enumbox} 

管理
\begin{enumbox} 
\item 达里奥原则
\item 法尔科尼管理方法
\item 阿代尔
\item 马利克
\item 好战略坏战略
\item 管理十诫
\item 为将之道
\item 活着/洞见/觉悟
\item 赋能
\end{enumbox} 

投资
\begin{enumbox} 
\item 李笑来财务自由之路
\item 穷查理宝典
\item 查理芒格的原则
\item 投资最重要的事
\item 聪明的投资者
\end{enumbox} 

历史和传记
\begin{enumbox} 
\item 中共党史
\item 陈云
\item 富兰克林自传
\end{enumbox} 

哲学
\begin{enumbox} 
\item 柏拉图对话集
\item 亚里士多德形而上学
\item 尼各马可伦理学
\item 新工具
\item 笛卡尔谈谈方法
\item 笛卡尔第一哲学沉思集
\item 斯宾诺莎伦理学
\item 斯宾诺莎知性改进论
\end{enumbox} 


\chapter{Wish}

WOOP

\section{反思}

博而寡要,劳而无功,知行脱节,有术而无道,术也就不可靠了,以往解决问题陷入物的支离境地,这是阳明学要解决的大问题。

迷信书本,买了那么多书,根本消化不了,浪费了大量的时间和精力,劳而无功。
书要看,但并非必须,都是要牢牢把握住为学头脑处。
盲目地去学习,反不如静下心来,仔细规划。

花在抱怨的时间秀的时间太多,当回归到修身为本。\hl{涵养须用敬,进学则在致知}。
此二语是今后的指导思想,少说大言,收敛身心,但踏踏实实做去,真积力久,自然或跃在渊。

遥想十年前,身处天津,前途茫茫,没有着力点。现在情形则要好很多,正是用心用力之时,不可随意荒废。

知识体系只是构成了一个背景,能体现价值的地方,还是要看解决了什么大问题,大问题是成长之树的树根。
看看黎曼猜想引发的全球关注,即可知。战略思维的大开大合,离不开大问题的驱动。

围绕着要解决的大问题,如何打造专业知识体系?

\section{立志}

富有之谓大业,日新之谓盛德,生生之谓易。

10X成长

\section{工具}

\subsection{参伍以变}

双线法则、圆点哲学、参伍以变形成为主要的方法论。

\subsection{六时书}

\begin{tcolorbox}
这是唯一让我们能够实实在在得到智慧的好处的最方便可行的方法, 如果不去记录六时书, 
我们实在是 “如入宝山, 空手而归”,让人可惜得扼腕而叹。

六时书可帮助我们建立“追踪体系”——帮助我们训练意识,因为身为在地球上的人,
几乎每天都是负面念头多于正面念头的,也就是说这是自然人的本来性质,
但这一性质必定导致大多数人陷入负面事件越来越多导致产生更多负面念头的恶性循环里。

这个规律就如同地心引力一样,是一个定理——一个大多数人都逃不开的规律。
但是人类既然能够逃脱地心引力上天入地,同样的,我们也可以摆脱这个“负多于正”的规律,
通过六时书的帮助——训练我们的念头,使负面念头越来越少,正面念头越来越多,
成为一个觉悟的、开悟的人。到时候,生活中不论发生什么事,
我们都能以最正确的态度去对待,一切问题都将不是问题。
\end{tcolorbox}

六时书这种结构化的方式较好,考虑问题趋于全面。

示例: 专心致志,不抱怨

\begin{lstlisting}
+ 正
- 反
O 合
\end{lstlisting}


\end{document}

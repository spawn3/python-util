% -*- coding: UTF-8 -*-
% hello.tex

\documentclass[UTF8]{ctexbook}

\usepackage{xeCJK}
\usepackage[utf8]{inputenc}

% load paralist before enumitem
\usepackage{paralist}

\usepackage{hyperref}
\hypersetup{pdftex,colorlinks=true,allcolors=blue}
\usepackage{hypcap}

\usepackage{color}
\usepackage[usenames, dvipsnames, svgnames, table]{xcolor}
% \pagecolor{gray}

\usepackage{makeidx}
\makeindex

\usepackage{amsmath}
\usepackage{mathtools}

\usepackage{listings}
\usepackage{multicol}
\usepackage{fancybox}
\usepackage{tcolorbox}
\usepackage{enumitem}
\usepackage{multirow}
\usepackage{longtable}

\usepackage{indentfirst}

% table
\setlength{\arrayrulewidth}{1pt}
\setlength{\tabcolsep}{16pt}
\renewcommand{\arraystretch}{2.5}
\newcolumntype{s}{>{\columncolor[HTML]{AAACED}} p{3cm}}

\arrayrulecolor[HTML]{DB5800}

% 摘录
\usepackage{verbatim}
\usepackage{libertine}
\usepackage{graphicx}
\usepackage{framed}

\lstset{%
    %alsolanguage=Java,
    %language={[ISO]C++}, %language为,还有{[Visual]C++}
    %alsolanguage=[ANSI]C, %可以添加很多个alsolanguage,如alsolanguage=matlab,alsolanguage=VHDL等
    %alsolanguage=tcl,
    %alsolanguage=XML,
    %alsolanguage=bash,
    tabsize=4, %
    frame=shadowbox, %把代码用带有阴影的框圈起来
    commentstyle=\color{red!50!green!50!blue!50},%浅灰色的注释
    rulesepcolor=\color{red!20!green!20!blue!20},%代码块边框为淡青色
    keywordstyle=\color{blue!90}\bfseries, %代码关键字的颜色为蓝色,粗体
    showstringspaces=false,%不显示代码字符串中间的空格标记
    stringstyle=\ttfamily, % 代码字符串的特殊格式
    keepspaces=true, %
    breakindent=22pt, %
    numbers=left,%左侧显示行号 往左靠,还可以为right,或none,即不加行号
    stepnumber=1,%若设置为2,则显示行号为1,3,5,即stepnumber为公差,默认stepnumber=1
    %numberstyle=\tiny, %行号字体用小号
    numberstyle={\color[RGB]{0,192,192}\tiny} ,%设置行号的大小,大小有tiny,scriptsize,footnotesize,small,normalsize,large等
    numbersep=8pt, %设置行号与代码的距离,默认是5pt
    basicstyle=\footnotesize, % 这句设置代码的大小
    showspaces=false, %
    flexiblecolumns=true, %
    breaklines=true, %对过长的代码自动换行
    breakautoindent=true,%
    breakindent=4em, %
    escapebegin=\begin{CJK*}{GBK}{hei},escapeend=\end{CJK*},
    aboveskip=1em, %代码块边框
    tabsize=2,
    showstringspaces=false, %不显示字符串中的空格
    backgroundcolor=\color[RGB]{245,245,244}, %代码背景色
    %backgroundcolor=\color[rgb]{0.91,0.91,0.91} %添加背景色
    escapeinside=``, %在``里显示中文
    %% added by http://bbs.ctex.org/viewthread.php?tid=53451
    fontadjust,
    captionpos=t,
    framextopmargin=2pt,framexbottommargin=2pt,abovecaptionskip=-3pt,belowcaptionskip=3pt,
    xleftmargin=4em,xrightmargin=4em, % 设定listing左右的空白
    texcl=true,
    % 设定中文冲突,断行,列模式,数学环境输入,listing数字的样式
    extendedchars=false,columns=flexible,mathescape=false
    % numbersep=-1em
}


\newenvironment{enumbox}[0]{
    \begin{tcolorbox}
    \begin{compactenum}
} {
    \end{compactenum}
    \end{tcolorbox}
}

\newenvironment{itembox}[0]{
    \begin{tcolorbox}
    \begin{compactitem}
} {
    \end{compactitem}
    \end{tcolorbox}
}

\newcommand{\hl}{\bgroup\markoverwith
  {\textcolor{yellow}{\rule[-.5ex]{2pt}{2.5ex}}}\ULon}

\newcommand*\openquote{\makebox(25,-22){\scalebox{5}{``}}}
\newcommand*\closequote{\makebox(25,-22){\scalebox{5}{''}}}
\colorlet{shadecolor}{Azure}

\makeatletter
\newif\if@right
\def\shadequote{\@righttrue\shadequote@i}
\def\shadequote@i{\begin{snugshade}\begin{quote}\openquote}
\def\endshadequote{%
\if@right\hfill\fi\closequote\end{quote}\end{snugshade}}
\@namedef{shadequote*}{\@rightfalse\shadequote@i}
\@namedef{endshadequote*}{\endshadequote}
\makeatother

\title{成长}
\author{炼金术士}
\date{\today}

% \bibliographystyle{plain}
% \bibliography{math}

\begin{document}

\maketitle
\tableofcontents

\chapter{技术}

在一个大的技术背景下,看分布式存储系统的理论和实战。

操作系统
\begin{enumbox}
\item Linux
\end{enumbox}

存储系统
\begin{enumbox}
\item FusionStor
\item Ceph
\end{enumbox}

共识
\begin{enumbox}
\item etcd
\item zookeeper
% \item Ceph Monitor
\end{enumbox}

文件系统
\begin{enumbox}
\item ext4 
\item xfs
\item fuse
\item CephFS
\item GlusterFS
\item Lustre
\item HDFS
\end{enumbox}

NoSQL
\begin{enumbox}
\item RocksDB/LevelDB
\item MongoDB
\item Redis
\item Hadoop
\end{enumbox}

NewSQL
\begin{enumbox}
\item OceanBase
\item PingCAP
\end{enumbox}

WebServer
\begin{enumbox}
\item Nginx
\item ElasticSearch ELK
\end{enumbox}

\chapter{志于道}

诸子:
\begin{enumbox}
\item 易经
\item 中庸
\item 道德经
\item 文子
\item 黄帝四经
\item 黄帝阴符经
\item 鬼谷子
\item 管子
\item 素书
\item 武经七书
\item 长短经
\item 武艺二书
\end{enumbox}

佛经
\begin{enumbox}
\item 心经
\item 金刚经
\item 坛经
\item 大乘起信论
\end{enumbox}

近人之著作
\begin{enumbox}
\item 东方战略学
\item 李小龙
\item 查理芒格
\item 孙正义
\item 人生算法
\item 第一性原理
\item 第二曲线
\item 基业长青
\item 从优秀到卓越
\item 黑天鹅
\item 反脆弱
\end{enumbox}

道是起点,也是归属。

志于道、据于德、依于仁、游于艺,此四事,实是一事,志于道,道之展开,囊括无遗。

立志,立何等志?本立而道生。道是目的也是方法。

道如太阳,行星围绕着它运转不息。

信解行证,道是一种信仰,生起好奇心,去了解、去上下求索、去身体力行。

道者,生生之道,增长之道,一气流行阴阳变化之道。
一者,道之纪也。

道是一中心范畴,千变万化而不离其本,是环的心,也是环自身。道有其体,又有在各个领域里的广泛应用。

深入道源去原道。观天之道,执天之行,尽矣!以天地之道去洞察万物,则易知易行。

道是人生算法,也是第二曲线的哲学;是第一性原理,也是10X增长。

\chapter{序章}

一学和PDCA构成五行结构,天有五贼,见之者昌。

轩辕三书

五轮书

一门深入、长时薰修

因指见月,目的在于见月。

\chapter{原则}

原则近似于我之所谓道,人何以之道?曰:心。心何以知?曰:虚一而静。
精于道者兼物物,一于道而以赞稽之,则万物官矣。

\section{尊道}

道德仁义礼,五者一体也,而道为之主,故第0原则即是尊道。
然何以尊之?需明法。

老子七善,三学六度

\subsection{至诚不息}

君子养心莫善乎诚,致诚则无它事也。

实事求是,拥抱现实,超越现实。

\subsection{虚壹而静}

是大原则,每临大事有静气,不信今时无古贤。

心善渊

将军之事,静以幽,正以治。

\subsection{机器之喻}

欲收无为而治之效,不能不着重在打磨机器、系统上,建立系统思维。
自组织、自进化的系统是工作的产物。

\section{工作之道}

以终为始

要事第一

全局优化(统合)

\section{生活之道}

闲居静思则通

\chapter{算法}

\section{概述}

人生需要核心算法。核心算法解决人生会遇到的大问题,最重要的一个问题就是如何更好的成长。

\subsection{模型}

顶级思维模型。

处处可见圆点哲学的影子。矛盾分析法,双线法则,两眼论,
一阴一阳之谓道,万物负阴抱阳冲气以为和。

回到核心,从核心出发。找到核心的过程,一靠直觉,二靠试错,低成本地试错。

老子第十六章,义理丰富。

以正治国,以奇用兵,以无事取天下。此意渊深,可为座右铭。

孙子兵法提供了一套方案,孙正义有自己的归纳总结。我欲清溪寻鬼谷,不论礼乐但论兵。
兵者是死生存亡的大事,社会暗流涌动,在温泉面纱下,竞争不可谓不激烈。
想要活出自己的人生,不能不考虑更重要的维度。

孙陶然五行管理兵法。

达里奥原则,培根新工具,笛卡尔方法谈,斯宾诺莎伦理学都旨在解决人生算法问题。

开发人生算法,喻颖正做出了表率。按守破离的节奏,先对标,再突破。最小核心最大化,
先要感知自己的核心,通过反复练习,使之价值最大化。

高筑墙,广积粮,缓称王。

\section{战略要素}

\subsection{目标}

第一,列出最重要的五个目标。双列表,10/10/10原则。

事业有成是因,财务自由是果。找到自己的核心算法,其它一切则水到渠成。
反复打磨核心,可以用爱因斯坦质能方程来描述:E=mc\^2。m是核心,c是大量重复练习,E是果。

\subsection{资源}

整合资源:客户,钱,人脉,客户。

\section{将略}

慎言!养成深沉厚重之心态。


\chapter{战略}

专业学习之外,把战略研究列为今后几年的重点。


为什么?

因为战略很重要,战略赢是大赢,战略输是大输。孙子:兵者,国之大事,死生之地,存亡之道,不可不察也。
不具备强大的战略思维能力,就很难实现远期的发展目标。

多聚变为一,一裂变为多,分合为变。以正治国,以奇用兵,以无事取天下。战略明,则可无事。否则,陷入事务之中,而无结果,可悲。

欲研习战略,需读经典,多实践,请教高人,开阔视野,放大格局。在解决现实问题中,融会贯通,知行合一。持续优化,不断把认知引向深入。
道德经,孙子兵法,商君书等,皆为经典。西方亦有经典,然不够精炼。待心有所主,则可进一步泛观博览。第一步,则在知止,懂取舍,有所不为。

战略罗盘之喻,精当。战略几何学,形象生动。柏拉图言:不懂几何者不得入内。以几何去研习战略,可收简洁精当,生动形象之效。

专业学习和战略研究,可谓一文一武,一阴一阳,一张一弛,相互促进,相得益彰。
战略研究要有更多的问题意识和进取精神,不仅仅是知识的获取,而为我所用,服务于最终目标的实现。
战略研究的境界,可用中庸的致广大而尽精微,极高明而道中庸形容之。
层次则有历史,科学,艺术,哲学。

\section{问题}

大学:物有本末,事有终始,知所先后,则近道矣。认识事物的轻重缓急,按其行动,就接近道了。按重要性和必要性维度进行分解,是柯维几本书的一个重要内容。
李笑来在财富自由之路中,一语中的,作为终极问题。对该问题的反复审问,是磨练价值观的利器,有助于提高选择和决策能力。别的问题都是术,这个问题近乎道。

所谓选择,即是增加必要的条件。尽量必要,尽量充分,最小完备集,奥卡姆剃刀原则在决策问题上的应用。李笑来所说的万能钥匙,即是NLP的换框法,转换视角。

\section{维度}

何为维度?升维思考,降维攻击。把重要维度都列出来,从中选择优势维度,扬长避短,有所取舍,特色组合,从而构建核心竞争力。
价值链分析如此,蓝海战略也如此。互联网,成本等都可以成为分析的重要维度,如差异化竞争,互联网对+,免费等常用的竞争策略。

维度,或者说条件,要素,李笑来有个认识:增加条件。

整合,跨界,爆裂,裂变都是这个核心思想的变形。借助技术手段,多维整合为一维,或一维细分为多维。
技术,市场和自然所谓3M力量,在塑造未来世界。

升维思考,降维贯通。维度一概念,用代数方式研究,线性空间。在数学和战略研究之间,架起会通的桥梁。数学和战略之间,有着深刻的联系。
构造结构,识别模式和关系,掌握变化的趋势。

蓝海战略的价值链分析,波特的五力分析着眼于产业竞争分析。价值链分析通过加减乘除来构建自己的优势维度,有所为有所不为。

维度是立体的,层次的,从分形的角度看,不仅仅有整数维,也有分数维,涌现出奇异的系统特性。

当代科学进展,描述了令人惊奇的时空结构。按照量子力学和相对论的认识,物质,能量,时空都呈现了超出直觉的功能和特性。
对微观粒子结构的认识在深入,对宏观时空结构的认识也在深入,在极小和极大的时空尺度上,都存在一些真正的大问题。

所谓认知,就是对维度的认识。志如其量,量如其识,三位一体,相互影响。量,开放心态,识,见识,认知水平。

\section{老子之道}

战略研究有层次境界之别,重点是理解并运用老子的一句话:道生一,一生二,二生三,三生万物。
其中一是重点的重点。侯王得一以为天下正,得一,则万事毕。

一生二,因二以济民行。一个模型是双线法则,守正出奇。守底线,抓关键。修道保法,故能为胜败之政。

老子之道,博大精深,内涵基本原则。马利克的管理,植根于系统论、控制论和仿生学等现代科学,耗散结构是另一值得注意的。
道生一,一即战略;精心守一,参悟商道。以道为中心,通达战略和管理,促进成长和进化。

半部老子治天下,围绕老子,通达道家思想。没有厚此薄彼,抱一而为天下式。老子文约义丰,且较为熟悉。

为自己打造一口深井,由老子承载大学之道,修齐治平,尽在其中。

从认知升级的维度读老庄,会有趣得多。

\section{圆点战略}

圆是格局,点是破局。圆是调查分析,点是指导方针和行动序列。

\section{双线法则}

\section{西方战略管理思想}

战略始于问题。战略七问,德鲁克五问,如何回答这些问题需要深入分析。黄金圈法则,明确了问题的顺序。

受到商业机构成功的启发,比尔·盖茨在今年的公开信中提出了一套“成功法则”:量化目标―选择策略―考量结果―调整策略―实现目标。
比尔·盖茨认为,这不仅仅是商业机构成功的秘诀,致力于扶贫帮困解决社会问题的非营利机构同样应遵循这一法则。

PDCA是唯一的管理方法,法尔科尼管理方法。

双环学习,前提批判,在表面原因之外,有更深层的原因或约束。
在尝试解决问题之前,需要更深入的调查分析,找到问题的根本原因,而不是流于表面,治标不治本。这引向了系统动力学的视野。

战略是可行性的假设,需要持续的压力测试,来检验其正确性和有效性。

机器的隐喻,有机体的正常运转。可以把组织看作一台机器或有机体,机器是在正常运转吗?
用控制论去理解,控制论适用于机器和生命领域。

\chapter{精选}

诸子:
\begin{enumbox}
\item 易经
\item 中庸
\item 道德经
\item 文子
\item 黄帝四经
\item 黄帝阴符经
\item 鬼谷子
\item 管子
\item 素书
\item 长短经
\item 太极图说
\item 通书
\item 武经七书
\item 武艺二书
\end{enumbox}

佛经
\begin{enumbox}
\item 心经
\item 金刚经
\item 坛经
\item 大乘起信论
\item 圆觉经
\end{enumbox}

近人之著作
\begin{enumbox}
\item 一二三哲学
\item 东方战略学
\item 李小龙
\item 查理芒格
\item 孙正义
\item 人生算法
\item 第一性原理
\item 第二曲线
\item 基业长青
\item 从优秀到卓越
\item 黑天鹅
\item 反脆弱
\end{enumbox}

\section{太极图说}

无极而太极。太极动而生阳,动极而静,静而生阴,静极复动。
一动一静,互为其根。分阴分阳,两仪立焉。
阳变阴合,而生水火木金土。五气顺布,四时行焉。

五行一阴阳也,阴阳一太极也,太极本无极也。
五行之生也,各一其性。\hl{无极之真,二五之精,妙合而凝}。
乾道成男,坤道成女。二气交感,化生万物。万物生生,而变化无穷焉。

惟人也得其秀而最灵。形既生矣,神发知矣。五性感动,而善恶分,万事出矣。
圣人定之以中正仁义而\hl{主静},立人极焉。

故圣人与天地合其德,日月合其明,四时合其序,鬼神合其吉凶。君子修之,吉;小人悖之,凶。

故曰:“立天之道,曰阴与阳。立地之道,曰柔与刚。立人之道,曰仁与义”。

又曰:“原始反终,故知死生之说”。

大哉易也,斯之至矣。

\section{太极拳谱}

\hl{太极者,无极而生,动静之机,阴阳之母也。
动之则分,静之则合}。无过不及,随曲就伸。
人刚我柔谓之走,我顺人背谓之粘。
动急则急应,动缓则缓随。
虽变化万端,而理唯一贯。
由招熟而渐悟懂劲,由懂劲而阶及神明。
然非用力之久,不能豁然贯通焉。

虚领顶劲,气沉丹田。不偏不倚,忽隐忽现。
左重则左虚,右重则右杳。
仰之则弥高,俯之则弥深,进之则愈长,退之则愈促。
一羽不能加,蝇虫不能落,人不知我,我独知人。
英雄所向无敌,盖皆由此而及也。

斯技旁门甚多,虽势有区别,概不外乎壮欺弱,慢让快耳。
有力打无力,手慢让手快,皆是先天自然之能,非关学力而有为也。
察四两拨千斤之句,显非力胜;观耄耋能御众之形,快何能为。
立如平/秤准,活似车轮。偏沉则随,双重则滞。
每见数年纯功,不能运化者,率皆自为人制,双重之病未悟耳。

欲避此病,须知阴阳。粘即是走,走即是粘。
阴不离阳,阳不离阴。阴阳相济,方为懂劲。

懂劲后,愈练愈精,默识揣摩,渐至从心所欲。
本是舍己从人,多误舍近求远。
所谓差之毫厘,谬之千里,学者不可不详辨焉。

\section{阴符经}

\subsection{原文}

\hl{观天之道,执天之行,尽矣}。
故天有五贼,见之者昌。
五贼在乎心,施行于天。宇宙在乎手,万化生乎身。
天性,人也;人心,机也。立天之道,以定人也。
天发杀机,移星易宿;地发杀机,龙蛇起陆;人发杀机,天地反覆;天人合发,万变定基。
性有巧拙,可以伏藏。九窍之邪,在乎三要,可以动静。
火生于木,祸发必克;奸生于国,时动必溃。知之修炼,谓之圣人。

天生天杀,道之理也。
天地,万物之盗;万物,人之盗;人,万物之盗。
三盗既宜,三才既安。故曰:食其时,百骸理;动其机,万化安。
人知其神而神,不知其不神之所以神也。
日月有数,大小有定,圣功生焉,神明出焉。
其盗机也,天下莫能见,莫能知也。君子得之固躬,小人得之轻命。 

瞽者善听,聋者善视。\hl{绝利一源,用师十倍。三返昼夜,用师万倍}。
心生于物,死于物,机在于目。
天之无恩而大恩生。迅雷烈风,莫不蠢然。
至乐性余,至静性廉。天之至私,用之至公。禽之制在炁。
生者死之根,死者生之根。恩生于害,害生于恩。
愚人以天地文理圣,我以时物文理哲。人以愚虞圣,我以不愚虞圣;人以奇期圣,我以不奇期圣。
故曰:沉水入火,自取灭亡。

\hl{自然之道静,故天地万物生}。
天地之道浸,故阴阳胜。
阴阳相推,而变化顺矣。是故圣人知自然之道不可违,因而制之。
至静之道。律历所不能契。
爰有奇器,是生万象,八卦甲子,神机鬼藏。
阴阳相胜之术,昭昭乎进于象矣。 

\subsection{释义}

以道心物三合之道来诠释,物者,意之所在。

观天之道,执天之行,尽矣:由心出发,体察天地之道,而后可以循道而行,此为道知,道尽为学处世之道。
观天之道是升维思考,执天之行是降维贯通,两相结合,就完备了。

天非茫茫之天,内蕴五行,能体察五行之气运,则可以昌盛。

心为能动的一方,以心受道体道,就可以立其环中,以应无穷,包括领导统御之术。

天人合发,万变定基:心与道合、与天合,这是做一切事的根基。

明了五行生克的结构与动态关系,进而内化于心,正心诚意,可称为圣人。

一事或成或败,皆有道理蕴含其中。天地-人-万物三合之道,尽心知性则知天矣,格物致知穷理,
尊德性而道问学,此三者相生相克,转圆而求其合。藏器于身待时而动,则万事如意,臻于中道。

绝利一源,一者何?道也,进乎技也。三返昼夜,循环至三,如昼夜交替,运行不废。
其功效甚大,有事半功倍之效果。一不能理解为具体的事,如此则器,君子不器,本立道生。
若心能体道,秉道御物,乘物游心,则三合之道可以大成。以道控势,顺势而为,与道浮沉。

执大象,天下往。往而不害,安平泰。大象无形,此无形之大象,即是道。
一生二,太极生两仪,有上下层次之别。两仪一阴一阳,有左右对称之美。

惟精惟一,志于道,若能志于道,而不废事,可入事事无碍法界。

大道至简,玄之又玄众妙之门。

真人者,同天而合道,执一而养产万类,怀天心,施德养,无为以包志虑思意而行威势者也。
鬼谷阴符七术之教。

阳明心学,心外无理心外无事,此心与道为一,即是道心、天心。

口目耳,此身之三要,心能制之。微信控,游戏控,则失心之所以为主,惑矣。

气韵生动

在二元对立的世界里,诗意地安居?一故神,两故化。

\section{洪范}

武王胜殷,杀受,立武庚,以箕子归。作《洪范》。

惟十有三祀,王访于箕子。王乃言曰:「呜呼!箕子。惟天阴骘下民,相协厥居,我不知其彝伦攸叙。」

箕子乃言曰:「我闻在昔,鲧堙洪水,汩陈其五行。帝乃震怒,不畀『洪范』九畴,彝伦攸斁。鲧则殛死,禹乃嗣兴,天乃锡禹『洪范』九畴,彝伦攸叙。

初一曰五行,次二曰敬用五事,次三曰农用八政,次四曰协用五纪,次五曰建用皇极,次六曰乂用三德,次七曰明用稽疑,次八曰念用庶征,次九曰向用五福,威用六极。

一、五行:一曰水,二曰火,三曰木,四曰金,五曰土。水曰润下,火曰炎上,木曰曲直,金曰从革,土爰稼穑。润下作咸,炎上作苦,曲直作酸,从革作辛,稼穑作甘。

二、五事:一曰貌,二曰言,三曰视,四曰听,五曰思。貌曰恭,言曰从,视曰明,听曰聪,思曰睿。恭作肃,从作乂,明作哲,聪作谋,睿作圣。

三、八政:一曰食,二曰货,三曰祀,四曰司空,五曰司徒,六曰司寇,七日宾,八曰师。

四、五纪:一曰岁,二曰月,三曰日,四曰星辰,五曰历数。

五、皇极:皇建其有极。敛时五福,用敷锡厥庶民。惟时厥庶民于汝极。锡汝保极:凡厥庶民,无有淫朋,人无有比德,惟皇作极。凡厥庶民,有猷有为有守,汝则念之。不协于极,不罹于咎,皇则受之。而康而色,曰:『予攸好德。』汝则锡之福。时人斯其惟皇之极。无虐茕独而畏高明,人之有能有为,使羞其行,而邦其昌。凡厥正人,既富方谷,汝弗能使有好于而家,时人斯其辜。于其无好德,汝虽锡之福,其作汝用咎。无偏无陂,遵王之义;无有作好,遵王之道;无有作恶,尊王之路。无偏无党,王道荡荡;无党无偏,王道平平;无反无侧,王道正直。会其有极,归其有极。曰:皇,极之敷言,是彝是训,于帝其训,凡厥庶民,极之敷言,是训是行,以近天子之光。曰:天子作民父母,以为天下王。

六、三德:一曰正直,二曰刚克,三曰柔克。平康,正直;强弗友,刚克;燮友,柔克。沈潜,刚克;高明,柔克。惟辟作福,惟辟作威,惟辟玉食。臣无有作福、作威、玉食。臣之有作福、作威、玉食,其害于而家,凶于而国。人用侧颇僻,民用僭忒。

七、稽疑:择建立卜筮人,乃命十筮。曰雨,曰霁,曰蒙,曰驿,曰克,曰贞,曰悔,凡七。卜五,占用二,衍忒。立时人作卜筮,三人占,则从二人之言。汝则有大疑,谋及乃心,谋及卿士,谋及庶人,谋及卜筮。汝则从,龟从,筮从,卿士从,庶民从,是之谓大同。身其康强,子孙其逢,汝则从,龟从,筮从,卿士逆,庶民逆吉。卿士从,龟从,筮从,汝则逆,庶民逆,吉。庶民从,龟从,筮从,汝则逆,卿士逆,吉。汝则从,龟从,筮逆,卿士逆,庶民逆,作内吉,作外凶。龟筮共违于人,用静吉,用作凶。

八、庶征:曰雨,曰暘,曰燠,曰寒,曰风。曰时五者来备,各以其叙,庶草蕃庑。一极备,凶;一极无,凶。曰休征;曰肃、时雨若;曰乂,时暘若;曰晰,时燠若;曰谋,时寒若;曰圣,时风若。曰咎征:曰狂,恒雨若;曰僭,恒暘若;曰豫,恒燠若;曰急,恒寒若;曰蒙,恒风若。曰王省惟岁,卿士惟月,师尹惟日。岁月日时无易,百谷用成,乂用民,俊民用章,家用平康。日月岁时既易,百谷用不成,乂用昏不明,俊民用微,家用不宁。庶民惟星,星有好风,星有好雨。日月之行,则有冬有夏。月之从星,则以风雨。

九、五福:一曰寿,二曰富,三曰康宁,四曰攸好德,五曰考终命。六极:一曰凶、短、折,二曰疾,三曰忧,四曰贫,五曰恶,六曰弱。

\section{大学}

大学之道,在明明德,在亲民,在止于至善。

知止而後有定,定而後能静,静而後能安,安而後能虑,虑而後能得。物有本末,事有终始。知所先後,则近道矣。

古之欲明明德于天下者,先治其国。欲治其国者,先齐其家,欲齐其家者,先修其身。欲修其身者,先正其心。欲正其心者,先诚其意。欲诚其意者,先致其知。致知在格物。

物格而後知至,知至而後意诚,意诚而後心正,心正而後身修,身修而後家齐,家齐而後国治,国治而後天下平。自天子以至于庶人,一是皆以修身为本。

其本乱而末治者否矣。其所厚者薄,而其所薄者厚,未之有也。此谓知本,此谓知之至也。

所谓诚其意者,毋自欺也。如恶恶臭,如好好色,此之谓自谦。故君子必慎其独也。小人闲居为不善,无所不至,见君子而後厌然,掩其不善,而著其善。人之视己,如见其肺肝然,则何益矣。此谓诚于中,形于外,故君子必慎其独也。曾子曰:“十目所视,十手所指,其严乎!”富润屋,德润身,心广体胖,故君子必诚其意。

《诗》云:“瞻彼淇澳,菉竹猗猗。有斐君子,如切如磋,如琢如磨。瑟兮僴兮,赫兮喧兮。有斐君子,终不可諠兮!”“如切如磋”者,道学也。“如琢如磨”者,自修也。“瑟兮僴兮”者,恂慄也。“赫兮喧兮”者,威仪也。“有斐君子,终不可諠兮”者,道盛德至善,民之不能忘也。

《诗》云:“於戏,前王不忘!”君子贤其贤而亲其亲,小人乐其乐而利其利,此以没世不忘也。

《康诰》曰:“克明德。”《大甲》曰:“顾諟天之明命。”《帝典》曰:“克明峻德。”皆自明也。

汤之《盘铭》曰:“苟日新,日日新,又日新。”《康诰》曰:“作新民。”《诗》曰:“周虽旧邦,其命维新。”是故君子无所不用其极。

《诗》云:“邦畿千里,维民所止。”《诗》云:“缗蛮黄鸟,止于丘隅。”子曰:“于止,知其所止,可以人而不如鸟乎?”《诗》云:“穆穆文王,于缉熙敬止!”为人君,止于仁;为人臣,止于敬;为人子,止于孝;为人父,止于慈;与国人交,止于信。

子曰:“听讼,吾犹人也。必也使无讼乎!”无情者不得尽其辞。大畏民志,此谓知本”。

所谓修身在正其心者,身有所忿懥,则不得其正,有所恐惧,则不得其正,有所好乐,则不得其正,有所忧患,则不得其正。心不在焉,视而不见,听而不闻,食而不知其味。此谓修身在正其心。

所谓齐其家在修其身者,人之其所亲爱而辟焉,之其所贱恶而辟焉,之其所畏敬而辟焉,之其所哀矜而辟焉,之其所敖惰而辟焉。故好而知其恶,恶而知其美者,天下鲜矣。故谚有之曰:“人莫知其子之恶,莫知其苗之硕。”此谓身不修不可以齐其家。

所谓治国必先齐其家者,其家不可教而能教人者,无之。故君子不出家而成教于国。孝者,所以事君也;弟者,所以事长也;慈者,所以使众也。《康诰》曰:“如保赤子。”心诚求之,虽不中不远矣。未有学养子而後嫁者也。一家仁,一国兴仁;一家让,一国兴让;一人贪戾,一国作乱:其机如此。此谓一言偾事,一人定国。尧、舜率天下以仁,而民从之。桀、纣率天下以暴,而民从之。其所令反其所好,而民不从。是故君子有诸己而後求诸人,无诸己而後非诸人。所藏乎身不恕,而能喻诸人者,未之有也。故治国在齐其家。《诗》云:“桃之夭夭,其叶蓁蓁。之子于归,宜其家人。”宜其家人,而後可以教国人。《诗》云:“宜兄宜弟。”宜兄宜弟,而後可以教国人。《诗》云:“其仪不忒,正是四国。”其为父子兄弟足法,而後民法之也。此谓治国在齐其家。

所谓平天下在治其国者,上老老而民兴孝,上长长而民兴弟,上恤孤而民不倍,是以君子有絜矩之道也。

所恶于上,毋以使下,所恶于下,毋以事上;所恶于前,毋以先後;所恶于後,毋以从前;所恶于右,毋以交于左;所恶于左,毋以交于右;此之谓絜矩之道。

《诗》云:“乐只君子,民之父母。”民之所好好之,民之所恶恶之,此之谓民之父母。《诗》云:“节彼南山,维石岩岩。赫赫师尹,民具尔瞻。”有国者不可以不慎,辟则为天下僇矣。《诗》云:“殷之未丧师,克配上帝。仪监于殷,峻命不易。”道得众则得国,失众则失国。

是故君子先慎乎德。有德此有人,有人此有土,有土此有财,有财此有用。德者本也,财者末也。外本内末,争民施夺。是故财聚则民散,财散则民聚。是故言悖而出者,亦悖而入;货悖而入者,亦悖而出。

《康诰》曰:“惟命不于常。”道善则得之,不善则失之矣。

《楚书》曰:“楚国无以为宝,惟善以为宝。”舅犯曰:“亡人无以为宝,仁亲以为宝。”

《秦誓》曰:“若有一个臣,断断兮无他技,其心休休焉,其如有容焉。人之有技,若己有之;人之彦圣,其心好之,不啻若自其口出。实能容之,以能保我子孙黎民,尚亦有利哉!人之有技,冒嫉以恶之;人之彦圣,而违之,俾不通:实不能容,以不能保我子孙黎民,亦曰殆哉!”唯仁人放流之,迸诸四夷,不与同中国。此谓唯仁人为能爱人,能恶人。见贤而不能举,举而不能先,命也;见不善而不能退,退而不能远,过也。好人之所恶,恶人之所好,是谓拂人之性,菑必逮夫身。是故君子有大道,必忠信以得之,骄泰以失之。

生财有大道,生之者众,食之者寡,为之者疾,用之者舒,则财恒足矣。仁者以财发身,不仁者以身发财。未有上好仁而下不好义者也,未有好义其事不终者也,未有府库财非其财者也。孟献子曰:“畜马乘不察于鸡豚,伐冰之家不畜牛羊,百乘之家不畜聚敛之臣。与其有聚敛之臣,宁有盗臣。”此谓国不以利为利,以义为利也。长国家而务财用者,必自小人矣。彼为善之,小人之使为国家,菑害并至。虽有善者,亦无如之何矣!此谓国不以利为利,以义为利也。

\section{中庸}

\hl{天命之谓性,率性之谓道,修道之谓教}。道也者,不可须臾离也;可离非道也。是故君子戒慎乎其所不睹,恐惧乎其所不闻。莫见乎隐,莫显乎微,故君子慎其独也。喜怒哀乐之未发,谓之中;发而皆中节,谓之和。中也者,天下之大本也;和也者,天下之达道也。致中和,天地位焉,万物育焉。

仲尼曰:“君子中庸,小人反中庸。君子之中庸也,君子而时中;小人之中庸也,小人而无忌惮也。”子曰:“中庸其至矣乎!民鲜能久矣!”

子曰:“道之不行也,我知之矣:知者过之,愚者不及也。道之不明也,我知之矣:贤者过之,不肖者不及也。人莫不饮食也,鲜能知味也。”子曰:“道其不行矣夫!”

子曰:“舜其大知也与!舜好问而好察迩言,隐恶而扬善,执其两端,用其中于民,其斯以为舜乎!”

子曰:“人皆曰予知,驱而纳诸罟擭陷阱之中,而莫之知辟也。人皆曰予知,择乎中庸,而不能期月守也。”

子曰:“回之为人也,择乎中庸,得一善,则拳拳服膺弗失之矣。”

子曰:“天下国家可均也,爵禄可辞也,白刃可蹈也,中庸不可能也。”

子路问强。子曰:“南方之强与?北方之强与?抑而强与?宽柔以教,不报无道,南方之强也,君子居之。衽金革,死而不厌,北方之强也,而强者居之。故君子和而不流,强哉矫!中立而不倚,强哉矫!国有道,不变塞焉,强哉矫!国无道,至死不变,强哉矫!”

子曰:“素隐行怪,後世有述焉,吾弗为之矣。君子遵道而行,半途而废,吾弗能已矣。君子依乎中庸,遁世不见知而不悔,唯圣者能之。”

君子之道费而隐。夫妇之愚,可以与知焉;及其至也,虽圣人亦有所不知焉。夫妇之不肖,可以能行焉;及其至也,虽圣人亦有所不能焉。天地之大也,人犹有所憾。故君子语大,天下莫能载焉;语小,天下莫能破焉。《诗》云:“鸢飞戾天,鱼跃于渊。”言其上下察也。君子之道,造端乎夫妇,及其至也,察乎天地。

子曰:“道不远人,人之为道而远人,不可以为道。《诗》云:‘伐柯伐柯,其则不远。’执柯以伐柯,睨而视之,犹以为远。故君子以人治人,改而止。忠恕违道不远,施诸己而不愿,亦勿施于人。君子之道四,丘未能一焉,所求乎子,以事父,未能也;所求乎臣,以事君,未能也;所求乎弟,以事兄,未能也;所求乎朋友先施之,未能也。庸德之行,庸言之谨;有所不足,不敢不勉,有馀不敢尽;言顾行,行顾言,君子胡不慥慥尔!”

君子素其位而行,不愿乎其外。素富贵,行乎富贵;素贫贱,行乎贫贱;素夷狄,行乎夷狄;素患难行乎患难,君子无入而不自得焉。在上位不陵下,在下位不援上,正己而不求于人,则无怨。上不怨天,下不尤人。故君子居易以俟命。小人行险以徼幸。子曰:“射有似乎君子,失诸正鹄,反求诸其身。”

君子之道,辟如行远必自迩,辟如登高必自卑。《诗》曰:“妻子好合,如鼓瑟琴。兄弟既翕,和乐且耽。宜尔室家,乐尔妻帑。”子曰:“父母其顺矣乎!”

子曰:“鬼神之为德,其盛矣乎?!视之而弗见,听之而弗闻,体物而不可遗,使天下之人齐明盛服,以承祭祀。洋洋乎如在其上,如在其左右。《诗》曰:‘神之格思,不可度思!矧可射思!’夫微之显,诚之不可揜如此夫。”

子曰:“舜其大孝也与!德为圣人,尊为天子,富有四海之内。宗庙飨之,子孙保之。故大德必得其位,必得其禄,必得其名,必得其寿。故天之生物,必因其材而笃焉。故栽者培之,倾者覆之。《诗》曰:‘嘉乐君子,宪宪令德。宜民宜人,受禄于天,保佑命之,自天申之。’故大德者必受命。”

子曰:“无忧者,其惟文王乎!以王季为父,以武王为子,父作之,子述之。武王缵大王、王季、文王之绪,壹戎衣而有天下。身不失天下之显名,尊为天子,富有四海之内。宗庙飨之,子孙保之。武王末受命,周公成文、武之德,追王大王、王季,上祀先公以天子之礼。斯礼也,达乎诸侯大夫,及士庶人。父为大夫,子为士,葬以大夫,祭以士。父为士,子为大夫,葬以士,祭以大夫。期之丧,达乎大夫。三年之丧,达乎天子。父母之丧,无贵贱,一也。”

子曰:“武王、周公,其达孝矣乎!夫孝者,善继人之志,善述人之事者也。春秋修其祖庙,陈其宗器,设其裳衣,荐其时食。宗庙之礼,所以序昭穆也。序爵,所以辨贵贱也。序事,所以辨贤也。旅酬下为上,所以逮贱也。燕毛,所以序齿也。践其位,行其礼,奏其乐,敬其所尊,爱其所亲,事死如事生,事亡如事存,孝之至也。郊社之礼,所以事上帝也。宗庙之礼,所以祀乎其先也。明乎郊社之礼、禘尝之义,治国其如示诸掌乎!”

哀公问政。子曰:“文武之政,布在方策。其人存,则其政举;其人亡,则其政息。人道敏政,地道敏树。夫政也者,蒲卢也。故为政在人,取人以身,修身以道,修道以仁。仁者人也,亲亲为大;义者宜也,尊贤为大。亲亲之杀,尊贤之等,礼所生也。在下位不获乎上,民不可得而治矣!故君子不可以不修身;思修身,不可以不事亲;思事亲,不可以不知人,思知人,不可以不知天。”

“天下之达道五,所以行之者三。曰:君臣也,父子也,夫妇也,昆弟也,朋友之交也,五者天下之达道也。\hl{知,仁,勇,三者天下之达德也,所以行之者一也}。或生而知之,或学而知之,或困而知之,及其知之,一也。或安而行之,或利而行之,或勉强而行之,及其成功,一也。子曰:好学近乎知,力行近乎仁,知耻近乎勇。知斯三者,则知所以修身;知所以修身,则知所以治人;知所以治人,则知所以治天下国家矣。”

“凡为天下国家有九经,曰:修身也。尊贤也,亲亲也,敬大臣也,体群臣也。子庶民也,来百工也,柔远人也,怀诸侯也。
修身则道立,尊贤则不惑,亲亲则诸父昆弟不怨,敬大臣则不眩,体群臣则士之报礼重,子庶民则百姓劝,来百工则财用足,柔远人则四方归之,怀诸侯则天下畏之。
齐明盛服,非礼不动。所以修身也;去谗远色,贱货而贵德,所以劝贤也;尊其位,重其禄,同其好恶,所以劝亲亲也;官盛任使,所以劝大臣也;忠信重禄,所以劝士也;
时使薄敛,所以劝百姓也;日省月试,既廪称事,所以劝百工也;送往迎来,嘉善而矜不能,所以柔远人也;
继绝世,举废国,治乱持危。朝聘以时,厚往而薄来,所以怀诸侯也。凡为天下国家有九经,所以行之者一也。”

“凡事豫则立,不豫则废。言前定则不跲,事前定则不困,行前定则不疚,道前定则不穷。在下位不获乎上,民不可得而治矣。
获乎上有道,不信乎朋友,不获乎上矣;信乎朋友有道,不顺乎亲,不信乎朋友矣;顺乎亲有道,反诸身不诚,不顺乎亲矣;诚身有道,不明乎善,不诚乎身矣。
\hl{诚者,天之道也;诚之者,人之道也。诚者不勉而中,不思而得,从容中道,圣人也。诚之者,择善而固执之者也}。”

“博学之,审问之,慎思之,明辨之,笃行之。有弗学,学之弗能,弗措也;有弗问,问之弗知,弗措也;有弗思,思之弗得,弗措也;有弗辨,辨之弗明,弗措也;有弗行,行之弗笃,弗措也。
人一能之己百之,人十能之己千之。果能此道矣。虽愚必明,虽柔必强。”

自诚明,谓之性。自明诚,谓之教。诚则明矣,明则诚矣。

\hl{唯天下至诚,为能尽其性;能尽其性,则能尽人之性;能尽人之性,则能尽物之性;能尽物之性,则可以赞天地之化育;可以赞天地之化育,则可以与天地参矣}。

其次致曲。曲能有诚,诚则形,形则著,著则明,明则动,动则变,变则化。唯天下至诚为能化。

至诚之道,可以前知。国家将兴,必有祯祥;国家将亡,必有妖孽。见乎蓍龟,动乎四体。祸福将至:善,必先知之;不善,必先知之。故至诚如神。

诚者自成也,而道自道也。诚者物之终始,不诚无物。是故君子诚之为贵。诚者非自成己而已也,所以成物也。
成己,仁也;成物,知也。性之德也,合外内之道也,故时措之宜也。

\hl{故至诚无息。不息则久,久则徵;徵则悠远,悠远则博厚,博厚则高明}。博厚,所以载物也;高明,所以覆物也;悠久,所以成物也。博厚配地,高明配天,悠久无疆。
如此者,不见而章,不动而变,无为而成。\hl{天地之道,可壹言而尽也。其为物不贰,则其生物不测}。天地之道:博也,厚也,高也,明也,悠也,久也。
今夫天,斯昭昭之多,及其无穷也,日月星辰系焉,万物覆焉。今夫地,一撮土之多,及其广厚,载华岳而不重,振河海而不泄,万物载焉。
今夫山,一卷石之多,及其广大,草木生之,禽兽居之,宝藏兴焉,今夫水,一勺之多,及其不测,鼋鼍、蛟龙、鱼鳖生焉,货财殖焉。
《诗》曰:“惟天之命,于穆不已!”盖曰天之所以为天也。“于乎不显,文王之德之纯!”盖曰文王之所以为文也,纯亦不已。

大哉!圣人之道洋洋乎!发育万物,峻极于天。优优大哉!礼仪三百,威仪三千,待其人然後行。故曰:苟不至德,至道不凝焉。
\hl{故君子尊德性而道问学;致广大而尽精微;极高明而道中庸;温故而知新,敦厚以崇礼}。是故居上不骄,为下不倍;国有道,其言足以兴;国无道,其默足以容。
《诗》曰:“既明且哲,以保其身。”其此之谓与!

子曰:“愚而好自用,贱而好自专,生乎今之世,反古之道:如此者,灾及其身者也。”非天子,不议礼,不制度,不考文。今天下车同轨,书同文,行同伦。虽有其位,苟无其德,不敢作礼乐焉;虽有其德。苟无其位,亦不敢作礼乐焉。子曰:“吾说夏礼,杞不足徵也。吾学殷礼,有宋存焉。吾学周礼,今用之,吾从周。”

王天下有三重焉,其寡过矣乎!上焉者虽善无徵,无徵不信,不信民弗从;下焉者虽善不尊,不尊不信,不信民弗从。故君子之道:本诸身,徵诸庶民,考诸三王而不缪,建诸天地而不悖,质诸鬼神而无疑,百世以俟圣人而不惑。质诸鬼神而无疑,知天也;百世以俟圣人而不惑,知人也。是故君子动而世为天下道,行而世为天下法,言而世为天下则。远之则有望,近之则不厌。《诗》曰:“在彼无恶,在此无射。庶几夙夜,以永终誉!”君子未有不如此而蚤有誉于天下者也。

仲尼祖述尧舜,宪章文武:上律天时,下袭水土。辟如天地之无不持载,无不覆帱,辟如四时之错行,如日月之代明。万物并育而不相害,道并行而不相悖,小德川流,大德敦化,此天地之这所以为大也。

唯天下至圣为能聪明睿知,足以有临也;宽裕温柔,足以有容也;发强刚毅,足以有执也;齐庄中正,足以有敬也;文理密察,足以有别也。溥博渊泉,而时出之。溥博如天,渊泉如渊。
见而民莫不敬,言而民莫不信,行而民莫不说。是以声名洋溢乎中国,施及蛮貊。舟车所至,人力所通,天之所覆,地之所载,日月所照,霜露所队,凡有血气者,莫不尊亲,故曰配天。

\hl{唯天下至诚,为能经纶天下之大经,立天下之大本,知天地之化育}。夫焉有所倚?肫肫其仁!渊渊其渊!浩浩其天!苟不固聪明圣知达天德者,其孰能知之?

《诗》曰:“衣锦尚絅。”恶其文之著也。故君子之道,暗然而日章;小人之道,的然而日亡。君子之道:淡而面不厌,简而文,温而理,知远之近,知风之自,知微之显,可与入德矣。
《诗》云:“潜虽伏矣,亦孔之昭!”故君子内省不疚,无恶于志。君子所不可及者,其唯人之所不见乎!
《诗》云:“相在尔室,尚不愧于屋漏。”故君子不动而敬,不言而信。
《诗》曰:“奏假无言,时靡有争。”是故君子不赏而民劝,不怒而民威于鈇钺。
《诗》曰:“不显惟德!百辟其刑之。”是故君子笃恭而天下平。
《诗》云:“予怀明德,不大声以色。”子曰:“声色之于以化民。末也。”
《诗》曰:“德如毛。”毛犹有伦,上天之载,无声无臭,至矣!

\section{管子}

\subsection{内业}

凡物之精,此则为生。下生五谷,上为列星。流于天地之间,谓之鬼神;藏于胸中,谓之圣人。
是故民气,杲乎如登于天,杳乎如入于渊,淖乎如在于海,卒乎如在于己。
是故此气也,不可止以力,而可安以德;不可呼以声,而可迎以音。敬守勿失,是谓成德,德成而智出,万物果得。

凡心之刑,自充自盈,自生自成。其所以失之,必以忧乐喜怒欲利。能去忧乐喜怒欲利,心乃反济。
彼心之情,利安以宁,勿烦勿乱,和乃自成。折折乎如在于侧,忽忽乎如将不得,渺渺乎如穷无极。
此稽不远,日用其德。

夫道者,所以充形也,而人不能固。其往不复,其来不舍。谋乎莫闻其音,卒乎乃在于心;冥冥乎不见其形,淫淫乎与我俱生。不见其形;不闻其声,而序其成,谓之道。
凡道无所,善心安爱。心静气理,道乃可止。彼道不远,民得以产;彼道不离,民因以知。是故卒乎其如可与索,眇眇乎其如穷无所。彼道之情,恶音与声,修心静音,道乃可得。
道也者,口之所不能言也,目之所不能视也,耳之所不能听也,所以修心而正形也;人之所失以死,所得以生也;事之所失以败,所得以成也。
凡道无根无茎,无叶无荣。万物以生,万物以成,命之曰道。

天主正,地主平,人主安静。春秋冬夏,天之时也;山陵川谷,地之枝也;喜怒取予,人之谋也。是故圣人与时变而不化,从物而不移。能正能静,然后能定。定心在中,耳目聪明,四肢坚固,可以为精舍。精也者,气之精者也。气,道乃生,生乃思,思乃知,知乃止矣。凡心之形,过知失生。

一物能化谓之神,一事能变谓之智。化不易气,变不易智,唯执一之君子能为此乎!执一不失,能君万物。君子使物,不为物使,得一之理。治心在于中,治言出于口,治事加于人,然则天下治矣。一言得而天下服,一言定而天下听,公之谓也。

形不正,德不来;中不静,心不治。正形摄德,天仁地义,则淫然而自至神明之极,照乎知万物。中义守不忒,不以物乱官,不以官乱心,是谓中得。

有神自在身,一往一来,奠之能思。失之必乱,得之必治。敬除其舍,精将自来。精想思之,宁念治之,严容畏敬,精将至定。得之而勿舍,耳目不淫。

心无他图,正心在中,万物得度。道满天下,普在民所,民不能知也。一言之解,上察于天,下极于地,蟠满九州。何谓解之?在于心安。我心治,官乃治,我心安,官乃安。治之者心也,安之者心也。

心以藏心,心之中又有心焉。彼心之心,音以先言。音然后形,形然后言,言然后使,使然后治。不治必乱,乱乃死。

精存自生,其外安荣,内藏以为泉原,浩然和平,以为气渊。渊之不涸,四体乃固;泉之不竭,九窍遂通。乃能穷天地,破四海。中无惑意,外无邪灾,心全于中,形全于外,不逢天灾,不遇人窖,谓之圣人。

人能正静,皮肤裕宽,耳目聪明,筋信而骨强。乃能戴大圜,而履大方,鉴于大清,视干大明。敬慎无忒,日新其德,遍知天下,穷于四极。敬发其充,是谓内得。然而不反,此生之忒。

凡道,必周必密,必宽必舒,必坚必固,守善勿舍,逐淫泽薄,既知其极,反于道德。全心在中,不可蔽匿,和于形容,见于肤色。善气迎人,亲于弟兄;恶气迎人,害于戎兵。不言之声,疾于雷鼓;心气之形,明于日月,察于父母。赏不足以劝善,刑不足以惩过,气意得而天下服,心意定而天下听。

搏气如神,万物备存。能搏乎?能一乎?能无卜筮而知吉凶乎?能止乎?能已乎?能勿求诸人而得之己乎?思之,思之,又重思之。思之而不通,鬼神将通之。非鬼神之力也,精气之极也。

四体既正,血气既静,一意搏心,耳目不淫,虽远若近。思索生知,慢易生忧,暴傲生怨,忧郁生疾,疾困乃死。思之而不舍,内困外薄,不早为图,生将巽舍。食莫若无饱,思莫若勿致,节适之齐,彼将自至。

凡人之生也,天出其精,地出其形,合此以为人。和乃生,不和不生。察和之道,其精不见,其征不丑。平正擅匈,论治在心。此以长寿。忿怒之失度,乃为之图。节其五欲,去其二凶,不喜不怒,平正擅匈。

凡人之生也,必以平正。所以失之,必以喜怒忧患。是故止怒莫若诗,去忧莫若乐,节乐莫若礼,守礼莫若敬,守敬莫若静。内静外敬,能反其性,性将大定。

凡食之道:大充,伤而形不臧;大摄,骨枯而血沍。充摄之间,此谓和成,精之所舍,而知之所生,饥饱之失度,乃为之图。饱则疾动,饥则广思,老则长虑。饱不疾动,气不通于四末;饥不广思,饱而不废;老不长虑,困乃速竭。大心而敢,宽气而广,其形安而不移,能守一而弃万苛,见利不诱,见害不俱,宽舒而仁,独乐其身,是谓云气,意行似天。

凡人之生也,必以其欢。忧则失纪,怒则失端。忧悲喜怒,道乃无处。爱欲静之,遇乱正之,勿引勿推,福将自归。彼道自来,可藉与谋,静则得之,躁则失之。灵气在心,一来一逝,其细无内,其大无外。所以失之,以躁为害。心能执静,道将自定。得道之人,理丞而屯泄,匈中无败。节欲之道,万物不害。

\subsection{心术上}

心之在体,君之位也;九窍之有职,官之分也。心处其道。九窍循理;嗜欲充益,目不见色,耳不闻声。故曰上离其道,下失其事。毋代马走,使尽其力;毋代鸟飞,使弊其羽翼;毋先物动,以观其则。动则失位,静乃自得。

道,不远而难极也,与人并处而难得也。虚其欲,神将入舍;扫除不洁,神乃留处。人皆欲智而莫索其所以智乎。智乎,智乎,投之海外无自夺,求之者不得处之者。夫正人无求之也,故能虚无。

虚无无形谓之道,化育万物谓之德,君臣父子人间之事谓之义,登降揖让、贵贱有等、亲疏之体谓之礼,简物、小未一道。杀僇禁诛谓之法。

大道可安而不可说。直人之言不义不颇,不出于口,不见于色,四海之人,又孰知其则?

天曰虚,地曰静,乃不伐。洁其宫,开其门,去私毋言,神明若存。纷乎其若乱,静之而自治。强不能遍立,智不能尽谋。物固有形,形固有名,名当,谓之圣人。故必知不言,无为之事,然后知道之纪。殊形异埶,不与万物异理,故可以为天下始。

人之可杀,以其恶死也;其可不利,以其好利也。是以君子不休乎好,不迫乎恶,恬愉无为,去智与故。其应也,非所设也;其动也,非所取也。过在自用,罪在变化。是故有道之君,其处也若无知,其应物也若偶之。静因之道也。

“心之在体,君之位也;九窍之有职,官之分也。”耳目者。视听之官也,心而无与于视听之事,则官得守其分矣。夫心有欲者,物过而目不见,声至而耳不闻也。故曰:“上离其道,下失其事。”故曰:心术者,无为而制窍者也。故曰“君”。“毋代马走”,“毋代鸟飞”,此言不夺能能,不与下诚也。“毋先物动”者,摇者不走,趮者不静,言动之不可以观也。“位”者”,谓其所立也。人主者立于阴,阴者静,故曰“动则失位”。阴则能制阳矣,静则能制动矣,攸曰,‘静乃自得。”

道在天地之间也,其大无外,其小无内,故曰“不远而难极也”。虚之与人也无间,唯圣人得虚道,故曰“并处而难得”。世人之所职者精也。去欲则宣,宣则静矣,静则精。精则独立矣,独则明,明则神矣。神者至贵也,故馆不辟除,则贵人不舍焉。故曰“不洁则神不处”。“人皆欲知而莫索之”,其所(以)知,彼也;其所以知,此也。不修之此,焉能知彼?修之此,莫能虚矣。虚者,无藏也。故曰去知则奚率求矣,无藏则奚设矣。无求无设则无虑,无虑则反复虚矣。

天之道,虚其无形。虚则不屈,无形则无所位迕,无所位迕,故遍流万物而不变,德者,道之舍,物得以生生,知得以职道之精。故德者得也。得也者,其谓所得以然也。以无为之谓道,舍之之谓德。故道之与德无间,故言之者不别也。间之理者,谓其所以舍也。义者,谓各处其宜也。礼者,因人之情,缘义之理,而为之节文者也,故礼者谓有理也。理也者,明分以谕义之意也。故礼出乎义,义出乎理,理因乎宜者也。法者所以同出,不得不然者也,故杀僇禁诛以一之也。故事督乎法,法出乎权,权出于道。

道也者、动不见其形,施不见其德,万物皆以得,然莫知其极。故曰“可以安而不可说”也。莫人,言至也。不宜,言应也。应也者,非吾所设,故能无宜也。不顾,言因也。因也者,非吾所顾,故无顾也。“不出于口,不见于色”,言无形也;“四海之人,孰知其则”,言深囿也。

天之道虚,地之道静。虚则不屈,静则不变,不变则无过,故曰“不伐”。“洁其宫,阙其门”:宫者,谓心也。心也者,智之舍也,故曰“宫”。洁之者,去好过也。门者,谓耳目也。耳目者,所以闻见也。“物固有形,形固有各”,此言不得过实、实不得延名。姑形以形,以形务名,督言正名,故曰“圣人”。“不言之言”,应也。应也者,以其为之人者也。执其名,务其应,所以成,之应之道也。“无为之道,因也。因也者,无益无损也。以其形因为之名,此因之术也。名者,圣人之所以纪万物也。人者立于强,务于善,未于能,动于故者也。圣人无之,无之则与物异矣。异则虚,虚者万物之始也,故曰“可以为天下始”。

人迫于恶,则失其所好;怵于好,则忘其所恶。非道也。故曰:“不怵乎好,不迫乎恶。”恶不失其理,欲不过其情,故曰:“君子”。“恬愉无为,去智与故”,言虚素也。“其应非所设也,其动非所取也”,此言因也。因也者,舍己而以物为法者也。感而后应,非所设也;缘理而动,非所取也,“过在自用,罪在变化”:自用则不虚,不虚则仵于物矣;变化则为生,为生则乱矣。故道贵因。因者,因其能者,言所用也。“君子之处也若无知”,言至虚也;“其应物也若偶之”,言时适也、若影之象形,响之应声也。故物至则应,过则舍矣。舍矣者,言复所于虚也。

\subsection{心术下}

形不正者,德不来;中不精者,心不冶。正形饰德,万物毕得,翼然自来,神莫知其极,昭知天下,通于四极。是故曰:无以物乱官,毋以官乱心,此之谓内德。是故意气定,然后反正。气者身之充也,行者正之义也。充不美则心不得,行不正则民不服。是故圣人若天然,无私覆也;若地然,无私载也。私者,乱天下者也。

凡物载名而来,圣人因而财之,而天下治。实不伤,不乱于天下,而天下治。专于意,一于心,耳目端,知远之证。能专乎?能一乎?能毋卜筮而知凶吉乎?能止乎?能已乎?能毋问于人而自得之于己乎?故曰,思之。思之不得,鬼神教之。非鬼神之力也。其精气之极也。

一气能变曰精、一事能变曰智。慕选者,所以等事也;极变者,所以应物也。慕选而不乱,极变而不烦,执一之君子执一而不失,能君万物,日月之与同光,天地之与同理。

圣人裁物,不为物使。心安,是国安也;心治,是国治也。治也者心也,安也者心也。治心在于中,治言出于口,治事加于民,故功作而民从,则百姓治矣。所以操者非刑也,所以危者非怒也。民人操,百姓治,道其本至也,至不至无,非所人而乱。

凡在有司执制者之利,非道也。圣人之道,若存若亡,援而用之,殁世不亡。与时变而不化,应物而不移,日用之而不化。

人能正静者,筋肕而骨强;能戴大圆者,体乎大方;镜大清者,视乎大明。正静不失,日新其德,昭知天下,通于四极。金心在中不可匿,外见于形容,可知于颜色。善气迎人,亲如弟兄;恶气迎人,害于戈兵。不言之言,闻于雷鼓。全心之形,明于日月,察于父母。昔者明王之爱天下,故天下可附;暴王之恶天下,故天下可离。故货之不足以为爱,刑之不足以为恶。货者爱之末也,刑者恶之末也。

凡民之生也,必以正平;所以失之者,必以喜乐哀怒,节怒莫若乐,节乐莫若礼,守礼莫若敬。外敬而内静者,必反其性。

岂无利事哉?我无利心。岂无安处哉?我无安心。心之中又有心。意以先言,意然后形,形然后思,思然后知。凡心之形,过知失生。

是故内聚以为原。泉之不竭,表里遂通;泉之不涸,四支坚固。能令用之,被及四固。

是故圣人一言解之,上察于天,下察于地。

\subsection{白心}

凡物之精,此则为生。下生五谷,上为列星。流于天地之间,谓之鬼神;藏于胸中,谓之圣人。是故民气,杲乎如登于天,杳乎如入于渊,淖乎如在于海,卒乎如在于己。是故此气也,不可止以力,而可安以德;不可呼以声,而可迎以音。敬守勿失,是谓成德,德成而智出,万物果得。

凡心之刑,自充自盈,自生自成。其所以失之,必以忧乐喜怒欲利。能去忧乐喜怒欲利,心乃反济。彼心之情,利安以宁,勿烦勿乱,和乃自成。折折乎如在于侧,忽忽乎如将不得,渺渺乎如穷无极。此稽不远,日用其德。

夫道者,所以充形也,而人不能固。其往不复,其来不舍。谋乎莫闻其音,卒乎乃在于心;冥冥乎不见其形,淫淫乎与我俱生。不见其形;不闻其声,而序其成,谓之道。凡道无所,善心安爱。心静气理,道乃可止。彼道不远,民得以产;彼道不离,民因以知。是故卒乎其如可与索,眇眇乎其如穷无所。彼道之情,恶音与声,修心静音,道乃可得。道也者,口之所不能言也,目之所不能视也,耳之所不能听也,所以修心而正形也;人之所失以死,所得以生也;事之所失以败,所得以成也。凡道无根无茎,无叶无荣。万物以生,万物以成,命之曰道。

天主正,地主平,人主安静。春秋冬夏,天之时也;山陵川谷,地之枝也;喜怒取予,人之谋也。是故圣人与时变而不化,从物而不移。能正能静,然后能定。定心在中,耳目聪明,四肢坚固,可以为精舍。精也者,气之精者也。气,道乃生,生乃思,思乃知,知乃止矣。凡心之形,过知失生。

一物能化谓之神,一事能变谓之智。化不易气,变不易智,唯执一之君子能为此乎!执一不失,能君万物。君子使物,不为物使,得一之理。治心在于中,治言出于口,治事加于人,然则天下治矣。一言得而天下服,一言定而天下听,公之谓也。

形不正,德不来;中不静,心不治。正形摄德,天仁地义,则淫然而自至神明之极,照乎知万物。中义守不忒,不以物乱官,不以官乱心,是谓中得。

有神自在身,一往一来,奠之能思。失之必乱,得之必治。敬除其舍,精将自来。精想思之,宁念治之,严容畏敬,精将至定。得之而勿舍,耳目不淫。

心无他图,正心在中,万物得度。道满天下,普在民所,民不能知也。一言之解,上察于天,下极于地,蟠满九州。何谓解之?在于心安。我心治,官乃治,我心安,官乃安。治之者心也,安之者心也。

心以藏心,心之中又有心焉。彼心之心,音以先言。音然后形,形然后言,言然后使,使然后治。不治必乱,乱乃死。

精存自生,其外安荣,内藏以为泉原,浩然和平,以为气渊。渊之不涸,四体乃固;泉之不竭,九窍遂通。乃能穷天地,破四海。中无惑意,外无邪灾,心全于中,形全于外,不逢天灾,不遇人窖,谓之圣人。

人能正静,皮肤裕宽,耳目聪明,筋信而骨强。乃能戴大圜,而履大方,鉴于大清,视干大明。敬慎无忒,日新其德,遍知天下,穷于四极。敬发其充,是谓内得。然而不反,此生之忒。

凡道,必周必密,必宽必舒,必坚必固,守善勿舍,逐淫泽薄,既知其极,反于道德。全心在中,不可蔽匿,和于形容,见于肤色。善气迎人,亲于弟兄;恶气迎人,害于戎兵。不言之声,疾于雷鼓;心气之形,明于日月,察于父母。赏不足以劝善,刑不足以惩过,气意得而天下服,心意定而天下听。

搏气如神,万物备存。能搏乎?能一乎?能无卜筮而知吉凶乎?能止乎?能已乎?能勿求诸人而得之己乎?思之,思之,又重思之。思之而不通,鬼神将通之。非鬼神之力也,精气之极也。

\include{tao/zhuangzi}
\include{tao/xunzi}

\chapter{Wish}

WOOP

\section{10X成长}

\begin{tcolorbox}
这是唯一让我们能够实实在在得到智慧的好处的最方便可行的方法, 如果不去记录六时书, 
我们实在是 “如入宝山, 空手而归”,让人可惜得扼腕而叹。

六时书可帮助我们建立“追踪体系”——帮助我们训练意识,因为身为在地球上的人,
几乎每天都是负面念头多于正面念头的,也就是说这是自然人的本来性质,
但这一性质必定导致大多数人陷入负面事件越来越多导致产生更多负面念头的恶性循环里。

这个规律就如同地心引力一样,是一个定理——一个大多数人都逃不开的规律。
但是人类既然能够逃脱地心引力上天入地,同样的,我们也可以摆脱这个“负多于正”的规律,
通过六时书的帮助——训练我们的念头,使负面念头越来越少,正面念头越来越多,
成为一个觉悟的、开悟的人。到时候,生活中不论发生什么事,
我们都能以最正确的态度去对待,一切问题都将不是问题。
\end{tcolorbox}

六时书这种结构化的方式较好,考虑问题趋于全面。

示例: 专心致志,不抱怨

\begin{lstlisting}
+ 正
- 反
O 合
\end{lstlisting}


\section{11}

\subsection{01}

编程注意事项
\begin{enumbox}
\item long time操作之前与之后,需要renew lease,防止lease失效,导致vctl切换
\item 读写、vfm\_get、vfm\_stat、stat等操作之前,必须chunk check。\hl{如果不检查,可能会返回ESTALE, see \_\_chunk\_read\_getnode}
\item chunk组织成tree,就意味着依赖性,需要从上到下依次保护,check
\item chunk上有lock保护
\item alloc/discard与io分离,故io加rdlock
\item \hl{困难之处在并发,tree结构上的并发,并且涉及到持久化、lease}
\end{enumbox}

诊断工具
\begin{enumbox}
\item 怎么快速得到各个controller的状态,包括pool、volume、snapshot等?
\item 怎么检查底层数据一致性?
\item 打印集群chunk tree,以及每个chunk的详细信息
\end{enumbox}

写到LUN结尾处,会自动扩容。ROW3这里有问题?

垃圾数据,干扰恢复,gc replica目前不能打开?

\subsection{02}

尽可能通过\hl{DBUG、GOTO}保留可跟踪线索,以方便线上调试。

mellanox交换机编程:SDK?


\section{12}

\subsection{03}

交换机SM HA不可用,更新OS到相同版本后可用。

熟悉RDMA

两副本,一副本处在chunk状态,且vfm非空,包括另一副本

\subsection{10}

熟悉iscsi/iser架构和代码,进一步需要熟悉spdk、nvmf等,这是块存储的趋势所在。

\begin{enumbox}
\item NVMe设备,通过设备文件和pci直接访问的不同
\item linux kernel的block layer
\item udev
\item mount是指定dirsync选项
\item dd在测试快照时,指定direct io
\item async和non blocking的异同
\item buffer和cache的异同
\item session和cookie
\item 内核提供了什么功能?现在又为什么在网络和存储上提倡kernel bypass。
\item vip and multipath
\item ceph
\item fs and object
\item db
\end{enumbox}

除了协议和生态部分,内核部分的主要模块也要持续思考,包括编程模型、元数据管理、HA等。
要具有提要钩玄的能力,勾勒出主要脉络,分清主次,带动全局。中医对生命科学的思考方式,很有借鉴意义。
人的生命是一个动态发展的整体。如何维持此整体的健康?

\subsection{11}

要深入体会网络协议栈的架构设计实现,充分地体现了分层架构的巨大威力。

要想富,先修路。网络编程的第一步就是连接管理。连接的建立、维护到断开。
一个或多个连接构成session。一个session见得唯一的target。target与lun什么关系?
为什么需要多个连接构成session?

关于学习:重视第一手资料,道听途说不能解决问题,善于提问,不要有畏难情绪。
比如RDMA,从原始文献好好体会,在iSER里进一步体会,温故知新,没有过不去的坎。

再回头看RDMA代码,当明白了一些原则后,就显得一目了然了。
iscsi和iser代码也是如此。为什么呢?需要理解消息交换的基本原则。
在一个循环内,既有post,又有poll。

scst与lio解决的是同一类问题,scst的代码分为三部分:core、target driver and storage driver。
lich也有如此模块。core起到存储虚拟化的作用。

学习一定要破除神秘感,循着正确方法,很容易攻克。即便如AI、数学,也不是不能捡起来。应该捡起来。

\subsection{12}

对个人来说,做什么才能产生复利效应?专业、认知。

一个lich系统,一个生命系统,都是系统。系统论具有普适性。

网络和scheduler的事件循环,也是一个圆运动。

\begin{enumbox}
\item 列出所有controller及其分布
\item 控制器为什么加载不成功?
\end{enumbox}

专注全闪,不做它想。不知自我克制,终将一事无成。

在docker里用文件模拟设备文件,因为文件io接口是一样的。

测试方法和工具:\hl{道法术器}几个层面,工具化、自动化至关重要,
但必须在正确策略的指引之下。

查找控制器为什么用mcast机制,而不是去admin上查(lease机制),
因为客户端有cache,也不会对admin带来额外负担。

vip才会有arp缓存失效的问题,需要主动更新client端arp缓存。

\subsection{13}

linux
\begin{enumbox}
\item 操作系统
\end{enumbox}

ceph
\begin{enumbox}
\item 两篇论文
\item 官方文档
\item 参考书
\item 源码
\end{enumbox}

\subsection{14}


\chapter{读后记}


\chapter{2019}

\section{01}

\subsection{01}

进入2019,斗志昂扬。2019是个丰收年。

最重要的事情是什么?切勿捡了芝麻、丢了西瓜。放长线才能钓大鱼。知止不殆,这是显而易见的事情,为什么一再地犯同样错误?
阅读也好,锻炼身体也好,专业知识的学习也好,事业也好,乃至一切,都是同样的。把握主要矛盾,寻求突破点,通于一,万事毕。

得一以为天下牧,黄帝四经:执一明三定二,是天人之道。据于道0,用德一之能量统筹全局,无极而太极,太极本无极。
方法论解决了,具体如何落实呢?

万千义理,不过0和1。由此0和1,演绎三千大千世界。

\subsection{02}

分析结果,\hl{须大力提升专业能力,作为主要的突破点}。有着东西看似重要,但不是现阶段的当务之急,比如管理、投资、扯谈等等。
如何提升专业能力,须有方法论。方法论要简单、能持续进化。更重要的是要有目标、有方向,\hl{道法术器}诸多层面层层展开。
\hl{得其环中,以应无穷},是近二十年工作生活经验之总结,不可不坚决,坚如磐石。

并行:
\begin{enumbox}
\item 软件管道与SOA
\item 面向模式的软件架构
\item 结构化并行程序设计
\end{enumbox}

并行问题,需要结合硬件和操作系统去理解。多处理器系统具有不同的内存访问特征,
目前的做法包括:\hl{cpu亲和性、自定义内存分配器}。以提升数据时间和空间的局部性。

软件管道理论非常吸引人,简洁又深刻。提出了三规则:
\begin{enumbox}
\item 输入等于输出
\item 下游处理能力大于等于上游
\item 分配器远大于下游处理能力
\end{enumbox}

由此去分析几个现象:
\begin{enumbox}
\item lich服务器框架的三层结构
\item 性能分析和预测
\item disk故障下的恢复
\item 节点故障的恢复
\item QoS和限流
\item ***
\item 与\hl{排队论和TOC理论}相会通
\item 与petri net有相同之处,甚至可以上升到量化、数学化描述的程度
\end{enumbox}

并行意味着并发,并发不一定是并行,并行是并发的subclass。

控制流/数据,串行/并行,定义\hl{加速比}的概念。先讨论数据管理模式。数据条带化有助于提升并行度。

选择控制器的方式:为什么按第一副本定义控制器所在节点?因为控制器频繁变动,所以localize的成本过高。
如果数据均匀分布,则每个节点都可以充当卷控制器,且性能波动不大。

mapreduce是控制流和数据流的统一。

可重入、线程安全

subvol延迟加载,并且只加载一次,适用于double check的场景。

问题集:
\begin{enumbox}
\item 何为伪共享?如何解决?
\item 何为ABA问题?如何解决?
\item 如何实现lock free数据结构?
\end{enumbox}

\subsection{03}

\subsection{04}

\section{201806}

\subsection{0609}

坚持写日记。

李小龙的武术与哲学,以武术诠释哲学,吸收了佛道精神。不拘泥于形式,追求简单实用高效。
阴阳是理论基础,阴阳是一元论,一而二、二而一,不是二元论,侧重于一体两面的统一性。
水与空杯是隐喻。

功夫是自我发现与自我实现之旅。

志于道、据于德、依于仁、游于艺。道是第一位的,通过德、仁、艺得以落实到平常日用间。
德是大学的明明德,也即是小龙的自我发现之旅。仁是外在的,老子三宝的慈、俭、不敢为天下先。

艺不可少,择一艺以终老,就是软件设计,亦日日所从事的,注入真气,求大成。
不可作为寻找事情来做,而是纳入自己的生命旅程。

道法术器,是另一个序列。从艺的角度去分析。艺人通过自己的作品说话,突出作品而忘记自我。

古有练剑师,注入自己的生命,十年磨一剑.各行各业都有此类艺人,诠释大匠精神。

练拳不练功,到老一场空。不可不重基本功,打下坚实基础,可以走得更从容更远。

\subsection{0610}

诵黄帝阴符经,观天之道,执天之行,尽矣。见识行事必有所本,其源头在天,建立天道格局,则大事可成。
师心自用,以一己私智,则事半功倍,结果不容乐观。

\subsection{0621}

\hl{志于道为人生第一等事}。一体万化,大道之行,万物毕得。

哲学、原则、算法,纳入道的大熔炉里淬炼,爰有奇器,是生万象。奇器,即为道的熔炉。

从器上升到奇器,是一认识飞跃。
器,本指专业方面的成就、著作、产品。
奇器则指道,在道的光照下,专业、技术的成绩会更有异彩。

寓奇于器,是为奇器。道不离器,而超越于器。
知的目标是道,行的目标是器,道器是知行的升级版。

明道而制器,备物致用,立成器以为天下利。

如此则月印万川、万川入海,执道之要,制胜无形。

\subsection{0629}

毕建勋造型本源,深契我心。中国画造型出自三源:物、心、道。

道是含二之一,一生二是演化过程,二而一是体道过程,逆而行之,推而上行。
早上诵太极图说、黄帝阴符经,心中喜悦。太极图说,包含了上行与下行两过程,妙合而凝。
阴符经:天地,万物之盗;万物,人之盗;人,万物之盗,三盗即宜,三才即安。
此段论述是很明显的三合结构。天地,引申为道。人与万物出于天地,各尽其分。
三者之间,存在是万有引力,循环相生相克。

静,引起特别的注意和兴趣。太极图说、阴符经、乃至老庄、淮南、管子等,都以静为究竟。
如阴符经:自然之道静,则天地万物生。静绝非如常识所理解的耽空滞寂,无所事事,没有活力,
而是有着巨大能量的状态与方法论原则,宜进一步深思。

静,涵戒定慧为一体。一静字,道破天机,是修行功夫的下手处。
老子云:致虚极,守静笃。大静方能有大动,不鸣则已一鸣惊人。尸居而龙现,渊默而雷声。

静,恒一而不杂。静,与天为和;静,顺势而为;静,保合太和乃利贞。
静,自然无为。即是稳定态,又携万千之力,静极复动。
静,不同于空,而大于空。

心欲安静,虑欲深远。

\begin{shadequote}
天道运而无所积,故万物成;帝道运而无所积,故天下归;
圣道运而无所积,故海内服。明于天,通于圣,六通四辟于帝王之德者,其自为也,昧然无不静者矣!
圣人之静也,非曰静也善,故静也。万物无足以挠心者,故静也。
水静则明烛须眉,平中准,大匠取法焉。水静犹明,而况精神!
圣人之心静乎!天地之鉴也,万物之镜也。夫虚 静恬淡寂漠无为者,天地之平而道德之至也。
故帝王圣人休焉。休则 虚,虚则实,实则伦矣。虚则静,静则动,动则得矣。静则无为,无为也,则任事者责矣。无为则俞俞。俞俞者,忧患不能处,年寿长矣 。
夫虚静恬淡寂漠无为者,万物之本也。明此以南乡,尧之为君也; 明此以北面,舜之为臣也。
以此处上,帝王天子之德也;以此处下, 玄圣素王之道也。
以此退居而闲游,江海山林之士服;
以此进为而抚世,则功大名显而天下一也。
静而圣,动而王,无为也而尊,朴素而 天下莫能与之争美。
\end{shadequote}

淮南子
\begin{shadequote}
夫精神气志者,静而日充者以壮,躁而日耗者以老。是故圣人将养其神,和弱其气,平夷其形,而与道沈浮俯仰。
恬然则纵之,迫则用之。其纵之也若委衣,其用之也若发机。如是,则万物之化无不遇,而百事之变无不应。
\end{shadequote}

道心合一、物我合一的功夫所在,就是正确理解的静字。
由静而上通九天、下贯九野。
由静而涤除玄鉴而无疵(10章六条,由一静字可得)。
心性修养,千言万语,由一静字可以概括。
静的效用、目的在于让心成为天地之鉴、万物之镜。由此三合之道大成。

由静以体道,二为一之门户,阴阳为太极之门户,由二以见一,舍二不能见一。
阴阳,太极之门。

践行止语练习。

由问题与范畴展开学习过程,\hl{利用搜索功能},以范畴为主线,贯串命题,形成完整的认知体系。

一个核心问题就是:道是什么?如何体道?在三合架构内,这些问题才能得到更好的理解。

\section{201807}

\subsection{0702}

虚静,摄无量义。

无我曰虚,归根曰静。无我而归诸道,心与道合,是为真人。

淡泊明志,宁静致远。

\subsection{0703}

123哲学是分子结构,再往上就是系统论。一个系统由子系统构成,形成层次结构。
系统具有分形属性,一即一切,一切即一。一花一世界,一叶一菩提。

抽身物外,胜物而不伤,勿死于物下。道提供了与物沟通的另一维度,
道者,万物之奥。道者,物之极。架构师与程序员的不同,主要也是在此。
精于道者兼物物,精于物者以物物,下学而上达。

道物,粗分有两个层次,然上通九天,下贯九野,一层功夫一层理。
合中有分,分中有合。

这一关确实不好过了,走还是留,是个问题。不管怎样,都要做好充分的准备。

管理不上路,财务不合规,关键是能不能虚心听取意见,
从中获得成长,一时的成败不是决定性的。

\subsection{0704}

我注六经,六经注我。我与六经之间是超越线性的关系。为今之计,发明心地,明心见性。

寻章摘句,君子不为。以虚壹而静之心态,拥抱现实及其变化,确立道为最高原则,尊道贵德。

归纳整理出我的原则,至关重要。

系统化的决策流程,决策攸关成败,有底层逻辑,有道有术。

守、破、离对应心物,心道、道物三线,成三角形。

\hl{做决策不是我什么什么还没准备好,要相信自己的基本功与学习能力}。
精于道者兼物物,致力于道,物不会是严重障碍。

顶角即是道,也是机器、系统,看到二中之一,看着物理学之后的形而上的东西。
形而上者谓之道,形而下者谓之器。此一上行下行的路径,揭示了更多可能性。

人生算法有认知闭环:感知-认知-决策-行动,是动词构成的,心道物三者,是名词构成的。
内核与外环,内核是最小化的那个点,外环是动力与使命。认知闭环发生在心物之间,
三角形的每个边都是一个认知闭环,PDCA循环。这些小车轮,架起了友谊的桥梁。

是节点问题,还是边问题?居于中心的是什么?

把道、原则、人生算法、多元思维模型、混沌大学课程这些模型融合起来。
打造自己的模型。

取势、明道、优术,取势在心,明道在道,优术在物。
外环由心发动。

夸克构成质子和中子,1:2的比例关系。

把最近围绕道的认知,应用到工作中,在知行中螺旋上升。
一是道心物三角,二是认知闭环,三是体道方法与心态虚壹而静。

稳住,静下来,搞点大事,五年磨一剑,一战定江山。

原则:心态、机器、系统。分生活、工作、投资等领域内归纳出的一些原则。

算法:认知闭环。

多元思维模型:从硬学科里提炼基础模型,形成体系,运用到各种决策场景。

混沌大学:用第一性原理,跨越第二曲线的不连续区。

道具有最终的统一性,众星拱之。

\hl{把分布式块存储系统列入最小内核},运用即即为广泛,深度也够,待解决问题也很重要。
怎么让它最大化呢?占据铁三角的物之一角。要呵护珍惜!

同时需要从别的领域吸收养分,但这个是核心,如果能立下来一个核心,
来华云不管遇到什么,都值了。

\subsection{0705}

体道者逸而不穷,任数者劳而无功。双线法则

战略,不在战,而在略。亮独观其大略。

用心体会虚壹而静四字。

道不欲杂,道是朴素的,一立而万物生矣。

% 如果钱能到,很好。任何时候,成长都是第一位的,如果因为钱影响了成长,就得不偿失。

成长如何衡量?曰道。道是一种信仰,有道则吉,无道则凶。道之有无取决于目标。

\hl{NLP思维逻辑层次}:精神、身份、信念、能力、行为、环境。

前五个都是我,把握当下。

精神=道,身份=我,信念=原则,自此以往,皆算法。

养神之所归诸道,身份是入口、枢纽、关节点。无我,上通于道,惟道是从。
道居于太极至尊的位置,至尊而不独尊。

内静外敬,性将大定。

\subsection{0706}

正念、良知是体道见性知天命的方法。

虚静,一是尊道;二是正念,如如不动。

\subsection{0709}

惟精惟一

打磨三合架构,整合原则、算法,去分析问题

\begin{enumbox}
\item 过以原则为基础的生活
\item 更高层次思考
\item 做一个超级现实的人
\item 极度的头脑开放
\item 五步流程方法
\item 如何做出好决策?
\end{enumbox}

在心物道三者间,持续转动。用三合结构分析达里奥的原则一书。

道者,物之极。升维思考物的真实价值。回到心,是否足够空灵高效有力量,心智模式。

心,极度真实、极度开放。

原则一书,也是升维降维,上帝视角,引入机器、系统,进行控制。
机器位于物的节点,分解为目标与结果、团队与规则。

五步流程法等同于设定目标+认知闭环(感知、认知、决策、行动)。

怎么做出好决策?

限制一下悟道的时间,不必太多,时时提起。

设定下一阶段的目标,全力以赴

帝者体太一、王者法阴阳、霸者则四时、君者用六律。
太一者,理解为目标、内核,有着更为深刻的内涵。

\subsection{0710}

霍金斯能量等级,让人耳目一新。正负能级的分水岭在勇气,
知耻近乎勇,勇气是自我成长的关键要素。

可以看做情商的元素周期表,每次考察一个元素,
改善之,努力向下一能级跃迁。

舍去这些人与事,内心才能真的平静,聚焦在最重要的事情上。
没有舍怎么得呢?幻想没有什么用,拥抱赤裸裸的真实才有出息。

如果让某些人与事影响内心的平静,真的非常不好,牢牢锁定自己的方向、做自己可控的事情。

胜人者有力,自胜者强。不能自胜,何谈其它?
有些事情,转念就好,顺其自然,岂能妄为?

好好消化原则一书,能极大地促进对道的理解。不要一头扎入细节之中,
做到以道观之,按原则行事。类似的书,还有用系统来工作,管道的故事。

先成长为真正的专家,比盲目地开公司,是更可控更现实的事情。
阿里P9、P10的财务收入已不少。在这个基础上,或投资、或创业,更可期待。
再给自己三年或五年的时间吧,不要急、慢慢来。

最近对道的探索,收获颇丰,感觉离大道更近了点,心态也变得更自主、积极,思路更开阔。
这是非常正确的选择,但不能着急。孔德之容,惟道是从。孔者,从容状。

体道会影响到很多方面,心与物。

用PCDA统筹目标五阶段,1+4。一是设定目标,4是四时,PDCA、元亨利贞、认知闭环、四象限等。

如何做出好决策?这是贯彻始终的一个大问题,渗透到每一个环节。

\subsection{0711}

昨日聚餐谈及PDCA,结合原则的1+4,霸者则四时,四时交替、运转不息,
3/4也是很重要的模式。

金字塔的逻辑结构,与太极生两仪暗合,金字塔原理更着重形式逻辑,
太极两仪偏重辩证逻辑。

机器生目标与结果两极,阳变阴合而生金木水火土,五气顺布,四时行焉。
这是二与五的结合,三与四在其中。无极之真,二五之精,妙合而凝。

有此六个数,足矣。

古之王者,建国君民,教学为先。学术是大本大源,故荀子开篇即是劝学。
博学、审问、慎思、明辨、笃行,环环相扣,一气流行。
气没有固定的形状,表示空。

思维格栅,如何才能形成?道、原则的体系展开。
广泛吸收重要学科的核心概念、理论与工具。

道法术器,一气流行。气韵生动,气表明一股存在的无形力量。
不仅要看到形,更要看到神。

\subsection{0712}

太极图说,黄帝阴符经要内化于心。宇宙在乎手,万化生乎身。宇宙、万化皆一心之所裁,本出于身手,由近及远。
三合结构普遍存在,如三盗既宜、三才既安的天地-人-万物之关系。我与非我,统一于大梵之境。
建立自我、追求无我,如此我无我皆入道矣。
偏于任何一方都非究竟之道。二元对立统一方为中道正见。一而不二,是谓知道。如何统一呢?

求道予人一大格局、大视野、大机趣、大静大动。

一存在于二中为三,一存在于四中为五,大部分情况已够用。
一二四是变化序列,三五隐然其中。运转PDCA而不知一,则怠,有术无道。

变化是维度的增加。

PDCA是个周而复始的循环过程,每一个循环就带来了新的可能性,把系统带入一个新高度。

管子有四时一章,论述详备。

李中莹心智力:这个世界由无数个系统组成,每个系统都用着同一套法则运行,称为系统动力。

\subsection{0713}

\section{08}

\subsection{01}

论自由,不仅是论,更在于得到。自由不仅是认知问题,更是实践问题。恒以一德,此一德就是自由。

庄子的自由离活泼泼的现实生活有点远。行动自由是兵家必争,自由则含义更广。抓手在哪里?

2019是自由元年,经过多年磨砺,心智渐趋成熟,可以更自由地去呼吸、去奋争,去为所欲为。

\hrulefill

计算机体系结构要好好学习,胡伟武的教材为主。在一个更广泛的范围内考虑架构问题。
\begin{myeasylist}{itemize}
& 伟大的计算原理
& 多核应用架构关键技术
& 排队论
& 响应式架构
& 计算机体系结构
& 性能之巅
\end{myeasylist}

\subsection{02}

\hl{hegel的哲学,与毕建勋的三合之道}很匹配,互参互证。精神从逻辑学开始,下降到自然界,再回归到精神自身,
从空泛到充实,充实之美。在征途中,攻城略地,海纳百川的气象,又有着攻城略地的开拓进取。

为什么把逻辑置于山顶?心物二元借着道的中介作用,交融为一。

年薪百万是个坎,并不难逾越。应有计划地突破之。此为目标的量化。

拿出三分精力做管理工作,应有章法,目标导向,严格要求。

\subsection{06}

思维科学
\begin{myeasylist}{itemize}
& 一二三哲学
& 轩辕三书
& 道德经
& 王阳明
& 毛泽东
& 大秦帝国
& 黑格尔
& 波普尔
\end{myeasylist}

绝利一源,用师十倍。

\subsection{07}

\begin{shadequote}
读书就应该读功用最大,价值最高的书。人们所做的一切不都是为了有所功用吗?小功用的书比比皆是,通俗易懂,无须费多少心力即可掌握一技一长。
但这是舍本逐末了,既然要读书就应当选择探究终极之道的书,以道贯通天下万事万物,实现最大的功用。

像这样至高经典的学习,决不能跟学习普通知识一样,看几遍就扔了。若想达到极高的效用,一定要熟读成诵本书,一时理解参透不了的,也不必着急,就像牛反刍一样,经典熟记了,
就会在你的思想中蕴化,为你的思想贯通万事万物打下基础。若要思想升华,只读《阴符经》肯定是不够的,因为道是幽微玄奥的,几部经典互通参照,才能更好的认识道。
在后面的文章中。道易学宫还会继续推出诸如《老子》这样的经典。
\end{shadequote}

\say{阴符经、道德经皆为至高经典。吾道一以贯之,用道贯通天地万物。三合之道中道居于顶点,无复多疑,一定要理解其中深意。
阳明良知说,弘扬人的主体性,但并非可以取代道的绝对性。在hegel哲学里,绝对理念是贯穿始终的东西,一步一步到达真理之境。}

\enquote{原则是道的简化,有限版的道。道是开放体系,hegel哲学并非如马克思所云是封闭的脚手架。是有限与无限的统一体。
知性和理性的区分非常重要。\hl{西方辩证法如何过渡到东方辩证法,情、势、节}?}

\hrulefill

\hl{大处着眼,小处着手}。好好悟这个,大处是思维境界,小处是工作境界,吾道一以贯之。
思维、工作、道法构成三合之道。升维思考,降维贯通。一升一降,生命之圆运动是也。

狐狸和刺猬,

继续\hl{租房}住吧,省事了,主要是省时间,省下来的时间好好用来学习,提高自己,再给一年时间,需要一个质的变化。

\subsection{21}

\hl{轩辕三书},千古之绝唱,无尽之宝藏。

用行动改造理念并形成制度。做事、成事,又不是盲目地做。

\hl{一之解,察于天地;一之理,施于四海}。一即是道,也是事。实事求是,才能一以贯之。
两者是贯通的、不是支离的。道寓于事物之中。在观念上是形而上的。
hegel哲学所揭示的,精神的力量、理念的力量、认识的力量甚至比物质的力量要大。
谋事在人成事也在人。

谋势、积微,缺一不可。

提出\hl{一的哲学},合一的力量。三合之道也不是究竟,上遂而及于一的哲学。

知行合一。

实践论、矛盾论

唯物论、辩证法

\subsection{23}

我的第一本书将叫做第一哲学、一的哲学。

\subsection{26}

一的哲学,简称一学。一学非一休,一文一武,一张一弛。

夫为一而不化,得道之本,得事之要。抱道执度,天下可一也。

化书,御一。

一能贯五,五能综一。

吾道一以贯之。知行合一。

用pcda做好管理工作。

\subsection{27}

一学,配合pdca作为主要的管理工具,pdca的每个字母都是一大课题。转动pdca循环,解决现实问题。
比如,联想方法论的复盘,就是c的深化。

目的是什么?思想方法和工作方法,更重要的是学以致用,用方法指导实践活动。
归纳下来,就是一个知一个行,知行合一。

\hl{道治天下},心道物三合,道具中枢地位。以道通物,以道观天下。
原则、多元思维模型都是法,法自道生。事督乎法,法出乎权,权出乎道。

四度,\hl{春生夏长秋收冬藏},周而复始,与pdca相当。


\end{document}

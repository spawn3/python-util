% -*- coding: UTF-8 -*-
% hello.tex

\documentclass[UTF8]{ctexbook}

\usepackage{xeCJK}
\usepackage[utf8]{inputenc}

% load paralist before enumitem
\usepackage{paralist}

\usepackage{hyperref}
\hypersetup{pdftex,colorlinks=true,allcolors=blue}
\usepackage{hypcap}

\usepackage{color}
\usepackage[usenames, dvipsnames, svgnames, table]{xcolor}
% \pagecolor{gray}

\usepackage{makeidx}
\makeindex

\usepackage{amsmath}
\usepackage{mathtools}

\usepackage{listings}
\usepackage{multicol}
\usepackage{fancybox}
\usepackage{tcolorbox}
\usepackage{enumitem}
\usepackage{multirow}
\usepackage{longtable}

\usepackage{indentfirst}

% table
\setlength{\arrayrulewidth}{1pt}
\setlength{\tabcolsep}{16pt}
\renewcommand{\arraystretch}{2.5}
\newcolumntype{s}{>{\columncolor[HTML]{AAACED}} p{3cm}}

\arrayrulecolor[HTML]{DB5800}

% 摘录
\usepackage{verbatim}
\usepackage{libertine}
\usepackage{graphicx}
\usepackage{framed}

\lstset{%
    %alsolanguage=Java,
    %language={[ISO]C++}, %language为,还有{[Visual]C++}
    %alsolanguage=[ANSI]C, %可以添加很多个alsolanguage,如alsolanguage=matlab,alsolanguage=VHDL等
    %alsolanguage=tcl,
    %alsolanguage=XML,
    %alsolanguage=bash,
    tabsize=4, %
    frame=shadowbox, %把代码用带有阴影的框圈起来
    commentstyle=\color{red!50!green!50!blue!50},%浅灰色的注释
    rulesepcolor=\color{red!20!green!20!blue!20},%代码块边框为淡青色
    keywordstyle=\color{blue!90}\bfseries, %代码关键字的颜色为蓝色,粗体
    showstringspaces=false,%不显示代码字符串中间的空格标记
    stringstyle=\ttfamily, % 代码字符串的特殊格式
    keepspaces=true, %
    breakindent=22pt, %
    numbers=left,%左侧显示行号 往左靠,还可以为right,或none,即不加行号
    stepnumber=1,%若设置为2,则显示行号为1,3,5,即stepnumber为公差,默认stepnumber=1
    %numberstyle=\tiny, %行号字体用小号
    numberstyle={\color[RGB]{0,192,192}\tiny} ,%设置行号的大小,大小有tiny,scriptsize,footnotesize,small,normalsize,large等
    numbersep=8pt, %设置行号与代码的距离,默认是5pt
    basicstyle=\footnotesize, % 这句设置代码的大小
    showspaces=false, %
    flexiblecolumns=true, %
    breaklines=true, %对过长的代码自动换行
    breakautoindent=true,%
    breakindent=4em, %
    escapebegin=\begin{CJK*}{GBK}{hei},escapeend=\end{CJK*},
    aboveskip=1em, %代码块边框
    tabsize=2,
    showstringspaces=false, %不显示字符串中的空格
    backgroundcolor=\color[RGB]{245,245,244}, %代码背景色
    %backgroundcolor=\color[rgb]{0.91,0.91,0.91} %添加背景色
    escapeinside=``, %在``里显示中文
    %% added by http://bbs.ctex.org/viewthread.php?tid=53451
    fontadjust,
    captionpos=t,
    framextopmargin=2pt,framexbottommargin=2pt,abovecaptionskip=-3pt,belowcaptionskip=3pt,
    xleftmargin=4em,xrightmargin=4em, % 设定listing左右的空白
    texcl=true,
    % 设定中文冲突,断行,列模式,数学环境输入,listing数字的样式
    extendedchars=false,columns=flexible,mathescape=false
    % numbersep=-1em
}


\newenvironment{enumbox}[0]{
    \begin{tcolorbox}
    \begin{compactenum}
} {
    \end{compactenum}
    \end{tcolorbox}
}

\newenvironment{itembox}[0]{
    \begin{tcolorbox}
    \begin{compactitem}
} {
    \end{compactitem}
    \end{tcolorbox}
}

\newcommand{\hl}{\bgroup\markoverwith
  {\textcolor{yellow}{\rule[-.5ex]{2pt}{2.5ex}}}\ULon}

\newcommand*\openquote{\makebox(25,-22){\scalebox{5}{``}}}
\newcommand*\closequote{\makebox(25,-22){\scalebox{5}{''}}}
\colorlet{shadecolor}{Azure}

\makeatletter
\newif\if@right
\def\shadequote{\@righttrue\shadequote@i}
\def\shadequote@i{\begin{snugshade}\begin{quote}\openquote}
\def\endshadequote{%
\if@right\hfill\fi\closequote\end{quote}\end{snugshade}}
\@namedef{shadequote*}{\@rightfalse\shadequote@i}
\@namedef{endshadequote*}{\endshadequote}
\makeatother

\title{成长}
\author{炼金术士}
\date{\today}

% \bibliographystyle{plain}
% \bibliography{math}

\begin{document}

\maketitle
\tableofcontents

\part{正心}

\chapter{Wish}

WOOP

\section{反思}

博而寡要,劳而无功,知行脱节,有术而无道,术也就不可靠了,以往解决问题陷入物的支离境地,这是阳明学要解决的大问题。

迷信书本,买了那么多书,根本消化不了,浪费了大量的时间和精力,劳而无功。
书要看,但并非必须,都是要牢牢把握住为学头脑处。
盲目地去学习,反不如静下心来,仔细规划。

花在抱怨的时间秀的时间太多,当回归到修身为本。\hl{涵养须用敬,进学则在致知}。
此二语是今后的指导思想,少说大言,收敛身心,但踏踏实实做去,真积力久,自然或跃在渊。

遥想十年前,身处天津,前途茫茫,没有着力点。现在情形则要好很多,正是用心用力之时,不可随意荒废。

知识体系只是构成了一个背景,能体现价值的地方,还是要看解决了什么大问题,大问题是成长之树的树根。
看看黎曼猜想引发的全球关注,即可知。战略思维的大开大合,离不开大问题的驱动。

围绕着要解决的大问题,如何打造专业知识体系?

\section{立志}

富有之谓大业,日新之谓盛德,生生之谓易。

10X成长

\section{工具}

\subsection{参伍以变}

双线法则、圆点哲学、参伍以变形成为主要的方法论。

\subsection{六时书}

\begin{tcolorbox}
这是唯一让我们能够实实在在得到智慧的好处的最方便可行的方法, 如果不去记录六时书, 
我们实在是 “如入宝山, 空手而归”,让人可惜得扼腕而叹。

六时书可帮助我们建立“追踪体系”——帮助我们训练意识,因为身为在地球上的人,
几乎每天都是负面念头多于正面念头的,也就是说这是自然人的本来性质,
但这一性质必定导致大多数人陷入负面事件越来越多导致产生更多负面念头的恶性循环里。

这个规律就如同地心引力一样,是一个定理——一个大多数人都逃不开的规律。
但是人类既然能够逃脱地心引力上天入地,同样的,我们也可以摆脱这个“负多于正”的规律,
通过六时书的帮助——训练我们的念头,使负面念头越来越少,正面念头越来越多,
成为一个觉悟的、开悟的人。到时候,生活中不论发生什么事,
我们都能以最正确的态度去对待,一切问题都将不是问题。
\end{tcolorbox}

六时书这种结构化的方式较好,考虑问题趋于全面。

示例: 专心致志,不抱怨

\begin{lstlisting}
+ 正
- 反
O 合
\end{lstlisting}

\chapter{用系统来工作}

\section{战略目标}

千事万事,只是一事。合内外,致良知,开物成务,赚大钱。

\section{经营总则}

采用\hl{PDCA循环}持续优化,僵化固化优化。

大学之道,在明明德,在亲民,在止于至善。

孔德之容,惟道是从。

选择,少即是多。精挑细选要从事的事情,做正确的事,才有望得到正确的结果。
精力有限,不能不慎重抉择。

做事从容,有节奏,有条理,有所不为。

诚则明,明则诚。

尊重专业的力量,尊贤使能,利用一切场所向专业人士学习。

早起

定量、有恒

坚持写日记。

识别独立的子系统,编制工作程序文档,并不断地自动化。

\section{工作程序}

\subsection{工作子系统}

此处记录通用部分,关于工作系统的详细信息,参加lich相关文档。

\subsection{学习子系统}

\subsection{生活子系统}

\chapter{阅读}

\section{方法}

以五行为中心

道
\begin{enumbox}
\item 道德经
\item 易经
\item 大学
\item 中庸
\item 传习录
\item 近思录
\item 画道精义
\item 一二三哲学
\item 大臣之道
\end{enumbox}

\section{专业}

存储
\begin{enumbox}
\item 大规模分布式存储系统
\item 数据存储技术与实战
\item 操作系统 (IO层)
\end{enumbox}

大问题
\begin{enumbox}
\item Chunk副本一致性
\item 事务管理:并发控制与故障恢复
\item QoS
\end{enumbox}

架构设计
\begin{enumbox}
\item 软件系统架构
\end{enumbox}

数据结构与算法
\begin{enumbox}
\item 算法导论
\end{enumbox}

Linux
\begin{enumbox}
\item Linux/Unix系统编程手册
\item Linux性能优化
\item Linux内核分析及应用
\end{enumbox}

Big Data
\begin{enumbox}
\item 精通数据科学
\end{enumbox}

系统
\begin{enumbox}
\item Ceph
\item SheepDog
\item GlusterFS
\item Lustre
\item LevelDB/RocksDB
\end{enumbox}

从客户端来看,设备文件对应分布式存储系统中的逻辑卷。
逻辑卷与物理卷具有相同的控制参数。

在分布式存储系统内部(单机存储引擎),NVMe有两种访问模式:设备文件、通过pci号直接访问。
后一种访问方式性能高。若经过设备文件,则通过kernel IO栈。

Ceph导出了\hl{对象层API},在其上构建统一存储:block,fs,object。fs具有最复杂的语义,block则对latency要求高。

\section{领导力}

\section{知识}

战略
\begin{enumbox}
\item 战略罗盘
\item 好战略、坏战略
\end{enumbox}

投资
\begin{enumbox}
\item 投资最重要的事
\end{enumbox}


\part{明道}

\chapter{导言}

\section{导言}

研判形势,淬炼心法,有所为,有所不为,乃至无为而无不为。

修道而保法,故能为胜败之政。

\begin{shadequote}

    道生一,一生二,二生三,三生万物。\\
    道生之,德蓄之,物形之,势成之。
\end{shadequote}

精一之学,体用兼备。
\begin{shadequote}

    天地之道,可一言而尽也:其为物不二,则其生物不测。\\
    天下之动,贞夫一者也。\\
    圣人抱一以为天下式。\\
    恒以一德。
\end{shadequote}

太极哲学,双线法则,圆点哲学,一分为三,提供了诸多值得反复体味的命题。

一,切己言之,就是事业,须更上一层楼。一是整体,是根据地,是不间断,也是突破点。

博厚,高明,悠久。

空灵之境,有无相生,有生于无。空非空寂,众缘所起,云行雨施,品物流行。

太极本无极。

上溯,万法归一,一归空。

五轮书,地水火风空。

建立自我,追求无我,是逆向工程。下学而上达。

\section{战略,或道}

\section{方法谈}

爱因斯坦说过这句话:我们不能用制造问题时同一水平的思维来解决问题。也许他意味着我们需要摆脱与我们对一个问题有关的消极的看法。如果我们对问题本身太投入,那么我们永远无法越过这个局面。

在一本叫“治愈与复原”中,David R. Hawkins详细阐述了这一点。他说,“问题最好不要在他们发生的同一水平上解决,而是在他们的上一个阶级上解决...通过超越他们,从更高的角度看待问题,问题很容易迎刃而解。
较高层次上,由于这种观点的转变,问题会自动解决,否则人们可能会看不到任何的问题。”

很多时候,我们面对一个问题时,总会把精力集中在问题上,一直问怎么“解决”呢?我们可能最终会走入死角,沮丧。
因为我们似乎找不到很好的解决办法。无论如何,不要把精力集中在问题本身上。花几分钟时间,花费你的时间和精力来正面地解析。
我们无法控制经常会有事情出现的,不要浪费时间担心这些事情;只花时间在你可以改变或控制的事情上。

\subsection{中庸}

\subsection{圆点哲学}
\subsection{双线法则}
\subsection{黄金分割率}
\subsection{80/20规则}
\subsection{黄金圈法则}

\subsection{达里奥的原则}

欲达到我们的目标,必须实事求是,客观公正地面对现实,正视自身的缺点和不足,而有以克服之。

这是真的吗?求真是第一位的,吾爱吾师,吾更爱真理。对道听途说的观念,我们固然要保持警觉和必要的批判精神。
对自我意识,也要慎思明辨。保持开放之心和专注之念,对自己的观念做压力测试,力求准确更准确。
而不能陷入先入之见,或自欺欺人,没有荣辱,只有是非。不当的虚荣心和自尊心会妨碍通向真正的目标。

对我们不知之物,保持谦卑,保持饥饿,保持愚蠢。

选择至关重要,我们必须承担选择的后果,为选择负起责任。
弱点,由弱点导致错误,皆在所难免。
但由此错误,吃一堑长一智,如果能通过反思而增强了自己,就是有益的。
从错误中学习,进步,进化,是通向成功的捷径。

任一选择,都带来其效应和影响。一阶效应也许不错,但二三阶效应可能已变形,
祸福相依,需要更多的洞见。关键的选择,决定了我们人生的质量。

成长,或曰进化,是唯一的目的。财富,名利皆是果,而不是因。
当我们围绕成长,而动心忍性,增益其所不能的时候,就是走在自我进化的路上。

自我进化,有一五步法可资遵循:
\begin{enumbox}
\item 设定清晰的目标
\item 觉知问题
\item 诊断问题
\item 设计方案
\item 执行方案
\end{enumbox}

五步法是迭代过程。每一步都需要投入必要的资源,做选择,做策划。

对比目标和输出的不同,找到不足,做出适当的调整,类似于PDCD。不妨想象,有台巨大的机器,作为输入输出的中介。
我们的核心任务,就是维持机器的良好运行和高效产出。

资源调度,采取开放的视角,并非一定需要我们亲力亲为。
我,即是设计者,也是执行者,主要作为设计者而存在。
任何人都非全知全能,而是有长有短,管理者的职责,在于知人善任。

不必为自己的弱点而沮丧,君子性非异也,善假于物。

唯一的目的,就是自我进化。唯一的事,就是打造机器。
机器是我们拥有的容器,即是心法,也是产品。我们是机器的架构师。

单纯观念,不足以动人。做出作品,持续产出,才能实现自我价值,立于不败之地。

实有诸己之谓德。默默地完成进化,是最明智的选择。围绕选定的一,厚积薄发,静水深流。

道生一,一即是战略,也是方法。

原则,架构起了价值和行动的桥梁。让我们有所遵循,持续积累,而不是茫然无措,本末倒置。

佛陀的教导

以戒为师。戒可释为原则,或良好习惯。

大乘起信论的一心二门的义理架构,予人深刻启示。

达里奥与王阳明

良知是比原则更基础的范畴,良知是一种元认知能力。
致吾心良知于事事物物,则事事物物皆得其理。

达里奥求真的意志和可操作性,较阳明为突出。
资本主义的熏陶,更适应于现实人生。

毛主席在其著作中,深入分析了认识的各种问题,如主观主义,教条主义,经验主义,
统称为主观主义,即主客观的分裂和不一致。以此指导行动,则误导行动。

马利克的管理学,采用系统论,控制论和仿生学等知识,以应对现实世界的复杂性。


\chapter{制胜之道}

制胜之道是在任一领域都可以胜出的道的探求,简称胜道,确立此点为今后数年的研究主题。
一切学问,都是为解决现实问题而来。知行并进是基本要求。

胜道离不开战略思维和系统思维的研修,是运用战略思维指导实践活动而心想事成,是目的和工具的统一。

胜道也即是幸福之道,胜利是为了幸福。

要紧紧把握住胜利这一根本目的和价值所在,此心不动,随机而动。

心能转物,即同如来。胜含义深广,人定胜天,是做一切事的目的所在。除了胜利,一无所求。
严肃认真地对待任何重要的事情,都要把终局的胜利作为考量的首要因素。

从这个角度来说,三合之道也是工具。

叔本华作为意志与表象的世界,尼采之全力意志,此所谓权力意志,即是求胜的意志,冲决罗网,百折不回。

金一南的系列著作,和君的三度修炼丛书,领域不同,具体目标也不同,但求胜制胜的意志则是一样的。
或为了国家的利益,或为了个人的幸福,修齐治平之道的根本,就是敢于胜利的意志。为了胜利,不择手段。

最激烈的对抗,发生在军事、经济、政治领域,拳击、中医、棋艺都是对抗的艺术。对抗的二元性是事物的本质。
问题的关键是:如何在对抗中,立于不败之地,而不失敌之败也?即是致人而不致于人。

回到自身、一切的答案来自于自身,自律是通向自由的桥梁。敬以直内,义以方外,敬义立而德不孤。
自身代表了能动性的一方面。

战略思维是制胜的核心。

\section{做好自己}

\begin{shadequote}

十年前我很关心全世界,结果我的日子过得非常艰难;五年前我很关心中国的命运,我也过得很艰难;
三年前我开始只关心公司,我的日子开始好起来。现在我只关心自己,越来越好。
\end{shadequote}

你若盛开,蝴蝶自来。吸引力、感召力不是靠宣传,而是用脚投票,自热而然。刻意去要求什么,反而毫无效果。

\section{三才之道}

\section{三合之道}

从三合之道说胜道,是致思原点。三合之道是易经和老子思想引出的一朵莲花。游心道物间,双环交融,终成正果。

三国演义中刘琦与诸葛亮谈到六合阵法,第一则是心法。

心的塑造能力,不能单纯地听天由命,高扬精神的主体性,斗志昂扬,化腐朽为神奇。不甘心做命运的奴隶,而是要命运的统帅。
这种主体性,努力地去探求制胜之道。心能转物,即同如来。

在三角形架构中,如果统一邵雍的心者太极也,大乘起信论的一心二门的命题?

依然把道放置在三角形顶点,心上通于道,而下及于物,诚神几曰圣人。

物外在于心,通过道而达一心物、合内外之境。心物为二,形而下者之谓器。心即理,致良知,良知并非现实的完成之物,而是生成的进化的状态。

\section{椭圆曲线}

心物分居椭圆的两焦点上,而道则是椭圆曲线形成的轨迹。道者,体常而应变,一隅不足以举之。

圆点哲学是椭圆曲线的极限形式,两点重合为一点。二合一,收敛为更为强大的力量。

尖椭圆

\section{PDCA}

参伍以变,错综其数。参即是上一节的三合之道,伍则是四象五行。

\section{战略罗盘}

\chapter{神圣几何}

战略几何学,数、形、理的结合。

数据结构,树状、图论。

\section{点}

\subsection{单点突破}

道生一,一是战略,又是战术。

\subsection{圆点哲学}

根据地思维,进可攻退可守。

\section{线}

\subsection{双线法则}

守住底线,抓住关键。

\subsection{三才之道}

天地定位

\subsection{S曲线}

\section{面}

\subsection{三合}

易有太极,是生两仪。

参伍以变,错综其数。

\subsection{五行}

1+4=5,2+3=5。内涵圆点哲学、双线法则、三才之道、三合之道,四象,是多个模型的融合。

道天地将法。

\subsection{河洛}

\subsection{六图思维模型}

\subsection{战略罗盘}

\subsection{PDCA}

\hl{法尔科尼管理的方法}提及,方法只有一种,即笛卡尔的方法,形式化为PDCA。
运转PDCA的目的在于解决问题、固化优化等。

集中在PDCA上,以PDCA为主,其它模型作为分析工具。五比三更为丰富的内涵、兼容了一二三四。
\hl{成中英}就有尚五的倾向。

1+4=5,一是目标,居于圆心,起统摄作用。

这与中国传统的五行思维法很契合,黄帝内经就是五行认知体系。

河洛都是五行结构。

孙子计篇,开门见山就说:\hl{经之以五事,校之以计,而索其情}。道天地将法,五事是正。守正方能出奇。

以个人成长为实例,运用该模型,实现10X成长。

\subsection{SWOT}

\subsection{坐标}

\subsection{正弦曲线}

\subsection{椭圆}

两个焦点,圆点哲学与三合之道的融合。

\subsection{因果循环图}

\section{体}

\subsection{双环}

垂直正交双环,电磁感应

\section{五行}

\subsection{三合}

心道物

\subsection{三才}

天地人

融合三合三才,则天地定位、内圣外王(心物)。

\subsection{五事}

道天地将法,将居于中央,将者,心也。天地定位,左道右法(范蠡答越王问),精神内敛,一心所摄。

天地是台阶,更上一层楼。善守者,攻于九天之上;善守者,藏于九地之下。拳打十万八千遍,一层功夫一层理。
理者,超认知。

九层之台起于累土;千里之行始于足下。

升维思考、降维贯通。

\subsection{战略罗盘}

设计、执行

内外

\subsection{战略菱形}

无战略、悲人生:全局性、长期性、风险性

强烈的问题意识、彻底性、进取性

\subsection{双环}

两个正交维度,一横一纵,纵横交错

八面体

\subsection{PDCA}

\chapter{周易}

以易立身

高扬心得主体地位,世界作为意志的表象,虽然不能不遵循至高无上的天道,但良知良能,神机妙算,确实切身的根本。
天地设位,而易行乎其中矣。成性存存,道义之门。心处天地之间,成三才格局。在三角形中,据于顶点,左右为道为物。
道居顶点,不能彰显心得主体性。心居顶点,则盛神、机发、良知良能、尽心、诚皆自然而然。




\chapter{志于道}

道是起点,也是归属。

志于道、据于德、依于仁、游于艺,此四事,实是一事,志于道,道之展开,囊括无遗。

立志,立何等志?本立而道生。道是目的也是方法。

惟精惟一,允执厥中。即是志于道,阳明以良知二字来统一,老子则以道为归宿。

志于道而不遗物,秋水:以道观之,物无贵贱,厚德载物,取舍得当。
如此老庄优劣,佛道儒的差异,并不重要,重要的是如何为我所用。

观天之道是升维思考,执天之行是降维贯通。
一升一降,则心物一元,游心与无为,处无为之事,行不言之教。

取势、明道、优术,皆明道事,道明则取势、优术在其中。此尚无法概况道之全域,
华严四法界说:事法界、理法界、理事无碍、事事无碍。
从理开始,进入思辨之境,理事圆融,事事无碍,齐物之论。

以下一的哲学、人生算法、原则、战略都属于道的范畴。
读书何为?在明道而养德。德成而智出,万物毕得。

\section{心术}

以心受道,以道控势。顺势而为,非道不行。有道则吉,无道则凶。
然则何为道也? 心如何受道得道,进而心想事成?

心性修养,不可谓不重。此精神力量,神存兵亡,乃为之形势,是做事的根本。
专业技能也好,领导力也好,为了更好的发展,是必要条件。

大学讲正心诚意,阳明心学大力发扬了这一命题。

道微妙难测,需要心术去把握。观天之道,以何观之?心眼也,思维方式也。

心生于物,死于物,机在目。目之所见,对心有正反两方面的作用,要扬长避短。

鬼谷著本经阴符七术,即要解决心术的问题。必有圣人之心,以不测之智,而通心术。
心能得一,乃有其术。

管子四篇:心术上下,内业,白心,关注的也是这一课题。

孙子曰:将军之事,静以幽,正以治。也是强调心性的力量。

\section{一体万化}

\subsection{空}

\begin{enumbox}
\item 空虚不毁万物为实
\item 天下万物生于有,有生于无。
\item 虚者道之常也,因者君之纲也。
\end{enumbox}

此空明之境,显示道的灵活性。

道生一,一生二,二生三,三生万物。一者,道之纪也。一具有中枢地位,万事开头难。

道是一,也是万,通书:是万为一,一实为万,万一各正,大小有定。

月印万川,一体万化,彰显了道的威力,一处悟入,处处得益。

道如太阳,行星围绕着它运转不息。
道如原子核,电子围之运转。

\subsection{无为}

道常无为而无不为,此语点出道的体相用。以无为体,以有为用。

从自然的角度讲,无为是因循自然,辅相万物之自然而不敢为。
从设计的角度讲,道则是大设计,透过系统设计,达到系统控制的目的。
这样两个角度、两个层次是统一的整体。

无为是出于自然而超乎自然的:裁成天地之道,辅相万物之宜。
并非一切顺其自然,而是为无为,设计出可以无为的系统来,而节约了劳动力,达到以小博大、无为而治的目的。

文化、制度、原则都是一些无为而治的措施,而不是所谓无政府主义、放任不管。

以正治国,以奇用兵,以无事取天下。这句霸气之极,没有奇正的合理应用,又怎么做到无事取天下呢?
无为是最高明的领导哲学,道生法,修道而保法,与法有着密切关系。

反思过往,不能不感到惭愧。无为理念知之很久了,依然把握不住关键,难得逍遥之旨。

顺天应人

\subsection{道艺合一}

\begin{enumbox}
\item 体道者逸而不穷,任数者劳而无功
\item 道者,一立而万物生矣
\item 得一之道,而以少治多
\item 执玄德于心,而化驰若神
\item 托小以包大,在中以制外
\item 万物之总,皆阅一孔;百事之根,皆出一门
\item 上通九天,下贯九野
\item 依道废智
\end{enumbox}

以上引言表达了一种辩证法,在其中若隐若现的道,有着神奇而巨大的力量,
是现实的扭曲力场。

体道者能牢牢把握事物发展内在的节律,合乎桑林之舞,乃中经首之会,
在方向与方法上都有切实的掌控感。

小智慧者,沉溺于一己的小得小失,迷失了大道,博而寡要,劳而无功。
一旦尊道而贵德,不但读书时更能领会其中的精要,内心也变得宁静祥和。
长线思维、立足不变、简洁、多元模型、第一性原理,都变得那么自然,
浑然一体,都是兵器库里的兵器,在内力的支撑下,任何兵器都变得得心应手。

现在谈道依然过于抽象,艺的考量变得更为重要。
我的艺是什么?软件编程与架构。这也是要求技术与艺术并有的领域,其中的二元对立比画道并不少见。
阴阳有无分合动静虚实,相生相克,有感斯应。

比如面向对象的分析与设计,就要求做到自然,就是切合实体以及实体之间关系的本义,接口交互起来自然而然。

再如需求与功能,需要进行艰苦的取舍。说需要外师造化,内得心源,并不为过。

毕建勋的画道研究,做出了很好的示范,沉潜画道二十余年。高焕堂的EIT造型,也有启发,以收触类旁通之效。

这是专业方面的启发。另一重要的方面,是事业与生活。专业做出事业,事业完善生活,彼此可为一体。
有道贯通其间,尽量做到有分有合,分中有合,合中有分,分分合合,旋转相生。

道上通九天,下贯九野,登高临下,无失所秉,履危行险,勿忘玄伏。

\section{信解行证}

\subsection{信}

道是一种信仰,生起好奇心,去了解、去上下求索、去身体力行。

道是一中心范畴,千变万化而不离其本,是环的心,也是环自身。道有其体,又有在各个领域里的广泛应用。

\subsection{解}

道者,生生之道,增长之道,一气流行阴阳变化之道。
一者,道之纪也。

深入道源去原道。观天之道,执天之行,尽矣!以天地之道去洞察万物,则易知易行。

道是人生算法,也是第二曲线的哲学;是第一性原理,也是10X增长。

\subsection{行}

\subsection{证}

\section{一的哲学}

多年困穷,不得其要,皆由于一之不立。不恒其德,或承之羞。

中庸言诚,至于至诚不息,不息则久。天地之道,可一言而尽,其为物不二,则其生物不测。
精一之学,体用兼备。绝学无忧。身怀绝技方可臻于自由之境,止于至善。

商君书:
\begin{shadequote}

治国而能抟民力而壹民务者强;能事本而禁末者富。

故圣王之治也,慎法,察务,归心于壹而已矣。
\end{shadequote}

恒道,持续行动的复利。专注,聚焦,真积力久。

\chapter{一二三哲学}

毕建勋的一二三哲学,一读之下,心有灵犀。

一二三哲学旨在为中国画学建立理论体系,但有着更为普适的意义在。
自庞朴提出一分为三论之后,对此问题思索已久,依然不得要领。
毕建勋的一二三哲学,多有精妙之论。

\section{三合结构}

\subsection{心}

\subsection{道}

道的演化路径

围绕老子“道生一,一生二,二生三,三生万物”,
周易“易有太极,是生两仪,两仪生四象,四象生八卦,八卦定吉凶,吉凶生大业”而立论。

兼三立两迭用推一,上行方法,升维到道的高度,与道的演化方向相反,正反合一,是完整二一关系。
一而二,二而一。以上行方法见道,以下行方法实践,是为道知,不出户而知天下,不窥牖而见天道。

阴阳三合,何本何化?

道具有三合结构,同时居于三合结构的最高点。道高于两级,基于两极。一故神,两故化。
道是价值论、认识论与方法论的统一。

气为道之体/相,有无为道之性,一二三为道之理,二一之法为道之法。三合为其空间结构。以上为一二三哲学。

道以有无为性,以气为体,以一二三为理,以二一为法。
此气化结构,内含数理法,体相用,数理法。

道以常有无为性体,以气为相,以万物为用。以一二三为数理,以二一为法。

观物取象,立象尽意。执大象,天下往。大象者,模型也。

\subsection{物}

\subsection{有无}

道者,有无之总名。

\subsection{理气}

\subsection{兼三}

\subsection{立两}

\subsection{二一}

三合之道有上下的层次关系,也有左右对称关系。层峦叠嶂,至于无穷。
体道活动就是立二参一,由现象之二,推而上行,臻于含二之一,即是太极笔法。

道之性是有无,故道变动不居,周流六虚,上下无常,刚柔相易,不可为典要,唯变所适。
道是确定性与不确定的统一。

运用到画学,则道心物为三合之道。

\section{脉络}

\subsection{一分为二}

\subsection{一分为三}

\subsection{中庸}

\subsection{道知}

鬼谷子

\subsection{天解}

淮南子

\subsection{致良知}

秉道通物:知良知,心与道合,事上磨练,心物合一。
心与道合,谓之道心,以道心应物如镜。

良知包含伦理之心,知觉能力、思维力、知识的唤醒等,不仅仅是直觉,也包含分析,是心之力的统称。

\subsection{志量识}

志,心之所主。

曾国藩:有志、有识、有恒。和君三度修炼:态度、气度、厚度,分别对应志量识,很是贴切。
态度决定命运,气度决定格局,底蕴的厚度决定人生的高度。

\section{感悟}

二叉树统一了老子与周易,即是一分为二的序列,也是阴阳三合的空间结构。

天命之谓性,率性之谓道,修道之谓教。心与道合,天人合一,即是人的天命,内圣外王之道,于斯立矣。
心与道合,有赖于心术。心性修养,心灵质地,是其基础。

三合之道成而万事备

有无者道之性妙之门

有无玄同大制不割,黑白之间灰度存焉

\subsection{参一治二}

一者道,二者心与物。始于心,心与道合,下及于物,此之谓秉道通物。

致良知,良知为何?在心之天理。何以致之?离不开事上磨练,知行合一的功夫。

本体与功夫,心与道合是便捷之路。执道要之柄,以游于无穷之地。

志于道、游于艺,两者互为因果,有无相生,两仪立而太极现。

在道心物的三合结构下,可以解释诸多问题和现象。

\chapter{原则}

原则近似于我之所谓道,人何以之道?曰:心。心何以知?曰:虚一而静。
精于道者兼物物,一于道而以赞稽之,则万物官矣。

\section{尊道}

道德仁义礼,五者一体也,而道为之主,故第0原则即是尊道。
然何以尊之?需明法。

老子七善,三学六度

\subsection{至诚不息}

君子养心莫善乎诚,致诚则无它事也。

实事求是,拥抱现实,超越现实。

\subsection{虚壹而静}

是大原则,每临大事有静气,不信今时无古贤。

心善渊

将军之事,静以幽,正以治。

\subsection{圆点哲学}

最小最大模型,立其环中,以应无穷。

\subsection{双线法则}

\subsection{123哲学}

\subsection{机器之喻}

欲收无为而治之效,不能不着重在打磨机器、系统上,建立系统思维。
自组织、自进化的系统是工作的产物。

用系统来工作

\section{工作之道}

以终为始

要事第一

全局优化(统合)

\section{生活之道}

闲居静思则通

\chapter{算法}

\section{排序算法}

\subsection{Insertion Sort}

\subsection{Shell Sort}

\subsection{Select Sort}

\subsection{Heap Sort}

\subsection{Bubble Sort}

\subsection{Quick Sort}

\subsection{Merge Sort}

归并,多路归并,外排序

\subsection{Counting Sort}

\subsection{Bucket Sort}

\subsection{Radix Sort}

\section{查找算法}

\subsection{二分查找}

\subsection{TOP N}

\chapter{战略}

专业学习之外,把战略研究列为今后几年的重点。

更喜欢演绎法,如欧几里得几何的那种公理化方法,斯宾诺莎的伦理学,达里奥的原则,金岳霖的论道都采用了该方法。
道生一,一生二,二生三,三生万物。一为取势,二为明道和优术,1+2=3,三者匹配,可运用于每一领域。


为什么?

因为战略很重要,战略赢是大赢,战略输是大输。孙子:兵者,国之大事,死生之地,存亡之道,不可不察也。
不具备强大的战略思维能力,就很难实现远期的发展目标。

多聚变为一,一裂变为多,分合为变。以正治国,以奇用兵,以无事取天下。战略明,则可无事。否则,陷入事务之中,而无结果,可悲。

欲研习战略,需读经典,多实践,请教高人,开阔视野,放大格局。在解决现实问题中,融会贯通,知行合一。持续优化,不断把认知引向深入。
道德经,孙子兵法,商君书等,皆为经典。西方亦有经典,然不够精炼。待心有所主,则可进一步泛观博览。第一步,则在知止,懂取舍,有所不为。

战略罗盘之喻,精当。战略几何学,形象生动。柏拉图言:不懂几何者不得入内。以几何去研习战略,可收简洁精当,生动形象之效。

专业学习和战略研究,可谓一文一武,一阴一阳,一张一弛,相互促进,相得益彰。
战略研究要有更多的问题意识和进取精神,不仅仅是知识的获取,而为我所用,服务于最终目标的实现。
战略研究的境界,可用中庸的致广大而尽精微,极高明而道中庸形容之。
层次则有历史,科学,艺术,哲学。

\section{问题}

大学:物有本末,事有终始,知所先后,则近道矣。认识事物的轻重缓急,按其行动,就接近道了。按重要性和必要性维度进行分解,是柯维几本书的一个重要内容。
李笑来在财富自由之路中,一语中的,作为终极问题。对该问题的反复审问,是磨练价值观的利器,有助于提高选择和决策能力。别的问题都是术,这个问题近乎道。

所谓选择,即是增加必要的条件。尽量必要,尽量充分,最小完备集,奥卡姆剃刀原则在决策问题上的应用。李笑来所说的万能钥匙,即是NLP的换框法,转换视角。

\section{维度}

何为维度?升维思考,降维攻击。把重要维度都列出来,从中选择优势维度,扬长避短,有所取舍,特色组合,从而构建核心竞争力。
价值链分析如此,蓝海战略也如此。互联网,成本等都可以成为分析的重要维度,如差异化竞争,互联网对+,免费等常用的竞争策略。

维度,或者说条件,要素,李笑来有个认识:增加条件。

整合,跨界,爆裂,裂变都是这个核心思想的变形。借助技术手段,多维整合为一维,或一维细分为多维。
技术,市场和自然所谓3M力量,在塑造未来世界。

升维思考,降维贯通。维度一概念,用代数方式研究,线性空间。在数学和战略研究之间,架起会通的桥梁。数学和战略之间,有着深刻的联系。
构造结构,识别模式和关系,掌握变化的趋势。

蓝海战略的价值链分析,波特的五力分析着眼于产业竞争分析。价值链分析通过加减乘除来构建自己的优势维度,有所为有所不为。

维度是立体的,层次的,从分形的角度看,不仅仅有整数维,也有分数维,涌现出奇异的系统特性。

当代科学进展,描述了令人惊奇的时空结构。按照量子力学和相对论的认识,物质,能量,时空都呈现了超出直觉的功能和特性。
对微观粒子结构的认识在深入,对宏观时空结构的认识也在深入,在极小和极大的时空尺度上,都存在一些真正的大问题。

所谓认知,就是对维度的认识。志如其量,量如其识,三位一体,相互影响。量,开放心态,识,见识,认知水平。

\section{老子之道}

战略研究有层次境界之别,重点是理解并运用老子的一句话:道生一,一生二,二生三,三生万物。
其中一是重点的重点。侯王得一以为天下正,得一,则万事毕。

一生二,因二以济民行。一个模型是双线法则,守正出奇。守底线,抓关键。修道保法,故能为胜败之政。

老子之道,博大精深,内涵基本原则。马利克的管理,植根于系统论、控制论和仿生学等现代科学,耗散结构是另一值得注意的。
道生一,一即战略;精心守一,参悟商道。以道为中心,通达战略和管理,促进成长和进化。

半部老子治天下,围绕老子,通达道家思想。没有厚此薄彼,抱一而为天下式。老子文约义丰,且较为熟悉。

为自己打造一口深井,由老子承载大学之道,修齐治平,尽在其中。

从认知升级的维度读老庄,会有趣得多。

\section{圆点战略}

圆是格局,点是破局。圆是调查分析,点是指导方针和行动序列。

\section{双线法则}

\section{西方战略管理思想}

战略始于问题。战略七问,德鲁克五问,如何回答这些问题需要深入分析。黄金圈法则,明确了问题的顺序。

受到商业机构成功的启发,比尔·盖茨在今年的公开信中提出了一套“成功法则”:量化目标―选择策略―考量结果―调整策略―实现目标。
比尔·盖茨认为,这不仅仅是商业机构成功的秘诀,致力于扶贫帮困解决社会问题的非营利机构同样应遵循这一法则。

PDCA是唯一的管理方法,法尔科尼管理方法。

双环学习,前提批判,在表面原因之外,有更深层的原因或约束。
在尝试解决问题之前,需要更深入的调查分析,找到问题的根本原因,而不是流于表面,治标不治本。这引向了系统动力学的视野。

战略是可行性的假设,需要持续的压力测试,来检验其正确性和有效性。

机器的隐喻,有机体的正常运转。可以把组织看作一台机器或有机体,机器是在正常运转吗?
用控制论去理解,控制论适用于机器和生命领域。

\section{价值链分析}

场景,价值链,商业模式。价值链拆分为要素和连接,归核,裂变,融合。效率,成本和差异化。

每一要素是一维度,如孙子兵法的五事七计,就是五个维度及其量化。五个维度构成一向量,或矩阵。
这是认识事物的一般方法。

德鲁克五问:

\chapter{精选}

诸子:
\begin{enumbox}
\item 易经
\item 中庸
\item 道德经
\item 文子
\item 黄帝四经
\item 黄帝阴符经
\item 鬼谷子
\item 管子
\item 素书
\item 长短经
\item 太极图说
\item 通书
\item 武经七书
\item 武艺二书
\end{enumbox}

佛经
\begin{enumbox}
\item 心经
\item 金刚经
\item 坛经
\item 大乘起信论
\item 圆觉经
\end{enumbox}

近人之著作
\begin{enumbox}
\item 一二三哲学
\item 东方战略学
\item 李小龙
\item 查理芒格
\item 孙正义
\item 人生算法
\item 第一性原理
\item 第二曲线
\item 基业长青
\item 从优秀到卓越
\item 黑天鹅
\item 反脆弱
\end{enumbox}

\section{太极图说}

无极而太极。太极动而生阳,动极而静,静而生阴,静极复动。
一动一静,互为其根。分阴分阳,两仪立焉。
阳变阴合,而生水火木金土。五气顺布,四时行焉。

五行一阴阳也,阴阳一太极也,太极本无极也。
五行之生也,各一其性。无极之真,二五之精,妙合而疑。
乾道成男,坤道成女。二气交感,化生万物。万物生生,而变化无穷焉。

惟人也得其秀而最灵。形既生矣,神发知矣。五性感动,而善恶分,万事出矣。
圣人定之以中正仁义而\hl{主静},立人极焉。

故圣人与天地合其德,日月合其明,四时合其序,鬼神合其吉凶。君子修之,吉;小人悖之,凶。

故曰:“立天之道,曰阴与阳。立地之道,曰柔与刚。立人之道,曰仁与义”。

又曰:“原始反终,故知死生之说”。

大哉易也,斯之至矣。

\section{太极拳谱}

\hl{太极者,无极而生,动静之机,阴阳之母也。
动之则分,静之则合}。无过不及,随曲就伸。
人刚我柔谓之走,我顺人背谓之粘。
动急则急应,动缓则缓随。
虽变化万端,而理唯一贯。
由招熟而渐悟懂劲,由懂劲而阶及神明。
然非用力之久,不能豁然贯通焉。

虚领顶劲,气沉丹田。不偏不倚,忽隐忽现。
左重则左虚,右重则右杳。
仰之则弥高,俯之则弥深,进之则愈长,退之则愈促。
一羽不能加,蝇虫不能落,人不知我,我独知人。
英雄所向无敌,盖皆由此而及也。

斯技旁门甚多,虽势有区别,概不外乎壮欺弱,慢让快耳。
有力打无力,手慢让手快,皆是先天自然之能,非关学力而有为也。
察四两拨千斤之句,显非力胜;观耄耋能御众之形,快何能为。
立如平/秤准,活似车轮。偏沉则随,双重则滞。
每见数年纯功,不能运化者,率皆自为人制,双重之病未悟耳。

欲避此病,须知阴阳。粘即是走,走即是粘。
阴不离阳,阳不离阴。阴阳相济,方为懂劲。

懂劲后,愈练愈精,默识揣摩,渐至从心所欲。
本是舍己从人,多误舍近求远。
所谓差之毫厘,谬之千里,学者不可不详辨焉。

\section{阴符经}

\subsection{原文}

\hl{观天之道,执天之行,尽矣}。
故天有五贼,见之者昌。
五贼在乎心,施行于天。宇宙在乎手,万化生乎身。
天性,人也;人心,机也。立天之道,以定人也。
天发杀机,移星易宿;地发杀机,龙蛇起陆;人发杀机,天地反覆;天人合发,万变定基。
性有巧拙,可以伏藏。九窍之邪,在乎三要,可以动静。
火生于木,祸发必克;奸生于国,时动必溃。知之修炼,谓之圣人。

天生天杀,道之理也。
天地,万物之盗;万物,人之盗;人,万物之盗。
三盗既宜,三才既安。故曰:食其时,百骸理;动其机,万化安。
人知其神而神,不知其不神之所以神也。
日月有数,大小有定,圣功生焉,神明出焉。
其盗机也,天下莫能见,莫能知也。君子得之固躬,小人得之轻命。 

瞽者善听,聋者善视。\hl{绝利一源,用师十倍。三返昼夜,用师万倍}。
心生于物,死于物,机在于目。
天之无恩而大恩生。迅雷烈风,莫不蠢然。
至乐性余,至静性廉。天之至私,用之至公。禽之制在炁。
生者死之根,死者生之根。恩生于害,害生于恩。
愚人以天地文理圣,我以时物文理哲。人以愚虞圣,我以不愚虞圣;人以奇期圣,我以不奇期圣。
故曰:沉水入火,自取灭亡。

自然之道静,故天地万物生。
天地之道浸,故阴阳胜。
阴阳相推,而变化顺矣。是故圣人知自然之道不可违,因而制之。
至静之道。律历所不能契。
爰有奇器,是生万象,八卦甲子,神机鬼藏。
阴阳相胜之术,昭昭乎进于象矣。 

\subsection{释义}

以道心物三合之道来诠释,物者,意之所在。

观天之道,执天之行,尽矣:由心出发,体察天地之道,而后可以循道而行,此为道知,道尽为学处世之道。
观天之道是升维思考,执天之行是降维贯通,两相结合,就完备了。

天非茫茫之天,内蕴五行,能体察五行之气运,则可以昌盛。

心为能动的一方,以心受道体道,就可以立其环中,以应无穷,包括领导统御之术。

天人合发,万变定基:心与道合、与天合,这是做一切事的根基。

明了五行生克的结构与动态关系,进而内化于心,正心诚意,可称为圣人。

一事或成或败,皆有道理蕴含其中。天地-人-万物三合之道,尽心知性则知天矣,格物致知穷理,
尊德性而道问学,此三者相生相克,转圆而求其合。藏器于身待时而动,则万事如意,臻于中道。

绝利一源,一者何?道也,进乎技也。三返昼夜,循环至三,如昼夜交替,运行不废。
其功效甚大,有事半功倍之效果。一不能理解为具体的事,如此则器,君子不器,本立道生。
若心能体道,秉道御物,乘物游心,则三合之道可以大成。以道控势,顺势而为,与道浮沉。

执大象,天下往。往而不害,安平泰。大象无形,此无形之大象,即是道。
一生二,太极生两仪,有上下层次之别。两仪一阴一阳,有左右对称之美。

惟精惟一,志于道,若能志于道,而不废事,可入事事无碍法界。

大道至简,玄之又玄众妙之门。

真人者,同天而合道,执一而养产万类,怀天心,施德养,无为以包志虑思意而行威势者也。
鬼谷阴符七术之教。

阳明心学,心外无理心外无事,此心与道为一,即是道心、天心。

口目耳,此身之三要,心能制之。微信控,游戏控,则失心之所以为主,惑矣。

气韵生动

\section{老子}

\section{管子}

\subsection{内业}

凡物之精,此则为生。下生五谷,上为列星。流于天地之间,谓之鬼神;藏于胸中,谓之圣人。
是故民气,杲乎如登于天,杳乎如入于渊,淖乎如在于海,卒乎如在于己。
是故此气也,不可止以力,而可安以德;不可呼以声,而可迎以音。敬守勿失,是谓成德,德成而智出,万物果得。

凡心之刑,自充自盈,自生自成。其所以失之,必以忧乐喜怒欲利。能去忧乐喜怒欲利,心乃反济。
彼心之情,利安以宁,勿烦勿乱,和乃自成。折折乎如在于侧,忽忽乎如将不得,渺渺乎如穷无极。
此稽不远,日用其德。

夫道者,所以充形也,而人不能固。其往不复,其来不舍。谋乎莫闻其音,卒乎乃在于心;冥冥乎不见其形,淫淫乎与我俱生。不见其形;不闻其声,而序其成,谓之道。
凡道无所,善心安爱。心静气理,道乃可止。彼道不远,民得以产;彼道不离,民因以知。是故卒乎其如可与索,眇眇乎其如穷无所。彼道之情,恶音与声,修心静音,道乃可得。
道也者,口之所不能言也,目之所不能视也,耳之所不能听也,所以修心而正形也;人之所失以死,所得以生也;事之所失以败,所得以成也。
凡道无根无茎,无叶无荣。万物以生,万物以成,命之曰道。

天主正,地主平,人主安静。春秋冬夏,天之时也;山陵川谷,地之枝也;喜怒取予,人之谋也。是故圣人与时变而不化,从物而不移。能正能静,然后能定。定心在中,耳目聪明,四肢坚固,可以为精舍。精也者,气之精者也。气,道乃生,生乃思,思乃知,知乃止矣。凡心之形,过知失生。

一物能化谓之神,一事能变谓之智。化不易气,变不易智,唯执一之君子能为此乎!执一不失,能君万物。君子使物,不为物使,得一之理。治心在于中,治言出于口,治事加于人,然则天下治矣。一言得而天下服,一言定而天下听,公之谓也。

形不正,德不来;中不静,心不治。正形摄德,天仁地义,则淫然而自至神明之极,照乎知万物。中义守不忒,不以物乱官,不以官乱心,是谓中得。

有神自在身,一往一来,奠之能思。失之必乱,得之必治。敬除其舍,精将自来。精想思之,宁念治之,严容畏敬,精将至定。得之而勿舍,耳目不淫。

心无他图,正心在中,万物得度。道满天下,普在民所,民不能知也。一言之解,上察于天,下极于地,蟠满九州。何谓解之?在于心安。我心治,官乃治,我心安,官乃安。治之者心也,安之者心也。

心以藏心,心之中又有心焉。彼心之心,音以先言。音然后形,形然后言,言然后使,使然后治。不治必乱,乱乃死。

精存自生,其外安荣,内藏以为泉原,浩然和平,以为气渊。渊之不涸,四体乃固;泉之不竭,九窍遂通。乃能穷天地,破四海。中无惑意,外无邪灾,心全于中,形全于外,不逢天灾,不遇人窖,谓之圣人。

人能正静,皮肤裕宽,耳目聪明,筋信而骨强。乃能戴大圜,而履大方,鉴于大清,视干大明。敬慎无忒,日新其德,遍知天下,穷于四极。敬发其充,是谓内得。然而不反,此生之忒。

凡道,必周必密,必宽必舒,必坚必固,守善勿舍,逐淫泽薄,既知其极,反于道德。全心在中,不可蔽匿,和于形容,见于肤色。善气迎人,亲于弟兄;恶气迎人,害于戎兵。不言之声,疾于雷鼓;心气之形,明于日月,察于父母。赏不足以劝善,刑不足以惩过,气意得而天下服,心意定而天下听。

搏气如神,万物备存。能搏乎?能一乎?能无卜筮而知吉凶乎?能止乎?能已乎?能勿求诸人而得之己乎?思之,思之,又重思之。思之而不通,鬼神将通之。非鬼神之力也,精气之极也。

四体既正,血气既静,一意搏心,耳目不淫,虽远若近。思索生知,慢易生忧,暴傲生怨,忧郁生疾,疾困乃死。思之而不舍,内困外薄,不早为图,生将巽舍。食莫若无饱,思莫若勿致,节适之齐,彼将自至。

凡人之生也,天出其精,地出其形,合此以为人。和乃生,不和不生。察和之道,其精不见,其征不丑。平正擅匈,论治在心。此以长寿。忿怒之失度,乃为之图。节其五欲,去其二凶,不喜不怒,平正擅匈。

凡人之生也,必以平正。所以失之,必以喜怒忧患。是故止怒莫若诗,去忧莫若乐,节乐莫若礼,守礼莫若敬,守敬莫若静。内静外敬,能反其性,性将大定。

凡食之道:大充,伤而形不臧;大摄,骨枯而血沍。充摄之间,此谓和成,精之所舍,而知之所生,饥饱之失度,乃为之图。饱则疾动,饥则广思,老则长虑。饱不疾动,气不通于四末;饥不广思,饱而不废;老不长虑,困乃速竭。大心而敢,宽气而广,其形安而不移,能守一而弃万苛,见利不诱,见害不俱,宽舒而仁,独乐其身,是谓云气,意行似天。

凡人之生也,必以其欢。忧则失纪,怒则失端。忧悲喜怒,道乃无处。爱欲静之,遇乱正之,勿引勿推,福将自归。彼道自来,可藉与谋,静则得之,躁则失之。灵气在心,一来一逝,其细无内,其大无外。所以失之,以躁为害。心能执静,道将自定。得道之人,理丞而屯泄,匈中无败。节欲之道,万物不害。

\subsection{心术上}

心之在体,君之位也;九窍之有职,官之分也。心处其道。九窍循理;嗜欲充益,目不见色,耳不闻声。故曰上离其道,下失其事。毋代马走,使尽其力;毋代鸟飞,使弊其羽翼;毋先物动,以观其则。动则失位,静乃自得。

道,不远而难极也,与人并处而难得也。虚其欲,神将入舍;扫除不洁,神乃留处。人皆欲智而莫索其所以智乎。智乎,智乎,投之海外无自夺,求之者不得处之者。夫正人无求之也,故能虚无。

虚无无形谓之道,化育万物谓之德,君臣父子人间之事谓之义,登降揖让、贵贱有等、亲疏之体谓之礼,简物、小未一道。杀僇禁诛谓之法。

大道可安而不可说。直人之言不义不颇,不出于口,不见于色,四海之人,又孰知其则?

天曰虚,地曰静,乃不伐。洁其宫,开其门,去私毋言,神明若存。纷乎其若乱,静之而自治。强不能遍立,智不能尽谋。物固有形,形固有名,名当,谓之圣人。故必知不言,无为之事,然后知道之纪。殊形异埶,不与万物异理,故可以为天下始。

人之可杀,以其恶死也;其可不利,以其好利也。是以君子不休乎好,不迫乎恶,恬愉无为,去智与故。其应也,非所设也;其动也,非所取也。过在自用,罪在变化。是故有道之君,其处也若无知,其应物也若偶之。静因之道也。

“心之在体,君之位也;九窍之有职,官之分也。”耳目者。视听之官也,心而无与于视听之事,则官得守其分矣。夫心有欲者,物过而目不见,声至而耳不闻也。故曰:“上离其道,下失其事。”故曰:心术者,无为而制窍者也。故曰“君”。“毋代马走”,“毋代鸟飞”,此言不夺能能,不与下诚也。“毋先物动”者,摇者不走,趮者不静,言动之不可以观也。“位”者”,谓其所立也。人主者立于阴,阴者静,故曰“动则失位”。阴则能制阳矣,静则能制动矣,攸曰,‘静乃自得。”

道在天地之间也,其大无外,其小无内,故曰“不远而难极也”。虚之与人也无间,唯圣人得虚道,故曰“并处而难得”。世人之所职者精也。去欲则宣,宣则静矣,静则精。精则独立矣,独则明,明则神矣。神者至贵也,故馆不辟除,则贵人不舍焉。故曰“不洁则神不处”。“人皆欲知而莫索之”,其所(以)知,彼也;其所以知,此也。不修之此,焉能知彼?修之此,莫能虚矣。虚者,无藏也。故曰去知则奚率求矣,无藏则奚设矣。无求无设则无虑,无虑则反复虚矣。

天之道,虚其无形。虚则不屈,无形则无所位迕,无所位迕,故遍流万物而不变,德者,道之舍,物得以生生,知得以职道之精。故德者得也。得也者,其谓所得以然也。以无为之谓道,舍之之谓德。故道之与德无间,故言之者不别也。间之理者,谓其所以舍也。义者,谓各处其宜也。礼者,因人之情,缘义之理,而为之节文者也,故礼者谓有理也。理也者,明分以谕义之意也。故礼出乎义,义出乎理,理因乎宜者也。法者所以同出,不得不然者也,故杀僇禁诛以一之也。故事督乎法,法出乎权,权出于道。

道也者、动不见其形,施不见其德,万物皆以得,然莫知其极。故曰“可以安而不可说”也。莫人,言至也。不宜,言应也。应也者,非吾所设,故能无宜也。不顾,言因也。因也者,非吾所顾,故无顾也。“不出于口,不见于色”,言无形也;“四海之人,孰知其则”,言深囿也。

天之道虚,地之道静。虚则不屈,静则不变,不变则无过,故曰“不伐”。“洁其宫,阙其门”:宫者,谓心也。心也者,智之舍也,故曰“宫”。洁之者,去好过也。门者,谓耳目也。耳目者,所以闻见也。“物固有形,形固有各”,此言不得过实、实不得延名。姑形以形,以形务名,督言正名,故曰“圣人”。“不言之言”,应也。应也者,以其为之人者也。执其名,务其应,所以成,之应之道也。“无为之道,因也。因也者,无益无损也。以其形因为之名,此因之术也。名者,圣人之所以纪万物也。人者立于强,务于善,未于能,动于故者也。圣人无之,无之则与物异矣。异则虚,虚者万物之始也,故曰“可以为天下始”。

人迫于恶,则失其所好;怵于好,则忘其所恶。非道也。故曰:“不怵乎好,不迫乎恶。”恶不失其理,欲不过其情,故曰:“君子”。“恬愉无为,去智与故”,言虚素也。“其应非所设也,其动非所取也”,此言因也。因也者,舍己而以物为法者也。感而后应,非所设也;缘理而动,非所取也,“过在自用,罪在变化”:自用则不虚,不虚则仵于物矣;变化则为生,为生则乱矣。故道贵因。因者,因其能者,言所用也。“君子之处也若无知”,言至虚也;“其应物也若偶之”,言时适也、若影之象形,响之应声也。故物至则应,过则舍矣。舍矣者,言复所于虚也。

\subsection{心术下}

形不正者,德不来;中不精者,心不冶。正形饰德,万物毕得,翼然自来,神莫知其极,昭知天下,通于四极。是故曰:无以物乱官,毋以官乱心,此之谓内德。是故意气定,然后反正。气者身之充也,行者正之义也。充不美则心不得,行不正则民不服。是故圣人若天然,无私覆也;若地然,无私载也。私者,乱天下者也。

凡物载名而来,圣人因而财之,而天下治。实不伤,不乱于天下,而天下治。专于意,一于心,耳目端,知远之证。能专乎?能一乎?能毋卜筮而知凶吉乎?能止乎?能已乎?能毋问于人而自得之于己乎?故曰,思之。思之不得,鬼神教之。非鬼神之力也。其精气之极也。

一气能变曰精、一事能变曰智。慕选者,所以等事也;极变者,所以应物也。慕选而不乱,极变而不烦,执一之君子执一而不失,能君万物,日月之与同光,天地之与同理。

圣人裁物,不为物使。心安,是国安也;心治,是国治也。治也者心也,安也者心也。治心在于中,治言出于口,治事加于民,故功作而民从,则百姓治矣。所以操者非刑也,所以危者非怒也。民人操,百姓治,道其本至也,至不至无,非所人而乱。

凡在有司执制者之利,非道也。圣人之道,若存若亡,援而用之,殁世不亡。与时变而不化,应物而不移,日用之而不化。

人能正静者,筋肕而骨强;能戴大圆者,体乎大方;镜大清者,视乎大明。正静不失,日新其德,昭知天下,通于四极。金心在中不可匿,外见于形容,可知于颜色。善气迎人,亲如弟兄;恶气迎人,害于戈兵。不言之言,闻于雷鼓。全心之形,明于日月,察于父母。昔者明王之爱天下,故天下可附;暴王之恶天下,故天下可离。故货之不足以为爱,刑之不足以为恶。货者爱之末也,刑者恶之末也。

凡民之生也,必以正平;所以失之者,必以喜乐哀怒,节怒莫若乐,节乐莫若礼,守礼莫若敬。外敬而内静者,必反其性。

岂无利事哉?我无利心。岂无安处哉?我无安心。心之中又有心。意以先言,意然后形,形然后思,思然后知。凡心之形,过知失生。

是故内聚以为原。泉之不竭,表里遂通;泉之不涸,四支坚固。能令用之,被及四固。

是故圣人一言解之,上察于天,下察于地。

\subsection{白心}

凡物之精,此则为生。下生五谷,上为列星。流于天地之间,谓之鬼神;藏于胸中,谓之圣人。是故民气,杲乎如登于天,杳乎如入于渊,淖乎如在于海,卒乎如在于己。是故此气也,不可止以力,而可安以德;不可呼以声,而可迎以音。敬守勿失,是谓成德,德成而智出,万物果得。

凡心之刑,自充自盈,自生自成。其所以失之,必以忧乐喜怒欲利。能去忧乐喜怒欲利,心乃反济。彼心之情,利安以宁,勿烦勿乱,和乃自成。折折乎如在于侧,忽忽乎如将不得,渺渺乎如穷无极。此稽不远,日用其德。

夫道者,所以充形也,而人不能固。其往不复,其来不舍。谋乎莫闻其音,卒乎乃在于心;冥冥乎不见其形,淫淫乎与我俱生。不见其形;不闻其声,而序其成,谓之道。凡道无所,善心安爱。心静气理,道乃可止。彼道不远,民得以产;彼道不离,民因以知。是故卒乎其如可与索,眇眇乎其如穷无所。彼道之情,恶音与声,修心静音,道乃可得。道也者,口之所不能言也,目之所不能视也,耳之所不能听也,所以修心而正形也;人之所失以死,所得以生也;事之所失以败,所得以成也。凡道无根无茎,无叶无荣。万物以生,万物以成,命之曰道。

天主正,地主平,人主安静。春秋冬夏,天之时也;山陵川谷,地之枝也;喜怒取予,人之谋也。是故圣人与时变而不化,从物而不移。能正能静,然后能定。定心在中,耳目聪明,四肢坚固,可以为精舍。精也者,气之精者也。气,道乃生,生乃思,思乃知,知乃止矣。凡心之形,过知失生。

一物能化谓之神,一事能变谓之智。化不易气,变不易智,唯执一之君子能为此乎!执一不失,能君万物。君子使物,不为物使,得一之理。治心在于中,治言出于口,治事加于人,然则天下治矣。一言得而天下服,一言定而天下听,公之谓也。

形不正,德不来;中不静,心不治。正形摄德,天仁地义,则淫然而自至神明之极,照乎知万物。中义守不忒,不以物乱官,不以官乱心,是谓中得。

有神自在身,一往一来,奠之能思。失之必乱,得之必治。敬除其舍,精将自来。精想思之,宁念治之,严容畏敬,精将至定。得之而勿舍,耳目不淫。

心无他图,正心在中,万物得度。道满天下,普在民所,民不能知也。一言之解,上察于天,下极于地,蟠满九州。何谓解之?在于心安。我心治,官乃治,我心安,官乃安。治之者心也,安之者心也。

心以藏心,心之中又有心焉。彼心之心,音以先言。音然后形,形然后言,言然后使,使然后治。不治必乱,乱乃死。

精存自生,其外安荣,内藏以为泉原,浩然和平,以为气渊。渊之不涸,四体乃固;泉之不竭,九窍遂通。乃能穷天地,破四海。中无惑意,外无邪灾,心全于中,形全于外,不逢天灾,不遇人窖,谓之圣人。

人能正静,皮肤裕宽,耳目聪明,筋信而骨强。乃能戴大圜,而履大方,鉴于大清,视干大明。敬慎无忒,日新其德,遍知天下,穷于四极。敬发其充,是谓内得。然而不反,此生之忒。

凡道,必周必密,必宽必舒,必坚必固,守善勿舍,逐淫泽薄,既知其极,反于道德。全心在中,不可蔽匿,和于形容,见于肤色。善气迎人,亲于弟兄;恶气迎人,害于戎兵。不言之声,疾于雷鼓;心气之形,明于日月,察于父母。赏不足以劝善,刑不足以惩过,气意得而天下服,心意定而天下听。

搏气如神,万物备存。能搏乎?能一乎?能无卜筮而知吉凶乎?能止乎?能已乎?能勿求诸人而得之己乎?思之,思之,又重思之。思之而不通,鬼神将通之。非鬼神之力也,精气之极也。

\section{庄子}

\subsection{天道}

天道运而无所积,故万物成;帝道运而无所积,故天下归;圣道运 而无所积,故海内服。明于天,通于圣,六通四辟于帝王之德者,其自为也,昧然无不静者矣!
圣人之静也,非曰静也善,故静也。万物 无足以挠心者,故静也。水静则明烛须眉,平中准,大匠取法焉。水静犹明,而况精神!
圣人之心静乎!天地之鉴也,万物之镜也。夫虚 静恬淡寂漠无为者,天地之平而道德之至也。故帝王圣人休焉。休则 虚,虚则实,实则伦矣。虚则静,静则动,动则得矣。
静则无为,无 为也,则任事者责矣。无为则俞俞。俞俞者,忧患不能处,年寿长矣 。
\hl{夫虚静恬淡寂漠无为者,万物之本也}。明此以南乡,尧之为君也; 明此以北面,舜之为臣也。以此处上,帝王天子之德也;以此处下, 玄圣素王之道也。
以此退居而闲游,江海山林之士服;以此进为而抚 世,则功大名显而天下一也。
静而圣,动而王,无为也而尊,朴素而 天下莫能与之争美。夫明白于天地之德者,此之谓大本大宗,与天和 者也。所以均调天下,与人和者也。与人和者,谓之人乐;与天和者 ,谓之天乐。
庄子曰:“吾师乎,吾师乎!赍万物而不为戾;泽及万 世而不为仁;长于上古而不为寿;覆载天地、刻雕众形而不为巧。” 此之谓天乐。
故曰:知天乐者,其生也天行,其死也物化。静而与阴 同德,动而与阳同波。故知天乐者,无天怨,无人非,无物累,无鬼 责。故曰:其动也天,其静也地,一心定而王天下;其鬼不祟,其魂 不疲,一心定而万物服。言以虚静推于天地,通于万物,此之谓天乐 。天乐者,圣人之心以畜天下也。

夫帝王之德,以天地为宗,以道德为主,以无为为常。无为也,则 用天下而有余;有为也,则为天下用而不足。故古之人贵夫无为也。 上无为也,下亦无为也,是下与上同德。下与上同德则不臣。下有为 也,上亦有为也,是上与下同道。上与下同道则不主。上必无为而用 下,下必有为为天下用。此不易之道也。

故古之王天下者,知虽落天地,不自虑也;辩虽雕万物,不自说也 ;能虽穷海内,不自为也。天不产而万物化,地不长而万物育,帝王 无为而天下功。故曰:莫神于天,莫富于地,莫大于帝王。故曰:帝 王之德配天地。此乘天地,驰万物,而用人群之道也。

本在于上,末在于下;要在于主,详在于臣。三军五兵之运,德之 末也;赏罚利害,五刑之辟,教之末也;礼法度数,刑名比详,治之 末也;钟鼓之音,羽旄之容,乐之末也;哭泣衰囗(左“纟”右“至 ”),隆杀之服,哀之末也。此五末者,须精神之运,心术之动,然 后从之者也。末学者,古人有之,而非所以先也。君先而臣从,父先 而子从,兄先而弟从,长先而少从,男先而女从,夫先而妇从。夫尊 卑先后,天地之行也,故圣人取象焉。天尊地卑,神明之位也;春夏 先,秋冬后,四时之序也;万物化作,萌区有状,盛衰之杀,变化之 流也。夫天地至神矣,而有尊卑先后之序,而况人道乎!宗庙尚亲, 朝廷尚尊,乡党尚齿,行事尚贤,大道之序也。语道而非其序者,非 其道也。语道而非其道者,安取道哉!

是故古之明大道者,先明天而道德次之,道德已明而仁义次之,仁 义已明而分守次之,分守已明而形名次之,形名已明而因任次之,因 任已明而原省次之,原省已明而是非次之,是非已明而赏罚次之,赏 罚已明而愚知处宜,贵贱履位,仁贤不肖袭情。必分其能,必由其名 。以此事上,以此畜下,以此治物,以此修身,知谋不用,必归其天 。此之谓大平,治之至也。故书曰:“有形有名。”形名者,古人有 之,而非所以先也。古之语大道者,五变而形名可举,九变而赏罚可 言也。骤而语形名,不知其本也;骤而语赏罚,不知其始也。倒道而 言,迕道而说者,人之所治也,安能治人!骤而语形名赏罚,此有知 治之具,非知治之道。可用于天下,不足以用天下。此之谓辩士,一 曲之人也。礼法数度,形名比详,古人有之。此下之所以事上,非上 之所以畜下也。

昔者舜问于尧曰:“天王之用心何如?”尧曰:“吾不敖无告,不 废穷民,苦死者,嘉孺子而哀妇人,此吾所以用心已。”舜曰:“美 则美矣,而未大也。”尧曰:“然则何如?”舜曰:“天德而出宁, 日月照而四时行,若昼夜之有经,云行而雨施矣!”尧曰:“胶胶扰 扰乎!子,天之合也;我,人之合也。”夫天地者,古之所大也,而 黄帝、尧、舜之所共美也。故古之王天下者,奚为哉?天地而已矣!

孔子西藏书于周室,子路谋曰:“由闻周之征藏史有老聃者,免而 归居,夫子欲藏书,则试往因焉。”孔子曰:“善。”往见老聃,而 老聃不许,于是囗(左“纟”右“番”音fan2)十二经以说。老 聃中其说,曰:“大谩,愿闻其要。”孔子曰:“要在仁义。”老聃 曰:“请问:仁义,人之性邪?”孔子曰:“然,君子不仁则不成, 不义则不生。仁义,真人之性也,又将奚为矣?”老聃曰:“请问: 何谓仁义?”孔子曰:“中心物恺,兼爱无私,此仁义之情也。”老 聃曰:“意,几乎后言!夫兼爱,不亦迂夫!无私焉,乃私也。夫子 若欲使天下无失其牧乎?则天地固有常矣,日月固有明矣,星辰固有 列矣,禽兽固有群矣,树木固有立矣。夫子亦放德而行,遁遁而趋, 已至矣!又何偈偈乎揭仁义,若击鼓而求亡子焉!意,夫子乱人之性 也。”

士成绮见老子而问曰:“吾闻夫子圣人也。吾固不辞远道而来愿见 ,百舍重趼而不敢息。今吾观子非圣人也,鼠壤有余蔬而弃妹,不仁 也!生熟不尽于前,而积敛无崖。”老子漠然不应。士成绮明日复见 ,曰:“昔者吾有剌于子,今吾心正囗(左“谷”右“阝”)矣,何 故也?”老子曰:“夫巧知神圣之人,吾自以为脱焉。昔者子呼我牛 也而谓之牛;呼我马也而谓之马。苟有其实,人与之名而弗受,再受 其殃。吾服也恒服,吾非以服有服。”士成绮雁行避影,履行遂进, 而问修身若何。老子曰:“而容崖然,而目冲然,而颡囗(左上“月 ”左下“廾”右“页”)然,而口阚然,而状义然。似系马而止也, 动而持,发也机,察而审,知巧而睹于泰,凡以为不信。边竟有人焉 ,其名为窃。”

老子曰:“夫道,于大不终,于小不遗,故万物备。广广乎其无不 容也,渊渊乎其不可测也。形德仁义,神之末也,非至人孰能定之! 夫至人有世,不亦大乎,而不足以为之累;天下奋柄而不与之偕;审 乎无假而不与利迁;极物之真,能守其本。故外天地,遗万物,而神 未尝有所困也。通乎道,合乎德,退仁义,宾礼乐,至人之心有所定 矣!”

世之所贵道者,书也。书不过语,语有贵也。语之所贵者,意也, 意有所随。意之所随者,不可以言传也,而世因贵言传书。世虽贵之 ,我犹不足贵也,为其贵非其贵也。故视而可见者,形与色也;听而 可闻者,名与声也。悲夫!世人以形色名声为足以得彼之情。夫形色 名声,果不足以得彼之情,则知者不言,言者不知,而世岂识之哉!

桓公读书于堂上,轮扁斫轮于堂下,释椎凿而上,问桓公曰:“敢 问:“公之所读者,何言邪?”公曰:“圣人之言也。”曰:“圣人 在乎?”公曰:“已死矣。”曰:“然则君之所读者,古人之糟粕已 夫!”桓公曰:“寡人读书,轮人安得议乎!有说则可,无说则死! ”轮扁曰:“臣也以臣之事观之。斫轮,徐则甘而不固,疾则苦而不 入,不徐不疾,得之于手而应于心,口不能言,有数存乎其间。臣不 能以喻臣之子,臣之子亦不能受之于臣,是以行年七十而老斫轮。古 之人与其不可传也死矣,然则君之所读者,古人之糟粕已夫!”

\section{荀子}

\subsection{解蔽}

凡人之患,蔽于一曲,而闇于大理。治则复经,两疑则惑矣。天下无二道,圣 人无两心。今诸侯异政,百家异说,则必或是或非,或治或乱。乱国之君,乱家之 人,此其诚心,莫不求正而以自为也。妒缪于道,而人诱其所迨也。私其所积,唯 恐闻其恶也。倚其所私,以观异术,唯恐闻其美也。是以与治虽走,而是己不辍也。 岂不蔽于一曲,而失正求也哉!心不使焉,则白黑在前而目不见,雷鼓在侧而耳不 闻,况于使者乎?德道之人,乱国之君非之上,乱家之人非之下,岂不哀哉!

故为蔽:欲为蔽,恶为蔽,始为蔽,终为蔽,远为蔽,近为蔽,博为蔽,浅为 蔽,古为蔽,今为蔽。凡万物异则莫不相为蔽,此心术之公患也。

昔人君之蔽者,夏桀殷纣是也。桀蔽于末喜斯观,而不知关龙逢,以惑其心, 而乱其行。桀蔽于妲己、飞廉,而不知微子启,以惑其心,而乱其行。故群臣去忠 而事私,百姓怨非而不用,贤良退处而隐逃,此其所以丧九牧之地,而虚宗庙之国 也。桀死于鬲山,纣县于赤旆。身不先知,人又莫之谏,此蔽塞之祸也。成汤监于 夏桀,故主其心而慎治之,是以能长用伊尹,而身不失道,此其所以代夏王而受九 有也。文王监于殷纣,故主其心而慎治之,是以能长用吕望,而身不失道,此其所 以代殷王而受九牧也。远方莫不致其珍;故目视备色,耳听备声,口食备味,形居 备宫,名受备号,生则天下歌,死则四海哭。夫是之谓至盛。诗曰:“凤凰秋秋, 其翼若干,其声若箫。有凤有凰,乐帝之心。”此不蔽之福也。

昔人臣之蔽者,唐鞅奚齐是也。唐鞅蔽于欲权而逐载子,奚齐蔽于欲国而罪申 生;唐鞅戮于宋,奚齐戮于晋。逐贤相而罪孝兄,身为刑戮,然而不知,此蔽塞之 祸也。故以贪鄙、背叛、争权而不危辱灭亡者,自古及今,未尝有之也。鲍叔、宁 戚、隰朋仁知且不蔽,故能持管仲,而名利福禄与管仲齐。召公、吕望仁知且不蔽, 故能持周公而名利福禄与周公齐。传曰:“知贤之为明,辅贤之谓能,勉之强之, 其福必长。”此之谓也。此不蔽之福也。

昔宾孟之蔽者,乱家是也。墨子蔽于用而不知文。宋子蔽于欲而不知得。慎子 蔽于法而不知贤。申子蔽于埶而不知知。惠子蔽于辞而不知实。庄子蔽于天而不知 人。故由用谓之道,尽利矣。由欲谓之道,尽嗛矣。由法谓之道,尽数矣。由埶谓 之道,尽便矣。由辞谓之道,尽论矣。由天谓之道,尽因矣。此数具者,皆道之一 隅也。夫道者体常而尽变,一隅不足以举之。曲知之人,观于道之一隅,而未之能 识也。故以为足而饰之,内以自乱,外以惑人,上以蔽下,下以蔽上,此蔽塞之祸 也。孔子仁知且不蔽,故学乱术足以为先王者也。一家得周道,举而用之,不蔽于 成积也。故德与周公齐,名与三王并,此不蔽之福也。

圣人知心术之患,见蔽塞之祸,故无欲、无恶、无始、无终、无近、无远、无 博、无浅、无古、无今,兼陈万物而中县衡焉。是故众异不得相蔽以乱其伦也。

何谓衡?曰:道。故心不可以不知道;心不知道,则不可道,而可非道。人孰 欲得恣,而守其所不可,以禁其所可?以其不可道之心取人,则必合于不道人,而 不合于道人。以其不可道之心与不道人论道人,乱之本也。夫何以知?曰:心知道, 然后可道;可道然后守道以禁非道。以其可道之心取人,则合于道人,而不合于不 道之人矣。以其可道之心与道人论非道,治之要也。何患不知?故治之要在于知道。

人何以知道?曰:心。心何以知?曰:虚壹而静。心未尝不臧也,然而有所谓 虚;心未尝不两也,然而有所谓壹;心未尝不动也,然而有所谓静。人生而有知, 知而有志;志也者,臧也;然而有所谓虚;不以所已臧害所将受谓之虚。心生而有 知,知而有异;异也者,同时兼知之;同时兼知之,两也;然而有所谓一;不以夫 一害此一谓之壹。心卧则梦,偷则自行,使之则谋;故心未尝不动也;然而有所谓 静;不以梦剧乱知谓之静。未得道而求道者,谓之虚壹而静。作之:则将须道者之 虚则人,将事道者之壹则尽,尽将思道者静则察。知道察,知道行,体道者也。虚 壹而静,谓之大清明。万物莫形而不见,莫见而不论,莫论而失位。坐于室而见四 海,处于今而论久远。疏观万物而知其情,参稽治乱而通其度,经纬天地而材官万 物,制割大理而宇宙里矣。恢恢广广,孰知其极?睪睪广广,孰知其德?涫涫纷纷, 孰知其形?明参日月,大满八极,夫是之谓大人。夫恶有蔽矣哉!

心者,形之君也,而神明之主也,出令而无所受令。自禁也,自使也,自夺也, 自取也,自行也,自止也。故口可劫而使墨云,形可劫而使诎申,心不可劫而使易 意,是之则受,非之则辞。故曰:心容--其择也无禁,必自现,其物也杂博,其 情之至也不贰。诗云:“采采卷耳,不盈倾筐。嗟我怀人,寘彼周行。”倾筐易满 也,卷耳易得也,然而不可以贰周行。故曰:心枝则无知,倾则不精,贰则疑惑。 以赞稽之,万物可兼知也。身尽其故则美。类不可两也,故知者择一而壹焉。

农精于田,而不可以为田师;贾精于市,而不可以为市师;工精于器,而不可 以为器师。有人也,不能此三技,而可使治三官。曰:精于道者也。精于物者也。 精于物者以物物,精于道者兼物物。故君子壹于道,而以赞稽物。壹于道则正,以 赞稽物则察;以正志行察论,则万物官矣。昔者舜之治天下也,不以事诏而万物成。 处一危之,其荣满侧;养一之微,荣矣而未知。故道经曰:“人心之危,道心之微。” 危微之几,惟明君子而后能知之。故人心譬如盘水,正错而勿动,则湛浊在下,而 清明在上,则足以见鬒眉而察理矣。微风过之,湛浊动乎下,清明乱于上,则不可 以得大形之正也。心亦如是矣。故导之以理,养之以清,物莫之倾,则足以定是非 决嫌疑矣。小物引之,则其正外易,其心内倾,则不足以决麤理矣。故好书者众矣, 而仓颉独传者,壹也;好稼者众矣,而后稷独传者,壹也。好乐者众矣,而夔独传 者,壹也;好义者众矣,而舜独传者,壹也。倕作弓,浮游作矢,而羿精于射;奚 仲作车,乘杜作乘马,而造父精于御:自古及今,未尝有两而能精者也。曾子曰: “是其庭可以搏鼠,恶能与我歌矣!”

空石之中有人焉,其名曰觙。其为人也,善射以好思。耳目之欲接,则败其思; 蚊虻之声闻,则挫其精。是以辟耳目之欲,而远蚊虻之声,闲居静思则通。思仁若 是,可谓微乎?孟子恶败而出妻,可谓能自强矣,未及思也;有子恶卧而焠掌,可 谓能自忍矣;未及好也。辟耳目之欲,远蚊虻之声,可谓危矣;未可谓微也。夫微 者,至人也。至人也,何忍!何强!何危!故浊明外景,清明内景,圣人纵其欲, 兼其情,而制焉者理矣;夫何强!何忍!何危!故仁者之行道也,无为也;圣人之 行道也,无强也。仁者之思也恭,圣者之思也乐。此治心之道也。

凡观物有疑,中心不定,则外物不清。吾虑不清,未可定然否也。冥冥而行者, 见寝石以为伏虎也,见植林以为后人也:冥冥蔽其明也。醉者越百步之沟,以为蹞 步之浍也;俯而出城门,以为小之闺也:酒乱其神也。厌目而视者,视一为两;掩 耳而听者,听漠漠而以为哅哅:埶乱其官也。故从山上望牛者若羊,而求羊者不下 牵也:远蔽其大也。从山下望木者,十仞之木若箸,而求箸者不上折也:高蔽其长 也。水动而景摇,人不以定美恶:水埶玄也。瞽者仰视而不见星,人不以定有无: 用精惑也。有人焉以此时定物,则世之愚者也。彼愚者之定物,以疑决疑,决必不 当。夫苟不当,安能无过乎?

夏首之南有人焉;曰涓蜀梁。其为人也,愚而善畏。明月而宵行,俯见其影, 以为伏鬼也;仰视其发,以为立魅也。背而走,比至其家,失气而死。岂不哀哉! 凡人之有鬼也,必以其感忽之间,疑玄之时定之。此人之所以无有而有无之时也, 而己以定事。故伤于湿而痹,痹而击鼓烹豚,则必有敝鼓丧豚之费矣,而未有俞疾 之福也。故虽不在夏首之南,则无以异矣。

凡以知,人之性也;可以知,物之理也。以可以知人之性,求可以知物之理, 而无所疑止之,则没世穷年不能无也。其所以贯理焉虽亿万,已不足浃万物之变, 与愚者若一。学、老身长子,而与愚者若一,犹不知错,夫是之谓妄人。故学也者, 固学止之也。恶乎止之?曰:止诸至足。曷谓至足?曰:圣王。圣也者,尽伦者也; 王也者,尽制者也;两尽者,足以为天下极矣。故学者以圣王为师,案以圣王之制 为法,法其法以求其统类,以务象效其人。向是而务,士也;类是而几,君子也; 知之,圣人也。故有知非以虑是,则谓之惧;有勇非以持是,则谓之贼;察孰非以 分是,则谓之篡;多能非以修荡是,则谓之知;辩利非以言是,则谓之詍。传曰: “天下有二:非察是,是察非。”谓合王制不合王制也。天下不以是为隆正也,然 而犹有能分是非、治曲直者邪?

若夫非分是非,非治曲直,非辨治乱,非治人道,虽能之无益于人,不能无损 于人;案直将治怪说,玩奇辞,以相挠滑也;案强钳而利口,厚颜而忍诟,无正而 恣孳,妄辨而几利;不好辞让,不敬礼节,而好相推挤:此乱世奸人之说也,则天 下之治说者,方多然矣。传曰:“析辞而为察,言物而为辨,君子贱之。博闻强志, 不合王制,君子贱之。”此之谓也。

为之无益于成也,求之无益于得也,忧戚之无益于几也,则广焉能弃之矣,不 以自妨也,不少顷干之胸中。不慕往,不闵来,无邑怜之心,当时则动,物至而应, 事起而辨,治乱可否,昭然明矣。

周而成,泄而败,明君无之有也。宣而成,隐而败,闇君无之有也。故人君者, 周则谗言至矣,直言反矣;小人迩而君子远矣!诗云:“墨以为明,狐狸而苍。” 此言上幽而下险也。君人者,宣则直言至矣,而谗言反矣;君子迩而小人远矣!诗 云:“明明在下,赫赫在上。”此言上明而下化也。



\part{日记}

\section{11}

\subsection{01}

编程注意事项
\begin{enumbox}
\item long time操作之前与之后,需要renew lease,防止lease失效,导致vctl切换
\item 读写、vfm\_get、vfm\_stat、stat等操作之前,必须chunk check。\hl{如果不检查,可能会返回ESTALE, see \_\_chunk\_read\_getnode}
\item chunk组织成tree,就意味着依赖性,需要从上到下依次保护,check
\item chunk上有lock保护
\item alloc/discard与io分离,故io加rdlock
\item \hl{困难之处在并发,tree结构上的并发,并且涉及到持久化、lease}
\end{enumbox}

诊断工具
\begin{enumbox}
\item 怎么快速得到各个controller的状态,包括pool、volume、snapshot等?
\item 怎么检查底层数据一致性?
\item 打印集群chunk tree,以及每个chunk的详细信息
\end{enumbox}

写到LUN结尾处,会自动扩容。ROW3这里有问题?

垃圾数据,干扰恢复,gc replica目前不能打开?

\subsection{02}

尽可能通过\hl{DBUG、GOTO}保留可跟踪线索,以方便线上调试。

mellanox交换机编程:SDK?


\section{12}

\subsection{03}

手腕扭伤快好了。真是旷日持久,一点点小问题,竟然如此,不可不慎重。病向浅中医,养生须趁早。

最重要的是下一阶段的工作问题。内壮则外强,不论何时何地,不要心怀幻想。\hl{放弃幻想,准备战斗}是战士的必然选择。
枕戈待旦是为将者的核心素养。

学习中医、太极,一是道,二是术,道术并重,前期以道为主,明理,后期深入术中,以道卫术。
有所偏重,交相互养。从更大的视野去看,道一而术多,都可以看作道的应用,博观约取。

对学习计算机技术也有好处,相互发明,在交叉地带有所领悟。为学日益为道日损。

徐特立老读书学习之法,定量、有恒,不可自乱方寸。

\subsection{04}

以心行气,以气运身。气是沟通身心的桥梁,区分先天气和后天气,呼吸是后天气,内气、真气从之的是先天气。
随外在呼吸,引领先天气之循环:沿着任脉而沉至丹田,沿着督脉而上达百会,形成闭环。任督通,则成小周天。
吸升呼降,升级的圆运动。医理拳理通。只有丹田聚气养气充沛,自然能打通任督二脉。

活动关节,找到途径各关节的主要穴位进行按摩,以起到拉伸的效果。循经找穴,进而通过经络上的关键穴位进行按摩。

\subsection{05}

洪范五福:《尚书》上所记载的五福\hl{一曰寿、二曰富、三曰康宁、四曰攸好德、五曰考终命},
东汉桓谭于《新论·辨惑第十三》把“考终命”更改,将五福改为“寿、富、贵、安乐、子孙众多”。
现代常把“五福临门”当作新春祝福使用。

六极:一曰凶、短、折, 二曰疾,三曰忧,四曰贫,五曰恶,六曰弱。

积贫积弱,富强的现代化中国。五福有可为有可为而不可期,如康宁、攸好德、富,一定程度上是可以做的。
这里有因果业力法则在起作用。

\begin{shadequote}
兴五福销六极

问:昔周著《九畴》之书,汉述《五行》之志,皆所以精究天人之际,穷探政化之源。然则五福之祥,何从而作;六极之沴,何故而生?
将欲辨行,可明本末。又今人财耗费,既贫且忧,时沴流行,或疾而夭。思欲销六极,致五福,殴一代于富寿,纳万人于康宁。何所施为,可致于此?

臣闻圣人兴五福销六极者,在乎\hl{立大中致大和}也。至哉中和之为德,不动而感,不劳而化,以之守则仁,以之用则神,卷之可以理一身,舒之可以济万物。
然则和者生于中也,中者生于不偏也,不邪也,不过也,不及也。若人君内非中勿思,外非中勿动,动静进退,皆得其中,故君得其中,则人得其所,人得其所,则和乐生焉。
是以君人之心和,则天地之气和,天地之气和,则万物之生和。于是乎三和之气,䜣合絪缊,积为寿,蓄为富,舒为康宁,敷为攸好德,益为考终命。
其羡者则融为甘露,凝为庆云,垂为德星,散为景风,流为醴泉。六气叶乎时,七曜顺乎轨,迨于巢穴羽毛之物,皆煦妪而自蕃,草木鳞介之祥,皆丛萃而继出。
夫然者,中和之气所致也。若人君内非中是思,外非中是动,动静进退,不得其中,故君不得其中,则人不得其所,人不得其所,则怨叹兴焉。
是以君人之心不和,则天地之气不和,天地之气不和,则万物之生不和。于是乎三不和之气,交错堙郁,伐为凶短折,攻为疾,聚为忧,损为贫,结为恶,耗为弱。
其羡者潜为伏阴,淫为愆阳,守为彗星,发为暴风,降为苦雨。四序失其节,三辰乱其行,迨乎襁褓卵胎之生,皆夭阏而不遂,木石华虫之怪,皆糅杂而毕呈。
夫然者,不中不和之气所致也。则天人交感之际,五福六极之来,岂不昭昭然哉。臣伏见比者兵赋未减,人鲜无忧,时沴所加,众或有疾。
德宗皇帝病人之病,忧人之忧,于是救之以广利之方,悦之以中和之乐,将使易忧为乐,变病为和,惠化之恩,莫斯甚也。

然臣窃闻善除害者察其本,善理疾者绝其源。伏惟陛下欲纾人之忧,先念忧之所自;欲救人之病,先思病之所由。知所自以绝之,则人忧自弭也;知所由以去之,则人病自瘳也。
然后申之以救疗之术,则人易康宁;鼓之以安乐之音,则人易和悦。斯必应疾而化速,利倍而功兼。六极待此而销,五福待此而作。
如是,可以陶三才缪滥之气,发为休祥;殴一代鄙夭之人,臻乎仁寿。中和之化,夫何远哉!
\end{shadequote}

东方治理学中阐述的黄帝四经基本方法论,作为基石。由此而展开为完整体系。理论核心在于体味阴阳中和的奥义。

撞丹田有上中下三个高度,每种高度20次为一组。完成5组为一轮。暂定如此,必须数量化。
\hl{任何活动都考虑量化},包括标准和验收。按由轻到重的原则,逐步加大运动量到一最大值,然后保持,允许有一定波动。
度数信,即是围绕中心轴的圆周运动,弦动方式。务必重视周期和节律。

张三丰打坐歌需要背诵。

意气运动要贯彻到一切运动中,这就意味着尽量慢,体会其中意气运行的节律。
数量并非第一重要,重要的是内在质量。

\subsection{06}

\subsection{07}

本周基本理解了RDMA,下周需要继续,列出以学习计划。

修炼也渐渐进入状态,坚持的并不算好。还需要加大力度和决心,通过一定手段调理身心是一辈子要进行的工作,不能轻视。
核心理念就是通经络,调气血,致中和。这也是世界运行的方式。

张首晟去世事件令人震惊,跨界资本失败造成的?也发人深思,\hl{专业是立身之本},下一步需要进一步深入去学习。
丹华资本是一步错棋?投资决策过于乐观?默默地做好自己的专业,再考虑趁势而起。不要给自己太大压力。\hl{从容中道乃最佳策略}。

\hl{太极混元桩站起},静中有动,气息流动,一刻不停。

每次集中攻克一个问题,目前静坐中,发现一些痛点。按瓶颈理论去处理。

\subsection{10}

站桩,太极混元桩、三体桩。为什么说万法出自三体桩。从无极桩、混元桩练习开始。
有足够的时间慢慢练习、体悟其原理。

动静交养,静坐、站桩、内家拳,都需要\hl{调心、调息、调身}。一呼一吸为一息,呼吸于生命而言,极为重要。
此中反思,对今后若干年具有重要价值。生活习惯、对生命的理解在这个过程中得以深化。
静坐、站桩、丹田功这些基本的内功心法,需要坚持下去,\hl{循序渐进、持之以恒}。信解行证,度数信。
基本方法无它,就是围绕一中心点,日积月累,以达豁然贯通之境。

\hl{脏腑、经络都可以看作中医的象数模型}。取象比类,就是一种模型思维。
至于模型是否反映真实情况,需要在实践中进行调节。

HY下一步会很艰难,是否陪着走下去是一个重要的抉择。是否有利于今后长期发展?是否有大的突破?
当断不断,必受其乱。选择的最重要标准就是下一步的发展。平台、领导、行业都极其重要。

简化简化再简化,放弃幻想、准备战斗,机会主义要不得。

\subsection{11}

平时行坐站卧都要注意姿势,功夫要下在平时,用正确的理念塑造良好的生活习惯。这个才是细水长流之道。
功夫怎么才能上身?不是机械地去练习,而是全身心地投入,用心领悟其精义。

体育运动与专业学习一样,遵循相同原则,如循序、有恒。刻意练习的理念,怎么用起来?
专项训练,全面提升。如何设计一套切实可行的健身法?应该包括:
\begin{enumbox}
\item 站桩
\item 静坐
\item 八段锦
\item 太极
\end{enumbox}

一些小敲门:
\begin{enumbox}
\item 腹式呼吸
\item 撞丹田
\item 跪式
\item 刷牙
\item 扣齿
\item 梳头
\item 提肛
\end{enumbox}

禅坐先不求速效,每天盘盘腿,重点关注一些痛点,就会有进步。日拱一卒的精神。
每一种功法都需要大量的时间积累,才能见效果。一个时期最好只有一种重点项目,待步上正轨时,再开始下一项。
采取一主多辅的架构。比如本阶段一撞丹田为主,以站桩、静坐为辅等等。
最后把所有的功法九九归一,抱元归一,回到旋极图的象征。

老子四十二章是最高哲学。\hl{道生一、一生二、二生三、三生万物。万物负阴而抱阳,中气以为和}。
这几句话包含了终极真理。要从中出发,展开为现实的力量。

近期的沉淀期,有其价值,沉下来,再出发。收敛到中间一点,收放自如。潜龙勿用,阳在下也。复其见天地之心。
最重要是保养此一团阳气,以直养而无害,则塞乎天地之间。越养越精神,以做持久战。

悟中道之理,成炼金术士。炼金术士者,能化腐朽为神奇。河洛五行,中土最为贵,乃调理转化之器。

放弃吧,现在变成很大的负能量,就这样不明不白的,有什么意思?现在的主要精力,应放在下一步的健康发展上。
真是内忧外患,有陷在泥潭里的感觉,世界那么大,为什么不走出去看看?

可以了解很多,最重要的确实基本功。

\subsection{14}

书本知识往往已经过时,紧跟会议、论文、各公司的实践活动。

\subsection{17}

\hl{大动不如小动,小动不如不动,不动之动乃生生不已之动}。此语精妙,极有启发。让人回归本源,无为而无不为,故大成拳是无为法。
剔除枝叶,一意本源。

在技术上事业上都有很好的启迪。处当今之际,HY可谓内忧外患,风雨飘摇。更有追问本质,以图活下去。
在技术上,一切围绕ABC,又分主次。主为linux、ceph等。通其一,万事毕。

大成拳心法功法俱为上佳,可以为一段时间的探索画上一个句号了。医武同源,同臻于道。道零德一而万物化生。
信解行证是一循环,转动不已。进而,立禅即意,此中禅意,渐渐融入生活中,行住坐卧,无往不在禅意中。
则与道合一。往日学习,偏于知解,缺乏体证,遂茫茫然不知所归,没有受用处。

大成拳与阳明心学都在唤醒自然本具之良知良能,栽培涵养,心中分别心生,则离道日远,佛理深邃,不可思议。
水性太极,妙悟圆觉,大成立禅,归心金刚。都需用心体证,勿自限于文字知解,何况不能达于文字般若,
妄生议论,枉费口舌,大可不必。

至此,各方面均有妥当安排,可以心无旁骛,尽心驰骋了。艺宗AB,拳归大成。

不管HY如何,这次一定要走。初步定为xsky,离家距离不算太远,发展势头蒸蒸日上。
最重要的是,与手头做的最为接近,可以全力以赴投入技术的进一步深造。

也需要安排几个候选,如京东云、联想、首都在线。至此,对下一步的职业规划基本定位清晰。

\subsection{18}

想不到华云窘迫至此,真是可叹!

贪多嚼不烂,全闪是唯一机会。

放弃幻想,准备战斗。积极准备找工作,静观其变吧。损失云云,不作为主要考虑项。
永远往前看,修之于身,其德乃真。

全用户态SDS,就照着这个目标努力。把握好AFA这个风口。
怎么建立相关知识体系呢?

运用心的综合能力,不要被细枝末节所遮蔽。

\subsection{19}

\subsection{20}

数据为王,存储是关键,必须坚持深入,建立完整的知识体系,同时有一门深入的定力和决心。
致广大而尽精微。

领域
\begin{enumbox}
\item 操作系统
\item 数据结构和算法
\item C/C++
\item RDMA
\item DPDK/SPDK
\item NVMe
\item FusionStor/Ceph
\item 其它分布式存储系统 (FS and DB etc)
\end{enumbox}

现在是一个重要的发展阶段。静下心来,好好转动PDCA循环。

linux io path非常重要,direct io绕过page cache,故需要对齐。aio依赖于direct io,也需要对齐。

块设备驱动的分层架构:\hl{vfs,page cache,general block,scsi},主机通过scsi协议连接到磁盘。
主机通过pcie的NVMe连接到NVMe SSD。

为什么1M块的读写如此慢?

latency和iops的关系:拿行驶在路上的车为例,没有饱和时,latency不受iops的影响;达到饱和后,如何维持稳定的iops就是一个大问题。
另外,故障对iops的影响也是设计上的大问题。故障后,进入降级运行状态,需要标记,以利恢复。\hl{论述并发和故障对iops的影响}。

\hl{理解异步操作},异步操作不同于非阻塞操作,依赖于非阻塞操作。aio、NVMe、RDMA、DPDK都采用了异步通信机制。
提交任务到队列,在polling线程里处理完成事件:
\begin{enumbox}
\item 提交
\item 完成(事件驱动、或PMD, callback)
\item 保持上下文
\end{enumbox}

提交和完成可以在同一线程里。

在异步操作之上可以构建同步操作。如rpc,提交请求后,进入yield状态;完成请求后,resume。
用状态机、或协程实现,原理都一样。

\hl{线程+队列}是中异常强大的模型。

\begin{enumbox}
\item aio syscall与eventfd结合,可以纳入epool/pool/select机制之中。
\item NUMA/cpuset和hugepage
\item 增删节点(异步化)
\end{enumbox}

其它如timerfd,signalfd都可以纳入epool机制,提供事件等待和通知机制。

启动若干aio polling线程,提交或检查完成状态。在aio api之上,用poll作为事件通知机制。
用到了两个eventfd,一个用于提交请求的通知,一个用于检查完成状态。

线程加一控制对象,对象包括上下文信息、队列等。

sqlite操作采用了类似机制。

\subsection{21}

linux内核分析和应用、大规模分布式存储系统是两本好书,按此知识体系按图索骥,\hl{致广大而尽精微}。
但这里的主题不够全面、深刻,更新更深的主题和知识点通过微信公众号、论文、blog等加以补充。

把握行动学习两原则和刻意练习、PDCA、黄金圈发展等基本理念,来指导自己的学习过程。

怎么高效地利用资源?

NUMA架构的cpu资源如何利用?线程core binding,私有hugepage分配器。
上层网络连接等任务hash到不同的core thread里。在一个core thread里处理提交、完成检查等基本操作。
同步操作,如io、网络、数据库访问,派生独立的线程池去处理。
如何做到内存本地化?如何做到thread均匀地映射到NUMA节点上。

如何减少上下文切换和数据copy的开销。

可抽象出独立的内存分配器,每个core有一个内存分配器的实例。

从线程的角度看,分为事件循环线程和工作线程,通过队列把各个线程连接起来,actor and channel模式、csp模型。

分析烧开水这个过程。打开开关后,我就可以走开去处理别的事情。由电水壶独立处理烧开水这个事,等它处理完成后,会鸣笛通知。
我得到通知,就可以使用其中的开水了。如果有多个电水壶,它们就可以并行工作。(\hl{用UML序列图来描述})

领导安排任务也遵循同样的逻辑。通过比类取象的方法去理解,并不难。

\hl{网中网}:主机与磁盘的连接,也可以看作是网络。如NVMe SSD,就是NVMe over PCIe,运行在PCIe上的NVMe。
普通的disk也是如此。PCIe采用串口技术,比并口更快是因为具有更高的频率。
从协议上看,NVMe对SCSI的优化体现在哪些地方? NVMe标准依然在快速演进。

与网络连接不同的是,主机与设备的连接是本地的,不是远程跨主机的。

从架构上,本地存储引擎这部分代码应该是简单的。本地磁盘管理:
\begin{enumbox}
\item pool与disk具有一对多的映射关系
\item 每个盘是个状态机
\item 磁盘多态:支持normal and NVMe disks
\item 分配磁盘位置,首先选择盘,再选择盘上的某个位置
\item 检查磁盘故障,并修复数据
\item 引用计数技术
\end{enumbox}

每个盘有属性和方法,包括分配、读写。每个disk有独立的分配工作线程。
一次分配请求,大部分情况下,是一块盘去满足。如果一块盘空间不足,就需要多盘。
要能够表达非连续的离散空间。请求端是同步操作,用一个lock来实现。由工作线程unlock。wuw序列。

normal disks采用aio方式,NVMe才有自己的方式(kernel bypass)。

chunk位置映射,有cache,读多写少。数量大,cache采用LRU置换算法。
\hl{要充分理解各种各样cache的重要性。作为一个重要的主题去掌握}。

参考lich架构,用c++写一个分布式全闪产品,是一个重要的想法。
这样,既可以从设计者的角度去理解lich,也可以引入一些优化项。
最重要是精简,不是一日之功,需要持久战,长期专注思考和coding。
就是她了。

\subsection{22}

lich采用了epoll+aio的io模型,epoll+同步io会引起线程主循环的效率。可以从两个层面看这个模型,
堵塞在epool上,就绪时进一步引入aio。事件和实际的io操作。aio是通过工作线程和多个队列共同完成的。

transfer,tcp和rdma,是可以共存的,都可以理解为条条大路通罗马,信息高速公路。在c/s之间建立了一条虚拟通路。
这条通路是在第三个对象的辅助之下建立的,监听socket。监听socket和连接socket可以统一地在一个事件处理框架内去处理。
针对每个socket,有独立的处理过程。

自转、公转两个环,polling处就是中心所在。消息有自转,也围绕中心公转,是双环,两个层面的事情。

提交和完成是两个过程,可以放在两个独立的线程里,也可以合二为一,用一个独立线程去做。队列需要各自独立。
队列+线程是强大的编程模型。如SEDA架构所描述的,可以有很多的变体。可以用来实现异步操作。

中断指的是cpu处理的中断,有外部事件发生时触发,cpu收到后,进入中断处理程序。与进程调度一起来考量。
中断时理解问题的一个关键概念。线程与协程的主要不同就在于此。线程可抢占,协程不可抢占。进程进入wait态后,需要中断去唤醒。

\hl{等待事件的线程,被中断唤醒后,重新加入ready队列,可以被调度执行}。

进一步,要理解mmap和sendfile的工作机制和带来的好处,主要从syscall讲起,如何减少上下文切换和内存copy的成本。

取象比类的方法理解计算的世界。

多存储网段,有多个port互联,若做bond,就是合成一条。
异步操作,烧开水。性能指标,汽车在路上行驶。确实有生动形象、易于理解的效果。中医的经络、藏象,方法上是相通的。
这是费曼提倡的方法。操作系统里的关键问题和技术,都是通过故事的形式引入的,如生产者消费者,哲学家就餐,背包等。

这比一味地在api之间晕头转向要好得多,并不是说api不需要掌握,而是明理后api的设计就是自热而然的过程。
这样设计出来的api自然贴切,持久稳定。

用解剖麻雀的方式解剖lich,认识其亮点和不足之处。在与ceph对比的过程中,进行创造性综合,集大成。

本周通过分析异步操作,终于有了贯通感,网络编程,包括TCP和RDMA、NVMe、iSER等,都采用了相似的设计模式。
深入理解进程、线程和协程,加上对队列的理解,就可以理解大部分问题了。

进而理解lua、erlang、go的协程模型。

道法术器,器是产品、系统,术是实现技术,法是架构和算法,道是原理。明理、善学,在术的层面要多进行刻意练习。

渐渐觉得,单纯看书不是学习的最佳方法。看书是第二位的事情,第一位的是在头脑里进行的综合分析和判断,就是用意不用力。
比如学习网络编程,大部分的书籍,都是在展现一个知识体系,至于最佳实践,往往显得过时而且没有针对性。

边想边做边读书,效果会更好。一直以来,有过于偏重读书的习惯,反而迷失了宗旨大义所在。
每本书,在知识的程序上,有重点和等级的不同。如linux环境编程更多讲一个一个api,以及api在linux kernel的实现方式。
这一层面是深入的知识,但从应用的角度去看,却远远不够直接。

道法更多是理念层面的,术器是实物层面,术是建筑系统这一大厦的一砖一瓦。道法则是指导建筑的原理和方法。

投资护城河,养生之理明,则进入专业和财富的领域,大道至简、一以贯之。

\subsection{23}

专业能力就是目前的一,有这个才有谈得上一生二后续过程,得一以为天下式。万不可再不重视了。全力以赴即可,至诚无息。
中庸是一部经典。

易经、道德经、中庸统摄黄帝内经、四经,一内一外,内圣外王之道备。

先从单机和分布式开始。

不应该开启超线程,会降低单个core的性能。NUMA节点内多核与内存是同一距离。在malloc时,可以注册RDMA。

完成一个任务可以有多种多样的方式。自己处理,或委托给别的线程处理。
自己处理比较简单,就是block,或非block的同步调用。

引入\hl{io multiplexing}的意义:监听多个描述符,只处理ready的描述符。
这就多了一层,成为两层,带来了非常强大的能力。

\subsection{24}

总是太乐观,HY处境一至于此。内忧外患,真是举步维艰。
抓紧呀,抓紧,真是危在旦夕,不能不做最坏打算。

下一步学习重点:\hl{操作系统、编程语言、算法和数据结构}。
这是最基础的,从这里引申出来,如存储、网络、文件系统、数据库等等。
CBA也离不开这个基础。勤练基本功,是做好任何事情的秘诀。

进程、地址空间、文件是OS的重要抽象,务必深入去理解。
磁盘等外设也是文件。io路径是个重点。ioctl能做什么?

线程是可调度的最小单元,同一进程的多个线程共享\hl{进程地址空间}。

外设中断cpu执行,进入中断处理程序,重新调度。
调度器具有最高权限,所以进程是可抢占的。

外设具有控制器,包括寄存器、数据缓冲器和调度算法。主机与外设如何通信?外设的数据缓冲器独立于进程地址空间。
direct io是越过page cache层,直接进入设备缓冲器,所以需要对齐到块边界。

删除卷或快照,分摊到各个节点上并行执行,如果引入回收站的功能则最好。
在存在离线节点的情况下,不应该违反\hl{safety和liveness}性质。

timer如何实现?

提交:直接提交,或提交到队列,然后通知负责提交队列的线程。所有的事件,都可以\hl{统一到epoll}框架内,
如enentfd,signalfd,timerfd,以及aio、socket fd等。

分解事件检查和实际处理过程。

线程
\begin{enumbox}
\item 一个core对应若干aio线程,
\item 一个disk对应一个allocator线程,
\item 一个sqlite db对应一个线程。
\end{enumbox}

都涉及到队列和多线程同步。队列的位置有所不同,有的归工作线程,有的归提交线程。
这存在非常大的灵活性。两种模式:\hl{hash到工作线程,工作线程polling task}。

协程是用户空间线程,依托于内核线程对象,不可被抢占。
因为OS感知不到,内核线程对象才是调度的最小单元。

协程维护1+N个上下文对象,N是协程的上下文。

\subsection{25}

VM failover,源vm自杀,目标vm才能启动,lease机制。

着重\hl{思考痛点和难点}。lich的难点在于故障下io无中断,数据副本一致性。

虚拟化和容器,都是需要探索的技术领域。要在\hl{操作系统、编程语言、算法和数据结构}的基础上,做出技术的Y型知识结构。

smartx的技术要求:分布式paxos、raft、etcd、zookeeper等。本地存储、存储协议(iSCSI、NVMf、SPDK等)。

基本功
\begin{enumbox}
\item 操作系统:io path, vfs、aio
\item 编程语言:c/c++,python、go、erlang等。
\item ext2/3/4, xfs, btrfs, f2fs, block layer
\item LevelDB/RocksDB
\item iSCSI, NFS, samba以及其它存储协议
\item HDFS/Ceph/Sheepdog/GlusterFS
\item SSD IO性能优化,有FTL开发经验
\end{enumbox}

每个core thread有线程本地变量,指向aio指针数组,共4个,前两个用于direct io,后两个用于元数据的sync io。
同一磁盘设备文件,以不同模式打开多次。这是fd与inode的多对一的关系。fd对应的对象一定存有状态信息。

redis的数据模式:每节点若干redis实例,每个实例保存:disk,metadata,raw记录。raw记录按vol组织成hash。
支持的操作:
\begin{enumbox}
\item 分配chunk
\item 回收chunk
\item 删除卷
\end{enumbox}

\hl{NUMA架构的拓扑结构},如何分配core和aio线程,如何分配内存?
接着看polling线程如何管理私有内存。

cat /proc/cpuinfo, cores与sibling相等,则没有开启hyper threading。若sibling是cores的二倍,则开启了超线程。
还有别的可能吗?

polling线程在NUMA节点之间均匀分布,aio线程优先选取超线程,其次选取同一物理cpu上的其它core。总的原则是局部性。

\hl{从core的初始化过程看起,接着重点关注polling线程的工作}。
外部线程如何与core线程通信,各类事件如何注册到core线程的?

统计内存使用量,统计各类资源的用法

怎么理解syscall,怎么理解内核空间和用户态?上下文切换和内存copy。如read过程,如何分析?如何优化?

\subsection{26}

从主体出发,从进程出发。进程调度、内存、文件和IO等。中断,异步机制。

每个cpu都有自己的线程队列,亲和力是进程的一个属性,指定可以运行在哪些cpu上。就是可以对应cpu的调度队列。

NUMA节点对应的内存,\hl{内存条应该在NUMA节点上对称插入}。否则,没有内存的NUMA节点需要访问远程内存,影响性能。

cpu拓扑是一棵树,叶子节点是逻辑cpu。polling线程binding到逻辑cpu上,
相应的aio线程选择与polling线程\hl{所在cpu最近的那个逻辑cpu}。
\hl{NUMA节点,多处理器,多core,超线程}会增加这棵树的层次。

NVMe设备自动发现会成为一个设备文件,通过kernel io路径进行存取。unbind设备驱动程序后,可以通过pci进行访问。
设备接入bus,通过port空间或内存映射访问设备。

在第一阶段,从各个NUMA节点上平均分配hugepage(mbind)。 hugepage与NUMA节点的这种关系得到保持。
mmap: posix\_memalign虚拟地址空间,然后mmap到对应的hugepage。然后初始化管理元数据,包括pages和buddy。
这个地方的代码可以优化,用统一方式管理global和private内存区。

为什么要用虚拟地址获取物理地址?方法是什么?物理地址在什么情况下被用到?

可以抽象出hugepage region这样的概念,用于管理多个hugepage,需要处理如下需求:
\begin{enumbox}
\item 指定NUMA节点
\item 用buddy算法管理多个hugepage的分配
\item 管理虚拟地址到物理地址的映射
\item 用统一方式管理global和private内存区
\item 统计内存区使用情况
\item 不需要借助hugetlbfs
\end{enumbox}

恢复策略:需要增加维护模式,在此期间,不进行数据修复。

学习路线图:
\begin{enumbox}
\item core
\item mm
\item rpc (tcp and rdma)
\item aio and nvme
\item iscsi/iser
\item NVMf
\item etcd
\item ***
\item snapshot
\item tier
\item bcache
\item EC
\end{enumbox}

还有什么可以改进的?

epoll机制的通用性
\begin{enumbox}
\item fd and socket
\item eventfd
\item signalfd
\item timerfd
\item pipe
\end{enumbox}

作为\hl{异步事件的监听机制},具有广泛的适用性。其它类似sem wait and post, \hl{wuw加锁机制}等。都可以用epoll来实现。
真正体现了event driven的特征。

\subsection{27}

六种同步方法:\hl{mutex, cond, sem, flock, spin, rwlock}。sem可以用于共享内存的同步,flock用于文件同步。
mutex, cond, spin, rwlock多多于多线程的同步。注意它们自己的不同。

mutex, spin, rwlock使用场景相似,性能有差别。涉及临界区和对共享变量的访问。
cond与mutex结合起来使用,是否一定要结合mutex?wait在一个外部条件上,这个外部条件/变量被多线程改变。
sem的使用场景又有所不同。如同步线程创建过程。串行化多个线程的创建过程等等。另外\hl{支持timedwait和try语义}。
至于文件锁,使用场景易于确认。

sem的wait和post是由不同线程执行的,一般的mutex、spin、rwlock则最好是由同一线程执行。否则会复杂化执行流。

简单的同步比较容易实施,比如多线程下保护一个全局变量。但要保护一个大型的数据结构,就比较困难,需要遵循一定的加锁、解锁协议/约定。
\hl{2PL,tree protocol}都是这样的协议。

场景一,一线程wait,另一线程需要唤醒它。用sem wait and post可以实现。

场景二,\hl{生产者在队列满时block,消费者在队列空时block}。队列满是一个条件,队列空是一个条件,所以需要两个sem。
同时对于队列长度的变化,需要加以并发同步。

\hl{所谓条件,只是可能性,而不是确定性的}。收到通知后,再次检查条件,进行相应处理。所以,这里通常是一个循环。

卷chunk树的cc比较复杂,每一个chunk都是一个cc的保护单元,又形成了层次结构。

\hrulefill

WAL和db共同构成完成的数据。这是大数据的alpha架构,什么是完整数据?如何构建完整数据。
为什么要先写入WAL?因为写入WAL相对性能高。WAL要支持UNDO和REDO操作。\hl{ARIES事务过程}。

fusionstor是没有commit log的,如何保障事务性?需要保障事务性吗?什么是顺序一致性。
严格\hl{区分提交的数据和未提交的数据},对两者的要求截然不同。
采用的clock机制,带来了系列问题,如周期性合并clock导致iops抖动,掉电情况下clock丢失,引起大量的恢复流量。unsafe\_clock机制的不自然。

\dotfill

一旦理解\hl{线程+队列}这种模型的强大威力,则scheduler、SEDA、alpha都比较容易理解了。
本质上是线程以及线程通信方式,有很多种组合情况。

线程模型有多种:M/1,1/1,M/N。按线程与KSE的比例。\hl{协程是M/1实现,pthread是1/1模型}。

scheduler resume当然可以由外部线程调用,只需要传入taskid参数、返回值和返回数据即可。
所谓resume就是重新加入runnable队列,进行重新调度。
被调度器选中后,继续执行schedule\_yield1的后半部分代码,返回值和数据也跟着传出。

\hl{scheduler和task的关系是1+N的关系。task可以用状态机来描述}。\hl{现代操作系统}一书对进程的描述,采用了这种方式。

\dotfill

用户和组是个正交的话题,留待最后再看。

进程和线程是活动的实体,即主体。内存分配、文件io等等都是在进程内执行的。
从一个到多个主体,引入了\hl{并发情况下通信和同步}的需求。

先理解同步机制,再理解IPC机制。最重要的IPC机制包括file和socket。
通信有\hl{共享内存和消息传递}两种基本形式。结合\hl{erlang和go语言的运行时模型}去理解这些概念。

服务端编程不能不深入理解epoll、aio等高级io特性。进一步理解其kernel实现原理。

\hrulefill

\hl{md5sum的计算},要合理安排\hl{存储和计算的位置}。计算放在控制器上去做是否更高效?
md5sum的计算是一个序列化的过程,只能一个完成后,进行下一个。如果vc切换,能否接着算?

按SEDA架构,组织成流水线计算,分两阶段:读取、计算,第一阶段可以并行,第二节点需要串行处理。
即fork and join并行模式。无法采用MapReduce计算框架。

\hl{卷在存储池间的离线复制和迁移},在线方式呢?会更复杂。
如果卷上有快照呢?从产品角度如何定义?

\hrulefill

\hl{研究nutanix的产品和技术}。

\subsection{28}

\subsection{29}

\subsection{30}


\chapter{2018}

\section{书单}

古籍
\begin{enumbox} 
\item 大学中庸
\item 黄帝阴符经
\item 鬼谷子
\item 大乘起信论
\item 曾国藩家书
\end{enumbox} 

学会学习
\begin{enumbox}
\item 东尼·博赞学习系列
\item NLP: 复制卓越的艺术
\item NLP执行师
\item 概念地图在教学和学习中的应用
\end{enumbox}

专业
\begin{enumbox}
\item Paxos/RAFT
\item 分布式系统: 概念与设计
\item Linux/UNIX系统编程手册
\item 性能之巅
\item 深度学习
\end{enumbox}

个人成长
\begin{enumbox}
\item 坎贝尔千面英雄
\item 卓越元素
\item 洛克菲勒家信
\item 摩根家信
\end{enumbox} 

管理
\begin{enumbox} 
\item 达里奥原则
\item 法尔科尼管理方法
\item 阿代尔
\item 马利克
\item 好战略坏战略
\item 管理十诫
\item 为将之道
\item 活着/洞见/觉悟
\item 赋能
\end{enumbox} 

投资
\begin{enumbox} 
\item 李笑来财务自由之路
\item 穷查理宝典
\item 查理芒格的原则
\item 投资最重要的事
\item 聪明的投资者
\end{enumbox} 

历史和传记
\begin{enumbox} 
\item 中共党史
\item 陈云
\item 富兰克林自传
\end{enumbox} 

哲学
\begin{enumbox} 
\item 柏拉图对话集
\item 亚里士多德形而上学
\item 尼各马可伦理学
\item 新工具
\item 笛卡尔谈谈方法
\item 笛卡尔第一哲学沉思集
\item 斯宾诺莎伦理学
\item 斯宾诺莎知性改进论
\end{enumbox} 

\chapter{2019}

\section{01}

\subsection{08}

用本文档记录日志,只维护一个即可。growth记录更高层次的指导原则,一般来说相对比较稳定。
如此就形成了一个相互促进的双环架构。

不要自我设限,一定如何如何?打基础,看机遇。

一些指导原则
\begin{enumbox}
\item 概念图、重要的是理解核心概念,费曼学习法,用自己的语言描述出来,取象比类,借助类比、联想等思维方式。
\item 在默认模式/执行模式,走神/专注等进行切换,\hl{文武之道、一张一弛},有助于激发出创造力。
\item 强调综合判断能力。
\item 慢练
\end{enumbox}

体系架构、操作系统和汇编语言是底层逻辑。数据结构、算法、编程语言结合起来进行学习。

坚持专业技术方面的发展,\hl{CBA都要有所涉猎}。现在正是最佳时机,不能再没有重点了。

重点应放在做过的项目和全闪上。\hl{多路径?多存储网段?SPDK、NVMe}等。
比如单机存储引擎,用sqlite3、redis、RocksDB,为什么?

从势、道、法、术、器等维度进行梳理。

\subsection{09}

clone后,没把源快照置为protected状态。

\hl{software pipelines和排队论是性能分析利器}。SEDA、actor和lambda是架构方法。
图论、网络流,老三论。按中医理论,人体由藏象经络组成,运行气血。与网络系统有很强的相似性。
图是最复杂的数据结构,降维去看其它数据结构会如何?petri net和有限状态机。
把以前生活中经历的点连成线,会更明晰。

专业的力量、能力变现、价值规律、写作是最好的自我投资、自我进化、放大核心优势都是很好的理念,关键在落实。

本质上是核模式的应用。先在一个小的专业领域建立根据地,再顺势裂变扩张。

眼观六路耳听八方,行业格局分析至关重要,技术加商业两条腿走路。

自己要变得强大,这是积极心态,接下来是如何才能变得强大,这是现实,
要做现实的造梦者,不做桃花源中人。

lich的io 路径需要控制器中转,是否影响性能?client直接读写数据会如何?ceph client与osd primary直接通信。

只有一个RDMA网卡,存在硬件上限。如何使用多个RDMA网卡呢?8k 100w iops。

接触斯多葛主义。内心的自由和平静。

Oracle asm?

\subsection{10}

向身边的人学习,虚心,不要老端着,自认为了不起,其实需要理解的东西很多。
差不多先生,含混不清是极大的弊端和缺陷,一定要清晰,清晰才有力量。
表达要简洁清澈清晰。空性不代表含混,而是包容。

写作是最好的自我投资,要深刻理解这句话,并每日打磨写作技能。
先记下流水账,固定时间进行归纳整理,提炼升华。

做事情太保守,\hl{视野不够高,格局不够大},现在呢?更多自以为是,
需要沉下心来,好好观察、琢磨。看清趋势。正心诚意,取势、明道、优术,然后是利器、举例。

集中一段时间,专攻技术。围绕关键问题,从广度和深度,\hl{致广大而尽精微}。
技术综述好做,选几本书多读几遍即可,难的还是深度。另外一条途径,就是多与人聊技术。

围绕全闪进行写作。全力以赴吧。

\hrulefill

\hl{重点整理在HY做过的工作}:
\begin{enumbox}
\item snapshot and LSV
\item Recovery
\item Balance
\item QoS
\item ***
\item ETCD
\item Network(TCP and RDMA)
\item ***
\item Scheduler
\item Memory Allocator
\item Disk Management (Local Storage Engine, +NVMe)
\item ***
\item iSCSI/iSER/NVMf
\end{enumbox}

\hl{恢复、平衡、删卷/snapshot、rollback、clone、flat}等操作,如何用统一的任务管理系统进行管理。
可以借鉴k8s等集群管理系统的经验。有了总控,可以更好地加入策略。

\hl{恢复和平衡}有些逻辑是通用的,如按pool处理,检测本地vctl变化,扫描本地vctl。
可以放到一个处理框架内进行处理。ceph如何进行数据平衡呢?

太大的调度粒度(vctl)不利于任务调度和负载均衡。qos如同节流阀。可以放到不同的位置上。

\hl{if-what}当前的恢复处理逻辑,如果发生\hl{网络分区故障},会如何?

为了支持多网卡,需要做什么?

为了支持多路径,需要做什么?

\hrulefill

研读google发表的系统相关论文,看真实需求以及技术演进方向。
性能、稳定可靠、容错高可用、可扩展是几大非功能属性,决定产品品质。

理解google的技术体系,不能老是泛泛而谈。borg衍生出k8s。
\hl{hdfs、bigtable、megastore、mapreduce}都已经实现了容器化和统一管理。

有bigtable的sstable演化出leveldb和rocksdb。redis可以与rocksdb进行组合,改变全内存数据库的局限性。

催生了hadoop生态,进一步衍生了spark等lambda架构的统一计算平台。

\hrulefill

光点图灵

RAID之上有LVM,都是单机环境下的产物。虚拟化程序越来越高。进而是盘阵,然后是分布式架构,
scale-out越来越强,灵活性越来越高,软件定义和超融合由此而来。

MongoDB直接做RAID0,然后利用其自身的集群能力,master/slave replication架构,
可以failover and failback。这相当于内置存储。当时对存储层面的理解相当有限。

对gridfs扩容是如何做的?通过多个目录做的,也不设计磁盘虚拟化方面,如LVM。
面对的就是一个扩展性问题。

\hl{迁入阿里云、采用七牛对象存储后},不自己维护服务器,避免了很大的麻烦。

\hrulefill

\hl{美地森的工作和成长经历}:

很长时间都理解不离aio、事件驱动的实质所在。
做技术确实死磕精神,重视贪多求大,结果反而欲速不达。

当时做edog又是如何做的呢?有什么可以反思的?
集群管理采用了Erlang,通过libvirt对接qemu,改了qemu的驱动,接入yfs作为底层存储。

对网络理解不够透彻,接触也少。前端时间获取交换机管理信息,技术收到都是类似的。
不过关于交换机,很多技术细节理解不到位。\hl{天马行空惯了,细节把握不到位}。
需要改进,做技术要认真再认真,程序化,每日记录、反思、总结、提炼。
既要有广度,深度更是必不可少。\hl{专精一件事,就可以立于不败之地}。
找到了自己的哲学、守中致和,核模式,就要学以致用,一以贯之,用来指导自己的言行。

接触了hadoop,为什么没有坚持呢?放着这么好的东西,不去跟进,
只是习惯于做些乱七八糟的东西,实在是有眼无珠呀。

虽然也有跟进,\hl{不坚定,不深入},当然更谈不上真的掌握。

\hrulefill

华胜天成:法国电信做手机的SIP协议,管理软件,一头雾水。

包括以前在\hl{天地伟业、鸿业科技},都没有深入业务,总是局限在一个小小的技术视角。
现在软硬件的变化,真是日新月异,不进则退。\hl{没有一个好的平台和机遇},就很难崛起。
如何把握好的平台和机遇,当然要靠本日一点一滴的积累,磨炼基本功,把握大趋势。

鸿业科技做的\hl{官网计算}回忆起来,倒是给了很好的启示,说明了管道理论的普遍性。

一是个人成长放在首位,接着就是睁大眼睛,静静地坚定地去找团队、平台和机遇。(\hl{人和事的胜任度})
两者匹配了,能力胜任度高,自然一切水到渠成。所以也不必着急焦虑,反反复复磨炼自己就够了,
金子总会发光的,好酒也怕巷子深。这些看似对立的格言,其实包含了深刻智慧。

这几年,技术发展真是太快了。风口一个也没把握住,原因何在呢?

\hl{接下来干什么呢}?

转移关注点,放低姿态,学习关键知识点,并记录在案。从整体系统中抽离,聚焦于更小的模块。

最近老王他们进行的性能优化工作是值得关注的架构调整。支持多路径,多网卡,甚至插到不同的交换机port上对性能都有影响(8x or 16x),以提升性能。
性能的理论上限如何预估?

更多地转变到SPDK(NVMe, NVMf)、RDMA上面来。全NVMe方案如何?怎么构建全局共享,client与数据不同于控制器直接存取?
可以参考的资源有什么?

纪翠叶问及fio的部分参数什么含义?这是器层面的知识点。

\subsection{11}

中庸是我的圣经。从心道物等诸多维度确定了必要的理念。

\hrulefill

不深入理解传统存储的高度和局限,就不能理解很多核心概念,比如多路径、共享、全局缓存等。
分布式架构在\hl{扩展性、灵活性}方面胜出。

通过真假latency,可以估计等效并发。方法如下,单并发测量固定时延。多并发测量iops,取倒数即为假时延。
固定时延和假时延的比值就是该配置下的等效并发度。

io path也有串行和并行两部分,从而引入加速比的概念,上面的等效并发度就是加速比。
可以评估整个体系结构的并行执行效果。

按pipelines理解体系结构,体系结构也是网中网的分形结构。
官网、电路等等可以作为分析模型,从流体动力学和物理学里吸收能量。
但也不必把问题复杂化。从最少的元素推导出更多的规则,解释更多的现象,是科学的研究方法。

从体系结构、操作系统的原理和知识,去深入存储子系统的相关理解。

\hrulefill

\hl{网络分区}下,恢复、平衡和一般的IO是如何进行的?

比如一个5节点集群,采用2副本策略。分区意味着什么呢?怎么理解多数派这一要求。
5节点2副本,也是只能容忍单个节点故障,\hl{故障容错度是有副本数决定的},与集群规模无关。
比如一个10节点的集群,2副本的情况下也只能容忍一个节点故障?

可以通过划分故障域改进这一点。副本数据分布到不同的故障域,容错按故障域来定义。

保护域和pool是等价的概念,保护域是一种物理隔离机制。把一个大集群分拆成小集群。所以基本的嵌套关系是\hl{集群-保护域-故障域-节点-磁盘}。

网络的VLAN是否也是一样的分区机制?

这一点与zk、etcd等协同共识系统存在很大的不同。

MySQL master/slave架构,能否自动切换?

\hrulefill

client直接与数据块通信,lich需要通过vctl中转,如何克服这一点?

多网卡支持,通过多网卡连接控制器,MP软件用来管理导出的LUN,确保只有一个。

SSD FTL中心任务是映射管理,与OS的页表、LSV的快照管理是一样的,关键是维护一个虚拟地址到物理地址的映射。
可以是一级,也可以是多级。\hl{伟大的计算原理}里面,讨论存储的时候提到四点:\hl{命名,映射,定位,授权}。
可以按此四个维度去思考林林总总的存储系统。

\hl{按主体-对象模型,授权有两种}:CL和ACL。CL以主体为中心,ACL以对象为中心(如文件系统的ACL)。

\hl{全力投入AFA},大量阅读,深度思考。知识就是力量,这点认识还是不够深刻。

地址空间管理,抽象成了一切皆文件。内存、磁盘、文件都是地址空间。文件系统的设计需要充分地认识到这一点。
SSD最核心的功能也是这一点。SSD FTL很类似于LSV,很多可类比的地方,如映射、GC。

数据结构主要也是映射,如hash、map、graph,graph是最通用的映射关系。
再如,函数也是映射。lambda架构也提及此,map-reduce更成了并行计算模式之一。

\hrulefill

多快好省,快和好是当前重点,多和省在此基础上,迭代优化。\hl{性能、负载、高可用}是几种集群形式,
bonding,多网卡支持。bonding具有什么特征?只是HA?而不能同时工作,聚合性能和均分负载?

\hl{全力投入AFA},大量阅读,深度思考。知识就是力量,这点认识还是不够深刻。

\subsection{12}

地址空间,如内存,文件都采用地址空间概念。多级页表可以支持稀疏地址空间,页表与inode采用的radix tree什么区别?
lich卷的chunk tree也采用了同样的结构。这样,需要维护虚拟地址到物理地址的映射。

对计算机体系结构、指令集、操作系统、汇编、编程语言有了初步的理解。对存储、网络、数据库也有一定的理解。
接下来就是围绕算法和数据结构为中心,\hl{以问题为主线}, 穿起来进行思考。伟大的计算原理给出了很好的总结。

分层去理解更透彻,如计算机组成结构化方法里,分为七层,网络也是分为若干层。

大的问题主要有调度、内存管理、地址空间管理、通信和同步、事务、分布式算法等。

以数据结构和算法为核心,去解决各类系统的核心问题。问题具有普遍性,算法也应理解为通用的算法。问题和算法是对应的。

提出问题,然后看应该如何去解题。解决问题是价值输出的重要方式。

分析io path就可以把一切知识点贯通了。一个写、一个读。集中在数据平面的相关问题。

\subsection{13}

在华云和光点的工作具有互补性,麻雀虽小五脏俱全,需要得到更好的总结、提炼和升华,一花一世界,由此去理解更广阔的技术世界。
譬如一个同心圆,虽有规模效应,规模引起一些质变,但稳固的核心,扎实的基本功,确是最为重要的,如此就拥有了强大的迁移能力。

不能不重视写作。

\hl{单一要素最大化,其它要素最小化}。算法可以作为下一阶段的单一要素,是不变的一。
算法的学习和理解脱离不了具体系统,一以贯之,就是以算法贯穿技术学习活动。

这里的算法是广义上的算法,不仅包括教科书里常常提到的算法,也包括协议、调度、事务等重要概念相关的算法。

\subsection{14}

\hl{化繁为简,打破学科边界},从实际问题及其方案开始理解。深入理解计算机系统即是采用这种叙述法。

计算机系统可分为cpu(\hl{进程、并发、中断、隔离})、memory、io子系统(storage/fs and network)等,按此结构循序展开。
io又分存储和网络等,组成tree架构。关键组件:\hl{调度器(thread、io),空间分配器(cache、memory、disk)}。

采用流一元论(flow dynamics)的描述框架,cpu居中调度,上下文切换和内存copy,可称为指令分配器。
\hl{flow由节点和边组成},节点是处理单元或节流阀,边是管道。处理方式可以有多种:转发、映射、压缩、消重等。

衡量流效率的指标是latency、iops、throughput等。流的平衡态由平衡方程描述,输入过多,会导致处理不过来,
发生拥塞、溢出等故障,故需要限流/节流。

memory copy消耗总线带宽,bus是如何利用的?

先理解一般case,加入cc和故障恢复,系统就变得复杂了。虚拟化、cc、持久性是操作系统研究的三大块。

cpu亲和性需要通盘考虑,不仅仅是内存,还有disk、network,读设计线程和buffer,是否与core线程处于一个NUMA节点?
还是跨NUMA节点进行内存copy? aio线程尽量与core线程距离近。\hl{如果只要一块网卡,则网卡属于哪个NUMA节点呢}?

\hrulefill

\hl{数据结构的内存layout},比如一个int应该怎么存?大端序、还是小端序?浮点数呢?结构体呢?
什么是补码表示?

任何一个变量、常量都占据一定的内存区域(addr,len)。代码也是,cpu从内存中取指、解码后执行。
一个函数是一个代码端,微观上是一个指令序列,符号表。一个对象呢?也是如此。内存布局,成员函数指向函数代码的指针。
模板在编译阶段实例化,用具体类型代替。最后都转化为变量和过程。\hl{class引入了对象作用域和类作用域}。构成了符号的层次结构。
可以假设最外层有一个global namespace,作为ns树的根。

把内存看做一维向量空间,就容易理解这个问题了。
最小地址单元是byte。指针运算,与数据类型有关,void *以byte为单元。

磁盘空间也是一维向量空间,分区、文件系统、文件就构成了层次结构,靠mapping建立虚拟地址到物理地址的映射,完成\hl{存储虚拟化}的功能。
RAID/LVM都是映射,需要\hl{计算或metadata}去实现这种映射。

snapshot、LSV也是meta+data,通过引入mapping实现特性。

\hrulefill

梳理lich中用到的数据结构和算法
\begin{enumbox}
\item array
\item list
\item stack
\item queue
\item heap
\item ***
\item bitmap
\item string
\item set
\item ***
\item skiplist
\item hash
\item tree
\item graph
\item ***
\item token bucket and leaky bucket
\end{enumbox}

\hl{key的结构和分布}非常重要。索引结构不仅要考虑内存结构,还有考虑磁盘结构。
一个线性空间中key的分布有规律可循,且有序。如果key是string,则分布是多样的。

为什么数据库的索引结构用hash和B+,而不是别的?

\hrulefill

bitmap也是用来管理1D线性空间,如磁盘空间管理,没有引入中间结构,key是offset。
bloom filter是一种特殊的存在性查询结构,不能用来检索,key是字符串。

页表、inode address space,都是trie(radix tree)结构。与邮政编码、url、dns域名类似。
name是有结构的,构成prefix tree。

trie与hash、RBtree、B+皆不同,它的key是有结构的,从而形成一定的层次。用来检索一个稀疏的线性空间。
lich volume即采用了该索引结构,定为三层。路由表采用radix tree设计。

array and list是基础存储结构,其它结构都是建立在此二者之上。

\hrulefill

以上是基础数据结构和算法,加上\hl{cc、事务、分布式、机器学习}等要求后,演化出更加多彩的算法世界。
分布式算法更多是交互协议设计。

\hl{墨菲法则},事情总比想象的要坏,真的不能再嘚瑟,认真考虑下一站是当务之急,损失什么的可以不用考虑太多。
重要的是,把握当下,面向未来。

可以参考的开源软件:
\begin{enumbox}
\item nginx
\item memcached
\item redis
\item mongodb
\item ***
\item LevelDB
\item RocksDB
\item Ceph
\item Sheepdog
\item ***
\item SPDK
\item DPDK
\item RDMA
\end{enumbox}

\hrulefill

由设备文件,可以\hl{分区、格式化、mount}。NVMe采用pci号实现kernel bypass。
读写设备文件和普通文件,采用同一套接口和语义。

io路径上\hl{page和block dismatch},page cache和buffer cache,buffer用指针引用page的区域。
故\hl{不经过page cache的io需要block对齐,包括direct io和aio}。

bcache? \hl{1+N虚拟出N个设备}。

bit + context,要深刻理解这一点。

\hrulefill

调度器:cpu调度器和io调度。多队列。

分配器:内存allocator和fs 空间管理,\hl{命名、映射}。

定位包括cache和replication。流动的数据。

\subsection{15}

\hl{行路难,行路难,多歧路,今安在}?好好沉淀一下吧,路真的不好走,一不留神就万劫不复。

忠于专业,这是安身立命之本。进一步的学习计划:
\begin{enumbox}
\item 数据结构和算法
\item software pipelines
\item SEDA and actor
\item lambda
\item queueing theory
\item network flow
\item *** 系统
\item \hl{操作系统(包括体系结构、OS和编程语言)}
\item 文件系统
\item 网络
\item 分布式系统
\item 数据库系统
\item *** 应用
\end{enumbox}

从\hl{scheduler、allocator和cache}几个维度去研究。

先研究scheduler,有\hl{线程调度、协程调度、io调度、事务调度器},甚至k8s一样的全局资源调度。
比如erlang、go的调度也很有特点。

\hl{任务状态,任务队列}。引入任务优先级,就是多队列。进程调度的任务就是选出下一个要运行的进程、线程。

任务上下文切换,scheduler和tasks之间是1+N的关系,交错执行。

调度分抢占式和非抢占式。线程相对于协程,多了中断方式的切换,包括硬中断和软中断。

\hl{调度器就是software pipelines里的分配器},影响系统整体性能。

scheduler采用什么数据结构来组织任务队列呢?

scheduler采用什么QoS策略?

多调度器下的任务漂移/窃取

调度器自身是不能有长时的block任务的,对这类任务采用异步方式处理。
线程调度是抢占式,\hl{协程任务是非抢占式,要杜绝出现block操作},由CQ来完成。

oracle asm架构

三层:\hl{并行、polling、事件处理},并行线程的cache、memory、pci bus要local。
尽量避免并发访问共享资源。share nothing是最好的,但有时共享、通信难以避免。

polling和事件处理研究单个scheduler的情况,并行研究多个scheduler的情况。

\hrulefill

所谓异步就是不等待任务处理结果,当任务完成时,会触发事件,进入cq。

\hl{以FusionStor为例},io有两种情况,\hl{普通disk方式和NVMe方式}。
普通disk采用aio,NVMe采用其自身的异步机制。

网络也有两种方式:TCP and RDMA。采用epoll多路机制,polling请求或完成事件。

\hl{RDMA和NVMe都是基于事件和多队列}的异步通信方式。

NVMe也很简单,用pci打开后,可以异步读写,\hl{poll时执行callback}。
Lich里主要处理了core线程和buffer的适配性。

NVMe没有使用aio线程,自身的异步机制,与coroutine结合很好。
aio还需要结合thread+epoll机制。

allocate时考虑数据块与disk的对应关系。
NVMe disk需要考虑NUMA亲和性?
普通盘连接南桥,不需要考虑?

sqlite3、redis相关的block操作,用工作线程pool。\hl{不同于aio,没有CQ这种机制}。
可以聚合起来进行批处理。因为缺少异步原生支持,可以称之为模拟aio。

网络io怎么做的?\hl{TCP和RDMA不同},RDMA是异步机制,TCP用得是recv和send。

统一的polling框架,包括aio and NVMe,TCP and RDMA,scheduler等。
通过eventfd block到一个block点上。

\hrulefill

状态机相对协程,是否更高效?对象的队列是少不了的,也需要schedule。
每个对象的状态与task不同,需要按每种类型进行分解。而task则与进程一样,有三种基本状态。

\hrulefill

\hl{总结性能优化方法}:NUMA和CPU亲和性,Hugepage,并行、流水线、聚合等方法。
locality考虑。

virtio, IOMMU,安装在client端,感知虚拟化。

\hl{硬件虚拟化}是指处理器支持虚拟化相关指令集,
\hl{VAAI}扩展了SCSI语义,加速数据copy速度,与sendfile类似。

\subsection{16}

每个逻辑Core有一个调度队列(RBtree),所谓NUMA和CPU亲和性,就是通过mask,指定线程可以进入的队列。
调度器是一个函数,从队列或多队列里取出一个或多个要处理的元素。调度器有控制线程,控制线程可以做其它的事情。
从这个角度说,把调度器理解为一个函数为合适。线程调度、io调度和事务调度,都是这种方式。

调度包括队列、调度entity和class。用class定制调度策略。

LICH中的coroutine调度,采用了简单的FIFO方式,有优化空间。

\hl{从底层物理资源去考虑},kernel起到资源管理的作用,即资源虚拟化。cpu、memory、devices。理解devices更复杂,
接入bus,被scan到,加载driver,然后就可以通过一定的方式进行通信,有外部事件时也可以中断方式通知cpu。
中断通知,必然是先准备好上下文。这个上下文的切换是个重点。

block用物理disk作为backing device,就是\hl{利用disk的物理特性},而不是虚拟分区。

\hl{时间片,优先级,可抢占}是内核线程调度的几大特征。

协程与状态机方式,都需要分两层看。调度器和状态推进。状态机选出下一个要处理的任务,执行任务状态机一步或多步后退出。

调度器相当于软件pipelines中的分配器。按epoll,需要把fd加入epoll的兴趣列表,
同时维护分离表/routor:fd到node(event handler以及context)的映射关系。

RDMA用了两层分离机制:epoll用于连接管理,自身的polling机制。

\subsection{17}

\hl{在lich里引入pg如何}?object-pg-osd的关系由两级映射完成,一是降低了metadata量,二是恢复平衡更容易做。
这融合了ceph和lich两者的优点。

如osd故障,pg处在降级状态。与lich引入vfm要解决的问题类似。\hl{引入最小副本数},会使得方案更优雅。
从osd故障到数据恢复,可以定制恢复策略,如时长等。

lich的chunk和replication,类似于RAID的条带化和mirroring。

传统盘阵的控制器结构,有双控、多控。双控关系有A/A,A/P。横向扩展。LUN的归属。
\hl{理解MPIO在多控架构下的工作原理}。

单处理器的false sharing。\hl{同一cache line}上有多个数据项,被不同线程更新。

多处理器的缓存一致性(同一存储位置),存储一致性(不同存储位置)。NAS的client 缓存一致性

ABA问题

本地文件系统
\begin{enumbox}
\item \hl{RAID/LVM}
\item ext2/3/4
\item XFS
\item ZFS
\item btrfs
\item ReiserFS
\item ***
\item Fuse
\end{enumbox}

分布式文件系统
\begin{enumbox}
\item GFS/HDFS
\item GlusterFS
\item Lustre
\end{enumbox}

与ceph对比
\begin{enumbox}
\item 数据分布(元数据管理)
\item 副本分布(节点为最小故障域粒度)
\item TP
\item 卷控(无日志,顺序一致性语义)
\item 恢复、平衡、流控
\item ***
\item \hl{RDMA/DPDK/SPDK}
\item \hl{iSER/NVMf}
\item Cache
\item Tier
\item ***
\item \hl{Snapshot}
\item Remote Replication
\item ***
\item EC
\item Dedup/Compress
\end{enumbox}

Lich优化项
\begin{enumbox}
\item pool下flat namespace,不需要支持多级目录
\item 引入类似pg的结构,二级映射
\item Node内管理磁盘,而没有把disk暴露
\item ***
\item 卷控平衡
\item 查找卷控位置采用了multicast机制,为何不去admin上根据lease情况找?
\item 数据访问需要vctl中转,client不能存取chunk (与array的双控架构做对比)
\item 卷控拆分成子卷,支持大容量卷
\item 恢复可以由卷控自己执行(智能卷控)
\item 停止恢复
\end{enumbox}

\hrulefill

Amazon EBS挂载到一个EC2实例上,EBS可以打快照,快照保存到S3上。
第一个快照是全量的,后续快照是增量的。

对一个pool,容错级别与副本数有关。故障域要大于等于副本数。\hl{EC能改进这个问题}。
降低同等或更高容错等级下的成本。
同等品质的硬件配置,pool越大,硬件发生故障的概率就越高。所以pool的规模也有上限。

ceph数据一致性检测和恢复是以PG为基本单元进行的。PG是对象的集合,PG对应的OSD存储了对象的副本。
副本一致性检测:\hl{peering, recovery, backfill, scrub}等过程。

PG上所有对象的写入,有primary OSD调度,串行化并记录有日志。

故障情况下,OSD(include primary and replica OSD)出现降级对象。

\hrulefill

RAID、副本、EC放在一块去理解。RAID有条带化、镜像、RAID5,在分布式系统中,对应\hl{分块,每块数据做副本orEC等机制}。
一是数据分块,放到多个节点上,提升并行度。再一个是容错等级,副本、EC是两种标准方式。
\hl{副本一致性和条带一致性是难点}。副本一致性通过RAFT协议去保障。

scrub可以发现静默错误,依赖的是checksum。

存储处在新旧之交,在快速演进。硬件和软件架构都在变革之中。
硬件引入了RDMA、SPDK、DPDK、NVMe等快速网络和存储介质、协议。
软件架构从传统阵列的SCSI、SAN、NAS到新的分布式架构。

新型存储SDS,包括Ceph、Amazon EC2、阿里的盘古等。

\hrulefill

快照,从COW到ROW到LSV,ROW用户数据共享,LSV不仅用户数据共享,而且数据出现多版本,所以需要GC。

ROW2和ROW3,都是带两级bitmap index。不同在于ROW2在新写时,按chunk发生了COW,
ROW3在新写入时不发生COW,但导致更严重的碎片化,影响读性能。

写有对齐写、不对齐写、overwrite覆盖写。读有对齐读和不对齐读。\hl{不对齐的写和读需要特别处理}。

\hrulefill

\hl{按对象、块、文件遍历本地和分布式系统},更复杂的NoSQL、NewSQL,暂时没有精力照顾到,后期也需要整理。
从哪些维度着眼?\hl{分布、副本/EC、sharding、快照}等。

先有disk,再有fs,fs里继续有disk(设备文件),我中有你,你中有我的关系。

怎么理解object、block、file之间的关系?\hl{NAS和SAN}在实际的应用场景中,有什么特别之处?
大家又是怎么使用的?

\hrulefill

\hl{纸上得来终觉浅,绝知此事要躬行}。看书、源码的时间要慢慢过渡到分析问题、解决问题。
头脑中要有big picture,在此基础上深化。如果没有这个,很容易迷失在技术的细枝末节里无力自拔。

更基础的知识放在长期的学习计划里,定量有恒,日积月累,定有所获。

\hl{戒定慧,此三学六度},实在是度人的船阀,还有很深刻简练的吗?
故称三无漏学,其道甚大,百物不废。何必舍近求远?实乃没有甚深定力和慧解的表现。

应渐渐地少看书了,多动脑、多动手,在解决问题的过程中,扩展知识和能力的边界。
(这段描述,有所不为而后可以有为,然后有所得) 知止是高级智慧。

AFA是方向,相关技术包括\hl{RDMA/DPDK/SPDK,NVMe、iSER、NVMf}等。
这些方面应该是下一步的重中之重,道不远人,如果舍近求远,恐有缘木求鱼的危害。

\subsection{18}

cache一致性与副本一致性存在很大的不同。cache一致性出现在共享内存多处理器、分布式文件系统client cache等场景。
副本一致性出现在replication和EC等场景。

cache一致性关注的是一个数据源,多个cache时的行为。副本一致性表示多个副本之间的一致性。
cache一致性可以用lease、oplock,可靠多播,MESI协议解决,副本一致性用RAFT协议解决。

\hl{完善性能分析和故障诊断toolbox}。

按\hl{主体-对象模型},一致性可分为观测一致性和数据一致性。

\hl{CAP和ACID},深入理解种种细分情形。

\hrulefill

lich meta里记录的是chkid到nids的映射,每个nid通过db查询到实际的disk loc。
相当于两级映射。

\hl{chunk tree过于复杂},难以做到事务性的要求。
allocate的时候,有可能会连续更新多级chunk。
迁移或故障的时候,也会引发级联更新。

lich无日志,如何保证一致性的?

\hl{lich chunk内并发},chunk内非覆盖区域,clock的连续性,理论上可以或并发或批量写入。
有一次排序、精简、聚合、并发的机会。覆盖区域,按fifo顺序提交。
如clock不连续,则需要等待。(\hl{paxos的日志写入,放松了该要求})

每个term包括\hl{选举、恢复、正常操作}诸阶段。

\hl{如何减少故障下的io中断时长}?admin、vctl可能发生切换、或reload。需要case by case分析。

实现LSV时采用了该方案。\hl{如果io size不规则,当如何}?
log和bitmap的写入,采用了pipeline模式。以确保log的并发提交,bitmap的顺序提交。

rcache比较复杂,开始按不同size建立多个cache是不明智的,应统一采用page cache,易于管理。
不过可以加入readahead等策略。

EC/dedup/compress/都需要建立新的映射。share底层数据块,但造成严重的碎片,对顺序读不友好。

\hl{回顾open vstorage的映射关系}。zerocopy snapshot?索引、共享、多版本?
多版本比共享有更复杂的引用关系。

\hrulefill

建立知识体系,依次scan各个分支领域。
\begin{enumbox}
\item \hl{CAP consensus, include paxos, RAFT and lease}
\item ACID transaction ARIES
\item metadata管理,如GFS、BigTable等。
\end{enumbox}

\hl{这次彻底把Paxos研究透彻,不能在一个地方跌倒两次。
否则就是有勇无谋,把握不住重点所在}。

戒定慧三字深邃,作为下一阶段的座右铭。

\subsection{19}

庄子内七篇、六祖坛经,从专业的角度去解读,很有意味。工匠精神、企业家精神是一而二、二而一。
回归专业是正确选择,一直知道、一直不能很好执行,知行不能合一。这次不能再偏离。

从分布式存储开始,建立问题和知识联动的体系,以助力其后的职业发展。

\subsection{21}

怎么查看cs和中断信息?

如何处理slow disk?

性能之颠:\hl{可观测指标=f(资源,workload)}。资源有使用率,基于时间(排队论?)、或容量。
在software pipelines里,workload是上游,可观测指标是下游,架构影响性能(化学)。

资源:\hl{CPU,memory,fs,disk and network}。

多网卡,MPIO。如何利用多块网卡?

面试经验谈
\begin{enumbox}
\item BAT,TMD,浪潮,联想等。
\item 仔细整理做过的东西,要更深入。不熟悉的东西宁可不写。要非常严谨
\item 用的什么型号的设备,测试的性能。
\item RDMA是重点,建立连接的流程,遇到的坑。
\item 基础编程题,本质上都是工程师
\item 要知道lich的不足之处
\end{enumbox}

引入kv用于元数据管理,加强client功能,client可以与object直接通信,更大的chunksize。
rich client,目前vctl就好像是这样的rich client。

恢复和数据平衡都依赖于控制器平衡,才能有更好的并发度?

\subsection{22}

CAP平衡ACID和BASE,进而引入单副本一致性问题,RSM的解决引入paxos、raft。

独立出metadata(KV)后,rich client,相当于controller前移到client,
如果允许多个client并发open一个LUN,会涉及一致性问题。

sheepdog采用DHT,不方便scan,所以recovery不好弄?

\subsection{24}

偏爱非对称架构,如有元数据服务器。再如SMP和NUMA。

控制器的粒度,从传统array、卷控制器,到primary osd。

io size影响到怎么处理,如果两个并发io发生了覆盖,则必须排序,然后在多个副本上按同一顺序apply。
\hl{如果不发生覆盖,则怎么执行都不会影响到最终结果}?元数据更新的io size通常较小,如64B。

ceph中pool不能过大,\hl{把pool限定在一个较小的节点集},避免过多的节点导致故障概率增加。

存储索引结构,\hl{btree,lsm},hash,bitmap、bloom filter。
bloom filter用在允许误判的情况下判定一个元素的存在性。

bcache用btree做什么?
bLSM是LSM和btree的组合体。

学习数据结构和算法,要自己动手推导,三部曲:\hl{猜想、重建、证实}。

\hrulefill

\hl{edog架构}。通过mnesia数据库管理vm和用户数据。
libvirt用于控制qemu-img,可以用xml配置文件。云管理平台通过libvirt来创建、启动、关闭、销毁vm。
qemu通过driver请求各种后台存储,如\hl{rbd、sheepdog、fusionstor}等。

\hrulefill

vfs不仅对理解文件系统有帮助,对理解任何存储系统都是有帮助的,包括对象、块等。
superblock是引导信息。inode维护着dir/file/volume到数据块的映射,dentry建立了global namespace。

zfs的pool化,\hl{FS/Volume}共存的架构令人耳目一新。传统的\hl{Volume/FS架构}。

存储有几级映射:
\begin{enumbox}
\item namespace (tree)
\item file to object mapping
\item object to location mapping
\end{enumbox}

比如ceph引入了pg,又加入了一层。bcache虚拟地址到物理地址的mapping。

ext2/3/4的间接块、extent tree数据结构。ext4区分inode为dir和file,
分别用htree和extent btree来实现。由此可见\hl{btree作为通用index结构}的强大之处。

\hrulefill

现在可认为建立了相对完善的知识体系。接下来就要围绕几个核心问题,
查漏补缺,\hl{致广大而尽精微},打造T型知识结构
\begin{enumbox}
\item 体系结构、操作系统、编程语言、数据结构和算法
\item 网络和分布式系统
\item 存储
\item TRIZ
\end{enumbox}

Lich需要进一步加入理解的地方
\begin{enumbox}
\item 数据模型
\item 数据分布
\item \hl{IO PATH}
\item 控制器架构
\item 本地磁盘空间管理
\item 副本一致性协议 (vctl, clock)
\item 恢复( +VFM )
\item 平衡
\item QoS(token bucket and leaky bucket)
\item 快照:COW、ROW、LSV
\item Slow and Broken Disk Detecter
\item ***
\item Redis Engine
\item BCache
\item Tier
\item EC
\item VDO
\item RR (sync and async)
\item ***
\item SCSI/iSCSI/iSER
\item NVMf
\item ***
\item Scheduler
\item Memory Allocator and Hugepage
\item RPC and corerpc (over TCP or RDMA)
\item RDMA/DPDK/SPDK
\item Multi NIC
\item MPIO
\end{enumbox}

\hrulefill

\hl{把操作系统作为主要隐喻},按操作系统的视角去看待形形色色的系统。
操作系统管理物理资源,提供syscall供上层应用使用。
在操作系统的层面有最好的抽象,如file、virtual memory and process。

\hl{分布式操作系统}会更强大,在OS之上加入分布式,则形成\hl{两层的知识体系},有分有总。
网络流,节点和边构成分布式,细化节点形成现代OS的设计。

\hl{分布式操作系统,简称操作系统},是看待世界的系统方法。

\hl{深耕操作系统这块良田},\hl{一分耕耘一分收获}。
这里的OS,既指实际的操作系统,也是一个更为广泛的隐喻。
坎贝尔神话隐喻予人以重大启示,终于找到了现阶段最重要的隐喻。
任正非是管道,OS是虚实结合的产物。

\hl{操作系统是软硬件协同设计的典范},体系结构、编程语言、存储、数据库和网络都是这个生态的积极参与者。
数据结构和算法。

用OS这个隐喻具有最大的开放性,同时又是专注的。
\hl{炼金术士手中的操作系统}整合了资源,且有着预定价值输出。
\hl{OS具有一即一切一切即一的大气魄}。

\subsection{25}

操作系统包括\hl{计算、存储、网络、协作}。从协作的视角来看,分布式系统算法。

卷管理 disk, raid and LVM,ZFS

卷的两种使用方式,一是Filesystem,二是iSCSI,分别对应NAS和SAN。
NAS文件系统在内部,SAN文件系统在外部。\hl{SAN通过NAS网关即可把NAS包含在内}。

体系结构、编程语言都是OS应有之义。OS封装体系结构供PL使用它的服务,OS又是PL构建的,\hl{Arch、OS、PL构成三连环}。
OS提供了最有用的抽象,File、Virtual Memory、Process and VM。

\hrulefill

\hl{删除卷}不同于一个一个free object,整体删除应有更便捷的方式。
在etcd上产生一个删除卷的JOB,包括每个节点上的task。由每个节点调度执行。

每个节点上,周期性地polling相关任务,加入本地任务队列。由独立线程或线程池处理该队列。
\hl{每个task建模为状态机}。处理完一个task后,从etcd上删除该task。

从replica cache、db和disk里删除vol相关记录。
注意操作顺序,应先解除ref关系,再回收disk空间。

\hl{删除pool}也应采用batch方式。保证后置条件,执行后不再有该pool数据。
因为pool id采用了name,\hl{what if}如果在此过程中,又创建了同名pool,会如何?
最简单的做法是禁止这样的行为。

\hrulefill

能用int表示的,会方便很多,如nid、diskid,需要保证在作用域内的唯一性。

CAS lock?

抽象,庄子齐物论

\hrulefill

bcache,拔盘后虚拟盘依然可写?

\hl{EIO下踢盘},这样做会引起什么问题?合适吗?
踢盘后,如果引起部分对象副本不可用,应该如何处理?

画草图,保留自己的笔记!

\subsection{26}

单核、SMP和NUMA,共享内存多处理器

先研究各类调度器,如linux进程调度、io调度,erlang/go协程调度,k8s资源调度等。

linux进程调度,要处理MP架构,减少上下文切换和cache miss等问题。

erlang/go的协程调度,是建立在内核线程上的非抢占式调度,支持大量协程调度,还要处理堵塞io带来的问题。
即\hl{调度与内存和io}要结合起来考量。io是通过异步提交和事件驱动的方式做的。
coroutine虽然是非抢占式的,但可以通过\hl{yield和resume机制}主动让出,由外部事件来唤醒。

allocator包括内存管理、磁盘、卷管理,乃至文件系统。

\subsection{28}

协议了解少,包括\hl{SCSI、NFS、CIFS}等,须强化。
RDMA/DPDK/SPDK、NVMe等慢慢成为主流

\hrulefill

分布式系统概念与设计。

分布式系统和操作系统是基础,需要专精。\hl{一门深入、长时薰修}。

理解分布式系统的起点是CAP定理,为了HA,引入副本,为了性能、引入cache。如此就要进一步解决带来的一致性问题。

\hl{CAP是思考原点},引入一致性、容错、可用性、性能,弹性、可伸缩性等等。
引入safety and liveness、ACID and BASE。
引入replication and cache,进而引入consensus (zk、etcd等)。paxos是核心算法。

分布式系统的很多手段在RAID里已经采用,如条带化、镜像、校验码。分别对应\hl{分块、复制/EC}。

Erlang/OTP在解决分布式系统的痛点上是怎么做的?如何拥抱失败?
协程调度,适应MP体系结构,同时把mm和io纳入考虑。进程有mailbox,异步通信。
\hl{go与此有所不同,采用了csp范式},把channel显式地提了出来,增加了灵活性(Q)。

分析磁盘故障的处理策略:如何检测?恢复和io策略?
\begin{enumbox}
\item 出现EIO应该马上踢盘吗?
\item 检测慢盘坏盘的方式是什么?
\item 无clean副本时,下一步该如何处理?
\item 一次恢复失败,接下来怎么办? retry, redo(all of disk)
\end{enumbox}

两种模式:强一致性和降级。如果采用强一致性,会堵塞io直到恢复完成,iops下降严重。
如果指定最小副本数,\hl{运行于降级模式,采用异步恢复策略}。就影响一致性,无法容忍连续故障。
\hl{性能是可用性的一个体现,这是CAP认识的深化}。
何况CA不是铁板一块,而是分层分模块有结构的,每个接口都需要单独分析,
在总的指导思想下,case by case地去处理。

节点故障:不同于磁盘故障,需要全面scan发现待修复的chunk。

两层:分布式和节点。

熟悉lich模块
\begin{enumbox}
\item scheduler
\item memory
\item io (normal, nvme, spdk)
\item rpc (including \hl{TCP/IP/RDMA/DPDK} etc)
\item ***
\item application protocols, iSCSI/NFS/CIFS
\end{enumbox}

每个core线程整合了scheduler和io,统一用epoll方式驱动(eventfd, timerfd)。
不同core线程通过corerpc进行通信。

为什么要依赖于物理时间戳?每个host时间相差要在5s以内?

知识体系:
\begin{enumbox}
\item 横(\hl{编程语言、算法、设计模式、架构})
\item 存储协议
\item 分布式系统
\item 本地引擎
\end{enumbox}

\subsection{29}

\hl{RDMA对NIC或switch有要求}。

DPDK是用户态TCP,用在以太网环境下。SPDK也是用户态NVMe驱动,并提供了NVMf。
如果不引入支持RDMA的交换机,在all以太网环境下,可以用SPDK/DPDK分别实现\hl{全用户态的存储和网络}。

输出价值的能力是最重要的能力。

没有时间深入研读DPDK/SPDK,只能做大体上的把握。
第一个问题就是\hl{如何在用户空间实现设备驱动}?

每个进程的地址空间分两部分:用户态和内核态。内核态只有一个地址空间,映射到不同的进程。
\hl{不同进程share一份内核空间},多进程trap后并发执行。

pci设备,通过mmap可以让用户态代码访问。\hl{每个pci设备都是映射到内核的一块内存区域},可以把该内存区域映射到进程地址空间。

能否在polling模式和中断模式之间动态切换?\hl{用户态接收不到中断,由uio接管}。polling模式检查状态寄存器的值。

缓冲区溢出攻击

\subsection{30}

\subsection{31}

\section{201806}

\subsection{0609}

坚持写日记。

李小龙的武术与哲学,以武术诠释哲学,吸收了佛道精神。不拘泥于形式,追求简单实用高效。
阴阳是理论基础,阴阳是一元论,一而二、二而一,不是二元论,侧重于一体两面的统一性。
水与空杯是隐喻。

功夫是自我发现与自我实现之旅。

志于道、据于德、依于仁、游于艺。道是第一位的,通过德、仁、艺得以落实到平常日用间。
德是大学的明明德,也即是小龙的自我发现之旅。仁是外在的,老子三宝的慈、俭、不敢为天下先。

艺不可少,择一艺以终老,就是软件设计,亦日日所从事的,注入真气,求大成。
不可作为寻找事情来做,而是纳入自己的生命旅程。

道法术器,是另一个序列。从艺的角度去分析。艺人通过自己的作品说话,突出作品而忘记自我。

古有练剑师,注入自己的生命,十年磨一剑.各行各业都有此类艺人,诠释大匠精神。

练拳不练功,到老一场空。不可不重基本功,打下坚实基础,可以走得更从容更远。

\subsection{0610}

诵黄帝阴符经,观天之道,执天之行,尽矣。见识行事必有所本,其源头在天,建立天道格局,则大事可成。
师心自用,以一己私智,则事半功倍,结果不容乐观。


\section{201807}

\subsection{0702}

虚静,摄无量义。

无我曰虚,归根曰静。无我而归诸道,心与道合,是为真人。

淡泊明志,宁静致远。

\subsection{0703}

123哲学是分子结构,再往上就是系统论。一个系统由子系统构成,形成层次结构。
系统具有分形属性,一即一切,一切即一。一花一世界,一叶一菩提。

抽身物外,胜物而不伤,勿死于物下。道提供了与物沟通的另一维度,
道者,万物之奥。道者,物之极。架构师与程序员的不同,主要也是在此。
精于道者兼物物,精于物者以物物,下学而上达。

道物,粗分有两个层次,然上通九天,下贯九野,一层功夫一层理。
合中有分,分中有合。

这一关确实不好过了,走还是留,是个问题。不管怎样,都要做好充分的准备。

管理不上路,财务不合规,关键是能不能虚心听取意见,
从中获得成长,一时的成败不是决定性的。

\subsection{0704}

我注六经,六经注我。我与六经之间是超越线性的关系。为今之计,发明心地,明心见性。

寻章摘句,君子不为。以虚壹而静之心态,拥抱现实及其变化,确立道为最高原则,尊道贵德。

归纳整理出我的原则,至关重要。

系统化的决策流程,决策攸关成败,有底层逻辑,有道有术。

守、破、离对应心物,心道、道物三线,成三角形。

\hl{做决策不是我什么什么还没准备好,要相信自己的基本功与学习能力}。
精于道者兼物物,致力于道,物不会是严重障碍。

顶角即是道,也是机器、系统,看到二中之一,看着物理学之后的形而上的东西。
形而上者谓之道,形而下者谓之器。此一上行下行的路径,揭示了更多可能性。

人生算法有认知闭环:感知-认知-决策-行动,是动词构成的,心道物三者,是名词构成的。
内核与外环,内核是最小化的那个点,外环是动力与使命。认知闭环发生在心物之间,
三角形的每个边都是一个认知闭环,PDCA循环。这些小车轮,架起了友谊的桥梁。

是节点问题,还是边问题?居于中心的是什么?

把道、原则、人生算法、多元思维模型、混沌大学课程这些模型融合起来。
打造自己的模型。

取势、明道、优术,取势在心,明道在道,优术在物。
外环由心发动。

夸克构成质子和中子,1:2的比例关系。

把最近围绕道的认知,应用到工作中,在知行中螺旋上升。
一是道心物三角,二是认知闭环,三是体道方法与心态虚壹而静。

稳住,静下来,搞点大事,五年磨一剑,一战定江山。

原则:心态、机器、系统。分生活、工作、投资等领域内归纳出的一些原则。

算法:认知闭环。

多元思维模型:从硬学科里提炼基础模型,形成体系,运用到各种决策场景。

混沌大学:用第一性原理,跨越第二曲线的不连续区。

道具有最终的统一性,众星拱之。

\hl{把分布式块存储系统列入最小内核},运用即即为广泛,深度也够,待解决问题也很重要。
怎么让它最大化呢?占据铁三角的物之一角。要呵护珍惜!

同时需要从别的领域吸收养分,但这个是核心,如果能立下来一个核心,
来华云不管遇到什么,都值了。

\subsection{0705}

体道者逸而不穷,任数者劳而无功。双线法则

战略,不在战,而在略。亮独观其大略。

用心体会虚壹而静四字。

道不欲杂,道是朴素的,一立而万物生矣。

% 如果钱能到,很好。任何时候,成长都是第一位的,如果因为钱影响了成长,就得不偿失。

成长如何衡量?曰道。道是一种信仰,有道则吉,无道则凶。道之有无取决于目标。

\hl{NLP思维逻辑层次}:精神、身份、信念、能力、行为、环境。

前五个都是我,把握当下。

精神=道,身份=我,信念=原则,自此以往,皆算法。

养神之所归诸道,身份是入口、枢纽、关节点。无我,上通于道,惟道是从。
道居于太极至尊的位置,至尊而不独尊。

内静外敬,性将大定。

\subsection{0706}

正念、良知是体道见性知天命的方法。

虚静,一是尊道;二是正念,如如不动。

\subsection{0709}

惟精惟一

打磨三合架构,整合原则、算法,去分析问题

\begin{enumbox}
\item 过以原则为基础的生活
\item 更高层次思考
\item 做一个超级现实的人
\item 极度的头脑开放
\item 五步流程方法
\item 如何做出好决策?
\end{enumbox}

在心物道三者间,持续转动。用三合结构分析达里奥的原则一书。

道者,物之极。升维思考物的真实价值。回到心,是否足够空灵高效有力量,心智模式。

心,极度真实、极度开放。

原则一书,也是升维降维,上帝视角,引入机器、系统,进行控制。
机器位于物的节点,分解为目标与结果、团队与规则。

五步流程法等同于设定目标+认知闭环(感知、认知、决策、行动)。

怎么做出好决策?

限制一下悟道的时间,不必太多,时时提起。

设定下一阶段的目标,全力以赴

\section{08}

\subsection{01}

dmidecode可以查询服务器型号

\subsection{02}

理解target,各种各样的target。host-target之间的transport和protocol是区分的关键。
\hl{类比TCP协议栈}去理解各种新的网络技术。

tgtctl是target和storage的交接点,体现在文件\hl{nvmf\_suzaku\_io}里。

spdk的NVMf导出bdev。如何对接分布式存储?

把libnvme用git管理起来\todo{git-libnvme}。

尝试用一台vm把suzaku跑起来。看看具体要求和配置是什么?

完善关键流程,补上漏洞。采用\hl{用系统来工作}的理念,完善过程。

test是什么状态?应该怎么做?

hazard相关文档。

排兵布阵,上知天文下知地理。

NVMe中buffer的表示,sge?

\subsection{06}

通过ipmi控制服务器。

一块nvme盘加不上,不知为什么?51,52,53上都是如此。51重新插拔盘解决,52、53拔掉电源,重启解决。

实则性能不如8.1版,为什么?观察到disk延迟高,对disk单独进行性能测试,剔除慢盘。
用4盘测试,性能达到600w+,但latency double了。

测量每块盘的平均队列深度和延时。为什么disk的latency突然变大了呢?
\begin{myeasylist}{itemize}
& 没有读过的盘,非稳态性能?
& bactl有问题?
& remote first后,iops显著下降,latency显著升高,磁盘压力小
& \hl{把单卷大小改为80G之后,性能提升上去了}。
\end{myeasylist}

mds\_rpc\_paget,并发高,导致rangectl的内存耗尽?

加入节点,rehash,等待lease timeout,io会中断。

三个client不要同时启动,而是错开几秒钟。

rdma 在提交和完成之间,可能会占用大量内存,导致内存耗尽。怎么解决?
内存不足时使用后备内存,以处理峰值情况。或者core内存管理动态化。

\hl{拆分为两个库,都需要用静态库},不能用so。

\subsection{07}

\subsection{08}

\subsection{09}


\section{201809}

\subsection{0901}

\subsection{0902}

\subsection{0903}

战略几何学、神圣几何:圆是时间,四方形、十字架是空间,三角形是存在,构成时-空-存在的结构。

双环系统可以解释一切,双环相交处是太极图。右手螺旋法则。

周末读书,关注到几个概念,心神、机发,心神论是黄帝内经的精髓,机发论是易道主义的理论枢纽。

机是什么?随机而动,机是变动不居的存在,但可以通过思维与实践的方式去认识和把握。
阳明心学的精髓:此心不动,随机而动,就是圆点结构。

一心一意到专业学习上,有道有术两个层次多个层次。所有的事情,都是培养心体。
要留出足够的时间去反思,并记录下反思的过程与结论。

这么多年,很遗憾的一点就是不能一心一意,也就是不诚,身在曹营心在汉,不能全心全力地投入到手头要紧的事情上,
老是觉得另有更大价值的事情,反而导致手头的机会也白白溜走。

今后当从容规划(转动PDCA循环),稳扎稳打,一步一个脚印,去实现目标。

几有多义,主要是微和危。事物的萌芽状态,看不透、想不明白,\hl{惟精是惟一的功夫},博文是约礼的功夫。这是阳明一贯的主张。
守住底线、抓住关键是方法,围绕一转动PDCA循环。

\hl{如何尽快实现财务自由}?四象限,打工、个体户、创业、投资。贯穿其中的是\hl{专业、工匠精神}。
只是有工匠精神依然不够,要有道。立足于当下,什么才是最重要的事情呢?

\hl{乾之九三给出了答案}。乾坤是易的门户,黄帝垂衣裳而治天下,盖取诸乾坤。

\subsection{0904}

乾坤是易的门户,易是通向现实世界的门户。这是非常重要的论断,因为一是学易之法,二是用易之法,学以致用,解决现实问题。
读书不在乎多,学宗大易,一部易经观天下。透过一部易经,而通达于现实世界,得偿所愿,心想事成。
通过易,撑起开物成务、进德修业的英雄梦想之旅。

六爻之动,三极之道。分而论之,初二为地道,三四为人道,五上为天道,匹配几、诚、神。

用\hl{三级火箭模型}分析创业公司的发展轨迹和着力点,什么是发动机?如何一环套一环?
产品和客户是任何公司的两极。设计理念与客户反馈要综合为用。

易之三义,变易、不易、易简。

\subsection{0905}

努力经营事业,开始物色各类人选,看看水浒传、三国演义,任何事业都不会想当然地一蹴而就,而是长期经营的结果。
事业是男人的第一支柱。

易经在这方面有着深沉的诉求,圣人以神道设教,抛开迷信的成分,易经是第一励志书,也是第一帝王书。
学习易经,方术方面了解即可,不作为主要方面,重要的是开物成务、进德修业方面的启发和辅助。

至九四,始入于上层,开启了自己的平台和事业。上下分际处是着力点。
或跃在渊,此一跃是多方面因素叠加的结果,主要还在于自己的野心、理念、认知。
此一跃,不回头。

一是因果律,二是神圣意志之发扬。乾卦就是这样的精神力量,乘云气而御飞龙,高扬进取意识。

更加open地去思考关键问题,包括行业、事业等等。思考、交流都是需要的。
进一步去了解别的产品,主要是把握趋势。

双环系统可以解释一切,双环相交处是太极图。

怎么通向现实世界呢?

\subsection{0906}

不能控制自己的情绪,太幼稚,这种东西纯粹影响发挥。当前第一要务是什么?事业,不容置疑。迄今没有起色。

第一个是专业环,这是安身立命之本。经过多年的摸索,是整理收割的时候了。

第二个是易经,全面拥抱易经,以之作为进德修业的重要基础。以此洗心,退藏于密,洗心,就是淬炼心智模式。
易经在思维方面,有着深度与广度。进入眼界的思维模型,都挂入易经这个思维格栅中去,易经就是太上老君的八卦炉,
淬炼出了孙悟空的火眼金睛。

另外,黄帝内经所蕴含的神本论以及机发论思想,在易经中也有深入的体现。洁净精微,易之教也。

环环相扣,专业与易经之环,碰撞出火花。工作与生活都需要大设计。

不要急、慢慢来。易经为起点,一部易经观天下,指导生活与工作之设计。专业是工作的一部分。
生活是进德,工作是修业,内外兼备,合内外,一物我。

一切的学习都不仅仅为了学习而学习,为了单纯的知识而学习,而是为了解决问题。

关键思想:
\begin{enumbox}
\item 确定易经作为最高指导思想,第三空间或虚或实,主要指的是这个,过有原则的生活,富有之谓大业、日新之谓盛德
\item PDCA作为执行方法
\item 双环系统分别对应生活和工作
\item 把\hl{视点/视角方法}作为架构描述语言
\item B:确定把分布式存储作为主要的技术领域
\item B:确定把QoS作为主要的研究方向之一
\end{enumbox}

\subsection{0911}

全力以赴到专业方向上,去解决关键问题,太极云尔,是反思框架。

心、道、物的三合之道,适合于下一阶段的学习过程。心就是阳明所谓良知,为学头脑所在,多问多思。道,原则,方法论,架构。
物是要研究的系统,要解决的问题。以道观之,以架构之眼看系统,当如庖丁解牛。

双环,一者三合之道,二者PDCA。双环正交?

对解决问题有腻烦心理,问题是前进的动力,当善待之,乐于去搞定她。

\subsection{0912}

心神主宰,以道观之,落实到物,以道的光华普照世界。寂然不动、感而遂通天下之故,这是二重性。

第一个小目标,100w,1000w,以此类推。明年大概就有100w了,坚定地走下去,不急不躁。重为轻根,静为躁君。

架构驱动的软件开发过程。

坚持用SWOT分析,是战略分析的起步。

\hl{本周末给出一个更明确的路线图}。第一,强化架构思维能力,视图视角是标准做法,IEEE STD 1471-2000。
视图可以视点集为模板,也可以单独定义。运用视角到视图之中,形成纵横交错的架构描述。

\subsection{0913}

\begin{shadequote}

能把诚神几统一起来的为圣人。北宋周敦颐在《通书》中提出的命题。“寂然不动者,诚也。感而遂通者,神也。
动而未形,有无之间者,几也。……诚神儿曰圣人”(《通书·圣》)。
诚是静无的,即“诚无为”(《通书·诚几德》)。神“感而遂通”,是诚的直接表现。几处于静无动有之间,是动之始。
诚是纯粹至善的,是一切道德行为的源泉。
神是诚的直接表现,故亦善。只有几“动于彼”,感外物而动,故兼有善恶。
《宋元学案·濂溪学案上》云:“常人之心,首病不诚。不诚故不儿而著。不几故不神。物焉而已。”
常人不能以诚贯几,受物之累而为恶。只有圣人才能以诚贯几,去几中之恶,把诚神几统一起来,故诚神几曰圣人。
\end{shadequote}

心道物,诚神几,有对应关系。把心置于三角形顶点处,似更体贴。

养心莫善乎诚,致诚则无它事。至诚之道,可以前知。惟天下至诚,为能经纶天下之大经,立天下之大本,成天地之化育。

圣人以神道设教,道则通神,一阴一阳之谓道,阴阳不测之谓神。何为神?妙万物而为言者。

几,人心惟危,道心惟微,几则合多义而言。机发论更提出制机的说法,乃易道主义的理论枢纽。
从机发论的角度理解,\hl{黄帝内经}灵枢,\hl{鬼谷子、阴符经}亦然。

\hl{此三角形居于左侧(符合右手螺旋法则),圆形+十字架构成的几何形状居于右侧(SWOT, PDCA, 2x2矩阵及其延伸,符合左手螺旋法则),
左右交错,形成太极之两仪}。大拇指都指向自己,反求诸己,建立自我,贵我通今,时变是守。
此参伍以变,错综其数的义理架构,实有进一步发挥的余地。

左为知、右为行,以此类推,大商之道的道术、变常、方圆、生死、利害、取予之对立统一,也是如此。

孙正义的25字诀,与\hl{周易、兵经百字、东方战略学},都是以字通道的卓越理念。

\subsection{0914}

观象玩辞,以字通道。建勋画论的三合之道,启人深思。道具太极位,则有商讨的余地。邵雍曰:心者太极也。华严经云:心如工画师,能画诸世间。
阳明心学也是如此。心是能动的一面,也是目的性的一面,使心居于太极位,乃应有之义。心秉道通物,心格物穷理,天性,人也,人心,机也。
立天之道,以定人也。此说并不否定或拉低道的价值,而是在建立自我的阶段,高扬心性,确立为学的头脑。道依然是那个道,
致吾心之良知于事事物物,则事事物物皆得其理。即满足了目的性要求,又满足了道的约束性原则性。

欲正其心者,先诚其意。在明道、格物的过程中,诚其意。事上磨练,皆在涵养此心之体。由物及心,完成此逆时针的环转关系。此右手螺旋法则。

如忽略道的环节,而直奔物的主题,则易于陷入事务主义的泥淖之中,事半功倍,乃至无功而返。
如过于强调理论,也有教条主义的倾向。

神者生之本。

\subsection{0918}

系统思考。

职场与理想的距离,靠三度修炼去完成。三度:态度、气度、厚度。读一艮卦,胜读一部华严。
中秋看王明夫主编的三度修炼,好好想一想下一步的规划。

\subsection{0920}

离开HY的可能较大,离不离开,都要以成长为主要标准。时间并不充裕,接下来到年底的一段时间,好好锤炼专业技能。

\hl{优先考虑开启自己的事业},专业技能的学习、知识体系的构建,不能脱离这个目的,才称得上学以致用。

\hl{全闪时代来临,离自己最近,怎能再次错过}?应采用包围式学习,地毯式学习,既要明确关键,又要面面俱到,点线面体,全面展开。

在多副本复制的场景下,由一控制器负责,如果控制器发生切换,则开启新纪元。在某一控制器的生存期内,
每次提交采用单调递增的版本号,所以二元序号的构成:(epoch, version/clock)。
卷控制器可管理很多chunk及其副本一致性,控制器位置与副本位置不具有对应关系。\hl{卷控制器可迁移}。

关于控制器的若干关键问题:
\begin{enumbox}
\item 如何选取控制器
\item 客户端如何定位控制器
\item 控制器发生切换又如何
\end{enumbox}

paxos的精髓是温故知新,一个实例产生一个值。如何标记序号?序号可以是二元结构,方便处理。

multi paxos与RAFT的异同?每一个控制器的生命周期包括三阶段:\hl{选主、恢复、正常操作}。

进一步,传统的2PC、3PC算法的不足和使用场景。这类算法是分布式系统的精髓,务必加以消化理解。

\hl{算法是程序员的金线},理应是下一阶段的重点。比如,通过token bucket或leaky bucket解决qos问题,实现很简单,设计很精妙。

马云定随舍三部曲,第一曲是定字诀。艮,止也。知止而后有定,定而后能静。

\hl{起居有常、饮食有节},乃养生之道,不仅此也,常与节有深意存焉。
财自道生、利缘义取,是大商之道。菩萨畏因,凡夫畏果。

\hl{多听多问}是领导之道,陈述句不如疑问句。

易经的卦图是直线,加上圆点哲学,三角形集两者之大成,融合双线法则、圆点哲学、三点论、一分为三诸论而为一,
算是多年思考、探索的一点结论。三生万物,由此而展开其广泛的运用过程,进入明体达用的第二阶段。
用太极两仪模式解读三角形各点之间的关系。

道是吾观物的门户、工具,不能僭越心的第一性,道物、道器、体用,分阴分阳,叠用柔刚。
\hl{吾有方圆之形}。五代表圆点哲学,PDCA等。以五为食,哈?口为口、为目,以五观之、观天下。

两个三角形,下一个代表物理资源,上一个代表虚拟资源,中间的交点是集群,物理资源总而为一,进一步生化出虚拟资源。

心道物三角,自身也有两种旋转方向,左手螺旋右手螺旋,标准图以心为顶点。

\subsection{0920}

战略一二三,美团十字架,参伍点成圆,乱环诀中诀。

智仁勇三达德,好学近乎智,力行近乎仁,知耻近乎勇。

在乱环之中,存在第一义,找到她!

架构、算法是内功心法,练拳不练功,到老一场空。

功业之心热灼,怎么开始?如何播种下第一粒种子?离什么最近? 立于中央,由近及远。

无所待,此时就是开始!此时此地,从心开始。

开始不难,终局判断如何?商业计划书?开始吧。

人钱事,搭班子、定战略、带队伍。做什么?怎么做?如何解决启动资金的问题?

如何整合资源?一二三级火箭分别是什么?

\subsection{0925}

心道物,以心为开始,以道为顶点,以物为落实。三者太极两仪,环转无尽,融归与太极大道之中。
如此排列,不压抑心的能动创造性和塑造能力。

心何以转物?以道为中介,诚神几,修心贵诚,通道故神,风起于青萍之末,挥之于泰山之本。
上通于道,下及于物。向道的跃迁层层递进,进一层有一层的道理。

任何一点都不足以表达正确的关系。

中秋假期间一个最重要的反思就是要有制胜的意志。\hl{善用兵者,修道而保法,故能为胜败之政}。
举凡人事百端,无不以胜或赢为最终的目的。取胜的方法多端,宗旨则一。

古今中外的人之共识,老庄虽然一直在说恬淡虚无,何尝忘记求以得,有罪以免也,故为天下贵。

从胜的角度,从修身为本三合之道的角度去诠释经典,别有一番风韵。

\hl{立足于专业技能,从战略的角度拓展未来成长空间},战略思维是一项极端重要的能力。

道天地将法,也是一重要的思考框架。尚五,五包含了一二三四,圆点哲学、太极阴阳、三才之道、四象/PDCA。

老子缺少点进取的意味,孙子则攻守兼备。

心到道的距离是认知差,\hl{道是超认知}。在不同的认识面上,相同的公理具有不同的内核,这就是hegel一直说的熟知非真知。
以道观之,在道的高度上,运用简单而普适的公理,可以达到非常好的结果。

人心惟危,道心惟微。危是指称思维的不可靠,微则是思维的神妙不测,真理与谬误在一线之间。
洞察思维的误区盲点,极深研几,就可以越来越接近道的境界。

查理芒格研究思维的错误模式,就是有鉴于此。普通的思维是靠不住的。
波普尔的证伪理论,索罗斯的反身性,都是解决这一问题的哲学努力。
更早,则有休谟的因果质疑。

\hl{枕戈待旦、厉兵秣马},为了最后之胜利,不能不如此。

心的综合能力,读书如果不思考,就破坏了这种能力,显得支离。

为什么要从心开始呢?虚心涵泳,切己体察。

架构师,工匠精神,粟裕尽打神仙仗。\hl{全力以赴投入到专业技能的学习和提升上去},主次不能颠倒。
说别的事情还显得太远,比如和君的国势、产业、资本、管理四库全书等。这是下一阶段的事情。

通过研读阳明学,更主要的是,通过建勋的画道提出的三合之道,确立了基础的思想方法和工作方法。

破局、突破,更上一层楼。

进一步提出经营方针和工作程序。

\subsection{0926}

六经注我,我注六经。阳明学提升了我的价值,先确立我,建立自我,第二步才是追求无我之境。

系统读书,一旦确立了我,读书就是为我所用的过程。志于道,游于艺,六艺摄于一心,如此,心物关系中,心为主,物为从,精神作为能动性的一方面,发挥了更为重要的作用。
即是在格物中诚意,在诚意中格物,尊德性而道问学。

留给自己的大机会不多了,需要极大地发挥精神的能动性,去慎思明辨。四十不惑,处在这个关键的转折点,怎能不好好地把握呢?
机遇偏好有准备的头脑,潜龙勿用,一定要静下心来,苦练内功,打好下一步发展的坚实基础。

战略致胜,战略是道的运用,以道莅天下。孙子兵法计篇:道天地将法。以五行对照之,道立于中央,天地定位,左将右法。
将作为能动性的一方,也不能不受道的制约,取胜的一切要素,都围绕着道而旋转,五行是更具体的模型。道具有目的性和工具性等多重价值。

\hl{搭班子、定战略、带队伍}是柳传志的联想方法,对应到将、道、法横轴上。
将是领导、法是管理。国势、产业、资本、管理,管理是创业之后的事情,且不可过于陷入微观管理的泥潭。
产业才是第一要考虑的领域,在国势下定位产业,资本、管理是随着产业而运转的。

\hl{战略罗盘}从内外、知行两个维度进行划分,从外到内。

借力打力,分布式存储最好的借力点是openstack和vmware、oracle等场景。顺势而为,方可事半功倍。
不懂得借力,没有生态思维,自行其是,往往落个狼狈下场。

认识差:红山为什么看不到云才是最大的趋势?华云错失超融合、全闪风口。后果严重。为什么大家往往看不到最好的机会?

光点的机会又是在哪里?从手机壁纸到游戏、区块链?

确实到了一个关节点,在专业上提炼总结,来一个大突破。

\subsection{0927}

十一在家研读阳明心学,以及企业家精神。阳明心学是与三合之道最为合拍,源于易经和道德经古老传统的三合之道,源远流长。
数年探索终将有所结论,涓涓细流汇成汪洋大海。悟后起修,悟后大有功夫在。

白立新的致良知四合院是阳明心学研究的一股清流,面向企业家也是有的放矢。

也不要在书面材料里太久,反观自心是最后的归宿。明心、净心,万事万物即在其中了。扩大心量,笼罩万物。
心生则种种法生,六祖坛经:心量广大,犹如虚空。

从专业工作者,进阶到企业家、投资家,打通四象限,是毕生追求。经过多年沉淀,人生进入第二阶段,唤起使命。
千面英雄的轮回,乾卦六龙御天,皆指示了人生的阶段性、周期性。40-50岁,正值人生壮年,争取走完第二阶段。开启第三阶段。

向前看,扭转思维的主视角。六合上下,立足当下,展望未来。能做什么? 高瞻远瞩
复盘重要,未来更重要。禅剑合一,心剑合一。引入第三点道,层次分明,动静无间,则曲得其妙。

不是观众,演员,要去做导演。

\subsection{0928}

S曲线是成长曲线,第一曲线跨越到第二曲线靠什么? 不要忘了当下最重要的目标,自我成长!

在公司住了一宿,感觉尚可。创造这样的条件。研究以下生活自理方面的,极简主义风格,体现了根据地的重要性。
建立根据地,然后进可攻退可守,极大地拓展了生活工作的战略半径。

\hl{根据地思维}不仅是个人的,工作、创业等都需要,是一个重要的战略思维。
高筑墙、广积粮、缓称王,说白了,就是根据地思维。
建立根据地后,就保有了一个\hl{极具弹性的战略空间}。
在三国争霸、国共之争中都体现得淋漓尽致。

在北城买房、甚至在郊区买别墅是一个战略构想,解决工作与居住地之间距离的矛盾。
确实,90\%的问题是money的问题。财务自由是个人发展的一个重要里程碑。

专业上,\hl{分布式存储系统就是一个战略根据地}。

\hl{国庆节计划在家研读阳明心学与企业家精神},认真思考下一步的作战规划。
从作战的视角来审视各种活动,会有更大洞见。
作战追求胜利,评估得失成败的根本原因。兵之形象水,因地制宜的灵活性。随机而动,追求胜利。
任何活动,都是项目,也是一场战斗、战役、战争。
没有求胜的坚决意志,就会错过最美的风景。
阳明在军事上的成绩,与悟道有关,也与他研读兵学有关。

兵贵胜,不贵久。

读\hl{阳明心学的管理智慧},三体世界的提法,与三合之道契合。主体世界、本体世界、客体世界,与心道物一一对应。
心上通于道、下及于物,这个上下反复的过程,完成了三者的互联互通。

心即理,知行合一、致良知、四句教等核心命题都可以在三合之道框架下,得以完美诠释。
知行合一贯通三体世界,故有三知三行之说。

心、场、道、法、术、器,细化了心道物三要素,引入了程序化、结构化的元素,如原则、用系统来工作等书所提及的,
约法三章,修道而保法,故能为胜败之政。

互联网思维的贯通,\hl{S2B2C商业模式的解读},很有新意,值得进一步品味。

国庆细读。阳明心学与企业家精神,围绕这个中心问题来进行阅读。
企业家是第二阶段的主线,一定要好好体会这个角色面临的机遇和挑战。

现在处内外交困之境地,内则杂念憧憧,外则工作事业皆不尽如人意。
当思如何破局?向前看,战略六看!

应以阳明学为中心,一心摄六艺,六艺摄于一心,六经注我开生面。
易经乃一经一艺,\hl{心易}的说法,有共鸣。

\hl{阳明学是很好的下手处,切问近思}。在三合之道的义理框架下,理解阳明学变得容易多了。
重要的不仅仅是理解,更在于知行合一。\hl{且知且行,咬定青山不放松}。

\subsection{0929}

从阳明心学可以辐射到传统和当代的诸多领域,比如对明治维新的影响,阳明心学复兴的意义。
周易古奥,难以直接受用。阳明心学从解决心物关系问题入手,确立了心的主体地位(张学智会通中西古今)。

高扬主体精神,心与道合,道济天下,以道莅天下,知行合一。PDCA,规划是知,执行是行,检查是知,调节优化是行。知行并进。

知,认知差,超认知。知己知彼、百战不殆。知机而后制机,知在人生旅途中,发挥着巨大的作用。
吴军写见识,无战略悲人生,处处都在说明知的极端重要性。认识升级、培育正念,是人生要务。

预知、先见之明,更是重要素养。道知,以道而知。损兑,专以心眼察也。阅读、旅行等只是获取知的具体途径,
开发心田,培育正念,时时处处在心体上下功夫。

涵养须用敬,进学则在致知。敬字提点人心,使不昏沉放逸,精神内守,为功大矣。
不仅此也,\hl{离致知而言内守,则有枯寂之弊}。

以良知之体、道之光华照耀大千世界。

尽快挣到足够的钱,是当前最大目标。如何做到呢?在因上用力。

早上学习专业,晚上用阳明心学作为反思批判的心法,做此二元分解,是为了突出专业精进的重要性,
否则就是知而不行,非真知也。阳明心学贯穿一切活动,

话有点多,自以为是。当渊默内敛,静心养神。

言行,君子之枢机。君子之所以动天地也,可不慎乎?公司各项时态之发展,固然重要,但不是最重要。
最重要的是成长,此为根本。此为重为静,终日行不离其辎重。

从平安和人寿的结果看,lich局部表现不错,应把这点作为一个标志性事件来看待,能看到其中蕴藏的巨大商机。
然后,all in,枕戈待旦,全力以赴地投入。好机会不会太多,不用太在意眼前的一点小小障碍。

\subsection{0930}

几点结论:
\begin{enumbox}
\item 人文以阳明心学为中心
\item 专业以分布式存储为中心、以全闪为破局点
\item 专业第一、人文第二,在专业中体现人文,精一之学
\item 建立根据地思维
\end{enumbox}

\section{10}

\subsection{03}

为学头脑处,此阳明念念不忘者。格物穷理,未免支离。头脑处在明明德,在心。龙场一悟,由外而转至于内。
精神之体相用,一而三、三而一,全体大用得以实现。

如何在心体上用功?在念头功夫。慎独之说,净念相继、都摄六根。正念是功夫、良知是本体。

先守住一部大学,体用兼备,兼采朱子阳明意,阳明为主,朱子为辅,尊德性而道问学。

\subsection{04}

理解阳明心学之真实义,及其演进脉络,首要在于切己体察,作为成长之一助力。以德性融摄知识,在诚意中格物。

熟读大学,以定其规模。大学格物致知,兼采朱子阳明,以阳明为主。三合之道,圆伊三点。

张学智在阐释阳明心学时,采用道德理性与知识理性一主一从、相辅相成的观点,深有启发。

为学日益,为道日损。道统摄学,达以简驭繁之效。

三五以变,为学处事的纲领。三摄太极两仪,五有空间时间。数年方法论探索的一综合结论。混沌大学的第一性原理,第二曲线,不连续性
也纳入这一体系内。三生万物。

\begin{enumbox}
\item 易经
\item 道德经
\item 大学
\item 中庸
\item 孙子兵法
\item 传习录
\item 画道精义
\item 一二三哲学
\item 原则
\item 用系统来工作
\item PDCA
\item 稻盛和夫
\end{enumbox}

读书诸原则:
\begin{enumbox}
\item 有的放矢,精读泛读相结合
\item 读原文、悟原理、知行合一
\end{enumbox}

\subsection{05}

从诚意去理解大学中庸,修身为本,则有下手处。喜怒哀乐之未发,谓之中;发而皆中节,谓之和。
中也者,天下之大本;和也者,天下之达道。致中和,天地位焉,万物育焉。

不反身,看不出一身毛病。

儒学,心学也。止于一为正,中和一是圣学根本。从内讲,至诚无息,纯亦不已。从外讲,尽性知天。
唯天下至诚,为能尽其性;能尽其性,则能尽人之性;能尽人之性,则能尽物之性。根源在诚意。

诚意与觉知、正念、冥想等略同,为明明德、致良知的功夫所在。

三五的中和一怎么理解?

\subsection{06}

悟后大有功夫在,专且有恒,不可泛滥无归。大学中庸,入道之书,当熟玩之,以奠定根本。

看三国电视剧,关羽、周瑜、杨修等人,皆以傲字取败。阳明国藩诲人,以傲字为第一凶德,可不警惕乎?
力去此病,劳谦、君子有终,吉。玩易既久,而不得真实受用,则与不读等,枉费精神而已,当思痛改之。

反身而诚,乐莫大焉。反之,反求诸己,不怨天不尤人,实为修身之首务。确立我是一切成败的根源,从而自强不息。

萧天石极言精功、内功之有益,宜重视。圣人定之以中正仁义而主静,立人极也。自然之道静,故天地万物生。
静而能生,宜深思。动有何敝?

专攻读一经,易经是也。易之妙,终身读之不能穷尽。大学中庸道德经等,皆易经之辅翼。太极两仪,此大学三纲领之义理结构,
从内修的角度去理解,修身为本,修身实为进德修业的根本。阳明龙场之悟,点出了一个重要的道路,突出了炼心,合心物内外,而成一元论。
由外而转向内,明明德、致良知皆是心性功夫,心性不废知识,相得益彰,一君二民,逡巡并进。

\subsection{07}

心道物三者循环往复,心的代表是稻盛和夫,道的代表是范蠡。道商,以道经商,是切合时代精神的选择。工匠精神、企业家精神有共通处,至诚之心,感天动地。
诚意是功夫头脑所在,论语与算盘,经济不仅是个人重要的一面,也是社会最重要的一面。致良知四合院是如何贯通两个领域的?企业家们学习致良知的价值有几何?

产业资本、金融资本之间的矛盾,金融资本得全球化之利,产业资本渐有全球化之害。

心性修养是一切的根源,内圣开出新外王,新时代,教育、科技、企业、资本是重点。

认知极重要,观三国演义、国共之争能强烈地感觉到,正确的见识有多么重要!寥寥数语,化腐朽为神奇。

国庆小结:

一、看电视剧三国演义、毛泽东,深知认知之重要,认知差极难弥补,超认知在事物的发展演进过程中,发挥着极大的设计塑造作用。
性格决定命运,人的事业发展的高度,视其性格即可见八九。由此可知,有无相生,无形的心性修养在事业中占据着极为重要的作用,
不可等闲视之。阳明心学融心学、知识修养与一炉,并非虚言。伟大人物如何看待一个事件,体现了其见识、洞察力。志、量、识不可或缺。

二、国庆之始,意在研读阳明学,中途多有变化,如看了萧天石、道商系列,沿着心道物的认知框架,逐步拓展。反求诸己是国庆最大的收获。
回归到正途,怨天尤人皆无用,风物长宜放眼量。不反身,看不出一身毛病;不反身,无以开发全部的潜能;不反身,就奠定不了以后发展的扎实基础。
行有不得,反求诸己,此君子之行也。不惟如此,一个团队、一个组织,也离不开沉下心来,如切如磋,十年一剑,磨炼内功。重视大学中庸,
如果仅仅在文字上打滚,也不会有大的收获。知行合一,方收大利。

三,既然立定了修身为本的纲领,诚意实为修身之要,慎独、主敬、求仁、习劳,是曾文正的心得。
不诚意,则旧习难改,因循度日,有恶不能改,有善不能迁,所失大矣。诚意是格物的主意,也是为学的头脑。按诸中庸之说,诚意之用,实为首要之责。
在心体上用功夫,则无支离无统之病。

四,重阳立教十五论有论学书一节,很得心学读书法之精要。心学的理念运用到读书、工作、生活中,当大有裨益。

五,道商学提出了六图思维模型。有极、无极、有无相生(太极、中极、真一),乃至大成。分四层。道常无为而无不为。有无之合,有中一至善之境。
真一图有近于圆点哲学。其中对三、五的解读,多有值得留意之处。四正的提法尤为精彩:智慧、生命、事业、兵法,合内外之道,由此可建立完善的知识体系。

六、关注到了八段锦,相比太极拳更为简易,工作闲暇之余,可玩味之,性命双修,养生之术,也当予以留意。神者生之本,另外一说也同样正确,好的身体
真的太重要。生命在于运动,生命也在于不运动(静)。静功性命双修,此南怀瑾、萧天石等前辈所明言。

\subsection{08}

地铁一路上看了张学智的几篇关于阳明学的论文,写的非常到位。德性与知识一主一从,相辅相成的观点,极有见地,令人醍醐灌顶之感。
国庆假期间读他的明代哲学史,粗粗翻阅,当进一步体味。论及熊十力的学术旨趣,承陆王而开一代新风,此中意味,尚不能体会。

道商学引入了六图思维模型,包括有极、无极、太极、中极、真一、大成等六图。
真一图与圆点哲学相似,不同之处是真一图是由有极无极二图融合而成。二宫尊德的一圆一元论,不及真一图深刻。

为学为道之分歧,至阳明而达到一大综合,双向流动,止于至善。熊十力更进一步,原儒里提及他的境论量论,乃本体论和知识论的结合。
量论有二:比量(观物正辞,穷神知化),证量(涵养性智)。可类比于hegel的感性、知性、理性。

心之力:直觉力、思维力,于念而离念。如此,不管身处何时何地,都可以进入学习状态。开发心,真是一剂灵药。在心体上用功,比在书本上用功,有效得多。
南怀瑾在太极拳与静坐一书,提及了剑仙的故事,王重阳在论学书,三国演义里诸葛亮初见周瑜的书房,发现没有一本书的对话。\hl{舍书探意采理}。
未有神仙不读书,不读书不行,迷失在书本里更不行。精神内敛,\hl{心生于物、死于物,机在目}。

缩小一下范围,本年度以传习录为重点读物,好好体会阳明彻底的一元论,道者含二之一。

心学逻辑学:
\begin{enumbox}
\item 求放心
\item 精神内敛
\item 先立其大者
\item 集义功夫、知行合一、体用不二
\item *
\item 为学头脑处
\item 诚意
\item 致良知
\item \hl{如何在心上用功}?静处涵养,事上磨练,心事合一。
\item *
\item 阳明学的生命学问
\item 阳明学的知识进度
\end{enumbox}

应用逻辑学
\begin{enumbox}
\item 易之辞曰初九潜龙勿用
\item 八段锦练起来!真是简易易学,贵在持之以恒。
\item 软件架构
\item 算法
\end{enumbox}

独立守神、抱圆守一。

以心学格心学

\subsection{09}

于念而离念,此语极精!念念相续,随即觉察,不被其牵引,超然独处。都摄六根净念相继,此说尚有勉强意思,于念而离念,则浑然天成。
盖诚意功夫深,则虚灵独耀,万象森然。怎么做功夫?

心-道-德-事业四部曲,有两种图形:四边形,菱形。含义大同小异。注意\hl{与正弦曲线的相似性}。

阴符经深刻,从心学理念读之,当更有收获。天地,万物之盗;万物,人之盗;人,万物之盗。
天地二分,则三合之道变成四合院:天地定位,左心右物(万物)。菱形结构由上下两个三角形组成,开拓出了更广阔的生存空间。
善守者,藏于九地之下;善攻者,动于九天之上,故能自保而全胜也。何为九天?何为九地?九地之下,九天之上,极言认知的沉潜与卓绝。

心生于物,死于物,机在目。心物关系,目为枢纽。目或者视觉,或指心眼慧眼,借我一双慧眼吧。鬼谷子有损兑,精神内敛,专以心眼察之。
塞其兑,闭其门,排除外界干扰诱惑。损外益内。

自然之道静,故天地万物生。静何以能生?心静则能照万物。

\subsection{10}

把个人与平台的关系理解明白,两者即独立又关联。个人要借助平台的力量,但不能依赖平台。

要直面最现实的问题,一切反求诸己。天地定位,修道保法,一心物,合内外。
心物之比例,开始阶段,心较小,须有意识地提起,待习熟后,则无往而非心也,凝而为一。
心外无物、心外无事、心外无理。心量广大,犹如虚空的境地。

在实现数据平衡的过程中,有赶工之嫌,心没有收到腔子里。阳明练习书法时,先静心,在心中练习,然后在纸上练习。
由此可见,做事情不能赶、胸有成竹,方可喷薄而出。

练拳也有这方面的说法,意走在前面,动作受意的导引。

比如练武,身体不动,冥想一招一式,达到非常成熟的地步。则功夫之提升,在不知不觉之中。

围绕诚意而展开,鬼谷子有实意一节,暗诵黄帝阴符经,也多有此类论述,如五贼在心,施行于天。

万化生乎身,不如万化生乎心。人心,机也。从信息和控制的角度去理解,从相盗的角度去理解,采天地之精华,为我所用。
精神收敛,胜物而不胜于物。一个开放系统,生命以负熵为食,与环境进行物质、能量、信息的交换。

\subsection{11}

纵横交错,纵是大学八条目,横是时间轴,过去、现在、未来。

假如创业又如何? 养生练习八段锦和因是子静坐法。

切割,犹如钻石,不是融合,而是打磨、重塑。

天生天杀,道之理也。

\hl{神仙抱道守一章,一是什么?心}!即阳明所谓致良知也,时时处处在心体上用功,
事业自然就在其中了,事业是心体发育的一个结果。心量渐次扩大其范围,涵容自然越来越丰富深刻。

太极两仪是天下最简单的道理,也是最深奥的道理。三角形是基因,或分形结构,自然构成层次结构的系统。

\subsection{15}

下一站,公有云大厂。公有云是趋势,私有云的空间慢慢被挤占,这是大趋势。
去大厂,可以专心做技术,两条线则更好。技术+管理。商业非己所长,可以学习。

小厂商做项目太费劲,要垫资,前期投入大。理想的情况是找到战略合作伙伴,各自发挥自身优势。

和君商学院有四大领域:国势、产业、管理、资本。道商提及四正:生命、事业、兵法、智慧。
构成生命之轮,投资、创业思维不能不有。投资是一个既简单又困难的事情,门欄不高,赚钱不易。
两条腿走路,技术(产业)和商业(资本、创业、投资)。

投资哲学、原则、体系等等,要有一致性,反复打磨。
\hl{PDCA的SDCA阶段固化}。在此基础上,进一步优化。

默而识之,慢慢修炼内功吧,要谁看?不鸣则已一鸣惊人,肉要丢在自己碗里好好吃,显摆出去,岂不愚蠢?

少说话比较好,不怨天不尤人。\hl{陶朱商经明奥理,鬼谷六韬藏玄机}。

预留的时间不会太长,需要抓紧一切时间进行准备工作。藏器于身、待时而动是明智理性的选择。

认知税、认知差、超认知,认知是个非常重要的问题,insight!

守破离,守住什么?

技术能力是底线、商业是上线(包括创业和投资)。往上提升,由实转虚。

皇极畴,五行畴,皇极畴在五行畴之中。

\subsection{16}

\hl{确立五行作为方法论的重要地位},是方法论研究的一个结论。点线面体、一二三四尽在其中。
五行系统有着悠久的传统,赋予其新意,与PDCA作为唯一的管理方法,立其环中,以应无穷。

五行生克,与因果循环图的增强连接和调节连接相似。

10X成长,需要在知识、方法、领导力诸方面持续进步。连接知识、领导力的是方法。
方法有心性修养,还有五行的思维模型。致良知、明明德为依归。

知识或技能是底线,不断提升,技进乎道。但如无道的指引,提升也是低效率的,
\hl{以神遇而不以目视},强调了心的统合和超越能力。

静坐、闭目养神、睡眠、八段锦作为涵养心性和养生的主要方式,一张一弛,一动一静,贵在持之以恒。

黄帝内经:夫五运阴阳者,天地之道也,万物之纲纪,变化之父母,生杀之本始,神明之府也,可不通乎?

文子:\hl{古之三皇,得道之统,立于中央,神与化游,以抚四方}。按左右前后四个方位,天地定位,构成六合之道。
木、火、金、水分别对应PDCA。P生发、进取,D全力以赴,热情似火,金收敛收获,水灵活调节。
此身立于中央之位,镇抚四方。(如此模型,也体现了双环互动)

五是两个阴阳(2D)正交的结果,交点居于中央。

全力以赴技术,仿佛将军枕戈待旦,致力于军事然。不擅长的事情不去想不去做了,积极打磨自己的核心优势,
\hl{绝利一源,用师十倍}。

方法和领导力,已有所了悟。然此时此地,不宜陷入其中,舍本逐末,忘了当前阶段的最重要任务。

\subsection{17}

黄帝系经典:黄帝阴符经、黄帝内经、黄帝四经,古朴、凝重。如何与大商之道相联通。
\hl{大商之道何处寻?阴阳五行贯通之}。

原来一直以来所感悟的,竟是阴阳五行学说。参伍以变,三合之道,圆点哲学,双线法则等等,
都能在阴阳五行体系下得到更好的理解。中国的术数学果然博大精深,辅助以科学的思维方式,将盛开美丽之花。

内经之中,智慧、生命、事业、兵法之道并存,一以贯之。

五行统一了所有的方法论,PDCA是唯一的管理方法。这些结论有待于进一步去实证。
阴阳五行思维,配合近代科学思维,则无往而不利。

进一步,致良知学说,明确了学习、悟道的一般程序。心居于五行之中央,道天地将法,以将喻心。

这种认识是否对学习专业技术有帮助?如何去运用这些理念去提升专业技能,更广泛地说,运用到一切重要的领域。

天有五贼,见之者昌。三盗既宜,三才既安。此可见五行之重要。孙正义二十五字诀,重要!

怎么去实证?验证一下这些思想的威力?

全力以赴技术,不再左右摇摆。绝利一源,用师十倍。三反昼夜,用师万倍。
把这几年学到的方法论,运用到接下来要做的事情上。务必去争取胜利。

多元思维模型

方法论小结:
\begin{enumbox}
\item 以心道物三合结构为致思的起点
\item PDCA
\item 学、问、思、辩、行(辩可以理解为刻意练习)
\item 行动学习
\item 双环学习
\item 阴阳五行
\end{enumbox}

反思以往,总是一心多用,得陇望蜀,身在曹营心在汉,不能全心全意地投入。结果也就很尴尬。

次则纠结于方法论的问题,不能准确理解格物致知的真义,方法论必须简单、一致

\subsection{18}

致良知-三合-PDCA,诠释了阴阳五行。人生算法提及4+2的结构,一个核心、一个外环,再加上认知闭环。
内圣是致良知,外王是PDCA,三合嫁接内圣外王。质能方程E=MC\^2,长长的坡道、厚厚的雪。

一个核心、一个外环的结构看起来有点怪,以致良知三合结构代替之。

认知闭环的感知、认知、决策构成PDCA的规划阶段,细分为几个子阶段。
但是没有把检查、调整作为独立的阶段,在现代科学的认知框架下,基于目标的检查、调节是非常重要的。
见系统论、信息论、控制论。

整个模型可以看作是阴阳五行模型的一个应用。阴阳五行模型是hegel所谓的逻辑学,\hl{承体启用},衍生出形形色色的应用逻辑学。
数学也有纯粹数学与应用数学之分。

抱一执中,守一守中、守中致和之说,在该模型里就是自然而然的事情了。

怎么用此模型去分析问题呢?比如阅读、学习、提升专业技能、商业等等,如何心想事成?如何得偿所愿?

黄帝阴符经的观天之道,天之道,如果理解为阴阳五行呢?

心生于物、死于物、机在于目。目者,深层理解即是认知。致吾心之良知,体天地之大道,则心物皆活。

算法,就是道、天理、良知等,内核有两层含义,一是指成长算法本身,二是指现实中要从事的事业、开发的产品。
外环呢?不如用三合之道去诠释内核和外环的含义。就要滚雪球下坡,又要推石头上山,两者是一个统一的过程。
上山、下山;下海、上海构成两个三角形的菱形结构。

自信找到了成长的基因、基本单元,就是阴阳五行的一个应用。由致良知、到三合、到PDCA,构成成长的五行模型。
确立内核具有里程碑的作用,如果没有得到这个一,劳动的价值就不能得到很好的放大,
所以第一要解决这个有没有的问题,好不好,实不实、快不快是第二层面的问题。

我们不能期望自己无所不能,认清能力半径,在能力半径内精耕细作。

大商之道何处寻?阴阳五行贯通之。于关键处有所领悟,是一大进步。
悟后起修,还需再接再厉,沉潜往复,乘胜追击,直挂云帆济沧海。

慈俭静和。

\subsection{19}

河出图、洛出书、圣人则之。河图洛书简称图书,先天而天弗违,后天而奉天时,河图代表先天之体,洛书代表后天之用。

五德、五贼之说,精当!喜怒哀乐之未发谓之中,发而皆中节谓之和。喜怒哀乐欲、仁义礼智信,构成万字符,拨乱反正。
此诚意慎独之功也!

五十以学易,可以无大过矣。\hl{五十代表河洛之中数},五显十藏,显诸仁藏诸用。

以中极图与三合之道互参。一升一降,上山下山,天地乾坤之间,生死轮转,往来无穷。

河图,经天纬地,立体河图显示为双环结构。

周末仔细思考人生算法,用阴阳五行、河图洛书构建起\hl{人生算法的第一版},然后逐步迭代。

\subsection{22}

以来知德太极图为基础模型,融合河洛、阴阳五行,开出新境界。

中级图内有一圆,无极而太极。外有一圆,阴阳一气升降往来于其间。

立天之道,以定人焉。唐诗太极图书心性学,以河洛之数理逻辑,演绎儒道心性之学,此书是继一二三哲学之后,另一极富冲击力之作品。

诠释的重要典籍:
\begin{enumbox}
\item *
\item 太极图说
\item 阴符经
\item 尚书洪范篇
\item 心经
\item *
\item 易传
\item 中庸
\item 孟子
\item 论语
\item 大学
\item 老庄
\item *
\item 刘一明道解周易
\item 周易函书
\item 杭辛斋易学七种
\item 七纬
\item 王船山
\item *
\item 圆运动的古中医学
\end{enumbox}

对周敦颐的太极图说的改造,点出了静的重要性。推重王船山,乾坤并建、成己成物并重的理念。
轮转生死之道,逆而行之,开出天地两级,三才运乎其中。河洛的价值于是得以彰显。

早上阅读微信订阅号,言及犹太人的思维,大卫之星、五行黄金教育法,\hl{犹太与中华}互有可以借鉴之处。

中心一圆,本体也,根据地也,进可攻退可守。犹如黑洞,具有超高质量,规定了演化的次序。

乾乾之变,从两个角度看中年危机。

执大象、天下往。中级图、真一图即是这样的大象。

主静、一源。

彭子益的圆运动的古中医学认为,中医之根在河图。河图天地定位,左旋右转,一升一降。
中气如轴,四维如轮。与天书解密的双环模型有相似之处(立体河图)。\hl{PDCA结构与此暗合}。

道心物的三合结构,可以看做是中极图的简化版。此模型易知简能,当珍爱之,由此开启一切方法论。
以图法解读传统文化典籍,实为切要。易图当在易文字之先。
大卫之星也可看着两个三合之道组合而成,终究是太极图的简化版。由此图,融合中华犹太文明于一体。

再来看人生算法的论述,确实有其精要处。为什么要建立内核?事业就是最大化内核,由最小化的内核展开。
为政以德、譬如北辰、居其所而众星拱之。或有根据地之说,都是此意,开启一本万化生生之途也。

守中而运营四方。

李善友的混沌大学课程,如何通过中极图加以描述?第一曲线、第二曲线、用第一性原理跨越不连续性!

\subsection{23}

人生算法,如何理解、践行、实证(信解行证)?好的理念,不必求多,大道至简。
信解行证分别对应木火金水五行方位。

养生禅武并重,以静坐、八段锦、八部金刚功、太极十三式为主体,穿插在每日的工作生活之中。

以大学之道,统领一切学与思。大学之道,是同心圆结构,内圣外王之道。
欲理解大学,不能不广泛地理解儒释道的精髓所在,进而落到修身为本这个最牢固的基点上。
修身不仅包括诚意正心的心性功夫,也有格物致知的专业素养。德性之知和闻见之知交相互养,相资为功。
成己成物,修己安人之道备矣。

有此图示,再去读儒道佛的文献,就易于理解了。庞朴以一分为三为打开传统文化宝藏的金钥匙,现在看来,不够准确。
河图洛书的图书体系,才是那把金钥匙。两者相同的地方,在于对中的理解。但是,河洛体系有着远为丰富的内涵。

制定一长期的人生算法,投入执行。其基因就是中极图,或曰来知德的太极图。
这是一个根本性的原理的图示。三合之道、大卫之星、第一性原理、在河洛的数理模型中,可得到简洁生动的呈现。

商道的六图思维,中心是中极图,中极乃至大成。中极图是圆点哲学的发展,
内在地包含了单点突破、双线法则、三才之道、四象限、圆点哲学,以及阴阳五行。

内核不必大,把分布式块做好,就可以纵横四海了。何必求多?

需要学习的知识
\begin{enumbox}
\item RDMA
\item SPDK
\end{enumbox}

\subsection{24}

黄宾虹:\hl{太极图是书画秘诀}。此语精彩,近日读书,其中启发最大者,是一些研究国画、美术的人的著作。
如杨成寅的太极哲学、毕建勋的一二三哲学、还有唐诗的太极图书心性学。盖书画所示的是象,若切入几何、义理的探索,
则象数义理相互启发。易传:圣人立象以尽意,设卦以尽情伪,系辞焉以尽其言。
易者,象也。老子言:执大象,天下往。玄牝之门,是为天地根,绵绵若存,用之不尽。
太极图就是这样的玄牝之门、天地之根、大象所寓焉。

神圣几何,几何是上帝创造世界的语言,从欧几里得的几何原本、笛卡尔的解析几何,计算几何?

不能停留在艺术的表述形式,调用自然科学的原理去诠释。

用几个经典的物理学公式: 万有引力定律,麦克斯韦方程(电磁)。
质能方程,熵,从经典物理学到量子物理学。

大卫之星都是简化形式,更容易掌握,但有所失真,包容不足。就行牛顿力学是相对论的简化形式一样,在一定范围是简洁有效的,超出该范围就会有较大的误差。
太极图则是最后的形式,有待逼近的真理。以圆点哲学为例,是接近太极图的形式,突出了两个维度。又如双线法则、三才之道的层级化的简单化的理解。
通过强化了某些要素,而达到鲜明的理论特色。

黄帝四经的执一明三定二,很有兴趣。

淮南子本经训:帝者本太一,王者法阴阳,霸者则四时,君者用六律。太极图就可称为太一,怎么理解气?气场是一种无形而有用的真实力量。
气韵生动,脉络井然,通过心眼察之。

黄宾虹值得好好研究,对太极有独到精深见解。

修身、工作等等活动可以统一了,天人合一,合一到太极模型上。物物一太极,理一分殊之说,宜深悟自得。

通过太极模型看世界,论语也好、道德经也好、易经也好,核心理念都是太极。
特别是来知德太极图,像极了银河系的星云图。

如王重阳所说:读书在采意,得意而舍书可也。如此,一切活动无法培养太极之心体。前一段穷究阳明的良知说,体会不深。
如进一步引入太极,则致吾心之太极,则事事物物皆得其理。阴符经:五贼在心,施行于天。宇宙在乎手,万化生乎身。
立天之道,以定人焉。这个天之道,即是由河图一路演化至今的太极也,如长江黄河,浩浩荡荡,奔流至海。千流万径尽入其中。

太极图不仅是书画秘诀,乃是天地之大道,万物之纲纪。

确立我\hl{太极一元论}的根本指导思想。

清单、方法论、原则、算法等等,从太极中涌出,约法三章的智慧,化繁为简,主宰自己的成长。

\subsection{25}

太极观天下,物物一太极,一本和生生,一体万化,理一分殊等等。

三合之道是太极的简化版,体现了升维、降维的观点。如把太极的圆心提起,则成圆锥,圆锥的2D投影,即是三角形。
一个是向上,一个是向内。向上即向内,向内即向上。透过简化太极,可以衍生出种种思维模型,凸显重要的理念。

太极乃圆,但不能立极,随便滑落下去,实在可惜。如昼夜之道,一日清晨难得,若浪费就实在太可惜。放大到一年一生,
也是如此。气韵、节律、节奏都是非常重要的理念。看不到、摸不着,但有生于无,作用却巨大。

信解行证,信则信矣,宜深信,由此深信,发起大愿,孜孜以求,解行证自在其中。

庄子天下篇评老子之道,建之以常无有,主之以太一。帝者体太一,此中境界极深远,不可轻忽。

身处HY,亦觉大痛苦,现实与理想有渐远之感?如何破局,惟在沉潜一志,修身为本,藏器于身,待时而动也。

反身而诚,乐莫大焉。行有不得、反求诸己。此逆转超越之道,四通八达,撑起朗朗乾坤。

把天道理解为发源于河图的太极之道、阴阳五行之道,是否解得通?试从阴符经而论之。

多言数穷,不如守中。事物的发展,沿着环形的轨迹。有无相生,难易相成,难的走着走着就变容易了,易的呢,走着走着就不易了。
曲则全,其次致曲、曲能有成,曲成万物而不遗。

\subsection{26}

读熊春锦的东方治理学和要略,心中踊跃,大为感动,明道授业解惑也。书中大为推重黄帝四经,有启蒙之功。
往日读书,不求甚解,故少受用处。近日甚爱中极图,也意识到通用太极图与中极图有所不同。
书中称中极图为旋极图,并指出太极图确实旋极图的中心能量轴,实能解心中之惑。

黄帝四经中的天执一、明三、定二、建八正、行七法,为治理学之纲领。还需要进一步体悟。
总之,熊师之思想,实能中我心,而须进一步加以探明体悟者也。\hl{旋极图具有巨大的能量,有待于去开发}。

万物负阴而抱阳,中气以为和。阴阳静和之美,于斯见之。圆运动的古中医学,亦有此论。
三论不能尽旋极图之至深奥义,确是其中至关重要的一个环节。

\hl{五行之惑}:两种排列方式如何通约?土居中,与土在环上。

物相与质象二境的分判,慧识性识与智识意识的区别,都发人深思。质象境,有质而无形,须以心眼察之,此大智慧人之洞见。
如中医经络、中气之说,皆可实证而确定不易。但以普通之知识度量之,则显得玄远迂阔。再如市场,其形态岂可尽以计算为之。
以静定心,深入质象境,乃能观其运行之大略,了了分明。

万法唯识所见,唯识学宜深究。

甚至运营一公司,架构一系统,不能不有质象之洞见,若执于物相,心为之凝滞,处处羁绊,难以神与化游,应于四方。
故金刚经云:应无所住而生其心,庄子云:故能胜物而不伤。需要注意的是,此不住之心,与执一守中不是一回事。
切勿理解为任意漂流,茫然无所归处。此不住之心,乃建立在守中执一的基础之上。
没有了守中执一的坚定信德,无所住则成空谈,流于缥缈之域,无所措其手足也。
\hl{随便滑落过去,沉沦生死苦海,蹉跎青春年华},如何能得一、德一。
故与佛家性空之见,不容不分辨。旋极图中心一圆,乃有无动静内外之分际,太极无极之玄机,当明辨之。

不能执一明三,则进入二元论矣。是非善恶动静有无,无从分辨。

五行又分阴阳,这是一伟大之架构,初见之于太极图书心性学,不期很快又见之于东方治理学,实为幸事。
太极图说之演化序列,二五之精,二中有五,五中复有二。

孙子计篇是SWOT分析的起点:经之以五事,校之以计,而索其情。天地者,机遇也。
道将法是针对此机遇而展开的战略战术活动。扩展为孙正义的二十五字诀,一流攻守群是战略选择。

顶情略其斗则是决策过程,

智信仁勇严的顺序至关重要,只是严,而没有前面的铺垫,必然行而不远。

风林火山海对战术的刻画极生动:其疾如风,其徐如林,侵略如火,不动如山,难知如阴,动如雷霆。
时机、节奏的把握是将略的直接体现。海者,言其大也。

参照波特的竞争战略(成本、差异化、集中)。法尔科尼的管理学(领导力,方法、知识),
威胁象限可用同心圆模型,波特的五力分析。

\subsection{29}

读熊春锦系列书,很是振奋。特别是提出的一系列很有启发的观点,
如旋极图、一元四素、帛书五行四个方法论、三因说、三元/三源说。
以旋极图为根本大象,研究黄帝四经、老子的关键命题,进而提出身国同治的德道之学。

一切都是河图演绎的结果。河图到旋极图,从旋极图去理解德道经和中庸的核心思想,极为生动形象。
道商六图中的四个对应到熊春锦的中心思想上,诠释上有同有异。

教育:开慧益智。以德一为统帅,整合慧智,慧是先天性识、智是后天意识。

治理:内圣外王

三境三元三因。质象境,气、光、音。须以几析法研究之,极深而研几。

一元四素四析,构成完整的方法论体系。一元,道与德,道生一、一生二、二生三、三生万物,万物负阴而抱阳,中气以为和。
二是事物的属性,一和三才是根据和动力所在。象数理气四法相互涵摄,层层深入。

为什么以黄帝四经的执一体系为治理学纲要?

修之身,其德乃真。修身为本,这个是根本。只是谈谈,满足理论上的兴趣,没多大意义。
修身是综合性的,包括进德修业两大方面。从修身的角度去理解传统文化,一元四素这个方法论。
一元四素是综合道德经、易经而来。黄帝四经、大学、中庸强化了某些主题。

说话太多,过于急切。当更沉静,勿躁动。中气以为和,守中致和。
中和之德,为最上境界。人物志论人之材质,中和最为贵矣。
中和者,精神内敛,中通外直,不蔓不枝。

心思不能宁静,总是有所待,不能真正地向内用功。内求法合外内之道。



\end{document}

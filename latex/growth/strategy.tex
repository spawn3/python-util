\chapter{战略}

专业学习之外,把战略研究列为今后几年的重点。


为什么?

因为战略很重要,战略赢是大赢,战略输是大输。孙子:兵者,国之大事,死生之地,存亡之道,不可不察也。
不具备强大的战略思维能力,就很难实现远期的发展目标。

多聚变为一,一裂变为多,分合为变。以正治国,以奇用兵,以无事取天下。战略明,则可无事。否则,陷入事务之中,而无结果,可悲。

欲研习战略,需读经典,多实践,请教高人,开阔视野,放大格局。在解决现实问题中,融会贯通,知行合一。持续优化,不断把认知引向深入。
道德经,孙子兵法,商君书等,皆为经典。西方亦有经典,然不够精炼。待心有所主,则可进一步泛观博览。第一步,则在知止,懂取舍,有所不为。

战略罗盘之喻,精当。战略几何学,形象生动。柏拉图言:不懂几何者不得入内。以几何去研习战略,可收简洁精当,生动形象之效。

专业学习和战略研究,可谓一文一武,一阴一阳,一张一弛,相互促进,相得益彰。
战略研究要有更多的问题意识和进取精神,不仅仅是知识的获取,而为我所用,服务于最终目标的实现。
战略研究的境界,可用中庸的致广大而尽精微,极高明而道中庸形容之。
层次则有历史,科学,艺术,哲学。

\section{问题}

大学:物有本末,事有终始,知所先后,则近道矣。认识事物的轻重缓急,按其行动,就接近道了。按重要性和必要性维度进行分解,是柯维几本书的一个重要内容。
李笑来在财富自由之路中,一语中的,作为终极问题。对该问题的反复审问,是磨练价值观的利器,有助于提高选择和决策能力。别的问题都是术,这个问题近乎道。

所谓选择,即是增加必要的条件。尽量必要,尽量充分,最小完备集,奥卡姆剃刀原则在决策问题上的应用。李笑来所说的万能钥匙,即是NLP的换框法,转换视角。

\section{维度}

何为维度?升维思考,降维攻击。把重要维度都列出来,从中选择优势维度,扬长避短,有所取舍,特色组合,从而构建核心竞争力。
价值链分析如此,蓝海战略也如此。互联网,成本等都可以成为分析的重要维度,如差异化竞争,互联网对+,免费等常用的竞争策略。

维度,或者说条件,要素,李笑来有个认识:增加条件。

整合,跨界,爆裂,裂变都是这个核心思想的变形。借助技术手段,多维整合为一维,或一维细分为多维。
技术,市场和自然所谓3M力量,在塑造未来世界。

升维思考,降维贯通。维度一概念,用代数方式研究,线性空间。在数学和战略研究之间,架起会通的桥梁。数学和战略之间,有着深刻的联系。
构造结构,识别模式和关系,掌握变化的趋势。

蓝海战略的价值链分析,波特的五力分析着眼于产业竞争分析。价值链分析通过加减乘除来构建自己的优势维度,有所为有所不为。

维度是立体的,层次的,从分形的角度看,不仅仅有整数维,也有分数维,涌现出奇异的系统特性。

当代科学进展,描述了令人惊奇的时空结构。按照量子力学和相对论的认识,物质,能量,时空都呈现了超出直觉的功能和特性。
对微观粒子结构的认识在深入,对宏观时空结构的认识也在深入,在极小和极大的时空尺度上,都存在一些真正的大问题。

所谓认知,就是对维度的认识。志如其量,量如其识,三位一体,相互影响。量,开放心态,识,见识,认知水平。

\section{老子之道}

战略研究有层次境界之别,重点是理解并运用老子的一句话:道生一,一生二,二生三,三生万物。
其中一是重点的重点。侯王得一以为天下正,得一,则万事毕。

一生二,因二以济民行。一个模型是双线法则,守正出奇。守底线,抓关键。修道保法,故能为胜败之政。

老子之道,博大精深,内涵基本原则。马利克的管理,植根于系统论、控制论和仿生学等现代科学,耗散结构是另一值得注意的。
道生一,一即战略;精心守一,参悟商道。以道为中心,通达战略和管理,促进成长和进化。

半部老子治天下,围绕老子,通达道家思想。没有厚此薄彼,抱一而为天下式。老子文约义丰,且较为熟悉。

为自己打造一口深井,由老子承载大学之道,修齐治平,尽在其中。

从认知升级的维度读老庄,会有趣得多。

\section{圆点战略}

圆是格局,点是破局。圆是调查分析,点是指导方针和行动序列。

\section{双线法则}

\section{西方战略管理思想}

战略始于问题。战略七问,德鲁克五问,如何回答这些问题需要深入分析。黄金圈法则,明确了问题的顺序。

受到商业机构成功的启发,比尔·盖茨在今年的公开信中提出了一套“成功法则”:量化目标―选择策略―考量结果―调整策略―实现目标。
比尔·盖茨认为,这不仅仅是商业机构成功的秘诀,致力于扶贫帮困解决社会问题的非营利机构同样应遵循这一法则。

PDCA是唯一的管理方法,法尔科尼管理方法。

双环学习,前提批判,在表面原因之外,有更深层的原因或约束。
在尝试解决问题之前,需要更深入的调查分析,找到问题的根本原因,而不是流于表面,治标不治本。这引向了系统动力学的视野。

战略是可行性的假设,需要持续的压力测试,来检验其正确性和有效性。

机器的隐喻,有机体的正常运转。可以把组织看作一台机器或有机体,机器是在正常运转吗?
用控制论去理解,控制论适用于机器和生命领域。

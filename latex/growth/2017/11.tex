\chapter{2017}

\section{11月}

\subsection{1127}

采用笔号:渊墨子。

毕达哥拉斯说:万物皆数。渊墨子说:万物皆machine。

物物一太极。这种古老的思维模型和现代机器思维的结合点。

为维护一机器良好运行,需要做什么?更重要的是什么?最重要的是什么?

软件架构和系统论,控制论,进化论,仿生学的融汇。

用系统来工作的主体思路是:战略/方针/流程,事事物物是由大大小小的系统构成。
如此,就可以采用系统科学的成果来解决现实问题。

从最少必要知识出发,用自己的话说出来。

唯一任务就是保证机器运转良好,佛挡杀佛,魔挡杀魔,落实到人和事。
以正确的方法做正确的事。简政精兵,一是事,二是人。
于此基础上,不能不讲求价值观,文化,方法论,设计。

TOC瓶颈理论

\subsection{1128}

渊,深而广。墨子,行者。

不抱怨!

走出大山,拥抱新世界。

\begin{shadequote}

慎独则心安

主敬则身强

求仁则人悦

习劳则神钦

厘清、辨理、探索可能性,思考之门尽在此。—— 李天命
\end{shadequote}

居安思危,决策须基于最坏的情况,做最大的努力。凡事预则立,不预则废。

当下重要的,一是道,二是术。

道类似于达里奥的原则,简单,普适,具体而微,致广大而尽精微。

术,术业有专攻。志于道,据于德,依于仁,游于艺。
老子云:见素抱朴,少私寡欲,绝学无忧。重点在于绝学无忧。

身语意,三业清净,从语开始修养,意则是根本。两舌、恶口、绮语、妄语。从否定的方面说。

如何对待错误,从错误中学习,检验我们的心理韧性和成长速度。
颜子不迁怒,不二过,成就复圣。

为什么会犯错?如果错误已成事实,当如何明智面对?

\subsubsection{费曼}

费曼的学习方法,有助于把思维从混沌的迷宫里,及时拯救出来。
而不是陷入细枝末节之中,无力自拔。

\subsubsection{达里奥}

加深对达里奥思想的理解,内化为己有。

求真是第一位的,不诚无物。求真,必有开放心态,进取,开放,灰度,
挑战已有认识,随时准备接受更好的观念。认识是一个流动性的过程,这是黑格尔的知性和理想的分别。

在追求目标的路上,错误在所难免,人非全知全能,求真也包括了直面错误和弱点。
直面,不意味着放任,而是更积极地寻求成长破局之道。

真是达里奥的核心概念,就如诚是中庸一书的核心范畴一样。
达里奥令人惊喜之处,在于再次唤醒了对原则的重视,并决心一究到底,为我所用。
对一心二门的颖悟,对控制论和机器思维的一以贯之,都说明了达里奥思想的理论力量。

达里奥所探究的是自然法则,由天道达于人事,道法自然。

毕达哥拉斯,物物一太极,万物皆机器。综合起来,象数义理兼备。
而机器思维,具有很强的可操作性,三种认识,统一到万物皆机器。

\subsubsection{TOC}

逻辑的力量

最需要做到位的,就是熟知的东西。比如早起,比如不抱怨,比如复盘,认真对待自己的目标等等。
世上没有万能药,在大格局下点点滴滴积累,才能成就,千里之行始于足下。

克己复礼,默默完成进化:否定的方面:不抱怨。肯定的方面:专注工作,明确目标,关注方法论。克服了否定的方面,
掌握了肯定的方面,然后就是长期的专注,持之以恒,终日乾乾,至诚不息。

持志如心痛,哪有功夫说闲话?

\subsection{1129}

简单是复杂系统的根本,模型化是认识系统的方法,最好用的模型本质上都是最简单的。
比如圆点哲学,太极周期,控制论,机器思维等。太极统摄象(机器),数,理。

从太极,或矛盾法则去理解达里奥的模型,真假,对错,是非是最核心的矛盾关系。
以真假为例,正确,或错误都是正常的,关键在于态度和选择。
每一个选择,都带来多级效应,不能不慎重以待。
这是真的吗?求真的问题意识贯穿一切思想和行为。

另一对矛盾关系是目标和产出,两者的不平衡是machine运转不灵的征兆。
two yous,设计/文化和人都是富有张力的结构。

自我进化五阶段是目标和问题导向的,围绕着目标和达成目标面临的问题和障碍而依次展开,
贯穿机器运转的全过程。

如此去解读达里奥的思想,就能把握其精髓,得心应手,意在笔先。
默而识之,不言而信,存乎德行。德在内,行在外,诚于中,形于外。

得其意,忘其言,用自己的话说出。

当前唯一的焦点,是学习和成长。列出明确的学习目标和计划,用达里奥的方法检验效果。

学习方面,大体分为几个方面:取势,明道,优术。

\subsection{1130}

渐渐能体会到哲学/思想的力量,苏格拉底,柏拉图,亚里士多德等等,
都有深入理解的必要,安排时间系统地学习一番,一体而多元,多样性的统一。

张世英提及,在认识过程中,天人二分,乃不可少的阶段。是为卓见。
不经历分解和还原论的阶段,无法深入到原子,分子,基因的层次,
无法提取组成复杂系统的简单形式,法则,模式。
简单是复杂的根源,学习的艺术最重要的就是该观点,
以及怎么运用到高效学习的运用,重视基础,重视组块。

大乘,从此案到彼岸,从现实到梦想,从目标到产出(逆向),起信,四信五行,
立义分则在一心二门三大。

乘,犹如渡船,或machine,借以渡河。河如横亘在面前的问题和障碍,
逢山开路,遇水搭桥,排除万难,一往无前,所向披靡。

大乘,大的工具(新工具,方法谈,知性改进论),要寻找这样的工具,并践行之。

一心二门的架构,意义深远。窄门宽门,乃是一门。马利克的管理控制论指出,
运行的两个层次:运营和管理。运营如肢体,管理如中枢神经系统。
两个层次有严格的分界线,不可混淆(混为一谈),但要合一。
运行是管理的本质,运行是比企业管理更为普适的范畴。
这就像hegel的逻辑学,贯穿自然哲学和精神哲学,或者如柏拉图的理式。
道器不一不二。不一,所以要分;不二,所以要合,分合为变。

定制的衣服更合身。可以学习别人总结的原则,更重要的是提炼出自己的原则,内化于心。

是时候了,开启财务自由的旅程。

控制论,人工智能,机器学习,控制论不仅仅是科学,也可以理解为更普遍的技能和思想方法。

\chapter{2017}

\section{11月}

\subsection{1127}

采用笔号:渊墨子。

毕达哥拉斯说:万物皆数。渊墨子说:万物皆machine。

物物一太极。这种古老的思维模型和现代机器思维的结合点。

为维护一机器良好运行,需要做什么?更重要的是什么?最重要的是什么?

软件架构和系统论,控制论,进化论,仿生学的融汇。

用系统来工作的主体思路是:战略/方针/流程,事事物物是由大大小小的系统构成。
如此,就可以采用系统科学的成果来解决现实问题。

从最少必要知识出发,用自己的话说出来。

唯一任务就是保证机器运转良好,佛挡杀佛,魔挡杀魔,落实到人和事。
以正确的方法做正确的事。简政精兵,一是事,二是人。
于此基础上,不能不讲求价值观,文化,方法论,设计。

TOC瓶颈理论

\subsection{1128}

渊,深而广。墨子,行者。

不抱怨!

走出大山,拥抱新世界。

\begin{shadequote}

慎独则心安

主敬则身强

求仁则人悦

习劳则神钦

厘清、辨理、探索可能性,思考之门尽在此。—— 李天命
\end{shadequote}

居安思危,决策须基于最坏的情况,做最大的努力。凡事预则立,不预则废。

当下重要的,一是道,二是术。

道类似于达里奥的原则,简单,普适,具体而微,致广大而尽精微。

术,术业有专攻。志于道,据于德,依于仁,游于艺。
老子云:见素抱朴,少私寡欲,绝学无忧。重点在于绝学无忧。

身语意,三业清净,从语开始修养,意则是根本。两舌、恶口、绮语、妄语。从否定的方面说。

如何对待错误,从错误中学习,检验我们的心理韧性和成长速度。
颜子不迁怒,不二过,成就复圣。

为什么会犯错?如果错误已成事实,当如何明智面对?

\subsubsection{费曼}

费曼的学习方法,有助于把思维从混沌的迷宫里,及时拯救出来。
而不是陷入细枝末节之中,无力自拔。

\subsubsection{达里奥}

加深对达里奥思想的理解,内化为己有。

求真是第一位的,不诚无物。求真,必有开放心态,进取,开放,灰度,
挑战已有认识,随时准备接受更好的观念。认识是一个流动性的过程,这是黑格尔的知性和理想的分别。

在追求目标的路上,错误在所难免,人非全知全能,求真也包括了直面错误和弱点。
直面,不意味着放任,而是更积极地寻求成长破局之道。

真是达里奥的核心概念,就如诚是中庸一书的核心范畴一样。
达里奥令人惊喜之处,在于再次唤醒了对原则的重视,并决心一究到底,为我所用。
对一心二门的颖悟,对控制论和机器思维的一以贯之,都说明了达里奥思想的理论力量。

达里奥所探究的是自然法则,由天道达于人事,道法自然。

毕达哥拉斯,物物一太极,万物皆机器。综合起来,象数义理兼备。
而机器思维,具有很强的可操作性,三种认识,统一到万物皆机器。

\subsubsection{TOC}

逻辑的力量

最需要做到位的,就是熟知的东西。比如早起,比如不抱怨,比如复盘,认真对待自己的目标等等。
世上没有万能药,在大格局下点点滴滴积累,才能成就,千里之行始于足下。

克己复礼,默默完成进化:否定的方面:不抱怨。肯定的方面:专注工作,明确目标,关注方法论。克服了否定的方面,
掌握了肯定的方面,然后就是长期的专注,持之以恒,终日乾乾,至诚不息。

持志如心痛,哪有功夫说闲话?

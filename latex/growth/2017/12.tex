% \chapter{2017}

\section{12月}

\subsection{1201}

转眼己是十二月,真是时光飞逝,越发突显了专注和把握重点的重要性。

记录,不能仅仅感性思考,更要把思考过程和结论记录下来。
只有记录下来,才能持续的改进提高。这是语言和思想的力量。
从这个角度来看,禅宗的不立文字不足为凭。
语言是淬炼思想的熔炉,修辞立其诚。

\begin{shadequote}

圣人立象以尽意,

设卦以尽情伪,

系辞焉以尽其言,

变而通之以尽利,

鼓之舞之以尽神。
\end{shadequote}

系统的模型化,取象丰富多端,模型则简单纯粹,一画开天地。
象数,义理各得其所,神而明之,存乎其人。

原则,用系统来工作是系统论的应用(包括控制论,信息论)。
大系统内嵌小系统,环环相扣,脉络井然。

诚,认真的创造性。

\subsection{1205}

诵读黄帝阴符经,若有所得,中庸,黄帝阴符经,大乘起信论,达里奥的原则诸书,
可作为今后一段时间的行为指南。在现实和理想之间,架起友谊的桥梁。

人心,机也。宇宙在乎手,万化生乎身。身心,切要。
手是实践,心是思想,意志和自由。

理论是行动的秘诀。此言深得我心。

阴符经已会背,更可用心玩味,与生活经验和目标相结合。
诫子书,太极图说,心要法门,都是值得细细咀嚼的精品。千古大道陆沉,只缘误解太极。
学问的高境界,在一以贯之。为学日益,为道日损。
下学上达,渐渐臻于融会贯通。

心迷法华转,心悟转法华。读书不可太多,心为之所夺。虚心涵泳,切己体察,最为读书者之心得。

坚持写作。写比读更重要,输出倒逼输入。写具有多项功能,太初有言,太初有为。
禅宗贬低文字,不妥。只有记录下来,才能形成一个可进化的体系,语言具有独立性。(波普尔的第三世界)

自主性,主体意识

编程,架构,写作,教学,演讲,管理这些活动具有统一性,可以互相促进,形成成长的多元效应。
机器和语言,是今后很长时间的主题。

达里奥的原则,构成他的机器的语言。言之道,岂能不认真讲究?

一事精致,足以动人。见素抱朴,绝学无忧。

\subsection{1206}

诚,涵摄绝利一源,实事求是之意。求真,求精。

常识,大逻辑,活力,灰度。重点在大逻辑和活力,常识和灰度是对大逻辑的必要辅助。
于此,可见柏拉图和黑格尔的深刻,更可见老子道论的深刻。朴,虽小,而天下莫能臣。
侯王若能守之,万物将自化。

会通古今中外。中西思想有大不同,但在更高的层次上,则具有根本的一致性。
以道观之,物无贵贱。CEO所要把握的,会是比具体业务形态更抽象的东西。
在经营上,各有各的特色;在管理上,百虑一致,殊途同归。道和器,体和相。

笛卡尔的方法,斯宾诺莎的伦理学,知性改进论,达里奥的原则,查理芒格的模型,
PDCA,控制论,TOC等都体系了欧几里得几何的精髓。

玄科之辨,有意义,唯求其是。

鬼谷子阴符经,可读。暗合天地之理。两篇阴符在握,何愁大业不成?
阴符,可通行无碍,乃权利之信物。利器,不可轻易示人。

\subsection{1206}

觉知当下,不要被思维牵着走。用志不分,乃凝于神。

尽心践行达里奥的原则,提炼出自己的原则,作为开展事业的基石。多并没有用,少即是多。

曾国藩的日课,富兰克林的修身十三条,是粗线条的原则,需要细化,充实其内容。
觉得一个理念好,并不管用,如果不能为我所用,内化为达成目标之路上的阻力,
就纯然是浪费,暴殄天物。

费曼学习法特别有益,特别是对喜欢陷入概念之中的人来说,更是如此。
更进一步,去理解事物的真相和本质,到底是什么?如何为我所用?

检讨过往的生活学习工作之路,无疑是走了很多弯路,需要正视,从中吸取经验和教训,
提炼出原则,作为今后的指南。

反思错误和弱点,固然痛苦,但不敢去正视,自欺欺人,麻木不仁,反而会迎来更多痛苦。

想要什么呢?目前的第一目标,当然是财富自由。金钱非万能,单可以解决80\%的日常问题。
财务自由是人生发展的里程碑,越过这个坎,会有更丰盛更多彩的未来。

从另一个角度来看,财务自由只是一个结果,而不是原因。原因在哪儿呢?

\subsection{1207}

多读书,读好书。在会通中西古今的大背景下,选书,读书,写书,输出为主,输入为辅。

打磨思维和认知的利剑,剑锋所指,在思想独立,财务自由。
禅可指称深刻而灵活的思想,禅剑合璧,所向无敌,如卡耐基所说的思考致富。

坎贝尔的神话学,令人印象深刻,一个套路,一个个人成长和进化的架构,
一个心象,可以激发内在潜能,开启内在和外在的英雄之旅。

idea精英,可理解为良知,决策就是找到好理念,与职位和人是不同的事情。
独裁制度和民主制度偏于两极,民主集中制合乎中道。

赋能,机器的设计者,赋能者。积极践行达里奥的原则,
这是目前的中心任务,反思是进步的基石。

反思职业生涯,有很多可以改进的地方,如个性,技能,选择等。
情商,学习方法,成为成长的瓶颈。

立天之道,以定人也。天之道,即是宇宙自然法则,也是为人处世之原则,制定原则,极其重要。

达里奥几十年来,坚持练习冥想。自然之道静,故天地万物生,冥想属静之道。

食其时,百骸理,动其机,万化安。输入输出,得其时机。

生者死之根,死者生之根。恩生于害,害生于恩。从失败中学习和提高。

爰有奇器,是生万象。机器思维。

达里奥的原则和黄帝阴符经比较研究:达里奥求真意识极为强烈充沛,自然之道不可违,
因而制之。至静之道,律历所不能契。

\subsection{1208}

真人者,与天为一。

真人者,同天而合道,执一而养产万类,怀天心,施德养,无为以包志虑思意,而行威势者也。

心能得一,乃有其术。阳明之学,是精一之学。真就是达里奥的一,truth,真知。
真的对立面是错误,错误不可怕,通过错误来学习,对立面的转化。

笛卡尔,斯宾诺莎也讲真观念,阳明讲良知。达里奥的优胜处,在于他的真知是可操作的原则和机器,
与现实密切联系,从现实中吸取宇宙之力。直面现实,承担责任。

揽彼造化力,持为我神通,参赞化育,自觉参与到宇宙的进化中来。

\subsection{1211}

GROW, WOOP, POA, 达里奥五步法,都是以goal为中心的实现梦想的过程。
WOOP具有更好的对称性,POA在各要素间引入了量化关系,且加入了人这一重要因素。
达里奥五步法强调了分析和设计的重要性。

谈谈方法,莱特兄弟的工作法,用系统来工作,原则,费曼学习法,
TOC(思维过程),PDCA,丰田工作法

10X思维,也是目标。常规思路行不通,需要出奇制胜。
守正出奇,正可立于不败之地,奇乃致胜于无形。

软件,写入,AI,大量的系统的阅读,奠定坚实基础。

学会如何学习是元知识。

固定心态和成长心态

进化是唯一重要的事。什么是进化?观天之道,执天之行,尽矣。

如何做出好的决策?更多的知识并不会带来更好的效果,适当的时候,就可以行动了。

\begin{tcolorbox}
艾伦推翻了那些典型的成功学书籍的方法。他告诉我们如何时时刻刻处在当下,
注意此时此刻所发生的事情,以达成卓越表现,因为这一刻就是你所拥有的一切。
我们常常会认为,想要提升自己或者他人的表现,增加知识很重要。
于是,我们去学习、培训,向他人灌输经验,但是往往表现和能力还是没有多大提升。
如何才能取得突破性的表现?本书作者艾伦•范恩发现,我们表现不佳往往不是不知道该怎么做,而是做不到我们所知道的。
太多的知识,不但不会帮助我们,反而会对我们造成选择上的干扰,阻碍了表现的提升。而消除干扰的关键在于专注。
\end{tcolorbox}

\subsection{1212}

既要吸收新知,更要培本固元。温故知新,循序渐进,纲举目张,提纲挈领。
以问题为主线,通过原则和清单把一切知识运用起来。

每个学科和领域,都有自己的模型和原则。查理芒格号召,积极掌握硬学科的基本概念和模型。
进化论和控制论最为切要。

真妄相待,波普尔的思想值得高度重视。

用上2017剩下的时间,好好消化吸收原则一书的精华,为我所用,构建我的原则。

守正出奇,持经达变是学习原则的必要态度,
既要有章法可循,又要具有高度灵活性。

要想击败市场,需要理解市场的运行机制。要想实现目标,需要理解实现目标的种种途径。

目标是什么?最重要的是什么?

\subsection{1213}

波普尔的三个世界:物质,心理,知识。原则属于知识的第三世界,
第一世界对第二世界开放,第二世界对第三世界开放。即各个世界存在作用和反作用关系。

第三世界的主要存在形式是语言构成的知识。所以记录下来很重要,
才能形成可以批判和进化的独立性。

说第三世界是自律的,具有客观实在性,不依意识而转移。任何科学都是不完全的。

禅宗不立文字的习气,实在弊大于利。立象以尽意,系辞以尽其言,才是正确态度。

面试王浩林,基础扎实,可视性和可诊断性,富国银行,提供dtrace报告。

SQ3R读书法,值得提倡。看代码也是如此。

理解,分成组块,刻意练习。以教为学,专项练习。

\subsection{1214}

用大乘起信论的架构涵摄一切学术。一心二门三大四信五行。
道德经,大学中庸,阳明心学,阴符经,笛卡尔,达里奥等。

大乘起信论的框架实在大美。一心二门三大。开放才能大。

四信,佛法僧。佛和僧,对应people(two yous);法对应文化和设计。
信根本真如,目的和产出。

五行涵摄万有(六度),布施是我对人,忍辱是人对我,持戒,精进,止观两两对偶。

以真或诚为最高标准。我爱我师,我更爱真理。
黑格尔,波普尔提供一个反思的维度和素材。

真是美的,也是善的。认真,尽性。从错误中学习,与求真是一体两面,一体两翼。
真需要强大的实力支撑。

打造机器,项目管理,bug管理。

凡治众如治寡,分数是也;斗众如斗寡,形名是也。组织结构不可不讲求,如刘强东。
林彪六大军事原则,三三制,四组一队是组织原则。

定标准,如libvirt的代码风格检查,单元测试框架。

\subsection{1215}

从容一生,从容中道,从容二字,心所向往。定目标,沉住气,悄悄干。九字真言。

以天为师。

精进亦有正邪,注意把握节奏。

系统思考之道:宇宙,动,平衡,理解,控制,事业,道的体系,模型,智慧。
三个一组,构成波普尔所说的三个世界。借助世界3,整合了世界1和世界2,
构成hegel所谓正反合的辩证过程。世界3,与柏拉图的理念世界有同有异。

制度是灯,觉知是电,控制是光。觉知和理解虽然属于无形之物,起到至关重要的作用。

\subsection{1216}

昨夜昏沉,杂念太多。如何克治?起信论五行,尽善尽美,加以适配,则能为我所用。
要处在于止观,止观不备,则大道不行。

什么更重要,什么最重要?10X意味着什么?换算成财富,约合年薪400w。
只算一不大不小的里程碑。加上时间因素,三年?五年?

唐双宁:矛盾论,光大集团。注意力是主要矛盾的主要方面,注意力>时间>金钱。
以这个价值观来衡量,则杂念太多,就颇为严重,需要尽早化解。

深潜(渊默),必要且必须。理论是行动的秘诀,构建理论,萃取原则很重要很重要,
这是以往的一大缺陷,知行脱节。

\section{04}

寻找人生算法。

量子模型,原子核、核外电子、能级轨道。跃迁至高能态,受到能量激发。

E=hv,v是频率,不仅要看到微观规律,更要注意宏观规律。

波粒二象性,形成量子思维。

狄拉克海,量子场论,反粒子,反物质。启示:数学的重要性。

量子力学,结构化学的学习过程,效率不高。光看书没有多大用处,重要的是思考,提出问题,解决问题,
才能激发创造性思维。即使把书背下来,依然不是自己的知识。区块链的学习过程中,也是这个现象。看了
几本书,依然不知所云。这是需要深刻反思的地方,犹如大物理学家费曼所说,
知道一个名词与知道名词代表的事物本质,有云泥之别。

以数学为护城河,触类旁通,人工智能、量子计算、区块链等技术进展,没有数学基础,是很难真正理解的。
这是纲领。另外重要的是,如何学好数学,并学以致用,运用到自己的研究、工作和生活中?

原子模型、太阳系行星模型有着惊人的同构性,这是自然最伟大的法则,也是重要的思维模型:圆点哲学。
波粒二象性,场论都是建立在基本的原子假说之上。所以,第一步是要掌握思维的原子模型,进而推广到更高级的阶段。
这是量子力学的高级阶段,最终形成万有的统一理论。

原子核由基本粒子构成,电子绕原子核高速运转,运转发生在不同的能级上,能级跃迁吸收或释放能量。
电子有自旋。质子和电子遵循电荷守恒,结合成为中子。质子和中子都有三个夸克胶合而成。这是其粒子性的一面。

上面提到的相似性,是一种分形结构。貌似混沌的复杂现象,底层逻辑往往非常简单。元素周期表总结了组成万物的为数不多的原子,
而原子也可以用更为基础的基本粒子来表述,弦论进一步统一了万物的构成单元。
对生命体而言,也是一样,细胞学说在生物学上具有划时代的重大意义。
高祖刘邦约法三章,市场经济的运行规则也是简单的:互通有无,等价交换。

系统科学不同于自然科学或社会科学,是在更抽象的层次来研究各种各样系统的现象和规律。

为什么在苦苦追寻,却越来越累,茫然无措?这是需要深刻反思之处,在旧纸堆里消耗生命,是否真有价值?
浮士德走出了昏暗的书房,迎来了色彩斑斓的现实生活。

找到了吗?找到自己的核心算法了吗?简洁的内核,就像夜空中的北极星,指引迷途的人以方向。
或跃在渊,以勇气走出大山,走向属于自己的星辰大海。

元素周期表是化学的殿堂,原子相互结合而成分子,分子形成化合物。HCON具有特别的重要性。个体是社会中的一原子,
不同元素的原子具有相应的特征,或活泼或沉稳。

道法自然。人类对自然规律的认知,已达到非常的深度和广度。光,电磁,热,力等现象逐渐走在统一的路上。数、理、化、生等自然科学
蓬勃发展。

收敛身心,投入到最重要的事上。ABCD就是其范围。

删繁就简,自律,极简,转动元亨利贞之环。

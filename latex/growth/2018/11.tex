\section{11}

\subsection{31}

中医、武术、书画、围棋等国术,不离易理。一元四素的全息方法论,务必要掌握。

技术之路还有很长,中间的深切反思是为了走更长的路,更快更稳。反思的结果,熊师有着全面的论述。
修之身,其德乃真。修身为本。感受身国的能量,爱之惜之,及时补充,以通为和。生命力乃是反熵为食,
不知不觉,五贼三盗,呼啸而至,而至于伤其本也。反者道之动,逆而转之,以求超越生死之途,而别开生面。

念诵、呼唤,此双可教育法,诵读心诀:恭熟忘合灵,层层递进,环环相扣。寻声觅音,音声相和,太极也有三阶段。
目前熟悉之经典,如太极图说、阴符经、大学中庸首章,皆当温故而知新。此由熟而阶于忘也。

尊道贵德,真有恭敬之心,忠信之情吗?须反思体认。意身清静上德明,心性修持大道生,离开修身,则离道日远。
道不远人,就在日用常行中,反求诸己,内求即可。阳明龙城悟道的本质,就是由外转向内,致良知、明明德。
尊德性而道问学,两者需要统一。德统慧智,道0而德1,道德也是一体两面,无极而太极,太极本无极,隐显有无之间,中气以为和。
慧智二分,一无一有,一隐一显,一上一下,统摄于德一之境。

此次反思,旨在开启慧境。始于阳明心学,历经曲折坎坷,迷悟相依,真妄相待,至今稍知门户,更应沉潜反复,信解行证,以求大成。
以经解经,以心印心。\hl{君子黄中通理,正位居体,美在其中,而畅于四肢,发于事业,美之至也}。

\hl{至此,可以开启第二阶段,进德修业}。天命之谓性,呼唤使命。
业即在德之中了。明明德,这是阳明的理路,以内统外,返本归元,执一以为天下牧。
所以,只有一事,明明德,何以明之?则有一些条目可以依循,大学中所列之次序即可。

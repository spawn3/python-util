\section{11}

\subsection{01}

中医、武术、书画、围棋等国术,不离易理。一元四素的全息方法论,务必要掌握。

技术之路还有很长,中间的深切反思是为了走更长的路,更快更稳。反思的结果,熊师有着全面的论述。
修之身,其德乃真。修身为本。感受身国的能量,爱之惜之,及时补充,以通为和。生命力乃是反熵为食,
不知不觉,五贼三盗,呼啸而至,而至于伤其本也。反者道之动,逆而转之,以求超越生死之途,而别开生面。

念诵、呼唤,此双可教育法,诵读心诀:恭熟忘合灵,层层递进,环环相扣。寻声觅音,音声相和,太极也有三阶段。
目前熟悉之经典,如太极图说、阴符经、大学中庸首章,皆当温故而知新。此由熟而阶于忘也。

尊道贵德,真有恭敬之心,忠信之情吗?须反思体认。意身清静上德明,心性修持大道生,离开修身,则离道日远。
道不远人,就在日用常行中,反求诸己,内求即可。阳明龙城悟道的本质,就是由外转向内,致良知、明明德。
尊德性而道问学,两者需要统一。德统慧智,道0而德1,道德也是一体两面,无极而太极,太极本无极,隐显有无之间,中气以为和。
慧智二分,一无一有,一隐一显,一上一下,统摄于德一之境。

此次反思,旨在开启慧境。始于阳明心学,历经曲折坎坷,迷悟相依,真妄相待,至今稍知门户,更应沉潜反复,信解行证,以求大成。
以经解经,以心印心。\hl{君子黄中通理,正位居体,美在其中,而畅于四肢,发于事业,美之至也}。

\hl{至此,可以开启第二阶段,进德修业}。天命之谓性,呼唤使命。
业即在德之中了。明明德,这是阳明的理路,以内统外,返本归元,执一以为天下牧。
所以,只有一事,明明德,何以明之?则有一些条目可以依循,大学中所列之次序即可。

针灸术,极有启发意义。身体具有特殊性,如何观察?如何治理?

\hl{旋极图是我的实践论、矛盾论},是我的唯物论、辩证法。
心物、知行、慧智、道器等矛盾在旋极图下得到完美解决。

世界观、认识论、方法论、价值论都在其中了。李小龙在道家阴阳哲学里,找到了功夫的哲学,即武道。
二宫尊德悟出了一圆之理。旋极图有着更为丰富的内涵,需要细细体认、发掘、为我所用。

历数历史上的大人物,多有其困顿颠沛之时,解决之道在理论创新、认知升级。

静坐时时可以体认,诵读也要养成习惯,八段锦要坚持下去。
\hl{静坐诵读八段锦}是开发慧识的功夫,此暗合熊师的双可教育法、太极修身体系。

读书就是采意采趣,以意逆志,采用王重阳十五论道书。
\begin{shadequote}

学书之道,不可寻文而乱目。

当宜采意以合心,舍书探意采理。合理采趣,来得趣则可以收入之心。
久久精诚,自然心光洋溢,智神踊跃,无所不通,无所不解。

若到此则可以收养,不可驰骋耳,恐失于性命。
若不穷书之本意,只欲记多念广。人前谈说,垮讶才俊。无益于修行,有伤于神气。虽多看书,与道何益。

既得书意,可深藏之。
\end{shadequote}

\subsubsection{道与艺}

人很难突破自己,再造生命。心里想的,与实际表现往往有较大差距,如何知行合一?需要下大工夫。
修身不是易事,正念、良知具备。不出大学之范围。岐黄医身,黄老医心。

软件与哲学也有密切的关系,即是道与艺的关系,现在有点偏了,没有很好地结合起来。
志于道、游于艺,如何统一?目前看来,是把重点放到艺的时候了。以道之光华照耀艺之征途。

哲学提供功夫的理论基础,重点在功夫上,技进乎道!
离艺而论道,有空泛之感。离道而修艺,有支离之惑。

旋极图提供了一个总的理论基础,但不能取代该做的重要工作。
举凡专业、工作、生活、事业等等诸多关系,用旋极图的理念一一分析,达到定二抱一的目的。

沉下来,如同U型理论所提到的,沉下来,升上去,此一体之圆运动。

\subsection{02}

一元四素,一元实乃三五之道,四素是四个方面、四个层次展开。一二三哲学,乃至四时五德,即是一元,包罗宏富。
现在能理解熊师的那个图了吧,左列一元、右列四素。中间是旋极图。由旋极图运化四方。

一气流行,此质象经内事,观此一气流行,化生万物。分开讲是一元四素,合起来,又是一元之旋极图。
无我能得一圆之理,明明德是根本。

一元四素是易经精华,东方思维的主要特征,与西方思维方式形成互补。东方西方,又是一个旋极图。
中体西用? 不妥。旋极有比体用更高明处。即体即用,体用不二。旋极之妙,在整合一切资源,为我所用。
具有极大的包容性和开放性。马云的太极、任正非的灰度,在旋极里,有着深刻的阐述。
\hl{更喜欢玄极这个名}。跳出三界外,不在五行中。离二元对立,而入无相之境。

庄子内七篇含内圣外王之意,人间世是过渡,逍遥游、齐物论、养生主、人间世、大宗师、德充符、应帝王。
真是气势恢宏,吞吐乾坤。齐则,归于一也。大者,无所不包也。应者,随机而动也,寂然不动、感而遂通。

当确立了自己的思想体系后,调整、巩固、充实、提高。
在信解行证中,反复地打磨、验证、优化,即展开完整的PDCA过程,严格要求自己。

内观是八卦炉、九鼎、是璇玑图。所读之书,是燃料,笼络天地、陶铸古今。
贾谊有赋:且夫天地为炉兮,造化为工;阴阳为炭兮,万物为铜。合散消息兮,安有常则?千变万化兮,未始有极!
(进入太极,不入璇玑之境,合散消息兮,有其常则;千变万化,信有其极)

切勿有沾沾自喜之心、阳明说一个傲字误平生,精神当内敛,不宜发散,养成中和沉潜之质。
从起心动念处,扎实用功,务求改天换地、变化气质。

周末采用新读书法,采意读书法。建立在牢固的基础之上。对熊师的思想体系,要慢慢地细细地消化吸收,化为自己分析问题、解决问题的思维方式。
这有着根本意义,因为这是哲学的核心地带。三合之道、太极图书心性学、以及熊师的大论,是近期极有启发的思想,带着自己,走向一个新高度。

这是长时间探索的一个结晶,从庞朴的一分为三开始,2013年,算来到现在已有六年了。六年磨一剑,很值。宝刀在手,接下来做什么?如何证实其价值?
马克思说的好,重要的不仅仅是解释世界,更重要的是改造世界。天人合一、禅剑合一。扩展自己,心想事成。

\subsection{03}

\subsection{04}

\subsection{05}

俞明三先生在三点论的基础上,提出了函学。印象深刻的是三因、五能演化说、并没有真的理解,引入术语过多,表达也有点晦涩。
熊春锦先生的著作,搜罗全了。旋极图是他分析问题的主要工具,道医学、儿童教育、治理学等。隐显三元不正好构成一个六画卦吗?
一上一下、一隐一显、一内一外。质元、物元、体元同样构成一个旋极图。这里不易理解,须借助现实的例子来喻析。

旋极图是一个重要工具,玄之又玄、众妙之门。现在开始,围绕它以建立完整的理论体系。
圆点论、三合之道是旋极图的一种简化形式。旋极图是易经、黄帝四经、道德经的中心思想,与河图洛书等皆可会通。
执大象、天下往,旋极图就是一大象,对理解诸子百家、儒释道非常有帮助。
矛盾论是基于两点论的,在论述上没有强调一的重要性。

\subsection{06}

在多个层面展开中和之德的思考。体用、内圣外王、修齐治平、三生万物。

守中致和是方法论,先要明白中和是什么。先确立圆点之至尊地位,达到临界点,精神内敛,真气从之。这些描述好像黑洞。

通观大学中庸,明德修身为本。

诵读,寻声觅音,内观之法,就好像阅读一本书,犹如电影一样从心中清晰地流过,一帧接着一帧。

太极图说是一篇雄文,义理丰富,无可与比。唐诗疑点有二:主静说,太极生两仪顺序说。太极动而生阳,还是静而生阳。
圣人定之以中正仁义而主静,立人极也。中正,似为中和之德,统摄仁与义。如何理解阴阳?

生是顺,五行一阴阳也,阴阳一太极也,太极本无极也是逆。顺逆各得其所。
无极者,道也;太极者,德也。无极而太极,太极本无极,居于道而用于德。
道与德实为一体两面,由德而有,物物一太极,化生万物。道为圆,德为圆心。
无极之真,二五之精,妙合而凝。凝聚而为一体,作为一个整体来把握,是全息分形结构。
是思维的原子模型。一元四素、三元论也在其中了。

厚德载物。

一阴一阳之谓道,阴阳不测之谓神,易传没有点出太极所在。

宇宙的终极模型,也是重要的思维模型。太极更多是质象境内的模型。

到了一个极为关键的转折点。找到了指导思想,下面的路会走得更顺利些,重要的是,要有一个突变,轻轻松松地实现大人虎变。
心中还有什么疑问?比较少了,情欲是主要的干扰,化干扰为助力,也是可以做到的。

一定要严于律己,走最经济的路线,达到自己的目的地。解决了思想问题,下面最有挑战的是专业。
学什么?达到什么程度? 会有什么产出?一定要严格地考评,力争上游。

修身为本,首先是身,身与心彼此呼应,养神。静养、动养交相互养。
闭目养神、静坐禅等静养,八段锦、太极拳等导引术。导引术有动静交养的功效。

修齐治平,一一归于太极模型的指导。从太极中,提炼一些关键理念,熊春锦的道学体系提供了一个框架,作为前进的基础。
最重要的一点,就是从三合之道进一步上升了旋极图的中和之德,为多年的探索画上了一个完美的句号。
从中可以推导处人生算法、李善友混沌大学的一些结论。中,就是第一性原理的本因。从第一曲线跨越到第二曲线,靠的就是这个中。
如何跨越靠的是和。中是天下之大本,和是天下之达用。中庸已经有非常好的揭示。

\subsection{07}

太极是核心理念,道商六图很全面,核心的核心是中极图,又称为旋极图、来知德太极图。
多读太极拳论以明理,慢慢加以习练,渐渐能有所成,不用太急功近利。不怕慢,就怕站。
站是犹豫彷徨,没有真信,不能力行。只要方向正确,或迟或早,总能抵达。

理解上去,就至简至易,众说纷纭,因为没有在根上用心。理通法随,中医如此,太极亦然。
先难后易,豁然贯通,则一日千里。

中和是至高境界,是普遍真理。中和是唯物论、辩证法。修齐治平,样样用得着。
但内心涵养,不可轻易吐露,吐露则后续少力量,这是精神修养的特点。

与人分享的初心是好的,但难免是虚荣心作怪,栽培了这么一个虚荣心,就是一大祸根。
需要严加克制,无我无为,才可通向更高楼。大学中庸处处说慎独、诚意,这是根本功夫,不能视而不见,
买椟还珠,不识庐山真面目。

喧嚣过后,洗尽铅华呈素姿。见素抱朴,精神内敛,一意涵养本源,才是正途。
小知炎炎,大知闲闲,真是生动的嘲讽!是安于小,还是慕于大呢?不言自明。
而言行上乃反其道而行之,真是不可救药呀!痛定思痛呀!

\hl{只有真信,才会切实吧。否则,悠忽度日,立不起来,随意滚落下去,真是可惜可叹}。

\subsection{08}

道/医,有助于明理;一静一动,静:闭目养神、静坐禅等;动,导引术,八段锦、太极拳等内家功夫。
如此,一动一静,身心都照顾到了。循序渐进,在感悟中提升功夫。

再进而扩充到生活、自医、性理、膳食、管理、成功等方面,\hl{张绪通的体系}值得关注。

这样一来,人之道就建立起来了,时时处处有规范,动静中礼。

习练有方,太极拳需要刻意练习的态度,用心揣摩其中义理精髓。招式并不是最重要的,最重要的是基础、内功,知其所以然。
在明理的基础上,十日练功,一日读书,练功明理交养,则进步快。

第一原理是中和之德,守中致和,贯彻到一切行动中。

不仅练拳,做任何事都要遵循慢字诀,事缓则圆。松柔静定,乃其中三昧也。
颇有相见恨晚之意,此番见识,不能丢。

\subsection{09}

基本功:静坐、呼吸、八段锦、马步、深蹲,最后用太极之理融会贯通。\hl{中医、琴棋书画等}也受太极之理的影响。
明理则习练有方,事半功倍。

物物一太极,太极之理精微又简易,对立而统一。阴阳相济,一气混元。
如何以现代科学阐发之?如何把简易的太极之理渗透到所做的每一件事情中?
以太极而悟太极,无处不太极。

这次修炼内容,逐步优化,重要的是要\hl{坚持下去,持之以恒},
不独强身健体,且能涵养心性,开慧益智,不能视为可有可无也。

战略上以一当十,战术上以十当一。这是两个层面,一流智者,在于运行两个看似矛盾的东西而并行不悖。
读书,精神内敛,采意为主。

又到周末了,好好总结近日的探索,择善固执。

不看新闻,具体来说,就是凤凰网和网易新闻。扰乱心神,无谓。精神内敛,一意本源。本源何在?

多言也是病,得治。自修则可,何必好为人师?桃李不言下自成蹊,
想想神奇的大黑洞,作为宇宙中最大的主宰力量,是如何做到的?
高度内敛,层层压缩,高速旋转,什么也不需要做,而众星拱之。

慢字诀,当先遵循,用心体会,身体力行,一招一式,不可马虎过去。比如压腿,当缓缓用力,日日不间断。

\subsection{13}

坚持静坐,为能双盘积极准备,八段锦、晃海等都有一些开跨、拉伸的动作,要能坚持,日日进步。
忍受一点疼,值得。疼是暂时的,而收获则能持久。

注意呼吸,胸式、腹式、体式等,至人之呼吸以踵,深且长。

按摩后溪穴?

\subsection{14}

存储系统的生命在于数据流。生命系统在于精气/能量流。

经络存想看上去不错,精气在全身如何流动,哪里发生了拥塞,这完全是一个网络系统。
精气是物质、能量,生命能,得之则生,失之则死。如何保养精气?管子内业首先提出精气之说,
人体如何与自然进行物质、能量、信息的交换?熵的概念有什么意义?生命以负熵为食,摄入负熵,则能量增加。
随着不断的消耗,则精气减少,不能平衡时,则免疫力下降,疾病就会找上来。

还精补脑是个巨大的创意,具体落实还需要一探究竟。

五脏六腑不能直接接触,但通过反射区可以施加间接作用,如前列腺按摩、鹿功、鹤功、龟功。

穴位是经络的关键节点,通过穴位可以影响经络的畅通无阻。气血循环,一是输送营养物质,二是排除废物,
即新陈代谢、吐故纳新的过程。心脏是气血循环的中心,七腺系统彼此感应,平衡最优。

呼吸经过肺,食物经过脾胃,初步处理后,随着心脏的搏动进入气血循环,膀胱、前列腺、尿液、精液。
大肠、小肠、大便排出。

肝胆、胰腺的作用?

中枢神经系统?消化、泌尿系统、生殖系统等,共同组成一个大系统。
这种理解,放到一般社会组织系统,也有一定的参考性。

一个重要的目标,学会双盘20分钟,需要准备,身体和心态的柔性。
另外就是鹿功相关的练习。边练习,边琢磨,听其言观其行。

\subsection{15}

围绕静坐进行锻炼,双盘要求很高,身心的柔性。欲降服其心,先降服其腿。松腰太极拳,用意不用力。
松沉静定,经常提醒自己,放松紧张的部位。一松一紧,才能相得益彰。

先能识别关键穴位,经络存想,打通任督二脉,大小周天。足三里、后溪可以联动。
为何要舌顶上颚,为什么要重视海底会阴? 因为二穴是任督二脉交汇处,由此一气联通。

杨定一的真原医和静坐,观点平实。

所谓循序渐进,宁可慢,也要保证动作的合理性。比如脊椎的运动,快了反而不好,慢才能体会骨节前后相随。
这也是学习的艺术里面画小圈的道理。虚领顶劲,以百会穴为端点,提领脊椎,使之中正从容。
从容中道,此语极妙。中庸一书,实在是一宝典,孔门最高之心法。与德道经可以会通,明明德,
明德者何?中和之德也。熊春锦所谓的德一境也。散而为五,合而为一。

按摩后溪,似乎有很大效果,右手已经恢复如常?当真如此的话,真是奇迹,更坚定了医武方面的信心。

万物负阴抱阳,中气以为和,是打坐的指导方针。\hl{阴阳所指,似乎与任督相反}?

不要急着宣告,沉潜反复,修之于身,其德乃真。身治则国治,修身为本,由内而外是正确方向。

\subsection{16}

重视腰,松柔静定。水性太极拳、心性太极拳有松腰的专项练习。为静坐打下坚实基础。
三个月应该能盘上,不过并不急,认清利弊,取其利、避其害。以11.12作为练习静坐的开始时间。

争取每天都不间断,循序增加难度,适可而止,不急于求成。把步骤和注意事项整理成一个清单。

\subsection{19}

经络确实神奇,也是下手处。312经络锻炼,算是入门功夫。循序深入。
经络也是一重要思维方式,点穴高手。对一复杂系统而言,如何控制?

312经络锻炼,可以理解得更广泛些,3是中医,1是静坐,2是武术、如太极拳等导引功夫。
3是基础,12是功夫,构成一个整体。经络为纲。

\subsection{20}

312以经络锻炼为中心,经络锻炼贯彻松空圆满的原则。静坐、武术的目的,也在于锻炼打通经络。
营血气,调阴阳,决生死,处百病。其作用岂能轻视乎?经络贯穿于道、医、武之全过程中。

机发论,于黄帝内经里有充分体现。诚神几曰圣人。如此,做事情就会高效。与心道物三合之道对应。
诚意、神道、知几其神乎?

时时处处都是做功夫之时,功夫所以进步慢,就在于不诚,诚则灵,至诚如神。一念闪动,震动十方。
一门深入、制心一处。都是刻画诚之为用。

\subsection{21}

练习静坐,考验意志力,欲克服之,一渐二恒三熬。跪式开始感觉做不来,但通过一蹲一蹲地慢慢柔化,就比较好切入状态。
不要有畏难情绪,直接做不到,就间接地去做,水滴石穿,坚冰渐渐涣然而释,柔化为水为气,水之三态,宜精心体味。
水性太极,化拙力为巧劲,充分利用万有引力,松空圆满。

我之二分,即便疼痛难忍,但如果放松,宾主分离,超然观之,也是能够熬下去的。一旦突破临界点,则一往无前,更上一层楼。

观察感知痛点所在,保持觉知。通过经络按摩等手法,慢慢调理,循序渐进。循序渐进,涵甚深奥义。
觉知痛点,柔化之。

痛也是必然的,没有这个锤炼,静坐的效果就会打折扣。静坐是全方位的修炼,虽然是形式上的,有超乎形式之外者。

一定要以积极心态,直面遇到的问题。在计划内,没有例外。所谓自律就是没有例外,如此也可形式坚定的意志力。
\hl{专注力和意志力},在这个过程中,得到很好的锻炼。

\subsection{22}

慢,打好基础,一层一层地打好基础,待做到松空圆满时,主练下一个痛点。慢,才能仔细体会痛点所在,觉知、柔化、克服。

放松,心无挂碍。现在依然坐不住,心有所牵挂。立定课程,快火煮、慢火温。一主多辅,一个时间关注于一个关键处,旁及于若干处。
八门五步,以五行解之。中央为土。

现在是沉淀期,厚积薄发,大器晚成。如果错失,则悔之何及?故当终日乾乾,精进不止。

乐者心之本体,不乐则难以持久,乐在其中,则不觉煎熬。唤醒乐感,则任何难关当不在话下。玩索而有得,打坐一事,也不要太拘束,
寓打坐于游玩之中,自然日日精进,有豁然贯通的一天。\hl{循序渐进,目前当多体会一个慢字诀}。化痛点为乐感。

苦中作乐,不如理解成机会难得,这是一场身与心的洗礼。不是难熬,而是发自内心的热爱。

迈出了这一步,就不后退,持之以恒,乐在其中。

静坐日课,进一步强化这个。\hl{2018年11月22日,做出以庄严承诺:将打坐进行到底,不成不休}。

日课
\begin{enumbox}
\item 每日清晨五点半开始
\item 每日晚上九点半开始
\item 日常注意行仪,视听言动思
\item 乐者心之本体,化痛点为心乐
\item 燕处超然
\item 循序渐进,体会慢字诀
\item \hl{对痛点,不可放过},觉知、驯服,一点一滴进步。千里之行始于足下。
\end{enumbox}

守破离,定随舍,三阶段。

软件、中医、系统论很多相通之处。生命系统和非生命系统,遵循着统一的运行原理。
维纳的控制论、香农的信息论都是基于这种相通性。物物一太极,事事一系统。不为良相,则为良医。
阴阳五行思维模型有着丰沛的生命力。

\subsection{23}

收回到呼吸,念头妄动是自然现象,犹如放风筝,呼吸就是手中线,必要的时候及时拉回来。

坛经无念为本,无住为宗,无相,无相乃实相,花许久的时间静坐,是否浪费?此时心中进行思考,是否合理?
此为一惑。

\begin{shadequote}
善知识。我此法门。从上以来。先立无念为宗。无相为体。无住为本。
无相者。于相而离相。无念者。于念而无念。无住者。人之本性。
\end{shadequote}

万缘放下,难!有点不务正业的感觉?打坐与所谓正业如何调适而上遂,相得益彰?
于功名利禄,心中汲汲,遗失根本。贵我,通今。

在静坐中,经络、太极理念极为有益。现阶段,以静坐为首,围绕静坐展开经络、太极的理解。
在静坐的过程中,解决前进的障碍。经络方面的认知:对痛点,揉按痛点,柔化。循经找穴,一点一点进步,终有豁然贯通的时刻。勿有畏难情绪。
太极:松柔静定,即是功夫,也是目的。松是心法。借由静坐打通身心二法。

静坐时,感觉极为敏感,加一毫力极有难以承受之感。不知为何?故务必重视保暖等措施。气机堵塞不通,如水之被坝阻挡,能量越积越多。
有其危险处,不可等闲视之。地心引力,流体等力的分析。

下坐时,酸痛感极短暂,很快就有舒畅轻安的感觉。以能量流动来理解,是能解释的。

经过几天的锻炼,有一定进步。下一步的重点,依然是基本功,所谓筑基。根基不固,不能建起万丈高楼。

静坐的影响是全方面的,养成沉静的气质,控制好做事的节奏,对长远的成功至关重要。
比如在工作中,也可以时不时地静下来,调节心神,充分发挥专注力和想象力。

动静结合,以静为主。静是回归先天,归根曰静,循理曰静。先后天的二层划分,至关重要。
静非不动,乃是更好地动。动静一如,守中致和。动中求静,动亦是静。
动静两者,在第三方理论的作用下,乃得其解,其实一也。

当下不正常的地方:
\begin{enumbox}
\item 发白
\item 牙齿
\item 胸前小硬块
\item 肾结石
\item 后背炎症
\item 小腿抽筋
\item 右手中指
\item 右脚大拇指
\end{enumbox}

每项逐步加以调理。求医做事当求精密,一分一毫之差,就可能造成大不同。静坐时可知,加一丝力,感觉就重如泰山。
右脚大拇指有伤处,感觉明显许多。空军飞行员对身体素质有着异常严格的要求。

\subsection{26}

重视经典,黄帝内经是中医之源头,不必圣化,却要致敬。接下来以黄帝内经、黄帝四经为中心,
于内圣外王之道,三致意焉!理论圆融,则进退适度。

熊师春锦,觉得四经比内经更有价值。且不论,苟能有益,何必分个高下?

张其成:一个中心,三个代表,两个基本点,概括国学,一源三流。看来我的理解,并没偏,只是深度不够。

\subsection{27}

静坐中可以融入很多内功锻炼,是全方位的实验,特别要注意经络,经络通畅,就很很顺利,不会太疼。所以,要把拍打拉筋等经络的手法融合进来。
详细觉知每个部分的感觉,对痛点加以特别的存想,就可以很好地驯服它了。

经络存想实在是伟大的构想,内视精气在经络内的流动,如果发生堵塞,就可以通过一定的手段疏通,保障通道的流畅。这一点具有巨大的价值,
在自我诊断并唤起自愈力方面,作用巨大。如何才能更好更快地掌握呢?

撞丹田一法,难度不高,是否有立竿见影之效果,需要实验。方法本身,可谓直指人心,针对性很强。
所有这一切,都在于明经络三字。经络是金钥匙。

目前的锻炼法有点凌乱无章,如何制订出一套具有一致性的简练锻炼法?其实,所有的锻炼法都应该围绕一个中心进行,即痛经活络。
这是内功的实质所在。明了这个,就可以时时处处地进行锻炼,用心揣摩。

沉淀期,认识更重要的事情,放慢节奏,好好思考人生。藏器于身,待时而动。

\subsection{28}

内功渐入正途,持之以恒,不必急于求成。作为日常生活中养精蓄锐、韬光养晦的一种生活方式即可。
猛火煮,慢火炖。掌握火候很重要。

按摩、敲敲打打有助于肢体的柔韧性。特别注意各种关节处,如脚踝、腰部等,静坐是容易疼的地方。
跪坐、还阳卧都可以去实验。

把一个一个小锻炼融入日常生活中,既有乐趣,也能收获健康,一举多得。调节心态,知轻重缓急。

医学不可不懂,易医同源,一直关注易经和古文的好处就彰显出来了。
会更能识别出关键点,建立起完善的知识体系。

阅读真是好习惯,从书本中可以得到指导成长和进化的指南、营养。但不能为阅读而阅读,这就本末倒置了。
近期关注中医和武术较多,也屯了不是资料。今后不断地完善医学资料库,作为一项重要的工作去做。
不过不要忘了,现阶段最重要的是什么?进德修业四个字足以概括。这是个人内在的革命,会经历各种情况,
信德信仪为什么重要?就在这里。信为一切功德母。锻炼身体就如爬山,上去的时候累点苦点,但过了坎,就有说不出的乐。

医学不仅仅关乎身体健康,也关乎心智成长。医学提供了看世界的一套思维方式,理法方药,望闻问切,辨证施治等等都是大智慧。

任何的宏图大志,离开了健康都显得如此的微乎渺哉,无足轻重。茨威格说:人生之幸运,莫过于在年富力强的时候,发现了天命所在。
这说出了一半的真理,至少同样重要的是,如何唤醒本具的自愈力,以避开风波海浪。

专研炖菜煲汤,煎菜炒菜少吃。伊尹由烹饪悟出治国之道,和实生物,同则不继。

求放心,经过最近的阶段性调整,心思活动开放而灵活,需要更专注了。设立日课,慢火炖的阶段。
最重要的是,回到工作重心上来,有主有辅,围绕一个中心,开展工作。万不可急于求成,自乱阵脚。

经典学习,以黄帝内经和黄帝四经为中心。内经又以经络为中心。经络明,则成竹在胸,纲要备矣。
不仅仅限于理论学习,更要在日常生活中发挥指导作用。比如静坐、太极以及一切活动。

近期心思有点野了,买了那么多的书、对那么多新鲜事物感兴趣。物有本末,事有终始,知所先后,则近道矣。

\subsection{29}

按摩下丹田,貌似原理有的小疙瘩区变得平滑多了。体会撞丹田功法,靠近墙体时顺其自然,用意不用力。不利用双手的力量,如何有效弹回?
随着距离的加大,弹回难度也更大。四肢、呼吸如何相互配合?尚没有找到感觉。一大问题,时间太紧凑,没有大段时间去体会。
周末在家也可练习。

静坐进入良性循环,单盘慢慢变得不费力,记得要全体放松。放松放松再放松。坚持下去,不能白白放弃。松字诀,松腰,松腿,松脚。
感觉气血不流通时,做小幅调整,可以这么做,也应该这么做。不能过于僵化,造成不可挽回的副作用。持之以恒,循序渐进是指导原则。
反过来说,急于求成,欲速则不达。有为、无为、无不为是三阶段。无为是道法自然,有为要以无为为指导原则,则以无为为目的为境界。
为者败之,执者失之。由此可见,老子的境界,貌似寻常,却是颠扑不破的真理。在学习黄帝内经的同时,契入老子之第一义。

经络按摩贯穿于养生、静坐、武术的全过程中。这个是关键是根本。对按摩中的各种感觉,采取旁观者的态度,燕处超然,疼归疼,我归我。放松而不紧张。
子午流注,每一时辰先选一穴位,重点关照。

肌肉时不时酸痛,睡一觉就好个七七八八。睡眠时是顺腹式呼吸。

\subsection{30}

马上进入12月份了。公司状态不定,据说又要搬家到昌平区,当有所准备。轻车简从,以不变应万变。
书和日常用品保持最小集合,同时要准备工作变动的可能,积极备战。

静坐、撞丹田、太极渐能入门,持之以恒地巩固提高,不能急于求成。

最重要的是什么?还是在业务上有新的突破,提升一个台阶,进入快速发展的轨道。
内功与外功相辅为用,交相互养。理解了中医的思维方式,对专业能力的提升也有很大的帮助,\hl{要主动地进行会通内外的工作}。

心沉下来,静下来。外界的变化不会扰动一颗宁静的心,宁静致远。

矫枉过正,单盘的准备阶段,可以分为三阶段:自由,正常单盘位,紧单盘位。由紧单盘位回到正常单盘位,就比较轻松。
锻炼宜稍稍超出正常水平,这样就会进步快一些。这也是一种循序渐进。

虚领顶劲,调整身形要注意这一点,先定脊柱,保障脊柱的中正安舒,含胸拔背,沉肩坠肘,收腰敛臀,就有了根基。

调息,任脉下行,督脉上行,形成小周天。呼气沿着任脉下沉,吸气沿着督脉上提。\hl{这是逆腹式呼吸}?
在静坐中,观呼吸是方便方面,其中细节,尚待进一步探索实验。

\section{11}

\subsection{01}

中医、武术、书画、围棋等国术,不离易理。一元四素的全息方法论,务必要掌握。

技术之路还有很长,中间的深切反思是为了走更长的路,更快更稳。反思的结果,熊师有着全面的论述。
修之身,其德乃真。修身为本。感受身国的能量,爱之惜之,及时补充,以通为和。生命力乃是反熵为食,
不知不觉,五贼三盗,呼啸而至,而至于伤其本也。反者道之动,逆而转之,以求超越生死之途,而别开生面。

念诵、呼唤,此双可教育法,诵读心诀:恭熟忘合灵,层层递进,环环相扣。寻声觅音,音声相和,太极也有三阶段。
目前熟悉之经典,如太极图说、阴符经、大学中庸首章,皆当温故而知新。此由熟而阶于忘也。

尊道贵德,真有恭敬之心,忠信之情吗?须反思体认。意身清静上德明,心性修持大道生,离开修身,则离道日远。
道不远人,就在日用常行中,反求诸己,内求即可。阳明龙城悟道的本质,就是由外转向内,致良知、明明德。
尊德性而道问学,两者需要统一。德统慧智,道0而德1,道德也是一体两面,无极而太极,太极本无极,隐显有无之间,中气以为和。
慧智二分,一无一有,一隐一显,一上一下,统摄于德一之境。

此次反思,旨在开启慧境。始于阳明心学,历经曲折坎坷,迷悟相依,真妄相待,至今稍知门户,更应沉潜反复,信解行证,以求大成。
以经解经,以心印心。\hl{君子黄中通理,正位居体,美在其中,而畅于四肢,发于事业,美之至也}。

\hl{至此,可以开启第二阶段,进德修业}。天命之谓性,呼唤使命。
业即在德之中了。明明德,这是阳明的理路,以内统外,返本归元,执一以为天下牧。
所以,只有一事,明明德,何以明之?则有一些条目可以依循,大学中所列之次序即可。

针灸术,极有启发意义。身体具有特殊性,如何观察?如何治理?

\hl{旋极图是我的实践论、矛盾论},是我的唯物论、辩证法。
心物、知行、慧智、道器等矛盾在旋极图下得到完美解决。

世界观、认识论、方法论、价值论都在其中了。李小龙在道家阴阳哲学里,找到了功夫的哲学,即武道。
二宫尊德悟出了一圆之理。旋极图有着更为丰富的内涵,需要细细体认、发掘、为我所用。

历数历史上的大人物,多有其困顿颠沛之时,解决之道在理论创新、认知升级。

静坐时时可以体认,诵读也要养成习惯,八段锦要坚持下去。
\hl{静坐诵读八段锦}是开发慧识的功夫,此暗合熊师的双可教育法、太极修身体系。

读书就是采意采趣,以意逆志,采用王重阳十五论道书。
\begin{shadequote}

学书之道,不可寻文而乱目。

当宜采意以合心,舍书探意采理。合理采趣,来得趣则可以收入之心。
久久精诚,自然心光洋溢,智神踊跃,无所不通,无所不解。

若到此则可以收养,不可驰骋耳,恐失于性命。
若不穷书之本意,只欲记多念广。人前谈说,垮讶才俊。无益于修行,有伤于神气。虽多看书,与道何益。

既得书意,可深藏之。
\end{shadequote}

\subsubsection{道与艺}

人很难突破自己,再造生命。心里想的,与实际表现往往有较大差距,如何知行合一?需要下大工夫。
修身不是易事,正念、良知具备。不出大学之范围。岐黄医身,黄老医心。

软件与哲学也有密切的关系,即是道与艺的关系,现在有点偏了,没有很好地结合起来。
志于道、游于艺,如何统一?目前看来,是把重点放到艺的时候了。以道之光华照耀艺之征途。

哲学提供功夫的理论基础,重点在功夫上,技进乎道!
离艺而论道,有空泛之感。离道而修艺,有支离之惑。

旋极图提供了一个总的理论基础,但不能取代该做的重要工作。
举凡专业、工作、生活、事业等等诸多关系,用旋极图的理念一一分析,达到定二抱一的目的。

沉下来,如同U型理论所提到的,沉下来,升上去,此一体之圆运动。

\subsection{02}

一元四素,一元实乃三五之道,四素是四个方面、四个层次展开。一二三哲学,乃至四时五德,即是一元,包罗宏富。
现在能理解熊师的那个图了吧,左列一元、右列四素。中间是旋极图。由旋极图运化四方。

一气流行,此质象经内事,观此一气流行,化生万物。分开讲是一元四素,合起来,又是一元之旋极图。
无我能得一圆之理,明明德是根本。

一元四素是易经精华,东方思维的主要特征,与西方思维方式形成互补。东方西方,又是一个旋极图。
中体西用? 不妥。旋极有比体用更高明处。即体即用,体用不二。旋极之妙,在整合一切资源,为我所用。
具有极大的包容性和开放性。马云的太极、任正非的灰度,在旋极里,有着深刻的阐述。
\hl{更喜欢玄极这个名}。跳出三界外,不在五行中。离二元对立,而入无相之境。

庄子内七篇含内圣外王之意,人间世是过渡,逍遥游、齐物论、养生主、人间世、大宗师、德充符、应帝王。
真是气势恢宏,吞吐乾坤。齐则,归于一也。大者,无所不包也。应者,随机而动也,寂然不动、感而遂通。

当确立了自己的思想体系后,调整、巩固、充实、提高。
在信解行证中,反复地打磨、验证、优化,即展开完整的PDCA过程,严格要求自己。

内观是八卦炉、九鼎、是璇玑图。所读之书,是燃料,笼络天地、陶铸古今。
贾谊有赋:且夫天地为炉兮,造化为工;阴阳为炭兮,万物为铜。合散消息兮,安有常则?千变万化兮,未始有极!
(进入太极,不入璇玑之境,合散消息兮,有其常则;千变万化,信有其极)

切勿有沾沾自喜之心、阳明说一个傲字误平生,精神当内敛,不宜发散,养成中和沉潜之质。
从起心动念处,扎实用功,务求改天换地、变化气质。

周末采用新读书法,采意读书法。建立在牢固的基础之上。对熊师的思想体系,要慢慢地细细地消化吸收,化为自己分析问题、解决问题的思维方式。
这有着根本意义,因为这是哲学的核心地带。三合之道、太极图书心性学、以及熊师的大论,是近期极有启发的思想,带着自己,走向一个新高度。

这是长时间探索的一个结晶,从庞朴的一分为三开始,2013年,算来到现在已有六年了。六年磨一剑,很值。宝刀在手,接下来做什么?如何证实其价值?
马克思说的好,重要的不仅仅是解释世界,更重要的是改造世界。天人合一、禅剑合一。扩展自己,心想事成。

\subsection{03}

\subsection{04}

\subsection{05}

俞明三先生在三点论的基础上,提出了函学。印象深刻的是三因、五能演化说、并没有真的理解,引入术语过多,表达也有点晦涩。
熊春锦先生的著作,搜罗全了。旋极图是他分析问题的主要工具,道医学、儿童教育、治理学等。隐显三元不正好构成一个六画卦吗?
一上一下、一隐一显、一内一外。质元、物元、体元同样构成一个旋极图。这里不易理解,须借助现实的例子来喻析。

旋极图是一个重要工具,玄之又玄、众妙之门。现在开始,围绕它以建立完整的理论体系。
圆点论、三合之道是旋极图的一种简化形式。旋极图是易经、黄帝四经、道德经的中心思想,与河图洛书等皆可会通。
执大象、天下往,旋极图就是一大象,对理解诸子百家、儒释道非常有帮助。
矛盾论是基于两点论的,在论述上没有强调一的重要性。

\subsection{06}

在多个层面展开中和之德的思考。体用、内圣外王、修齐治平、三生万物。

守中致和是方法论,先要明白中和是什么。先确立圆点之至尊地位,达到临界点,精神内敛,真气从之。这些描述好像黑洞。

通观大学中庸,明德修身为本。

诵读,寻声觅音,内观之法,就好像阅读一本书,犹如电影一样从心中清晰地流过,一帧接着一帧。

太极图说是一篇雄文,义理丰富,无可与比。唐诗疑点有二:主静说,太极生两仪顺序说。太极动而生阳,还是静而生阳。
圣人定之以中正仁义而主静,立人极也。中正,似为中和之德,统摄仁与义。如何理解阴阳?

生是顺,五行一阴阳也,阴阳一太极也,太极本无极也是逆。顺逆各得其所。
无极者,道也;太极者,德也。无极而太极,太极本无极,居于道而用于德。
道与德实为一体两面,由德而有,物物一太极,化生万物。道为圆,德为圆心。
无极之真,二五之精,妙合而凝。凝聚而为一体,作为一个整体来把握,是全息分形结构。
是思维的原子模型。一元四素、三元论也在其中了。

厚德载物。

一阴一阳之谓道,阴阳不测之谓神,易传没有点出太极所在。

宇宙的终极模型,也是重要的思维模型。太极更多是质象境内的模型。

到了一个极为关键的转折点。找到了指导思想,下面的路会走得更顺利些,重要的是,要有一个突变,轻轻松松地实现大人虎变。
心中还有什么疑问?比较少了,情欲是主要的干扰,化干扰为助力,也是可以做到的。

一定要严于律己,走最经济的路线,达到自己的目的地。解决了思想问题,下面最有挑战的是专业。
学什么?达到什么程度? 会有什么产出?一定要严格地考评,力争上游。

修身为本,首先是身,身与心彼此呼应,养神。静养、动养交相互养。
闭目养神、静坐禅等静养,八段锦、太极拳等导引术。导引术有动静交养的功效。

修齐治平,一一归于太极模型的指导。从太极中,提炼一些关键理念,熊春锦的道学体系提供了一个框架,作为前进的基础。
最重要的一点,就是从三合之道进一步上升了旋极图的中和之德,为多年的探索画上了一个完美的句号。
从中可以推导处人生算法、李善友混沌大学的一些结论。中,就是第一性原理的本因。从第一曲线跨越到第二曲线,靠的就是这个中。
如何跨越靠的是和。中是天下之大本,和是天下之达用。中庸已经有非常好的揭示。

\subsection{07}

太极是核心理念,道商六图很全面,核心的核心是中极图,又称为旋极图、来知德太极图。
多读太极拳论以明理,慢慢加以习练,渐渐能有所成,不用太急功近利。不怕慢,就怕站。
站是犹豫彷徨,没有真信,不能力行。只要方向正确,或迟或早,总能抵达。

理解上去,就至简至易,众说纷纭,因为没有在根上用心。理通法随,中医如此,太极亦然。
先难后易,豁然贯通,则一日千里。

中和是至高境界,是普遍真理。中和是唯物论、辩证法。修齐治平,样样用得着。
但内心涵养,不可轻易吐露,吐露则后续少力量,这是精神修养的特点。

与人分享的初心是好的,但难免是虚荣心作怪,栽培了这么一个虚荣心,就是一大祸根。
需要严加克制,无我无为,才可通向更高楼。大学中庸处处说慎独、诚意,这是根本功夫,不能视而不见,
买椟还珠,不识庐山真面目。

喧嚣过后,洗尽铅华呈素姿。见素抱朴,精神内敛,一意涵养本源,才是正途。
小知炎炎,大知闲闲,真是生动的嘲讽!是安于小,还是慕于大呢?不言自明。
而言行上乃反其道而行之,真是不可救药呀!痛定思痛呀!

\hl{只有真信,才会切实吧。否则,悠忽度日,立不起来,随意滚落下去,真是可惜可叹}。

\section{201806}

\subsection{0609}

坚持写日记。

李小龙的武术与哲学,以武术诠释哲学,吸收了佛道精神。不拘泥于形式,追求简单实用高效。
阴阳是理论基础,阴阳是一元论,一而二、二而一,不是二元论,侧重于一体两面的统一性。
水与空杯是隐喻。

功夫是自我发现与自我实现之旅。

志于道、据于德、依于仁、游于艺。道是第一位的,通过德、仁、艺得以落实到平常日用间。
德是大学的明明德,也即是小龙的自我发现之旅。仁是外在的,老子三宝的慈、俭、不敢为天下先。

艺不可少,择一艺以终老,就是软件设计,亦日日所从事的,注入真气,求大成。
不可作为寻找事情来做,而是纳入自己的生命旅程。

道法术器,是另一个序列。从艺的角度去分析。艺人通过自己的作品说话,突出作品而忘记自我。

古有练剑师,注入自己的生命,十年磨一剑.各行各业都有此类艺人,诠释大匠精神。

练拳不练功,到老一场空。不可不重基本功,打下坚实基础,可以走得更从容更远。

\subsection{0610}

诵黄帝阴符经,观天之道,执天之行,尽矣。见识行事必有所本,其源头在天,建立天道格局,则大事可成。
师心自用,以一己私智,则事半功倍,结果不容乐观。

\subsection{0621}

\hl{志于道为人生第一等事}。一体万化,大道之行,万物毕得。

哲学、原则、算法,纳入道的大熔炉里淬炼,爰有奇器,是生万象。奇器,即为道的熔炉。

从器上升到奇器,是一认识飞跃。
器,本指专业方面的成就、著作、产品。
奇器则指道,在道的光照下,专业、技术的成绩会更有异彩。

寓奇于器,是为奇器。道不离器,而超越于器。
知的目标是道,行的目标是器,道器是知行的升级版。

明道而制器,备物致用,立成器以为天下利。

如此则月印万川、万川入海,执道之要,制胜无形。

\section{201806}

\subsection{0609}

坚持写日记。

李小龙的武术与哲学,以武术诠释哲学,吸收了佛道精神。不拘泥于形式,追求简单实用高效。
阴阳是理论基础,阴阳是一元论,一而二、二而一,不是二元论,侧重于一体两面的统一性。
水与空杯是隐喻。

功夫是自我发现与自我实现之旅。

志于道、据于德、依于仁、游于艺。道是第一位的,通过德、仁、艺得以落实到平常日用间。
德是大学的明明德,也即是小龙的自我发现之旅。仁是外在的,老子三宝的慈、俭、不敢为天下先。

艺不可少,择一艺以终老,就是软件设计,亦日日所从事的,注入真气,求大成。
不可作为寻找事情来做,而是纳入自己的生命旅程。

道法术器,是另一个序列。从艺的角度去分析。艺人通过自己的作品说话,突出作品而忘记自我。

古有练剑师,注入自己的生命,十年磨一剑.各行各业都有此类艺人,诠释大匠精神。

练拳不练功,到老一场空。不可不重基本功,打下坚实基础,可以走得更从容更远。

\subsection{0610}

诵黄帝阴符经,观天之道,执天之行,尽矣。见识行事必有所本,其源头在天,建立天道格局,则大事可成。
师心自用,以一己私智,则事半功倍,结果不容乐观。

\subsection{0621}

\hl{志于道为人生第一等事}。一体万化,大道之行,万物毕得。

哲学、原则、算法,纳入道的大熔炉里淬炼,爰有奇器,是生万象。奇器,即为道的熔炉。

从器上升到奇器,是一认识飞跃。
器,本指专业方面的成就、著作、产品。
奇器则指道,在道的光照下,专业、技术的成绩会更有异彩。

寓奇于器,是为奇器。道不离器,而超越于器。
知的目标是道,行的目标是器,道器是知行的升级版。

明道而制器,备物致用,立成器以为天下利。

如此则月印万川、万川入海,执道之要,制胜无形。

\subsection{0629}

毕建勋造型本源,深契我心。中国画造型出自三源:物、心、道。

道是含二之一,一生二是演化过程,二而一是体道过程,逆而行之,推而上行。
早上诵太极图说、黄帝阴符经,心中喜悦。太极图说,包含了上行与下行两过程,妙合而凝。
阴符经:天地,万物之盗;万物,人之盗;人,万物之盗,三盗即宜,三才即安。
此段论述是很明显的三合结构。天地,引申为道。人与万物出于天地,各尽其分。
三者之间,存在是万有引力,循环相生相克。

静,引起特别的注意和兴趣。太极图说、阴符经、乃至老庄、淮南、管子等,都以静为究竟。
如阴符经:自然之道静,则天地万物生。静绝非如常识所理解的耽空滞寂,无所事事,没有活力,
而是有着巨大能量的状态与方法论原则,宜进一步深思。

静,涵戒定慧为一体。一静字,道破天机,是修行功夫的下手处。
老子云:致虚极,守静笃。大静方能有大动,不鸣则已一鸣惊人。尸居而龙现,渊默而雷声。

静,恒一而不杂。静,与天为和;静,顺势而为;静,保合太和乃利贞。
静,自然无为。即是稳定态,又携万千之力,静极复动。
静,不同于空,而大于空。

心欲安静,虑欲深远。

\begin{shadequote}
天道运而无所积,故万物成;帝道运而无所积,故天下归;
圣道运而无所积,故海内服。明于天,通于圣,六通四辟于帝王之德者,其自为也,昧然无不静者矣!
圣人之静也,非曰静也善,故静也。万物无足以挠心者,故静也。
水静则明烛须眉,平中准,大匠取法焉。水静犹明,而况精神!
圣人之心静乎!天地之鉴也,万物之镜也。夫虚 静恬淡寂漠无为者,天地之平而道德之至也。
故帝王圣人休焉。休则 虚,虚则实,实则伦矣。虚则静,静则动,动则得矣。静则无为,无为也,则任事者责矣。无为则俞俞。俞俞者,忧患不能处,年寿长矣 。
夫虚静恬淡寂漠无为者,万物之本也。明此以南乡,尧之为君也; 明此以北面,舜之为臣也。
以此处上,帝王天子之德也;以此处下, 玄圣素王之道也。
以此退居而闲游,江海山林之士服;
以此进为而抚世,则功大名显而天下一也。
静而圣,动而王,无为也而尊,朴素而 天下莫能与之争美。
\end{shadequote}

淮南子
\begin{shadequote}
夫精神气志者,静而日充者以壮,躁而日耗者以老。是故圣人将养其神,和弱其气,平夷其形,而与道沈浮俯仰。
恬然则纵之,迫则用之。其纵之也若委衣,其用之也若发机。如是,则万物之化无不遇,而百事之变无不应。
\end{shadequote}

道心合一、物我合一的功夫所在,就是正确理解的静字。
由静而上通九天、下贯九野。
由静而涤除玄鉴而无疵(10章六条,由一静字可得)。
心性修养,千言万语,由一静字可以概括。
静的效用、目的在于让心成为天地之鉴、万物之镜。由此三合之道大成。

由静以体道,二为一之门户,阴阳为太极之门户,由二以见一,舍二不能见一。
阴阳,太极之门。

践行止语练习。

由问题与范畴展开学习过程,\hl{利用搜索功能},以范畴为主线,贯串命题,形成完整的认知体系。

一个核心问题就是:道是什么?如何体道?在三合架构内,这些问题才能得到更好的理解。

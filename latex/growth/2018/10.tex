\section{201810}

\subsection{1003}

为学头脑处,此阳明念念不忘者。格物穷理,未免支离。头脑处在明明德,在心。龙场一悟,由外而转至于内。
精神之体相用,一而三、三而一,全体大用得以实现。

如何在心体上用功?在念头功夫。慎独之说,净念相继、都摄六根。正念是功夫、良知是本体。

先守住一部大学,体用兼备,兼采朱子阳明意,阳明为主,朱子为辅,尊德性而道问学。

\subsection{1004}

理解阳明心学之真实义,及其演进脉络,首要在于切己体察,作为成长之一助力。以德性融摄知识,在诚意中格物。

熟读大学,以定其规模。大学格物致知,兼采朱子阳明,以阳明为主。三合之道,圆伊三点。

张学智在阐释阳明心学时,采用道德理性与知识理性一主一从、相辅相成的观点,深有启发。

为学日益,为道日损。道统摄学,达以简驭繁之效。

三五以变,为学处事的纲领。三摄太极两仪,五有空间时间。数年方法论探索的一综合结论。混沌大学的第一性原理,第二曲线,不连续性
也纳入这一体系内。三生万物。

\begin{enumbox}
\item 易经
\item 道德经
\item 大学
\item 中庸
\item 孙子兵法
\item 传习录
\item 画道精义
\item 一二三哲学
\item 原则
\item 用系统来工作
\item PDCA
\item 稻盛和夫
\end{enumbox}

读书诸原则:
\begin{enumbox}
\item 有的放矢,精读泛读相结合
\item 读原文、悟原理、知行合一
\end{enumbox}

\subsection{1004}

从诚意去理解大学中庸,修身为本,则有下手处。喜怒哀乐之未发,谓之中;发而皆中节,谓之和。
中也者,天下之大本;和也者,天下之达道。致中和,天地位焉,万物育焉。

不反身,看不出一身毛病。

儒学,心学也。止于一为正,中和一是圣学根本。从内讲,至诚无息,纯亦不已。从外讲,尽性知天。
唯天下至诚,为能尽其性;能尽其性,则能尽人之性;能尽人之性,则能尽物之性。根源在诚意。

诚意与觉知、正念、冥想等略同,为明明德、致良知的功夫所在。

三五的中和一怎么理解?

\subsection{1006}

悟后大有功夫在,专且有恒,不可泛滥无归。大学中庸,入道之书,当熟玩之,以奠定根本。

看三国电视剧,关羽、周瑜、杨修等人,皆以傲字取败。阳明国藩诲人,以傲字为第一凶德,可不警惕乎?
力去此病,劳谦、君子有终,吉。玩易既久,而不得真实受用,则与不读等,枉费精神而已,当思痛改之。

反身而诚,乐莫大焉。反之,反求诸己,不怨天不尤人,实为修身之首务。确立我是一切成败的根源,从而自强不息。

萧天石极言精功、内功之有益,宜重视。圣人定之以中正仁义而主静,立人极也。自然之道静,故天地万物生。
静而能生,宜深思。动有何敝?

专攻读一经,易经是也。易之妙,终身读之不能穷尽。大学中庸道德经等,皆易经之辅翼。太极两仪,此大学三纲领之义理结构,
从内修的角度去理解,修身为本,修身实为进德修业的根本。阳明龙场之悟,点出了一个重要的道路,突出了炼心,合心物内外,而成一元论。
由外而转向内,明明德、致良知皆是心性功夫,心性不废知识,相得益彰,一君二民,逡巡并进。

\subsection{1007}

心道物三者循环往复,心的代表是稻盛和夫,道的代表是范蠡。道商,以道经商,是切合时代精神的选择。工匠精神、企业家精神有共通处,至诚之心,感天动地。
诚意是功夫头脑所在,论语与算盘,经济不仅是个人重要的一面,也是社会最重要的一面。致良知四合院是如何贯通两个领域的?企业家们学习致良知的价值有几何?

产业资本、金融资本之间的矛盾,金融资本得全球化之利,产业资本渐有全球化之害。

心性修养是一切的根源,内圣开出新外王,新时代,教育、科技、企业、资本是重点。

认知极重要,观三国演义、国共之争能强烈地感觉到,正确的见识有多么重要!寥寥数语,化腐朽为神奇。

国庆小结:

一、看电视剧三国演义、毛泽东,深知认知之重要,认知差极难弥补,超认知在事物的发展演进过程中,发挥着极大的设计塑造作用。
性格决定命运,人的事业发展的高度,视其性格即可见八九。由此可知,有无相生,无形的心性修养在事业中占据着极为重要的作用,
不可等闲视之。阳明心学融心学、知识修养与一炉,并非虚言。伟大人物如何看待一个事件,体现了其见识、洞察力。志、量、识不可或缺。

二、国庆之始,意在研读阳明学,中途多有变化,如看了萧天石、道商系列,沿着心道物的认知框架,逐步拓展。反求诸己是国庆最大的收获。
回归到正途,怨天尤人皆无用,风物长宜放眼量。不反身,看不出一身毛病;不反身,无以开发全部的潜能;不反身,就奠定不了以后发展的扎实基础。
行有不得,反求诸己,此君子之行也。不惟如此,一个团队、一个组织,也离不开沉下心来,如切如磋,十年一剑,磨炼内功。重视大学中庸,
如果仅仅在文字上打滚,也不会有大的收获。知行合一,方收大利。

三,既然立定了修身为本的纲领,诚意实为修身之要,慎独、主敬、求仁、习劳,是曾文正的心得。
不诚意,则旧习难改,因循度日,有恶不能改,有善不能迁,所失大矣。诚意是格物的主意,也是为学的头脑。按诸中庸之说,诚意之用,实为首要之责。
在心体上用功夫,则无支离无统之病。

四,重阳立教十五论有论学书一节,很得心学读书法之精要。心学的理念运用到读书、工作、生活中,当大有裨益。

五,道商学提出了六图思维模型。有极、无极、有无相生(太极、中极、真一),乃至大成。分四层。道常无为而无不为。有无之合,有中一至善之境。
真一图有近于圆点哲学。其中对三、五的解读,多有值得留意之处。四正的提法尤为精彩:智慧、生命、事业、兵法,合内外之道,由此可建立完善的知识体系。

六、关注到了八段锦,相比太极拳更为简易,工作闲暇之余,可玩味之,性命双修,养生之术,也当予以留意。神者生之本,另外一说也同样正确,好的身体
真的太重要。生命在于运动,生命也在于不运动(静)。静功性命双修,此南怀瑾、萧天石等前辈所明言。


\section{201810}

\subsection{1003}

为学头脑处,此阳明念念不忘者。格物穷理,未免支离。头脑处在明明德,在心。龙场一悟,由外而转至于内。
精神之体相用,一而三、三而一,全体大用得以实现。

如何在心体上用功?在念头功夫。慎独之说,净念相继、都摄六根。正念是功夫、良知是本体。

先守住一部大学,体用兼备,兼采朱子阳明意,阳明为主,朱子为辅,尊德性而道问学。

\subsection{1004}

理解阳明心学之真实义,及其演进脉络,首要在于切己体察,作为成长之一助力。以德性融摄知识,在诚意中格物。

熟读大学,以定其规模。大学格物致知,兼采朱子阳明,以阳明为主。三合之道,圆伊三点。

张学智在阐释阳明心学时,采用道德理性与知识理性一主一从、相辅相成的观点,深有启发。

为学日益,为道日损。道统摄学,达以简驭繁之效。

三五以变,为学处事的纲领。三摄太极两仪,五有空间时间。数年方法论探索的一综合结论。混沌大学的第一性原理,第二曲线,不连续性
也纳入这一体系内。三生万物。

\begin{enumbox}
\item 易经
\item 道德经
\item 大学
\item 中庸
\item 孙子兵法
\item 传习录
\item 画道精义
\item 一二三哲学
\item 原则
\item 用系统来工作
\item PDCA
\end{enumbox}

读书诸原则:
\begin{enumbox}
\item 有的放矢,精读泛读相结合
\item 读原文、悟原理、知行合一
\end{enumbox}

\subsection{1004}

从诚意去理解大学中庸,修身为本,则有下手处。喜怒哀乐之未发,谓之中;发而皆中节,谓之和。
中也者,天下之大本;和也者,天下之达道。致中和,天地位焉,万物育焉。

不反身,看不出一身毛病。

儒学,心学也。止于一为正,中和一是圣学根本。从内讲,至诚无息,纯亦不已。从外讲,尽性知天。
唯天下至诚,为能尽其性;能尽其性,则能尽人之性;能尽人之性,则能尽物之性。根源在诚意。

诚意与觉知、正念、冥想等略同,为明明德、致良知的功夫所在。

三五的中和一怎么理解?

\subsection{1004}

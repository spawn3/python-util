\section{201804}

年近不惑,适逢困局。如何突破?守破离,戒定慧。守住自己的优势,有所不为,
坚持战略定力,才能生发智慧。

马克思哲学的原则:量变产生质变。要做好持久抗战之准备,风物长宜放眼量。
找到自己的核心,围绕核心,调动一切资源,持续升级。圆点哲学是最根本的思考模型。
圆可以是渠道,可以是一切积极的力量。比如太阳系模型,同心圆。或如原子模型,不同能级的轨道。

故制定奇点行动,以突破当下之困局。若能持续个两三年,期于小成。

\begin{compactenum}
\item 反思历史,找准发力点
\item 交换、比较、反复
\end{compactenum}

\subsection{0417}

积分入户,社保缴纳年限不够。积极准备吧,需要一突破。难以满足就不争,争能争的。大争之世界。
北京户口重要,但不是最重要。

以问题为指引的学习和工作模式。一味地被动地吸收知识并不高效,转换为主动地萃取,如同蜜蜂采蜜一样。
核心是什么?分布式存储系统+AI。转变为问题驱动的学习模式,刻意练习,念兹在兹。

分布式系统,分布式存储系统。共识问题是分布式系统的关键问题。
存储系统范围很广,包括数据库,统一存储(块,对象,文件)。

分布式系统的属性:可靠、稳定、高性能。透明性指分布式系统像一台机器一样,但提供了复制、容错等分布式特有的特征。

系统编程:进程、通信、资源、体系架构、分布式系统。

把共识问题作为一个突破口,深入加以研究,建立对分布式系统的sense。包括paxos,raft,multi paxos,fast paxos等。
切分开来,识别关键问题,反复推演。这是理解很多关键问题的钥匙。

进一步,学习存储系统的基本分析模型。学习任何知识,都要先建立分析模型/隐喻,提炼、分离关键要素,反复推演。
问题驱动,刻意练习。

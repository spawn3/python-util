\section{201804}

年近不惑,适逢困局。如何突破?守破离,戒定慧。守住自己的优势,有所不为,
坚持战略定力,才能生发智慧。

圆点哲学是第一性原理,用它指导自己的生活和事业,融合其它有用的思维模型。
在圆和点的嬗变中,定位自己,回归核心,创造价值。

圆点哲学是成长战略,引领成长,在成长中持续深化和完善。

马克思哲学的原则:量变产生质变。要做好持久抗战之准备,风物长宜放眼量。
找到自己的核心,围绕核心,调动一切资源,持续升级。圆点哲学是最根本的思考模型。
圆可以是渠道,可以是一切积极的力量。比如太阳系模型,同心圆。或如原子模型,不同能级的轨道。

故制定奇点行动,以突破当下之困局。若能持续个两三年,期于小成。

知识管理,并不需要那么多。时间管理,划分时间单元,一时一事。自由来自自律。

\begin{compactenum}
\item 反思历史,找准发力点
\item 交换、比较、反复
\begin{compactenum}

看司马懿,忍人所不能忍,功业考量和心志磨炼。善败,败而不耻,败而不伤,磨炼心志。
这一切建立在能力的基础上。

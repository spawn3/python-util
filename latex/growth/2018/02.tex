\section{02}

\subsection{0222}

假期将尽,马上就要开工。2018年的目标是什么?怎么才能完成?这些问题是要反复提醒自己的。脱离了目标,工作和努力就变得没有多大意义。
重要的是创造价值,深度开发自己的作品集。

战略和专业是一体两翼,一体是进化,改变,成长,进步,跃迁,逆袭,活着和发展,或任何类似的概念。改变是中间的,变则通,不断地变化
不断地调整,使工作和生活的机器运行良好,收获美好的结果。今年的重点依然是工作,用正确的方法做正确的事情。
变化居于中空的位置,意味着以开放的空杯心态,接纳一切有助于促成变化的营养,时止则止,时行则行,动静不失其时,其道光明。
空性很重要,无知之知,不断地拓展自己的认知边界。

投资自己的教育和成长是最合适的选择。从投资的角度看待自己的工作,最重要的指标就是投入产出比,或杠杆率。投机,或赌徒的心态,不符合性格,
也不会有长期的收益。

创业,工作都是投资,种瓜得瓜,种豆得豆,一分耕耘一分收获。表面上看,做的事情不同,底层逻辑和模型则相通。当风险和收益不匹配的时候,就需要停下来反思了。

控制论是最根本的模型,适用于机器的世界,也适用于有机体的生命世界。耗散结构论揭示了更深刻的认知,控制论是一个近乎平衡的场景,耗散结构适用于远离平衡态
的场景。当然,两者都是开放的,熵是魔鬼,是乱序之度量,引入控制,是逆转该过程,使之有助于实现既定目标。一的思想,就是一个反熵。

拓展交际圈,默默耕耘,发挥能动性。反者道之动,弱者道之用,天下万物生于有,有生于无。弱非软弱,而是沉静,古朴,柔性,反脆弱。
潜龙勿用,这是当下的理性选择。沉下心来,好好打磨自己。


快行动,慢思考一书中的shiftpoint,极形象,换挡点。驾驶过程中,需要不断地进行微调,才能保证行驶在正确的道路上。

回顾多年的阅读史,暴露了几个缺陷:知行不一,没养成记录的习惯,为读书而读书,现实感不强,问题意识不强。

老子和孙子兵法,一柔一刚,经营未来。一阴一阳之谓道。提炼核心原则,让这些原则内化到思考和行动中。
商战如同兵战,是残酷的,充满竞争的。蓝海战略不是基于竞争,并不意味着没有竞争,只是暂时避开了竞争,另辟蹊径。
军事原则会带来最多启示。

多个维度组成价值曲线,加减乘除创造市场空间。

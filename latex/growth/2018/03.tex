\subsection{03}

\subsubsection{0304}

从人生算法,得质能方程。成就等于核心算法和大量重复动作的函数关系。
乾之九三:终日乾乾,与时偕行。终日乾乾,反复道也。
在取势的基础上,明道,优术,如切如磋,如琢如磨。

施振荣的王道经营,创造价值,平衡利益和永续经营。六面向价值总账和创值兵法。
联系道德经42章,生-冲-和的宇宙观,生对应创造价值,冲对应平衡利益,两者之和,是太极,是永续经营。

创造价值为阳,平衡利益为阴,阴阳平衡,方可永续经营之目的。价值为何?由什么构成?
分为隐显三对:现在-未来,直接-间接,有形-无形。隐显并重,在第一曲线的基础上,开辟第二曲线。
创值兵法:一以贯之,以终为始,吐故纳新,价畅其流。

华为文化:以客户为中心,以奋斗者为本,长期坚持艰苦奋斗。也是一生二二生一,两仪合而为太极的结构。

易经、道德经、孙子。

回到闭环认知上来。围绕核心,高度运转认知闭环。感知-认知-决策-行动。外环还不能很好地理解。
两仪、双线、圆点都体现分解的智慧,在分合为变中,求其大成。

识别核心是最重要的事情,打磨价值观的好问题:什么更重要?什么最重要?

另外就是选择平台,努力赶不上结构因素。小平台也有好处,但需要具备更好的成长性。


\subsubsection{0310}

用问题来引导方向。收集各类关键问题,持续跟进。

庄子天下篇:建之以常无有,主之以太一,以濡弱谦下为表,以空虚不毁万物为实。太一之说,精当。

乾坤并建以为功,复其见天地之心者乎?

易道:体卷与用卷,前若圆之心,后若圆之轮。圆心在于提取而立,圆轮在于施展而从。心立轮从,前定后变。
立于定圆的道理,从于变通的方法。

于是,提取者不厌其精,施展者不厌其广。道理越精圆,方法越广通。精圆于有极,广通于无限。有极与无限,
道理与方法,内定外变,互承同在。是故,一心定无涯,全文变二幅。一体一用,前立后从。圆之一说,通之二卷。
概曰:体卷、用卷。

一体万化

鬼谷子转圆

最小系统,从最小必要知识出发,应对人生百态。公理化系统如此,学习任何知识也当如此。找到关键点,重点突破,再及其余。

行星的椭圆轨道,与太极图的相似性。椭圆轨道的特征是具有两个焦点,行星围绕恒星作周期运动。

以正治国,以奇用兵,以无事取天下。此语极高明。

机发论,破局点。“安汝止,唯几唯康”,不习无不利。找到事物的破局点,则可以小博大,不劳而成。
习,含妄动之意。无为在于道法自然,而达无不为的效果,以无事取天下。

贵无,抱一,取势。枢机之发,荣辱之主。将欲动变,必先养志、伏意、以视间。

按最小系统的原则,给出自己的哲学,以指导以后的思想和行动。

一切实际事物都是多项不同组元合成的。

道是实际的综合因果性,理是认识的分析因果性。

周易管理智慧:三才定目标,大象理拐点,流程定框架。

奇正:

凡战者,以正合,以奇胜。故善出奇者,如循环之无端,孰能穷之?守正出奇,何为正?何为奇?

生冲和,冲为对立,和为平衡。创生,对冲,平衡三部曲。描述了事物发生发展的一般过程。

守破离,有守有破而臻于离。守住底线,抓住关键。离,远离平衡态的耗散结构。
守是格局,破是破局,离则螺旋上升,进入新的稳态。

断舍离。想整理出一本手册,记录核心概念,原则,用于指导今后的工作和生活,收到无为而无不为的积极效果。
达里奥的原则,斯宾诺莎的伦理学提供了借鉴。

圆点哲学,双线法则,一分为三,战略几何学,易经战略,战略罗盘。

厘清价值观。

物理是最好的人生指南。包括广为宣传的第一性原理,统一之路上的遐思。最为人知的是质能方程,人生算法、软价值公式、复利公式都基于此。

归核,再出发。

万事万物,归为元素周期表。更为甚者,H占据了大头,这有什么启示呢?最简单的H是构成宇宙的基本物质。

寻找人生算法,必然是基于简洁的形式。如最小作用量原理的普遍性。

元亨利贞,匹配软价值公式。人生算法、复利公式是更基础的单元。


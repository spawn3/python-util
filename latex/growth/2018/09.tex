\section{201809}

\subsection{0901}

\subsection{0902}

\subsection{0903}

战略几何学、神圣几何:圆是时间,四方形、十字架是空间,三角形是存在,构成时-空-存在的结构。

双环系统可以解释一切,双环相交处是太极图。右手螺旋法则。

周末读书,关注到几个概念,心神、机发,心神论是黄帝内经的精髓,机发论是易道主义的理论枢纽。

机是什么?随机而动,机是变动不居的存在,但可以通过思维与实践的方式去认识和把握。
阳明心学的精髓:此心不动,随机而动,就是圆点结构。

一心一意到专业学习上,有道有术两个层次多个层次。所有的事情,都是培养心体。
要留出足够的时间去反思,并记录下反思的过程与结论。

这么多年,很遗憾的一点就是不能一心一意,也就是不诚,身在曹营心在汉,不能全心全力地投入到手头要紧的事情上,
老是觉得另有更大价值的事情,反而导致手头的机会也白白溜走。

今后当从容规划(转动PDCA循环),稳扎稳打,一步一个脚印,去实现目标。

几有多义,主要是微和危。事物的萌芽状态,看不透、想不明白,\hl{惟精是惟一的功夫},博文是约礼的功夫。这是阳明一贯的主张。
守住底线、抓住关键是方法,围绕一转动PDCA循环。

\hl{如何尽快实现财务自由}?四象限,打工、个体户、创业、投资。贯穿其中的是\hl{专业、工匠精神}。
只是有工匠精神依然不够,要有道。立足于当下,什么才是最重要的事情呢?

\hl{乾之九三给出了答案}。乾坤是易的门户,黄帝垂衣裳而治天下,盖取诸乾坤。

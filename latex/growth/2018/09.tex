\section{201809}

\subsection{0901}

\subsection{0902}

\subsection{0903}

战略几何学、神圣几何:圆是时间,四方形、十字架是空间,三角形是存在,构成时-空-存在的结构。

双环系统可以解释一切,双环相交处是太极图。右手螺旋法则。

周末读书,关注到几个概念,心神、机发,心神论是黄帝内经的精髓,机发论是易道主义的理论枢纽。

机是什么?随机而动,机是变动不居的存在,但可以通过思维与实践的方式去认识和把握。
阳明心学的精髓:此心不动,随机而动,就是圆点结构。

一心一意到专业学习上,有道有术两个层次多个层次。所有的事情,都是培养心体。
要留出足够的时间去反思,并记录下反思的过程与结论。

这么多年,很遗憾的一点就是不能一心一意,也就是不诚,身在曹营心在汉,不能全心全力地投入到手头要紧的事情上,
老是觉得另有更大价值的事情,反而导致手头的机会也白白溜走。

今后当从容规划(转动PDCA循环),稳扎稳打,一步一个脚印,去实现目标。

几有多义,主要是微和危。事物的萌芽状态,看不透、想不明白,\hl{惟精是惟一的功夫},博文是约礼的功夫。这是阳明一贯的主张。
守住底线、抓住关键是方法,围绕一转动PDCA循环。

\hl{如何尽快实现财务自由}?四象限,打工、个体户、创业、投资。贯穿其中的是\hl{专业、工匠精神}。
只是有工匠精神依然不够,要有道。立足于当下,什么才是最重要的事情呢?

\hl{乾之九三给出了答案}。乾坤是易的门户,黄帝垂衣裳而治天下,盖取诸乾坤。

\subsection{0904}

乾坤是易的门户,易是通向现实世界的门户。这是非常重要的论断,因为一是学易之法,二是用易之法,学以致用,解决现实问题。
读书不在乎多,学宗大易,一部易经观天下。透过一部易经,而通达于现实世界,得偿所愿,心想事成。
通过易,撑起开物成务、进德修业的英雄梦想之旅。

六爻之动,三极之道。分而论之,初二为地道,三四为人道,五上为天道,匹配几、诚、神。

用\hl{三级火箭模型}分析创业公司的发展轨迹和着力点,什么是发动机?如何一环套一环?
产品和客户是任何公司的两极。设计理念与客户反馈要综合为用。

易之三义,变易、不易、易简。

\subsection{0905}

努力经营事业,开始物色各类人选,看看水浒传、三国演义,任何事业都不会想当然地一蹴而就,而是长期经营的结果。
事业是男人的第一支柱。

易经在这方面有着深沉的诉求,圣人以神道设教,抛开迷信的成分,易经是第一励志书,也是第一帝王书。
学习易经,方术方面了解即可,不作为主要方面,重要的是开物成务、进德修业方面的启发和辅助。

至九四,始入于上层,开启了自己的平台和事业。上下分际处是着力点。
或跃在渊,此一跃是多方面因素叠加的结果,主要还在于自己的野心、理念、认知。
此一跃,不回头。

一是因果律,二是神圣意志之发扬。乾卦就是这样的精神力量,乘云气而御飞龙,高扬进取意识。

更加open地去思考关键问题,包括行业、事业等等。思考、交流都是需要的。
进一步去了解别的产品,主要是把握趋势。

双环系统可以解释一切,双环相交处是太极图。

怎么通向现实世界呢?

\subsection{0906}

不能控制自己的情绪,太幼稚,这种东西纯粹影响发挥。当前第一要务是什么?事业,不容置疑。迄今没有起色。

第一个是专业环,这是安身立命之本。经过多年的摸索,是整理收割的时候了。

第二个是易经,全面拥抱易经,以之作为进德修业的重要基础。以此洗心,退藏于密,洗心,就是淬炼心智模式。
易经在思维方面,有着深度与广度。进入眼界的思维模型,都挂入易经这个思维格栅中去,易经就是太上老君的八卦炉,
淬炼出了孙悟空的火眼金睛。

另外,黄帝内经所蕴含的神本论以及机发论思想,在易经中也有深入的体现。洁净精微,易之教也。

环环相扣,专业与易经之环,碰撞出火花。工作与生活都需要大设计。

不要急、慢慢来。易经为起点,一部易经观天下,指导生活与工作之设计。专业是工作的一部分。
生活是进德,工作是修业,内外兼备,合内外,一物我。

一切的学习都不仅仅为了学习而学习,为了单纯的知识而学习,而是为了解决问题。

关键思想:
\begin{enumbox}
\item 确定易经作为最高指导思想,第三空间或虚或实,主要指的是这个,过有原则的生活,富有之谓大业、日新之谓盛德
\item PDCA作为执行方法
\item 双环系统分别对应生活和工作
\item 把\hl{视点/视角方法}作为架构描述语言
\item B:确定把分布式存储作为主要的技术领域
\item B:确定把QoS作为主要的研究方向之一
\end{enumbox}

\subsection{0911}

全力以赴到专业方向上,去解决关键问题,太极云尔,是反思框架。

心、道、物的三合之道,适合于下一阶段的学习过程。心就是阳明所谓良知,为学头脑所在,多问多思。道,原则,方法论,架构。
物是要研究的系统,要解决的问题。以道观之,以架构之眼看系统,当如庖丁解牛。

双环,一者三合之道,二者PDCA。双环正交?

对解决问题有腻烦心理,问题是前进的动力,当善待之,乐于去搞定她。

\subsection{0912}

心神主宰,以道观之,落实到物,以道的光华普照世界。寂然不动、感而遂通天下之故,这是二重性。

第一个小目标,100w,1000w,以此类推。明年大概就有100w了,坚定地走下去,不急不躁。重为轻根,静为躁君。

架构驱动的软件开发过程。

坚持用SWOT分析,是战略分析的起步。

\hl{本周末给出一个更明确的路线图}。第一,强化架构思维能力,视图视角是标准做法,IEEE STD 1471-2000。
视图可以视点集为模板,也可以单独定义。运用视角到视图之中,形成纵横交错的架构描述。

\subsection{0913}

\begin{shadequote}

能把诚神几统一起来的为圣人。北宋周敦颐在《通书》中提出的命题。“寂然不动者,诚也。感而遂通者,神也。
动而未形,有无之间者,几也。……诚神儿曰圣人”(《通书·圣》)。
诚是静无的,即“诚无为”(《通书·诚几德》)。神“感而遂通”,是诚的直接表现。几处于静无动有之间,是动之始。
诚是纯粹至善的,是一切道德行为的源泉。
神是诚的直接表现,故亦善。只有几“动于彼”,感外物而动,故兼有善恶。
《宋元学案·濂溪学案上》云:“常人之心,首病不诚。不诚故不儿而著。不几故不神。物焉而已。”
常人不能以诚贯几,受物之累而为恶。只有圣人才能以诚贯几,去几中之恶,把诚神几统一起来,故诚神几曰圣人。
\end{shadequote}

心道物,诚神几,有对应关系。把心置于三角形顶点处,似更体贴。

养心莫善乎诚,致诚则无它事。至诚之道,可以前知。惟天下至诚,为能经纶天下之大经,立天下之大本,成天地之化育。

圣人以神道设教,道则通神,一阴一阳之谓道,阴阳不测之谓神。何为神?妙万物而为言者。

几,人心惟危,道心惟微,几则合多义而言。机发论更提出制机的说法,乃易道主义的理论枢纽。
从机发论的角度理解,\hl{黄帝内经}灵枢,\hl{鬼谷子、阴符经}亦然。

\hl{此三角形居于左侧(符合右手螺旋法则),圆形+十字架构成的几何形状居于右侧(SWOT, PDCA, 2x2矩阵及其延伸,符合左手螺旋法则),
左右交错,形成太极之两仪}。大拇指都指向自己,反求诸己,建立自我,贵我通今,时变是守。
此参伍以变,错综其数的义理架构,实有进一步发挥的余地。

左为知、右为行,以此类推,大商之道的道术、变常、方圆、生死、利害、取予之对立统一,也是如此。

孙正义的25字诀,与\hl{周易、兵经百字、东方战略学},都是以字通道的卓越理念。

\subsection{0914}

观象玩辞,以字通道。建勋画论的三合之道,启人深思。道具太极位,则有商讨的余地。邵雍曰:心者太极也。华严经云:心如工画师,能画诸世间。
阳明心学也是如此。心是能动的一面,也是目的性的一面,使心居于太极位,乃应有之义。心秉道通物,心格物穷理,天性,人也,人心,机也。
立天之道,以定人也。此说并不否定或拉低道的价值,而是在建立自我的阶段,高扬心性,确立为学的头脑。道依然是那个道,
致吾心之良知于事事物物,则事事物物皆得其理。即满足了目的性要求,又满足了道的约束性原则性。

欲正其心者,先诚其意。在明道、格物的过程中,诚其意。事上磨练,皆在涵养此心之体。由物及心,完成此逆时针的环转关系。此右手螺旋法则。

如忽略道的环节,而直奔物的主题,则易于陷入事务主义的泥淖之中,事半功倍,乃至无功而返。
如过于强调理论,也有教条主义的倾向。

神者生之本。

\subsection{0918}

系统思考。

职场与理想的距离,靠三度修炼去完成。三度:态度、气度、厚度。读一艮卦,胜读一部华严。
中秋看王明夫主编的三度修炼,好好想一想下一步的规划。

\subsection{0920}

离开HY的可能较大,离不离开,都要以成长为主要标准。时间并不充裕,接下来到年底的一段时间,好好锤炼专业技能。

\hl{优先考虑开启自己的事业},专业技能的学习、知识体系的构建,不能脱离这个目的,才称得上学以致用。

\hl{全闪时代来临,离自己最近,怎能再次错过}?应采用包围式学习,地毯式学习,既要明确关键,又要面面俱到,点线面体,全面展开。

在多副本复制的场景下,由一控制器负责,如果控制器发生切换,则开启新纪元。在某一控制器的生存期内,
每次提交采用单调递增的版本号,所以二元序号的构成:(epoch, version/clock)。
卷控制器可管理很多chunk及其副本一致性,控制器位置与副本位置不具有对应关系。\hl{卷控制器可迁移}。

关于控制器的若干关键问题:
\begin{enumbox}
\item 如何选取控制器
\item 客户端如何定位控制器
\item 控制器发生切换又如何
\end{enumbox}

paxos的精髓是温故知新,一个实例产生一个值。如何标记序号?序号可以是二元结构,方便处理。

multi paxos与RAFT的异同?每一个控制器的生命周期包括三阶段:\hl{选主、恢复、正常操作}。

进一步,传统的2PC、3PC算法的不足和使用场景。这类算法是分布式系统的精髓,务必加以消化理解。

\hl{算法是程序员的金线},理应是下一阶段的重点。比如,通过token bucket或leaky bucket解决qos问题,实现很简单,设计很精妙。

马云定随舍三部曲,第一曲是定字诀。艮,止也。知止而后有定,定而后能静。

\hl{起居有常、饮食有节},乃养生之道,不仅此也,常与节有深意存焉。
财自道生、利缘义取,是大商之道。菩萨畏因,凡夫畏果。

\hl{多听多问}是领导之道,陈述句不如疑问句。

易经的卦图是直线,加上圆点哲学,三角形集两者之大成,融合双线法则、圆点哲学、三点论、一分为三诸论而为一,
算是多年思考、探索的一点结论。三生万物,由此而展开其广泛的运用过程,进入明体达用的第二阶段。
用太极两仪模式解读三角形各点之间的关系。

道是吾观物的门户、工具,不能僭越心的第一性,道物、道器、体用,分阴分阳,叠用柔刚。
\hl{吾有方圆之形}。五代表圆点哲学,PDCA等。以五为食,哈?口为口、为目,以五观之、观天下。

两个三角形,下一个代表物理资源,上一个代表虚拟资源,中间的交点是集群,物理资源总而为一,进一步生化出虚拟资源。

心道物三角,自身也有两种旋转方向,左手螺旋右手螺旋,标准图以心为顶点。

\subsection{0920}

战略一二三,美团十字架,参伍点成圆,乱环诀中诀。

智仁勇三达德,好学近乎智,力行近乎仁,知耻近乎勇。

在乱环之中,存在第一义,找到她!

架构、算法是内功心法,练拳不练功,到老一场空。

功业之心热灼,怎么开始?如何播种下第一粒种子?离什么最近? 立于中央,由近及远。

无所待,此时就是开始!此时此地,从心开始。

开始不难,终局判断如何?商业计划书?开始吧。

人钱事,搭班子、定战略、带队伍。做什么?怎么做?如何解决启动资金的问题?

如何整合资源?一二三级火箭分别是什么?

\subsection{0925}

心道物,以心为开始,以道为顶点,以物为落实。三者太极两仪,环转无尽,融归与太极大道之中。
如此排列,不压抑心的能动创造性和塑造能力。

心何以转物?以道为中介,诚神几,修心贵诚,通道故神,风起于青萍之末,挥之于泰山之本。
上通于道,下及于物。向道的跃迁层层递进,进一层有一层的道理。

任何一点都不足以表达正确的关系。

中秋假期间一个最重要的反思就是要有制胜的意志。\hl{善用兵者,修道而保法,故能为胜败之政}。
举凡人事百端,无不以胜或赢为最终的目的。取胜的方法多端,宗旨则一。

古今中外的人之共识,老庄虽然一直在说恬淡虚无,何尝忘记求以得,有罪以免也,故为天下贵。

从胜的角度,从修身为本三合之道的角度去诠释经典,别有一番风韵。

\hl{立足于专业技能,从战略的角度拓展未来成长空间},战略思维是一项极端重要的能力。

道天地将法,也是一重要的思考框架。尚五,五包含了一二三四,圆点哲学、太极阴阳、三才之道、四象/PDCA。

老子缺少点进取的意味,孙子则攻守兼备。

心到道的距离是认知差,\hl{道是超认知}。在不同的认识面上,相同的公理具有不同的内核,这就是hegel一直说的熟知非真知。
以道观之,在道的高度上,运用简单而普适的公理,可以达到非常好的结果。

人心惟危,道心惟微。危是指称思维的不可靠,微则是思维的神妙不测,真理与谬误在一线之间。
洞察思维的误区盲点,极深研几,就可以越来越接近道的境界。

查理芒格研究思维的错误模式,就是有鉴于此。普通的思维是靠不住的。
波普尔的证伪理论,索罗斯的反身性,都是解决这一问题的哲学努力。
更早,则有休谟的因果质疑。

\hl{枕戈待旦、厉兵秣马},为了最后之胜利,不能不如此。

心的综合能力,读书如果不思考,就破坏了这种能力,显得支离。

为什么要从心开始呢?虚心涵泳,切己体察。

架构师,工匠精神,粟裕尽打神仙仗。\hl{全力以赴投入到专业技能的学习和提升上去},主次不能颠倒。
说别的事情还显得太远,比如和君的国势、产业、资本、管理四库全书等。这是下一阶段的事情。

通过研读阳明学,更主要的是,通过建勋的画道提出的三合之道,确立了基础的思想方法和工作方法。

破局、突破,更上一层楼。

进一步提出经营方针和工作程序。

\subsection{0926}

六经注我,我注六经。阳明学提升了我的价值,先确立我,建立自我,第二步才是追求无我之境。

系统读书,一旦确立了我,读书就是为我所用的过程。志于道,游于艺,六艺摄于一心,如此,心物关系中,心为主,物为从,精神作为能动性的一方面,发挥了更为重要的作用。
即是在格物中诚意,在诚意中格物,尊德性而道问学。

留给自己的大机会不多了,需要极大地发挥精神的能动性,去慎思明辨。四十不惑,处在这个关键的转折点,怎能不好好地把握呢?
机遇偏好有准备的头脑,潜龙勿用,一定要静下心来,苦练内功,打好下一步发展的坚实基础。

战略致胜,战略是道的运用,以道莅天下。孙子兵法计篇:道天地将法。以五行对照之,道立于中央,天地定位,左将右法。
将作为能动性的一方,也不能不受道的制约,取胜的一切要素,都围绕着道而旋转,五行是更具体的模型。道具有目的性和工具性等多重价值。

\hl{搭班子、定战略、带队伍}是柳传志的联想方法,对应到将、道、法横轴上。
将是领导、法是管理。国势、产业、资本、管理,管理是创业之后的事情,且不可过于陷入微观管理的泥潭。
产业才是第一要考虑的领域,在国势下定位产业,资本、管理是随着产业而运转的。

\hl{战略罗盘}从内外、知行两个维度进行划分,从外到内。

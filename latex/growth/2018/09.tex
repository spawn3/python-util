\section{201809}

\subsection{0901}

\subsection{0902}

\subsection{0903}

战略几何学、神圣几何:圆是时间,四方形、十字架是空间,三角形是存在,构成时-空-存在的结构。

双环系统可以解释一切,双环相交处是太极图。右手螺旋法则。

周末读书,关注到几个概念,心神、机发,心神论是黄帝内经的精髓,机发论是易道主义的理论枢纽。

机是什么?随机而动,机是变动不居的存在,但可以通过思维与实践的方式去认识和把握。
阳明心学的精髓:此心不动,随机而动,就是圆点结构。

一心一意到专业学习上,有道有术两个层次多个层次。所有的事情,都是培养心体。
要留出足够的时间去反思,并记录下反思的过程与结论。

这么多年,很遗憾的一点就是不能一心一意,也就是不诚,身在曹营心在汉,不能全心全力地投入到手头要紧的事情上,
老是觉得另有更大价值的事情,反而导致手头的机会也白白溜走。

今后当从容规划(转动PDCA循环),稳扎稳打,一步一个脚印,去实现目标。

几有多义,主要是微和危。事物的萌芽状态,看不透、想不明白,\hl{惟精是惟一的功夫},博文是约礼的功夫。这是阳明一贯的主张。
守住底线、抓住关键是方法,围绕一转动PDCA循环。

\hl{如何尽快实现财务自由}?四象限,打工、个体户、创业、投资。贯穿其中的是\hl{专业、工匠精神}。
只是有工匠精神依然不够,要有道。立足于当下,什么才是最重要的事情呢?

\hl{乾之九三给出了答案}。乾坤是易的门户,黄帝垂衣裳而治天下,盖取诸乾坤。

\subsection{0904}

乾坤是易的门户,易是通向现实世界的门户。这是非常重要的论断,因为一是学易之法,二是用易之法,学以致用,解决现实问题。
读书不在乎多,学宗大易,一部易经观天下。透过一部易经,而通达于现实世界,得偿所愿,心想事成。
通过易,撑起开物成务、进德修业的英雄梦想之旅。

六爻之动,三极之道。分而论之,初二为地道,三四为人道,五上为天道,匹配几、诚、神。

用\hl{三级火箭模型}分析创业公司的发展轨迹和着力点,什么是发动机?如何一环套一环?
产品和客户是任何公司的两极。设计理念与客户反馈要综合为用。

易之三义,变易、不易、易简。

\subsection{0905}

努力经营事业,开始物色各类人选,看看水浒传、三国演义,任何事业都不会想当然地一蹴而就,而是长期经营的结果。
事业是男人的第一支柱。

易经在这方面有着深沉的诉求,圣人以神道设教,抛开迷信的成分,易经是第一励志书,也是第一帝王书。
学习易经,方术方面了解即可,不作为主要方面,重要的是开物成务、进德修业方面的启发和辅助。

至九四,始入于上层,开启了自己的平台和事业。上下分际处是着力点。
或跃在渊,此一跃是多方面因素叠加的结果,主要还在于自己的野心、理念、认知。
此一跃,不回头。

一是因果律,二是神圣意志之发扬。乾卦就是这样的精神力量,乘云气而御飞龙,高扬进取意识。

更加open地去思考关键问题,包括行业、事业等等。思考、交流都是需要的。
进一步去了解别的产品,主要是把握趋势。

双环系统可以解释一切,双环相交处是太极图。

怎么通向现实世界呢?

\subsection{0906}

不能控制自己的情绪,太幼稚,这种东西纯粹影响发挥。当前第一要务是什么?事业,不容置疑。迄今没有起色。

第一个是专业环,这是安身立命之本。经过多年的摸索,是整理收割的时候了。

第二个是易经,全面拥抱易经,以之作为进德修业的重要基础。以此洗心,退藏于密,洗心,就是淬炼心智模式。
易经在思维方面,有着深度与广度。进入眼界的思维模型,都挂入易经这个思维格栅中去,易经就是太上老君的八卦炉,
淬炼出了孙悟空的火眼金睛。

另外,黄帝内经所蕴含的神本论以及机发论思想,在易经中也有深入的体现。洁净精微,易之教也。

环环相扣,专业与易经之环,碰撞出火花。工作与生活都需要大设计。

不要急、慢慢来。易经为起点,一部易经观天下,指导生活与工作之设计。专业是工作的一部分。
生活是进德,工作是修业,内外兼备,合内外,一物我。

一切的学习都不仅仅为了学习而学习,为了单纯的知识而学习,而是为了解决问题。

关键思想:
\begin{enumbox}
\item 确定易经作为最高指导思想,第三空间或虚或实,主要指的是这个,过有原则的生活,富有之谓大业、日新之谓盛德
\item PDCA作为执行方法
\item 双环系统分别对应生活和工作
\item 把\hl{视点/视角方法}作为架构描述语言
\item B:确定把分布式存储作为主要的技术领域
\item B:确定把QoS作为主要的研究方向之一
\end{enumbox}

\subsection{0911}

全力以赴到专业方向上,去解决关键问题,太极云尔,是反思框架。

心、道、物的三合之道,适合于下一阶段的学习过程。心就是阳明所谓良知,为学头脑所在,多问多思。道,原则,方法论,架构。
物是要研究的系统,要解决的问题。以道观之,以架构之眼看系统,当如庖丁解牛。

双环,一者三合之道,二者PDCA。双环正交?

对解决问题有腻烦心理,问题是前进的动力,当善待之,乐于去搞定她。

\subsection{0912}

心神主宰,以道观之,落实到物,以道的光华普照世界。寂然不动、感而遂通天下之故,这是二重性。

第一个小目标,100w,1000w,以此类推。明年大概就有100w了,坚定地走下去,不急不躁。重为轻根,静为躁君。

架构驱动的软件开发过程。

坚持用SWOT分析,是战略分析的起步。

\hl{本周末给出一个更明确的路线图}。第一,强化架构思维能力,视图视角是标准做法,IEEE STD 1471-2000。
视图可以视点集为模板,也可以单独定义。运用视角到视图之中,形成纵横交错的架构描述。

\subsection{0913}

\begin{shadequote}

能把诚神几统一起来的为圣人。北宋周敦颐在《通书》中提出的命题。“寂然不动者,诚也。感而遂通者,神也。
动而未形,有无之间者,几也。……诚神儿曰圣人”(《通书·圣》)。
诚是静无的,即“诚无为”(《通书·诚几德》)。神“感而遂通”,是诚的直接表现。几处于静无动有之间,是动之始。
诚是纯粹至善的,是一切道德行为的源泉。
神是诚的直接表现,故亦善。只有几“动于彼”,感外物而动,故兼有善恶。
《宋元学案·濂溪学案上》云:“常人之心,首病不诚。不诚故不儿而著。不几故不神。物焉而已。”
常人不能以诚贯几,受物之累而为恶。只有圣人才能以诚贯几,去几中之恶,把诚神几统一起来,故诚神几曰圣人。
\end{shadequote}

心道物,诚神几,有对应关系。把心置于三角形顶点处,似更体贴。

养心莫善乎诚,致诚则无它事。至诚之道,可以前知。惟天下至诚,为能经纶天下之大经,立天下之大本,成天地之化育。

圣人以神道设教,道则通神,一阴一阳之谓道,阴阳不测之谓神。何为神?妙万物而为言者。

几,人心惟危,道心惟微,几则合多义而言。机发论更提出制机的说法,乃易道主义的理论枢纽。
从机发论的角度理解,\hl{黄帝内经}灵枢,\hl{鬼谷子、阴符经}亦然。

\hl{此三角形居于左侧(符合右手螺旋法则),圆形+十字架构成的几何形状居于右侧(SWOT, PDCA, 2x2矩阵及其延伸,符合左手螺旋法则),
左右交错,形成太极之两仪}。大拇指都指向自己,反求诸己,建立自我,贵我通今,时变是守。
此参伍以变,错综其数的义理架构,实有进一步发挥的余地。

左为知、右为行,以此类推,大商之道的道术、变常、方圆、生死、利害、取予之对立统一,也是如此。

孙正义的25字诀,与\hl{周易、兵经百字、东方战略学},都是以字通道的卓越理念。

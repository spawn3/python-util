\section{12}

\subsection{1203}

手腕扭伤快好了。真是旷日持久,一点点小问题,竟然如此,不可不慎重。病向浅中医,养生须趁早。

最重要的是下一阶段的工作问题。内壮则外强,不论何时何地,不要心怀幻想。\hl{放弃幻想,准备战斗}是战士的必然选择。
枕戈待旦是为将者的核心素养。

学习中医、太极,一是道,二是术,道术并重,前期以道为主,明理,后期深入术中,以道卫术。
有所偏重,交相互养。从更大的视野去看,道一而术多,都可以看作道的应用,博观约取。

对学习计算机技术也有好处,相互发明,在交叉地带有所领悟。为学日益为道日损。

徐特立老读书学习之法,定量、有恒,不可自乱方寸。

\subsection{1204}

以心行气,以气运身。气是沟通身心的桥梁,区分先天气和后天气,呼吸是后天气,内气、真气从之的是先天气。
随外在呼吸,引领先天气之循环:沿着任脉而沉至丹田,沿着督脉而上达百会,形成闭环。任督通,则成小周天。
吸升呼降,升级的圆运动。医理拳理通。只有丹田聚气养气充沛,自然能打通任督二脉。

活动关节,找到途径各关节的主要穴位进行按摩,以起到拉伸的效果。循经找穴,进而通过经络上的关键穴位进行按摩。

\subsection{1205}

洪范五福:《尚书》上所记载的五福\hl{一曰寿、二曰富、三曰康宁、四曰攸好德、五曰考终命},
东汉桓谭于《新论·辨惑第十三》把“考终命”更改,将五福改为“寿、富、贵、安乐、子孙众多”。
现代常把“五福临门”当作新春祝福使用。

六极:一曰凶、短、折, 二曰疾,三曰忧,四曰贫,五曰恶,六曰弱。

积贫积弱,富强的现代化中国。五福有可为有可为而不可期,如康宁、攸好德、富,一定程度上是可以做的。
这里有因果业力法则在起作用。

\begin{shadequote}
兴五福销六极

问:昔周著《九畴》之书,汉述《五行》之志,皆所以精究天人之际,穷探政化之源。然则五福之祥,何从而作;六极之沴,何故而生?
将欲辨行,可明本末。又今人财耗费,既贫且忧,时沴流行,或疾而夭。思欲销六极,致五福,殴一代于富寿,纳万人于康宁。何所施为,可致于此?

臣闻圣人兴五福销六极者,在乎\hl{立大中致大和}也。至哉中和之为德,不动而感,不劳而化,以之守则仁,以之用则神,卷之可以理一身,舒之可以济万物。
然则和者生于中也,中者生于不偏也,不邪也,不过也,不及也。若人君内非中勿思,外非中勿动,动静进退,皆得其中,故君得其中,则人得其所,人得其所,则和乐生焉。
是以君人之心和,则天地之气和,天地之气和,则万物之生和。于是乎三和之气,䜣合絪缊,积为寿,蓄为富,舒为康宁,敷为攸好德,益为考终命。
其羡者则融为甘露,凝为庆云,垂为德星,散为景风,流为醴泉。六气叶乎时,七曜顺乎轨,迨于巢穴羽毛之物,皆煦妪而自蕃,草木鳞介之祥,皆丛萃而继出。
夫然者,中和之气所致也。若人君内非中是思,外非中是动,动静进退,不得其中,故君不得其中,则人不得其所,人不得其所,则怨叹兴焉。
是以君人之心不和,则天地之气不和,天地之气不和,则万物之生不和。于是乎三不和之气,交错堙郁,伐为凶短折,攻为疾,聚为忧,损为贫,结为恶,耗为弱。
其羡者潜为伏阴,淫为愆阳,守为彗星,发为暴风,降为苦雨。四序失其节,三辰乱其行,迨乎襁褓卵胎之生,皆夭阏而不遂,木石华虫之怪,皆糅杂而毕呈。
夫然者,不中不和之气所致也。则天人交感之际,五福六极之来,岂不昭昭然哉。臣伏见比者兵赋未减,人鲜无忧,时沴所加,众或有疾。
德宗皇帝病人之病,忧人之忧,于是救之以广利之方,悦之以中和之乐,将使易忧为乐,变病为和,惠化之恩,莫斯甚也。

然臣窃闻善除害者察其本,善理疾者绝其源。伏惟陛下欲纾人之忧,先念忧之所自;欲救人之病,先思病之所由。知所自以绝之,则人忧自弭也;知所由以去之,则人病自瘳也。
然后申之以救疗之术,则人易康宁;鼓之以安乐之音,则人易和悦。斯必应疾而化速,利倍而功兼。六极待此而销,五福待此而作。
如是,可以陶三才缪滥之气,发为休祥;殴一代鄙夭之人,臻乎仁寿。中和之化,夫何远哉!
\end{shadequote}

东方治理学中阐述的黄帝四经基本方法论,作为基石。由此而展开为完整体系。理论核心在于体味阴阳中和的奥义。

撞丹田有上中下三个高度,每种高度20次为一组。完成5组为一轮。暂定如此,必须数量化。
\hl{任何活动都考虑量化},包括标准和验收。按由轻到重的原则,逐步加大运动量到一最大值,然后保持,允许有一定波动。
度数信,即是围绕中心轴的圆周运动,弦动方式。务必重视周期和节律。

张三丰打坐歌需要背诵。

意气运动要贯彻到一切运动中,这就意味着尽量慢,体会其中意气运行的节律。
数量并非第一重要,重要的是内在质量。

\subsection{1206}

\subsection{1207}

\subsection{1208}

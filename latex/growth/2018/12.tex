\section{12}

\subsection{03}

手腕扭伤快好了。真是旷日持久,一点点小问题,竟然如此,不可不慎重。病向浅中医,养生须趁早。

最重要的是下一阶段的工作问题。内壮则外强,不论何时何地,不要心怀幻想。\hl{放弃幻想,准备战斗}是战士的必然选择。
枕戈待旦是为将者的核心素养。

学习中医、太极,一是道,二是术,道术并重,前期以道为主,明理,后期深入术中,以道卫术。
有所偏重,交相互养。从更大的视野去看,道一而术多,都可以看作道的应用,博观约取。

对学习计算机技术也有好处,相互发明,在交叉地带有所领悟。为学日益为道日损。

徐特立老读书学习之法,定量、有恒,不可自乱方寸。

\subsection{04}

以心行气,以气运身。气是沟通身心的桥梁,区分先天气和后天气,呼吸是后天气,内气、真气从之的是先天气。
随外在呼吸,引领先天气之循环:沿着任脉而沉至丹田,沿着督脉而上达百会,形成闭环。任督通,则成小周天。
吸升呼降,升级的圆运动。医理拳理通。只有丹田聚气养气充沛,自然能打通任督二脉。

活动关节,找到途径各关节的主要穴位进行按摩,以起到拉伸的效果。循经找穴,进而通过经络上的关键穴位进行按摩。

\subsection{05}

洪范五福:《尚书》上所记载的五福\hl{一曰寿、二曰富、三曰康宁、四曰攸好德、五曰考终命},
东汉桓谭于《新论·辨惑第十三》把“考终命”更改,将五福改为“寿、富、贵、安乐、子孙众多”。
现代常把“五福临门”当作新春祝福使用。

六极:一曰凶、短、折, 二曰疾,三曰忧,四曰贫,五曰恶,六曰弱。

积贫积弱,富强的现代化中国。五福有可为有可为而不可期,如康宁、攸好德、富,一定程度上是可以做的。
这里有因果业力法则在起作用。

\begin{shadequote}
兴五福销六极

问:昔周著《九畴》之书,汉述《五行》之志,皆所以精究天人之际,穷探政化之源。然则五福之祥,何从而作;六极之沴,何故而生?
将欲辨行,可明本末。又今人财耗费,既贫且忧,时沴流行,或疾而夭。思欲销六极,致五福,殴一代于富寿,纳万人于康宁。何所施为,可致于此?

臣闻圣人兴五福销六极者,在乎\hl{立大中致大和}也。至哉中和之为德,不动而感,不劳而化,以之守则仁,以之用则神,卷之可以理一身,舒之可以济万物。
然则和者生于中也,中者生于不偏也,不邪也,不过也,不及也。若人君内非中勿思,外非中勿动,动静进退,皆得其中,故君得其中,则人得其所,人得其所,则和乐生焉。
是以君人之心和,则天地之气和,天地之气和,则万物之生和。于是乎三和之气,䜣合絪缊,积为寿,蓄为富,舒为康宁,敷为攸好德,益为考终命。
其羡者则融为甘露,凝为庆云,垂为德星,散为景风,流为醴泉。六气叶乎时,七曜顺乎轨,迨于巢穴羽毛之物,皆煦妪而自蕃,草木鳞介之祥,皆丛萃而继出。
夫然者,中和之气所致也。若人君内非中是思,外非中是动,动静进退,不得其中,故君不得其中,则人不得其所,人不得其所,则怨叹兴焉。
是以君人之心不和,则天地之气不和,天地之气不和,则万物之生不和。于是乎三不和之气,交错堙郁,伐为凶短折,攻为疾,聚为忧,损为贫,结为恶,耗为弱。
其羡者潜为伏阴,淫为愆阳,守为彗星,发为暴风,降为苦雨。四序失其节,三辰乱其行,迨乎襁褓卵胎之生,皆夭阏而不遂,木石华虫之怪,皆糅杂而毕呈。
夫然者,不中不和之气所致也。则天人交感之际,五福六极之来,岂不昭昭然哉。臣伏见比者兵赋未减,人鲜无忧,时沴所加,众或有疾。
德宗皇帝病人之病,忧人之忧,于是救之以广利之方,悦之以中和之乐,将使易忧为乐,变病为和,惠化之恩,莫斯甚也。

然臣窃闻善除害者察其本,善理疾者绝其源。伏惟陛下欲纾人之忧,先念忧之所自;欲救人之病,先思病之所由。知所自以绝之,则人忧自弭也;知所由以去之,则人病自瘳也。
然后申之以救疗之术,则人易康宁;鼓之以安乐之音,则人易和悦。斯必应疾而化速,利倍而功兼。六极待此而销,五福待此而作。
如是,可以陶三才缪滥之气,发为休祥;殴一代鄙夭之人,臻乎仁寿。中和之化,夫何远哉!
\end{shadequote}

东方治理学中阐述的黄帝四经基本方法论,作为基石。由此而展开为完整体系。理论核心在于体味阴阳中和的奥义。

撞丹田有上中下三个高度,每种高度20次为一组。完成5组为一轮。暂定如此,必须数量化。
\hl{任何活动都考虑量化},包括标准和验收。按由轻到重的原则,逐步加大运动量到一最大值,然后保持,允许有一定波动。
度数信,即是围绕中心轴的圆周运动,弦动方式。务必重视周期和节律。

张三丰打坐歌需要背诵。

意气运动要贯彻到一切运动中,这就意味着尽量慢,体会其中意气运行的节律。
数量并非第一重要,重要的是内在质量。

\subsection{06}

\subsection{07}

本周基本理解了RDMA,下周需要继续,列出以学习计划。

修炼也渐渐进入状态,坚持的并不算好。还需要加大力度和决心,通过一定手段调理身心是一辈子要进行的工作,不能轻视。
核心理念就是通经络,调气血,致中和。这也是世界运行的方式。

张首晟去世事件令人震惊,跨界资本失败造成的?也发人深思,\hl{专业是立身之本},下一步需要进一步深入去学习。
丹华资本是一步错棋?投资决策过于乐观?默默地做好自己的专业,再考虑趁势而起。不要给自己太大压力。\hl{从容中道乃最佳策略}。

\hl{太极混元桩站起},静中有动,气息流动,一刻不停。

每次集中攻克一个问题,目前静坐中,发现一些痛点。按瓶颈理论去处理。

\subsection{10}

站桩,太极混元桩、三体桩。为什么说万法出自三体桩。从无极桩、混元桩练习开始。
有足够的时间慢慢练习、体悟其原理。

动静交养,静坐、站桩、内家拳,都需要\hl{调心、调息、调身}。一呼一吸为一息,呼吸于生命而言,极为重要。
此中反思,对今后若干年具有重要价值。生活习惯、对生命的理解在这个过程中得以深化。
静坐、站桩、丹田功这些基本的内功心法,需要坚持下去,\hl{循序渐进、持之以恒}。信解行证,度数信。
基本方法无它,就是围绕一中心点,日积月累,以达豁然贯通之境。

\hl{脏腑、经络都可以看作中医的象数模型}。取象比类,就是一种模型思维。
至于模型是否反映真实情况,需要在实践中进行调节。

HY下一步会很艰难,是否陪着走下去是一个重要的抉择。是否有利于今后长期发展?是否有大的突破?
当断不断,必受其乱。选择的最重要标准就是下一步的发展。平台、领导、行业都极其重要。

简化简化再简化,放弃幻想、准备战斗,机会主义要不得。

\subsection{11}

平时行坐站卧都要注意姿势,功夫要下在平时,用正确的理念塑造良好的生活习惯。这个才是细水长流之道。
功夫怎么才能上身?不是机械地去练习,而是全身心地投入,用心领悟其精义。

体育运动与专业学习一样,遵循相同原则,如循序、有恒。刻意练习的理念,怎么用起来?
专项训练,全面提升。如何设计一套切实可行的健身法?应该包括:
\begin{enumbox}
\item 站桩
\item 静坐
\item 八段锦
\item 太极
\end{enumbox}

一些小敲门:
\begin{enumbox}
\item 腹式呼吸
\item 撞丹田
\item 跪式
\item 刷牙
\item 扣齿
\item 梳头
\item 提肛
\end{enumbox}

禅坐先不求速效,每天盘盘腿,重点关注一些痛点,就会有进步。日拱一卒的精神。
每一种功法都需要大量的时间积累,才能见效果。一个时期最好只有一种重点项目,待步上正轨时,再开始下一项。
采取一主多辅的架构。比如本阶段一撞丹田为主,以站桩、静坐为辅等等。
最后把所有的功法九九归一,抱元归一,回到旋极图的象征。

老子四十二章是最高哲学。\hl{道生一、一生二、二生三、三生万物。万物负阴而抱阳,中气以为和}。
这几句话包含了终极真理。要从中出发,展开为现实的力量。

近期的沉淀期,有其价值,沉下来,再出发。收敛到中间一点,收放自如。潜龙勿用,阳在下也。复其见天地之心。
最重要是保养此一团阳气,以直养而无害,则塞乎天地之间。越养越精神,以做持久战。

悟中道之理,成炼金术士。炼金术士者,能化腐朽为神奇。河洛五行,中土最为贵,乃调理转化之器。

放弃吧,现在变成很大的负能量,就这样不明不白的,有什么意思?现在的主要精力,应放在下一步的健康发展上。
真是内忧外患,有陷在泥潭里的感觉,世界那么大,为什么不走出去看看?

可以了解很多,最重要的确实基本功。

\subsection{14}

书本知识往往已经过时,紧跟会议、论文、各公司的实践活动。

\subsection{17}

\hl{大动不如小动,小动不如不动,不动之动乃生生不已之动}。此语精妙,极有启发。让人回归本源,无为而无不为,故大成拳是无为法。
剔除枝叶,一意本源。

在技术上事业上都有很好的启迪。处当今之际,HY可谓内忧外患,风雨飘摇。更有追问本质,以图活下去。
在技术上,一切围绕ABC,又分主次。主为linux、ceph等。通其一,万事毕。

大成拳心法功法俱为上佳,可以为一段时间的探索画上一个句号了。医武同源,同臻于道。道零德一而万物化生。
信解行证是一循环,转动不已。进而,立禅即意,此中禅意,渐渐融入生活中,行住坐卧,无往不在禅意中。
则与道合一。往日学习,偏于知解,缺乏体证,遂茫茫然不知所归,没有受用处。

大成拳与阳明心学都在唤醒自然本具之良知良能,栽培涵养,心中分别心生,则离道日远,佛理深邃,不可思议。
水性太极,妙悟圆觉,大成立禅,归心金刚。都需用心体证,勿自限于文字知解,何况不能达于文字般若,
妄生议论,枉费口舌,大可不必。

至此,各方面均有妥当安排,可以心无旁骛,尽心驰骋了。艺宗AB,拳归大成。

不管HY如何,这次一定要走。初步定为xsky,离家距离不算太远,发展势头蒸蒸日上。
最重要的是,与手头做的最为接近,可以全力以赴投入技术的进一步深造。

也需要安排几个候选,如京东云、联想、首都在线。至此,对下一步的职业规划基本定位清晰。

\subsection{18}

想不到华云窘迫至此,真是可叹!

贪多嚼不烂,全闪是唯一机会。

放弃幻想,准备战斗。积极准备找工作,静观其变吧。损失云云,不作为主要考虑项。
永远往前看,修之于身,其德乃真。

全用户态SDS,就照着这个目标努力。把握好AFA这个风口。
怎么建立相关知识体系呢?

运用心的综合能力,不要被细枝末节所遮蔽。

\subsection{19}

\subsection{20}

数据为王,存储是关键,必须坚持深入,建立完整的知识体系,同时有一门深入的定力和决心。
致广大而尽精微。

领域
\begin{enumbox}
\item 操作系统
\item 数据结构和算法
\item C/C++
\item RDMA
\item DPDK/SPDK
\item NVMe
\item FusionStor/Ceph
\item 其它分布式存储系统 (FS and DB etc)
\end{enumbox}

现在是一个重要的发展阶段。静下心来,好好转动PDCA循环。

linux io path非常重要,direct io绕过page cache,故需要对齐。aio依赖于direct io,也需要对齐。

块设备驱动的分层架构:\hl{vfs,page cache,general block,scsi},主机通过scsi协议连接到磁盘。
主机通过pcie的NVMe连接到NVMe SSD。

为什么1M块的读写如此慢?

latency和iops的关系:拿行驶在路上的车为例,没有饱和时,latency不受iops的影响;达到饱和后,如何维持稳定的iops就是一个大问题。
另外,故障对iops的影响也是设计上的大问题。故障后,进入降级运行状态,需要标记,以利恢复。\hl{论述并发和故障对iops的影响}。

\hl{理解异步操作},异步操作不同于非阻塞操作,依赖于非阻塞操作。aio、NVMe、RDMA、DPDK都采用了异步通信机制。
提交任务到队列,在polling线程里处理完成事件:
\begin{enumbox}
\item 提交
\item 完成(事件驱动、或PMD, callback)
\item 保持上下文
\end{enumbox}

提交和完成可以在同一线程里。

在异步操作之上可以构建同步操作。如rpc,提交请求后,进入yield状态;完成请求后,resume。
用状态机、或协程实现,原理都一样。

\hl{线程+队列}是中异常强大的模型。

\begin{enumbox}
\item aio syscall与eventfd结合,可以纳入epool/pool/select机制之中。
\item NUMA/cpuset和hugepage
\item 增删节点(异步化)
\end{enumbox}

其它如timerfd,signalfd都可以纳入epool机制,提供事件等待和通知机制。

启动若干aio polling线程,提交或检查完成状态。在aio api之上,用poll作为事件通知机制。
用到了两个eventfd,一个用于提交请求的通知,一个用于检查完成状态。

线程加一控制对象,对象包括上下文信息、队列等。

sqlite操作采用了类似机制。

\subsection{21}

linux内核分析和应用、大规模分布式存储系统是两本好书,按此知识体系按图索骥,\hl{致广大而尽精微}。
但这里的主题不够全面、深刻,更新更深的主题和知识点通过微信公众号、论文、blog等加以补充。

把握行动学习两原则和刻意练习、PDCA、黄金圈发展等基本理念,来指导自己的学习过程。

怎么高效地利用资源?

NUMA架构的cpu资源如何利用?线程core binding,私有hugepage分配器。
上层网络连接等任务hash到不同的core thread里。在一个core thread里处理提交、完成检查等基本操作。
同步操作,如io、网络、数据库访问,派生独立的线程池去处理。
如何做到内存本地化?如何做到thread均匀地映射到NUMA节点上。

如何减少上下文切换和数据copy的开销。

可抽象出独立的内存分配器,每个core有一个内存分配器的实例。

从线程的角度看,分为事件循环线程和工作线程,通过队列把各个线程连接起来,actor and channel模式、csp模型。

分析烧开水这个过程。打开开关后,我就可以走开去处理别的事情。由电水壶独立处理烧开水这个事,等它处理完成后,会鸣笛通知。
我得到通知,就可以使用其中的开水了。如果有多个电水壶,它们就可以并行工作。(\hl{用UML序列图来描述})

领导安排任务也遵循同样的逻辑。通过比类取象的方法去理解,并不难。

\hl{网中网}:主机与磁盘的连接,也可以看作是网络。如NVMe SSD,就是NVMe over PCIe,运行在PCIe上的NVMe。
普通的disk也是如此。PCIe采用串口技术,比并口更快是因为具有更高的频率。
从协议上看,NVMe对SCSI的优化体现在哪些地方? NVMe标准依然在快速演进。

与网络连接不同的是,主机与设备的连接是本地的,不是远程跨主机的。

从架构上,本地存储引擎这部分代码应该是简单的。本地磁盘管理:
\begin{enumbox}
\item pool与disk具有一对多的映射关系
\item 每个盘是个状态机
\item 磁盘多态:支持normal and NVMe disks
\item 分配磁盘位置,首先选择盘,再选择盘上的某个位置
\item 检查磁盘故障,并修复数据
\item 引用计数技术
\end{enumbox}

每个盘有属性和方法,包括分配、读写。每个disk有独立的分配工作线程。
一次分配请求,大部分情况下,是一块盘去满足。如果一块盘空间不足,就需要多盘。
要能够表达非连续的离散空间。请求端是同步操作,用一个lock来实现。由工作线程unlock。wuw序列。

normal disks采用aio方式,NVMe才有自己的方式(kernel bypass)。

chunk位置映射,有cache,读多写少。数量大,cache采用LRU置换算法。
\hl{要充分理解各种各样cache的重要性。作为一个重要的主题去掌握}。

参考lich架构,用c++写一个分布式全闪产品,是一个重要的想法。
这样,既可以从设计者的角度去理解lich,也可以引入一些优化项。
最重要是精简,不是一日之功,需要持久战,长期专注思考和coding。
就是她了。

\subsection{22}

lich采用了epoll+aio的io模型,epoll+同步io会引起线程主循环的效率。可以从两个层面看这个模型,
堵塞在epool上,就绪时进一步引入aio。事件和实际的io操作。aio是通过工作线程和多个队列共同完成的。

transfer,tcp和rdma,是可以共存的,都可以理解为条条大路通罗马,信息高速公路。在c/s之间建立了一条虚拟通路。
这条通路是在第三个对象的辅助之下建立的,监听socket。监听socket和连接socket可以统一地在一个事件处理框架内去处理。
针对每个socket,有独立的处理过程。

自转、公转两个环,polling处就是中心所在。消息有自转,也围绕中心公转,是双环,两个层面的事情。

提交和完成是两个过程,可以放在两个独立的线程里,也可以合二为一,用一个独立线程去做。队列需要各自独立。
队列+线程是强大的编程模型。如SEDA架构所描述的,可以有很多的变体。可以用来实现异步操作。

中断指的是cpu处理的中断,有外部事件发生时触发,cpu收到后,进入中断处理程序。与进程调度一起来考量。
中断时理解问题的一个关键概念。线程与协程的主要不同就在于此。线程可抢占,协程不可抢占。进程进入wait态后,需要中断去唤醒。

\hl{等待事件的线程,被中断唤醒后,重新加入ready队列,可以被调度执行}。

进一步,要理解mmap和sendfile的工作机制和带来的好处,主要从syscall讲起,如何减少上下文切换和内存copy的成本。

取象比类的方法理解计算的世界。

多存储网段,有多个port互联,若做bond,就是合成一条。
异步操作,烧开水。性能指标,汽车在路上行驶。确实有生动形象、易于理解的效果。中医的经络、藏象,方法上是相通的。
这是费曼提倡的方法。操作系统里的关键问题和技术,都是通过故事的形式引入的,如生产者消费者,哲学家就餐,背包等。

这比一味地在api之间晕头转向要好得多,并不是说api不需要掌握,而是明理后api的设计就是自热而然的过程。
这样设计出来的api自然贴切,持久稳定。

用解剖麻雀的方式解剖lich,认识其亮点和不足之处。在与ceph对比的过程中,进行创造性综合,集大成。

本周通过分析异步操作,终于有了贯通感,网络编程,包括TCP和RDMA、NVMe、iSER等,都采用了相似的设计模式。
深入理解进程、线程和协程,加上对队列的理解,就可以理解大部分问题了。

进而理解lua、erlang、go的协程模型。

道法术器,器是产品、系统,术是实现技术,法是架构和算法,道是原理。明理、善学,在术的层面要多进行刻意练习。

渐渐觉得,单纯看书不是学习的最佳方法。看书是第二位的事情,第一位的是在头脑里进行的综合分析和判断,就是用意不用力。
比如学习网络编程,大部分的书籍,都是在展现一个知识体系,至于最佳实践,往往显得过时而且没有针对性。

边想边做边读书,效果会更好。一直以来,有过于偏重读书的习惯,反而迷失了宗旨大义所在。
每本书,在知识的程序上,有重点和等级的不同。如linux环境编程更多讲一个一个api,以及api在linux kernel的实现方式。
这一层面是深入的知识,但从应用的角度去看,却远远不够直接。

道法更多是理念层面的,术器是实物层面,术是建筑系统这一大厦的一砖一瓦。道法则是指导建筑的原理和方法。

投资护城河,养生之理明,则进入专业和财富的领域,大道至简、一以贯之。

\subsection{23}

专业能力就是目前的一,有这个才有谈得上一生二后续过程,得一以为天下式。万不可再不重视了。全力以赴即可,至诚无息。
中庸是一部经典。

易经、道德经、中庸统摄黄帝内经、四经,一内一外,内圣外王之道备。

先从单机和分布式开始。

不应该开启超线程,会降低单个core的性能。NUMA节点内多核与内存是同一距离。在malloc时,可以注册RDMA。

完成一个任务可以有多种多样的方式。自己处理,或委托给别的线程处理。
自己处理比较简单,就是block,或非block的同步调用。

引入\hl{io multiplexing}的意义:监听多个描述符,只处理ready的描述符。
这就多了一层,成为两层,带来了非常强大的能力。

\subsection{24}

总是太乐观,HY处境一至于此。内忧外患,真是举步维艰。
抓紧呀,抓紧,真是危在旦夕,不能不做最坏打算。

下一步学习重点:\hl{操作系统、编程语言、算法和数据结构}。
这是最基础的,从这里引申出来,如存储、网络、文件系统、数据库等等。
CBA也离不开这个基础。勤练基本功,是做好任何事情的秘诀。

进程、地址空间、文件是OS的重要抽象,务必深入去理解。
磁盘等外设也是文件。io路径是个重点。ioctl能做什么?

线程是可调度的最小单元,同一进程的多个线程共享\hl{进程地址空间}。

外设中断cpu执行,进入中断处理程序,重新调度。
调度器具有最高权限,所以进程是可抢占的。

外设具有控制器,包括寄存器、数据缓冲器和调度算法。主机与外设如何通信?外设的数据缓冲器独立于进程地址空间。
direct io是越过page cache层,直接进入设备缓冲器,所以需要对齐到块边界。

删除卷或快照,分摊到各个节点上并行执行,如果引入回收站的功能则最好。
在存在离线节点的情况下,不应该违反\hl{safety和liveness}性质。

timer如何实现?

提交:直接提交,或提交到队列,然后通知负责提交队列的线程。所有的事件,都可以\hl{统一到epoll}框架内,
如enentfd,signalfd,timerfd,以及aio、socket fd等。

分解事件检查和实际处理过程。

线程
\begin{enumbox}
\item 一个core对应若干aio线程,
\item 一个disk对应一个allocator线程,
\item 一个sqlite db对应一个线程。
\end{enumbox}

都涉及到队列和多线程同步。队列的位置有所不同,有的归工作线程,有的归提交线程。
这存在非常大的灵活性。两种模式:\hl{hash到工作线程,工作线程polling task}。

协程是用户空间线程,依托于内核线程对象,不可被抢占。
因为OS感知不到,内核线程对象才是调度的最小单元。

协程维护1+N个上下文对象,N是协程的上下文。

\subsection{25}

VM failover,源vm自杀,目标vm才能启动,lease机制。

着重\hl{思考痛点和难点}。lich的难点在于故障下io无中断,数据副本一致性。

虚拟化和容器,都是需要探索的技术领域。要在\hl{操作系统、编程语言、算法和数据结构}的基础上,做出技术的Y型知识结构。

smartx的技术要求:分布式paxos、raft、etcd、zookeeper等。本地存储、存储协议(iSCSI、NVMf、SPDK等)。

基本功
\begin{enumbox}
\item 操作系统:io path, vfs、aio
\item 编程语言:c/c++,python、go、erlang等。
\item ext2/3/4, xfs, btrfs, f2fs, block layer
\item LevelDB/RocksDB
\item iSCSI, NFS, samba以及其它存储协议
\item HDFS/Ceph/Sheepdog/GlusterFS
\item SSD IO性能优化,有FTL开发经验
\end{enumbox}

每个core thread有线程本地变量,指向aio指针数组,共4个,前两个用于direct io,后两个用于元数据的sync io。
同一磁盘设备文件,以不同模式打开多次。这是fd与inode的多对一的关系。fd对应的对象一定存有状态信息。

redis的数据模式:每节点若干redis实例,每个实例保存:disk,metadata,raw记录。raw记录按vol组织成hash。
支持的操作:
\begin{enumbox}
\item 分配chunk
\item 回收chunk
\item 删除卷
\end{enumbox}

\hl{NUMA架构的拓扑结构},如何分配core和aio线程,如何分配内存?
接着看polling线程如何管理私有内存。

cat /proc/cpuinfo, cores与sibling相等,则没有开启hyper threading。若sibling是cores的二倍,则开启了超线程。
还有别的可能吗?

polling线程在NUMA节点之间均匀分布,aio线程优先选取超线程,其次选取同一物理cpu上的其它core。总的原则是局部性。

\hl{从core的初始化过程看起,接着重点关注polling线程的工作}。
外部线程如何与core线程通信,各类事件如何注册到core线程的?

统计内存使用量,统计各类资源的用法

怎么理解syscall,怎么理解内核空间和用户态?上下文切换和内存copy。如read过程,如何分析?如何优化?

\subsection{26}

从主体出发,从进程出发。进程调度、内存、文件和IO等。中断,异步机制。

每个cpu都有自己的线程队列,亲和力是进程的一个属性,指定可以运行在哪些cpu上。就是可以对应cpu的调度队列。

NUMA节点对应的内存,\hl{内存条应该在NUMA节点上对称插入}。否则,没有内存的NUMA节点需要访问远程内存,影响性能。

cpu拓扑是一棵树,叶子节点是逻辑cpu。polling线程binding到逻辑cpu上,
相应的aio线程选择与polling线程\hl{所在cpu最近的那个逻辑cpu}。
\hl{NUMA节点,多处理器,多core,超线程}会增加这棵树的层次。

NVMe设备自动发现会成为一个设备文件,通过kernel io路径进行存取。unbind设备驱动程序后,可以通过pci进行访问。
设备接入bus,通过port空间或内存映射访问设备。

在第一阶段,从各个NUMA节点上平均分配hugepage(mbind)。 hugepage与NUMA节点的这种关系得到保持。
mmap: posix\_memalign虚拟地址空间,然后mmap到对应的hugepage。然后初始化管理元数据,包括pages和buddy。
这个地方的代码可以优化,用统一方式管理global和private内存区。

为什么要用虚拟地址获取物理地址?方法是什么?物理地址在什么情况下被用到?

可以抽象出hugepage region这样的概念,用于管理多个hugepage,需要处理如下需求:
\begin{enumbox}
\item 指定NUMA节点
\item 用buddy算法管理多个hugepage的分配
\item 管理虚拟地址到物理地址的映射
\item 用统一方式管理global和private内存区
\item 统计内存区使用情况
\item 不需要借助hugetlbfs
\end{enumbox}

恢复策略:需要增加维护模式,在此期间,不进行数据修复。

学习路线图:
\begin{enumbox}
\item core
\item mm
\item rpc (tcp and rdma)
\item aio and nvme
\item iscsi/iser
\item NVMf
\item etcd
\item ***
\item snapshot
\item tier
\item bcache
\item EC
\end{enumbox}

还有什么可以改进的?

epoll机制的通用性
\begin{enumbox}
\item fd and socket
\item eventfd
\item signalfd
\item timerfd
\item pipe
\end{enumbox}

作为\hl{异步事件的监听机制},具有广泛的适用性。其它类似sem wait and post, \hl{wuw加锁机制}等。都可以用epoll来实现。
真正体现了event driven的特征。

epollout事件标示缓冲区可写,cond变量表示可能性。惊群现象。

\subsection{27}

六种同步方法:\hl{mutex, cond, sem, flock, spin, rwlock}。sem可以用于共享内存的同步,flock用于文件同步。
mutex, cond, spin, rwlock多多于多线程的同步。注意它们自己的不同。

mutex, spin, rwlock使用场景相似,性能有差别。涉及临界区和对共享变量的访问。
cond与mutex结合起来使用,是否一定要结合mutex?wait在一个外部条件上,这个外部条件/变量被多线程改变。
sem的使用场景又有所不同。如同步线程创建过程。串行化多个线程的创建过程等等。另外\hl{支持timedwait和try语义}。
至于文件锁,使用场景易于确认。

sem的wait和post是由不同线程执行的,一般的mutex、spin、rwlock则最好是由同一线程执行。否则会复杂化执行流。

简单的同步比较容易实施,比如多线程下保护一个全局变量。但要保护一个大型的数据结构,就比较困难,需要遵循一定的加锁、解锁协议/约定。
\hl{2PL,tree protocol}都是这样的协议。

场景一,一线程wait,另一线程需要唤醒它。用sem wait and post可以实现。

场景二,\hl{生产者在队列满时block,消费者在队列空时block}。队列满是一个条件,队列空是一个条件,所以需要两个sem。
同时对于队列长度的变化,需要加以并发同步。

\hl{所谓条件,只是可能性,而不是确定性的}。收到通知后,再次检查条件,进行相应处理。所以,这里通常是一个循环。

卷chunk树的cc比较复杂,每一个chunk都是一个cc的保护单元,又形成了层次结构。

\hrulefill

WAL和db共同构成完成的数据。这是大数据的alpha架构,什么是完整数据?如何构建完整数据。
为什么要先写入WAL?因为写入WAL相对性能高。WAL要支持UNDO和REDO操作。\hl{ARIES事务过程}。

fusionstor是没有commit log的,如何保障事务性?需要保障事务性吗?什么是顺序一致性。
严格\hl{区分提交的数据和未提交的数据},对两者的要求截然不同。
采用的clock机制,带来了系列问题,如周期性合并clock导致iops抖动,掉电情况下clock丢失,引起大量的恢复流量。unsafe\_clock机制的不自然。

\dotfill

一旦理解\hl{线程+队列}这种模型的强大威力,则scheduler、SEDA、alpha都比较容易理解了。
本质上是线程以及线程通信方式,有很多种组合情况。

线程模型有多种:M/1,1/1,M/N。按线程与KSE的比例。\hl{协程是M/1实现,pthread是1/1模型}。

scheduler resume当然可以由外部线程调用,只需要传入taskid参数、返回值和返回数据即可。
所谓resume就是重新加入runnable队列,进行重新调度。
被调度器选中后,继续执行schedule\_yield1的后半部分代码,返回值和数据也跟着传出。

\hl{scheduler和task的关系是1+N的关系。task可以用状态机来描述}。\hl{现代操作系统}一书对进程的描述,采用了这种方式。

\dotfill

用户和组是个正交的话题,留待最后再看。

进程和线程是活动的实体,即主体。内存分配、文件io等等都是在进程内执行的。
从一个到多个主体,引入了\hl{并发情况下通信和同步}的需求。

先理解同步机制,再理解IPC机制。最重要的IPC机制包括file和socket。
通信有\hl{共享内存和消息传递}两种基本形式。结合\hl{erlang和go语言的运行时模型}去理解这些概念。

服务端编程不能不深入理解epoll、aio等高级io特性。进一步理解其kernel实现原理。

\hrulefill

\hl{md5sum的计算},要合理安排\hl{存储和计算的位置}。计算放在控制器上去做是否更高效?
md5sum的计算是一个序列化的过程,只能一个完成后,进行下一个。如果vc切换,能否接着算?

按SEDA架构,组织成流水线计算,分两阶段:读取、计算,第一阶段可以并行,第二节点需要串行处理。
即fork and join并行模式。无法采用MapReduce计算框架。

\hl{卷在存储池间的离线复制和迁移},在线方式呢?会更复杂。
如果卷上有快照呢?从产品角度如何定义?

多核情况下,服务器端架构为多层:\hl{多线程/epoll/多线程(aio等)}。
先hash道事件循环线程,对每个事件循环线程,不能出现任何堵塞式操作。
堵塞式操作如何处理?采用异步方式转化之。

如果解决polling模式下cpu占用率过高的问题,是否能检测到\hl{中断和polling两种模式}之间的临界点,而作动态切换呢?

内存使用量过高,可通过动态化来解决。

\hrulefill

改进学习方法,不求面面俱到,而要抓住重点概念,围绕关键概念进行延伸。
比如,如何理解async编程?由此可以引申出很多知识点。甚至包括erlang、go等语言的设计理念。

观摩优美的设计,临摹,守破离。书本是第二位的,第一位的是综合的理解和判断能力。

面向模式的软件架构,渐渐能读得懂了,\hl{从问题出发,边学边干,边干边学},达到知行合一、得心应手的妙境。

lich采用父子双进程结构,父进程监听子进程,如子进程异常退出,则重新启动子进程。
关键的业务逻辑在子进程内通过多线程的方式完成。
其中的一些线程引入了独特的调度机制,实现为协程,如polling线程。

参考重要的开源项目,如\hl{nginx、redis、memcached、erlang、go}等,借鉴其解决一些重要问题的理念和方法。
一方面夯实基本功,另一方面博观约取,提高创新能力。在此一下一上的过程中,驾轻就熟。

\hl{研究nutanix的产品和技术}。

\subsection{28}

go语言的调度机制,go routine可以在不同线程之间漂移,一般情况下是亲和性的。
routine stacksize大约为2k,应该是可以动态伸缩的。所以其数量级远远大于线程,但依然是有限的。

一个goroutine,在执行堵塞式调用的时候,需要剥离处理,用独立的线程池进行处理。否则,会堵塞住调度器的主循环。
对异步io,当不存在此问题。

协程与状态机模型是等价的,怎么进行等价变换?

调度器和协程之间的关系是1/M。每个实体都维护有上下文信息。有了上下文信息,就可以在它们之间自由穿梭。
调度器的任务就是选出下一个要执行的任务。任务执行完成或yield之后,返回调度器,继续进行调度。

调度器和协程的上下文信息是如何创建的?

\hrulefill

有三套rpc,rpc和corerpc都采用了调度器和协程机制。请求端有几个map:nid到sock,sockfd到node,pending requests以匹配reply。

rpc table的solt维护在途消息的信息,以匹配reply。同时注册有callback,用来唤醒yield的task。
每一个msg需要分配唯一的msgid标示。传回的reply也包含该标示。

corerpc的不同之处,在于\hl{建立连接时需要协商hash},每个节点上的每个core线程上维护一套corenet的连接信息。
相当于相同hash的core维护一条通道。有多个polling thread的情况下,每个节点就维护了多个corenet。
而监听port只需要一个就够了。

false share

\hrulefill

切入以算法和数据结构为主的学习阶段。一是各个击破,二是在解决实际问题的过程中,运用自如。
在调度、缓存、qos、分布式系统等系统的构建过程中,在存储引擎的构建过程中,算法和数据结构的身影,随处可见。

以\hl{问题和算法}去统领操作系统和编程语言,以及其它系统。设计、架构、算法是一体的。数据结构的选择也从属于算法的要求。
算法有其抽象的一方面,但只有在解决实际问题的过程中,才能对算法有更深入的理解和挖掘。

人生算法与机器算法不同,底层逻辑却相通。\hl{重要的是解决关键问题},在解决问题的过程中,得到提炼和升华、进化。
算法不能脱离事物的自然本性,\hl{认识事物的本然,然后设计算法,使之为我所用,才是最重要的}。

\hl{数字化、智能化离不开算法。架构、模式、算法的学习和领悟应该贯穿于人生和专业的全过程}。
软件=数据结构+算法。算法不局限在软件领域,而是存在于事事物物之中。数据结构是起点,算法是重点。

\dotfill

渐渐能理解计算机的底层工作原理了,cpu为什么叫做cpu。cpu是控制和运算单元,与内存进行密切交互。
取指、解码和执行,每条指令的操作数,存在寄存器和内存里,需要先行准备好寄存器和内存。

为什么需要寄存器?直接从内存取数据,并把结果存入内存不行吗?每条指令对寄存器都有着特定的要求。

\hl{进程上下文切换}需要保存什么信息?保存到哪儿?

要理解\hl{bus、外设、中断}的作用。控制流和数据流,

考虑最简单的情况,一个处理器,启动两个进程的情况。1+2,kernel潜在地进行进程调度。一个cpu上完整的指令序列是1+2调度模型。
这里所说的进程A和B是用户创建的进程,有相互隔离的虚拟地址空间。

调度器是一个进程吗?是一个特殊的进程,不加入调度队列,是背后的主宰。
调度器所需信息存在哪里?应该是不属于进程A和B的虚拟地址空间的,而是内核专有的一块内存区域。

设备也有自己的处理器,可能也有嵌入式操作系统,只是没有独立电源。\hl{主机通知设备也是通过中断方式}?

\dotfill

回过头来,再去理解NVMe和RDMA的工作原理。NVMe设备通过PCIe总线与主机相连,主机和NVMe的通信协议是NVMe协议。
NVMf是NVMe over fabric,在IB等fabric上运行NVMe协议。PCIe和fabric都是物理链路,NVMe则是通信协议。
NVMe被设计出来,用于取代SCSI协议。NVMe协议的传输性能大大高于SCSI协议。怎么做到的?

RDMA需要网卡和交换机等硬件支持,与TCP同为transfer协议。

NVMe和RDMA都是kernel bypass技术,SCSI运行在RDMA上是iSER,NVMe运行在RDMA上是NVMf。
\hl{NVMe是存储协议,RDMA是网络层传输协议。两者都采用了异步编程模型}。

\subsection{29}

每个inode包含address space成员,用于管理page cache。cache有writeback和writethrough模式,direct io类似于writethrough模式。
aio依赖于direct io。page cache被组织成一颗radix tree,再加上LRU list,就起到检索和置换的双重目的。

radix tree用来管理稀疏数据块,分层的,不存在的项不消耗存储空间,同时查询时间也可控。
\hl{lich中卷的chunk tree},就可以理解成是一个radix tree的数据结构,通过这种方式实现了精简配置。

dentry把inode组织成tree状结构,建立了path到inode的映射关系。file则对应进程打开的文件描述符,进程可以多次以不同模式打开一个文件(inode),
每打开一次,就增加一个不同的fd到file对象的映射。\hl{这些file对象指向一个inode对象}。

每个目录或文件项对应一个inode,inode的数据块部分含义不同、组织结构也迥异。不同文件系统有不同的组织方式。
如ext3用间接块来组织,ext4用entent tree来组织(B+ tree)。

内存采用page、vfs采用page,block层采用block,在vfs和block层之间是dismatch,通过一定的缓存区来消解这种dismatch。
不同的设备,对block的定义也不同,这里所说的block层是抽象的通用block层,体现了多态性。

\dotfill

\hl{内核管理物理资源},如用buddy管理物理内存的分配,用文件系统管理块设备。
用户空间代码一般没有权限直接与外设通信,\hl{RDMA和NVMe看上去是个例外}?

每种外设都有一定的处理能力,包括控制器芯片、寄存器和内存,甚至有微型嵌入式OS。
这是具体而微的分形结构,无处不在的处理逻辑。

联想主机CPU的工作原理,在寄存器和内存的辅助下,执行一条条指令。bus把所有组件联系起来成为一个整体。
\hl{与人体经络何其相似乃尔}。藏象和经络一起,通则不疼。
用排队论或网络流的数学工具去分析。性能如何?瓶颈点在哪里?

\dotfill

线程和队列,可以模拟异步操作,这也是该组合强大威力的一种体现。

\subsection{30}

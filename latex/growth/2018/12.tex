\section{12}

\subsection{03}

手腕扭伤快好了。真是旷日持久,一点点小问题,竟然如此,不可不慎重。病向浅中医,养生须趁早。

最重要的是下一阶段的工作问题。内壮则外强,不论何时何地,不要心怀幻想。\hl{放弃幻想,准备战斗}是战士的必然选择。
枕戈待旦是为将者的核心素养。

学习中医、太极,一是道,二是术,道术并重,前期以道为主,明理,后期深入术中,以道卫术。
有所偏重,交相互养。从更大的视野去看,道一而术多,都可以看作道的应用,博观约取。

对学习计算机技术也有好处,相互发明,在交叉地带有所领悟。为学日益为道日损。

徐特立老读书学习之法,定量、有恒,不可自乱方寸。

\subsection{04}

以心行气,以气运身。气是沟通身心的桥梁,区分先天气和后天气,呼吸是后天气,内气、真气从之的是先天气。
随外在呼吸,引领先天气之循环:沿着任脉而沉至丹田,沿着督脉而上达百会,形成闭环。任督通,则成小周天。
吸升呼降,升级的圆运动。医理拳理通。只有丹田聚气养气充沛,自然能打通任督二脉。

活动关节,找到途径各关节的主要穴位进行按摩,以起到拉伸的效果。循经找穴,进而通过经络上的关键穴位进行按摩。

\subsection{05}

洪范五福:《尚书》上所记载的五福\hl{一曰寿、二曰富、三曰康宁、四曰攸好德、五曰考终命},
东汉桓谭于《新论·辨惑第十三》把“考终命”更改,将五福改为“寿、富、贵、安乐、子孙众多”。
现代常把“五福临门”当作新春祝福使用。

六极:一曰凶、短、折, 二曰疾,三曰忧,四曰贫,五曰恶,六曰弱。

积贫积弱,富强的现代化中国。五福有可为有可为而不可期,如康宁、攸好德、富,一定程度上是可以做的。
这里有因果业力法则在起作用。

\begin{shadequote}
兴五福销六极

问:昔周著《九畴》之书,汉述《五行》之志,皆所以精究天人之际,穷探政化之源。然则五福之祥,何从而作;六极之沴,何故而生?
将欲辨行,可明本末。又今人财耗费,既贫且忧,时沴流行,或疾而夭。思欲销六极,致五福,殴一代于富寿,纳万人于康宁。何所施为,可致于此?

臣闻圣人兴五福销六极者,在乎\hl{立大中致大和}也。至哉中和之为德,不动而感,不劳而化,以之守则仁,以之用则神,卷之可以理一身,舒之可以济万物。
然则和者生于中也,中者生于不偏也,不邪也,不过也,不及也。若人君内非中勿思,外非中勿动,动静进退,皆得其中,故君得其中,则人得其所,人得其所,则和乐生焉。
是以君人之心和,则天地之气和,天地之气和,则万物之生和。于是乎三和之气,䜣合絪缊,积为寿,蓄为富,舒为康宁,敷为攸好德,益为考终命。
其羡者则融为甘露,凝为庆云,垂为德星,散为景风,流为醴泉。六气叶乎时,七曜顺乎轨,迨于巢穴羽毛之物,皆煦妪而自蕃,草木鳞介之祥,皆丛萃而继出。
夫然者,中和之气所致也。若人君内非中是思,外非中是动,动静进退,不得其中,故君不得其中,则人不得其所,人不得其所,则怨叹兴焉。
是以君人之心不和,则天地之气不和,天地之气不和,则万物之生不和。于是乎三不和之气,交错堙郁,伐为凶短折,攻为疾,聚为忧,损为贫,结为恶,耗为弱。
其羡者潜为伏阴,淫为愆阳,守为彗星,发为暴风,降为苦雨。四序失其节,三辰乱其行,迨乎襁褓卵胎之生,皆夭阏而不遂,木石华虫之怪,皆糅杂而毕呈。
夫然者,不中不和之气所致也。则天人交感之际,五福六极之来,岂不昭昭然哉。臣伏见比者兵赋未减,人鲜无忧,时沴所加,众或有疾。
德宗皇帝病人之病,忧人之忧,于是救之以广利之方,悦之以中和之乐,将使易忧为乐,变病为和,惠化之恩,莫斯甚也。

然臣窃闻善除害者察其本,善理疾者绝其源。伏惟陛下欲纾人之忧,先念忧之所自;欲救人之病,先思病之所由。知所自以绝之,则人忧自弭也;知所由以去之,则人病自瘳也。
然后申之以救疗之术,则人易康宁;鼓之以安乐之音,则人易和悦。斯必应疾而化速,利倍而功兼。六极待此而销,五福待此而作。
如是,可以陶三才缪滥之气,发为休祥;殴一代鄙夭之人,臻乎仁寿。中和之化,夫何远哉!
\end{shadequote}

东方治理学中阐述的黄帝四经基本方法论,作为基石。由此而展开为完整体系。理论核心在于体味阴阳中和的奥义。

撞丹田有上中下三个高度,每种高度20次为一组。完成5组为一轮。暂定如此,必须数量化。
\hl{任何活动都考虑量化},包括标准和验收。按由轻到重的原则,逐步加大运动量到一最大值,然后保持,允许有一定波动。
度数信,即是围绕中心轴的圆周运动,弦动方式。务必重视周期和节律。

张三丰打坐歌需要背诵。

意气运动要贯彻到一切运动中,这就意味着尽量慢,体会其中意气运行的节律。
数量并非第一重要,重要的是内在质量。

\subsection{06}

\subsection{07}

本周基本理解了RDMA,下周需要继续,列出以学习计划。

修炼也渐渐进入状态,坚持的并不算好。还需要加大力度和决心,通过一定手段调理身心是一辈子要进行的工作,不能轻视。
核心理念就是通经络,调气血,致中和。这也是世界运行的方式。

张首晟去世事件令人震惊,跨界资本失败造成的?也发人深思,\hl{专业是立身之本},下一步需要进一步深入去学习。
丹华资本是一步错棋?投资决策过于乐观?默默地做好自己的专业,再考虑趁势而起。不要给自己太大压力。\hl{从容中道乃最佳策略}。

\hl{太极混元桩站起},静中有动,气息流动,一刻不停。

每次集中攻克一个问题,目前静坐中,发现一些痛点。按瓶颈理论去处理。

\subsection{10}

站桩,太极混元桩、三体桩。为什么说万法出自三体桩。从无极桩、混元桩练习开始。
有足够的时间慢慢练习、体悟其原理。

动静交养,静坐、站桩、内家拳,都需要\hl{调心、调息、调身}。一呼一吸为一息,呼吸于生命而言,极为重要。
此中反思,对今后若干年具有重要价值。生活习惯、对生命的理解在这个过程中得以深化。
静坐、站桩、丹田功这些基本的内功心法,需要坚持下去,\hl{循序渐进、持之以恒}。信解行证,度数信。
基本方法无它,就是围绕一中心点,日积月累,以达豁然贯通之境。

\hl{脏腑、经络都可以看作中医的象数模型}。取象比类,就是一种模型思维。
至于模型是否反映真实情况,需要在实践中进行调节。

HY下一步会很艰难,是否陪着走下去是一个重要的抉择。是否有利于今后长期发展?是否有大的突破?
当断不断,必受其乱。选择的最重要标准就是下一步的发展。平台、领导、行业都极其重要。

简化简化再简化,放弃幻想、准备战斗,机会主义要不得。

\subsection{11}

平时行坐站卧都要注意姿势,功夫要下在平时,用正确的理念塑造良好的生活习惯。这个才是细水长流之道。
功夫怎么才能上身?不是机械地去练习,而是全身心地投入,用心领悟其精义。

体育运动与专业学习一样,遵循相同原则,如循序、有恒。刻意练习的理念,怎么用起来?
专项训练,全面提升。如何设计一套切实可行的健身法?应该包括:
\begin{enumbox}
\item 站桩
\item 静坐
\item 八段锦
\item 太极
\end{enumbox}

一些小敲门:
\begin{enumbox}
\item 腹式呼吸
\item 撞丹田
\item 跪式
\item 刷牙
\item 扣齿
\item 梳头
\item 提肛
\end{enumbox}

禅坐先不求速效,每天盘盘腿,重点关注一些痛点,就会有进步。日拱一卒的精神。
每一种功法都需要大量的时间积累,才能见效果。一个时期最好只有一种重点项目,待步上正轨时,再开始下一项。
采取一主多辅的架构。比如本阶段一撞丹田为主,以站桩、静坐为辅等等。
最后把所有的功法九九归一,抱元归一,回到旋极图的象征。

老子四十二章是最高哲学。\hl{道生一、一生二、二生三、三生万物。万物负阴而抱阳,中气以为和}。
这几句话包含了终极真理。要从中出发,展开为现实的力量。

近期的沉淀期,有其价值,沉下来,再出发。收敛到中间一点,收放自如。潜龙勿用,阳在下也。复其见天地之心。
最重要是保养此一团阳气,以直养而无害,则塞乎天地之间。越养越精神,以做持久战。

悟中道之理,成炼金术士。炼金术士者,能化腐朽为神奇。河洛五行,中土最为贵,乃调理转化之器。

放弃吧,现在变成很大的负能量,就这样不明不白的,有什么意思?现在的主要精力,应放在下一步的健康发展上。
真是内忧外患,有陷在泥潭里的感觉,世界那么大,为什么不走出去看看?

可以了解很多,最重要的确实基本功。

\subsection{14}

书本知识往往已经过时,紧跟会议、论文、各公司的实践活动。

\subsection{17}

\hl{大动不如小动,小动不如不动,不动之动乃生生不已之动}。此语精妙,极有启发。让人回归本源,无为而无不为,故大成拳是无为法。
剔除枝叶,一意本源。

在技术上事业上都有很好的启迪。处当今之际,HY可谓内忧外患,风雨飘摇。更有追问本质,以图活下去。
在技术上,一切围绕ABC,又分主次。主为linux、ceph等。通其一,万事毕。

大成拳心法功法俱为上佳,可以为一段时间的探索画上一个句号了。医武同源,同臻于道。道零德一而万物化生。
信解行证是一循环,转动不已。进而,立禅即意,此中禅意,渐渐融入生活中,行住坐卧,无往不在禅意中。
则与道合一。往日学习,偏于知解,缺乏体证,遂茫茫然不知所归,没有受用处。

大成拳与阳明心学都在唤醒自然本具之良知良能,栽培涵养,心中分别心生,则离道日远,佛理深邃,不可思议。
水性太极,妙悟圆觉,大成立禅,归心金刚。都需用心体证,勿自限于文字知解,何况不能达于文字般若,
妄生议论,枉费口舌,大可不必。

至此,各方面均有妥当安排,可以心无旁骛,尽心驰骋了。艺宗AB,拳归大成。

不管HY如何,这次一定要走。初步定为xsky,离家距离不算太远,发展势头蒸蒸日上。
最重要的是,与手头做的最为接近,可以全力以赴投入技术的进一步深造。

也需要安排几个候选,如京东云、联想、首都在线。至此,对下一步的职业规划基本定位清晰。

\subsection{18}

想不到华云窘迫至此,真是可叹!

贪多嚼不烂,全闪是唯一机会。

放弃幻想,准备战斗。积极准备找工作,静观其变吧。损失云云,不作为主要考虑项。
永远往前看,修之于身,其德乃真。

全用户态SDS,就照着这个目标努力。把握好AFA这个风口。
怎么建立相关知识体系呢?

\subsection{19}

\subsection{20}

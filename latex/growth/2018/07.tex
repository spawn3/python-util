\section{201807}

\subsection{0702}

虚静,摄无量义。

无我曰虚,归根曰静。无我而归诸道,心与道合,是为真人。

淡泊明志,宁静致远。

\subsection{0703}

123哲学是分子结构,再往上就是系统论。一个系统由子系统构成,形成层次结构。
系统具有分形属性,一即一切,一切即一。一花一世界,一叶一菩提。

抽身物外,胜物而不伤,勿死于物下。道提供了与物沟通的另一维度,
道者,万物之奥。道者,物之极。架构师与程序员的不同,主要也是在此。
精于道者兼物物,精于物者以物物,下学而上达。

道物,粗分有两个层次,然上通九天,下贯九野,一层功夫一层理。
合中有分,分中有合。

这一关确实不好过了,走还是留,是个问题。不管怎样,都要做好充分的准备。

管理不上路,财务不合规,关键是能不能虚心听取意见,
从中获得成长,一时的成败不是决定性的。

\subsection{0704}

我注六经,六经注我。我与六经之间是超越线性的关系。为今之计,发明心地,明心见性。

寻章摘句,君子不为。以虚壹而静之心态,拥抱现实及其变化,确立道为最高原则,尊道贵德。

归纳整理出我的原则,至关重要。

系统化的决策流程,决策攸关成败,有底层逻辑,有道有术。

守、破、离对应心物,心道、道物三线,成三角形。

\hl{做决策不是我什么什么还没准备好,要相信自己的基本功与学习能力}。
精于道者兼物物,致力于道,物不会是严重障碍。

顶角即是道,也是机器、系统,看到二中之一,看着物理学之后的形而上的东西。
形而上者谓之道,形而下者谓之器。此一上行下行的路径,揭示了更多可能性。

人生算法有认知闭环:感知-认知-决策-行动,是动词构成的,心道物三者,是名词构成的。
内核与外环,内核是最小化的那个点,外环是动力与使命。认知闭环发生在心物之间,
三角形的每个边都是一个认知闭环,PDCA循环。这些小车轮,架起了友谊的桥梁。

是节点问题,还是边问题?居于中心的是什么?

把道、原则、多元思维模型、混沌大学课程、人生算法这些模型融合起来。
打造自己的模型。

取势、明道、优术,取势在心,明道在道,优术在物。
外环由心发动。

夸克构成质子和中子,1:2的比例关系。

\section{201807}

\subsection{0702}

虚静,摄无量义。

无我曰虚,归根曰静。无我而归诸道,心与道合,是为真人。

淡泊明志,宁静致远。

\subsection{0703}

123哲学是分子结构,再往上就是系统论。一个系统由子系统构成,形成层次结构。
系统具有分形属性,一即一切,一切即一。一花一世界,一叶一菩提。

抽身物外,胜物而不伤,勿死于物下。道提供了与物沟通的另一维度,
道者,万物之奥。道者,物之极。架构师与程序员的不同,主要也是在此。
精于道者兼物物,精于物者以物物,下学而上达。

道物,粗分有两个层次,然上通九天,下贯九野,一层功夫一层理。
合中有分,分中有合。

这一关确实不好过了,走还是留,是个问题。不管怎样,都要做好充分的准备。

管理不上路,财务不合规,关键是能不能虚心听取意见,
从中获得成长,一时的成败不是决定性的。

\subsection{0704}

我注六经,六经注我。我与六经之间是超越线性的关系。为今之计,发明心地,明心见性。

寻章摘句,君子不为。以虚壹而静之心态,拥抱现实及其变化,确立道为最高原则,尊道贵德。

归纳整理出我的原则,至关重要。

系统化的决策流程,决策攸关成败,有底层逻辑,有道有术。

守、破、离对应心物,心道、道物三线,成三角形。

\hl{做决策不是我什么什么还没准备好,要相信自己的基本功与学习能力}。
精于道者兼物物,致力于道,物不会是严重障碍。

顶角即是道,也是机器、系统,看到二中之一,看着物理学之后的形而上的东西。
形而上者谓之道,形而下者谓之器。此一上行下行的路径,揭示了更多可能性。

人生算法有认知闭环:感知-认知-决策-行动,是动词构成的,心道物三者,是名词构成的。
内核与外环,内核是最小化的那个点,外环是动力与使命。认知闭环发生在心物之间,
三角形的每个边都是一个认知闭环,PDCA循环。这些小车轮,架起了友谊的桥梁。

是节点问题,还是边问题?居于中心的是什么?

把道、原则、人生算法、多元思维模型、混沌大学课程这些模型融合起来。
打造自己的模型。

取势、明道、优术,取势在心,明道在道,优术在物。
外环由心发动。

夸克构成质子和中子,1:2的比例关系。

把最近围绕道的认知,应用到工作中,在知行中螺旋上升。
一是道心物三角,二是认知闭环,三是体道方法与心态虚壹而静。

稳住,静下来,搞点大事,五年磨一剑,一战定江山。

原则:心态、机器、系统。分生活、工作、投资等领域内归纳出的一些原则。

算法:认知闭环。

多元思维模型:从硬学科里提炼基础模型,形成体系,运用到各种决策场景。

混沌大学:用第一性原理,跨越第二曲线的不连续区。

道具有最终的统一性,众星拱之。

\hl{把分布式块存储系统列入最小内核},运用即即为广泛,深度也够,待解决问题也很重要。
怎么让它最大化呢?占据铁三角的物之一角。要呵护珍惜!

同时需要从别的领域吸收养分,但这个是核心,如果能立下来一个核心,
来华云不管遇到什么,都值了。

\subsection{0705}

体道者逸而不穷,任数者劳而无功。双线法则

战略,不在战,而在略。亮独观其大略。

用心体会虚壹而静四字。

道不欲杂,道是朴素的,一立而万物生矣。

% 如果钱能到,很好。任何时候,成长都是第一位的,如果因为钱影响了成长,就得不偿失。

成长如何衡量?曰道。道是一种信仰,有道则吉,无道则凶。道之有无取决于目标。

\hl{NLP思维逻辑层次}:精神、身份、信念、能力、行为、环境。

前五个都是我,把握当下。

精神=道,身份=我,信念=原则,自此以往,皆算法。

养神之所归诸道,身份是入口、枢纽、关节点。无我,上通于道,惟道是从。
道居于太极至尊的位置,至尊而不独尊。

内静外敬,性将大定。

\subsection{0706}

正念、良知是体道见性知天命的方法。

虚静,一是尊道;二是正念,如如不动。

\subsection{0709}

惟精惟一

打磨三合架构,整合原则、算法,去分析问题

\begin{enumbox}
\item 过以原则为基础的生活
\item 更高层次思考
\item 做一个超级现实的人
\item 极度的头脑开放
\item 五步流程方法
\item 如何做出好决策?
\end{enumbox}

在心物道三者间,持续转动。用三合结构分析达里奥的原则一书。

道者,物之极。升维思考物的真实价值。回到心,是否足够空灵高效有力量,心智模式。

心,极度真实、极度开放。

原则一书,也是升维降维,上帝视角,引入机器、系统,进行控制。
机器位于物的节点,分解为目标与结果、团队与规则。

五步流程法等同于设定目标+认知闭环(感知、认知、决策、行动)。

怎么做出好决策?

限制一下悟道的时间,不必太多,时时提起。

设定下一阶段的目标,全力以赴

帝者体太一、王者法阴阳、霸者则四时、君者用六律。
太一者,理解为目标、内核,有着更为深刻的内涵。

\subsection{0710}

霍金斯能量等级,让人耳目一新。正负能级的分水岭在勇气,
知耻近乎勇,勇气是自我成长的关键要素。

可以看做情商的元素周期表,每次考察一个元素,
改善之,努力向下一能级跃迁。

舍去这些人与事,内心才能真的平静,聚焦在最重要的事情上。
没有舍怎么得呢?幻想没有什么用,拥抱赤裸裸的真实才有出息。

如果让某些人与事影响内心的平静,真的非常不好,牢牢锁定自己的方向、做自己可控的事情。

胜人者有力,自胜者强。不能自胜,何谈其它?
有些事情,转念就好,顺其自然,岂能妄为?

好好消化原则一书,能极大地促进对道的理解。不要一头扎入细节之中,
做到以道观之,按原则行事。类似的书,还有用系统来工作,管道的故事。

先成长为真正的专家,比盲目地开公司,是更可控更现实的事情。
阿里P9、P10的财务收入已不少。在这个基础上,或投资、或创业,更可期待。
再给自己三年或五年的时间吧,不要急、慢慢来。

最近对道的探索,收获颇丰,感觉离大道更近了点,心态也变得更自主、积极,思路更开阔。
这是非常正确的选择,但不能着急。孔德之容,惟道是从。孔者,从容状。

体道会影响到很多方面,心与物。

用PCDA统筹目标五阶段,1+4。一是设定目标,4是四时,PDCA、元亨利贞、认知闭环、四象限等。

如何做出好决策?这是贯彻始终的一个大问题,渗透到每一个环节。

\section{201807}

\subsection{0702}

虚静,摄无量义。

无我曰虚,归根曰静。无我而归诸道,心与道合,是为真人。

淡泊明志,宁静致远。

\subsection{0703}

123哲学是分子结构,再往上就是系统论。一个系统由子系统构成,形成层次结构。
系统具有分形属性,一即一切,一切即一。一花一世界,一叶一菩提。

抽身物外,胜物而不伤,勿死于物下。道提供了与物沟通的另一维度,
道者,万物之奥。道者,物之极。架构师与程序员的不同,主要也是在此。
精于道者兼物物,精于物者以物物,下学而上达。

道物,粗分有两个层次,然上通九天,下贯九野,一层功夫一层理。
合中有分,分中有合。

这一关确实不好过了,走还是留,是个问题。不管怎样,都要做好充分的准备。

管理不上路,财务不合规,关键是能不能虚心听取意见,
从中获得成长,一时的成败不是决定性的。

\subsection{0704}

我注六经,六经注我。我与六经之间是超越线性的关系。为今之计,发明心地,明心见性。

寻章摘句,君子不为。以虚壹而静之心态,拥抱现实及其变化,确立道为最高原则,尊道贵德。

归纳整理出我的原则,至关重要。

系统化的决策流程,决策攸关成败,有底层逻辑,有道有术。

守、破、离对应心物,心道、道物三线,成三角形。

\hl{做决策不是我什么什么还没准备好,要相信自己的基本功与学习能力}。
精于道者兼物物,致力于道,物不会是严重障碍。

顶角即是道,也是机器、系统,看到二中之一,看着物理学之后的形而上的东西。
形而上者谓之道,形而下者谓之器。此一上行下行的路径,揭示了更多可能性。

人生算法有认知闭环:感知-认知-决策-行动,是动词构成的,心道物三者,是名词构成的。
内核与外环,内核是最小化的那个点,外环是动力与使命。认知闭环发生在心物之间,
三角形的每个边都是一个认知闭环,PDCA循环。这些小车轮,架起了友谊的桥梁。

是节点问题,还是边问题?居于中心的是什么?

把道、原则、人生算法、多元思维模型、混沌大学课程这些模型融合起来。
打造自己的模型。

取势、明道、优术,取势在心,明道在道,优术在物。
外环由心发动。

夸克构成质子和中子,1:2的比例关系。

把最近围绕道的认知,应用到工作中,在知行中螺旋上升。
一是道心物三角,二是认知闭环,三是体道方法与心态虚壹而静。

稳住,静下来,搞点大事,五年磨一剑,一战定江山。

原则:心态、机器、系统。分生活、工作、投资等领域内归纳出的一些原则。

算法:认知闭环。

多元思维模型:从硬学科里提炼基础模型,形成体系,运用到各种决策场景。

混沌大学:用第一性原理,跨越第二曲线的不连续区。

道具有最终的统一性,众星拱之。

\hl{把分布式块存储系统列入最小内核},运用即即为广泛,深度也够,待解决问题也很重要。
怎么让它最大化呢?占据铁三角的物之一角。要呵护珍惜!

同时需要从别的领域吸收养分,但这个是核心,如果能立下来一个核心,
来华云不管遇到什么,都值了。

\subsection{0705}

体道者逸而不穷,任数者劳而无功。双线法则

战略,不在战,而在略。亮独观其大略。

用心体会虚壹而静四字。

道不欲杂,道是朴素的,一立而万物生矣。

% 如果钱能到,很好。任何时候,成长都是第一位的,如果因为钱影响了成长,就得不偿失。

成长如何衡量?曰道。道是一种信仰,有道则吉,无道则凶。道之有无取决于目标。

\hl{NLP思维逻辑层次}:精神、身份、信念、能力、行为、环境。

前五个都是我,把握当下。

精神=道,身份=我,信念=原则,自此以往,皆算法。

养神之所归诸道,身份是入口、枢纽、关节点。无我,上通于道,惟道是从。
道居于太极至尊的位置,至尊而不独尊。

内静外敬,性将大定。

\subsection{0706}

正念、良知是体道见性知天命的方法。

虚静,一是尊道;二是正念,如如不动。

\subsection{0709}

惟精惟一

打磨三合架构,整合原则、算法,去分析问题

\begin{enumbox}
\item 过以原则为基础的生活
\item 更高层次思考
\item 做一个超级现实的人
\item 极度的头脑开放
\item 五步流程方法
\item 如何做出好决策?
\end{enumbox}

在心物道三者间,持续转动。用三合结构分析达里奥的原则一书。

道者,物之极。升维思考物的真实价值。回到心,是否足够空灵高效有力量,心智模式。

心,极度真实、极度开放。

原则一书,也是升维降维,上帝视角,引入机器、系统,进行控制。
机器位于物的节点,分解为目标与结果、团队与规则。

五步流程法等同于设定目标+认知闭环(感知、认知、决策、行动)。

怎么做出好决策?

限制一下悟道的时间,不必太多,时时提起。

设定下一阶段的目标,全力以赴

帝者体太一、王者法阴阳、霸者则四时、君者用六律。
太一者,理解为目标、内核,有着更为深刻的内涵。

\subsection{0710}

霍金斯能量等级,让人耳目一新。正负能级的分水岭在勇气,
知耻近乎勇,勇气是自我成长的关键要素。

可以看做情商的元素周期表,每次考察一个元素,
改善之,努力向下一能级跃迁。

舍去这些人与事,内心才能真的平静,聚焦在最重要的事情上。
没有舍怎么得呢?幻想没有什么用,拥抱赤裸裸的真实才有出息。

如果让某些人与事影响内心的平静,真的非常不好,牢牢锁定自己的方向、做自己可控的事情。

胜人者有力,自胜者强。不能自胜,何谈其它?
有些事情,转念就好,顺其自然,岂能妄为?

好好消化原则一书,能极大地促进对道的理解。不要一头扎入细节之中,
做到以道观之,按原则行事。类似的书,还有用系统来工作,管道的故事。

先成长为真正的专家,比盲目地开公司,是更可控更现实的事情。
阿里P9、P10的财务收入已不少。在这个基础上,或投资、或创业,更可期待。
再给自己三年或五年的时间吧,不要急、慢慢来。

最近对道的探索,收获颇丰,感觉离大道更近了点,心态也变得更自主、积极,思路更开阔。
这是非常正确的选择,但不能着急。孔德之容,惟道是从。孔者,从容状。

体道会影响到很多方面,心与物。

用PCDA统筹目标五阶段,1+4。一是设定目标,4是四时,PDCA、元亨利贞、认知闭环、四象限等。

如何做出好决策?这是贯彻始终的一个大问题,渗透到每一个环节。

\subsection{0711}

昨日聚餐谈及PDCA,结合原则的1+4,霸者则四时,四时交替、运转不息,
3/4也是很重要的模式。

金字塔的逻辑结构,与太极生两仪暗合,金字塔原理更着重形式逻辑,
太极两仪偏重辩证逻辑。

机器生目标与结果两极,阳变阴合而生金木水火土,五气顺布,四时行焉。
这是二与五的结合,三与四在其中。无极之真,二五之精,妙合而凝。

有此六个数,足矣。

古之王者,建国君民,教学为先。学术是大本大源,故荀子开篇即是劝学。
博学、审问、慎思、明辨、笃行,环环相扣,一气流行。
气没有固定的形状,表示空。

思维格栅,如何才能形成?道、原则的体系展开。
广泛吸收重要学科的核心概念、理论与工具。

道法术器,一气流行。气韵生动,气表明一股存在的无形力量。
不仅要看到形,更要看到神。

\subsection{0712}

太极图说,黄帝阴符经要内化于心。宇宙在乎手,万化生乎身。宇宙、万化皆一心之所裁,本出于身手,由近及远。
三合结构普遍存在,如三盗既宜、三才既安的天地-人-万物之关系。我与非我,统一于大梵之境。
建立自我、追求无我,如此我无我皆入道矣。
偏于任何一方都非究竟之道。二元对立统一方为中道正见。一而不二,是谓知道。如何统一呢?

求道予人一大格局、大视野、大机趣、大静大动。

一存在于二中为三,一存在于四中为五,大部分情况已够用。
一二四是变化序列,三五隐然其中。运转PDCA而不知一,则怠,有术无道。

变化是维度的增加。

PDCA是个周而复始的循环过程,每一个循环就带来了新的可能性,把系统带入一个新高度。

管子有四时一章,论述详备。

李中莹心智力:这个世界由无数个系统组成,每个系统都用着同一套法则运行,称为系统动力。

王小川从生物学去领悟管理,马斯克从物理学领悟第一性原理,共通的是系统论。
生命系统与非生命系统的同异分别是什么?所谓求道,就在于领悟系统运行的原理。

亚里士多德的四因说:目的因,形式因,资料因,动力因。

体道者逸而不穷,即从这个角度说。任数者劳而无功,此说不透彻,道与数不离,定性与定量相偶。
体道者与任数者都是为了逸而不穷这个目的而来的。

\hl{认真写专利,这个有大用处}。

围绕道构建心智模式、思维模型,同时积极融合各学科的重要概念与思维模型。

道德经、阴符经、太极图说是重要的思想资源,反复打磨对基础概念的理解,天网恢恢疏而不漏。
构建思维的天网,疏略简单,然而能囊括万事万物。

涤除玄鉴,能无疵乎?玄鉴能洞幽烛微,看到世界运行的各种系统、机器,而董理之,为我所用。
以道观天下,易在其中矣。然不止于观,观天之道,执天之行,方为尽矣。

阴符、盗、贼,皆暗中之事,不知不觉中,发生着巨变,怎能不特别地留意呢?
比如时间流逝,消无声息,需要警惕,提起觉知力,明察秋毫,把握住最重要的事情。

原则、用系统来工作、控制论是西方学术传统下的产物,与东方文化有明显不同,形成互补之势。
道中有术、术中有道,道术一体,勇往直前。

读书快慢结合,关键地方反复体味,可收一日千里之效;泛观博览,广泛吸取营养,用以涵养本源处。
我与六经,心与法华,于是统一于道。归纳演绎,相得益彰。

通天下,桓古今,无非一气而已。气一元论,为思维引入了极大空灵的感觉。气无形无相,用之不竭。

代码里的气韵生动

\subsection{0713}

12345五个数字足够深刻,宇宙在乎手,万化生乎身。其中1是关键,道生一。认知闭环四阶段是属于PDCA的PD,不过分析思路可用。
分析问题,有节点问题,有连接问题,也有内核与外环问题,同时深化了对P阶段的理解。内核相当于中央之一,五行属土。
内核之一分阴分阳,摄内核外环,目标使命,解决目的因与动力因。离开中央之一的PDCA不完整,所以四时转化为五行。

尚书洪范篇提出了五行、皇极,皇极即中央之一。皇建有其极,乃成王道。

复盘属于PDCA的check环节。

\subsection{0716}

运行PDCA循环。帝者体太一,王者法阴阳,霸者则四时,君者用六律。在这一指导思想下,广泛运用PDCA。

目前工作的主要目标是移动集采,事前要进行多次演习。在团队内运行PDCA,会有不同的规定性。
进一步,在全公司内运行PDCA。

平衡记分卡、目标五阶段、TM四象限等采用的也是四时五行的模型,2x2矩阵。

\hl{参伍以变,错综其数。通其变,遂成天下之文,极其数,遂定天下之象}。
参,三合之道。五,五行,PDCA。五行代表圆点哲学。三合者,太极生两仪,阴阳本太极。
转动PDCA之环,勇往直前。

中庸的诚字,是通往专家之路的最好指南。诚之者,择善而固执之者也。
诚则明,明则诚。至诚无息。

择善,天命之谓性,天命的呼唤。正念、良知,皆诚意也。故诚,涵正念、良知、天命而为一。

孙言及Q4的安排,技术、管理统一于道,能深切地领会一下管理,也是一个大机会。

\subsection{0717}

兵法家最切实用。兵之胜败,本在于政。明一者皇,察道者帝,通德者王。谋得兵胜者霸。

凡战之道:位欲严;政欲栗;力欲窕;气欲闲;心欲一。

夫圆者,天也;方者,地也。天圆而无端,故不可得而观;地方而无垠,故莫能窥其门。天化育而无形象,地生长而无计量,浑浑沉沉,孰知其藏。
凡物有朕,唯道无朕。所以无朕者,以其无常形势也。轮转而无穷,象日月之运行,若春秋有代谢,若日月有昼夜,终而复始,明而复晦,莫能得其纪。
制刑而无刑,故功可成;物物而不物,故胜而不屈。刑兵之极也,至于无刑,可谓极之矣。
是故大兵无创,与鬼神通,五兵不厉,天下莫之敢当。
建鼓不出库,诸侯莫不慑㥄沮胆其处。故\hl{庙战者帝,神化者王。所谓庙战者,法天道也;神化者,法四时也}。
修政于境内,而远方慕其德;制胜于未战,而诸侯服其威。内政治也。

道生法,法生兵。

\subsection{0718}

昔者,越王句践问范子曰:“古之贤主、圣王之治,何左何右?何去何取?”
范子对曰:“臣闻\hl{圣主之治,左道右术,去末取实}。”
越王曰:“何谓道?何谓术?何谓末?何谓实?”
范子对曰:“道者,天地先生,不知老;曲成万物,不名巧。故谓之道。道生气,气生阴,阴生阳,阳生天地。天地立,然后有寒暑、燥湿、日月、星辰、四时,而万物备。
术者,天意也。盛夏之时,万物遂长。圣人缘天心,助天喜,乐万物之长。故舜弹五弦之琴,歌南风之诗,而天下治。言其乐与天下同也。当是之时,颂声作。
所谓末者,名也。故名过实,则百姓不附亲,贤士不为用。而外□诸侯,圣主不为也。
所谓实者,谷□也,得人心,任贤士也。凡此四者,邦之宝也。”

天下大乱,贤圣不明,道德不一,天下多得一察焉以自好。譬如耳目鼻口,皆有所明,不能相通。犹百家众技也,皆有所长,时有所用。
虽然,不该不遍,一曲之士也。判天地之美,析万物之理,察古人之全,寡能备于天地之美,称神明之容。
是故\hl{内圣外王之道},暗而不明,郁而不发,天下之人各为其所欲焉以自为方。
悲夫!百家往而不反,必不合矣。后世之学者,不幸不见天地之纯,古人之大体,道术将为天下裂。

圣人内修道术而不外饰仁义,知九窍四支之宜,而游乎精神之和,此圣人之游也。

是故圣人持养其神,和弱其气,平夷其形,而与道浮沉,如此则万物之化无不偶也,百事之变无不应也。按:形-神-气,三者内。与道浮沉,上行下行。

\subsection{0719}

\subsection{0720}

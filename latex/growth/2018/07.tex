\section{201807}

\subsection{0702}

虚静,摄无量义。

无我曰虚,归根曰静。无我而归诸道,心与道合,是为真人。

淡泊明志,宁静致远。

\subsection{0703}

123哲学是分子结构,再往上就是系统论。一个系统由子系统构成,形成层次结构。
系统具有分形属性,一即一切,一切即一。一花一世界,一叶一菩提。

抽身物外,胜物而不伤,勿死于物下。道提供了与物沟通的另一维度,
道者,万物之奥。道者,物之极。架构师与程序员的不同,主要也是在此。
精于道者兼物物,精于物者以物物,下学而上达。

道物,粗分有两个层次,然上通九天,下贯九野,一层功夫一层理。
合中有分,分中有合。

这一关确实不好过了,走还是留,是个问题。不管怎样,都要做好充分的准备。

管理不上路,财务不合规,关键是能不能虚心听取意见,
从中获得成长,一时的成败不是决定性的。

\subsection{0704}

我注六经,六经注我。我与六经之间是超越线性的关系。为今之计,发明心地,明心见性。

寻章摘句,君子不为。以虚壹而静之心态,拥抱现实及其变化,确立道为最高原则,尊道贵德。

归纳整理出我的原则,至关重要。

系统化的决策流程,决策攸关成败,有底层逻辑,有道有术。

守、破、离对应心物,心道、道物三线,成三角形。

\hl{做决策不是我什么什么还没准备好,要相信自己的基本功与学习能力}。
精于道者兼物物,致力于道,物不会是严重障碍。

顶角即是道,也是机器、系统,看到二中之一,看着物理学之后的形而上的东西。
形而上者谓之道,形而下者谓之器。此一上行下行的路径,揭示了更多可能性。

人生算法有认知闭环:感知-认知-决策-行动,是动词构成的,心道物三者,是名词构成的。
内核与外环,内核是最小化的那个点,外环是动力与使命。认知闭环发生在心物之间,
三角形的每个边都是一个认知闭环,PDCA循环。这些小车轮,架起了友谊的桥梁。

是节点问题,还是边问题?居于中心的是什么?

把道、原则、人生算法、多元思维模型、混沌大学课程这些模型融合起来。
打造自己的模型。

取势、明道、优术,取势在心,明道在道,优术在物。
外环由心发动。

夸克构成质子和中子,1:2的比例关系。

把最近围绕道的认知,应用到工作中,在知行中螺旋上升。
一是道心物三角,二是认知闭环,三是体道方法与心态虚壹而静。

稳住,静下来,搞点大事,五年磨一剑,一战定江山。

原则:心态、机器、系统。分生活、工作、投资等领域内归纳出的一些原则。

算法:认知闭环。

多元思维模型:从硬学科里提炼基础模型,形成体系,运用到各种决策场景。

混沌大学:用第一性原理,跨越第二曲线的不连续区。

道具有最终的统一性,众星拱之。

\hl{把分布式块存储系统列入最小内核},运用即即为广泛,深度也够,待解决问题也很重要。
怎么让它最大化呢?占据铁三角的物之一角。要呵护珍惜!

同时需要从别的领域吸收养分,但这个是核心,如果能立下来一个核心,
来华云不管遇到什么,都值了。

\subsection{0705}

体道者逸而不穷,任数者劳而无功。双线法则

战略,不在战,而在略。亮独观其大略。

用心体会虚壹而静四字。

道不欲杂,道是朴素的,一立而万物生矣。

% 如果钱能到,很好。任何时候,成长都是第一位的,如果因为钱影响了成长,就得不偿失。

成长如何衡量?曰道。道是一种信仰,有道则吉,无道则凶。道之有无取决于目标。

\hl{NLP思维逻辑层次}:精神、身份、信念、能力、行为、环境。

前五个都是我,把握当下。

精神=道,身份=我,信念=原则,自此以往,皆算法。

养神之所归诸道,身份是入口、枢纽、关节点。无我,上通于道,惟道是从。
道居于太极至尊的位置,至尊而不独尊。

内静外敬,性将大定。

\subsection{0706}

正念、良知是体道见性知天命的方法。

虚静,一是尊道;二是正念,如如不动。

\subsection{0709}

惟精惟一

打磨三合架构,整合原则、算法,去分析问题

\begin{enumbox}
\item 过以原则为基础的生活
\item 更高层次思考
\item 做一个超级现实的人
\item 极度的头脑开放
\item 五步流程方法
\item 如何做出好决策?
\end{enumbox}

在心物道三者间,持续转动。用三合结构分析达里奥的原则一书。

道者,物之极。升维思考物的真实价值。回到心,是否足够空灵高效有力量,心智模式。

心,极度真实、极度开放。

原则一书,也是升维降维,上帝视角,引入机器、系统,进行控制。
机器位于物的节点,分解为目标与结果、团队与规则。

五步流程法等同于设定目标+认知闭环(感知、认知、决策、行动)。

怎么做出好决策?

限制一下悟道的时间,不必太多,时时提起。

设定下一阶段的目标,全力以赴

帝者体太一、王者法阴阳、霸者则四时、君者用六律。
太一者,理解为目标、内核,有着更为深刻的内涵。

\subsection{0710}

霍金斯能量等级,让人耳目一新。正负能级的分水岭在勇气,
知耻近乎勇,勇气是自我成长的关键要素。

可以看做情商的元素周期表,每次考察一个元素,
改善之,努力向下一能级跃迁。

舍去这些人与事,内心才能真的平静,聚焦在最重要的事情上。
没有舍怎么得呢?幻想没有什么用,拥抱赤裸裸的真实才有出息。

如果让某些人与事影响内心的平静,真的非常不好,牢牢锁定自己的方向、做自己可控的事情。

胜人者有力,自胜者强。不能自胜,何谈其它?
有些事情,转念就好,顺其自然,岂能妄为?

好好消化原则一书,能极大地促进对道的理解。不要一头扎入细节之中,
做到以道观之,按原则行事。类似的书,还有用系统来工作,管道的故事。

先成长为真正的专家,比盲目地开公司,是更可控更现实的事情。
阿里P9、P10的财务收入已不少。在这个基础上,或投资、或创业,更可期待。
再给自己三年或五年的时间吧,不要急、慢慢来。

最近对道的探索,收获颇丰,感觉离大道更近了点,心态也变得更自主、积极,思路更开阔。
这是非常正确的选择,但不能着急。孔德之容,惟道是从。孔者,从容状。

体道会影响到很多方面,心与物。

用PCDA统筹目标五阶段,1+4。一是设定目标,4是四时,PDCA、元亨利贞、认知闭环、四象限等。

如何做出好决策?这是贯彻始终的一个大问题,渗透到每一个环节。

\subsection{0711}

昨日聚餐谈及PDCA,结合原则的1+4,霸者则四时,四时交替、运转不息,
3/4也是很重要的模式。

金字塔的逻辑结构,与太极生两仪暗合,金字塔原理更着重形式逻辑,
太极两仪偏重辩证逻辑。

机器生目标与结果两极,阳变阴合而生金木水火土,五气顺布,四时行焉。
这是二与五的结合,三与四在其中。无极之真,二五之精,妙合而凝。

有此六个数,足矣。

古之王者,建国君民,教学为先。学术是大本大源,故荀子开篇即是劝学。
博学、审问、慎思、明辨、笃行,环环相扣,一气流行。
气没有固定的形状,表示空。

思维格栅,如何才能形成?道、原则的体系展开。
广泛吸收重要学科的核心概念、理论与工具。

道法术器,一气流行。气韵生动,气表明一股存在的无形力量。
不仅要看到形,更要看到神。

\subsection{0712}

太极图说,黄帝阴符经要内化于心。宇宙在乎手,万化生乎身。宇宙、万化皆一心之所裁,本出于身手,由近及远。
三合结构普遍存在,如三盗既宜、三才既安的天地-人-万物之关系。我与非我,统一于大梵之境。
建立自我、追求无我,如此我无我皆入道矣。
偏于任何一方都非究竟之道。二元对立统一方为中道正见。一而不二,是谓知道。如何统一呢?

求道予人一大格局、大视野、大机趣、大静大动。

一存在于二中为三,一存在于四中为五,大部分情况已够用。
一二四是变化序列,三五隐然其中。运转PDCA而不知一,则怠,有术无道。

变化是维度的增加。

PDCA是个周而复始的循环过程,每一个循环就带来了新的可能性,把系统带入一个新高度。

管子有四时一章,论述详备。

李中莹心智力:这个世界由无数个系统组成,每个系统都用着同一套法则运行,称为系统动力。

王小川从生物学去领悟管理,马斯克从物理学领悟第一性原理,共通的是系统论。
生命系统与非生命系统的同异分别是什么?所谓求道,就在于领悟系统运行的原理。

亚里士多德的四因说:目的因,形式因,资料因,动力因。

体道者逸而不穷,即从这个角度说。任数者劳而无功,此说不透彻,道与数不离,定性与定量相偶。
体道者与任数者都是为了逸而不穷这个目的而来的。

\hl{认真写专利,这个有大用处}。

围绕道构建心智模式、思维模型,同时积极融合各学科的重要概念与思维模型。

道德经、阴符经、太极图说是重要的思想资源,反复打磨对基础概念的理解,天网恢恢疏而不漏。
构建思维的天网,疏略简单,然而能囊括万事万物。

涤除玄鉴,能无疵乎?玄鉴能洞幽烛微,看到世界运行的各种系统、机器,而董理之,为我所用。
以道观天下,易在其中矣。然不止于观,观天之道,执天之行,方为尽矣。

阴符、盗、贼,皆暗中之事,不知不觉中,发生着巨变,怎能不特别地留意呢?
比如时间流逝,消无声息,需要警惕,提起觉知力,明察秋毫,把握住最重要的事情。

原则、用系统来工作、控制论是西方学术传统下的产物,与东方文化有明显不同,形成互补之势。
道中有术、术中有道,道术一体,勇往直前。

读书快慢结合,关键地方反复体味,可收一日千里之效;泛观博览,广泛吸取营养,用以涵养本源处。
我与六经,心与法华,于是统一于道。归纳演绎,相得益彰。

通天下,桓古今,无非一气而已。气一元论,为思维引入了极大空灵的感觉。气无形无相,用之不竭。

代码里的气韵生动

\subsection{0713}

12345五个数字足够深刻,宇宙在乎手,万化生乎身。其中1是关键,道生一。认知闭环四阶段是属于PDCA的PD,不过分析思路可用。
分析问题,有节点问题,有连接问题,也有内核与外环问题,同时深化了对P阶段的理解。内核相当于中央之一,五行属土。
内核之一分阴分阳,摄内核外环,目标使命,解决目的因与动力因。离开中央之一的PDCA不完整,所以四时转化为五行。

尚书洪范篇提出了五行、皇极,皇极即中央之一。皇建有其极,乃成王道。

复盘属于PDCA的check环节。

\subsection{0716}

运行PDCA循环。帝者体太一,王者法阴阳,霸者则四时,君者用六律。在这一指导思想下,广泛运用PDCA。

目前工作的主要目标是移动集采,事前要进行多次演习。在团队内运行PDCA,会有不同的规定性。
进一步,在全公司内运行PDCA。

平衡记分卡、目标五阶段、TM四象限等采用的也是四时五行的模型,2x2矩阵。

\hl{参伍以变,错综其数。通其变,遂成天下之文,极其数,遂定天下之象}。
参,三合之道。五,五行,PDCA。五行代表圆点哲学。三合者,太极生两仪,阴阳本太极。
转动PDCA之环,勇往直前。

中庸的诚字,是通往专家之路的最好指南。诚之者,择善而固执之者也。
诚则明,明则诚。至诚无息。

择善,天命之谓性,天命的呼唤。正念、良知,皆诚意也。故诚,涵正念、良知、天命而为一。

孙言及Q4的安排,技术、管理统一于道,能深切地领会一下管理,也是一个大机会。

\subsection{0717}

兵法家最切实用。兵之胜败,本在于政。明一者皇,察道者帝,通德者王。谋得兵胜者霸。

凡战之道:位欲严;政欲栗;力欲窕;气欲闲;心欲一。

夫圆者,天也;方者,地也。天圆而无端,故不可得而观;地方而无垠,故莫能窥其门。天化育而无形象,地生长而无计量,浑浑沉沉,孰知其藏。
凡物有朕,唯道无朕。所以无朕者,以其无常形势也。轮转而无穷,象日月之运行,若春秋有代谢,若日月有昼夜,终而复始,明而复晦,莫能得其纪。
制刑而无刑,故功可成;物物而不物,故胜而不屈。刑兵之极也,至于无刑,可谓极之矣。
是故大兵无创,与鬼神通,五兵不厉,天下莫之敢当。
建鼓不出库,诸侯莫不慑㥄沮胆其处。故\hl{庙战者帝,神化者王。所谓庙战者,法天道也;神化者,法四时也}。
修政于境内,而远方慕其德;制胜于未战,而诸侯服其威。内政治也。

道生法,法生兵。

\subsection{0718}

昔者,越王句践问范子曰:“古之贤主、圣王之治,何左何右?何去何取?”
范子对曰:“臣闻\hl{圣主之治,左道右术,去末取实}。”
越王曰:“何谓道?何谓术?何谓末?何谓实?”
范子对曰:“道者,天地先生,不知老;曲成万物,不名巧。故谓之道。道生气,气生阴,阴生阳,阳生天地。天地立,然后有寒暑、燥湿、日月、星辰、四时,而万物备。
术者,天意也。盛夏之时,万物遂长。圣人缘天心,助天喜,乐万物之长。故舜弹五弦之琴,歌南风之诗,而天下治。言其乐与天下同也。当是之时,颂声作。
所谓末者,名也。故名过实,则百姓不附亲,贤士不为用。而外□诸侯,圣主不为也。
所谓实者,谷□也,得人心,任贤士也。凡此四者,邦之宝也。”

天下大乱,贤圣不明,道德不一,天下多得一察焉以自好。譬如耳目鼻口,皆有所明,不能相通。犹百家众技也,皆有所长,时有所用。
虽然,不该不遍,一曲之士也。判天地之美,析万物之理,察古人之全,寡能备于天地之美,称神明之容。
是故\hl{内圣外王之道},暗而不明,郁而不发,天下之人各为其所欲焉以自为方。
悲夫!百家往而不反,必不合矣。后世之学者,不幸不见天地之纯,古人之大体,道术将为天下裂。

圣人内修道术而不外饰仁义,知九窍四支之宜,而游乎精神之和,此圣人之游也。

是故圣人持养其神,和弱其气,平夷其形,而与道浮沉,如此则万物之化无不偶也,百事之变无不应也。
按:形-神-气,三者内。与道浮沉,上行下行。

以政治国,以奇用兵,以无事取天下。

以道观之,提供了一个方法论,基于原则的生活。原则,用系统来工作等理念,都是如此。

反面的做法是什么?以道观之的方法提供了什么好处呢?

心与物的直接接触,易于陷入事务、琐碎的境地而不自觉,迷失于物之中。
物是具体的万,而道则是抽象的一。

阳明的良知学也走上了同样的路,一方面致良知,另方面则是事上磨练。
所谓致我心之良知,则事事物物皆得其理。道心即是良知,心与道合,则光明普照万物,一一得其神采。

一个致字,指明不仅仅是感而后应,更在于主动地运用心中的道,去重塑外在的世界(reshape)。
这一点体现了极大的自由度。

\subsection{0719}

三合结构,作为阳明心学的解释框架。从心出发,红尘练心。
良知即是道、天理、原则。物,意之所在,扩大了物的内涵。
内圣外王之道,性之德也,合内外之道,故时措之意也。

进入定量有恒地体道阶段,道以优游故能化,道,贵周贵密,贵舒贵宽。心欲一,气欲闲。
别的时间专注在专业领域。第三空间,或傅钟所说的每天21个小时,余下3个小时是用来思考的。
所以,在生活、工作之外,开辟第三空间,志于道,太极元气,含三为一,三者最终达到致中和的理想状态。
此内圣外王之道也。

\subsection{0720}

阳明、国藩、泽东所以能立功、立言、立德,在于以学为大本大源,用学术道术统御,此心不动随机而动。
心术、道术、治术。以学术为动力。一学化三术,本立而道生。

读书在涵养心体,不解。涵养须用敬,进学则在致知。致知、良知,不易理解。认知升级、认知闭环。
用三合架构诠释阳明心学,如合符节。三合纵横交错,太极生两仪,两仪互为其根,与道浮沉。

八正道摄于三学,三学摄于一心。马一浮六艺摄于一心,心,一通向道,二通向物。
上行下行的分界,朱陆异同。

大学问,致良知,乃为学头脑处。尊德性而道问学,道不仅仅是德性之知,万物得之而成,失之而亡的道的内涵,要丰富得多。

苏格拉底等希腊三贤,约略对应于心、道、物三角,以柏拉图的理念论为顶点。

\subsection{0723}

用系统来工作:分离原理。5年时间,解决生活、事业的重大问题。

为学日益,为道日损。损之又损,以至于无为,无为而无不为。取天下常以无事,及其有事,不足以取天下。

如何才能无事取天下?有赖于道法、原则、系统等理念,知行合一。理解受控系统的工作机制,持续不断地检查调节,最优化。

PDCA与金字塔原理是形式,内容则是要达成的目标、要解决的问题。

当前主要的目标是道艺两方面的成长,也可以摄于一件事,致良知。参一治二,道艺、知行、体用合一。
从书本里走出来,发明心体。格物穷理的旧路,有个范式迁移运动。

首先要管理的是学习子系统,这个占据了太大的比例。其次是工作与生活、理财等方面。

今年要掌握的主要思维模型:三合之道、五行。三合之道突出道的主导性作用,五行突出中央的目标。

财务自由是做正确事的结果,根据观察,以下途径有望快速实现财务自由:
\begin{enumbox}
\item 专业
\item 创业
\item 跟对领导
\item 买房
\item 理财 p2p
\item Bit币
\end{enumbox}

专业是底线,专业方面的成长具有根本性的意义。心道艺三合。

为道日损,轻装上阵。不停买书、不停看书,又有哪些真正服务于达成目标呢?诚然,知道了一些知识,但不能转化,又有什么用?
目标是什么?尽快达成财务自由,应围绕目标组织一切活动。

第一,最根本的是思维方式、方法论,或者直接叫做道。然后运用此道到事事物物,则事事物物皆得其理。

事事物物中最重要的是专业能力。在心性与道的加持下,去完成专业能力的积累和变现。

道不可见,唯有具有道心方能见道。惟天下至诚,为能尽其性。
心的质地是第一位的,阳明致良知的起点与归属,都是心。

大学、中庸、道德经已经包括了所有的理念。

从心出发,志于道、游于艺。

\subsection{0724}

读古籍需要诠释系统,没有诠释系统犹如盲人摸象,把握不住要点。毕建勋围绕画道构建的123哲学,就是解读传统文化的一把金钥匙。
不仅如此,123哲学有更普适的价值。先攻下这个山头,印证所得。

从一点提炼出精义,再由精义引申到全体,一即一切、一切即一,一花一世界一叶一菩提,这是读书做学问干事业的一个要点。
单点突破,全面开花。一花与一世界之间,是一座山,透过道的力量,才能翻山越岭,顺利抵达目的地。

所以读书不要想着背,而是围绕关键字,画龙点睛,以字通道,乃合乎精一之宗旨。

天网恢恢疏而不漏,网上扭结就是范畴、概念。此方法,并不会遗漏什么,而独观其大略,进而为我所用。
若陷入支离破碎的文字之中,离道则远矣。

学习的艺术:画小圈,让时间慢下来的策略,同样适用。

道生一、一生二、二生三、三生万物。二生三则成闭合结构。

心善渊,幽深貌。静以幽、正以治,正道而行。

取势、明道、优术,道总要放在顶点的位置,尊道贵德。不具有道心,又如何能成功取势呢?

\subsection{0725}

心物道三角,二中见一和三是功夫。一是形而上的先在、同在,兼三故成三合之道。

道者,物之极,万物之宗。形而上者之谓道,形而上者之谓器。道心惟微,幽微之间。

有无,相反相成,谓之玄同。

% \subsection{0726}

% \subsection{0727}

\subsection{0730}

阳明归宗于致良知,惟精是惟一的功夫,博文是约礼的功夫。毕建勋提出了一二三哲学,进而建立了道心物的三合架构。
在三合架构下去理解阳明学,很是给力。道居于顶点,与物无对,乃绝对之存在。
心物构成两基础,精神文明与物质文明,两手都要硬。
三而一,一而三,良知具有三位一体的性质,通过心-理-知-行,

格物致知,缺少了道的中介,就事倍功半。引入了道,原则、系统,则依道而行,以一而治万,一本万殊,
自然无为,事少而功多。进而达到无为而治的理想境地。

认识是S曲线发展的,犹如太极图的中间曲线(正弦曲线)。

毕建勋对一二三哲学的处理过程,是归纳与演绎的统一。从老子四十二章,提炼出一二三精义,以此一二三精义,
去诠释老子文本,进一步提炼出一二三哲学,以此一二三哲学,作为画道的精髓。
正三角和倒三角衔接起来,是正弦曲线的简化形态。

S曲线是成长曲线,第二曲线与第一曲线是不连续的鸿沟,第一性原理用来跨越不连续的鸿沟,
即以道观之,第一曲线与第二曲线则是稳定的连续的。

致良知是有无、动静、本体与功夫、本末、道艺、体用的统一。无善无恶心之体四句教,有生于无,体用一源、显微无间。

致良知是内圣外王之道,如何通向王道经营的唯一道路。致吾心良知之天理于事事物物,则事事物物皆得其理。

易与天地准,故能弥论天地之道,知周乎万物而道济天下而不过,范围天地之化而不过,曲成万物而不遗,通乎昼夜之道而知。
故神无方而易无体。

曲则全,枉则直,洼则盈,敝则新,少则得,多则惑,是以圣人抱一以为天下式。

诚则明,明则诚。诚者天之道也,诚之者人之道也。诚者,物之终始,不诚无物。

养心莫善于诚,致诚则无它事。

精于道者兼物物,荀子解蔽有精义。虚壹而静。

反思是唯一的路,心以藏心,心之中又有心焉。一心二门,人心道心,迷悟之间。解蔽、玄鉴,大圆镜智。
放在良知的前面,判定其是非得失善恶,加减乘除。扩充心,大其心,至于美大圣神,天地万物一体之仁。
心外无物,在实践过程中,能力圈,扩大自己的能力圈和影响圈,能量场。天地与我并生,万物与我为一。

唤醒、激活。以简御繁。

本体、主体,实体即主体。纯粹理性、实践理性,我与非我。德国古典哲学。希腊三贤。东西比较。

专业:志于道游于艺,更进一步,关于职业事业的考虑。

了凡四训,立命之学、改过之法、积善之方、谦德之效

\subsection{0731}

太极生两仪,阴阳一太极,太极与两仪的二一关系去理解动静,则得之。定为动静之体,体为静,用为动。
寂然不动,感而遂通。理者气之条理,气者理之流行。解蔽纠偏,求静之心,则喜静厌动,不能应事接物。
动中求静,静中求动,动静一如,显微无间。其实一也,循理则静。

良知是为学头脑处。为道是为学的本体,为学是为道的功夫。反求本体,承体启用,曲成万物而不遗。

在心体上用功,何解?

乐是心之本体。常快活即是功夫。什么是孔颜之乐?

如何化解当前之困局?

对致良知,将信将疑。信解行证,则是正确选择。确信阳明不欺人,求解其真含义,知行合一,证悟得道。
闻思修,我今见闻得受持,愿解如来真实义。

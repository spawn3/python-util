\subsection{0225}

建立认知学,关于认知之重要性,至阳明心学而彰显。

西方哲学则对此认知早。从柏拉图始,至hegel而盛。

\subsection{0228}

读瑜颖正的人生算法,master夜访孤独大脑,两眼论,受益匪浅。特别是闭环认知,可谓集大成之作。
从李笑来,傅盛,吴军等,知道了认知,见识的重要性。船山说:志如其量,量如其识。
志、量、识,三者都重要,而见识为先。

傅盛是站在CEO的角度写作的,成长、管理、战略三部曲,以认知为中心,才知道什么是最重要的,什么是好的。
吴军具有国际化视野,李笑来经历丰富,有很多心得体会。
而瑜颖正则不知其经历,是这几个人中归纳能力超群者。

为了更好地工作和生活,不断地进行认知升级,实具有大价值。高频地运转自己的闭环认知系统,实现复利效应和指数级增长。

先简略归纳一下人生算法的大略:
\begin{compactenum}
\item 万物运转,有其算法。上帝设计算法,而不废自由意志
\item 向厉害的人学习,学习的要点就是其底层算法
\item 决策、科学、算法、人文,两两相对,对立而统一
\item 详细介绍了闭环认知的操作系统(4+2)
\item 以九段论践行闭环认知,分为两个阶段:回到核心、从核心出发
\end{compactenum}

能量=核心算法乘以大量重复动作的平方。提及了专注力的重要性,专注力加上质能方程,可所向披靡,势如破竹。

闭环认知系统可以作为分析、诊断的基本工具。节点、连接、内核、外环四大要素。比简单的知行合一更精细了一层。
核心对应李笑来的大刀,外环对应万能钥匙。

时至今日,需要深刻地反思,以规划好前进的道路。战略罗盘,PDCA都是四象限系统。取势、明道、优术,都类似。
贵无、致知、取势、明道、优术,则五行备。

进化论、基因、神经网络。把闭环认知植入基因深处。

闭环认知可作为学习系统,感知到某种有趣现象或问题,深入认知的基础上,做出决策,付之行动,周而复始。
外环是换框,必要时切换赛道,发生跃迁。高速运转该框架,达到快速成长的目的。
以此去分析成长、管理、战略等主题,会有什么新发现?

鬼谷子有转圆之说,切题。PDCA,控制论、圆点、双线大体类似。

节奏,经常STOP,如高尔夫的挥杆动作分解。划小圈也。

以上知识总结起来,学以致用。

最最重要是切割出核心算法,珍之重之。

分布式块存储系统是目前的核心。

\section{201808}

\subsection{0801}

近思录、传习录,细品皆极为有味。宋明学术,坐谈心性,不可少。内圣外王之道,于是乎明。

涵养须用敬,进学则在致知。细品这句话,意味深长。敬天爱人,致良知,合内外,为学须有头脑。

虚心涵泳,切己体察。朱子学,并不全是支离,但立言不易,差之毫厘谬以千里,不可不明辨。

知行合一,知行为太极之两仪,两仪之一不能代替太极本身,分二合一,要在明了立言宗旨,勿为纷纷知解,失却身心有用之学。
不知而行,知而不行,都是常见病痛。两者俱要真切笃实,才得受用。

阳明爱精一之旨,抱一以为天下式。

白天游于艺,晚上志于道。志道游艺亦为太极之两仪,彼此涵盖,交相辉映。道中有艺,艺中有道,道艺合一。
道艺、知行、体用合一。

以艺代物字,道心艺,开物成务。

运转PDCA,P为知,D为行,事前事中事后。

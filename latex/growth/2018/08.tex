\section{201808}

\subsection{0801}

近思录、传习录,细品皆极为有味。宋明学术,坐谈心性,不可少。内圣外王之道,于是乎明。

涵养须用敬,进学则在致知。细品这句话,意味深长。敬天爱人,致良知,合内外,为学须有头脑。

虚心涵泳,切己体察。朱子学,并不全是支离,但立言不易,差之毫厘谬以千里,不可不明辨。

知行合一,知行为太极之两仪,两仪之一不能代替太极本身,分二合一,要在明了立言宗旨,勿为纷纷知解,失却身心有用之学。
不知而行,知而不行,都是常见病痛。两者俱要真切笃实,才得受用。

阳明爱精一之旨,抱一以为天下式。

以艺代物字,道心艺,开物成务。

运转PDCA,P为知,D为行,事前事中事后。

\hl{艺也分为基础与项目知识},彻上彻下,理事无碍,事事无碍。

通乎昼夜之道而知。白天游于艺,晚上志于道。志道游艺亦为太极之两仪,彼此涵盖,交相辉映。
道中有艺,艺中有道,道艺合一。道艺、知行、体用合一。

志于道,当作何解?如何评估是否进步?阳明良知之教,可以作为目标。信解行证,知行合一。

\hl{参伍以变,错综其数}。三对应原则,五则对应系统来工作,对每一子系统按PDCA进行滚动。
三是空间结构,五是动态演化。志于道,三合之道;游于艺,PDCA 4+1=5。

参考资料:
\begin{enumbox}
\item 道德经
\item 系辞传
\item 近思录
\item 传习录
\item 谈谈方法
\item 穷查理宝典
\item 原则
\item 用系统来工作
\item PDCA
\end{enumbox}

不要急着看书,收敛身心,立志,迁善改过。勿忘勿助,欲速则不达。
收敛身心,一意本源,致良知是也。

千事万事,只是一事。读书云云,属于物这一极。三极之道,有三角形、八卦两种形式。

双线法则,原点哲学,也是层次与环转结构。

转化消极的情绪,直面现实,走好明天的路。

2+1 = 3, 4 + 1 = 5,2,4常见,一重要,加一后,有能级跃迁。

\subsection{0802}

倒三角,更见功夫,一刻不能涵养省察,则左摇右摆,远离中道。
以道莅天下,物各付物,各得其位。

一生二,二生三,三与一重叠则成三角,三与一相对则成四角,四时运转不废。
得一则天清地宁,侯王得一以为天下正。

程朱陆王的差别并不那么大,陆王学有头脑,先立其大者,则小者不能夺。

\section{201808}

\subsection{0801}

近思录、传习录,细品皆极为有味。宋明学术,坐谈心性,不可少。内圣外王之道,于是乎明。

涵养须用敬,进学则在致知。细品这句话,意味深长。敬天爱人,致良知,合内外,为学须有头脑。

虚心涵泳,切己体察。朱子学,并不全是支离,但立言不易,差之毫厘谬以千里,不可不明辨。

知行合一,知行为太极之两仪,两仪之一不能代替太极本身,分二合一,要在明了立言宗旨,勿为纷纷知解,失却身心有用之学。
不知而行,知而不行,都是常见病痛。两者俱要真切笃实,才得受用。

阳明爱精一之旨,抱一以为天下式。

以艺代物字,道心艺,开物成务。

运转PDCA,P为知,D为行,事前事中事后。

\hl{艺也分为基础与项目知识},彻上彻下,理事无碍,事事无碍。

通乎昼夜之道而知。白天游于艺,晚上志于道。志道游艺亦为太极之两仪,彼此涵盖,交相辉映。
道中有艺,艺中有道,道艺合一。道艺、知行、体用合一。

志于道,当作何解?如何评估是否进步?阳明良知之教,可以作为目标。信解行证,知行合一。

\hl{参伍以变,错综其数}。三对应原则,五则对应系统来工作,对每一子系统按PDCA进行滚动。
三是空间结构,五是动态演化。志于道,三合之道;游于艺,PDCA 4+1=5。

参考资料:
\begin{enumbox}
\item 道德经
\item 系辞传
\item 近思录
\item 传习录
\item 谈谈方法
\item 穷查理宝典
\item 原则
\item 用系统来工作
\item PDCA
\end{enumbox}

不要急着看书,收敛身心,立志,迁善改过。勿忘勿助,欲速则不达。
收敛身心,一意本源,致良知是也。

千事万事,只是一事。读书云云,属于物这一极。三极之道,有三角形、八卦两种形式。

双线法则,原点哲学,也是层次与环转结构。

转化消极的情绪,直面现实,走好明天的路。

2+1 = 3, 4 + 1 = 5,2,4常见,一重要,加一后,有能级跃迁。

\subsection{0802}

倒三角,更见功夫,一刻不能涵养省察,则左摇右摆,远离中道。
以道莅天下,物各付物,各得其位。

一生二,二生三,三与一重叠则成三角,三与一相对则成四角,四时运转不废。
得一则天清地宁,侯王得一以为天下正。

程朱陆王的差别并不那么大,陆王学有头脑,先立其大者,则小者不能夺。

\subsection{0803}

谋划走出大山。上山西乡,走出大山,意义非凡。定战略,建班子,带队伍。

SSD是一场革命,从硬件席卷软件,能做些什么?

Y既然冥顽不化,辅助之是否划得来?团结一切可以团结的力量,做出一番事业来。
既然是开启事业,就要潜谋于庙堂之上,决胜于千里之外。

80\%的时间应该放在专业上,20\%的时间用来悟道,目前看,知行两端,行的力度差点。

良知说来说去,切勿被绕进去,本意是真切简洁,真切故简洁,简洁故真切。
如果又进入所谓经院哲学,则惑矣。

12345,上山打老虎。12345,五个数字,包含了全部的方法论。
四五即是PDCA,一切的管理方法都是笛卡尔方法。在形下的经验世界,是唯一的方法。
既是结构,又是时序。

对这些人,学习其优点,不予置评了,纯粹浪费时间。必有事焉,关注于自身的大事要事。
在力所能及的范围内,最小化内核,最大化价值。

回到价值创造的源头,立定脚跟。

知行合一,都是学,不可离开学。一念发动处,即是知,也是行。

做个有心人,潜移默化之中,去做自己的事情。好高骛远,不是坏事,分时间阶段。

舍我其谁哉?

进入B模式,ESBI,B模式,生意靠系统,生意即系统。

周末细细梳理一下。

\subsection{0805}

椭圆是圆点哲学的更普遍形式,圆点是椭圆思维的极限形式,即两个焦点重叠为一。

椭圆的两个焦点可以构造123哲学,椭圆的圆可以构造四时无形。用椭圆结构统一12345哲学。
参伍以变,错综其数。通其变,遂成天下之文;极其数,遂定天下之象。

三角形的顶点在椭圆轨道上滑动,体常而应变,是原则性与灵活性的统一。

U型理论,沉下来沉下来,高纬度思考法,升维、俯瞰。

一分为三、圆点哲学、双线法则、三合之道、三才之道等,四象限,构成新哲学。

\subsection{0806}

过去、现在、未来构成三角,顶点是现在、当下。两个脉络,把握当下、经营未来。

过去-未来构成时间轴,为何把现在抽离出来,放在顶点呢?唯一可以把握的是当下之一念,通今。

求道者,要自己开辟通向未来的道路,遵循合适的方法,通达未来。

转动PDCA之环,各个击破。

把一个节点展开为一个过程。4+1,对一的分析用三。由一二三的分析,得出目标,用PDCA去完成该目标。
目标是层次结构,相应地PDCA也是层次结构。

全面总结多年来的技术经验,构建知识体系。重视基础、算法,有强烈的问题意思。
行动学习的两个要素:洞察力的问题,程序化的知识。

\subsection{0807}

日本经营四圣,梦想力,走出大山,万事开头难,从经营与管理的角度去看待专业问题。
既要学习管理学家的书,更要学习大企业家的书,
体会其关键思想。越过生灭不定的表象,洞察不生不灭的本性。

至于人事等外缘,合则留,不合则去,无复独多虑。开创出一番事业,激流在心底涌动。
整合能整合的一切资源,团结一切可以团结的力量。易经何为者也?开物成务,冒天下之道,如斯而已。
夫易,圣人之所以崇德而广业也。不知易,不足为将相。从这个角度去读易经,当有大收获。

太极图说、黄帝阴符经、心要法门等三篇文章,义理丰富,反复道也,而大易为宗。

用阳明良知之学去检验所学,及时发现问题,克己复礼。

执古之道,以御今之有。大易为纲,会通中西古今。

易为规范,乾坤是样板。这个解读有新意,易经每一卦是一个系统,构成更大的系统,系统分类原则很简单。
如此解读,统一了原则、用系统来工作等理念。

并非当务之急,慢慢来。

什么导致成功?精通导致成功与热情。技术人,精通,守破离。诚则明,明则诚,通向成功。

用三合之道去理解,热爱、专注、绝活,三者是相乘的关系。

稻盛和夫的成功方程式:成功=思维方法-热诚-能力。思维方式是良知、是明道。心-道-物,
物是产品,道生之、德蓄之、物形之、势成之。

三合之道即是成功的定义,也是成功的方程式。三合之道是成功方程式。

常常运行三合之道,去分析梳理理念,体用不二,有无相生,动静一如。

鼓起探索未知的勇气,再厉害的人,都有盲区。在不断尝试中得到成长。

孙子兵法的组织能力。

成功的why、what、how

学习方法:行动学习,始于洞察力的问题,是以君子将有为也、将有行也,问焉而以言,其受命也如响,
无有远近幽深,遂知来物,非天下之至精,其孰能与于此哉?

本质上是二一关系,成三合之道。

陈云十五字诀

\subsection{0808}

成功之道,精通重于热诚,在擅长的领域开创事业。

把易经学习作为项目管理起来,走PDCA循环。

\begin{enumbox}
\item 易经
\item 易传
\item 易学
\item 易图
\end{enumbox}

用来回答所面临的重大问题:个人与组织成功之道。
修己安人,己欲立而立人,先把自己立起来。潜龙勿用,厚积薄发。

首要的一点,就是明确自己的使命,确立志向、目标、任务。

学习易经,重点在于掌握其唯物辩证的思维方式,三十辐共一毂,端立中央,独立不改、周行不怠。
为将者,天地定位,修道保法,运筹帷幄之中,决胜千里之外。

为什么说易经是众经之首、三玄之冠、百家之源?读易已久,为何不见受用处?什么才是正确的学易法?

心易、心为太极。三点贯通,乃成三合。

逐段解读系辞传,勿贪多,提炼其精义,循序渐进。乾知大始、坤作成物,惟乾之健能克服万事开头难的难,乾之动非盲动,而是反复道也。

一字一义,一字多义。

易为经、旁学为纬,经纬分明,纵横交织,纲举目张。

知常曰明,不知常妄作凶。动静有常,刚柔断矣。

\subsection{0809}

PDCA的4+1循环是战略执行的形式,知行合一。内容则有赖于三合之道,背后蕴含着基本前提与价值观。
每一个象限都有丰富内容,规划包含达里奥目标五阶段的三个节点:识别问题、分析根本原因、设计解决方案。
检查对应于柳传志提倡的复盘、日稽查。规划的输出应是用系统来工作的三级文档。

三中有五、五中有三、三三五五、大道至简。每个要素皆内在地含有无极之真、二五之精,层层无尽。

分析的重点是三合之道,如夸克、中微子皆呈现三态。三五错综,极尽天下事。

问、学、习,问代表好奇心,不仅如此,更有深刻的意义。强烈的问题意识,是做事的一个起点。

PDCA可用圆或椭圆为几何形式,其数一二三四是也,数与形结合,奠定易经象数为根本的旨趣。

回到易经,不在于易经本身,尽信书不如无书,而在于以此为文本基础,阐发体悟我之象数学。
世界之本源在数,毕达哥拉斯的这一论断,极富启发意义。柏拉图进一步指出了共相论,在物理学之后,有形而上学在焉。
形而上者之谓道,形而下者之谓器。此有无两重世界,一心二门,是思想的一个显著特征。

大小精粗,其运无处不在。

易数,乃天地之数,一气流行,隐显聚散不同,森然万象罗列。

阳明良知说解决了一个大问题,但不是全部内容。易的世界远为广大。夫易广矣大矣,易之为书也不可远,为道也屡迁,
易道广大,百物不废,惧以终始,其要无咎,其用柔中也。

易之为书也,广大悉备。

老子道论、阳明良知说,皆发挥易道之精义。良知说可看做行道的方法。道体非空非有,不生不灭,求之不得,弃之不理,如何才能把握?
至于庄子心斋,淮南天解、鬼谷道知,皆语焉不详,至阳明始详细论述悟道得道之要津,譬如渡河,良知为筏。
佛言六度八正道三十七道品,致良知一心以摄之。此精一之学。天下之动,贞夫一者。

严格按照PDCA运转每一个事情,包括生活、学习、工作等等。上班时间专注于技术与业务,第三空间才更多读易经等等,以相互促进。
必须\hl{转移到以技术与业务为中心的状态}, 下面的路才好走,才更容易取得突破。多年的脚踏两条船,从实际效果看来,恨不能令人满意。

PDCA可以统摄原则、用系统来工作等内容,相互借鉴。变通莫大乎四时,春秋秋冬、元亨利贞。
更深刻的模型是椭圆。从圆的一个中心进化到椭圆的两个中心。

道生一、一生二、二生三、三生万物。万物运行在PDCA的椭圆轨道上。道与物不二,道在物中,在物上。
椭圆轨道,体常而应变,既有原则,又有灵活性,独立不改,周行不怠。

少飞提及任督二脉,数据平面与控制平面,有点启发。

三合与五行。首要的试金石是解决职场问题。工作多年,没有大的起色,不能不反思其中缘由。

\subsection{0810}

重要的思维模型,在易经中已有很好体现,并且具有高境界,在道、战略的层面进行的。

易有太极,是生两仪,两仪生四象,四象生八卦,八卦定吉凶,吉凶生大业。

其言曲而中,其事肆而隐,因二以济民行,以明失得之报。

易大传中有所提点,关节处融会贯通,化而裁之、推而行之,裁成天地之道,辅相万物之宜。
一处通,处处通。

对称与对称破缺是有趣有用的主体。道与阴阳是对称破缺,一阴一阳是对称。

\subsection{0813}

乾坤衍,引申乾坤之义。由字通词,由词通道。一字之义,当贯群经。简练以为揣摩(用心深细)。

约,失之鲜矣;诚,乐莫大焉。

其数123456是也,一是易经象数系统,二是数学物理。易学与科学相辅相成。

以三来说,天地人三才之道,贯通则为王为丰。三才,天地人,人立于天地之间,当顶天立地。
我们的事业是什么,我们的事业依靠什么?

如何安排道艺,依旧没有打通。明理易,用理为难,用理在于圣时、权权。
由一字一义而明理,由圣时权权而用理,则三合之道成。

专利?

三五,三是结构性分析,五对三分析后的每个节点是如何落地如何达到。
有此两方面足矣,不必再骑驴觅驴,头上安头,画蛇添足。

卦、象、爻,象特指三画卦,卦指六画卦,爻有时位阴阳之别。
爻组合而成象,象组合而成卦。

三是分析的起点,三生万物,三是三合之道。三三组合而成六。

接下来的五年,年年都重要,要去完成几件大事。

心中已有一框架,构建于易经、道德经、孙子兵法。

% 高筑墙,广积粮,缓称王
致良知,行王道,赚大钱

为学为道,为艺为道,孰轻孰重?目前稍显纠结,没有圆融无碍,当做合适安排。
为学日益,为道日损。道不离器,器不离道。诚如阳明所说,为学当有头脑,致良知是头脑。
道心物,学也是一物,在正确的心态与思维方式的指引下,能力的提升。
脱离了艺与器,抽象地谈论道,离道日远。

南辕北辙,缘木求鱼。艺仍是第一位的,不攻下这个山头,道就显示不出来,不能证明自己。
\hl{从明日始,严格制定下一阶段的学习工作计划,日日审视之}。

涵养须用敬,进学则在致知。敬,敬业。敬天爱人,重要。

心物之间本为直线,却非距离最短。曲成万物而不遗,须以道为中介,执玄德于心而化驰若神。
这种间接路线,反而能最快达到,违反直觉,却实实在在如此,宜深思。

如何论证道的无上威力?把本职工作做好,把握大机遇,己立立人,己达达人,掉书袋不可取。

\subsection{0814}

提问是一门艺术。如何问,如何答,构成对话。依照对话对象的不同,有自我对话等等。
自我对话是反思的主要形式,自问自答是学习的主要方式。

学习不是记问之学,而是聚焦于关键问题,持续地追问,打破砂锅问到底。博学之,审问之,慎思之,明辨之,笃行之。
问是思维的原点,在好奇心的引导下,辅之以博学、慎思,明辨,笃行的功夫,以问题为中心的过程的展开。

学记值得好好读,师说。

提问离不开目的与目标,不是漫无边际,围绕目标,一问一答。经营公司,必善于提问。个人发展,必善于提问。

最近念念不忘的三五哲学,如何简练地表达出来?如何运用到实际的生活、学习与工作过程中来,变现为现实的力量?
提问为三五哲学注入了动力。

陈述体系是封闭的,而提问则开放而专注。提问能聚焦力量,何道何术,何去何取。战略是舍九取一,舍而后得。

最重要的是什么? 为什么如此焦虑?时间还有,何不从容计议?

\subsection{0820}

123是道,456是术,道术一体,道展开为术。反向修义,借假修真,层层修分,处处求中。此一求三修之心法,值得细细揣摩。

一部易经观天下,已经不适合多看书,少而精是正解。

扬长避短,发挥优势,最小化内核,最大化价值。

一卦可变成任意卦,要在机。机发论是易道主义的理论枢纽,极深而研几,阴符经论机,值得关注。

恒卦当重视。不恒其德,或承其羞,言犹在耳,今日困局,乃自己所致,怨不得人。
行有不得,反求诸己。

思考周易的构建原则,无极而太极,易有太极,是生两仪,两仪生四象,四象生八卦,八卦定吉凶,吉凶生大业。
不要把易经想象得很难,理解一点运用一点,日新之谓盛德,潜移默化之中,大道存焉。

盛德大业是易经的本质追求,富有之谓大业,日新之谓盛德,生生之谓易。问题是,易经是如何实现这两点的呢?
君子进德修业,欲及时也,故无咎。

\subsection{0821}

层次:六十四卦,卦,象,爻。通变之谓事,易穷则变,变则通,通则久。爻变导致象变卦变,在六十四卦里准确地定位。

变发生在很多层面。变通莫大乎四时,崇高莫大乎富贵。

\subsection{0822}

看书有点看不进去了,以后重点放在输出上,输出倒逼输入。此正所谓练拳不练功,到老一场空。

中国武学、书画都与易经密切相关,大道存乎易经,易经存乎天地自然之间,太极拳是一种哲学拳,以拳证道。

不汲汲于阅读,就会有更多时间放在重要问题的思考上,一念清静,觉知当下。
时至今日,年近不惑,更觉时光宝贵,不容浪费。

大学、中庸、易经、道德经皆至理,细细玩味,必有所得,不必穷索。以有用之精神,役无用之事业,怎能算得上是明智之举呢?

道乃含二之一,即是一阴一阳之谓道。

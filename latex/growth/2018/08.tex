\section{201808}

\subsection{0801}

近思录、传习录,细品皆极为有味。宋明学术,坐谈心性,不可少。内圣外王之道,于是乎明。

涵养须用敬,进学则在致知。细品这句话,意味深长。敬天爱人,致良知,合内外,为学须有头脑。

虚心涵泳,切己体察。朱子学,并不全是支离,但立言不易,差之毫厘谬以千里,不可不明辨。

知行合一,知行为太极之两仪,两仪之一不能代替太极本身,分二合一,要在明了立言宗旨,勿为纷纷知解,失却身心有用之学。
不知而行,知而不行,都是常见病痛。两者俱要真切笃实,才得受用。

阳明爱精一之旨,抱一以为天下式。

以艺代物字,道心艺,开物成务。

运转PDCA,P为知,D为行,事前事中事后。

\hl{艺也分为基础与项目知识},彻上彻下,理事无碍,事事无碍。

通乎昼夜之道而知。白天游于艺,晚上志于道。志道游艺亦为太极之两仪,彼此涵盖,交相辉映。
道中有艺,艺中有道,道艺合一。道艺、知行、体用合一。

志于道,当作何解?如何评估是否进步?阳明良知之教,可以作为目标。信解行证,知行合一。

\hl{参伍以变,错综其数}。三对应原则,五则对应系统来工作,对每一子系统按PDCA进行滚动。
三是空间结构,五是动态演化。志于道,三合之道;游于艺,PDCA 4+1=5。

参考资料:
\begin{enumbox}
\item 道德经
\item 系辞传
\item 近思录
\item 传习录
\item 谈谈方法
\item 穷查理宝典
\item 原则
\item 用系统来工作
\item PDCA
\end{enumbox}

不要急着看书,收敛身心,立志,迁善改过。勿忘勿助,欲速则不达。
收敛身心,一意本源,致良知是也。

千事万事,只是一事。读书云云,属于物这一极。三极之道,有三角形、八卦两种形式。

双线法则,原点哲学,也是层次与环转结构。

转化消极的情绪,直面现实,走好明天的路。

2+1 = 3, 4 + 1 = 5,2,4常见,一重要,加一后,有能级跃迁。

\subsection{0802}

倒三角,更见功夫,一刻不能涵养省察,则左摇右摆,远离中道。
以道莅天下,物各付物,各得其位。

一生二,二生三,三与一重叠则成三角,三与一相对则成四角,四时运转不废。
得一则天清地宁,侯王得一以为天下正。

程朱陆王的差别并不那么大,陆王学有头脑,先立其大者,则小者不能夺。

\subsection{0803}

谋划走出大山。上山西乡,走出大山,意义非凡。定战略,建班子,带队伍。

SSD是一场革命,从硬件席卷软件,能做些什么?

Y既然冥顽不化,辅助之是否划得来?团结一切可以团结的力量,做出一番事业来。
既然是开启事业,就要潜谋于庙堂之上,决胜于千里之外。

80\%的时间应该放在专业上,20\%的时间用来悟道,目前看,知行两端,行的力度差点。

良知说来说去,切勿被绕进去,本意是真切简洁,真切故简洁,简洁故真切。
如果又进入所谓经院哲学,则惑矣。

12345,上山打老虎。12345,五个数字,包含了全部的方法论。
四五即是PDCA,一切的管理方法都是笛卡尔方法。在形下的经验世界,是唯一的方法。
既是结构,又是时序。

对这些人,学习其优点,不予置评了,纯粹浪费时间。必有事焉,关注于自身的大事要事。
在力所能及的范围内,最小化内核,最大化价值。

回到价值创造的源头,立定脚跟。

知行合一,都是学,不可离开学。一念发动处,即是知,也是行。

做个有心人,潜移默化之中,去做自己的事情。好高骛远,不是坏事,分时间阶段。

舍我其谁哉?

进入B模式,ESBI,B模式,生意靠系统,生意即系统。

周末细细梳理一下。

\subsection{0806}

过去、现在、未来构成三角,顶点是现在、当下。两个脉络,把握当下、经营未来。

过去-未来构成时间轴,为何把现在抽离出来,放在顶点呢?唯一可以把握的是当下之一念,通今。

求道者,要自己开辟通向未来的道路,遵循合适的方法,通达未来。

转动PDCA之环,各个击破。

把一个节点展开为一个过程。4+1,对一的分析用三。由一二三的分析,得出目标,用PDCA去完成该目标。
目标是层次结构,相应地PDCA也是层次结构。

全面总结多年来的技术经验,构建知识体系。重视基础、算法,有强烈的问题意思。
行动学习的两个要素:洞察力的问题,程序化的知识。

\subsection{0807}

日本经营四圣,梦想力,走出大山,万事开头难,从经营与管理的角度去看待专业问题。
既要学习管理学家的书,更要学习大企业家的书,
体会其关键思想。越过生灭不定的表象,洞察不生不灭的本性。

至于人事等外缘,合则留,不合则去,无复独多虑。开创出一番事业,激流在心底涌动。
整合能整合的一切资源,团结一切可以团结的力量。易经何为者也?开物成务,冒天下之道,如斯而已。
夫易,圣人之所以崇德而广业也。不知易,不足为将相。从这个角度去读易经,当有大收获。

太极图说、黄帝阴符经、心要法门等三篇文章,义理丰富,反复道也,而大易为宗。

用阳明良知之学去检验所学,及时发现问题,克己复礼。

执古之道,以御今之有。大易为纲,会通中西古今。

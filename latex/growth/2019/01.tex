\chapter{2019}

\section{01}

问题集:
\begin{enumbox}
\item 何为伪共享?
\item 何为ABA问题?
\item CAS过程
\item 什么是future?
\item 如何实现lock free数据结构?
\item 可重入、线程安全
\item 为什么选择redis替代sqlite3?
\item *** 元问题
\item 黄金圈法则
\item 最重要的是什么?本质是什么?
\item \hl{测试问题:这里坏了会如何?}
\item 更好的做法是什么?
\item 五个why
\end{enumbox}

\subsection{01}

进入2019,斗志昂扬。2019是个丰收年。

最重要的事情是什么?切勿捡了芝麻、丢了西瓜。放长线才能钓大鱼。知止不殆,这是显而易见的事情,为什么一再地犯同样错误?
阅读也好,锻炼身体也好,专业知识的学习也好,事业也好,乃至一切,都是同样的。把握主要矛盾,寻求突破点,通于一,万事毕。

得一以为天下牧,黄帝四经:执一明三定二,是天人之道。据于道0,用德一之能量统筹全局,无极而太极,太极本无极。
方法论解决了,具体如何落实呢?

万千义理,不过0和1。由此0和1,演绎三千大千世界。

\subsection{02}

分析结果,\hl{须大力提升专业能力,作为主要的突破点}。有着东西看似重要,但不是现阶段的当务之急,比如管理、投资、扯谈等等。
如何提升专业能力,须有方法论。方法论要简单、能持续进化。更重要的是要有目标、有方向,\hl{道法术器}诸多层面层层展开。
\hl{得其环中,以应无穷},是近二十年工作生活经验之总结,不可不坚决,坚如磐石。

并行:
\begin{enumbox}
\item 软件管道与SOA
\item 面向模式的软件架构
\item 结构化并行程序设计
\end{enumbox}

并行问题,需要结合硬件和操作系统去理解。多处理器系统具有不同的内存访问特征,
目前的做法包括:\hl{cpu亲和性、自定义内存分配器}。以提升数据时间和空间的局部性。

软件管道理论非常吸引人,简洁又深刻。提出了三规则:
\begin{enumbox}
\item 输入等于输出
\item 下游处理能力大于等于上游
\item 分配器远大于下游处理能力
\end{enumbox}

由此去分析几个现象:
\begin{enumbox}
\item lich服务器框架的三层结构
\item 性能分析和预测
\item disk故障下的恢复
\item 节点故障的恢复
\item QoS和限流
\item ***
\item 与\hl{排队论和TOC理论}相会通
\item 与petri net有相同之处,甚至可以上升到量化、数学化描述的程度
\end{enumbox}

并行意味着并发,并发不一定是并行,并行是并发的subclass。

控制流/数据,串行/并行,定义\hl{加速比}的概念。先讨论数据管理模式。数据条带化有助于提升并行度。

选择控制器的方式:为什么按第一副本定义控制器所在节点?因为控制器频繁变动,所以localize的成本过高。
如果数据均匀分布,则每个节点都可以充当卷控制器,且性能波动不大。

mapreduce是控制流和数据流的统一。

subvol延迟加载,并且只加载一次,适用于double check的场景。

理解了计算机的工作原理,就容易理解多线程调度情况下,一个语句可能对应多个机器指令,
调度可以发生在任意两个机器指令之间,有可能破坏了程序执行的原子性。

一个处理器内多核访问本地内存延迟一样,多个处理器才有所谓ccNUMA问题。
处理器结构分三级:\hl{多处理器、多核、超线程}。

TLB是虚拟地址到物理地址的缓存,如果命中,就不需要查进程页表了。

\subsection{03}

管道的故事,与其临渊羡鱼,不如退而结网。李嘉诚的奋斗过程,值得深思。学习、知识是改变命运最强大的力量。

改变命运没有什么诀窍,做好自己份内的事情,做到极致,就能形成有力突破点,破局而出、化茧为蝶。
若一味好高骛远,难免顾此失彼,积累不够,无法突破。常立志,不如立长志。

人生能做成那么一两件大事,已经不容易,想得太多,难以专注专精,自然打不出水。

好好研读\hl{李嘉诚、任正非、稻盛和夫}等企业家,看看是如何纵横捭阖,建立起自己的商业帝国。
\hl{六度是人生的向导},以戒为师,持戒是第一位的,空洞地谈论禅定、智慧,是无源之水无本之木。

计算机体系结构的演进:单处理器,多核,多处理器,集群,并行和分布式计算成为主流。操作系统的设计、汇编语言、高级语言、数据结构和算法。
这是都是基础中的基础,务必保持长期学习。

fork and join是一种并行模式,pipeline也是。pipeline是多阶段的,每个独立线程处理一个阶段,然后传给下一个处理线程。
扩展开来,就是SEDA架构。每个处理线程扩展为线程池,并且引入队列和调度策略。

统计词频是mapreduce并行算法的生动演示。

加锁导致串行化执行,降低加速比。加速比定义了并行加速的上限,与串行代码的比例成倒数关系。
工作量-跨度模型是更好的模型,\hl{用工作量和跨度的比值定义加速比的上限}。

并行陷阱,有些可用software pipelines理论解释,有些可用工作量-跨度模型解释,有些可用计算机体系结构解释。
需要从硬件和软件多个层面去说明。

贪婪调度策略,理想化方法,原子操作(单个内存变量)

中断具有最高优先级,可以引起进程的重新调度。0x0是时钟中断,0x80是syscall中断。

\hl{改进学习方法,以我为主},围绕关键概念来展开,概念图和思维导图。

研读leveldb和rocksdb,从bigtable提炼出的kv库,采用lsmt数据结构。lsv与这个项目的目标相同,转化随机写为顺序写。
引入了新的问题,一个key对应多个版本,需要gc。读需要合并,也需要cache来优化读性能。

多看论文,研究多种索引结构,如\hl{hash、dht、b tree,lsmt,bloom filter、bitmap、tier}等等及其变体和改进。
索引结构:bitmap、hash、b+tree、lsmt等。每种操作的成本分析

\subsection{04}

中和二字是真诀,守中致和、持经达变、守正出奇是两个方面的平衡。得其环中,以应无穷。
\hl{中是优势能力核},和是裂变式创新。中气以为和,中是圆心、和是圆环,抱一复中。

中和是天道,也是人道。人物志中说道:\hl{凡人之质,中和为最}。中和之质必平淡无味,故能调和五味,变化应节。
白居易兴五福消六极的方法就是中和,能守中致和则五福可兴,六极可消。

中就是专精一事的一事,即是思维方法,更是评估准则。即是价值观,又是方法论。
李嘉诚的演讲处处可见一种矛盾的张力,如果在对立两极之间达致平衡。最典型的就是\hl{建立自我,追求无我}这一命题。
谦虚、感恩、自律、达观。任正非也是,矛盾法则运用得炉火纯青。这是真正的太极哲学,不用再有怀疑。

数据结构:trie中key是有结构的,每一层对应key的一部分。radix tree是特殊的trie,key space是连续的线性空间。
如进程虚拟内存空间,lich中卷的chunk,叶子节点代表空间中的元素,中间节点是索引。

hash和b+tree的key都是原子性的,没有结构,可以进行比较。字典序或数值大小。btree的不同,在于利用了底层存储的特征。
按page组织每个节点,节点内部由若干条记录。

为什么linux内核采用radix tree来实现page cache,而不是hash?
tag dirty

bitmap表达能力有限,1bit表示两种状态,是或否,used or free。
mbind和cpu亲和性都是用bit mask来指定的。
bloom filter用于判定存在与否,但存在误判的可能。且delete操作难以表达。

基础数据结构也有融合趋势,每种都有使用场景和局限。
选择一个数据结构,要分析各操作的复杂度,以及并发、性能和扩展性,内存和外存的存储结构。
内存结构和磁盘结构要能一一对应,但明显是不同的。要能从磁盘结构方便地load到内存中。

存储结构有两种:\hl{数组、链表}。

从元数据、索引的角度来看,复杂度以此为\hl{对象、块、文件}。
文件具有最高的复杂度,表现在ceph里,就是有额外的MDS来提供元数据服务。

lich过于依赖ssh,etcd独立出来有很大的好处。用于管理集群级信息。
\hl{结合UMP考虑整体架构}。etcd+lichd+ump构成一个完整的系统。哪些地方可以改进?

HA始终是一个难度,如何做到全系统无单点故障?

年前主要做好\hl{快照、SPDK、RDMA}方面的工作。为下一步发展做好充分准备。

要充分理解选择每种数据结构和算法的缘由、利弊得失?

\subsection{05}

如何进入那1\%?核模式、华为的压强原则,都是守中致和。

黑洞吸收一切物质和能量,喷射射线暴,这是怎样有启发性的一个模型呀!核模式的升级版就是黑洞模式。

心法只有一条:守中致和。此乃公理,引发一切定律、引理。何为中?如何守?处处是开放问题,需要深度思考。
核模式是一之多元,聚变为核、裂变为万物。第一曲线与第二曲线之间。

守中致和即是太极哲学公理。软件管道的公理是输入输出守恒,
太极哲学公理则是守中致和,由一达万,欲求10x成长,此乃大道坦途。

\hl{好谋而成,分段治事,不疾而速,无为而治}。境行果,即体即用。好谋而成,所谋所成者即在中与和。
格局、布局尽在中与和之中。

\subsection{06}

把所有资源all in到不变的事情上。不变的事情,最主要是个人成长。心性、知识等各个方面。

反身而诚,乐莫大焉。诚是中庸精华。

\chapter{2019}

\section{01}

\subsection{01}

进入2019,斗志昂扬。2019是个丰收年。

最重要的事情是什么?切勿捡了芝麻、丢了西瓜。放长线才能钓大鱼。知止不殆,这是显而易见的事情,为什么一再地犯同样错误?
阅读也好,锻炼身体也好,专业知识的学习也好,事业也好,乃至一切,都是同样的。把握主要矛盾,寻求突破点,通于一,万事毕。

得一以为天下牧,黄帝四经:执一明三定二,是天人之道。据于道0,用德一之能量统筹全局,无极而太极,太极本无极。
方法论解决了,具体如何落实呢?

万千义理,不过0和1。由此0和1,演绎三千大千世界。

\subsection{02}

分析结果,\hl{须大力提升专业能力,作为主要的突破点}。有着东西看似重要,但不是现阶段的当务之急,比如管理、投资、扯谈等等。
如何提升专业能力,须有方法论。方法论要简单、能持续进化。更重要的是要有目标、有方向,\hl{道法术器}诸多层面层层展开。
\hl{得其环中,以应无穷},是近二十年工作生活经验之总结,不可不坚决,坚如磐石。

并行:
\begin{enumbox}
\item 软件管道与SOA
\item 面向模式的软件架构
\item 结构化并行程序设计
\end{enumbox}

并行问题,需要结合硬件和操作系统去理解。多处理器系统具有不同的内存访问特征,
目前的做法包括:\hl{cpu亲和性、自定义内存分配器}。以提升数据时间和空间的局部性。

软件管道理论非常吸引人,简洁又深刻。提出了三规则:
\begin{enumbox}
\item 输入等于输出
\item 下游处理能力大于等于上游
\item 分配器远大于下游处理能力
\end{enumbox}

由此去分析几个现象:
\begin{enumbox}
\item lich服务器框架的三层结构
\item 性能分析和预测
\item disk故障下的恢复
\item 节点故障的恢复
\item QoS和限流
\item ***
\item 与\hl{排队论和TOC理论}相会通
\item 与petri net有相同之处,甚至可以上升到量化、数学化描述的程度
\end{enumbox}

并行意味着并发,并发不一定是并行,并行是并发的subclass。

控制流/数据,串行/并行,定义\hl{加速比}的概念。先讨论数据管理模式。数据条带化有助于提升并行度。

选择控制器的方式:为什么按第一副本定义控制器所在节点?因为控制器频繁变动,所以localize的成本过高。
如果数据均匀分布,则每个节点都可以充当卷控制器,且性能波动不大。

mapreduce是控制流和数据流的统一。

可重入、线程安全

subvol延迟加载,并且只加载一次,适用于double check的场景。

问题集:
\begin{enumbox}
\item 何为伪共享?如何解决?
\item 何为ABA问题?如何解决?
\item 如何实现lock free数据结构?
\end{enumbox}

\subsection{03}

\subsection{04}

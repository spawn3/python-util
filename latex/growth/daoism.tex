\chapter{志于道}

诸子:
\begin{enumbox}
\item 易经
\item 中庸
\item 道德经
\item 文子
\item 黄帝四经
\item 黄帝阴符经
\item 鬼谷子
\item 管子
\item 素书
\item 武经七书
\item 长短经
\item 武艺二书
\end{enumbox}

佛经
\begin{enumbox}
\item 心经
\item 金刚经
\item 坛经
\item 大乘起信论
\end{enumbox}

近人之著作
\begin{enumbox}
\item 东方战略学
\item 李小龙
\item 查理芒格
\item 孙正义
\item 人生算法
\item 第一性原理
\item 第二曲线
\item 基业长青
\item 从优秀到卓越
\item 黑天鹅
\item 反脆弱
\end{enumbox}

道是起点,也是归属。

志于道、据于德、依于仁、游于艺,此四事,实是一事,志于道,道之展开,囊括无遗。

立志,立何等志?本立而道生。道是目的也是方法。

道如太阳,行星围绕着它运转不息。

信解行证,道是一种信仰,生起好奇心,去了解、去上下求索、去身体力行。

道者,生生之道,增长之道,一气流行阴阳变化之道。
一者,道之纪也。

道是一中心范畴,千变万化而不离其本,是环的心,也是环自身。道有其体,又有在各个领域里的广泛应用。

深入道源去原道。观天之道,执天之行,尽矣!以天地之道去洞察万物,则易知易行。

道是人生算法,也是第二曲线的哲学;是第一性原理,也是10X增长。

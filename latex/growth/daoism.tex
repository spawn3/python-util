\chapter{志于道}

诸子:
\begin{enumbox}
\item 易经
\item 中庸
\item 道德经
\item 文子
\item 黄帝四经
\item 黄帝阴符经
\item 鬼谷子
\item 管子
\item 素书
\item 长短经
\item 太极图说
\item 通书
\item 武经七书
\item 武艺二书
\end{enumbox}

佛经
\begin{enumbox}
\item 心经
\item 金刚经
\item 坛经
\item 大乘起信论
\item 圆觉经
\end{enumbox}

近人之著作
\begin{enumbox}
\item 一二三哲学
\item 东方战略学
\item 李小龙
\item 查理芒格
\item 孙正义
\item 人生算法
\item 第一性原理
\item 第二曲线
\item 基业长青
\item 从优秀到卓越
\item 黑天鹅
\item 反脆弱
\end{enumbox}

道是起点,也是归属。

志于道、据于德、依于仁、游于艺,此四事,实是一事,志于道,道之展开,囊括无遗。

立志,立何等志?本立而道生。道是目的也是方法。

\section{一体万化}

\subsection{空}

\begin{enumbox}
\item 空虚不毁万物为实
\item 天下万物生于有,有生于无。
\item 虚者道之常也,因者君之纲也。
\end{enumbox}

此空明之境,显示道的灵活性。

道生一,一生二,二生三,三生万物。一者,道之纪也。一具有中枢地位,万事开头难。

道是一,也是万,通书:是万为一,一实为万,万一各正,大小有定。

月印万川,一体万化,彰显了道的威力,一处悟入,处处得益。

道如太阳,行星围绕着它运转不息。
道如原子核,电子围之运转。

\subsection{无为}

道常无为而无不为,此语点出道的体相用。以无为体,以有为用。

从自然的角度讲,无为是因循自然,辅相万物之自然而不敢为。
从设计的角度讲,道则是大设计,透过系统设计,达到系统控制的目的。
这样两个角度、两个层次是统一的整体。

无为是出于自然而超乎自然的:裁成天地之道,辅相万物之宜。
并非一切顺其自然,而是为无为,设计出可以无为的系统来,而节约了劳动力,达到以小博大、无为而治的目的。

文化、制度、原则都是一些无为而治的措施,而不是所谓无政府主义、放任不管。

以正治国,以奇用兵,以无事取天下。这句霸气之极,没有奇正的合理应用,又怎么做到无事取天下呢?
无为是最高明的领导哲学,道生法,修道而保法,与法有着密切关系。

反思过往,不能不感到惭愧。无为理念知之很久了,依然把握不住关键,难得逍遥之旨。

\subsection{心术}

以心受道,以道控势。顺势而为,非道不行。有道则吉,无道则凶。
然则何为道也? 心如何受道得道,进而心想事成?

心性修养,不可谓不重。此精神力量,神存兵亡,乃为之形势,是做事的根本。
专业技能也好,领导力也好,为了更好的发展,是必要条件。

大学讲正心诚意,阳明心学大力发扬了这一命题。

道微妙难测,需要心术去把握。观天之道,以何观之?心眼也,思维方式也。

心生于物,死于物,机在目。目之所见,对心有正反两方面的作用,要扬长避短。

鬼谷著本经阴符七术,即要解决心术的问题。必有圣人之心,以不测之智,而通心术。
心能得一,乃有其术。

管子四篇:心术上下,内业,白心,关注的也是这一课题。

孙子曰:将军之事,静以幽,正以治。也是强调心性的力量。

\section{信解行证}

\subsection{信}

道是一种信仰,生起好奇心,去了解、去上下求索、去身体力行。

道是一中心范畴,千变万化而不离其本,是环的心,也是环自身。道有其体,又有在各个领域里的广泛应用。

\subsection{解}

道者,生生之道,增长之道,一气流行阴阳变化之道。
一者,道之纪也。

深入道源去原道。观天之道,执天之行,尽矣!以天地之道去洞察万物,则易知易行。

道是人生算法,也是第二曲线的哲学;是第一性原理,也是10X增长。

\subsection{行}

\subsection{证}

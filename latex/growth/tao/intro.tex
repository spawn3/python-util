\chapter{导言}

\section{导言}

研判形势,淬炼心法,有所为,有所不为,乃至无为而无不为。

修道而保法,故能为胜败之政。

\begin{shadequote}

    道生一,一生二,二生三,三生万物。\\
    道生之,德蓄之,物形之,势成之。
\end{shadequote}

精一之学,体用兼备。
\begin{shadequote}

    天地之道,可一言而尽也:其为物不二,则其生物不测。\\
    天下之动,贞夫一者也。\\
    圣人抱一以为天下式。\\
    恒以一德。
\end{shadequote}

太极哲学,双线法则,圆点哲学,一分为三,提供了诸多值得反复体味的命题。

一,切己言之,就是事业,须更上一层楼。一是整体,是根据地,是不间断,也是突破点。

博厚,高明,悠久。

空灵之境,有无相生,有生于无。空非空寂,众缘所起,云行雨施,品物流行。

太极本无极。

上溯,万法归一,一归空。

五轮书,地水火风空。

建立自我,追求无我,是逆向工程。下学而上达。

\section{战略,或道}

\section{方法谈}

爱因斯坦说过这句话:我们不能用制造问题时同一水平的思维来解决问题。也许他意味着我们需要摆脱与我们对一个问题有关的消极的看法。如果我们对问题本身太投入,那么我们永远无法越过这个局面。

在一本叫“治愈与复原”中,David R. Hawkins详细阐述了这一点。他说,“问题最好不要在他们发生的同一水平上解决,而是在他们的上一个阶级上解决...通过超越他们,从更高的角度看待问题,问题很容易迎刃而解。
较高层次上,由于这种观点的转变,问题会自动解决,否则人们可能会看不到任何的问题。”

很多时候,我们面对一个问题时,总会把精力集中在问题上,一直问怎么“解决”呢?我们可能最终会走入死角,沮丧。
因为我们似乎找不到很好的解决办法。无论如何,不要把精力集中在问题本身上。花几分钟时间,花费你的时间和精力来正面地解析。
我们无法控制经常会有事情出现的,不要浪费时间担心这些事情;只花时间在你可以改变或控制的事情上。

\subsection{中庸}

\subsection{圆点哲学}
\subsection{双线法则}
\subsection{黄金分割率}
\subsection{80/20规则}
\subsection{黄金圈法则}

\subsection{达里奥的原则}

欲达到我们的目标,必须实事求是,客观公正地面对现实,正视自身的缺点和不足,而有以克服之。

这是真的吗?求真是第一位的,吾爱吾师,吾更爱真理。对道听途说的观念,我们固然要保持警觉和必要的批判精神。
对自我意识,也要慎思明辨。保持开放之心和专注之念,对自己的观念做压力测试,力求准确更准确。
而不能陷入先入之见,或自欺欺人,没有荣辱,只有是非。不当的虚荣心和自尊心会妨碍通向真正的目标。

对我们不知之物,保持谦卑,保持饥饿,保持愚蠢。

选择至关重要,我们必须承担选择的后果,为选择负起责任。
弱点,由弱点导致错误,皆在所难免。
但由此错误,吃一堑长一智,如果能通过反思而增强了自己,就是有益的。
从错误中学习,进步,进化,是通向成功的捷径。

任一选择,都带来其效应和影响。一阶效应也许不错,但二三阶效应可能已变形,
祸福相依,需要更多的洞见。关键的选择,决定了我们人生的质量。

成长,或曰进化,是唯一的目的。财富,名利皆是果,而不是因。
当我们围绕成长,而动心忍性,增益其所不能的时候,就是走在自我进化的路上。

自我进化,有一五步法可资遵循:
\begin{enumbox}
\item 设定清晰的目标
\item 觉知问题
\item 诊断问题
\item 设计方案
\item 执行方案
\end{enumbox}

五步法是迭代过程。每一步都需要投入必要的资源,做选择,做策划。

对比目标和输出的不同,找到不足,做出适当的调整,类似于PDCD。不妨想象,有台巨大的机器,作为输入输出的中介。
我们的核心任务,就是维持机器的良好运行和高效产出。

资源调度,采取开放的视角,并非一定需要我们亲力亲为。
我,即是设计者,也是执行者,主要作为设计者而存在。
任何人都非全知全能,而是有长有短,管理者的职责,在于知人善任。

不必为自己的弱点而沮丧,君子性非异也,善假于物。

唯一的目的,就是自我进化。唯一的事,就是打造机器。
机器是我们拥有的容器,即是心法,也是产品。我们是机器的架构师。

单纯观念,不足以动人。做出作品,持续产出,才能实现自我价值,立于不败之地。

实有诸己之谓德。默默地完成进化,是最明智的选择。围绕选定的一,厚积薄发,静水深流。

道生一,一即是战略,也是方法。

原则,架构起了价值和行动的桥梁。让我们有所遵循,持续积累,而不是茫然无措,本末倒置。

佛陀的教导

以戒为师。戒可释为原则,或良好习惯。

大乘起信论的一心二门的义理架构,予人深刻启示。

达里奥与王阳明

良知是比原则更基础的范畴,良知是一种元认知能力。
致吾心良知于事事物物,则事事物物皆得其理。

达里奥求真的意志和可操作性,较阳明为突出。
资本主义的熏陶,更适应于现实人生。

毛主席在其著作中,深入分析了认识的各种问题,如主观主义,教条主义,经验主义,
统称为主观主义,即主客观的分裂和不一致。以此指导行动,则误导行动。

马利克的管理学,采用系统论,控制论和仿生学等知识,以应对现实世界的复杂性。


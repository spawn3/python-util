\section{庄子}

\subsection{天道}

天道运而无所积,故万物成;帝道运而无所积,故天下归;圣道运 而无所积,故海内服。明于天,通于圣,六通四辟于帝王之德者,其自为也,昧然无不静者矣!
圣人之静也,非曰静也善,故静也。万物 无足以挠心者,故静也。水静则明烛须眉,平中准,大匠取法焉。水静犹明,而况精神!
圣人之心静乎!天地之鉴也,万物之镜也。夫虚 静恬淡寂漠无为者,天地之平而道德之至也。故帝王圣人休焉。休则 虚,虚则实,实则伦矣。虚则静,静则动,动则得矣。
静则无为,无 为也,则任事者责矣。无为则俞俞。俞俞者,忧患不能处,年寿长矣 。
\hl{夫虚静恬淡寂漠无为者,万物之本也}。明此以南乡,尧之为君也; 明此以北面,舜之为臣也。以此处上,帝王天子之德也;以此处下, 玄圣素王之道也。
以此退居而闲游,江海山林之士服;以此进为而抚 世,则功大名显而天下一也。
静而圣,动而王,无为也而尊,朴素而 天下莫能与之争美。夫明白于天地之德者,此之谓大本大宗,与天和 者也。所以均调天下,与人和者也。与人和者,谓之人乐;与天和者 ,谓之天乐。
庄子曰:“吾师乎,吾师乎!赍万物而不为戾;泽及万 世而不为仁;长于上古而不为寿;覆载天地、刻雕众形而不为巧。” 此之谓天乐。
故曰:知天乐者,其生也天行,其死也物化。静而与阴 同德,动而与阳同波。故知天乐者,无天怨,无人非,无物累,无鬼 责。故曰:其动也天,其静也地,一心定而王天下;其鬼不祟,其魂 不疲,一心定而万物服。言以虚静推于天地,通于万物,此之谓天乐 。天乐者,圣人之心以畜天下也。

夫帝王之德,以天地为宗,以道德为主,以无为为常。无为也,则 用天下而有余;有为也,则为天下用而不足。故古之人贵夫无为也。 上无为也,下亦无为也,是下与上同德。下与上同德则不臣。下有为 也,上亦有为也,是上与下同道。上与下同道则不主。上必无为而用 下,下必有为为天下用。此不易之道也。

故古之王天下者,知虽落天地,不自虑也;辩虽雕万物,不自说也 ;能虽穷海内,不自为也。天不产而万物化,地不长而万物育,帝王 无为而天下功。故曰:莫神于天,莫富于地,莫大于帝王。故曰:帝 王之德配天地。此乘天地,驰万物,而用人群之道也。

本在于上,末在于下;要在于主,详在于臣。三军五兵之运,德之 末也;赏罚利害,五刑之辟,教之末也;礼法度数,刑名比详,治之 末也;钟鼓之音,羽旄之容,乐之末也;哭泣衰囗(左“纟”右“至 ”),隆杀之服,哀之末也。此五末者,须精神之运,心术之动,然 后从之者也。末学者,古人有之,而非所以先也。君先而臣从,父先 而子从,兄先而弟从,长先而少从,男先而女从,夫先而妇从。夫尊 卑先后,天地之行也,故圣人取象焉。天尊地卑,神明之位也;春夏 先,秋冬后,四时之序也;万物化作,萌区有状,盛衰之杀,变化之 流也。夫天地至神矣,而有尊卑先后之序,而况人道乎!宗庙尚亲, 朝廷尚尊,乡党尚齿,行事尚贤,大道之序也。语道而非其序者,非 其道也。语道而非其道者,安取道哉!

是故古之明大道者,先明天而道德次之,道德已明而仁义次之,仁 义已明而分守次之,分守已明而形名次之,形名已明而因任次之,因 任已明而原省次之,原省已明而是非次之,是非已明而赏罚次之,赏 罚已明而愚知处宜,贵贱履位,仁贤不肖袭情。必分其能,必由其名 。以此事上,以此畜下,以此治物,以此修身,知谋不用,必归其天 。此之谓大平,治之至也。故书曰:“有形有名。”形名者,古人有 之,而非所以先也。古之语大道者,五变而形名可举,九变而赏罚可 言也。骤而语形名,不知其本也;骤而语赏罚,不知其始也。倒道而 言,迕道而说者,人之所治也,安能治人!骤而语形名赏罚,此有知 治之具,非知治之道。可用于天下,不足以用天下。此之谓辩士,一 曲之人也。礼法数度,形名比详,古人有之。此下之所以事上,非上 之所以畜下也。

昔者舜问于尧曰:“天王之用心何如?”尧曰:“吾不敖无告,不 废穷民,苦死者,嘉孺子而哀妇人,此吾所以用心已。”舜曰:“美 则美矣,而未大也。”尧曰:“然则何如?”舜曰:“天德而出宁, 日月照而四时行,若昼夜之有经,云行而雨施矣!”尧曰:“胶胶扰 扰乎!子,天之合也;我,人之合也。”夫天地者,古之所大也,而 黄帝、尧、舜之所共美也。故古之王天下者,奚为哉?天地而已矣!

孔子西藏书于周室,子路谋曰:“由闻周之征藏史有老聃者,免而 归居,夫子欲藏书,则试往因焉。”孔子曰:“善。”往见老聃,而 老聃不许,于是囗(左“纟”右“番”音fan2)十二经以说。老 聃中其说,曰:“大谩,愿闻其要。”孔子曰:“要在仁义。”老聃 曰:“请问:仁义,人之性邪?”孔子曰:“然,君子不仁则不成, 不义则不生。仁义,真人之性也,又将奚为矣?”老聃曰:“请问: 何谓仁义?”孔子曰:“中心物恺,兼爱无私,此仁义之情也。”老 聃曰:“意,几乎后言!夫兼爱,不亦迂夫!无私焉,乃私也。夫子 若欲使天下无失其牧乎?则天地固有常矣,日月固有明矣,星辰固有 列矣,禽兽固有群矣,树木固有立矣。夫子亦放德而行,遁遁而趋, 已至矣!又何偈偈乎揭仁义,若击鼓而求亡子焉!意,夫子乱人之性 也。”

士成绮见老子而问曰:“吾闻夫子圣人也。吾固不辞远道而来愿见 ,百舍重趼而不敢息。今吾观子非圣人也,鼠壤有余蔬而弃妹,不仁 也!生熟不尽于前,而积敛无崖。”老子漠然不应。士成绮明日复见 ,曰:“昔者吾有剌于子,今吾心正囗(左“谷”右“阝”)矣,何 故也?”老子曰:“夫巧知神圣之人,吾自以为脱焉。昔者子呼我牛 也而谓之牛;呼我马也而谓之马。苟有其实,人与之名而弗受,再受 其殃。吾服也恒服,吾非以服有服。”士成绮雁行避影,履行遂进, 而问修身若何。老子曰:“而容崖然,而目冲然,而颡囗(左上“月 ”左下“廾”右“页”)然,而口阚然,而状义然。似系马而止也, 动而持,发也机,察而审,知巧而睹于泰,凡以为不信。边竟有人焉 ,其名为窃。”

老子曰:“夫道,于大不终,于小不遗,故万物备。广广乎其无不 容也,渊渊乎其不可测也。形德仁义,神之末也,非至人孰能定之! 夫至人有世,不亦大乎,而不足以为之累;天下奋柄而不与之偕;审 乎无假而不与利迁;极物之真,能守其本。故外天地,遗万物,而神 未尝有所困也。通乎道,合乎德,退仁义,宾礼乐,至人之心有所定 矣!”

世之所贵道者,书也。书不过语,语有贵也。语之所贵者,意也, 意有所随。意之所随者,不可以言传也,而世因贵言传书。世虽贵之 ,我犹不足贵也,为其贵非其贵也。故视而可见者,形与色也;听而 可闻者,名与声也。悲夫!世人以形色名声为足以得彼之情。夫形色 名声,果不足以得彼之情,则知者不言,言者不知,而世岂识之哉!

桓公读书于堂上,轮扁斫轮于堂下,释椎凿而上,问桓公曰:“敢 问:“公之所读者,何言邪?”公曰:“圣人之言也。”曰:“圣人 在乎?”公曰:“已死矣。”曰:“然则君之所读者,古人之糟粕已 夫!”桓公曰:“寡人读书,轮人安得议乎!有说则可,无说则死! ”轮扁曰:“臣也以臣之事观之。斫轮,徐则甘而不固,疾则苦而不 入,不徐不疾,得之于手而应于心,口不能言,有数存乎其间。臣不 能以喻臣之子,臣之子亦不能受之于臣,是以行年七十而老斫轮。古 之人与其不可传也死矣,然则君之所读者,古人之糟粕已夫!”

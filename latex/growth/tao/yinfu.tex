\chapter{精选}

诸子:
\begin{enumbox}
\item 易经
\item 中庸
\item 道德经
\item 文子
\item 黄帝四经
\item 黄帝阴符经
\item 鬼谷子
\item 管子
\item 素书
\item 长短经
\item 太极图说
\item 通书
\item 武经七书
\item 武艺二书
\end{enumbox}

佛经
\begin{enumbox}
\item 心经
\item 金刚经
\item 坛经
\item 大乘起信论
\item 圆觉经
\end{enumbox}

近人之著作
\begin{enumbox}
\item 一二三哲学
\item 东方战略学
\item 李小龙
\item 查理芒格
\item 孙正义
\item 人生算法
\item 第一性原理
\item 第二曲线
\item 基业长青
\item 从优秀到卓越
\item 黑天鹅
\item 反脆弱
\end{enumbox}

\section{阴符经}

\subsection{原文}

\hl{观天之道,执天之行,尽矣}。
故天有五贼,见之者昌。
五贼在乎心,施行于天。宇宙在乎手,万化生乎身。
天性,人也;人心,机也。立天之道,以定人也。
天发杀机,移星易宿;地发杀机,龙蛇起陆;人发杀机,天地反覆;天人合发,万变定基。
性有巧拙,可以伏藏。九窍之邪,在乎三要,可以动静。
火生于木,祸发必克;奸生于国,时动必溃。知之修炼,谓之圣人。

天生天杀,道之理也。
天地,万物之盗;万物,人之盗;人,万物之盗。
三盗既宜,三才既安。故曰:食其时,百骸理;动其机,万化安。
人知其神而神,不知其不神之所以神也。
日月有数,大小有定,圣功生焉,神明出焉。
其盗机也,天下莫能见,莫能知也。君子得之固躬,小人得之轻命。 

瞽者善听,聋者善视。\hl{绝利一源,用师十倍。三返昼夜,用师万倍}。
心生于物,死于物,机在于目。
天之无恩而大恩生。迅雷烈风,莫不蠢然。
至乐性余,至静性廉。天之至私,用之至公。禽之制在炁。
生者死之根,死者生之根。恩生于害,害生于恩。
愚人以天地文理圣,我以时物文理哲。人以愚虞圣,我以不愚虞圣;人以奇期圣,我以不奇期圣。
故曰:沉水入火,自取灭亡。

自然之道静,故天地万物生。
天地之道浸,故阴阳胜。
阴阳相推,而变化顺矣。是故圣人知自然之道不可违,因而制之。
至静之道。律历所不能契。
爰有奇器,是生万象,八卦甲子,神机鬼藏。
阴阳相胜之术,昭昭乎进于象矣。 

\subsection{释义}

以道心物三合之道来诠释,物者,意之所在。

观天之道,执天之行,尽矣:由心出发,体察天地之道,而后可以循道而行,此为道知,道尽为学处世之道。
观天之道是升维思考,执天之行是降维贯通,两相结合,就完备了。

天非茫茫之天,内蕴五行,能体察五行之气运,则可以昌盛。

心为能动的一方,以心受道体道,就可以立其环中,以应无穷,包括领导统御之术。

天人合发,万变定基:心与道合、与天合,这是做一切事的根基。

明了五行生克的结构与动态关系,进而内化于心,正心诚意,可称为圣人。

一事或成或败,皆有道理蕴含其中。天地-人-万物三合之道,尽心知性则知天矣,格物致知穷理,
尊德性而道问学,此三者相生相克,转圆而求其合。藏器于身待时而动,则万事如意,臻于中道。

绝利一源,一者何?道也,进乎技也。三返昼夜,循环至三,如昼夜交替,运行不废。
其功效甚大,有事半功倍之效果。一不能理解为具体的事,如此则器,君子不器,本立道生。
若心能体道,秉道御物,乘物游心,则三合之道可以大成。以道控势,顺势而为,与道浮沉。

执大象,天下往。往而不害,安平泰。大象无形,此无形之大象,即是道。
一生二,太极生两仪,有上下层次之别。两仪一阴一阳,有左右对称之美。

惟精惟一,志于道,若能志于道,而不废事,可入事事无碍法界。

大道至简,玄之又玄众妙之门。

真人者,同天而合道,执一而养产万类,怀天心,施德养,无为以包志虑思意而行威势者也。
鬼谷阴符七术之教。

阳明心学,心外无理心外无事,此心与道为一,即是道心、天心。

口目耳,此身之三要,心能制之。微信控,游戏控,则失心之所以为主,惑矣。

气韵生动

\section{太极拳谱}

太极者,无极而生,动静之机,阴阳之母也。
动之则分,静之则合。无过不及,随曲就伸。
人刚我柔谓之走,我顺人背谓之粘。
动急则急应,动缓则缓随。
虽变化万端,而理唯一贯。
由招熟而渐悟懂劲,由懂劲而阶及神明。
然非用力之久,不能豁然贯通焉。

虚领顶劲,气沉丹田。不偏不倚,忽隐忽现。
左重则左虚,右重则右杳。
仰之则弥高,俯之则弥深,进之则愈长,退之则愈促。
一羽不能加,蝇虫不能落,人不知我,我独知人。
英雄所向无敌,盖皆由此而及也。

斯技旁门甚多,虽势有区别,概不外乎壮欺弱,慢让快耳。
有力打无力,手慢让手快,皆是先天自然之能,非关学力而有为也。
察四两拨千斤之句,显非力胜;观耄耋能御众之形,快何能为。
立如平/秤准,活似车轮。偏沉则随,双重则滞。
每见数年纯功,不能运化者,率皆自为人制,双重之病未悟耳。

欲避此病,须知阴阳。粘即是走,走即是粘。
阴不离阳,阳不离阴。阴阳相济,方为懂劲。

懂劲后,愈练愈精,默识揣摩,渐至从心所欲。
本是舍己从人,多误舍近求远。
所谓差之毫厘,谬之千里,学者不可不详辨焉。

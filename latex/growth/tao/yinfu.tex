\chapter{精选}

诸子:
\begin{enumbox}
\item 易经
\item 中庸
\item 道德经
\item 文子
\item 黄帝四经
\item 黄帝阴符经
\item 鬼谷子
\item 管子
\item 素书
\item 长短经
\item 太极图说
\item 通书
\item 武经七书
\item 武艺二书
\end{enumbox}

佛经
\begin{enumbox}
\item 心经
\item 金刚经
\item 坛经
\item 大乘起信论
\item 圆觉经
\end{enumbox}

近人之著作
\begin{enumbox}
\item 一二三哲学
\item 东方战略学
\item 李小龙
\item 查理芒格
\item 孙正义
\item 人生算法
\item 第一性原理
\item 第二曲线
\item 基业长青
\item 从优秀到卓越
\item 黑天鹅
\item 反脆弱
\end{enumbox}

\section{太极图说}

无极而太极。太极动而生阳,动极而静,静而生阴,静极复动。
一动一静,互为其根。分阴分阳,两仪立焉。
阳变阴合,而生水火木金土。五气顺布,四时行焉。

五行一阴阳也,阴阳一太极也,太极本无极也。
五行之生也,各一其性。\hl{无极之真,二五之精,妙合而凝}。
乾道成男,坤道成女。二气交感,化生万物。万物生生,而变化无穷焉。

惟人也得其秀而最灵。形既生矣,神发知矣。五性感动,而善恶分,万事出矣。
圣人定之以中正仁义而\hl{主静},立人极焉。

故圣人与天地合其德,日月合其明,四时合其序,鬼神合其吉凶。君子修之,吉;小人悖之,凶。

故曰:“立天之道,曰阴与阳。立地之道,曰柔与刚。立人之道,曰仁与义”。

又曰:“原始反终,故知死生之说”。

大哉易也,斯之至矣。

\section{太极拳谱}

\hl{太极者,无极而生,动静之机,阴阳之母也。
动之则分,静之则合}。无过不及,随曲就伸。
人刚我柔谓之走,我顺人背谓之粘。
动急则急应,动缓则缓随。
虽变化万端,而理唯一贯。
由招熟而渐悟懂劲,由懂劲而阶及神明。
然非用力之久,不能豁然贯通焉。

虚领顶劲,气沉丹田。不偏不倚,忽隐忽现。
左重则左虚,右重则右杳。
仰之则弥高,俯之则弥深,进之则愈长,退之则愈促。
一羽不能加,蝇虫不能落,人不知我,我独知人。
英雄所向无敌,盖皆由此而及也。

斯技旁门甚多,虽势有区别,概不外乎壮欺弱,慢让快耳。
有力打无力,手慢让手快,皆是先天自然之能,非关学力而有为也。
察四两拨千斤之句,显非力胜;观耄耋能御众之形,快何能为。
立如平/秤准,活似车轮。偏沉则随,双重则滞。
每见数年纯功,不能运化者,率皆自为人制,双重之病未悟耳。

欲避此病,须知阴阳。粘即是走,走即是粘。
阴不离阳,阳不离阴。阴阳相济,方为懂劲。

懂劲后,愈练愈精,默识揣摩,渐至从心所欲。
本是舍己从人,多误舍近求远。
所谓差之毫厘,谬之千里,学者不可不详辨焉。

\section{阴符经}

\subsection{原文}

\hl{观天之道,执天之行,尽矣}。
故天有五贼,见之者昌。
五贼在乎心,施行于天。宇宙在乎手,万化生乎身。
天性,人也;人心,机也。立天之道,以定人也。
天发杀机,移星易宿;地发杀机,龙蛇起陆;人发杀机,天地反覆;天人合发,万变定基。
性有巧拙,可以伏藏。九窍之邪,在乎三要,可以动静。
火生于木,祸发必克;奸生于国,时动必溃。知之修炼,谓之圣人。

天生天杀,道之理也。
天地,万物之盗;万物,人之盗;人,万物之盗。
三盗既宜,三才既安。故曰:食其时,百骸理;动其机,万化安。
人知其神而神,不知其不神之所以神也。
日月有数,大小有定,圣功生焉,神明出焉。
其盗机也,天下莫能见,莫能知也。君子得之固躬,小人得之轻命。 

瞽者善听,聋者善视。\hl{绝利一源,用师十倍。三返昼夜,用师万倍}。
心生于物,死于物,机在于目。
天之无恩而大恩生。迅雷烈风,莫不蠢然。
至乐性余,至静性廉。天之至私,用之至公。禽之制在炁。
生者死之根,死者生之根。恩生于害,害生于恩。
愚人以天地文理圣,我以时物文理哲。人以愚虞圣,我以不愚虞圣;人以奇期圣,我以不奇期圣。
故曰:沉水入火,自取灭亡。

\hl{自然之道静,故天地万物生}。
天地之道浸,故阴阳胜。
阴阳相推,而变化顺矣。是故圣人知自然之道不可违,因而制之。
至静之道。律历所不能契。
爰有奇器,是生万象,八卦甲子,神机鬼藏。
阴阳相胜之术,昭昭乎进于象矣。 

\subsection{释义}

以道心物三合之道来诠释,物者,意之所在。

观天之道,执天之行,尽矣:由心出发,体察天地之道,而后可以循道而行,此为道知,道尽为学处世之道。
观天之道是升维思考,执天之行是降维贯通,两相结合,就完备了。

天非茫茫之天,内蕴五行,能体察五行之气运,则可以昌盛。

心为能动的一方,以心受道体道,就可以立其环中,以应无穷,包括领导统御之术。

天人合发,万变定基:心与道合、与天合,这是做一切事的根基。

明了五行生克的结构与动态关系,进而内化于心,正心诚意,可称为圣人。

一事或成或败,皆有道理蕴含其中。天地-人-万物三合之道,尽心知性则知天矣,格物致知穷理,
尊德性而道问学,此三者相生相克,转圆而求其合。藏器于身待时而动,则万事如意,臻于中道。

绝利一源,一者何?道也,进乎技也。三返昼夜,循环至三,如昼夜交替,运行不废。
其功效甚大,有事半功倍之效果。一不能理解为具体的事,如此则器,君子不器,本立道生。
若心能体道,秉道御物,乘物游心,则三合之道可以大成。以道控势,顺势而为,与道浮沉。

执大象,天下往。往而不害,安平泰。大象无形,此无形之大象,即是道。
一生二,太极生两仪,有上下层次之别。两仪一阴一阳,有左右对称之美。

惟精惟一,志于道,若能志于道,而不废事,可入事事无碍法界。

大道至简,玄之又玄众妙之门。

真人者,同天而合道,执一而养产万类,怀天心,施德养,无为以包志虑思意而行威势者也。
鬼谷阴符七术之教。

阳明心学,心外无理心外无事,此心与道为一,即是道心、天心。

口目耳,此身之三要,心能制之。微信控,游戏控,则失心之所以为主,惑矣。

气韵生动

在二元对立的世界里,诗意地安居?一故神,两故化。

\section{洪范}

武王胜殷,杀受,立武庚,以箕子归。作《洪范》。

惟十有三祀,王访于箕子。王乃言曰:「呜呼!箕子。惟天阴骘下民,相协厥居,我不知其彝伦攸叙。」

箕子乃言曰:「我闻在昔,鲧堙洪水,汩陈其五行。帝乃震怒,不畀『洪范』九畴,彝伦攸斁。鲧则殛死,禹乃嗣兴,天乃锡禹『洪范』九畴,彝伦攸叙。

初一曰五行,次二曰敬用五事,次三曰农用八政,次四曰协用五纪,次五曰建用皇极,次六曰乂用三德,次七曰明用稽疑,次八曰念用庶征,次九曰向用五福,威用六极。

一、五行:一曰水,二曰火,三曰木,四曰金,五曰土。水曰润下,火曰炎上,木曰曲直,金曰从革,土爰稼穑。润下作咸,炎上作苦,曲直作酸,从革作辛,稼穑作甘。

二、五事:一曰貌,二曰言,三曰视,四曰听,五曰思。貌曰恭,言曰从,视曰明,听曰聪,思曰睿。恭作肃,从作乂,明作哲,聪作谋,睿作圣。

三、八政:一曰食,二曰货,三曰祀,四曰司空,五曰司徒,六曰司寇,七日宾,八曰师。

四、五纪:一曰岁,二曰月,三曰日,四曰星辰,五曰历数。

五、皇极:皇建其有极。敛时五福,用敷锡厥庶民。惟时厥庶民于汝极。锡汝保极:凡厥庶民,无有淫朋,人无有比德,惟皇作极。凡厥庶民,有猷有为有守,汝则念之。不协于极,不罹于咎,皇则受之。而康而色,曰:『予攸好德。』汝则锡之福。时人斯其惟皇之极。无虐茕独而畏高明,人之有能有为,使羞其行,而邦其昌。凡厥正人,既富方谷,汝弗能使有好于而家,时人斯其辜。于其无好德,汝虽锡之福,其作汝用咎。无偏无陂,遵王之义;无有作好,遵王之道;无有作恶,尊王之路。无偏无党,王道荡荡;无党无偏,王道平平;无反无侧,王道正直。会其有极,归其有极。曰:皇,极之敷言,是彝是训,于帝其训,凡厥庶民,极之敷言,是训是行,以近天子之光。曰:天子作民父母,以为天下王。

六、三德:一曰正直,二曰刚克,三曰柔克。平康,正直;强弗友,刚克;燮友,柔克。沈潜,刚克;高明,柔克。惟辟作福,惟辟作威,惟辟玉食。臣无有作福、作威、玉食。臣之有作福、作威、玉食,其害于而家,凶于而国。人用侧颇僻,民用僭忒。

七、稽疑:择建立卜筮人,乃命十筮。曰雨,曰霁,曰蒙,曰驿,曰克,曰贞,曰悔,凡七。卜五,占用二,衍忒。立时人作卜筮,三人占,则从二人之言。汝则有大疑,谋及乃心,谋及卿士,谋及庶人,谋及卜筮。汝则从,龟从,筮从,卿士从,庶民从,是之谓大同。身其康强,子孙其逢,汝则从,龟从,筮从,卿士逆,庶民逆吉。卿士从,龟从,筮从,汝则逆,庶民逆,吉。庶民从,龟从,筮从,汝则逆,卿士逆,吉。汝则从,龟从,筮逆,卿士逆,庶民逆,作内吉,作外凶。龟筮共违于人,用静吉,用作凶。

八、庶征:曰雨,曰暘,曰燠,曰寒,曰风。曰时五者来备,各以其叙,庶草蕃庑。一极备,凶;一极无,凶。曰休征;曰肃、时雨若;曰乂,时暘若;曰晰,时燠若;曰谋,时寒若;曰圣,时风若。曰咎征:曰狂,恒雨若;曰僭,恒暘若;曰豫,恒燠若;曰急,恒寒若;曰蒙,恒风若。曰王省惟岁,卿士惟月,师尹惟日。岁月日时无易,百谷用成,乂用民,俊民用章,家用平康。日月岁时既易,百谷用不成,乂用昏不明,俊民用微,家用不宁。庶民惟星,星有好风,星有好雨。日月之行,则有冬有夏。月之从星,则以风雨。

九、五福:一曰寿,二曰富,三曰康宁,四曰攸好德,五曰考终命。六极:一曰凶、短、折,二曰疾,三曰忧,四曰贫,五曰恶,六曰弱。

\section{大学}

大学之道,在明明德,在亲民,在止于至善。

知止而後有定,定而後能静,静而後能安,安而後能虑,虑而後能得。物有本末,事有终始。知所先後,则近道矣。

古之欲明明德于天下者,先治其国。欲治其国者,先齐其家,欲齐其家者,先修其身。欲修其身者,先正其心。欲正其心者,先诚其意。欲诚其意者,先致其知。致知在格物。

物格而後知至,知至而後意诚,意诚而後心正,心正而後身修,身修而後家齐,家齐而後国治,国治而後天下平。自天子以至于庶人,一是皆以修身为本。

其本乱而末治者否矣。其所厚者薄,而其所薄者厚,未之有也。此谓知本,此谓知之至也。

所谓诚其意者,毋自欺也。如恶恶臭,如好好色,此之谓自谦。故君子必慎其独也。小人闲居为不善,无所不至,见君子而後厌然,掩其不善,而著其善。人之视己,如见其肺肝然,则何益矣。此谓诚于中,形于外,故君子必慎其独也。曾子曰:“十目所视,十手所指,其严乎!”富润屋,德润身,心广体胖,故君子必诚其意。

《诗》云:“瞻彼淇澳,菉竹猗猗。有斐君子,如切如磋,如琢如磨。瑟兮僴兮,赫兮喧兮。有斐君子,终不可諠兮!”“如切如磋”者,道学也。“如琢如磨”者,自修也。“瑟兮僴兮”者,恂慄也。“赫兮喧兮”者,威仪也。“有斐君子,终不可諠兮”者,道盛德至善,民之不能忘也。

《诗》云:“於戏,前王不忘!”君子贤其贤而亲其亲,小人乐其乐而利其利,此以没世不忘也。

《康诰》曰:“克明德。”《大甲》曰:“顾諟天之明命。”《帝典》曰:“克明峻德。”皆自明也。

汤之《盘铭》曰:“苟日新,日日新,又日新。”《康诰》曰:“作新民。”《诗》曰:“周虽旧邦,其命维新。”是故君子无所不用其极。

《诗》云:“邦畿千里,维民所止。”《诗》云:“缗蛮黄鸟,止于丘隅。”子曰:“于止,知其所止,可以人而不如鸟乎?”《诗》云:“穆穆文王,于缉熙敬止!”为人君,止于仁;为人臣,止于敬;为人子,止于孝;为人父,止于慈;与国人交,止于信。

子曰:“听讼,吾犹人也。必也使无讼乎!”无情者不得尽其辞。大畏民志,此谓知本”。

所谓修身在正其心者,身有所忿懥,则不得其正,有所恐惧,则不得其正,有所好乐,则不得其正,有所忧患,则不得其正。心不在焉,视而不见,听而不闻,食而不知其味。此谓修身在正其心。

所谓齐其家在修其身者,人之其所亲爱而辟焉,之其所贱恶而辟焉,之其所畏敬而辟焉,之其所哀矜而辟焉,之其所敖惰而辟焉。故好而知其恶,恶而知其美者,天下鲜矣。故谚有之曰:“人莫知其子之恶,莫知其苗之硕。”此谓身不修不可以齐其家。

所谓治国必先齐其家者,其家不可教而能教人者,无之。故君子不出家而成教于国。孝者,所以事君也;弟者,所以事长也;慈者,所以使众也。《康诰》曰:“如保赤子。”心诚求之,虽不中不远矣。未有学养子而後嫁者也。一家仁,一国兴仁;一家让,一国兴让;一人贪戾,一国作乱:其机如此。此谓一言偾事,一人定国。尧、舜率天下以仁,而民从之。桀、纣率天下以暴,而民从之。其所令反其所好,而民不从。是故君子有诸己而後求诸人,无诸己而後非诸人。所藏乎身不恕,而能喻诸人者,未之有也。故治国在齐其家。《诗》云:“桃之夭夭,其叶蓁蓁。之子于归,宜其家人。”宜其家人,而後可以教国人。《诗》云:“宜兄宜弟。”宜兄宜弟,而後可以教国人。《诗》云:“其仪不忒,正是四国。”其为父子兄弟足法,而後民法之也。此谓治国在齐其家。

所谓平天下在治其国者,上老老而民兴孝,上长长而民兴弟,上恤孤而民不倍,是以君子有絜矩之道也。

所恶于上,毋以使下,所恶于下,毋以事上;所恶于前,毋以先後;所恶于後,毋以从前;所恶于右,毋以交于左;所恶于左,毋以交于右;此之谓絜矩之道。

《诗》云:“乐只君子,民之父母。”民之所好好之,民之所恶恶之,此之谓民之父母。《诗》云:“节彼南山,维石岩岩。赫赫师尹,民具尔瞻。”有国者不可以不慎,辟则为天下僇矣。《诗》云:“殷之未丧师,克配上帝。仪监于殷,峻命不易。”道得众则得国,失众则失国。

是故君子先慎乎德。有德此有人,有人此有土,有土此有财,有财此有用。德者本也,财者末也。外本内末,争民施夺。是故财聚则民散,财散则民聚。是故言悖而出者,亦悖而入;货悖而入者,亦悖而出。

《康诰》曰:“惟命不于常。”道善则得之,不善则失之矣。

《楚书》曰:“楚国无以为宝,惟善以为宝。”舅犯曰:“亡人无以为宝,仁亲以为宝。”

《秦誓》曰:“若有一个臣,断断兮无他技,其心休休焉,其如有容焉。人之有技,若己有之;人之彦圣,其心好之,不啻若自其口出。实能容之,以能保我子孙黎民,尚亦有利哉!人之有技,冒嫉以恶之;人之彦圣,而违之,俾不通:实不能容,以不能保我子孙黎民,亦曰殆哉!”唯仁人放流之,迸诸四夷,不与同中国。此谓唯仁人为能爱人,能恶人。见贤而不能举,举而不能先,命也;见不善而不能退,退而不能远,过也。好人之所恶,恶人之所好,是谓拂人之性,菑必逮夫身。是故君子有大道,必忠信以得之,骄泰以失之。

生财有大道,生之者众,食之者寡,为之者疾,用之者舒,则财恒足矣。仁者以财发身,不仁者以身发财。未有上好仁而下不好义者也,未有好义其事不终者也,未有府库财非其财者也。孟献子曰:“畜马乘不察于鸡豚,伐冰之家不畜牛羊,百乘之家不畜聚敛之臣。与其有聚敛之臣,宁有盗臣。”此谓国不以利为利,以义为利也。长国家而务财用者,必自小人矣。彼为善之,小人之使为国家,菑害并至。虽有善者,亦无如之何矣!此谓国不以利为利,以义为利也。

\section{中庸}

\hl{天命之谓性,率性之谓道,修道之谓教}。道也者,不可须臾离也;可离非道也。是故君子戒慎乎其所不睹,恐惧乎其所不闻。莫见乎隐,莫显乎微,故君子慎其独也。喜怒哀乐之未发,谓之中;发而皆中节,谓之和。中也者,天下之大本也;和也者,天下之达道也。致中和,天地位焉,万物育焉。

仲尼曰:“君子中庸,小人反中庸。君子之中庸也,君子而时中;小人之中庸也,小人而无忌惮也。”子曰:“中庸其至矣乎!民鲜能久矣!”

子曰:“道之不行也,我知之矣:知者过之,愚者不及也。道之不明也,我知之矣:贤者过之,不肖者不及也。人莫不饮食也,鲜能知味也。”子曰:“道其不行矣夫!”

子曰:“舜其大知也与!舜好问而好察迩言,隐恶而扬善,执其两端,用其中于民,其斯以为舜乎!”

子曰:“人皆曰予知,驱而纳诸罟擭陷阱之中,而莫之知辟也。人皆曰予知,择乎中庸,而不能期月守也。”

子曰:“回之为人也,择乎中庸,得一善,则拳拳服膺弗失之矣。”

子曰:“天下国家可均也,爵禄可辞也,白刃可蹈也,中庸不可能也。”

子路问强。子曰:“南方之强与?北方之强与?抑而强与?宽柔以教,不报无道,南方之强也,君子居之。衽金革,死而不厌,北方之强也,而强者居之。故君子和而不流,强哉矫!中立而不倚,强哉矫!国有道,不变塞焉,强哉矫!国无道,至死不变,强哉矫!”

子曰:“素隐行怪,後世有述焉,吾弗为之矣。君子遵道而行,半途而废,吾弗能已矣。君子依乎中庸,遁世不见知而不悔,唯圣者能之。”

君子之道费而隐。夫妇之愚,可以与知焉;及其至也,虽圣人亦有所不知焉。夫妇之不肖,可以能行焉;及其至也,虽圣人亦有所不能焉。天地之大也,人犹有所憾。故君子语大,天下莫能载焉;语小,天下莫能破焉。《诗》云:“鸢飞戾天,鱼跃于渊。”言其上下察也。君子之道,造端乎夫妇,及其至也,察乎天地。

子曰:“道不远人,人之为道而远人,不可以为道。《诗》云:‘伐柯伐柯,其则不远。’执柯以伐柯,睨而视之,犹以为远。故君子以人治人,改而止。忠恕违道不远,施诸己而不愿,亦勿施于人。君子之道四,丘未能一焉,所求乎子,以事父,未能也;所求乎臣,以事君,未能也;所求乎弟,以事兄,未能也;所求乎朋友先施之,未能也。庸德之行,庸言之谨;有所不足,不敢不勉,有馀不敢尽;言顾行,行顾言,君子胡不慥慥尔!”

君子素其位而行,不愿乎其外。素富贵,行乎富贵;素贫贱,行乎贫贱;素夷狄,行乎夷狄;素患难行乎患难,君子无入而不自得焉。在上位不陵下,在下位不援上,正己而不求于人,则无怨。上不怨天,下不尤人。故君子居易以俟命。小人行险以徼幸。子曰:“射有似乎君子,失诸正鹄,反求诸其身。”

君子之道,辟如行远必自迩,辟如登高必自卑。《诗》曰:“妻子好合,如鼓瑟琴。兄弟既翕,和乐且耽。宜尔室家,乐尔妻帑。”子曰:“父母其顺矣乎!”

子曰:“鬼神之为德,其盛矣乎?!视之而弗见,听之而弗闻,体物而不可遗,使天下之人齐明盛服,以承祭祀。洋洋乎如在其上,如在其左右。《诗》曰:‘神之格思,不可度思!矧可射思!’夫微之显,诚之不可揜如此夫。”

子曰:“舜其大孝也与!德为圣人,尊为天子,富有四海之内。宗庙飨之,子孙保之。故大德必得其位,必得其禄,必得其名,必得其寿。故天之生物,必因其材而笃焉。故栽者培之,倾者覆之。《诗》曰:‘嘉乐君子,宪宪令德。宜民宜人,受禄于天,保佑命之,自天申之。’故大德者必受命。”

子曰:“无忧者,其惟文王乎!以王季为父,以武王为子,父作之,子述之。武王缵大王、王季、文王之绪,壹戎衣而有天下。身不失天下之显名,尊为天子,富有四海之内。宗庙飨之,子孙保之。武王末受命,周公成文、武之德,追王大王、王季,上祀先公以天子之礼。斯礼也,达乎诸侯大夫,及士庶人。父为大夫,子为士,葬以大夫,祭以士。父为士,子为大夫,葬以士,祭以大夫。期之丧,达乎大夫。三年之丧,达乎天子。父母之丧,无贵贱,一也。”

子曰:“武王、周公,其达孝矣乎!夫孝者,善继人之志,善述人之事者也。春秋修其祖庙,陈其宗器,设其裳衣,荐其时食。宗庙之礼,所以序昭穆也。序爵,所以辨贵贱也。序事,所以辨贤也。旅酬下为上,所以逮贱也。燕毛,所以序齿也。践其位,行其礼,奏其乐,敬其所尊,爱其所亲,事死如事生,事亡如事存,孝之至也。郊社之礼,所以事上帝也。宗庙之礼,所以祀乎其先也。明乎郊社之礼、禘尝之义,治国其如示诸掌乎!”

哀公问政。子曰:“文武之政,布在方策。其人存,则其政举;其人亡,则其政息。人道敏政,地道敏树。夫政也者,蒲卢也。故为政在人,取人以身,修身以道,修道以仁。仁者人也,亲亲为大;义者宜也,尊贤为大。亲亲之杀,尊贤之等,礼所生也。在下位不获乎上,民不可得而治矣!故君子不可以不修身;思修身,不可以不事亲;思事亲,不可以不知人,思知人,不可以不知天。”

“天下之达道五,所以行之者三。曰:君臣也,父子也,夫妇也,昆弟也,朋友之交也,五者天下之达道也。\hl{知,仁,勇,三者天下之达德也,所以行之者一也}。或生而知之,或学而知之,或困而知之,及其知之,一也。或安而行之,或利而行之,或勉强而行之,及其成功,一也。子曰:好学近乎知,力行近乎仁,知耻近乎勇。知斯三者,则知所以修身;知所以修身,则知所以治人;知所以治人,则知所以治天下国家矣。”

“凡为天下国家有九经,曰:修身也。尊贤也,亲亲也,敬大臣也,体群臣也。子庶民也,来百工也,柔远人也,怀诸侯也。
修身则道立,尊贤则不惑,亲亲则诸父昆弟不怨,敬大臣则不眩,体群臣则士之报礼重,子庶民则百姓劝,来百工则财用足,柔远人则四方归之,怀诸侯则天下畏之。
齐明盛服,非礼不动。所以修身也;去谗远色,贱货而贵德,所以劝贤也;尊其位,重其禄,同其好恶,所以劝亲亲也;官盛任使,所以劝大臣也;忠信重禄,所以劝士也;
时使薄敛,所以劝百姓也;日省月试,既廪称事,所以劝百工也;送往迎来,嘉善而矜不能,所以柔远人也;
继绝世,举废国,治乱持危。朝聘以时,厚往而薄来,所以怀诸侯也。凡为天下国家有九经,所以行之者一也。”

“凡事豫则立,不豫则废。言前定则不跲,事前定则不困,行前定则不疚,道前定则不穷。在下位不获乎上,民不可得而治矣。
获乎上有道,不信乎朋友,不获乎上矣;信乎朋友有道,不顺乎亲,不信乎朋友矣;顺乎亲有道,反诸身不诚,不顺乎亲矣;诚身有道,不明乎善,不诚乎身矣。
\hl{诚者,天之道也;诚之者,人之道也。诚者不勉而中,不思而得,从容中道,圣人也。诚之者,择善而固执之者也}。”

“博学之,审问之,慎思之,明辨之,笃行之。有弗学,学之弗能,弗措也;有弗问,问之弗知,弗措也;有弗思,思之弗得,弗措也;有弗辨,辨之弗明,弗措也;有弗行,行之弗笃,弗措也。
人一能之己百之,人十能之己千之。果能此道矣。虽愚必明,虽柔必强。”

自诚明,谓之性。自明诚,谓之教。诚则明矣,明则诚矣。

\hl{唯天下至诚,为能尽其性;能尽其性,则能尽人之性;能尽人之性,则能尽物之性;能尽物之性,则可以赞天地之化育;可以赞天地之化育,则可以与天地参矣}。

其次致曲。曲能有诚,诚则形,形则著,著则明,明则动,动则变,变则化。唯天下至诚为能化。

至诚之道,可以前知。国家将兴,必有祯祥;国家将亡,必有妖孽。见乎蓍龟,动乎四体。祸福将至:善,必先知之;不善,必先知之。故至诚如神。

诚者自成也,而道自道也。诚者物之终始,不诚无物。是故君子诚之为贵。诚者非自成己而已也,所以成物也。
成己,仁也;成物,知也。性之德也,合外内之道也,故时措之宜也。

\hl{故至诚无息。不息则久,久则徵;徵则悠远,悠远则博厚,博厚则高明}。博厚,所以载物也;高明,所以覆物也;悠久,所以成物也。博厚配地,高明配天,悠久无疆。
如此者,不见而章,不动而变,无为而成。\hl{天地之道,可壹言而尽也。其为物不贰,则其生物不测}。天地之道:博也,厚也,高也,明也,悠也,久也。
今夫天,斯昭昭之多,及其无穷也,日月星辰系焉,万物覆焉。今夫地,一撮土之多,及其广厚,载华岳而不重,振河海而不泄,万物载焉。
今夫山,一卷石之多,及其广大,草木生之,禽兽居之,宝藏兴焉,今夫水,一勺之多,及其不测,鼋鼍、蛟龙、鱼鳖生焉,货财殖焉。
《诗》曰:“惟天之命,于穆不已!”盖曰天之所以为天也。“于乎不显,文王之德之纯!”盖曰文王之所以为文也,纯亦不已。

大哉!圣人之道洋洋乎!发育万物,峻极于天。优优大哉!礼仪三百,威仪三千,待其人然後行。故曰:苟不至德,至道不凝焉。
\hl{故君子尊德性而道问学;致广大而尽精微;极高明而道中庸;温故而知新,敦厚以崇礼}。是故居上不骄,为下不倍;国有道,其言足以兴;国无道,其默足以容。
《诗》曰:“既明且哲,以保其身。”其此之谓与!

子曰:“愚而好自用,贱而好自专,生乎今之世,反古之道:如此者,灾及其身者也。”非天子,不议礼,不制度,不考文。今天下车同轨,书同文,行同伦。虽有其位,苟无其德,不敢作礼乐焉;虽有其德。苟无其位,亦不敢作礼乐焉。子曰:“吾说夏礼,杞不足徵也。吾学殷礼,有宋存焉。吾学周礼,今用之,吾从周。”

王天下有三重焉,其寡过矣乎!上焉者虽善无徵,无徵不信,不信民弗从;下焉者虽善不尊,不尊不信,不信民弗从。故君子之道:本诸身,徵诸庶民,考诸三王而不缪,建诸天地而不悖,质诸鬼神而无疑,百世以俟圣人而不惑。质诸鬼神而无疑,知天也;百世以俟圣人而不惑,知人也。是故君子动而世为天下道,行而世为天下法,言而世为天下则。远之则有望,近之则不厌。《诗》曰:“在彼无恶,在此无射。庶几夙夜,以永终誉!”君子未有不如此而蚤有誉于天下者也。

仲尼祖述尧舜,宪章文武:上律天时,下袭水土。辟如天地之无不持载,无不覆帱,辟如四时之错行,如日月之代明。万物并育而不相害,道并行而不相悖,小德川流,大德敦化,此天地之这所以为大也。

唯天下至圣为能聪明睿知,足以有临也;宽裕温柔,足以有容也;发强刚毅,足以有执也;齐庄中正,足以有敬也;文理密察,足以有别也。溥博渊泉,而时出之。溥博如天,渊泉如渊。
见而民莫不敬,言而民莫不信,行而民莫不说。是以声名洋溢乎中国,施及蛮貊。舟车所至,人力所通,天之所覆,地之所载,日月所照,霜露所队,凡有血气者,莫不尊亲,故曰配天。

\hl{唯天下至诚,为能经纶天下之大经,立天下之大本,知天地之化育}。夫焉有所倚?肫肫其仁!渊渊其渊!浩浩其天!苟不固聪明圣知达天德者,其孰能知之?

《诗》曰:“衣锦尚絅。”恶其文之著也。故君子之道,暗然而日章;小人之道,的然而日亡。君子之道:淡而面不厌,简而文,温而理,知远之近,知风之自,知微之显,可与入德矣。
《诗》云:“潜虽伏矣,亦孔之昭!”故君子内省不疚,无恶于志。君子所不可及者,其唯人之所不见乎!
《诗》云:“相在尔室,尚不愧于屋漏。”故君子不动而敬,不言而信。
《诗》曰:“奏假无言,时靡有争。”是故君子不赏而民劝,不怒而民威于鈇钺。
《诗》曰:“不显惟德!百辟其刑之。”是故君子笃恭而天下平。
《诗》云:“予怀明德,不大声以色。”子曰:“声色之于以化民。末也。”
《诗》曰:“德如毛。”毛犹有伦,上天之载,无声无臭,至矣!

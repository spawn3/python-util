\chapter{原则}

原则近似于我之所谓道,人何以之道?曰:心。心何以知?曰:虚一而静。
精于道者兼物物,一于道而以赞稽之,则万物官矣。

\section{尊道}

道德仁义礼,五者一体也,而道为之主,故第0原则即是尊道。
然何以尊之?需明法。

老子七善,三学六度

\subsection{至诚不息}

君子养心莫善乎诚,致诚则无它事也。

实事求是,拥抱现实,超越现实。

\subsection{虚壹而静}

是大原则,每临大事有静气,不信今时无古贤。

心善渊

将军之事,静以幽,正以治。

\subsection{圆点哲学}

最小最大模型,立其环中,以应无穷。

\subsection{双线法则}

\subsection{123哲学}

\subsection{机器之喻}

欲收无为而治之效,不能不着重在打磨机器、系统上,建立系统思维。
自组织、自进化的系统是工作的产物。

用系统来工作

\section{工作之道}

以终为始

要事第一

全局优化(统合)

\section{生活之道}

闲居静思则通

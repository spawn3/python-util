\section{管子}

\subsection{内业}

凡物之精,此则为生。下生五谷,上为列星。流于天地之间,谓之鬼神;藏于胸中,谓之圣人。
是故民气,杲乎如登于天,杳乎如入于渊,淖乎如在于海,卒乎如在于己。
是故此气也,不可止以力,而可安以德;不可呼以声,而可迎以音。敬守勿失,是谓成德,德成而智出,万物果得。

凡心之刑,自充自盈,自生自成。其所以失之,必以忧乐喜怒欲利。能去忧乐喜怒欲利,心乃反济。
彼心之情,利安以宁,勿烦勿乱,和乃自成。折折乎如在于侧,忽忽乎如将不得,渺渺乎如穷无极。
此稽不远,日用其德。

夫道者,所以充形也,而人不能固。其往不复,其来不舍。谋乎莫闻其音,卒乎乃在于心;冥冥乎不见其形,淫淫乎与我俱生。不见其形;不闻其声,而序其成,谓之道。
凡道无所,善心安爱。心静气理,道乃可止。彼道不远,民得以产;彼道不离,民因以知。是故卒乎其如可与索,眇眇乎其如穷无所。彼道之情,恶音与声,修心静音,道乃可得。
道也者,口之所不能言也,目之所不能视也,耳之所不能听也,所以修心而正形也;人之所失以死,所得以生也;事之所失以败,所得以成也。
凡道无根无茎,无叶无荣。万物以生,万物以成,命之曰道。

天主正,地主平,人主安静。春秋冬夏,天之时也;山陵川谷,地之枝也;喜怒取予,人之谋也。是故圣人与时变而不化,从物而不移。能正能静,然后能定。定心在中,耳目聪明,四肢坚固,可以为精舍。精也者,气之精者也。气,道乃生,生乃思,思乃知,知乃止矣。凡心之形,过知失生。

一物能化谓之神,一事能变谓之智。化不易气,变不易智,唯执一之君子能为此乎!执一不失,能君万物。君子使物,不为物使,得一之理。治心在于中,治言出于口,治事加于人,然则天下治矣。一言得而天下服,一言定而天下听,公之谓也。

形不正,德不来;中不静,心不治。正形摄德,天仁地义,则淫然而自至神明之极,照乎知万物。中义守不忒,不以物乱官,不以官乱心,是谓中得。

有神自在身,一往一来,奠之能思。失之必乱,得之必治。敬除其舍,精将自来。精想思之,宁念治之,严容畏敬,精将至定。得之而勿舍,耳目不淫。

心无他图,正心在中,万物得度。道满天下,普在民所,民不能知也。一言之解,上察于天,下极于地,蟠满九州。何谓解之?在于心安。我心治,官乃治,我心安,官乃安。治之者心也,安之者心也。

心以藏心,心之中又有心焉。彼心之心,音以先言。音然后形,形然后言,言然后使,使然后治。不治必乱,乱乃死。

精存自生,其外安荣,内藏以为泉原,浩然和平,以为气渊。渊之不涸,四体乃固;泉之不竭,九窍遂通。乃能穷天地,破四海。中无惑意,外无邪灾,心全于中,形全于外,不逢天灾,不遇人窖,谓之圣人。

人能正静,皮肤裕宽,耳目聪明,筋信而骨强。乃能戴大圜,而履大方,鉴于大清,视干大明。敬慎无忒,日新其德,遍知天下,穷于四极。敬发其充,是谓内得。然而不反,此生之忒。

凡道,必周必密,必宽必舒,必坚必固,守善勿舍,逐淫泽薄,既知其极,反于道德。全心在中,不可蔽匿,和于形容,见于肤色。善气迎人,亲于弟兄;恶气迎人,害于戎兵。不言之声,疾于雷鼓;心气之形,明于日月,察于父母。赏不足以劝善,刑不足以惩过,气意得而天下服,心意定而天下听。

搏气如神,万物备存。能搏乎?能一乎?能无卜筮而知吉凶乎?能止乎?能已乎?能勿求诸人而得之己乎?思之,思之,又重思之。思之而不通,鬼神将通之。非鬼神之力也,精气之极也。

四体既正,血气既静,一意搏心,耳目不淫,虽远若近。思索生知,慢易生忧,暴傲生怨,忧郁生疾,疾困乃死。思之而不舍,内困外薄,不早为图,生将巽舍。食莫若无饱,思莫若勿致,节适之齐,彼将自至。

凡人之生也,天出其精,地出其形,合此以为人。和乃生,不和不生。察和之道,其精不见,其征不丑。平正擅匈,论治在心。此以长寿。忿怒之失度,乃为之图。节其五欲,去其二凶,不喜不怒,平正擅匈。

凡人之生也,必以平正。所以失之,必以喜怒忧患。是故止怒莫若诗,去忧莫若乐,节乐莫若礼,守礼莫若敬,守敬莫若静。内静外敬,能反其性,性将大定。

凡食之道:大充,伤而形不臧;大摄,骨枯而血沍。充摄之间,此谓和成,精之所舍,而知之所生,饥饱之失度,乃为之图。饱则疾动,饥则广思,老则长虑。饱不疾动,气不通于四末;饥不广思,饱而不废;老不长虑,困乃速竭。大心而敢,宽气而广,其形安而不移,能守一而弃万苛,见利不诱,见害不俱,宽舒而仁,独乐其身,是谓云气,意行似天。

凡人之生也,必以其欢。忧则失纪,怒则失端。忧悲喜怒,道乃无处。爱欲静之,遇乱正之,勿引勿推,福将自归。彼道自来,可藉与谋,静则得之,躁则失之。灵气在心,一来一逝,其细无内,其大无外。所以失之,以躁为害。心能执静,道将自定。得道之人,理丞而屯泄,匈中无败。节欲之道,万物不害。

\subsection{心术上}

心之在体,君之位也;九窍之有职,官之分也。心处其道。九窍循理;嗜欲充益,目不见色,耳不闻声。故曰上离其道,下失其事。毋代马走,使尽其力;毋代鸟飞,使弊其羽翼;毋先物动,以观其则。动则失位,静乃自得。

道,不远而难极也,与人并处而难得也。虚其欲,神将入舍;扫除不洁,神乃留处。人皆欲智而莫索其所以智乎。智乎,智乎,投之海外无自夺,求之者不得处之者。夫正人无求之也,故能虚无。

虚无无形谓之道,化育万物谓之德,君臣父子人间之事谓之义,登降揖让、贵贱有等、亲疏之体谓之礼,简物、小未一道。杀僇禁诛谓之法。

大道可安而不可说。直人之言不义不颇,不出于口,不见于色,四海之人,又孰知其则?

天曰虚,地曰静,乃不伐。洁其宫,开其门,去私毋言,神明若存。纷乎其若乱,静之而自治。强不能遍立,智不能尽谋。物固有形,形固有名,名当,谓之圣人。故必知不言,无为之事,然后知道之纪。殊形异埶,不与万物异理,故可以为天下始。

人之可杀,以其恶死也;其可不利,以其好利也。是以君子不休乎好,不迫乎恶,恬愉无为,去智与故。其应也,非所设也;其动也,非所取也。过在自用,罪在变化。是故有道之君,其处也若无知,其应物也若偶之。静因之道也。

“心之在体,君之位也;九窍之有职,官之分也。”耳目者。视听之官也,心而无与于视听之事,则官得守其分矣。夫心有欲者,物过而目不见,声至而耳不闻也。故曰:“上离其道,下失其事。”故曰:心术者,无为而制窍者也。故曰“君”。“毋代马走”,“毋代鸟飞”,此言不夺能能,不与下诚也。“毋先物动”者,摇者不走,趮者不静,言动之不可以观也。“位”者”,谓其所立也。人主者立于阴,阴者静,故曰“动则失位”。阴则能制阳矣,静则能制动矣,攸曰,‘静乃自得。”

道在天地之间也,其大无外,其小无内,故曰“不远而难极也”。虚之与人也无间,唯圣人得虚道,故曰“并处而难得”。世人之所职者精也。去欲则宣,宣则静矣,静则精。精则独立矣,独则明,明则神矣。神者至贵也,故馆不辟除,则贵人不舍焉。故曰“不洁则神不处”。“人皆欲知而莫索之”,其所(以)知,彼也;其所以知,此也。不修之此,焉能知彼?修之此,莫能虚矣。虚者,无藏也。故曰去知则奚率求矣,无藏则奚设矣。无求无设则无虑,无虑则反复虚矣。

天之道,虚其无形。虚则不屈,无形则无所位迕,无所位迕,故遍流万物而不变,德者,道之舍,物得以生生,知得以职道之精。故德者得也。得也者,其谓所得以然也。以无为之谓道,舍之之谓德。故道之与德无间,故言之者不别也。间之理者,谓其所以舍也。义者,谓各处其宜也。礼者,因人之情,缘义之理,而为之节文者也,故礼者谓有理也。理也者,明分以谕义之意也。故礼出乎义,义出乎理,理因乎宜者也。法者所以同出,不得不然者也,故杀僇禁诛以一之也。故事督乎法,法出乎权,权出于道。

道也者、动不见其形,施不见其德,万物皆以得,然莫知其极。故曰“可以安而不可说”也。莫人,言至也。不宜,言应也。应也者,非吾所设,故能无宜也。不顾,言因也。因也者,非吾所顾,故无顾也。“不出于口,不见于色”,言无形也;“四海之人,孰知其则”,言深囿也。

天之道虚,地之道静。虚则不屈,静则不变,不变则无过,故曰“不伐”。“洁其宫,阙其门”:宫者,谓心也。心也者,智之舍也,故曰“宫”。洁之者,去好过也。门者,谓耳目也。耳目者,所以闻见也。“物固有形,形固有各”,此言不得过实、实不得延名。姑形以形,以形务名,督言正名,故曰“圣人”。“不言之言”,应也。应也者,以其为之人者也。执其名,务其应,所以成,之应之道也。“无为之道,因也。因也者,无益无损也。以其形因为之名,此因之术也。名者,圣人之所以纪万物也。人者立于强,务于善,未于能,动于故者也。圣人无之,无之则与物异矣。异则虚,虚者万物之始也,故曰“可以为天下始”。

人迫于恶,则失其所好;怵于好,则忘其所恶。非道也。故曰:“不怵乎好,不迫乎恶。”恶不失其理,欲不过其情,故曰:“君子”。“恬愉无为,去智与故”,言虚素也。“其应非所设也,其动非所取也”,此言因也。因也者,舍己而以物为法者也。感而后应,非所设也;缘理而动,非所取也,“过在自用,罪在变化”:自用则不虚,不虚则仵于物矣;变化则为生,为生则乱矣。故道贵因。因者,因其能者,言所用也。“君子之处也若无知”,言至虚也;“其应物也若偶之”,言时适也、若影之象形,响之应声也。故物至则应,过则舍矣。舍矣者,言复所于虚也。

\subsection{心术下}

形不正者,德不来;中不精者,心不冶。正形饰德,万物毕得,翼然自来,神莫知其极,昭知天下,通于四极。是故曰:无以物乱官,毋以官乱心,此之谓内德。是故意气定,然后反正。气者身之充也,行者正之义也。充不美则心不得,行不正则民不服。是故圣人若天然,无私覆也;若地然,无私载也。私者,乱天下者也。

凡物载名而来,圣人因而财之,而天下治。实不伤,不乱于天下,而天下治。专于意,一于心,耳目端,知远之证。能专乎?能一乎?能毋卜筮而知凶吉乎?能止乎?能已乎?能毋问于人而自得之于己乎?故曰,思之。思之不得,鬼神教之。非鬼神之力也。其精气之极也。

一气能变曰精、一事能变曰智。慕选者,所以等事也;极变者,所以应物也。慕选而不乱,极变而不烦,执一之君子执一而不失,能君万物,日月之与同光,天地之与同理。

圣人裁物,不为物使。心安,是国安也;心治,是国治也。治也者心也,安也者心也。治心在于中,治言出于口,治事加于民,故功作而民从,则百姓治矣。所以操者非刑也,所以危者非怒也。民人操,百姓治,道其本至也,至不至无,非所人而乱。

凡在有司执制者之利,非道也。圣人之道,若存若亡,援而用之,殁世不亡。与时变而不化,应物而不移,日用之而不化。

人能正静者,筋肕而骨强;能戴大圆者,体乎大方;镜大清者,视乎大明。正静不失,日新其德,昭知天下,通于四极。金心在中不可匿,外见于形容,可知于颜色。善气迎人,亲如弟兄;恶气迎人,害于戈兵。不言之言,闻于雷鼓。全心之形,明于日月,察于父母。昔者明王之爱天下,故天下可附;暴王之恶天下,故天下可离。故货之不足以为爱,刑之不足以为恶。货者爱之末也,刑者恶之末也。

凡民之生也,必以正平;所以失之者,必以喜乐哀怒,节怒莫若乐,节乐莫若礼,守礼莫若敬。外敬而内静者,必反其性。

岂无利事哉?我无利心。岂无安处哉?我无安心。心之中又有心。意以先言,意然后形,形然后思,思然后知。凡心之形,过知失生。

是故内聚以为原。泉之不竭,表里遂通;泉之不涸,四支坚固。能令用之,被及四固。

是故圣人一言解之,上察于天,下察于地。

\subsection{白心}

凡物之精,此则为生。下生五谷,上为列星。流于天地之间,谓之鬼神;藏于胸中,谓之圣人。是故民气,杲乎如登于天,杳乎如入于渊,淖乎如在于海,卒乎如在于己。是故此气也,不可止以力,而可安以德;不可呼以声,而可迎以音。敬守勿失,是谓成德,德成而智出,万物果得。

凡心之刑,自充自盈,自生自成。其所以失之,必以忧乐喜怒欲利。能去忧乐喜怒欲利,心乃反济。彼心之情,利安以宁,勿烦勿乱,和乃自成。折折乎如在于侧,忽忽乎如将不得,渺渺乎如穷无极。此稽不远,日用其德。

夫道者,所以充形也,而人不能固。其往不复,其来不舍。谋乎莫闻其音,卒乎乃在于心;冥冥乎不见其形,淫淫乎与我俱生。不见其形;不闻其声,而序其成,谓之道。凡道无所,善心安爱。心静气理,道乃可止。彼道不远,民得以产;彼道不离,民因以知。是故卒乎其如可与索,眇眇乎其如穷无所。彼道之情,恶音与声,修心静音,道乃可得。道也者,口之所不能言也,目之所不能视也,耳之所不能听也,所以修心而正形也;人之所失以死,所得以生也;事之所失以败,所得以成也。凡道无根无茎,无叶无荣。万物以生,万物以成,命之曰道。

天主正,地主平,人主安静。春秋冬夏,天之时也;山陵川谷,地之枝也;喜怒取予,人之谋也。是故圣人与时变而不化,从物而不移。能正能静,然后能定。定心在中,耳目聪明,四肢坚固,可以为精舍。精也者,气之精者也。气,道乃生,生乃思,思乃知,知乃止矣。凡心之形,过知失生。

一物能化谓之神,一事能变谓之智。化不易气,变不易智,唯执一之君子能为此乎!执一不失,能君万物。君子使物,不为物使,得一之理。治心在于中,治言出于口,治事加于人,然则天下治矣。一言得而天下服,一言定而天下听,公之谓也。

形不正,德不来;中不静,心不治。正形摄德,天仁地义,则淫然而自至神明之极,照乎知万物。中义守不忒,不以物乱官,不以官乱心,是谓中得。

有神自在身,一往一来,奠之能思。失之必乱,得之必治。敬除其舍,精将自来。精想思之,宁念治之,严容畏敬,精将至定。得之而勿舍,耳目不淫。

心无他图,正心在中,万物得度。道满天下,普在民所,民不能知也。一言之解,上察于天,下极于地,蟠满九州。何谓解之?在于心安。我心治,官乃治,我心安,官乃安。治之者心也,安之者心也。

心以藏心,心之中又有心焉。彼心之心,音以先言。音然后形,形然后言,言然后使,使然后治。不治必乱,乱乃死。

精存自生,其外安荣,内藏以为泉原,浩然和平,以为气渊。渊之不涸,四体乃固;泉之不竭,九窍遂通。乃能穷天地,破四海。中无惑意,外无邪灾,心全于中,形全于外,不逢天灾,不遇人窖,谓之圣人。

人能正静,皮肤裕宽,耳目聪明,筋信而骨强。乃能戴大圜,而履大方,鉴于大清,视干大明。敬慎无忒,日新其德,遍知天下,穷于四极。敬发其充,是谓内得。然而不反,此生之忒。

凡道,必周必密,必宽必舒,必坚必固,守善勿舍,逐淫泽薄,既知其极,反于道德。全心在中,不可蔽匿,和于形容,见于肤色。善气迎人,亲于弟兄;恶气迎人,害于戎兵。不言之声,疾于雷鼓;心气之形,明于日月,察于父母。赏不足以劝善,刑不足以惩过,气意得而天下服,心意定而天下听。

搏气如神,万物备存。能搏乎?能一乎?能无卜筮而知吉凶乎?能止乎?能已乎?能勿求诸人而得之己乎?思之,思之,又重思之。思之而不通,鬼神将通之。非鬼神之力也,精气之极也。

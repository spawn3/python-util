\chapter{一二三哲学}

毕建勋的一二三哲学,一读之下,心有灵犀。

一二三哲学旨在为中国画学建立理论体系,但有着更为普适的意义在。
自庞朴提出一分为三论之后,对此问题思索已久,依然不得要领。
毕建勋的一二三哲学,多有精妙之论。

\section{三合结构}

\subsection{道}

道的演化路径

围绕老子“道生一,一生二,二生三,三生万物”,
周易“易有太极,是生两仪,两仪生四象,四象生八卦,八卦定吉凶,吉凶生大业”而立论。

兼三立两迭用推一,上行方法,升维到道的高度,与道的演化方向相反,正反合一,是完整二一关系。
一而二,二而一。以上行方法见道,以下行方法实践,是为道知,不出户而知天下,不窥牖而见天道。

阴阳三合,何本何化?

道具有三合结构,同时居于三合结构的最高点。道高于两级,基于两极。一故神,两故化。
道是价值论、认识论与方法论的统一。

气为道之体/相,有无为道之性,一二三为道之理,二一之法为道之法。三合为其空间结构。以上为一二三哲学。

道以有无为性,以气为体,以一二三为理,以二一为法。
此气化结构,内含数理法,体相用,数理法。

道以常有无为性体,以气为相,以万物为用。以一二三为数理,以二一为法。

观物取象,立象尽意。执大象,天下往。大象者,模型也。

\subsection{有无}

道者,有无之总名。

\subsection{理气}

\subsection{兼三}

\subsection{立两}

\subsection{二一}

三合之道有上下的层次关系,也有左右对称关系。层峦叠嶂,至于无穷。
体道活动就是立二参一,由现象之二,推而上行,臻于含二之一,即是太极笔法。

道之性是有无,故道变动不居,周流六虚,上下无常,刚柔相易,不可为典要,唯变所适。
道是确定性与不确定的统一。

运用到画学,则道心物为三合之道。

\section{流}

\subsection{一分为二}

\subsection{一分为三}

\subsection{中庸}

\subsection{道知}

鬼谷子

\subsection{天解}

淮南子

\subsection{致良知}

秉道通物:知良知,心与道合,事上磨练,心物合一。
心与道合,谓之道心,以道心应物如镜。

\subsection{志量识}

志,心之所主。

曾国藩:有志、有识、有恒。和君三度修炼:态度、气度、厚度,分别对应志量识,很是贴切。
态度决定命运,气度决定格局,底蕴的厚度决定人生的高度。

\section{感悟}

二叉树统一了老子与周易,即是一分为二的序列,也是阴阳三合的空间结构。

天命之谓性,率性之谓道,修道之谓教。心与道合,天人合一,即是人的天命,内圣外王之道,于斯立矣。
心与道合,有赖于心术。心性修养,心灵质地,是其基础。

三合之道成而万事备

有无者道之性妙之门

有无玄同大制不割,黑白之间灰度存焉

\subsection{参一治二}

一者道,二者心与物。始于心,心与道合,下及于物,此之谓秉道通物。

致良知,良知为何?在心之天理。何以致之?离不开事上磨练,知行合一的功夫。

本体与功夫,心与道合是便捷之路。执道要之柄,以游于无穷之地。

志于道、游于艺,两者互为因果,有无相生,两仪立而太极现。

在道心物的三合结构下,可以解释诸多问题和现象。

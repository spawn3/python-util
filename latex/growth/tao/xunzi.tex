\section{荀子}

\subsection{解蔽}

凡人之患,蔽于一曲,而闇于大理。治则复经,两疑则惑矣。天下无二道,圣 人无两心。今诸侯异政,百家异说,则必或是或非,或治或乱。乱国之君,乱家之 人,此其诚心,莫不求正而以自为也。妒缪于道,而人诱其所迨也。私其所积,唯 恐闻其恶也。倚其所私,以观异术,唯恐闻其美也。是以与治虽走,而是己不辍也。 岂不蔽于一曲,而失正求也哉!心不使焉,则白黑在前而目不见,雷鼓在侧而耳不 闻,况于使者乎?德道之人,乱国之君非之上,乱家之人非之下,岂不哀哉!

故为蔽:欲为蔽,恶为蔽,始为蔽,终为蔽,远为蔽,近为蔽,博为蔽,浅为 蔽,古为蔽,今为蔽。凡万物异则莫不相为蔽,此心术之公患也。

昔人君之蔽者,夏桀殷纣是也。桀蔽于末喜斯观,而不知关龙逢,以惑其心, 而乱其行。桀蔽于妲己、飞廉,而不知微子启,以惑其心,而乱其行。故群臣去忠 而事私,百姓怨非而不用,贤良退处而隐逃,此其所以丧九牧之地,而虚宗庙之国 也。桀死于鬲山,纣县于赤旆。身不先知,人又莫之谏,此蔽塞之祸也。成汤监于 夏桀,故主其心而慎治之,是以能长用伊尹,而身不失道,此其所以代夏王而受九 有也。文王监于殷纣,故主其心而慎治之,是以能长用吕望,而身不失道,此其所 以代殷王而受九牧也。远方莫不致其珍;故目视备色,耳听备声,口食备味,形居 备宫,名受备号,生则天下歌,死则四海哭。夫是之谓至盛。诗曰:“凤凰秋秋, 其翼若干,其声若箫。有凤有凰,乐帝之心。”此不蔽之福也。

昔人臣之蔽者,唐鞅奚齐是也。唐鞅蔽于欲权而逐载子,奚齐蔽于欲国而罪申 生;唐鞅戮于宋,奚齐戮于晋。逐贤相而罪孝兄,身为刑戮,然而不知,此蔽塞之 祸也。故以贪鄙、背叛、争权而不危辱灭亡者,自古及今,未尝有之也。鲍叔、宁 戚、隰朋仁知且不蔽,故能持管仲,而名利福禄与管仲齐。召公、吕望仁知且不蔽, 故能持周公而名利福禄与周公齐。传曰:“知贤之为明,辅贤之谓能,勉之强之, 其福必长。”此之谓也。此不蔽之福也。

昔宾孟之蔽者,乱家是也。墨子蔽于用而不知文。宋子蔽于欲而不知得。慎子 蔽于法而不知贤。申子蔽于埶而不知知。惠子蔽于辞而不知实。庄子蔽于天而不知 人。故由用谓之道,尽利矣。由欲谓之道,尽嗛矣。由法谓之道,尽数矣。由埶谓 之道,尽便矣。由辞谓之道,尽论矣。由天谓之道,尽因矣。此数具者,皆道之一 隅也。夫道者体常而尽变,一隅不足以举之。曲知之人,观于道之一隅,而未之能 识也。故以为足而饰之,内以自乱,外以惑人,上以蔽下,下以蔽上,此蔽塞之祸 也。孔子仁知且不蔽,故学乱术足以为先王者也。一家得周道,举而用之,不蔽于 成积也。故德与周公齐,名与三王并,此不蔽之福也。

圣人知心术之患,见蔽塞之祸,故无欲、无恶、无始、无终、无近、无远、无 博、无浅、无古、无今,兼陈万物而中县衡焉。是故众异不得相蔽以乱其伦也。

何谓衡?曰:道。故心不可以不知道;心不知道,则不可道,而可非道。人孰 欲得恣,而守其所不可,以禁其所可?以其不可道之心取人,则必合于不道人,而 不合于道人。以其不可道之心与不道人论道人,乱之本也。夫何以知?曰:心知道, 然后可道;可道然后守道以禁非道。以其可道之心取人,则合于道人,而不合于不 道之人矣。以其可道之心与道人论非道,治之要也。何患不知?故治之要在于知道。

人何以知道?曰:心。心何以知?曰:虚壹而静。心未尝不臧也,然而有所谓 虚;心未尝不两也,然而有所谓壹;心未尝不动也,然而有所谓静。人生而有知, 知而有志;志也者,臧也;然而有所谓虚;不以所已臧害所将受谓之虚。心生而有 知,知而有异;异也者,同时兼知之;同时兼知之,两也;然而有所谓一;不以夫 一害此一谓之壹。心卧则梦,偷则自行,使之则谋;故心未尝不动也;然而有所谓 静;不以梦剧乱知谓之静。未得道而求道者,谓之虚壹而静。作之:则将须道者之 虚则人,将事道者之壹则尽,尽将思道者静则察。知道察,知道行,体道者也。虚 壹而静,谓之大清明。万物莫形而不见,莫见而不论,莫论而失位。坐于室而见四 海,处于今而论久远。疏观万物而知其情,参稽治乱而通其度,经纬天地而材官万 物,制割大理而宇宙里矣。恢恢广广,孰知其极?睪睪广广,孰知其德?涫涫纷纷, 孰知其形?明参日月,大满八极,夫是之谓大人。夫恶有蔽矣哉!

心者,形之君也,而神明之主也,出令而无所受令。自禁也,自使也,自夺也, 自取也,自行也,自止也。故口可劫而使墨云,形可劫而使诎申,心不可劫而使易 意,是之则受,非之则辞。故曰:心容--其择也无禁,必自现,其物也杂博,其 情之至也不贰。诗云:“采采卷耳,不盈倾筐。嗟我怀人,寘彼周行。”倾筐易满 也,卷耳易得也,然而不可以贰周行。故曰:心枝则无知,倾则不精,贰则疑惑。 以赞稽之,万物可兼知也。身尽其故则美。类不可两也,故知者择一而壹焉。

农精于田,而不可以为田师;贾精于市,而不可以为市师;工精于器,而不可 以为器师。有人也,不能此三技,而可使治三官。曰:精于道者也。精于物者也。 精于物者以物物,精于道者兼物物。故君子壹于道,而以赞稽物。壹于道则正,以 赞稽物则察;以正志行察论,则万物官矣。昔者舜之治天下也,不以事诏而万物成。 处一危之,其荣满侧;养一之微,荣矣而未知。故道经曰:“人心之危,道心之微。” 危微之几,惟明君子而后能知之。故人心譬如盘水,正错而勿动,则湛浊在下,而 清明在上,则足以见鬒眉而察理矣。微风过之,湛浊动乎下,清明乱于上,则不可 以得大形之正也。心亦如是矣。故导之以理,养之以清,物莫之倾,则足以定是非 决嫌疑矣。小物引之,则其正外易,其心内倾,则不足以决麤理矣。故好书者众矣, 而仓颉独传者,壹也;好稼者众矣,而后稷独传者,壹也。好乐者众矣,而夔独传 者,壹也;好义者众矣,而舜独传者,壹也。倕作弓,浮游作矢,而羿精于射;奚 仲作车,乘杜作乘马,而造父精于御:自古及今,未尝有两而能精者也。曾子曰: “是其庭可以搏鼠,恶能与我歌矣!”

空石之中有人焉,其名曰觙。其为人也,善射以好思。耳目之欲接,则败其思; 蚊虻之声闻,则挫其精。是以辟耳目之欲,而远蚊虻之声,闲居静思则通。思仁若 是,可谓微乎?孟子恶败而出妻,可谓能自强矣,未及思也;有子恶卧而焠掌,可 谓能自忍矣;未及好也。辟耳目之欲,远蚊虻之声,可谓危矣;未可谓微也。夫微 者,至人也。至人也,何忍!何强!何危!故浊明外景,清明内景,圣人纵其欲, 兼其情,而制焉者理矣;夫何强!何忍!何危!故仁者之行道也,无为也;圣人之 行道也,无强也。仁者之思也恭,圣者之思也乐。此治心之道也。

凡观物有疑,中心不定,则外物不清。吾虑不清,未可定然否也。冥冥而行者, 见寝石以为伏虎也,见植林以为后人也:冥冥蔽其明也。醉者越百步之沟,以为蹞 步之浍也;俯而出城门,以为小之闺也:酒乱其神也。厌目而视者,视一为两;掩 耳而听者,听漠漠而以为哅哅:埶乱其官也。故从山上望牛者若羊,而求羊者不下 牵也:远蔽其大也。从山下望木者,十仞之木若箸,而求箸者不上折也:高蔽其长 也。水动而景摇,人不以定美恶:水埶玄也。瞽者仰视而不见星,人不以定有无: 用精惑也。有人焉以此时定物,则世之愚者也。彼愚者之定物,以疑决疑,决必不 当。夫苟不当,安能无过乎?

夏首之南有人焉;曰涓蜀梁。其为人也,愚而善畏。明月而宵行,俯见其影, 以为伏鬼也;仰视其发,以为立魅也。背而走,比至其家,失气而死。岂不哀哉! 凡人之有鬼也,必以其感忽之间,疑玄之时定之。此人之所以无有而有无之时也, 而己以定事。故伤于湿而痹,痹而击鼓烹豚,则必有敝鼓丧豚之费矣,而未有俞疾 之福也。故虽不在夏首之南,则无以异矣。

凡以知,人之性也;可以知,物之理也。以可以知人之性,求可以知物之理, 而无所疑止之,则没世穷年不能无也。其所以贯理焉虽亿万,已不足浃万物之变, 与愚者若一。学、老身长子,而与愚者若一,犹不知错,夫是之谓妄人。故学也者, 固学止之也。恶乎止之?曰:止诸至足。曷谓至足?曰:圣王。圣也者,尽伦者也; 王也者,尽制者也;两尽者,足以为天下极矣。故学者以圣王为师,案以圣王之制 为法,法其法以求其统类,以务象效其人。向是而务,士也;类是而几,君子也; 知之,圣人也。故有知非以虑是,则谓之惧;有勇非以持是,则谓之贼;察孰非以 分是,则谓之篡;多能非以修荡是,则谓之知;辩利非以言是,则谓之詍。传曰: “天下有二:非察是,是察非。”谓合王制不合王制也。天下不以是为隆正也,然 而犹有能分是非、治曲直者邪?

若夫非分是非,非治曲直,非辨治乱,非治人道,虽能之无益于人,不能无损 于人;案直将治怪说,玩奇辞,以相挠滑也;案强钳而利口,厚颜而忍诟,无正而 恣孳,妄辨而几利;不好辞让,不敬礼节,而好相推挤:此乱世奸人之说也,则天 下之治说者,方多然矣。传曰:“析辞而为察,言物而为辨,君子贱之。博闻强志, 不合王制,君子贱之。”此之谓也。

为之无益于成也,求之无益于得也,忧戚之无益于几也,则广焉能弃之矣,不 以自妨也,不少顷干之胸中。不慕往,不闵来,无邑怜之心,当时则动,物至而应, 事起而辨,治乱可否,昭然明矣。

周而成,泄而败,明君无之有也。宣而成,隐而败,闇君无之有也。故人君者, 周则谗言至矣,直言反矣;小人迩而君子远矣!诗云:“墨以为明,狐狸而苍。” 此言上幽而下险也。君人者,宣则直言至矣,而谗言反矣;君子迩而小人远矣!诗 云:“明明在下,赫赫在上。”此言上明而下化也。


\chapter{旋极图}

中和是最强大的力量,各方力量得到最好的安排。

中和二字,体用不二,内圣外王之道备矣。

资料汇编
\begin{enumbox}
\item 易经
\item 黄帝四经
\item 道德经
\item 中庸
\item 道枢
\item 道学通论
\item 圆运动的古中医学
\item 吴清源中的精神
\item 气韵生动
\end{enumbox}

\section{征古}

人物志:凡人之质量,中和最贵矣。中和之质,必平淡无味;故能调成五材,变化应节。
是故,观人察质,必先察其平淡,而后求其聪明。

\section{范畴}

中和之德

守中致和

阴阳平衡、不平衡之美

\section{方法论}

守中:多言数穷,不如守中

致和:协同一切因素来守中

人生算法的质能公式:E=mc^2。m即是中,c就是致和,最快速度。
方法论上确立这两条即可,长长的坡道厚厚的雪。
复利公司也是如此。从一个基数开始,若能天天向上,进步就神速了。

怎么与快速成长挂上钩?从中庸对中和的赞叹,可知中和之德,尽矣。
观天之道、执天之行,天之道即是中和、天之行就是守中致和,无疑矣。
致中和,天地位焉,万物育焉。

在积极吸收新的营养成分的时候,不要违背了守中致和的原则。守中致和是第一原理。

天体运行模型,形成大黑洞,无物能逃逸,精神内敛至极,宇宙中最厉害的力量。

河图,河是银河。飞流直下三千尺,疑似银河落九天。

这是旋极图给出的启示。任正非对灰度的理解、中庸、平衡、无为而治的理解,
都可以由此进行推演。力出一孔、利出一孔。这是最根本的定位。

真信厚土,滋养万物,五德归一,一归于道。

中有多义:度、数、信。诚、神、几。位。

中和之德,也即是熊春锦先生所说的德一,居于道、用于德,整个世界由此展开。

抱一以为天下式,执一明三定二,两生而参视。

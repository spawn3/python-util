\chapter{算法}

以来氏太极图为指导,实验制胜之道。

做实验的方式,研究怎么加快自身的成长速度。

\section{目的的意义}

生活,工作之外,开辟第三空间,一而三,三而一,同时能够做到和谐无间。

学会如何学习。不用忙着做事,学会如何学习更重要。

财务自由是必要的,是正确做事的一个附加结果。

% 爱让人沉醉,但不要为其所伤。

人生需要核心算法。核心算法解决人生会遇到的大问题,最重要的一个问题就是如何更好的成长。

你找到你的核心算法了吗?

得一以为天下贞。一就是核心算法,得之后可以all in,如同滚雪球,越滚越大。
如滚圆石于万仞之山,势也。

守正出奇,核心算法是正,到处apply是奇,转圆而求其合。
核心算法是圆心,立足于核心算法,运圆于心,无为而无不为。

守什么?守核心算法。破什么?核心算法破局而出。

高筑墙,广积粮,缓称王。

\section{人生算法一文的启示}

思想有堆砌的倾向,阳明格物之旨,归于一心也。

\section{来知德太极图}

\hl{来知德太极图是最核心的思维模型},作为宇宙第一模型,可以推演出其它重要的模型。
主宰者理,对待者数,流行者气。执大象、天下往。此为第一大象也。

阳明悟致良知,叹为千古圣人相传一点骨血也。

曾国藩顿悟刚柔相济,寓道法于黄老之中,而后无往而不利。

二宫尊德悟一圆之理,开辟新境界。

稻盛和夫的成功方程式,第一要素即是思维方式,来式太极图解决了思维方式的问题。
建中立极。

商道所揭示的六图思维模型,中极图统摄有极无极图,导向真一、大成,具有枢纽地位。

\subsection{主宰义}

一内一外两个圆,界定了范围。

\subsection{对待义}

\subsection{流行义}

周期律动,正弦曲线

\section{引申}

乾坤衍

\subsection{思维模型}

顶级思维模型。

处处可见圆点哲学的影子。矛盾分析法,双线法则,两眼论,
一阴一阳之谓道,万物负阴抱阳冲气以为和。

回到核心,从核心出发。找到核心的过程,一靠直觉,二靠试错,低成本地试错。

\subsection{孙正义二十五字诀}

\begin{shadequote}
\item 道天地将法
\item 顶情略七斗
\item 一流攻守群
\item 智信仁勇严
\item 风林火山海
\end{shadequote}

\subsection{单点突破}

\subsection{点线面体}

老子第十六章,义理丰富。

以正治国,以奇用兵,以无事取天下。此意渊深,可为座右铭。

孙子兵法提供了一套方案,孙正义有自己的归纳总结。\hl{我欲清溪寻鬼谷,不论礼乐但论兵}。
兵者是死生存亡的大事,社会暗流涌动,在温泉面纱下,竞争不可谓不激烈。
想要活出自己的人生,不能不考虑更重要的维度。

达里奥原则,培根新工具,笛卡尔方法谈,斯宾诺莎伦理学都旨在解决人生算法问题。

\hl{孙陶然五行管理兵法}。

开发人生算法,喻颖正做出了表率。按守破离的节奏,先对标,再突破。最小核心最大化,
先要感知自己的核心,通过反复练习,使之价值最大化。

TRIZ创新算法。

价值链分析

\subsection{三合之道}

\subsection{大卫之星}

\subsection{五行}

\section{方法论}

PDCA

信解行证

守破离

取势、明道、优术

道、天、地、将、法

定战略、搭班子、带队伍

\subsection{将略}

慎言!养成深沉厚重之心态。

破局,出奇制胜

从必然王国到自由王国

\subsection{战略要素}

\subsubsection{目标}

第一,列出最重要的五个目标。双列表,10/10/10原则。

事业有成是因,财务自由是果。找到自己的核心算法,其它一切则水到渠成。
反复打磨核心,可以用爱因斯坦质能方程来描述:E=mc\^2。m是核心,c是大量重复练习,E是果。

\subsubsection{打法}

\subsubsection{资源}

整合资源:客户,钱,人脉

\subsubsection{激励}

\chapter{算法}

你找到你的核心算法了吗?

得一以为天下贞。一就是核心算法,得之后可以all in,如同滚雪球,越滚越大。
如滚圆石于万仞之山,势也。

守正出奇,核心算法是正,到处apply是奇,转圆而求其合。
核心算法是圆心,立足于核心算法,运圆于心,无为而无不为。

守什么?守核心算法。破什么?核心算法破局而出。

高筑墙,广积粮,缓称王。

\section{Goal}

生活,工作之外,开辟第三空间,一而三,三而一,同时能够做到和谐无间。

学会如何学习。不用忙着做事,学会如何学习更重要。

财务自由是必要的,是正确做事的一个附加结果。

爱让人沉醉,但不要为其所伤。

\section{模型}

人生需要核心算法。核心算法解决人生会遇到的大问题,最重要的一个问题就是如何更好的成长。

\subsection{思维模型}

顶级思维模型。

处处可见圆点哲学的影子。矛盾分析法,双线法则,两眼论,
一阴一阳之谓道,万物负阴抱阳冲气以为和。

回到核心,从核心出发。找到核心的过程,一靠直觉,二靠试错,低成本地试错。

老子第十六章,义理丰富。

以正治国,以奇用兵,以无事取天下。此意渊深,可为座右铭。

孙子兵法提供了一套方案,孙正义有自己的归纳总结。我欲清溪寻鬼谷,不论礼乐但论兵。
兵者是死生存亡的大事,社会暗流涌动,在温泉面纱下,竞争不可谓不激烈。
想要活出自己的人生,不能不考虑更重要的维度。

达里奥原则,培根新工具,笛卡尔方法谈,斯宾诺莎伦理学都旨在解决人生算法问题。

孙陶然五行管理兵法。

开发人生算法,喻颖正做出了表率。按守破离的节奏,先对标,再突破。最小核心最大化,
先要感知自己的核心,通过反复练习,使之价值最大化。

高筑墙,广积粮,缓称王。

TRIZ创新算法。

价值链分析

\section{守}

取势、明道、优术

道、天、地、将、法

定战略、搭班子、带队伍

\subsection{战略要素}

\subsubsection{目标}

第一,列出最重要的五个目标。双列表,10/10/10原则。

事业有成是因,财务自由是果。找到自己的核心算法,其它一切则水到渠成。
反复打磨核心,可以用爱因斯坦质能方程来描述:E=mc\^2。m是核心,c是大量重复练习,E是果。

\subsubsection{打法}

\subsubsection{资源}

整合资源:客户,钱,人脉

\subsubsection{激励}

\subsection{将略}

慎言!养成深沉厚重之心态。

\section{破}

破局,出奇制胜

\section{离}

从必然王国到自由王国

\chapter{神圣几何}

战略几何学,数、形、理的结合。

数据结构,树状、图论。

\section{点}

\subsection{单点突破}

道生一,一是战略,又是战术。

\subsection{圆点哲学}

根据地思维,进可攻退可守。

\section{线}

\subsection{双线法则}

守住底线,抓住关键。

\subsection{三才之道}

天地定位

\subsection{S曲线}

\section{面}

\subsection{三合}

易有太极,是生两仪。

参伍以变,错综其数。

\subsection{五行}

1+4=5,2+3=5。内涵圆点哲学、双线法则、三才之道、三合之道,四象,是多个模型的融合。

道天地将法。

\subsection{河洛}

\subsection{六图思维模型}

\subsection{战略罗盘}

\subsection{PDCA}

\hl{法尔科尼管理的方法}提及,方法只有一种,即笛卡尔的方法,形式化为PDCA。
运转PDCA的目的在于解决问题、固化优化等。

集中在PDCA上,以PDCA为主,其它模型作为分析工具。五比三更为丰富的内涵、兼容了一二三四。
\hl{成中英}就有尚五的倾向。

1+4=5,一是目标,居于圆心,起统摄作用。

这与中国传统的五行思维法很契合,黄帝内经就是五行认知体系。

河洛都是五行结构。

孙子计篇,开门见山就说:\hl{经之以五事,校之以计,而索其情}。道天地将法,五事是正。守正方能出奇。

以个人成长为实例,运用该模型,实现10X成长。

\subsection{SWOT}

\subsection{坐标}

\subsection{正弦曲线}

\subsection{椭圆}

两个焦点,圆点哲学与三合之道的融合。

\subsection{因果循环图}

\section{体}

\subsection{双环}

垂直正交双环,电磁感应

\section{五行}

\subsection{三合}

心道物

\subsection{三才}

天地人

融合三合三才,则天地定位、内圣外王(心物)。

\subsection{五事}

道天地将法,将居于中央,将者,心也。天地定位,左道右法(范蠡答越王问),精神内敛,一心所摄。

天地是台阶,更上一层楼。善守者,攻于九天之上;善守者,藏于九地之下。拳打十万八千遍,一层功夫一层理。
理者,超认知。

九层之台起于累土;千里之行始于足下。

升维思考、降维贯通。

\subsection{战略罗盘}

设计、执行

内外

\subsection{战略菱形}

无战略、悲人生:全局性、长期性、风险性

强烈的问题意识、彻底性、进取性

\subsection{双环}

两个正交维度,一横一纵,纵横交错

八面体

\subsection{PDCA}

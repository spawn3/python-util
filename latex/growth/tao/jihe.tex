\chapter{神圣几何}

战略几何学,数、形、理的结合。

数据结构,树状、图论。

\section{点}

\subsection{单点突破}

道生一,一是战略,又是战术。

\subsection{圆点哲学}

根据地思维,进可攻退可守。

\section{线}

\subsection{双线法则}

守住底线,抓住关键。

\subsection{三才之道}

天地定位

\subsection{S曲线}

\section{面}

\subsection{SWOT}

\subsection{三合之道}

易有太极,是生两仪。

参伍以变,错综其数。

\subsection{PDCA}

1+4=5,一是目标,居于圆心,起统摄作用。

\subsection{五行}

道天地将法。

\subsection{河洛}

\subsection{六图思维模型}

\subsection{坐标}

\subsection{正弦曲线}

\subsection{椭圆}

两个焦点,圆点哲学与三合之道的融合。

\subsection{因果循环图}

\section{体}

\subsection{双环}

垂直正交双环,电磁感应

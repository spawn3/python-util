\chapter{制胜之道}

制胜之道是在任一领域都可以胜出的道的探求,简称胜道,确立此点为今后数年的研究主题。
一切学问,都是为解决现实问题而来。知行并进是基本要求。

胜道离不开战略思维和系统思维的研修,是运用战略思维指导实践活动而心想事成,是目的和工具的统一。

胜道也即是幸福之道,胜利是为了幸福。

要紧紧把握住胜利这一根本目的和价值所在,此心不动,随机而动。

心能转物,即同如来。胜含义深广,人定胜天,是做一切事的目的所在。除了胜利,一无所求。
严肃认真地对待任何重要的事情,都要把终局的胜利作为考量的首要因素。

从这个角度来说,三合之道也是工具。

叔本华作为意志与表象的世界,尼采之全力意志,此所谓权力意志,即是求胜的意志,冲决罗网,百折不回。

金一南的系列著作,和君的三度修炼丛书,领域不同,具体目标也不同,但求胜制胜的意志则是一样的。
或为了国家的利益,或为了个人的幸福,修齐治平之道的根本,就是敢于胜利的意志。为了胜利,不择手段。

最激烈的对抗,发生在军事、经济、政治领域,拳击、中医、棋艺都是对抗的艺术。对抗的二元性是事物的本质。
问题的关键是:如何在对抗中,立于不败之地,而不失敌之败也?即是致人而不致于人。

回到自身、一切的答案来自于自身,自律是通向自由的桥梁。敬以直内,义以方外,敬义立而德不孤。
自身代表了能动性的一方面。

战略思维是制胜的核心。

\section{做好自己}

\begin{shadequote}

十年前我很关心全世界,结果我的日子过得非常艰难;五年前我很关心中国的命运,我也过得很艰难;
三年前我开始只关心公司,我的日子开始好起来。现在我只关心自己,越来越好。
\end{shadequote}

你若盛开,蝴蝶自来。吸引力、感召力不是靠宣传,而是用脚投票,自热而然。刻意去要求什么,反而毫无效果。

\section{三才之道}

\section{三合之道}

从三合之道说胜道,是致思原点。三合之道是易经和老子思想引出的一朵莲花。游心道物间,双环交融,终成正果。

三国演义中刘琦与诸葛亮谈到六合阵法,第一则是心法。

心的塑造能力,不能单纯地听天由命,高扬精神的主体性,斗志昂扬,化腐朽为神奇。不甘心做命运的奴隶,而是要命运的统帅。
这种主体性,努力地去探求制胜之道。心能转物,即同如来。

在三角形架构中,如果统一邵雍的心者太极也,大乘起信论的一心二门的命题?

依然把道放置在三角形顶点,心上通于道,而下及于物,诚神几曰圣人。

物外在于心,通过道而达一心物、合内外之境。心物为二,形而下者之谓器。心即理,致良知,良知并非现实的完成之物,而是生成的进化的状态。

\section{椭圆曲线}

心物分居椭圆的两焦点上,而道则是椭圆曲线形成的轨迹。道者,体常而应变,一隅不足以举之。

圆点哲学是椭圆曲线的极限形式,两点重合为一点。二合一,收敛为更为强大的力量。

尖椭圆

\section{PDCA}

参伍以变,错综其数。参即是上一节的三合之道,伍则是四象五行。

\section{战略罗盘}

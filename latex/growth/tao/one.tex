\chapter{一的哲学}

语出庄子天下篇:建之以常无有,主之以太一。

思方剑已就,天人琴鸣否?佩思方剑,抚天人琴

多年困穷,不得其要,皆由于一之不立。不恒其德,或承之羞。

中庸言诚,至于至诚不息,不息则久。天地之道,可一言而尽,其为物不二,则其生物不测。
精一之学,体用兼备。绝学无忧。身怀绝技方可臻于自由之境,止于至善。

天下之动,贞夫一也。

元亨利贞

古典哲学、现代科学有丰富的资源,更多的素材则来自活生生的现实,此乃无字天书。

商君书:
\begin{shadequote}

治国而能抟民力而壹民务者强;能事本而禁末者富。

故圣王之治也,慎法,察务,归心于壹而已矣。
\end{shadequote}

恒道,持续行动的复利。专注,聚焦,真积力久。

\section{体论}

\subsection{一是什么}

道生一,一生二,二生三,三生万物。万物负阴而抱阳,冲气以为和。最重要的是深刻理解道生一的含义,这是后续逻辑展开的第一站。
舍此,就不会有后续的发展。道如虚空,涵无量义。无极而太极,太极而无极。
非贵有,非贵无,而贵一,无余子也。

城市乡巴佬:坚守一事。

\subsection{为何要守一}

原子模型是物理学的伟大发现,对H原子模型的研究,引发了量子力学革命,物理学进入新纪元。
原子由原子核和核外电子构成,核外电子运行在不同的能级规定上。
开始,这个模型是从太阳系模型类比而得到的。
不同的是,对电子的运行轨道,采用了几率波的解释方式。

世界万事万物都是由原子构成。原子构成分子,分子构成高分子,进而形成丰富多彩的物质和生命现象。
原子也是可分的。采用一近似原子模型来展开论述。

小至原子,大至太阳系,甚至银河系,都是按核心加外围的方式进行组织。
大爆炸理论,黑洞奇点理论,莫不如此。

原子模型加上五行八卦,构成最主要的思维模型,太极,阴阳,三才,四象,五行,八卦,九宫等。
圆点哲学提供了极大的启示,更多理论和抒情,没有在应用层展开。

同心圆是原子模型之一义。

精于一事,可触类旁通

第一性原理

基本问题

\section{用论}

多方面的应用,最重要是用于当下和未来的进化,原子模型提供了强力的思维工具。
工欲善其事必先利其器,器已利,不用何为?

用此模型去融通一些重要的主题,如提问,刻意练习等。

刻意练习:分解成原子等基础动作,重复练习,注重反馈,持续迭代

学和问的关系:尊德性而道问学,致广大而尽精微,极高明而道中庸

在人生算法一文中,提出认知闭环,最小核心最大化,E=mc2等新解。最重要的是两阶段:找到核心,核心最大化。
这是即专注又开放的迭代过程。

引申到个人成长和组织发展上,也是如此。华为,BAT,都有着强大的核心,广泛的生态建设。

丁肇中

\subsection{华为}

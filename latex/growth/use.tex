\section{用}

\subsection{财务自由的价值}

财务自由不是终点,但是一个必要的里程碑。李笑来的这个观点和成长曲线令人印象深刻。
成长应该是加速的,过了某一临界点,就进入了自由境界。

在达到里程碑之前的策略,是果断投资自己,重视成长甚于重视金钱。
在达到里程碑以后的策略,依然如此。

李笑来的万能钥匙,是转念。其大刀是终极问题:更重要的是什么?最重要的是什么?
由此展开思维和行动的历程。

财务自由,作为今后五年的首要任务。由此展开自我进化的历程。

\begin{enumbox}
\item 如何做选择?增加必要的条件。
\item 如何开始?重视最少必要知识。
\item 最重要的是什么?元认知能力。
\item 如何才能融会贯通,连点成线。
\item 如何鉴别聪明人?拥有清晰必要准确的概念及其联系。
\item 创投的第一性原理是成长率。
\end{enumbox}

参考资料:
\begin{enumbox}
\item 穷爸爸,富爸爸
\item 管道的故事
\item 思考致富
\end{enumbox}

\subsection{控制二分法}

一寸光阴一寸金,寸金难买寸光阴。注意力和时间宝贵,把注意力更多放在可以控制和改变的事情上。

区分内在目标和外在目标,菩萨畏因,凡夫畏果。

\subsection{以进化论来指导修身养性}

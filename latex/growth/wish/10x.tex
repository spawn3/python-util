\chapter{Wish}

WOOP

\section{反思}

博而寡要,劳而无功,知行脱节,有术而无道,术也就不可靠了,以往解决问题陷入物的支离境地,这是阳明学要解决的大问题。

迷信书本,买了那么多书,根本消化不了,浪费了大量的时间和精力,劳而无功。
书要看,但并非必须,都是要牢牢把握住为学头脑处。
盲目地去学习,反不如静下心来,仔细规划。

花在抱怨的时间秀的时间太多,当回归到修身为本。\hl{涵养须用敬,进学则在致知}。
此二语是今后的指导思想,少说大言,收敛身心,但踏踏实实做去,真积力久,自然或跃在渊。

遥想十年前,身处天津,前途茫茫,没有着力点。现在情形则要好很多,正是用心用力之时,不可随意荒废。

知识体系只是构成了一个背景,能体现价值的地方,还是要看解决了什么大问题,大问题是成长之树的树根。
看看黎曼猜想引发的全球关注,即可知。战略思维的大开大合,离不开大问题的驱动。

围绕着要解决的大问题,如何打造专业知识体系?

\section{立志}

富有之谓大业,日新之谓盛德,生生之谓易。

10X成长

\section{工具}

\subsection{参伍以变}

双线法则、圆点哲学、参伍以变形成为主要的方法论。

\subsection{六时书}

\begin{tcolorbox}
这是唯一让我们能够实实在在得到智慧的好处的最方便可行的方法, 如果不去记录六时书, 
我们实在是 “如入宝山, 空手而归”,让人可惜得扼腕而叹。

六时书可帮助我们建立“追踪体系”——帮助我们训练意识,因为身为在地球上的人,
几乎每天都是负面念头多于正面念头的,也就是说这是自然人的本来性质,
但这一性质必定导致大多数人陷入负面事件越来越多导致产生更多负面念头的恶性循环里。

这个规律就如同地心引力一样,是一个定理——一个大多数人都逃不开的规律。
但是人类既然能够逃脱地心引力上天入地,同样的,我们也可以摆脱这个“负多于正”的规律,
通过六时书的帮助——训练我们的念头,使负面念头越来越少,正面念头越来越多,
成为一个觉悟的、开悟的人。到时候,生活中不论发生什么事,
我们都能以最正确的态度去对待,一切问题都将不是问题。
\end{tcolorbox}

六时书这种结构化的方式较好,考虑问题趋于全面。

示例: 专心致志,不抱怨

\begin{lstlisting}
+ 正
- 反
O 合
\end{lstlisting}

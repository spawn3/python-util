\chapter{算法}

\section{概述}

人生需要核心算法。核心算法解决人生会遇到的大问题,最重要的一个问题就是如何更好的成长。

\subsection{模型}

顶级思维模型。

处处可见圆点哲学的影子。矛盾分析法,双线法则,两眼论,
一阴一阳之谓道,万物负阴抱阳冲气以为和。

回到核心,从核心出发。找到核心的过程,一靠直觉,二靠试错,低成本地试错。

老子第十六章,义理丰富。

以正治国,以奇用兵,以无事取天下。此意渊深,可为座右铭。

孙子兵法提供了一套方案,孙正义有自己的归纳总结。我欲清溪寻鬼谷,不论礼乐但论兵。
兵者是死生存亡的大事,社会暗流涌动,在温泉面纱下,竞争不可谓不激烈。
想要活出自己的人生,不能不考虑更重要的维度。

孙陶然五行管理兵法。

达里奥原则,培根新工具,笛卡尔方法谈,斯宾诺莎伦理学都旨在解决人生算法问题。

开发人生算法,喻颖正做出了表率。按守破离的节奏,先对标,再突破。最小核心最大化,
先要感知自己的核心,通过反复练习,使之价值最大化。

高筑墙,广积粮,缓称王。

\section{战略要素}

\subsection{目标}

第一,列出最重要的五个目标。双列表,10/10/10原则。

事业有成是因,财务自由是果。找到自己的核心算法,其它一切则水到渠成。
反复打磨核心,可以用爱因斯坦质能方程来描述:E=mc\^2。m是核心,c是大量重复练习,E是果。

\subsection{资源}

整合资源:客户,钱,人脉,客户。

\section{将略}

慎言!养成深沉厚重之心态。


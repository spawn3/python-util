\chapter{SSAN性能问题}

\section{性能}

\subsection{线程结构}

\begin{enumbox}
\item 把slow操作从main线程里分离出来,避免堵塞main线程,改进系统的响应能力。
\item 双进程结构,保证osd意外退出后,按照一定策略进行处理
\item 后续可横向扩展main线程,以解决系统的单点性能瓶颈。
\end{enumbox}

\todo{双进程结构}构成监工/工作者结构,保证osd意外退出后,按照一定策略进行处理。

\todo{横向扩展main线程}进行负载均衡,以提升系统的整体调度能力

\subsection{内存分配}

\todo{hugepage和内存池}

\subsection{网络通信}

\begin{enumbox}
\item 多节点相同命令的并行化,比如修复过程中,primary向所有节点请求数据,若串行则耗时过长
\item 网络通信异步化(改动较大,优先级较低)
\end{enumbox}

\todo{并行、异步网络通信}

\subsection{磁盘访问}

\todo{直接管理裸盘}引入管理元数据层。

\subsection{数据组织方式}

\begin{enumbox}
\item 数据目录组织方式优化
\item object version
\end{enumbox}

\section{恢复下的性能}

\subsection{QoS}

已基本实现恢复QoS,控制恢复带宽。

\subsection{异步修复}

\todo{异步修复}磁盘不采用一致性hash,而是采用hash slot机制。
可参考Redis Cluster Slot、Ceph PG的设计思想,但引入了相当的复杂度。

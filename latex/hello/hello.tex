% -*- coding: UTF-8 -*-
% hello.tex

\documentclass[UTF8]{ctexart}

\usepackage{hyperref}
\hypersetup{pdftex,colorlinks=true,allcolors=blue}
\usepackage{hypcap}

\usepackage{color}
\usepackage[usenames, dvipsnames, svgnames, table]{xcolor}
% \pagecolor{gray}

\usepackage{makeidx}
\makeindex

\usepackage{amsmath}
\usepackage{mathtools}

\usepackage{listings}
\usepackage{multicol}
\usepackage{fancybox}
\usepackage{tcolorbox}
\usepackage{enumitem}

\title{LICH架构文档}
\author{董冠军}
\date{\today}

% \bibliographystyle{plain}
% \bibliography{math}

\begin{document}

\maketitle
\tableofcontents

\section{当前任务清单}

\subsection{关键特性和过程}

\begin{enumerate}
    \item 存储分层
    \item 保护模式
    \item flush and load
    \item recovery
\end{enumerate}

\subsection{快照}

\begin{enumerate}
    \item 支持60000+快照
    \item consistency group
\end{enumerate}

\subsection{一致性/正确性}

\begin{enumerate}
    \item 内置质量,各模块添加自校验机制,方便诊断数据正确性问题(assert + log + test)
    \item 加强断言:pre和post条件,变量变化规则,不变式,基本假设等
    \item 日志用tag/keywork和timeline,以便于跟踪一个对象的变化历史,用一个或多个维度贯穿起来,用于辅助诊断
    \item 增加CHUNK\_HISTORY,以时间线方式,跟踪记录CHUNK变化的生命周期
\end{enumerate}

\subsection{性能}

\begin{enumerate}
    \item \textcolor{red}{rcache 顺序IO随机化问题}
    \item wbuf 顺序IO随机化问题
    \item 系统启动时间
    \item GC策略和算法
    \item 预填充lich chunk
    \item 统计基础操作的开销,作为性能分析的基础
\end{enumerate}

\subsection{兼容性}

\begin{enumerate}
    \item 版本演进
\end{enumerate}

\subsection{Misc}

\begin{enumerate}
    \item VAAI [+xcopy]
    \item FC
    \item Remote copy
    \item SSD cache
    \item EC
\end{enumerate}


\section{基础理论}

Lich涉及许多底层理论和系统,包括并行计算和分布式系统,操作系统,文件系统,数据库系统,网络,还包括相应的底层硬件架构。
需要对数据结构和算法,有良好基础。所以,对基本问题和理论,要有清晰和深入的掌握,才能“运用之妙,存乎一心”。

按道法术器组织结构,写成大文章。

\section{对象模型}

pool, volume, snapshot等基本概念。

\subsection{Pool}
\subsection{Volume}
\subsection{Snapshot}
\subsection{Volume Group}

\section{数据管理}

\subsection{分配和回收}
\subsection{分布}
\subsection{平衡}
\subsection{恢复}

\section{优化项}

\subsection{时间优化}

\begin{itemize}
    \item localize
    \item auto tier
    \item ssd cache
\end{itemize}

\subsection{空间优化}

\begin{itemize}
    \item 精简配置 (Thin provisioning)
    \item EC
    \item Dedup
    \item Compress
\end{itemize}

\section{企业级特性}

\subsection{Security}
\subsection{QOS}
\subsection{Quota}
\subsection{Multipath}
\subsection{DR}
\subsection{CDP}

\section{实现相关}

\begin{itemize}
    \item polling
    \item coroutine and scheduler
    \item kernel bypass
    \item mbuffer
\end{itemize}

\section{硬件架构}

NVMe

RDMA/DPDK/SPDK

AFA


\section{LSV}

现有Lich raw卷,存在性能问题,COW快照也不便于扩展。所以实现了Log structured Volume,
转化随机IO为顺序IO,基于其上,实现了ROW快照。

特别要注意的是,实现中应着力避免顺序IO随机化,会引起IO放大,从而极大地降低性能。


\subsection{Volume}

Volume模块负责空间管理。提供malloc/free接口,也可批量分配和回收。采用bitmap和free list多种管理方式。
freelist充当分配缓冲区的角色,可持久化,也可不持久化。

lsv-lich raw-disk的chunk空间存在两级映射关系,会影响到读写性能。

底层空间宜按固定大小的段来组织。每个段空间管理的开销是固定的。
目前支持两级存储分级:
\begin{tcolorbox}
\begin{itemize}
    \item 0:ssd
    \item 1:hdd
\end{itemize}
\end{tcolorbox}

\subsection{Bitmap}

Bitmap负责元数据管理,有两层:L1和L2,按类似页表的方式组织。和Log层数据一起,构成三层。

L1是Bitmap的头部,大小固定,属于卷或快照私有。L2按需分配,在快照之间共享。在Clone的情况下,会涉及跨卷读。

通过Bitmap层,支持快照的全部特性,多个快照构成快照树。

\subsection{Log}

在Volume模块之上做了简单封装,表示卷的数据,Bitmap表示卷的元数据。在覆盖更新的情况下,Bitmap指向新的数据页,
导致原来的数据页失效,可以回收。在有快照的情况下,会变得较为复杂。

Log模块无需要持久化的信息。

\subsection{WBuf}

Wbuf有两个序列:WAL和Wbuf的提交序。在wbuf中读出的最新数据和提交后通过bitmap+log读取的数据,应该一致。

IO内,LBA不同,无冲突,页序;IO间,LBA可能相同,有冲突,需要串行化。

\subsection{RCache}

多级缓存机制,需要注意针对多种读场景进行优化,如顺序读。因为经过虚拟页表映射,虚拟地址空间和物理地址空间,顺序可能是交叉的。
应着力避免出现顺序变随机导致读放大的情况。

预读很重要,也比较困难,需要构建学习模型。

\subsection{GC}

log功能单一化,gc模块独立出来。gc要解决的问题有二:
\begin{enumerate}
    \item 跟踪所有log
    \item 在所有log中,根据一定策略(qos),选择回收价值最大者进行回收
\end{enumerate}

目前的实现,是局域的解,而不是全局最优解,是bottom-up的分代垃圾回收器。可增量并行执行,与前台赋值器需要同步机制。
回收器和赋值器需要读写barrier。

\subsection{Recovery}

正常关机的情况下,各个模块会flush必要的数据,下次启动的时候,load出来即可。

异常关机的情况下,各个模块没有机会flush数据,导致丢失部分内存状态信息。
这样,在下次启动的时候,需要执行恢复过程。

需要flush数据的模块有:
\begin{itemize}
    \item Wbuf
    \item GC
    \item Volume
\end{itemize}

提出几个问题:
\begin{enumerate}
    \item 正常关机时,需要flush什么信息?
    \item 恢复过程,从X恢复出Y,X是什么?Y是什么?(X是日志,Y是最新状态)
    \item 怎么理解提交等基础操作?
    \item 恢复的性能如何?如何通过检查点机制改善恢复性能?
\end{enumerate}

针对以上问题,每个模块的恢复机制有所不同,但分析方法具有通用性。

\subsubsection{Volume Recovery}

 $U = (A - B) + C + D$

tail标记了可见空间,可见空间=已分配+可分配(free list)。free list组织成内存和磁盘两部分。flush时,需要持久化freelist的内存部分。

在调用malloc和free接口的时候,会同步更新用于空间管理的bitmap。为1的为已分配,为0的为可分配,这个关系总成立。

为了支持批量malloc和free接口,引入dirty page bitmap,类似于GC中提到的卡表,可以实现\textcolor{red}{多次更新,一次提交}的设计模式。

主要操作:
\begin{tcolorbox}
\begin{itemize}
    \item malloc操作:依次从C,D,U里取可用chunk。
    \item free操作:把释放的chunk放入C,如果C满,则转化为D。
\end{itemize}
\end{tcolorbox}

这里的提交操作可以理解为:C转化为D的过程,并没有记录检查点。
所以恢复操作,要全扫描bitmap,从bitmap重建C和D。

\subsubsection{GC Recovery}

GC recovery过程可以理解为:从gc bitmap重建内存状态。

所有的log,分为两部分:old storage和bitmap。bitmap相当于journalling。进入check queue的logctrl,先登记到bitmap。
在提交时,即从heap移入old storage时,清除/注销相应的bitmap项。

\subsubsection{Wbuf Recovery}

谁充当了日志的角色?在wbuf模块很明确,有专门的WAL。写入阶段登记,commit阶段回收。

\subsection{LSV测试}

LSV(\textcolor{red}{Log Structured Volume})基于Lich原生卷,实现了日志结构的卷格式,支持快照的各种操作。

相对于Lich原生卷,LSV有几点优势:
\begin{tcolorbox}
    \begin{itemize}
        \item 转化随机IO为顺序IO,混合存储情况下有更高性能
        \item 实现为ROW快照,zero-copy快照,\textcolor{red}{支持快照树}
    \end{itemize}
\end{tcolorbox}

LSV的关键过程:
\begin{tcolorbox}
    \begin{description}[style=nextline]
        \item [写] 写入wal和wbuf后,即可返回。wbuf积聚到1M时,提交log+bitmap后台异步任务。
        \item [读] 从wbuf读取最新数据,如果没有命中,则依次从rcache,bitmap+log读取。
        \item [GC] 垃圾回收,后台异步任务,按一定策略,回收无效页。
        \item [重启] 分两种情况,正常和异常情况。正常情况下会刷新内存状态,重启时直接加载即可;异常情况下,进入recovery过程。
    \end{description}
\end{tcolorbox}


LSV测试,主要分为功能,正确性和性能几个方面,\textcolor{red}{正确性和性能按标准测试用例}执行即可。下面列出一期测试计划。

\subsubsection{GIT分支}

lsv\_pipeline

\subsubsection{特性}

%\begin{tcolorbox}
\begin{lstlisting}[language=bash,frame=single]
# 创建LSV卷:
lichbd vol create p1/v1 --size 100Gi -F lsv -p iscsi

# 快照功能,与原来一样,部分命令示例:
lichbd snap create p1/v1@snap1 -p iscsi
lichbd snap create p1/v1@snap2 -p iscsi
lichbd snap ls p1/v1 -p iscsi

# 暂不支持flat操作

\end{lstlisting}
%\end{tcolorbox}

\subsubsection{性能/正确性测试清单}

\begin{itemize}
    \item 与lich原生卷全面的性能对比
    \item 资源利用率(包括磁盘,内存)
    \item 系统启动时间
    \item 重启系统的恢复过程
    \item 存储分层
    \item 扩展到多卷
\end{itemize}

\subsubsection{注意事项}

\begin{itemize}
    \item /opt/fusionstack/log/lich.log (echo 5 > /dev/shm/lich4/msgctl/level)
\end{itemize}


\section{FAQ}

\subsection{IOMeter测试,256K,Lich顺序和随机IO性能差别大}

\end{document}

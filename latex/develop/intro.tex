\chapter{导言}

构造和解释框架

进程,通信,命名,同步,复制/缓存,一致性,容错,安全

CAP

事务管理,并发和容错

团队

\section{人}

后续LICH代码维护和开发模式,可称之为\hl{“分进合击”}。
所谓分进,指的是维护工作分层分模块,责任到人,平时通过串讲讨论等多种沟通形式,进行知识总结/分享;
所谓合击,指的是开发新功能时,组成多人小组,联合设计/开发,提高协同能力,避免孤军作战。

从组织结构上来说,进一步分为开发组和架构组,在共同目标之下,承担职责,协同作战。

自我驱动,开放进取,集中优势,协同进化

元数据和IO路径

\subsection{模块及其负责人}

产品,场景及其生态

\begin{longtable}{|c|c|c|}
\caption{LICH模块和责任人}\\
\hline
    & 模块 & 责任人\\
\hline\hline
\endhead

\multirow{2}*{管理层} & Monitoring & 王少飞 \\
                      & Alert &   \\
\hline

\multirow{2}*{应用层} & OpenStack & \\
                      & VAAI &   \\
\hline

\multirow{2}*{接口层} & ISCSI &  \\
                      & FC    &  \\
                      & RDMA/SPDK &  \\
\hline

\multirow{5}*{特性层} & Recovery &  \\
                      & Balance &   \\
                      & Thin Provisioning & \\
                      & QoS &   \\
                      & Cache* &  \\
                      & Tier &  \\
                      & EC* &  \\
                      & Backup &  \\
                      & DR &  \\
                      & Dedup &  \\
                      & Compression &  \\
\hline

\multirow{7}*{引擎层} & Resource(CPU/Memory/Disk/Network) &  \\
                      & Cluster(Pool/Volume/Snapshot) & \\
                      & Metadata Management &   \\
                      & RAS(Reliability/Availability/Serviceability) &  \\
                      & Performance &   \\
                      & Extentablity &   \\
\hline

\multirow{4}*{标准库} & Data Structures &  \\
                      & Algorithm &   \\
                      & ... &   \\
\hline
\end{longtable}

\section{事}

% 流程,方法和工具

% 领域模型

% 基础库

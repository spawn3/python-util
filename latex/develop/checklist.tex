\chapter{LICH系统故障排查检查表}

\section{故障诊断}

故障诊断要结合集群和节点状态,core,日志,硬件,数据等因素。

\subsection{检查集群健康状态}

\begin{lstlisting}
# 节点列表,检查latency,uptime, nid等
lich list -v

# 节点,存储池和容量
lich stat

# 检查恢复状态
lich health
\end{lstlisting}

\subsection{检查ETCD}

\begin{lstlisting}
# etcd状态
etcdctl ls -r

# 版本
\end{lstlisting}

\subsection{检查系统时间}

\subsection{检查lichd版本}

\subsection{检查卷控制器的平衡性}

\begin{lstlisting}
# 节点间平衡
lich.balance --scan
lich.balance --balance

# 节点内core平衡
\end{lstlisting}

\subsection{查看CORE}

\begin{lstlisting}
./lscore.sh
\end{lstlisting}

\subsection{查看日志}

必要时,开启backtrace

必要时,开启DBUG日志等级

\subsection{检查硬件和操作系统状态}

CPU,内存,磁盘和网络

\subsubsection{磁盘分区满}

\lstinputlisting{code/du.sh}

%\begin{verbatim}
%TMP='/';for i in `/bin/ls $TMP`; do du -sh $TMP/$i; done
%\end{verbatim}

\subsection{检查数据一致性}

节点内sqlite/disk bitmap一致性检查

\begin{lstlisting}
python diskcheck.py --scan
\end{lstlisting}

chunk副本一致性检查

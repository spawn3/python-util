\chapter{软件架构}

\section{ETCD}

\section{Admin Node}

问题集:
\begin{compactenum}
\item admin承担什么职责?
\item 怎么选举出admin节点?
\item admin是否成为系统瓶颈?可扩展性如何?
\item 什么会导致admin发生切换?admin切换会影响到什么?
\end{compactenum}

管理vip。

nodetable: 每个存储池下有若干节点。节点状态有可写,只读,删除中等。

diskmap:admin维护每个节点的nodetable结构和每个pool的diskmap结构。
每个节点有存储池的磁盘集合,上报到admin后,构成存储池,故障域,节点(total,used)的层次结构。

节点下线,nodetable和diskmap如何随之而变化?参考nodetable\_offline函数。
检查到网络错误,会触发该过程。

lease admin: controller与admin获取lease,并周期性renew。

管理后台任务

\section{Normal Node}

normal以心跳方式向admin注册,并周期性地更新节点信息,代码在node\_srv.c。

可以成为各种controller。为了保证controller的唯一性,采用了lease机制。
一个卷的所有操作都经过controller执行,包括元数据操作和IO操作。

在iscsi层,通过core\_attach关联卷到core上,进入scheduler,所有IO消息都通过corerpc进行。

真实的io采用aio机制,由外部线程执行。sqlite,rpc,disk allocator都采用了异步模式。
core thread和外部线程通过队列对象通信。总而言之,core thread不能执行同步或堵塞代码,
如write,fsync等,否则会严重影响调度器性能。(倾向于不区分本地和远程,利用rpc timeout特性)

远程更新元数据,是通过rpc机制,同步写入;远程更新raw数据,采用了corerpc机制。

元数据的分配和回收,开销大,且易错。需要多次io,
如果中间插入故障点,可能会导致数据不一致。

本地空间管理:采用sqlite管理本地磁盘空间。replica cache占用大量内存。

tier和cache属于节点内特性,另外,全局cache也是需要的。

writeback and writethrough

\hl{节点两两之间的连接管理}:net\_table。连接断开后调用注册的reset handler。
连接之上注册有心跳过程,可以及时更新连接状态,提高感知网络故障的及时性。
% 读副本时利用这个信息,进行负载均衡处理。

rpc构建与net table之上,加入了request/reply相关逻辑。

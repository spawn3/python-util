\chapter{灾备}

目前收集到的信息:

\begin{compactenum}
\item 方案1:forward 控制流和数据流,故障情况下异步传输
\item 方案2:基于libiscsi和iscsi协议(张胜玉提供,以前做过相关功能)
\item 方案3:老王提供的方案,先写入远程站点,故障情况下,从本地站点全量flush到远程站点
\end{compactenum}

中科院魏征当时实现了一部分,没有完成,类似方案1,考虑这部分代码是否可以复用。

主要的困难:
\begin{compactenum}
\item 选择网络拓扑,传输协议,复制策略
\item 控制流/数据流的保序
\item 故障处理
\item failover
\item 传输性能
\end{compactenum}

\begin{lstlisting}
我解释下iscsi的优势:
1. 我们系统本身就支持了iscsi server,因此无需再去实现一套server,实现一套server需要额外的代码,关键是对应的管理功能
2. 我们可以利用scsi协议预留给厂商的指令,封装一些控制指令(比如建立快照等)很好保证时序。
3. 有利于将来扩展,相比以上两点,这个重要性更高,因为我们的系统对外不管是iscsi,iser还是光纤,
最后里面都是要走scsi协议,一些scsi协议需要redirect到远端。如果不用这个方案,需要每个特殊命令单独实现一个同步指令。

控制流/数据流的保序,iscsi本身有时序处理
- 故障处理   - failover, 发生中断要进行本地记录增量日志,待恢复后将数据刷过去。
- 传输性能, iscsi协议本身无此瓶颈。
\end{lstlisting}

业务流量和复制流量隔离

多复制流量多通路

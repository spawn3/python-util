\chapter{问题集}

任务管理:
\begin{lstlisting}
每个节点启动若干core线程,并监听17902端口。

在iscsi层,core\_attach加入core的poll模式。即进入core线程进行处理。
通过corerpc与不同节点上的core进行通信。corerpc会把相关请求发送到该节点具有相同core hash的core上。
两个节点上两个core有sock连接。

对每一连接,按core hash到相应的core(包括corenet和corenet\_mapping)

每个core监听独立的epool fd,把该连接socket关联到对应core的epoll fd上即可。

rpc的worker\_t和corerpc的core\_t具有相似性,都包括epoll fd和scheduler。
coreprc通过corenet管理epoll fd,用来监视跟踪的sd的io就绪状态。
accept建立连接后,即把相应sd加入epoll的兴趣列表。
\end{lstlisting}

内存:
\begin{lstlisting}
hugepage

mbuffer
\end{lstlisting}

磁盘:
\begin{lstlisting}
\end{lstlisting}

网络层:
\begin{lstlisting}
为什么有多种RPC机制?

corenet是各节点上相同core hash的core构成的网络?

为什么需要vnode\_location?与md\_map\_getsrv不是相同的吗?

命令行工具通过rpc与lichd通信,不能位于fusionstor集群之外。命令行工具进程具有什么nid?
命令行进程具有空的nid,lichbd不依赖于nid就可以进行通信。
集群外的情况没考虑,涉及版本和通信的二进制兼容性问题。

netable什么用?管理节点间的网络连接,同时收集latenccy信息,用于节点间的负载均衡。

如何采集多副本读负载均衡所需的信息?

rpc table用于匹配request和reply,如果reply找不到相应的request,则丢弃。

net\_islocal会被取代,本地通信也用rpc机制,主要是因为rpc有timeout,可以发现一些异常情况。

heartbeat是如何工作的?频率如何?在建立连接的时候,启动hb timer。

minirpc采用UDP,进行一些简单的通信,corerpc用于io数据流量,rpc用于一般的集群通信元数据流量。

\end{lstlisting}

分布式:
\begin{lstlisting}
lease能用etcd替代吗?

VC什么时候会发生切换?PC呢?VC切换后iscsi session如何变化?切换需要多久才能完成?

admin何时发生切换?会导致所有lease失效?
\end{lstlisting}

元数据:
\begin{lstlisting}
谁负责创建存储池的root目录?/目录下有iscsi,nbd等二级目录。

区分存储池和目录,区分内存结构和磁盘结构。
\end{lstlisting}

ssd cache
\begin{lstlisting}
cache容量不计入有效容量,tier则会计入。

模拟cache的方法,导致ssd盘首先被消耗完。

界面上没有节点级的磁盘使用量和使用率。
\end{lstlisting}

内存cache
\begin{lstlisting}
controller cache
replica cache

以及各种mapping关系,如net table,corerpc mapping等。
\end{lstlisting}

架构方面:
\begin{lstlisting}

IO中断与代码的什么方面有关?如何优化?

客户端任务派遣模型: 1对1,1对m。

是否需要引入MQ?哪些地方用得着?

AFA是什么?RDMA,iSER,SPDK怎么理解?

与ceph的全方位对比?

如何提高系统的可视性和可诊断性?
\end{lstlisting}

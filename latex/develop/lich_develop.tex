% -*- coding: UTF-8 -*-
% lich_checklist_2017Q4.tex

\documentclass[UTF8,oneside]{ctexbook}

% \usepackage{xeCJK}
\usepackage[utf8]{inputenc}

% load paralist before enumitem
\usepackage{paralist}

\usepackage{hyperref}
\hypersetup{pdftex,colorlinks=true,allcolors=blue}
\usepackage{hypcap}

\usepackage{color}
\usepackage[usenames, dvipsnames, svgnames, table]{xcolor}
% \pagecolor{gray}

\usepackage{makeidx}
\makeindex

\usepackage{amsmath}
\usepackage{mathtools}

\usepackage{listings}
\usepackage{multicol}
\usepackage{fancybox}
\usepackage{tcolorbox}
\usepackage{enumitem}
\usepackage{multirow}
\usepackage{longtable}

\usepackage{indentfirst}

\lstset{%
    %alsolanguage=Java,
    %language={[ISO]C++}, %language为,还有{[Visual]C++}
    %alsolanguage=[ANSI]C, %可以添加很多个alsolanguage,如alsolanguage=matlab,alsolanguage=VHDL等
    %alsolanguage=tcl,
    %alsolanguage=XML,
    %alsolanguage=bash,
    tabsize=4, %
    frame=shadowbox, %把代码用带有阴影的框圈起来
    commentstyle=\color{red!50!green!50!blue!50},%浅灰色的注释
    rulesepcolor=\color{red!20!green!20!blue!20},%代码块边框为淡青色
    keywordstyle=\color{blue!90}\bfseries, %代码关键字的颜色为蓝色,粗体
    showstringspaces=false,%不显示代码字符串中间的空格标记
    stringstyle=\ttfamily, % 代码字符串的特殊格式
    keepspaces=true, %
    breakindent=22pt, %
    numbers=left,%左侧显示行号 往左靠,还可以为right,或none,即不加行号
    stepnumber=1,%若设置为2,则显示行号为1,3,5,即stepnumber为公差,默认stepnumber=1
    %numberstyle=\tiny, %行号字体用小号
    numberstyle={\color[RGB]{0,192,192}\tiny} ,%设置行号的大小,大小有tiny,scriptsize,footnotesize,small,normalsize,large等
    numbersep=8pt, %设置行号与代码的距离,默认是5pt
    basicstyle=\footnotesize, % 这句设置代码的大小
    showspaces=false, %
    flexiblecolumns=true, %
    breaklines=true, %对过长的代码自动换行
    breakautoindent=true,%
    breakindent=4em, %
    escapebegin=\begin{CJK*}{GBK}{hei},escapeend=\end{CJK*},
    aboveskip=1em, %代码块边框
    tabsize=2,
    showstringspaces=false, %不显示字符串中的空格
    backgroundcolor=\color[RGB]{245,245,244}, %代码背景色
    %backgroundcolor=\color[rgb]{0.91,0.91,0.91} %添加背景色
    escapeinside=``, %在``里显示中文
    %% added by http://bbs.ctex.org/viewthread.php?tid=53451
    fontadjust,
    captionpos=t,
    framextopmargin=2pt,framexbottommargin=2pt,abovecaptionskip=-3pt,belowcaptionskip=3pt,
    xleftmargin=4em,xrightmargin=4em, % 设定listing左右的空白
    texcl=true,
    % 设定中文冲突,断行,列模式,数学环境输入,listing数字的样式
    extendedchars=false,columns=flexible,mathescape=false
    % numbersep=-1em
}

\newenvironment{enumbox}[0]{
    \begin{tcolorbox}
    \begin{compactenum}
} {
    \end{compactenum}
    \end{tcolorbox}
}

\newenvironment{itembox}[0]{
    \begin{tcolorbox}
    \begin{compactitem}
} {
    \end{compactitem}
    \end{tcolorbox}
}

% table
\setlength{\arrayrulewidth}{1pt}
\setlength{\tabcolsep}{16pt}
%\renewcommand{\arraystretch}{1.5}
\newcolumntype{s}{>{\columncolor[HTML]{AAACED}} p{3cm}}

\arrayrulecolor[HTML]{DB5800}

\usepackage{tikz,mathpazo}
\usetikzlibrary{positioning, fit, matrix, shapes, arrows, chains, trees, arrows.meta}

% \bibliographystyle{plain}
% \bibliography{math}

\tikzset{%
  >={Latex[width=2mm,length=2mm]},
  % Specifications for style of nodes:
            base/.style = {rectangle, rounded corners, draw=black,
                           minimum width=4cm, minimum height=1cm,
                           text centered, font=\sffamily},
  activityStarts/.style = {base, fill=blue!30},
       startstop/.style = {base, fill=red!30},
    activityRuns/.style = {base, fill=green!30},
         process/.style = {base, minimum width=2.5cm, fill=orange!15,
                           font=\ttfamily},
}

% 摘录
\usepackage{verbatim}
\usepackage{libertine}
\usepackage{graphicx}
\usepackage{framed}

\newcommand*\openquote{\makebox(25,-22){\scalebox{5}{``}}}
\newcommand*\closequote{\makebox(25,-22){\scalebox{5}{''}}}
\colorlet{shadecolor}{Azure}

\makeatletter
\newif\if@right
\def\shadequote{\@righttrue\shadequote@i}
\def\shadequote@i{\begin{snugshade}\begin{quote}\openquote}
\def\endshadequote{%
\if@right\hfill\fi\closequote\end{quote}\end{snugshade}}
\@namedef{shadequote*}{\@rightfalse\shadequote@i}
\@namedef{endshadequote*}{\endshadequote}
\makeatother

\usepackage[normalem]{ulem}

\newcommand{\hl}{\bgroup\markoverwith
  {\textcolor{yellow}{\rule[-.5ex]{2pt}{2.5ex}}}\ULon}

%\usepackage{soul}

%\newcommand{\hlc}[2][yellow]{{%
%    \colorlet{foo}{#1}%
%    \sethlcolor{foo}\hl{#2}}%
%}

% todonode
\usepackage{lipsum}                     % Dummytext
\usepackage{xargs}                      % Use more than one optional parameter in a new commands
% 
\usepackage[colorinlistoftodos,prependcaption,textsize=tiny]{todonotes}
\newcommandx{\unsure}[2][1=]{\todo[linecolor=red,backgroundcolor=red!25,bordercolor=red,#1]{#2}}
\newcommandx{\change}[2][1=]{\todo[linecolor=blue,backgroundcolor=blue!25,bordercolor=blue,#1]{#2}}
\newcommandx{\info}[2][1=]{\todo[linecolor=OliveGreen,backgroundcolor=OliveGreen!25,bordercolor=OliveGreen,#1]{#2}}
\newcommandx{\improvement}[2][1=]{\todo[linecolor=Plum,backgroundcolor=Plum!25,bordercolor=Plum,#1]{#2}}
\newcommandx{\thiswillnotshow}[2][1=]{\todo[disable,#1]{#2}}
%

\usepackage[simplified]{pgf-umlcd}



\title{LICH开发者指南}
\author{董冠军}
\date{\today}

\begin{document}

\maketitle
\tableofcontents

\chapter{检查清单}

\section{检查清单checklist}

先宏观,后微观,致广大而尽精微

\begin{compactenum}
\item 集群健康情况
\item 硬件
    \begin{compactenum}
    \item 磁盘
    \item 网络
    \item 内存
    \item CPU
    \item 操作系统
    \end{compactenum}
\item 服务
    \begin{compactenum}
    \item 后台任务,包括恢复,删除,快照后台任务等
    \item 日志
    \item core
    \end{compactenum}
\item 数据一致性检查
\end{compactenum}

\chapter{技术}

在一个大的技术背景下,看分布式存储系统的理论和实战。

操作系统
\begin{enumbox}
\item Linux
\end{enumbox}

存储系统
\begin{enumbox}
\item FusionStor
\item Ceph
\end{enumbox}

共识
\begin{enumbox}
\item etcd
\item zookeeper
% \item Ceph Monitor
\end{enumbox}

文件系统
\begin{enumbox}
\item ext4 
\item xfs
\item fuse
\item CephFS
\item GlusterFS
\item Lustre
\item HDFS
\end{enumbox}

NoSQL
\begin{enumbox}
\item RocksDB/LevelDB
\item MongoDB
\item Redis
\item Hadoop
\end{enumbox}

NewSQL
\begin{enumbox}
\item OceanBase
\item PingCAP
\end{enumbox}

WebServer
\begin{enumbox}
\item Nginx
\item ElasticSearch ELK
\end{enumbox}

\chapter{Tools}

\section{Deployment}

\subsection{ntpdate}

\mygraphics{../imgs/tool/ntp-date.png}

\subsection{ansible}

\section{Development}

\subsection{cmake}

\mygraphics{../imgs/tool/cmake-link-static.png}

生成静态库
\begin{myeasylist}{itemize}
& SHARED  -> STATIC
& LIBRARY -> ARCHIVE
\end{myeasylist}

\subsection{gdb}

\begin{myeasylist}{itemize}
& ~/.gdbinit
& info registers
& info sharedlibrary
& gdb -p
\end{myeasylist}

gdb -p发现了mbuffer\_writefile进入死循环,原因是count==0。

猜想是重入了一个锁。

\subsection{debug}

trace msgid来跟踪消息流。

\subsection{wireshark}

\subsection{SoftRoce}

spdk/scripts/setup.sh

\section{Test}

\subsection{fio}

\subsection{spdk/perf}

\subsection{hazard}

\mygraphics{../imgs/tool/hazard-1.jpeg}
\mygraphics{../imgs/tool/hazard-2.jpeg}
\mygraphics{../imgs/tool/hazard-3.jpeg}

\chapter{模型}

\begin{tikzpicture}[show background grid]
    \begin{class}{Disk}{6, 0}
    \end{class}
    \begin{class}{Storage Pool}{6, 2}
    \end{class}
    \begin{class}{Volume}{6, 4}
    \end{class}
    \begin{class}{Host}{6, 6}
    \end{class}
    \begin{class}{Cluster}{0, 2}
    \end{class}
    \begin{class}{Snapshot}{0, 4}
    \end{class}

    \composition{Cluster}{pools}{1..*}{Storage Pool}
    \composition{Storage Pool}{disks}{1..*}{Disk}
    \composition{Storage Pool}{volumes}{1..*}{Volume}
    \composition{Volume}{mapping}{*..*}{Host}
    \composition{Volume}{snapshots}{1..*}{Snapshot}
\end{tikzpicture}

\section{Cluster}

整体

\section{Protection Domain}

把物理节点划分为不同的保护域,一个卷的所有数据只出现在一个保护域内。卷可以跨保护域进行复制和迁移。

默认一个,包括所有节点。

% 保护域是物理节点的划分,存储池是存储介质的划分。每块盘只能出现在一个存储池里。

\section{Pool}

逻辑容器

\section{Storage Pool}

与存储池有什么同和异?存储池可以看做关联了磁盘的pool,可以看做pool的子类。

属性:
\begin{enumbox}
    \item 磁盘列表
    \item 定义精简池
    \item 存储池上可以指定卷的副本数
    \item \hl{有足够的故障域,且不同故障域配置一致的资源量}
\end{enumbox}

操作:
\begin{enumbox}
    \item 创建
    \item 删除
    \item 扩展(添加磁盘到\hl{已存在的存储池},该映射关系持久化到本地,同步到admin节点)
    \item 缩容(从存储池中移除磁盘,引发数据重建过程)
    \item \hl{自动或手动按磁盘速率进行存储池分级划分}
    \item 不同存储池之间,卷的复制
    \item 不同存储池之间,卷的迁移,可在线或离线
    \item 存储池级别的统计信息
\end{enumbox}

% 存储池是disk的集合,与节点无关。但disk所在的节点构成存储池的节点列表,不同存储池的节点可能覆盖。

存储池下,可以创建volume。没有关联磁盘的存储池,不能创建卷。

\hl{chunkid到磁盘物理位置有两级映射:chunk的副本节点列表,节点内chunkid到物理地址的映射}。

在为卷分配chunk的时候,需要确定各个副本的物理存储位置。当前实现是返回不同副本的节点列表。
如果指定了存储池,就需要在存储池所在的节点范围内进行分配。同时要满足故障域和数据均衡规则。

\begin{tcolorbox}
移动采集中存储池要求,相比于目前的逻辑pool,更多是一种设计上的退步。
存储虚拟化的目标,是物理位置无关。我们可以基于逻辑容器,实现基于策略的管理。
所以,\hl{从实现层面,要保留当前pool的功能,按照系统配置确定pool的类型}。
\end{tcolorbox}

% 存储池内,要满足故障域规则(\ref{rule:faultset})

\section{Fault Set}

故障域有粒度之分,如磁盘,节点,机架,机柜,数据中心。

存储池内,要满足故障域规则:一个chunk的不同副本,分布在不同的故障域内。\label{rule:faultset}

在初次分配,再平衡和恢复等过程中,都需遵循这些规则。

\section{Volume}

属性:

操作:
\begin{compactenum}
    \item rename
    \item resize \info{在线扩容}
    \item mv
    \item copy \change{全量拷贝/增量拷贝} \change{跨存储池拷贝} % change不能出现在box里
\end{compactenum}

\section{Snapshot}

snapshot隶属于卷,无卷则无快照,快照组织成快照树,其中有且只有一个快照是可写快照,即卷的写入点。

\section{Mapping}

数据隔离/ACL,数据保护

卷对主机的可见性。一个卷只有映射给了某主机,才可以被该主机访问。

采用白名单机制,但是,cinder需要无验证地访问一个pool。

在建立mapping时,需要host信息。如果有host列表,则可让管理系统去选择。

iscsi采用chap认证。

每个pool上,需要一个属性,表明是开放的,还是封闭的访问模式。如果是封闭的,检查白名单进行认证。


\section{Consistency Group}

一致性卷组

\begin{shadequote}
Consistency Groups could be useful for Data Protection (snapshots, backups) and
Remote Replication (Mirroring).

The Mirroring support will allow to setup mirroring of multiple volumes in the
same consistency group (i.e. attaching multiple RBD images to the same journal
to ensure consistent replay).

There is already an interest to implement this functionality as a part Mirroring feature:
http://tracker.ceph.com/issues/13295

The snapshot support will allow snapshots of multiple volumes in the same
consistency group to be taken at the same point-in-time to ensure data
consistency.
\end{shadequote}

\chapter{核心数据结构}

\section{ETCD}

集群状态

\section{元数据管理}

对象标识和寻址机制

\section{SQLITE}

异步化

\section{DISK BITMAP}

空间分配器

\section{LEASE}

\section{CLOCK}

副本数据一致性

\section{MSGQUEUE}

离线消息处理

异步化,是否可提取出异步化框架?

\section{控制器缓存}

采用引用计数技术。

控制器切换和平衡

\section{副本缓存}

\section{扩展属性}

\section{快照}

\section{克隆卷}

\section{CORE AND Scheduler}

每个core有独立scheduler。为了与其它节点上的core进行通信,抽象出了corenet的概念。

实体标识和命名:hostname,nid,core hash,sockid。

RPC有发送端和接收端。

\chapter{关键过程}

过程分析分两种:正常情况和异常情况。特别是故障情况下,过程之间存在广泛的交互作用,变得更为复杂。

过程需要具备一些重要属性,如事务ACID,safety和liveness等,要具体情况具体分析。
值得注意的是,若干过程需要实现为可重入的。

其它注意事项:
\begin{compactenum}
\item 必须正确处理过程的返回值。
\item 一定要检查加锁的返回值,返回值失败,不能unlock,从而保证加锁解锁的对称性
\item 资源的分配和释放,使用goto方法,用stack方式管理
\item 过程运行在协程内,还是协程外?
\item 变量作用域,在作用域以外访问变量
\item 过程是否并发安全?
\item 过程执行时间是否过长,且中间无中断?比如有大循环,有同步IO操作等
\end{compactenum}

实体有复合和简单之分。简单实体的基本操作包括增删改查,复合实体在此基础上多出了子节点的相关操作。

按一颗树来组织,简单实体是叶子节点,复合实体是中间节点或根节点。

if-then/what-if假设分析:
\begin{compactenum}
\item 资源管理的对称性(分配/释放, 资源包括fd,lock,malloc等)
\item 如果此处发生故障,会如何?容错
\item 如果有多个task进入,会如何?并发
\item 会有什么不良影响吗?safety
\item 过程能完成吗?liveness
\end{compactenum}

事务分析:
\begin{compactenum}
\item 可串行化(两阶段锁,树协议)
\item 事务日志:持久化每一个操作,包括必须的上下文信息
\item 重启时,REDO/UNDO(原子性)
\item 为减少需要REDO的操作,记录检查点
\end{compactenum}

\section{集群}

\subsection{加入节点}
\subsection{删除节点}

\section{存储池}

\subsection{创建存储池}

\subsection{删除存储池}

\subsection{添加磁盘}

% \subsection{添加缓存盘}

\subsection{拔盘}

拔盘会产生一系列的影响,如IO抖动,控制器主副本丢失,存储池降级等。
与读写,控制器切换,QoS策略,恢复,平衡等过程都有密切关系。

\begin{compactenum}
\item 更高效的检测方法
\item 何时关闭fd,应避免fd重用造成的影响
\item 如何降低check过程对性能的干扰
\item lich health clean是否可以自动化
\item 盘会不会重新加入,从而造成副本数增多
\item RAID的影响,干扰别的盘,造成IO中断
\item rescan,不必等到下次恢复周期
\end{compactenum}

diskid字段上加索引

\subsection{空间分配}

admin维护着集群的拓扑结构。

故障域规则

分两层:副本的节点位置和副本的磁盘位置。

注意多个chunk的局部性对性能和恢复性能的影响。

\section{卷}

卷属性:
\begin{lstlisting}
- 副本数
- 精简配置
- 当前链接
\end{lstlisting}

\subsection{创建卷}

\subsection{删除卷}

\subsection{加载卷}

\begin{compactenum}
\item 延迟加载table2
\item 预加载table2
\item 获取allocate属性的方式和性能
\end{compactenum}

\subsection{分配卷空间}

局部性

\subsection{unmap卷空间}

\subsection{查询卷属性}

\subsection{计算卷md5sum}

存在不能返回的情况

\subsection{resize}

扩容,不允许缩容

\subsection{rename}

前置条件:不能跨存储池

\subsection{拷贝}

两种实现方式:读写,基于快照

考虑在server端做!

可控的并发度

前置条件和后置条件

不变式

\subsection{迁移}

同池迁移,同rename

跨池迁移,前置条件

\subsection{切换卷控制器}

\begin{compactenum}
\item 源端volume\_proto是怎么回收的?
\item lease机制怎么影响本过程?
\end{compactenum}

\subsection{write}

\subsection{read}

\section{快照}

\subsection{创建快照}

\subsection{删除快照}

\subsection{回滚快照}

\subsection{克隆}

克隆卷后,需要保护其源快照。目前,克隆关系是单向的,克隆卷记录了其源快照信息,快照没有记录克隆卷的信息。

\subsection{FLATTEN}

\section{主机映射}

\section{后台任务}

\subsection{监控磁盘状态}

\subsection{恢复}

局部性

恢复性能与批量分配有关。如果连续的chunk,被分配到部分节点的部分盘上,就会影响到恢复性能。
恢复是按卷顺序扫描,调整该顺序,可以提高并行度。

删除卷

QoS,slow start

恢复要能有效处理多种故障情况,做到高效及时,QoS。

\subsection{平衡}

平衡分控制器平衡和数据平衡。

卷在节点间和节点内的平衡,节点内core间平衡,可以引入一hash table来解决,
带来的问题是什么?可以容易地克服吗?

通过迁移控制器来实现卷在节点和corenet上的平衡。每个corenet指的是各个节点上具有相同core hash的core组成的网络。

平衡算法要保证输出的稳定性,分两阶段:定位和迁移。定位阶段确定所有卷控制器的位置,每个卷最多迁移一次。

\subsection{回收卷}

\subsection{回收快照}

\subsection{恢复快照}

\subsection{FLAT卷}

flat性能分析:读,分配,写多个阶段。能否批处理?并发度如何?

快照树

精简配置

删除卷

因为table2的写入特性,顺序处理各个chunk,并发度不高。

\subsection{存储池状态}

\begin{compactitem}
\item Available
\item Degraded
\item Readonly
\item Unavailable
\end{compactitem}

异步处理

cron后台执行一定的策略,如nagios等监控系统,处理结果放入/dev/shm

时间戳

pending状态

\chapter{运行时结构}

\begin{compactitem}
\item core threads
\item disk checker
\item disk allocator
\item disk aio
\item network
\item network heartbeat
\item timer
\item recovery
\item balance
\end{compactitem}

\section{lichd}

入口:\_\_lichd\_run

\section{Node}

\subsection{\_\_node\_start}

启动节点服务,支持两种模式: master and normal。

\section{CORE}

\subsection{\_\_core\_worker}

\section{磁盘管理}

\subsection{diskctl\_start}

\chapter{标准库}

\section{CPU}

\subsection{pthread}
\subsection{coroutine}

\section{Memory}

\subsection{ymalloc}
\subsection{mem\_cache}
\subsection{mbuffer}

\section{Disk}

\subsection{Local File and Directory}
\subsection{AIO}

\section{Network}

\subsection{minirpc}
\subsection{rpc}
\subsection{corerpc}

\section{String}
\section{List}
\section{Hash}
\section{Skip List}
\section{Cache with ref}
\section{Date and Time}
\section{JSON}
\section{Lease}
\section{ETCD}

\chapter{运维视图}

FAQ

\section{问题集}

问题集
\begin{enumbox}
\item lichd --init做了什么
\item 每个节点的nid保存在哪儿,如何分配的?
\item /opt/fusionstack/data目录布局
\item /dev/shm/lich4目录布局
\item cleanup
\item clock机制
\item hsm
\end{enumbox}

\begin{compactitem}
\item coroutine and scheduler
\item polling \change{polling}
\item kernel bypass
\item mbuffer
\end{compactitem}

问题集:
\begin{enumbox}
\item /的位置信息
\item 当前分配的最大卷ID?
\end{enumbox}

P1: IOMeter测试,256K,Lich顺序和随机IO性能差别大

P2: lsv\_gc\_check断言失败

P3: Error Handling

\section{约束}

\subsection{最大副本数}

6

\subsection{单卷最大快照数}


\section{Performance}

\subsection{估算}

HDD:
\begin{itembox}
\item 顺序1M 200
\item 随机1M 50
\item 顺序4K IOPS
\item 随机4K IOPS
\end{itembox}

集群部署
\begin{enumbox}
\item 网络
\item 磁盘
\item 副本数
\end{enumbox}

\subsection{精简卷顺序1M写入性能低}

subvol写锁,导致串行化

用allocate命令先批量填充,性能有显著提升。


\end{document}

\chapter{分析方法}

\section{开发流程}

分进合击

\subsection{版本控制GIT}

\subsection{文档工具Doxygen}

\subsection{编辑器VIM/Clion}

\section{测试方法}

\subsection{基于docker的自动化测试}

\begin{lstlisting}
# 以下过程只需要执行一次
python test.py --pull

# 如遇docker build失败,重启docker服务:systemctl restart docker,然后重试
python test.py --build
python test.py --run
python test.py --conf

# 可以多次执行test过程
python test.py --test
python test.py --test --nofail
\end{lstlisting}

注意事项:
\begin{compactenum}
\item 共创建8个docker容器,IP地址为:172.17.0.[2-9]
\item 检查到docker容器的ssh连接是否正常,取消docker容器内部的/etc/resolv.conf设定
\item time ./run.sh 10,共运行10次test,遇到异常则退出
\end{compactenum}

\subsection{单元测试框架cmock}

\section{调试方法}

\subsection{基于core的调试}

\begin{lstlisting}
ulimit -c
ulimit -c unlimited

cat /proc/sys/kernel/core_pattern
echo "/opt/fusionstack/core/core-%e-%p-%s" > /proc/sys/kernel/core_pattern
\end{lstlisting}

\subsection{调试运行中的程序}

\section{日志分析}

\subsection{打开backtrace}

通过backtrace可以跟踪GOTO的退出路径,方便定位错误的根源。

\begin{lstlisting}
方法1:echo 1 > /dev/shm/lich4/msgctl/backtrace
方法2:配置文件/opt/fusionstack/etc/lich.conf globals中添加: backtrace on;
\end{lstlisting}

\subsection{调整日志等级}

\begin{lstlisting}
echo n > /dev/shm/lich4/msgctl/level

其中,n的取值范围:
    1: DBUG (配合子模块使用)
    2: DINFO
    3: DWARN
    4: DERROR
    5: DFATAL

echo 1 > /dev/shm/lich4/msgctl/sub/<module>

module的取值范围:
    ylib
    ynet
    schedule
    storage
    cluster
    interface
    lsv
\end{lstlisting}


\subsection{检查lichd进程是否重启}

% \begin{itembox}
% \item grep \verb+'run as\|exit\|process\|assert' /opt/fusionstack/log/lich.log+
% \end{itembox}

\begin{lstlisting}
grep 'run as\|exit\|process\|assert' /opt/fusionstack/log/lich.log
\end{lstlisting}

\subsection{检查卷是否reload}

有时前端无IO,可能与卷反复地重新加载有关。

\subsection{全历史日志分析法}

比如调试hazard错误,需要跟踪对象生命周期内的所有行为,才能精准地定位问题的根源。

\section{内存分析}

\subsection{valgrind的用法}

\begin{lstlisting}
先停止lichd进程,然后用valgrind启动lichd进程:
    valgrind --tool=memcheck --leak-check=full --show-reachable=yes -v /opt/fusionstack/lich/sbin/lichd --home /opt/fusionstack/data -f > log.txt 2>&1

运行一段时间后可以使用ctrl + c 中止,但需要等valgrind分析完内存,再查看错误:
    grep 'illegal\|inappropriate\|inadequate\|overlapping\|memory leak\|overlap\|invalid\|definitely' log.txt

比较严重的错误:   
    invalid read (内存非法读取)
    invalid write(内存非法写入)
    definitely lost (内存泄漏)
\end{lstlisting}

\section{故障诊断}

故障诊断要结合集群和节点状态,core,日志,硬件,数据等因素。

\subsection{检查系统时间}

\subsection{检查lichd和etcd版本}

\subsection{检查集群健康状态}

\begin{lstlisting}
# etcd状态
etcdctl ls -r

# 节点,存储池和容量
lich stat

# 节点列表
lich list

# 检查恢复状态
lich health
\end{lstlisting}

\subsection{检查卷控制器的平衡性}

\subsection{查看CORE}

\begin{lstlisting}
./lscore.sh
\end{lstlisting}

\subsection{查看日志}

必要时,开启backtrace

必要时,开启DBUG日志等级

\subsection{检查硬件和操作系统状态}

CPU,内存,磁盘和网络

\subsubsection{磁盘分区满}

\lstinputlisting{code/du.sh}

%\begin{verbatim}
%TMP='/';for i in `/bin/ls $TMP`; do du -sh $TMP/$i; done
%\end{verbatim}

\subsection{检查数据一致性}

节点内sqlite/disk bitmap一致性检查

chunk副本一致性检查

\section{性能优化}

\subsection{启用performance\_analysis}

\begin{enumbox}
\item 在代码里采用<ANALYSIS\_BEGIN, ANALYSIS\_QUEUE>对,测量过程的执行时间
\item 在/opt/fusionstack/etc/lich.conf里设定:performance\_analysis on
\item 输出一:相关性能取样的条目会出现在日志文件:/opt/fusionstack/log/perf.log
\item 输出二:kill -USR1 <pid> 命令,聚合输出到日志文件:tail -f /opt/fusionstack/log/lich.log|grep --color analysis
\end{enumbox}

注意事项:
\begin{compactenum}
\item 查询lichd pid的方法:pgrep lichd,通常输出两行,取第二行的pid
\end{compactenum}

\subsection{动态追踪工具SystemTap}

\subsection{火焰图}

\chapter{元数据管理}

etcd,存储池,目录,卷,快照,映射

\section{元数据}

最重要是chkinfo结构体,表示一个chunk的若干属性,包括副本位置信息。

区分磁盘结构和内存结构。

三级元数据。是一颗大树。集群/存储池/目录/卷/chunk/replica。

\section{一致性协议}

元数据和数据的更新方式为何不同?

块存储的顺序一致性要求。成功返回的各副本强一致性,
不成功返回的,各副本可能处在不一致的状态,且无需修复。

元数据是否会进入某种不一致的状态?在整体掉电的情况下。

sqlite错误记录,指向无效的chunk?

元数据写入是sync的

事务操作,简单模型和组合模型。

\subsection{clock的作用}

分析三部曲:先分析正常情况,后分析并发场景,然后分析异常情况,
有各种各样的故障,逐一加以说明。

正常情况下,用clock维护一个chunk各副本的一致性。

每次io,携带有clock,即是chkstat.chkstat\_clock。在副本级,维护有chkid到<clock,dirty>的映射关系。
每个副本按clock递增(+1)的顺序写入。

重新载入卷,chkstat.chk\_clock设为0。

恢复一个卷的顺序,影响到并发度。

chkstat.chkstat\_clock与replica.clock相等时,说明本次IO写入完成。

写meta和raw不同,与此有关?

写入过程,副本上的io更新序,wlist。

\subsection{若干故障情况}

每个chunk的各副本,状态由几部分构成:卷控制器上的chkinfo和chkstat,
replica级别的clock信息,网络ltime。

分析每一个条件:
\begin{compactenum}
\item offline
\item reset (ltime是与nid的网络连接状态,异常时重置)
\item replica status
\item chkstat.chkstat\_clock (chunk级,io级,不需持久化,每次重启设为0)
\item clock and its dirty status
\end{compactenum}

一个节点clock丢失,所有写流程都会到达这个节点(拔盘的情况与此类似)

在chunk\_push前,会从replica上同步clockstat,然后对比。

一个副本clock丢失

所有clock丢失

重启整个系统,

\section{分配}

精简配置

\section{回收}

\section{加载}

\section{恢复}

\section{均衡}

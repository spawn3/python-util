\chapter{核心数据结构}

计算机程序的构造和解释,从数据结构开始理解软件。

\section{卷}

\subsection{Thin Provisioning}

存入fileinfo,xattr有点不便

创建的快照和基于快照clone出的卷,都是精简配置,与原卷是否是精简配置无关。

在卷迁移和复制的具体实现中,用到了快照和克隆。此时,需要保留原卷相关属性。

\subsection{扩展属性}

\section{快照}

快照是一特殊的卷,克隆也是。卷,快照,克隆从数据组织上基本一致,支持的部分操作有所不同。

目前快照实现无数据索引,导致rollback,delete,read等操作性能低下。

\section{元数据管理}

对象标识和寻址机制

\subsection{SQLITE}

索引

批处理

异步化

\subsection{DISK BITMAP}

空间分配器

\section{ETCD}

集群状态

\section{LEASE}

\section{CLOCK}

副本数据一致性

当重启一个节点的时候,clock文件可能丢失,引发数据检查和修复过程,降低系统整体性能。

\section{MSGQUEUE}

离线消息处理

异步化,是否可提取出异步化框架?

理想情况下,拔出的盘,回收完成后,其disk bitmap大小为1,sqlite无关联记录。

\section{Cache}

\subsection{控制器缓存}

采用引用计数技术。

控制器切换和平衡

\subsection{副本缓存}

\section{CORE AND Scheduler}

每个core有独立scheduler。为了与其它节点上的core进行通信,抽象出了corenet的概念。

实体标识和命名:hostname,nid,core hash,sockid。

RPC有发送端和接收端。

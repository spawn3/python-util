\chapter{核心数据结构}

\section{ETCD}

集群状态

\section{元数据管理}

对象标识和寻址机制

\section{SQLITE}

异步化

\section{DISK BITMAP}

空间分配器

\section{LEASE}

\section{CLOCK}

副本数据一致性

\section{MSGQUEUE}

离线消息处理

异步化,是否可提取出异步化框架?

\section{控制器缓存}

采用引用计数技术。

控制器切换和平衡

\section{副本缓存}

\section{扩展属性}

\section{快照}

\section{克隆卷}

\section{CORE AND Scheduler}

每个core有独立scheduler。为了与其它节点上的core进行通信,抽象出了corenet的概念。

实体标识和命名:hostname,nid,core hash,sockid。

RPC有发送端和接收端。

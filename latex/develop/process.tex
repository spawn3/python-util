\chapter{关键过程}

过程分析分两种:正常情况和异常情况。特别是故障情况下,过程之间存在广泛的交互作用,变得更为复杂。

过程需要具备一些重要属性,如事务ACID,safety和liveness等,要具体情况具体分析。
值得注意的是,若干过程需要实现为可重入的。

其它注意事项:
\begin{compactenum}
\item 必须正确处理过程的返回值。
\item 一定要检查加锁的返回值,返回值失败,不能unlock,从而保证加锁解锁的对称性
\item 资源的分配和释放,使用goto方法,用stack方式管理
\item 过程运行在协程内,还是协程外?
\item 变量作用域,在作用域以外访问变量
\item 过程是否并发安全?
\item 过程执行时间是否过长,且中间无中断?比如有大循环,有同步IO操作等
\end{compactenum}

实体有复合和简单之分。简单实体的基本操作包括增删改查,复合实体在此基础上多出了子节点的相关操作。

按一颗树来组织,简单实体是叶子节点,复合实体是中间节点或根节点。

if-then/what-if假设分析:
\begin{compactenum}
\item 资源管理的对称性(分配/释放, 资源包括fd,lock,malloc等)
\item 如果此处发生故障,会如何?容错
\item 如果有多个task进入,会如何?并发
\item 会有什么不良影响吗?safety
\item 过程能完成吗?liveness
\end{compactenum}

事务分析:
\begin{compactenum}
\item 可串行化(两阶段锁,树协议)
\item 事务日志:持久化每一个操作,包括必须的上下文信息
\item 重启时,REDO/UNDO(原子性)
\item 为减少需要REDO的操作,记录检查点
\end{compactenum}

\section{集群}

\subsection{加入节点}
\subsection{删除节点}

\section{存储池}

\subsection{创建存储池}

\subsection{删除存储池}

\subsection{添加磁盘}

% \subsection{添加缓存盘}

\subsection{拔盘}

拔盘会产生一系列的影响,如IO抖动,控制器主副本丢失,存储池降级等。
与读写,控制器切换,QoS策略,恢复,平衡等过程都有密切关系。

\begin{compactenum}
\item 更高效的检测方法
\item 何时关闭fd,应避免fd重用造成的影响
\item 如何降低check过程对性能的干扰
\item lich health clean是否可以自动化
\item 盘会不会重新加入,从而造成副本数增多
\item RAID的影响,干扰别的盘,造成IO中断
\item rescan,不必等到下次恢复周期
\end{compactenum}

diskid字段上加索引

\subsection{空间分配}

admin维护着集群的拓扑结构。

故障域规则

分两层:副本的节点位置和副本的磁盘位置。

注意多个chunk的局部性对性能和恢复性能的影响。

\section{卷}

卷属性:
\begin{lstlisting}
- 副本数
- 精简配置
- 当前链接
\end{lstlisting}

\subsection{创建卷}

\subsection{删除卷}

\subsection{加载卷}

\begin{compactenum}
\item 延迟加载table2
\item 预加载table2
\item 获取allocate属性的方式和性能
\end{compactenum}

\subsection{分配卷空间}

局部性

\subsection{unmap卷空间}

\subsection{查询卷属性}

\subsection{计算卷md5sum}

存在不能返回的情况

\subsection{resize}

扩容,不允许缩容

\subsection{rename}

前置条件:不能跨存储池

\subsection{拷贝}

两种实现方式:读写,基于快照

考虑在server端做!

可控的并发度

前置条件和后置条件

不变式

\subsection{迁移}

同池迁移,同rename

跨池迁移,前置条件

\subsection{切换卷控制器}

\begin{compactenum}
\item 源端volume\_proto是怎么回收的?
\item lease机制怎么影响本过程?
\end{compactenum}

\subsection{write}

\subsection{read}

\section{快照}

\subsection{创建快照}

\subsection{删除快照}

\subsection{回滚快照}

\subsection{克隆}

克隆卷后,需要保护其源快照。目前,克隆关系是单向的,克隆卷记录了其源快照信息,快照没有记录克隆卷的信息。

\subsection{FLATTEN}

\section{主机映射}

\section{后台任务}

\subsection{监控磁盘状态}

\subsection{恢复}

局部性

恢复性能与批量分配有关。如果连续的chunk,被分配到部分节点的部分盘上,就会影响到恢复性能。
恢复是按卷顺序扫描,调整该顺序,可以提高并行度。

删除卷

QoS,slow start

恢复要能有效处理多种故障情况,做到高效及时,QoS。

\subsection{平衡}

平衡分控制器平衡和数据平衡。

卷在节点间和节点内的平衡,节点内core间平衡,可以引入一hash table来解决,
带来的问题是什么?可以容易地克服吗?

通过迁移控制器来实现卷在节点和corenet上的平衡。每个corenet指的是各个节点上具有相同core hash的core组成的网络。

平衡算法要保证输出的稳定性,分两阶段:定位和迁移。定位阶段确定所有卷控制器的位置,每个卷最多迁移一次。

\subsection{回收卷}

\subsection{回收快照}

\subsection{恢复快照}

\subsection{FLAT卷}

flat性能分析:读,分配,写多个阶段。能否批处理?并发度如何?

快照树

精简配置

删除卷

因为table2的写入特性,顺序处理各个chunk,并发度不高。

\subsection{存储池状态}

\begin{compactitem}
\item Available
\item Degraded
\item Readonly
\item Unavailable
\end{compactitem}

异步处理

cron后台执行一定的策略,如nagios等监控系统,处理结果放入/dev/shm

时间戳

pending状态

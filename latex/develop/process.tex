\chapter{关键过程}

过程分析分两种:正常情况和异常情况。特别是故障情况下,过程之间存在广泛的交互作用,变得更为复杂。

过程需要具备一些重要属性,如事务ACID,safety和liveness等,要具体情况具体分析。
值得注意的是,若干过程需要实现为可重入的。

其它注意事项:
\begin{compactenum}
\item 必须正确处理过程的返回值。
\item 过程运行在协程内,还是协程外?
\item 过程是否并发安全?
\item 过程执行时间是否过长,且中间无中断?比如有大循环,有同步IO操作等
\item 一定要检查加锁的返回值,返回值失败,不能unlock,从而保证加锁解锁的对称性
\item 资源的分配和释放,使用goto方法,用stack方式管理
\end{compactenum}

实体有复合和简单之分。简单实体的基本操作包括增删改查,复合实体在此基础上多出了子节点的相关操作。

按一颗树来组织,简单实体是叶子节点,复合实体是中间节点或根节点。

\section{集群}

\subsection{加入节点}
\subsection{删除节点}

\section{存储池}

\subsection{创建存储池}

\subsection{删除存储池}

\subsection{添加磁盘}

% \subsection{添加缓存盘}

\subsection{拔盘}

拔盘会产生一系列的影响,如IO抖动,控制器主副本丢失,存储池降级等。
与读写,控制器切换,QoS策略,恢复,平衡等过程都有密切关系。

\begin{compactenum}
\item 更高效的检测方法
\item 何时关闭fd,应避免fd重用造成的影响
\item 如何降低check过程对性能的干扰
\item lich health clean是否可以自动化
\item 盘会不会重新加入,从而造成副本数增多
\item RAID的影响,干扰别的盘,造成IO中断
\item rescan,不必等到下次恢复周期
\end{compactenum}

diskid字段上加索引

\subsection{空间分配}

admin维护着集群的拓扑结构。

故障域规则

分两层:副本的节点位置和副本的磁盘位置。

注意多个chunk的局部性对性能和恢复性能的影响。

\section{卷}

\subsection{创建卷}

\subsection{删除卷}

\subsection{加载卷}

\begin{compactenum}
\item 延迟加载table2
\item 预加载table2
\item 获取allocate属性的方式和性能
\end{compactenum}

\subsection{切换卷控制器}

\begin{compactenum}
\item 源端volume\_proto是怎么回收的?
\item lease机制怎么影响本过程?
\end{compactenum}

\subsection{分配卷空间}

局部性

\subsection{unmap卷空间}

\subsection{resize}

\subsection{rename}

\subsection{拷贝}

前置条件和后置条件

不变式

考虑在server端做!

\subsection{迁移}

\subsection{write}
\subsection{read}

\section{快照}

\subsection{创建快照}
\subsection{删除快照}
\subsection{回滚快照}
\subsection{克隆}
\subsection{FLATTEN}

\section{主机映射}

\section{后台任务}

\subsection{监控磁盘状态}

\subsection{恢复}

局部性

恢复性能与批量分配有关。如果连续的chunk,被分配到部分节点的部分盘上,就会影响到恢复性能。

恢复是按卷顺序扫描,调整该顺序,可以提高并行度。

\subsection{平衡}

平衡分控制器平衡和数据平衡。

\subsection{回收卷}

\subsection{回收快照}

\subsection{恢复快照}

\subsection{FLAT卷}

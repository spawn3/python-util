\chapter{关键过程}

过程分析分两种:正常情况和异常情况。特别是故障情况下,过程之间存在广泛的交互作用,变得更为复杂。

过程需要具备一些重要属性,如事务ACID,safety和liveness等,要具体情况具体分析。
值得注意的是,若干过程需要实现为可重入的。

其它注意事项:
\begin{compactenum}
\item 必须正确处理过程的返回值。
\item 过程运行在协程内,还是协程外?
\item 过程是否并发安全?
\item 过程执行时间是否过长,且中间无中断?比如有大循环,有同步IO操作等
\end{compactenum}

\section{存储池}

\subsection{创建存储池}

\subsection{添加磁盘}

% \subsection{添加缓存盘}

\subsection{空间分配}

admin维护着集群的拓扑结构。

故障域规则

分两层:副本的节点位置和副本的磁盘位置。

注意多个chunk的局部性对性能和恢复性能的影响。

\subsection{拔盘}

拔盘会产生一系列的影响,如IO抖动,控制器主副本丢失,存储池降级等。
与读写,控制器切换,QoS策略,恢复,平衡等过程都有密切关系。

\subsection{删除存储池}

\section{卷}

\subsection{创建卷}

\subsection{删除卷}

\subsection{加载卷}

\begin{compactenum}
\item 延迟加载table2
\item 预加载table2
\item 获取allocate属性的方式和性能
\end{compactenum}

\subsection{分配卷空间}

\subsection{unmap卷空间}

\subsection{切换卷控制器}

\begin{compactenum}
\item 源端volume\_proto是怎么回收的?
\item lease机制怎么影响本过程?
\end{compactenum}

\section{快照}

\subsection{创建快照}
\subsection{删除快照}
\subsection{回滚快照}
\subsection{克隆}
\subsection{FLATTEN}

\section{主机映射}

\section{后台任务}

\subsection{恢复}

\subsection{平衡}

平衡分控制器平衡和数据平衡。

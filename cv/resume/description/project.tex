\resheading{项目经验}
  \begin{itemize}[leftmargin=*]
      \item \textbf{2017.11: 维护分布式块存储引擎FusionStor:} FusionStor是采用协程调度机制、PMD方式驱动的分布式块存储系统,Linux/C。
          {\small
          \begin{itemize}
              \item 实现批量allocate,以改进精简配置卷的性能
              \item 基于token bucket和lecky bucket算法,实现卷QoS和恢复QoS
              \item 采用SEDA(Stage Event Drivern Arch)实现恢复、平衡等特性。引入最小副本数的概念,支持降级写/异步恢复策略,平衡一致性和性能
              %\item 优化故障下的IO中断时间
              \item 延迟加载subvol数据块,以提升卷加载速度
              \item 改进创建快照操作的事务性语义
              \item 支持大容量卷
              \item 实现异步任务管理框架,支持删除卷、快照、revert、flatten等操作
              \item 重构本地磁盘管理代码,支持NVMe内核态/用户态两种驱动方式
              \item 调研本地bcache缓存方案
          \end{itemize}
          }

      \item \textbf{2017.08-2017.11: 实现COW树型快照:} 删除线性快照的某一快照后,所有后续快照(如果有的话)都将被删除,
          故引入树型快照,卷及其快照构成单根树型结构。

      \item \textbf{2017.02-2017.06: Log Structure Volume原型:} 调研并原型化Log structured Volume(LSV)格式,
          以改进随机写性能,并支持ROW快照。

      \item \textbf{用户行为日志分析:} 采集用户行为数据,经由Redis导入Hadoop集群。
          后端采用MapReduce计算框架计算关键指标,结果存入MongoDB供前端可视化使用。
          基本能满足产品/运营部门的数据需求。

      \item \textbf{安卓壁纸和动态壁纸APP后端基础架构开发和运维:} 分层架构(接入层/业务层/数据层)
          {\small
          \begin{itemize}
              \item 业务层(Python/Tornado)无状态可横向扩展
              \item 数据层采用MongoDB集群,各层充分利用index/cache提升性能
              \item 采用动静分离策略,把图片等静态资源分离出来
              \item 采用ElasticSearch支持网站全文检索
          \end{itemize}
          }

      \item \textbf{EDOG虚拟机管理系统:} B/S架构,后端采用Erlang/OTP,通过Libvirt控制VM,通过QEMU驱动对接自研分布式共享存储系统,
          实现了VM的生命周期管理、动态调度、以及故障下的HA等特性。

      \item \textbf{鸿业市政管线设计软件:} 在Autodesk平台上采用C++/AutoLisp开发。独立完成的主要工作有:自动裁图、管网平差计算、
          通过C++/Boost graph对管网建模,帮助市政设计人员来模拟运行和评估参数。

  \end{itemize}

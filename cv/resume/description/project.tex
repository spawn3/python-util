\resheading{项目经验}
  \begin{itemize}[leftmargin=*]
      \item \textbf{2017.11: 分布式块存储引擎:} 研发分布式块存储系统(Linux/C),改进数据恢复、平衡、QoS等模块。
          采用SEDA架构实现数据恢复、平衡等特性。引入最小副本数,允许降级写/异步恢复,平衡CAP定理的C和A。

      \item \textbf{2017.08-2017.11: 支持COW树型快照:} 删除线性快照的某一快照后,所有后续快照(如果有的话)都将被删除,
          故引入树型快照,卷和它的所有快照构成单根树型结构。

      \item \textbf{2017.02-2017.07: Log Structure Volume原型:}

      \item \textbf{用户行为日志分析:} 采集用户行为数据,经由Redis导入Hadoop集群。
          后端采用MapReduce计算框架计算关键指标,结果存入MongoDB供前端可视化使用。

      \item \textbf{安卓壁纸和动态壁纸APP后端基础架构开发和运维:} 分层架构(接入层/业务层/数据层),
          业务层(Python/Tornado)无状态可横向扩展,
          数据层采用MongoDB集群,各层充分利用index/cache提升性能。
          采用动静分离策略,把图片等静态资源分离出来。采用ElasticSearch支持网站全文检索。

      \item \textbf{EDOG虚拟机管理系统:} B/S架构,后端采用Erlang/OTP,通过Libvirt控制VM,通过QEMU驱动对接自研分布式共享存储系统,
          实现了VM的生命周期管理、动态调度、以及故障下的HA等特性。

      \item \textbf{鸿业市政管线设计软件:} 在Autodesk平台上采用C++/AutoLisp开发。独立完成的主要工作有:自动裁图、管网平差计算、
          通过C++/Boost graph对管网建模,帮助市政设计人员来模拟运行和评估参数。

  \end{itemize}
